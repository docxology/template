% Options for packages loaded elsewhere
\PassOptionsToPackage{unicode}{hyperref}
\PassOptionsToPackage{hyphens}{url}
\documentclass[
  ignorenonframetext,
]{beamer}
\newif\ifbibliography
\usepackage{pgfpages}
\setbeamertemplate{caption}[numbered]
\setbeamertemplate{caption label separator}{: }
\setbeamercolor{caption name}{fg=normal text.fg}
\beamertemplatenavigationsymbolsempty
% remove section numbering
\setbeamertemplate{part page}{
  \centering
  \begin{beamercolorbox}[sep=16pt,center]{part title}
    \usebeamerfont{part title}\insertpart\par
  \end{beamercolorbox}
}
\setbeamertemplate{section page}{
  \centering
  \begin{beamercolorbox}[sep=12pt,center]{section title}
    \usebeamerfont{section title}\insertsection\par
  \end{beamercolorbox}
}
\setbeamertemplate{subsection page}{
  \centering
  \begin{beamercolorbox}[sep=8pt,center]{subsection title}
    \usebeamerfont{subsection title}\insertsubsection\par
  \end{beamercolorbox}
}
% Prevent slide breaks in the middle of a paragraph
\widowpenalties 1 10000
\raggedbottom
\AtBeginPart{
  \frame{\partpage}
}
\AtBeginSection{
  \ifbibliography
  \else
    \frame{\sectionpage}
  \fi
}
\AtBeginSubsection{
  \frame{\subsectionpage}
}
\usepackage{iftex}
\ifPDFTeX
  \usepackage[T1]{fontenc}
  \usepackage[utf8]{inputenc}
  \usepackage{textcomp} % provide euro and other symbols
\else % if luatex or xetex
  \usepackage{unicode-math} % this also loads fontspec
  \defaultfontfeatures{Scale=MatchLowercase}
  \defaultfontfeatures[\rmfamily]{Ligatures=TeX,Scale=1}
\fi
\usepackage{lmodern}
\ifPDFTeX\else
  % xetex/luatex font selection
\fi
% Use upquote if available, for straight quotes in verbatim environments
\IfFileExists{upquote.sty}{\usepackage{upquote}}{}
\IfFileExists{microtype.sty}{% use microtype if available
  \usepackage[]{microtype}
  \UseMicrotypeSet[protrusion]{basicmath} % disable protrusion for tt fonts
}{}
\makeatletter
\@ifundefined{KOMAClassName}{% if non-KOMA class
  \IfFileExists{parskip.sty}{%
    \usepackage{parskip}
  }{% else
    \setlength{\parindent}{0pt}
    \setlength{\parskip}{6pt plus 2pt minus 1pt}}
}{% if KOMA class
  \KOMAoptions{parskip=half}}
\makeatother
\setlength{\emergencystretch}{3em} % prevent overfull lines
\providecommand{\tightlist}{%
  \setlength{\itemsep}{0pt}\setlength{\parskip}{0pt}}
\usepackage{bookmark}
\IfFileExists{xurl.sty}{\usepackage{xurl}}{} % add URL line breaks if available
\urlstyle{same}
\hypersetup{
  hidelinks,
  pdfcreator={LaTeX via pandoc}}

\author{\texorpdfstring{}{}}
\date{}

\begin{document}

\section{Conclusion}\label{sec:conclusion}

\begin{frame}{Summary of Ento-Linguistic Contributions}
\protect\phantomsection\label{summary-of-ento-linguistic-contributions}
This work establishes Ento-Linguistic analysis as a critical framework
for understanding how scientific language constitutes knowledge rather
than merely representing it. Our main contributions demonstrate that
terminology in entomology creates systematic patterns of ambiguity and
framing that influence research practice across six key domains: Unit of
Individuality, Behavior and Identity, Power \& Labor, Sex \&
Reproduction, Kin, and Economics.
\end{frame}

\begin{frame}{Key Findings and Theoretical Achievements}
\protect\phantomsection\label{key-findings-and-theoretical-achievements}
\begin{block}{Constitutive Role of Scientific Language}
\protect\phantomsection\label{constitutive-role-of-scientific-language}
Our mixed-methodology framework revealed that scientific terminology is
not transparent but actively shapes research possibilities:

\textbf{Terminology Network Structure}: Computational analysis of 1,578
terms across 12,847 relationships demonstrated modular network
structures where domains develop specialized terminological dialects.

\textbf{Context-Dependent Meaning}: 73.4\% of analyzed terminology
exhibits context-dependent meanings, creating ambiguity that influences
research interpretation.

\textbf{Framing Assumptions}: Systematic identification of
anthropomorphic (67.3\%), hierarchical (45.8\%), and economic (23.1\%)
framings that impose human social structures on ant biology.

\textbf{Domain-Specific Patterns}: Each Ento-Linguistic domain shows
characteristic terminological structures, from the rigid hierarchies of
Power \& Labor to the fluid identities of Behavior and Identity domains.
\end{block}

\begin{block}{Speech and Thought Entanglement}
\protect\phantomsection\label{speech-and-thought-entanglement}
The ethical motivation articulated in Section \ref{sec:introduction}
finds empirical support in our analysis: scientific language creates
invisible constraints on inquiry that researchers must actively address
to achieve communicative clarity.
\end{block}
\end{frame}

\begin{frame}{Broader Impact on Scientific Practice}
\protect\phantomsection\label{broader-impact-on-scientific-practice}
\begin{block}{Implications for Scientific Communication}
\protect\phantomsection\label{implications-for-scientific-communication}
Our findings establish principles for more conscious scientific language
use:

\textbf{Clarity as Ethical Imperative}: In value-aligned scientific
communities, clear communication becomes an ethical responsibility
rather than optional practice.

\textbf{Terminological Stewardship}: Scientific communities should
actively curate terminology to ensure it serves research goals rather
than perpetuating historical conceptual limitations.

\textbf{Meta-Standards Development}: Our work provides foundations for
evaluating scientific communication quality alongside methodological
rigor.
\end{block}

\begin{block}{Applications Across Scientific Disciplines}
\protect\phantomsection\label{applications-across-scientific-disciplines}
The Ento-Linguistic framework developed here has applications beyond
entomology:

\textbf{Biological Sciences}: Analysis of anthropomorphic terminology in
evolutionary biology, neuroscience, and ecology.

\textbf{Interdisciplinary Research}: Understanding how specialized
terminological dialects create communication barriers between
disciplines.

\textbf{Science Education}: Developing frameworks for teaching students
about how language shapes scientific understanding.

\textbf{Peer Review Processes}: Integrating language analysis into
evaluation of research clarity and appropriateness.
\end{block}
\end{frame}

\begin{frame}{Future Directions and Meta-Standards}
\protect\phantomsection\label{future-directions-and-meta-standards}
\begin{block}{Immediate Extensions}
\protect\phantomsection\label{immediate-extensions}
Several critical areas for immediate development emerged from our
analysis:

\textbf{Multilingual Analysis}: Extending Ento-Linguistic analysis to
non-English scientific literature to identify cross-cultural
terminological patterns. For example, comparing how German ``Staaten''
(states) vs.~English ``colony'' terminology influences understandings of
social insect organization.

\textbf{Longitudinal Studies}: Tracking terminological evolution over
time to understand how scientific language changes with theoretical
developments. This could reveal how the shift from ``superorganism'' to
``colonial'' perspectives altered research questions in entomology.

\textbf{Interactive Tools}: Developing software tools that help
researchers navigate terminological complexity and identify appropriate
language use. Such tools could provide real-time feedback on term
appropriateness and suggest clearer alternatives.
\end{block}

\begin{block}{Theoretical Advancements}
\protect\phantomsection\label{theoretical-advancements}
\textbf{Extended Discourse Frameworks}: Developing more sophisticated
theories of how scientific language constitutes research objects and
relationships.

\textbf{Comparative Disciplinary Analysis}: Applying Ento-Linguistic
methods across scientific disciplines to identify general principles of
scientific communication.

\textbf{Semantic Network Integration}: Incorporating advanced semantic
analysis techniques to better capture conceptual relationships in
scientific terminology.
\end{block}

\begin{block}{Practical Applications}
\protect\phantomsection\label{practical-applications}
\textbf{Terminology Guidelines}: Creating evidence-based guidelines for
clear scientific communication across biological disciplines.

\textbf{Educational Interventions}: Developing training programs that
help researchers understand how language shapes their work.

\textbf{Peer Review Integration}: Incorporating language clarity
assessment into scientific peer review processes.
\end{block}
\end{frame}

\begin{frame}{Meta-Standards for Scientific Communication}
\protect\phantomsection\label{meta-standards-for-scientific-communication}
Our work establishes foundational principles for meta-standards that
scientific communities can use to evaluate and improve communication
practices:

\textbf{Clarity Standards}: Terminology should maximize understanding
while minimizing unnecessary ambiguity and confusion.

\textbf{Appropriateness Standards}: Language should be appropriate to
the phenomena described, avoiding inappropriate projections of human
social categories onto natural systems.

\textbf{Consistency Standards}: Within research communities, terminology
should be used consistently to facilitate communication and knowledge
accumulation.

\textbf{Evolution Standards}: Communities should maintain mechanisms for
terminological evolution as scientific understanding develops and
research questions change.
\end{frame}

\begin{frame}{Final Reflections}
\protect\phantomsection\label{final-reflections}
This work demonstrates that scientific language is not a neutral tool
for representing biological reality, but an active constituent of
scientific knowledge production. By making visible the constitutive
effects of terminology in entomology, we provide a foundation for more
responsible and effective scientific communication.

The entanglement of speech and thought in scientific practice creates
both challenges and opportunities. The challenge lies in recognizing how
established terminology creates invisible constraints on inquiry. The
opportunity lies in developing conscious practices for terminological
stewardship that enhance rather than limit scientific understanding.

As scientific research becomes increasingly complex and
interdisciplinary, the quality of scientific communication becomes ever
more critical. Our work provides both analytical tools and theoretical
insights for addressing this challenge, establishing Ento-Linguistic
analysis as a vital methodology for understanding and improving how
scientists communicate about the natural world.

The meta-standards developed here offer a pathway toward scientific
communities that communicate with greater clarity, precision, and
ethical awareness---advancing not just what we know about the world, but
how we know it.
\end{frame}

\end{document}
