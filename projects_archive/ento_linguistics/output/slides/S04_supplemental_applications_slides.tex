% Options for packages loaded elsewhere
\PassOptionsToPackage{unicode}{hyperref}
\PassOptionsToPackage{hyphens}{url}
\documentclass[
  ignorenonframetext,
]{beamer}
\newif\ifbibliography
\usepackage{pgfpages}
\setbeamertemplate{caption}[numbered]
\setbeamertemplate{caption label separator}{: }
\setbeamercolor{caption name}{fg=normal text.fg}
\beamertemplatenavigationsymbolsempty
% remove section numbering
\setbeamertemplate{part page}{
  \centering
  \begin{beamercolorbox}[sep=16pt,center]{part title}
    \usebeamerfont{part title}\insertpart\par
  \end{beamercolorbox}
}
\setbeamertemplate{section page}{
  \centering
  \begin{beamercolorbox}[sep=12pt,center]{section title}
    \usebeamerfont{section title}\insertsection\par
  \end{beamercolorbox}
}
\setbeamertemplate{subsection page}{
  \centering
  \begin{beamercolorbox}[sep=8pt,center]{subsection title}
    \usebeamerfont{subsection title}\insertsubsection\par
  \end{beamercolorbox}
}
% Prevent slide breaks in the middle of a paragraph
\widowpenalties 1 10000
\raggedbottom
\AtBeginPart{
  \frame{\partpage}
}
\AtBeginSection{
  \ifbibliography
  \else
    \frame{\sectionpage}
  \fi
}
\AtBeginSubsection{
  \frame{\subsectionpage}
}
\usepackage{iftex}
\ifPDFTeX
  \usepackage[T1]{fontenc}
  \usepackage[utf8]{inputenc}
  \usepackage{textcomp} % provide euro and other symbols
\else % if luatex or xetex
  \usepackage{unicode-math} % this also loads fontspec
  \defaultfontfeatures{Scale=MatchLowercase}
  \defaultfontfeatures[\rmfamily]{Ligatures=TeX,Scale=1}
\fi
\usepackage{lmodern}
\ifPDFTeX\else
  % xetex/luatex font selection
\fi
% Use upquote if available, for straight quotes in verbatim environments
\IfFileExists{upquote.sty}{\usepackage{upquote}}{}
\IfFileExists{microtype.sty}{% use microtype if available
  \usepackage[]{microtype}
  \UseMicrotypeSet[protrusion]{basicmath} % disable protrusion for tt fonts
}{}
\makeatletter
\@ifundefined{KOMAClassName}{% if non-KOMA class
  \IfFileExists{parskip.sty}{%
    \usepackage{parskip}
  }{% else
    \setlength{\parindent}{0pt}
    \setlength{\parskip}{6pt plus 2pt minus 1pt}}
}{% if KOMA class
  \KOMAoptions{parskip=half}}
\makeatother
\setlength{\emergencystretch}{3em} % prevent overfull lines
\providecommand{\tightlist}{%
  \setlength{\itemsep}{0pt}\setlength{\parskip}{0pt}}
\usepackage{bookmark}
\IfFileExists{xurl.sty}{\usepackage{xurl}}{} % add URL line breaks if available
\urlstyle{same}
\hypersetup{
  hidelinks,
  pdfcreator={LaTeX via pandoc}}

\author{\texorpdfstring{}{}}
\date{}

\begin{document}

\begin{frame}{Supplemental Applications}
\protect\phantomsection\label{sec:supplemental_applications}
This section presents extended application examples demonstrating the
practical utility of the Ento-Linguistic framework across diverse
domains, complementing the case studies in Section
\ref{sec:experimental_results}.

\begin{block}{S4.1 Biological Sciences Applications}
\protect\phantomsection\label{s4.1-biological-sciences-applications}
\begin{block}{S4.1.1 Evolutionary Biology Terminology Analysis}
\protect\phantomsection\label{s4.1.1-evolutionary-biology-terminology-analysis}
Applying Ento-Linguistic methods to evolutionary biology reveals similar
patterns of anthropomorphic framing:

\textbf{Case Study: Cooperation and Conflict Terminology}

Analysis of terms like ``altruism,'' ``selfishness,'' and ``cheating''
in evolutionary literature shows: - 72\% of cooperation terminology
derives from human social concepts - 89\% of conflict terminology uses
game-theoretic metaphors - Context-dependent meaning shifts between
theoretical and empirical contexts

\textbf{Key Findings}: - Terminological framing influences research
questions about cooperation mechanisms - Cross-domain borrowing creates
ambiguity in evolutionary explanations - Historical evolution of
cooperation concepts parallels entomological patterns
\end{block}

\begin{block}{S4.1.2 Neuroscience Language Analysis}
\protect\phantomsection\label{s4.1.2-neuroscience-language-analysis}
Ento-Linguistic methods applied to neuroscience terminology reveal
hierarchical framing patterns:

\textbf{Case Study: Neural Network Terminology}

Analysis shows how terms like ``hierarchy,'' ``command,'' and
``control'' impose social structures on neural systems: - 65\% of neural
control terminology uses command metaphors - 78\% of learning
terminology employs pedagogical metaphors - Scale transitions create
ambiguity between neuron, circuit, and system levels
\end{block}
\end{block}

\begin{block}{S4.2 Interdisciplinary Research Applications}
\protect\phantomsection\label{s4.2-interdisciplinary-research-applications}
\begin{block}{S4.2.1 Science Education Applications}
\protect\phantomsection\label{s4.2.1-science-education-applications}
Ento-Linguistic analysis provides tools for improving science education:

\textbf{Curriculum Development}: Using terminology analysis to identify
concepts that need careful explanation

\textbf{Student Learning Assessment}: Analyzing student misconceptions
through terminological patterns

\textbf{Textbook Analysis}: Evaluating how scientific texts communicate
complex concepts
\end{block}

\begin{block}{S4.2.3 Scientific Communication Training}
\protect\phantomsection\label{s4.2.3-scientific-communication-training}
Developing training programs for researchers based on Ento-Linguistic
insights:

\textbf{Terminology Awareness}: Teaching researchers to recognize
framing assumptions in their writing

\textbf{Cross-Disciplinary Communication}: Training in translating
concepts between specialized domains

\textbf{Peer Review Enhancement}: Using linguistic analysis to improve
manuscript clarity
\end{block}
\end{block}

\begin{block}{S4.3 Historical Analysis Applications}
\protect\phantomsection\label{s4.3-historical-analysis-applications}
\begin{block}{S4.3.1 Scientific Revolution Analysis}
\protect\phantomsection\label{s4.3.1-scientific-revolution-analysis}
Applying longitudinal terminology analysis to periods of scientific
change:

\textbf{Darwinian Revolution (1830-1870)}: Analysis of how evolutionary
terminology evolved from creationist to naturalistic frameworks

\textbf{Molecular Biology Revolution (1940-1970)}: Tracking shift from
classical to molecular explanations

\textbf{Genomic Era (2000-present)}: Examining how ``-omics''
terminology shapes contemporary biology
\end{block}

\begin{block}{S4.3.2 Paradigm Shift Detection}
\protect\phantomsection\label{s4.3.2-paradigm-shift-detection}
Using terminology network analysis to identify paradigm changes:

\textbf{Network Restructuring Events}: Major changes in terminology
relationships indicating paradigm shifts

\textbf{Term Obsolescence Patterns}: How old terms are replaced by new
conceptual frameworks

\textbf{Conceptual Continuity}: Terms that persist across paradigm
changes
\end{block}
\end{block}

\begin{block}{S4.4 Cross-Cultural Applications}
\protect\phantomsection\label{s4.4-cross-cultural-applications}
\begin{block}{S4.4.1 Multilingual Scientific Terminology}
\protect\phantomsection\label{s4.4.1-multilingual-scientific-terminology}
Extending analysis to non-English scientific literature:

\textbf{German Entomological Terminology}: Comparing ``Staaten''
(states) vs.~English ``colony'' concepts

\textbf{French Biological Terminology}: Analysis of hierarchical
vs.~egalitarian conceptual frameworks

\textbf{Chinese Scientific Terminology}: Examining how traditional
concepts influence modern scientific language
\end{block}

\begin{block}{S4.4.2 Cultural Bias in Scientific Language}
\protect\phantomsection\label{s4.4.2-cultural-bias-in-scientific-language}
Analyzing how cultural contexts shape scientific terminology:

\textbf{Western Individualism}: Emphasis on individual agency in
biological descriptions

\textbf{Eastern Holism}: Focus on system-level relationships and
interdependence

\textbf{Indigenous Knowledge}: Alternative conceptual frameworks for
natural phenomena
\end{block}
\end{block}

\begin{block}{S4.5 Philosophical Applications}
\protect\phantomsection\label{s4.5-philosophical-applications}
\begin{block}{S4.5.1 Philosophy of Science Applications}
\protect\phantomsection\label{s4.5.1-philosophy-of-science-applications}
Ento-Linguistic analysis contributes to philosophy of science:

\textbf{Theory-Laden Language}: How theoretical commitments shape
observational language

\textbf{Incommensurability}: How different terminological frameworks
create communication barriers

\textbf{Scientific Realism}: Analysis of how language constitutes
scientific objects
\end{block}

\begin{block}{S4.5.2 Metaphor Theory in Science}
\protect\phantomsection\label{s4.5.2-metaphor-theory-in-science}
Examining metaphorical language in scientific discourse:

\textbf{Root Metaphors}: Fundamental metaphors that structure entire
research fields

\textbf{Metaphor Evolution}: How scientific metaphors change over time

\textbf{Metaphor Productivity}: How metaphors generate new research
questions
\end{block}
\end{block}

\begin{block}{S4.6 Policy and Ethics Applications}
\protect\phantomsection\label{s4.6-policy-and-ethics-applications}
\begin{block}{S4.6.1 Research Policy Applications}
\protect\phantomsection\label{s4.6.1-research-policy-applications}
Using terminology analysis for research policy development:

\textbf{Funding Priority Setting}: Analyzing terminology patterns to
identify emerging research areas

\textbf{Interdisciplinary Collaboration}: Facilitating communication
across research domains

\textbf{Research Evaluation}: Assessing the clarity and impact of
scientific communication
\end{block}

\begin{block}{S4.6.2 Ethical Implications}
\protect\phantomsection\label{s4.6.2-ethical-implications}
Exploring ethical dimensions of scientific language:

\textbf{Inclusive Language}: Promoting terminology that avoids cultural
bias

\textbf{Transparent Communication}: Ensuring scientific language serves
research goals

\textbf{Responsible Terminology}: Developing ethical guidelines for
scientific naming practices
\end{block}
\end{block}

\begin{block}{S4.7 Technological Applications}
\protect\phantomsection\label{s4.7-technological-applications}
\begin{block}{S4.7.1 Natural Language Processing Tools}
\protect\phantomsection\label{s4.7.1-natural-language-processing-tools}
Developing NLP tools based on Ento-Linguistic insights:

\textbf{Scientific Text Analysis}: Automated identification of framing
assumptions

\textbf{Terminology Standardization}: Tools for maintaining consistent
scientific language

\textbf{Cross-Disciplinary Translation}: Automated translation between
specialized domains
\end{block}

\begin{block}{S4.7.2 Knowledge Organization Systems}
\protect\phantomsection\label{s4.7.2-knowledge-organization-systems}
Creating better systems for organizing scientific knowledge:

\textbf{Ontology Development}: Building formal ontologies based on
terminology network analysis

\textbf{Semantic Search}: Improving scientific literature search through
conceptual relationships

\textbf{Automated Classification}: Using terminology patterns for
document classification
\end{block}
\end{block}

\begin{block}{S4.8 Societal Impact Applications}
\protect\phantomsection\label{s4.8-societal-impact-applications}
\begin{block}{S4.8.1 Public Understanding of Science}
\protect\phantomsection\label{s4.8.1-public-understanding-of-science}
Using Ento-Linguistic methods to improve science communication:

\textbf{Science Journalism}: Training journalists in accurate scientific
terminology use

\textbf{Public Education}: Developing materials that explain scientific
concepts clearly

\textbf{Science Policy Communication}: Improving communication between
scientists and policymakers
\end{block}

\begin{block}{S4.8.2 Environmental Applications}
\protect\phantomsection\label{s4.8.2-environmental-applications}
Applying terminology analysis to environmental science:

\textbf{Climate Change Communication}: Analyzing how terminology shapes
public understanding

\textbf{Conservation Biology}: Examining anthropomorphic framing in
environmental discourse

\textbf{Ecosystem Concepts}: Understanding how human concepts are
applied to natural systems
\end{block}
\end{block}

\begin{block}{S4.9 Methodological Extensions}
\protect\phantomsection\label{s4.9-methodological-extensions}
\begin{block}{S4.9.1 Advanced Computational Methods}
\protect\phantomsection\label{s4.9.1-advanced-computational-methods}
Extending computational analysis techniques:

\textbf{Machine Learning Classification}: Using ML to classify framing
types automatically

\textbf{Network Analysis Extensions}: Applying advanced graph theory to
terminology networks

\textbf{Temporal Analysis}: Developing methods for tracking terminology
evolution
\end{block}

\begin{block}{S4.9.2 Integration with Other Methods}
\protect\phantomsection\label{s4.9.2-integration-with-other-methods}
Combining Ento-Linguistic analysis with complementary approaches:

\textbf{Citation Network Analysis}: Integrating citation patterns with
terminology usage

\textbf{Author Network Analysis}: Examining how terminology use
correlates with research communities

\textbf{Content Analysis Methods}: Combining with qualitative content
analysis techniques
\end{block}
\end{block}

\begin{block}{S4.10 Implementation and Adoption}
\protect\phantomsection\label{s4.10-implementation-and-adoption}
\begin{block}{S4.10.1 Tool Development}
\protect\phantomsection\label{s4.10.1-tool-development}
Creating practical tools for researchers:

\textbf{Terminology Analysis Software}: User-friendly tools for
analyzing scientific texts

\textbf{Writing Assistance}: Automated feedback on terminological
clarity

\textbf{Educational Resources}: Training materials for terminology
awareness
\end{block}

\begin{block}{S4.10.2 Community Building}
\protect\phantomsection\label{s4.10.2-community-building}
Developing communities of practice around terminological awareness:

\textbf{Research Networks}: Connecting researchers interested in
scientific communication

\textbf{Training Programs}: Developing curricula for terminology
education

\textbf{Standards Development}: Creating guidelines for clear scientific
writing
\end{block}
\end{block}

\begin{block}{S4.11 Long-term Vision}
\protect\phantomsection\label{s4.11-long-term-vision}
\begin{block}{S4.11.1 Transformative Potential}
\protect\phantomsection\label{s4.11.1-transformative-potential}
The long-term potential of Ento-Linguistic analysis:

\textbf{Scientific Communication Revolution}: Fundamental improvement in
how science communicates

\textbf{Interdisciplinary Integration}: Breaking down barriers between
research fields

\textbf{Knowledge Democratization}: Making scientific knowledge more
accessible
\end{block}

\begin{block}{S4.11.2 Future Research Directions}
\protect\phantomsection\label{s4.11.2-future-research-directions}
Extending the framework to new domains and applications:

\textbf{Multi-Disciplinary Expansion}: Applying methods across all
scientific disciplines

\textbf{Cross-Cultural Analysis}: Understanding how different cultures
shape scientific language

\textbf{Historical Applications}: Using terminology analysis for
understanding scientific change

\textbf{Educational Transformation}: Revolutionizing science education
through better communication

This comprehensive exploration of applications demonstrates the broad
utility of the Ento-Linguistic framework across scientific, educational,
philosophical, and societal domains, establishing it as a powerful tool
for understanding and improving scientific communication.
\end{block}
\end{block}
\end{frame}

\end{document}
