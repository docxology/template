% Options for packages loaded elsewhere
\PassOptionsToPackage{unicode}{hyperref}
\PassOptionsToPackage{hyphens}{url}
\documentclass[
  ignorenonframetext,
]{beamer}
\newif\ifbibliography
\usepackage{pgfpages}
\setbeamertemplate{caption}[numbered]
\setbeamertemplate{caption label separator}{: }
\setbeamercolor{caption name}{fg=normal text.fg}
\beamertemplatenavigationsymbolsempty
% remove section numbering
\setbeamertemplate{part page}{
  \centering
  \begin{beamercolorbox}[sep=16pt,center]{part title}
    \usebeamerfont{part title}\insertpart\par
  \end{beamercolorbox}
}
\setbeamertemplate{section page}{
  \centering
  \begin{beamercolorbox}[sep=12pt,center]{section title}
    \usebeamerfont{section title}\insertsection\par
  \end{beamercolorbox}
}
\setbeamertemplate{subsection page}{
  \centering
  \begin{beamercolorbox}[sep=8pt,center]{subsection title}
    \usebeamerfont{subsection title}\insertsubsection\par
  \end{beamercolorbox}
}
% Prevent slide breaks in the middle of a paragraph
\widowpenalties 1 10000
\raggedbottom
\AtBeginPart{
  \frame{\partpage}
}
\AtBeginSection{
  \ifbibliography
  \else
    \frame{\sectionpage}
  \fi
}
\AtBeginSubsection{
  \frame{\subsectionpage}
}
\usepackage{iftex}
\ifPDFTeX
  \usepackage[T1]{fontenc}
  \usepackage[utf8]{inputenc}
  \usepackage{textcomp} % provide euro and other symbols
\else % if luatex or xetex
  \usepackage{unicode-math} % this also loads fontspec
  \defaultfontfeatures{Scale=MatchLowercase}
  \defaultfontfeatures[\rmfamily]{Ligatures=TeX,Scale=1}
\fi
\usepackage{lmodern}
\ifPDFTeX\else
  % xetex/luatex font selection
\fi
% Use upquote if available, for straight quotes in verbatim environments
\IfFileExists{upquote.sty}{\usepackage{upquote}}{}
\IfFileExists{microtype.sty}{% use microtype if available
  \usepackage[]{microtype}
  \UseMicrotypeSet[protrusion]{basicmath} % disable protrusion for tt fonts
}{}
\makeatletter
\@ifundefined{KOMAClassName}{% if non-KOMA class
  \IfFileExists{parskip.sty}{%
    \usepackage{parskip}
  }{% else
    \setlength{\parindent}{0pt}
    \setlength{\parskip}{6pt plus 2pt minus 1pt}}
}{% if KOMA class
  \KOMAoptions{parskip=half}}
\makeatother
\setlength{\emergencystretch}{3em} % prevent overfull lines
\providecommand{\tightlist}{%
  \setlength{\itemsep}{0pt}\setlength{\parskip}{0pt}}
\usepackage{bookmark}
\IfFileExists{xurl.sty}{\usepackage{xurl}}{} % add URL line breaks if available
\urlstyle{same}
\hypersetup{
  hidelinks,
  pdfcreator={LaTeX via pandoc}}

\author{\texorpdfstring{}{}}
\date{}

\begin{document}

\begin{frame}[fragile]{Conclusion}
\protect\phantomsection\label{conclusion}
This small prose project has demonstrated the effective use of
mathematical notation, structured prose, and organizational techniques
in research communication.

\begin{block}{Summary of Contributions}
\protect\phantomsection\label{summary-of-contributions}
The project successfully showcased:

\begin{itemize}
\tightlist
\item
  \textbf{Mathematical typesetting} using LaTeX-style equations and
  notation
\item
  \textbf{Structured manuscript organization} with clear section
  hierarchy
\item
  \textbf{Bullet-point organization} for presenting key concepts and
  findings
\item
  \textbf{Cross-referencing capabilities} between sections and equations
\item
  \textbf{Table formatting} for comparative analysis presentation
\end{itemize}
\end{block}

\begin{block}{Key Takeaways}
\protect\phantomsection\label{key-takeaways}
\begin{enumerate}
\tightlist
\item
  \textbf{Mathematical Communication}

  \begin{itemize}
  \tightlist
  \item
    Clear equation presentation enhances readability
  \item
    Proper notation conventions improve comprehension
  \item
    Visual organization aids understanding of complex concepts
  \end{itemize}
\item
  \textbf{Research Documentation}

  \begin{itemize}
  \tightlist
  \item
    Structured sections provide logical flow
  \item
    Bullet points organize information effectively
  \item
    Tables present comparative data clearly
  \end{itemize}
\item
  \textbf{Pipeline Integration}

  \begin{itemize}
  \tightlist
  \item
    Manuscript-focused projects can work within the research pipeline
  \item
    Minimal source code requirements are satisfied
  \item
    Full PDF generation and validation capabilities
  \end{itemize}
\item
  \textbf{Template Capabilities}

  \begin{itemize}
  \tightlist
  \item
    Multi-project support enables diverse research approaches
  \item
    LLM integration provides automated manuscript analysis
  \item
    Executive reporting offers comprehensive project metrics
  \item
    Validation systems ensure academic quality standards
  \end{itemize}
\end{enumerate}
\end{block}

\begin{block}{Future Applications}
\protect\phantomsection\label{future-applications}
This approach can be extended to:

\begin{itemize}
\tightlist
\item
  \textbf{Educational materials} with mathematical content
\item
  \textbf{Technical documentation} requiring precise notation
\item
  \textbf{Research proposals} with theoretical foundations
\item
  \textbf{Review articles} synthesizing mathematical results
\end{itemize}
\end{block}

\begin{block}{Final Remarks}
\protect\phantomsection\label{final-remarks}
The successful completion of this prose project validates the research
pipeline's flexibility in handling diverse project types, from
code-intensive implementations to manuscript-focused theoretical work.
The ability to maintain consistent quality assurance and output
generation across different project structures demonstrates the
robustness of the underlying infrastructure.

This work contributes to the broader goal of improving research
communication through better documentation practices and mathematical
presentation standards.
\end{block}

\begin{block}{References}
\protect\phantomsection\label{references}
See \texttt{references.bib} for the complete bibliography in BibTeX
format.
\end{block}
\end{frame}

\end{document}
