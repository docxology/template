% Options for packages loaded elsewhere
\PassOptionsToPackage{unicode}{hyperref}
\PassOptionsToPackage{hyphens}{url}
\documentclass[
  ignorenonframetext,
]{beamer}
\newif\ifbibliography
\usepackage{pgfpages}
\setbeamertemplate{caption}[numbered]
\setbeamertemplate{caption label separator}{: }
\setbeamercolor{caption name}{fg=normal text.fg}
\beamertemplatenavigationsymbolsempty
% remove section numbering
\setbeamertemplate{part page}{
  \centering
  \begin{beamercolorbox}[sep=16pt,center]{part title}
    \usebeamerfont{part title}\insertpart\par
  \end{beamercolorbox}
}
\setbeamertemplate{section page}{
  \centering
  \begin{beamercolorbox}[sep=12pt,center]{section title}
    \usebeamerfont{section title}\insertsection\par
  \end{beamercolorbox}
}
\setbeamertemplate{subsection page}{
  \centering
  \begin{beamercolorbox}[sep=8pt,center]{subsection title}
    \usebeamerfont{subsection title}\insertsubsection\par
  \end{beamercolorbox}
}
% Prevent slide breaks in the middle of a paragraph
\widowpenalties 1 10000
\raggedbottom
\AtBeginPart{
  \frame{\partpage}
}
\AtBeginSection{
  \ifbibliography
  \else
    \frame{\sectionpage}
  \fi
}
\AtBeginSubsection{
  \frame{\subsectionpage}
}
\usepackage{iftex}
\ifPDFTeX
  \usepackage[T1]{fontenc}
  \usepackage[utf8]{inputenc}
  \usepackage{textcomp} % provide euro and other symbols
\else % if luatex or xetex
  \usepackage{unicode-math} % this also loads fontspec
  \defaultfontfeatures{Scale=MatchLowercase}
  \defaultfontfeatures[\rmfamily]{Ligatures=TeX,Scale=1}
\fi
\usepackage{lmodern}
\ifPDFTeX\else
  % xetex/luatex font selection
\fi
% Use upquote if available, for straight quotes in verbatim environments
\IfFileExists{upquote.sty}{\usepackage{upquote}}{}
\IfFileExists{microtype.sty}{% use microtype if available
  \usepackage[]{microtype}
  \UseMicrotypeSet[protrusion]{basicmath} % disable protrusion for tt fonts
}{}
\makeatletter
\@ifundefined{KOMAClassName}{% if non-KOMA class
  \IfFileExists{parskip.sty}{%
    \usepackage{parskip}
  }{% else
    \setlength{\parindent}{0pt}
    \setlength{\parskip}{6pt plus 2pt minus 1pt}}
}{% if KOMA class
  \KOMAoptions{parskip=half}}
\makeatother
\setlength{\emergencystretch}{3em} % prevent overfull lines
\providecommand{\tightlist}{%
  \setlength{\itemsep}{0pt}\setlength{\parskip}{0pt}}
\usepackage{bookmark}
\IfFileExists{xurl.sty}{\usepackage{xurl}}{} % add URL line breaks if available
\urlstyle{same}
\hypersetup{
  hidelinks,
  pdfcreator={LaTeX via pandoc}}

\author{\texorpdfstring{}{}}
\date{}

\begin{document}

\begin{frame}{Abstract}
\protect\phantomsection\label{abstract}
This research presents a comprehensive mathematical framework for
optimization theory, focusing on rigorous analysis of convergence
properties, stability characteristics, and computational methods. We
develop theoretical foundations for understanding optimization problems
with both inequality and equality constraints, establishing connections
between fundamental mathematical concepts including the Fundamental
Theorem of Calculus, chain rule applications, and gradient-based
optimization techniques.

Our methodology employs structured mathematical exposition to
demonstrate key theoretical results, including the relationship between
differentiation and integration, gradient definitions for multivariable
functions, and constraint-based optimization formulations. Through
rigorous mathematical analysis, we establish theoretical bounds and
provide clear frameworks for understanding complex optimization
landscapes.

The results demonstrate effective integration of mathematical notation,
structured prose, and cross-referenced equations within a
publication-quality manuscript format. We show how precise mathematical
typesetting, logical argument flow, and proper equation numbering
contribute to clear technical communication. The work contributes to
best practices in mathematical exposition by demonstrating effective
manuscript organization and integration of theoretical concepts with
computational considerations.

This research provides a foundation for understanding optimization
theory through rigorous mathematical analysis, establishing connections
between fundamental calculus principles and modern optimization
techniques. The framework developed here serves as a template for
mathematical research communication, demonstrating how complex
theoretical concepts can be effectively presented through structured
documentation and precise notation.
\end{frame}

\end{document}
