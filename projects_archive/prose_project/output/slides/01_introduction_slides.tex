% Options for packages loaded elsewhere
\PassOptionsToPackage{unicode}{hyperref}
\PassOptionsToPackage{hyphens}{url}
\documentclass[
  ignorenonframetext,
]{beamer}
\newif\ifbibliography
\usepackage{pgfpages}
\setbeamertemplate{caption}[numbered]
\setbeamertemplate{caption label separator}{: }
\setbeamercolor{caption name}{fg=normal text.fg}
\beamertemplatenavigationsymbolsempty
% remove section numbering
\setbeamertemplate{part page}{
  \centering
  \begin{beamercolorbox}[sep=16pt,center]{part title}
    \usebeamerfont{part title}\insertpart\par
  \end{beamercolorbox}
}
\setbeamertemplate{section page}{
  \centering
  \begin{beamercolorbox}[sep=12pt,center]{section title}
    \usebeamerfont{section title}\insertsection\par
  \end{beamercolorbox}
}
\setbeamertemplate{subsection page}{
  \centering
  \begin{beamercolorbox}[sep=8pt,center]{subsection title}
    \usebeamerfont{subsection title}\insertsubsection\par
  \end{beamercolorbox}
}
% Prevent slide breaks in the middle of a paragraph
\widowpenalties 1 10000
\raggedbottom
\AtBeginPart{
  \frame{\partpage}
}
\AtBeginSection{
  \ifbibliography
  \else
    \frame{\sectionpage}
  \fi
}
\AtBeginSubsection{
  \frame{\subsectionpage}
}
\usepackage{iftex}
\ifPDFTeX
  \usepackage[T1]{fontenc}
  \usepackage[utf8]{inputenc}
  \usepackage{textcomp} % provide euro and other symbols
\else % if luatex or xetex
  \usepackage{unicode-math} % this also loads fontspec
  \defaultfontfeatures{Scale=MatchLowercase}
  \defaultfontfeatures[\rmfamily]{Ligatures=TeX,Scale=1}
\fi
\usepackage{lmodern}
\ifPDFTeX\else
  % xetex/luatex font selection
\fi
% Use upquote if available, for straight quotes in verbatim environments
\IfFileExists{upquote.sty}{\usepackage{upquote}}{}
\IfFileExists{microtype.sty}{% use microtype if available
  \usepackage[]{microtype}
  \UseMicrotypeSet[protrusion]{basicmath} % disable protrusion for tt fonts
}{}
\makeatletter
\@ifundefined{KOMAClassName}{% if non-KOMA class
  \IfFileExists{parskip.sty}{%
    \usepackage{parskip}
  }{% else
    \setlength{\parindent}{0pt}
    \setlength{\parskip}{6pt plus 2pt minus 1pt}}
}{% if KOMA class
  \KOMAoptions{parskip=half}}
\makeatother
\setlength{\emergencystretch}{3em} % prevent overfull lines
\providecommand{\tightlist}{%
  \setlength{\itemsep}{0pt}\setlength{\parskip}{0pt}}
\usepackage{bookmark}
\IfFileExists{xurl.sty}{\usepackage{xurl}}{} % add URL line breaks if available
\urlstyle{same}
\hypersetup{
  hidelinks,
  pdfcreator={LaTeX via pandoc}}

\author{\texorpdfstring{}{}}
\date{}

\begin{document}

\begin{frame}{Introduction}
\protect\phantomsection\label{introduction}
This small prose project demonstrates manuscript-focused research with
mathematical equations, structured prose, and bullet-point organization.
The project contains minimal source code to satisfy pipeline
requirements but focuses on demonstrating the manuscript rendering
pipeline, including automatic title page generation from metadata
configuration.

\begin{block}{Research Context}
\protect\phantomsection\label{research-context}
Mathematical research often involves complex equations and structured
argumentation. This project showcases:

\begin{itemize}
\tightlist
\item
  \textbf{Mathematical notation} using LaTeX-style equations
\item
  \textbf{Structured prose} with clear paragraphs and sections
\item
  \textbf{Bullet-point organization} for key concepts
\item
  \textbf{Cross-references} between sections and equations
\end{itemize}
\end{block}

\begin{block}{Key Concepts}
\protect\phantomsection\label{key-concepts}
The following equation demonstrates a fundamental mathematical
relationship \cite{stewart2015calculus, apostol1974mathematical}:

\begin{equation}
\label{eq:fundamental_theorem}
\frac{d}{dx} \int_a^x f(t) \, dt = f(x)
\end{equation}

This is the Fundamental Theorem of Calculus, which connects
differentiation and integration.
\end{block}

\begin{block}{Template Capabilities Demonstrated}
\protect\phantomsection\label{template-capabilities-demonstrated}
This project showcases the research template's comprehensive
capabilities for mathematical exposition:

\begin{itemize}
\tightlist
\item
  \textbf{Multi-format rendering}: Automatic generation of PDF
  manuscripts with professional formatting
\item
  \textbf{LLM-powered analysis}: Automated scientific review and
  technical validation
\item
  \textbf{Executive reporting}: Cross-project metrics and comparative
  analysis
\item
  \textbf{Comprehensive validation}: Automated checking of mathematical
  notation and references
\item
  \textbf{Flexible project types}: Support for both code-intensive and
  prose-focused research
\end{itemize}
\end{block}

\begin{block}{Methodology}
\protect\phantomsection\label{methodology}
Our approach involves:

\begin{itemize}
\tightlist
\item
  \textbf{Theoretical analysis} of mathematical relationships
\item
  \textbf{Equation derivation} using standard techniques
\item
  \textbf{Documentation} of results in structured format
\item
  \textbf{Validation} through mathematical consistency checks
\end{itemize}
\end{block}

\begin{block}{Research Questions}
\protect\phantomsection\label{research-questions}
This project addresses:

\begin{enumerate}
\tightlist
\item
  \textbf{How can mathematical concepts be effectively communicated?}

  \begin{itemize}
  \tightlist
  \item
    Through clear prose and notation
  \item
    Using structured manuscript organization
  \item
    Employing appropriate mathematical typesetting
  \end{itemize}
\item
  \textbf{What are the key elements of mathematical exposition?}

  \begin{itemize}
  \tightlist
  \item
    Precise mathematical notation
  \item
    Logical flow of arguments
  \item
    Clear section organization
  \item
    Proper equation numbering and referencing
  \end{itemize}
\end{enumerate}
\end{block}

\begin{block}{Expected Contributions}
\protect\phantomsection\label{expected-contributions}
This work contributes to the understanding of mathematical communication
by demonstrating:

\begin{itemize}
\tightlist
\item
  Effective use of mathematical typesetting
\item
  Structured manuscript organization
\item
  Integration of prose and equations
\item
  Best practices for technical documentation
\end{itemize}
\end{block}
\end{frame}

\end{document}
