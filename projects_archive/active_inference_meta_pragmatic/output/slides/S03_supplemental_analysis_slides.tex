% Options for packages loaded elsewhere
\PassOptionsToPackage{unicode}{hyperref}
\PassOptionsToPackage{hyphens}{url}
\documentclass[
  ignorenonframetext,
]{beamer}
\newif\ifbibliography
\usepackage{pgfpages}
\setbeamertemplate{caption}[numbered]
\setbeamertemplate{caption label separator}{: }
\setbeamercolor{caption name}{fg=normal text.fg}
\beamertemplatenavigationsymbolsempty
% remove section numbering
\setbeamertemplate{part page}{
  \centering
  \begin{beamercolorbox}[sep=16pt,center]{part title}
    \usebeamerfont{part title}\insertpart\par
  \end{beamercolorbox}
}
\setbeamertemplate{section page}{
  \centering
  \begin{beamercolorbox}[sep=12pt,center]{section title}
    \usebeamerfont{section title}\insertsection\par
  \end{beamercolorbox}
}
\setbeamertemplate{subsection page}{
  \centering
  \begin{beamercolorbox}[sep=8pt,center]{subsection title}
    \usebeamerfont{subsection title}\insertsubsection\par
  \end{beamercolorbox}
}
% Prevent slide breaks in the middle of a paragraph
\widowpenalties 1 10000
\raggedbottom
\AtBeginPart{
  \frame{\partpage}
}
\AtBeginSection{
  \ifbibliography
  \else
    \frame{\sectionpage}
  \fi
}
\AtBeginSubsection{
  \frame{\subsectionpage}
}
\usepackage{iftex}
\ifPDFTeX
  \usepackage[T1]{fontenc}
  \usepackage[utf8]{inputenc}
  \usepackage{textcomp} % provide euro and other symbols
\else % if luatex or xetex
  \usepackage{unicode-math} % this also loads fontspec
  \defaultfontfeatures{Scale=MatchLowercase}
  \defaultfontfeatures[\rmfamily]{Ligatures=TeX,Scale=1}
\fi
\usepackage{lmodern}
\ifPDFTeX\else
  % xetex/luatex font selection
\fi
% Use upquote if available, for straight quotes in verbatim environments
\IfFileExists{upquote.sty}{\usepackage{upquote}}{}
\IfFileExists{microtype.sty}{% use microtype if available
  \usepackage[]{microtype}
  \UseMicrotypeSet[protrusion]{basicmath} % disable protrusion for tt fonts
}{}
\makeatletter
\@ifundefined{KOMAClassName}{% if non-KOMA class
  \IfFileExists{parskip.sty}{%
    \usepackage{parskip}
  }{% else
    \setlength{\parindent}{0pt}
    \setlength{\parskip}{6pt plus 2pt minus 1pt}}
}{% if KOMA class
  \KOMAoptions{parskip=half}}
\makeatother
\setlength{\emergencystretch}{3em} % prevent overfull lines
\providecommand{\tightlist}{%
  \setlength{\itemsep}{0pt}\setlength{\parskip}{0pt}}
\usepackage{bookmark}
\IfFileExists{xurl.sty}{\usepackage{xurl}}{} % add URL line breaks if available
\urlstyle{same}
\hypersetup{
  hidelinks,
  pdfcreator={LaTeX via pandoc}}

\author{\texorpdfstring{}{}}
\date{}

\begin{document}

\begin{frame}{Supplemental Analysis}
\protect\phantomsection\label{sec:supplemental_analysis}
This section provides extended theoretical analysis of meta-cognitive
frameworks and their implications.

\begin{block}{Meta-Cognitive Framework Analysis}
\protect\phantomsection\label{sec:meta_cognitive_frameworks}
\begin{block}{Hierarchical Meta-Cognition}
\protect\phantomsection\label{hierarchical-meta-cognition}
\textbf{Level 1 Meta-Cognition:} Monitoring basic inference processes
\textbf{Level 2 Meta-Cognition:} Monitoring meta-cognitive processes
themselves \textbf{Level 3 Meta-Cognition:} Framework-level monitoring
and adaptation
\end{block}

\begin{block}{Self-Modeling Requirements}
\protect\phantomsection\label{self-modeling-requirements}
Active Inference requires systems to model themselves within the same
formalism used to model the world, creating recursive self-reference.
\end{block}

\begin{block}{Framework Coherence}
\protect\phantomsection\label{framework-coherence}
Meta-cognitive frameworks must maintain internal consistency while
adapting to changing circumstances.
\end{block}
\end{block}

\begin{block}{Theoretical Extensions}
\protect\phantomsection\label{sec:theoretical_extensions}
\begin{block}{Multi-Agent Active Inference}
\protect\phantomsection\label{multi-agent-active-inference}
Extension of the framework to social cognition and multi-agent systems.
\end{block}

\begin{block}{Temporal Meta-Cognition}
\protect\phantomsection\label{temporal-meta-cognition}
Incorporation of temporal dynamics into meta-cognitive processing.
\end{block}

\begin{block}{Cultural Cognitive Frameworks}
\protect\phantomsection\label{cultural-cognitive-frameworks}
Analysis of how cultural contexts shape meta-cognitive frameworks.
\end{block}
\end{block}

\begin{block}{Implementation Considerations}
\protect\phantomsection\label{sec:implementation_considerations}
\begin{block}{Computational Constraints}
\protect\phantomsection\label{computational-constraints}
Practical limitations and optimization strategies for meta-level
processing.
\end{block}

\begin{block}{Learning Dynamics}
\protect\phantomsection\label{learning-dynamics}
How meta-cognitive frameworks develop and evolve over time.
\end{block}

\begin{block}{Robustness Properties}
\protect\phantomsection\label{robustness-properties}
Ensuring meta-cognitive systems remain stable under perturbation.
\end{block}
\end{block}
\end{frame}

\end{document}
