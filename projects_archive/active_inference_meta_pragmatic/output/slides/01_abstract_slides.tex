% Options for packages loaded elsewhere
\PassOptionsToPackage{unicode}{hyperref}
\PassOptionsToPackage{hyphens}{url}
\documentclass[
  ignorenonframetext,
]{beamer}
\newif\ifbibliography
\usepackage{pgfpages}
\setbeamertemplate{caption}[numbered]
\setbeamertemplate{caption label separator}{: }
\setbeamercolor{caption name}{fg=normal text.fg}
\beamertemplatenavigationsymbolsempty
% remove section numbering
\setbeamertemplate{part page}{
  \centering
  \begin{beamercolorbox}[sep=16pt,center]{part title}
    \usebeamerfont{part title}\insertpart\par
  \end{beamercolorbox}
}
\setbeamertemplate{section page}{
  \centering
  \begin{beamercolorbox}[sep=12pt,center]{section title}
    \usebeamerfont{section title}\insertsection\par
  \end{beamercolorbox}
}
\setbeamertemplate{subsection page}{
  \centering
  \begin{beamercolorbox}[sep=8pt,center]{subsection title}
    \usebeamerfont{subsection title}\insertsubsection\par
  \end{beamercolorbox}
}
% Prevent slide breaks in the middle of a paragraph
\widowpenalties 1 10000
\raggedbottom
\AtBeginPart{
  \frame{\partpage}
}
\AtBeginSection{
  \ifbibliography
  \else
    \frame{\sectionpage}
  \fi
}
\AtBeginSubsection{
  \frame{\subsectionpage}
}
\usepackage{iftex}
\ifPDFTeX
  \usepackage[T1]{fontenc}
  \usepackage[utf8]{inputenc}
  \usepackage{textcomp} % provide euro and other symbols
\else % if luatex or xetex
  \usepackage{unicode-math} % this also loads fontspec
  \defaultfontfeatures{Scale=MatchLowercase}
  \defaultfontfeatures[\rmfamily]{Ligatures=TeX,Scale=1}
\fi
\usepackage{lmodern}
\ifPDFTeX\else
  % xetex/luatex font selection
\fi
% Use upquote if available, for straight quotes in verbatim environments
\IfFileExists{upquote.sty}{\usepackage{upquote}}{}
\IfFileExists{microtype.sty}{% use microtype if available
  \usepackage[]{microtype}
  \UseMicrotypeSet[protrusion]{basicmath} % disable protrusion for tt fonts
}{}
\makeatletter
\@ifundefined{KOMAClassName}{% if non-KOMA class
  \IfFileExists{parskip.sty}{%
    \usepackage{parskip}
  }{% else
    \setlength{\parindent}{0pt}
    \setlength{\parskip}{6pt plus 2pt minus 1pt}}
}{% if KOMA class
  \KOMAoptions{parskip=half}}
\makeatother
\setlength{\emergencystretch}{3em} % prevent overfull lines
\providecommand{\tightlist}{%
  \setlength{\itemsep}{0pt}\setlength{\parskip}{0pt}}
\usepackage{bookmark}
\IfFileExists{xurl.sty}{\usepackage{xurl}}{} % add URL line breaks if available
\urlstyle{same}
\hypersetup{
  hidelinks,
  pdfcreator={LaTeX via pandoc}}

\author{\texorpdfstring{}{}}
\date{}

\begin{document}

\begin{frame}{Abstract}
\protect\phantomsection\label{sec:abstract}
Active Inference provides a unified formalism for understanding agents
that minimize variational free energy through perception and action.
Beyond a theory of surprise minimization, Active Inference operates at
the \emph{meta-level}: it is \emph{meta-pragmatic} and
\emph{meta-epistemic}, allowing modelers to specify the frameworks
within which cognition occurs.

A \(2 \times 2\) matrix (Data/Meta-Data \(\times\)
Cognitive/Meta-Cognitive) organizes Active Inference's contributions
across four quadrants. This structure reveals how Active Inference
transcends reinforcement learning by enabling specification of both
epistemic structures (what can be known: matrices \(A\), \(B\), \(D\))
and pragmatic landscapes (what matters: matrix \(C\)).

The Expected Free Energy (EFE) formulation operates at a meta-level
where modeler choices define the boundaries of both epistemic and
pragmatic domains. Unlike fixed reward functions, Active Inference makes
framework specification itself a research question.

Implications extend to cognitive security, where meta-level processing
becomes crucial for defending against manipulation of belief formation
and value structures, and to AI safety, where framework specification
provides principled value alignment.

\textbf{Keywords:} active inference, free energy principle,
meta-cognition, meta-pragmatic, meta-epistemic, cognitive science,
cognitive security, framework specification, generative models

\textbf{MSC2020:} 68T01 (Artificial intelligence), 91E10 (Cognitive
science), 92B05 (Neural networks)
\end{frame}

\end{document}
