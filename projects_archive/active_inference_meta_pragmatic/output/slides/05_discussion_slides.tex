% Options for packages loaded elsewhere
\PassOptionsToPackage{unicode}{hyperref}
\PassOptionsToPackage{hyphens}{url}
\documentclass[
  ignorenonframetext,
]{beamer}
\newif\ifbibliography
\usepackage{pgfpages}
\setbeamertemplate{caption}[numbered]
\setbeamertemplate{caption label separator}{: }
\setbeamercolor{caption name}{fg=normal text.fg}
\beamertemplatenavigationsymbolsempty
% remove section numbering
\setbeamertemplate{part page}{
  \centering
  \begin{beamercolorbox}[sep=16pt,center]{part title}
    \usebeamerfont{part title}\insertpart\par
  \end{beamercolorbox}
}
\setbeamertemplate{section page}{
  \centering
  \begin{beamercolorbox}[sep=12pt,center]{section title}
    \usebeamerfont{section title}\insertsection\par
  \end{beamercolorbox}
}
\setbeamertemplate{subsection page}{
  \centering
  \begin{beamercolorbox}[sep=8pt,center]{subsection title}
    \usebeamerfont{subsection title}\insertsubsection\par
  \end{beamercolorbox}
}
% Prevent slide breaks in the middle of a paragraph
\widowpenalties 1 10000
\raggedbottom
\AtBeginPart{
  \frame{\partpage}
}
\AtBeginSection{
  \ifbibliography
  \else
    \frame{\sectionpage}
  \fi
}
\AtBeginSubsection{
  \frame{\subsectionpage}
}
\usepackage{iftex}
\ifPDFTeX
  \usepackage[T1]{fontenc}
  \usepackage[utf8]{inputenc}
  \usepackage{textcomp} % provide euro and other symbols
\else % if luatex or xetex
  \usepackage{unicode-math} % this also loads fontspec
  \defaultfontfeatures{Scale=MatchLowercase}
  \defaultfontfeatures[\rmfamily]{Ligatures=TeX,Scale=1}
\fi
\usepackage{lmodern}
\ifPDFTeX\else
  % xetex/luatex font selection
\fi
% Use upquote if available, for straight quotes in verbatim environments
\IfFileExists{upquote.sty}{\usepackage{upquote}}{}
\IfFileExists{microtype.sty}{% use microtype if available
  \usepackage[]{microtype}
  \UseMicrotypeSet[protrusion]{basicmath} % disable protrusion for tt fonts
}{}
\makeatletter
\@ifundefined{KOMAClassName}{% if non-KOMA class
  \IfFileExists{parskip.sty}{%
    \usepackage{parskip}
  }{% else
    \setlength{\parindent}{0pt}
    \setlength{\parskip}{6pt plus 2pt minus 1pt}}
}{% if KOMA class
  \KOMAoptions{parskip=half}}
\makeatother
\setlength{\emergencystretch}{3em} % prevent overfull lines
\providecommand{\tightlist}{%
  \setlength{\itemsep}{0pt}\setlength{\parskip}{0pt}}
\usepackage{bookmark}
\IfFileExists{xurl.sty}{\usepackage{xurl}}{} % add URL line breaks if available
\urlstyle{same}
\hypersetup{
  hidelinks,
  pdfcreator={LaTeX via pandoc}}

\author{\texorpdfstring{}{}}
\date{}

\begin{document}

\begin{frame}{Discussion}
\protect\phantomsection\label{sec:discussion}
The \(2 \times 2\) matrix (Data/Meta-Data × Cognitive/Meta-Cognitive)
positions Active Inference as a meta-level methodology with far-reaching
implications for cognitive science, artificial intelligence, and our
understanding of intelligence itself. Framework specification---not just
inference---becomes the research variable.

\begin{block}{Theoretical Contributions}
\protect\phantomsection\label{sec:theoretical_contributions}
\begin{block}{Value Landscapes Beyond Scalar Rewards}
\protect\phantomsection\label{value-landscapes-beyond-scalar-rewards}
Active Inference's meta-pragmatic nature transcends traditional
approaches to goal-directed behavior. Unlike reinforcement learning,
which specifies rewards as scalar values:

\begin{equation}
R(s,a) \in \mathbb{R}
\label{eq:traditional_reward}
\end{equation}

Active Inference enables specification of preference landscapes:

\begin{equation}
C(o) \in \mathbb{R}^{|\mathcal{O}|}
\label{eq:active_inference_preferences}
\end{equation}

This supports modeling of value systems far richer than scalar rewards:
- \textbf{Complex Value Structures:} Multi-dimensional preferences with
trade-offs - \textbf{Ethical Considerations:} Moral and social values in
the preference landscape - \textbf{Contextual Goals:}
Situation-dependent value hierarchies - \textbf{Meta-Preferences:}
Preferences about preference structures themselves
\end{block}

\begin{block}{Epistemological Framework Specification}
\protect\phantomsection\label{epistemological-framework-specification}
Active Inference supports specification of epistemic frameworks through
matrices \(A\), \(B\), and \(D\), making epistemology a design
parameter:

\textbf{Empirical Framework:}

\begin{equation}
A_{\text{empirical}} = \begin{pmatrix} 0.95 & 0.05 \\ 0.05 & 0.95 \end{pmatrix}
\label{eq:empirical_framework}
\end{equation}

High confidence in observations, rapid inference.

\textbf{Skeptical Framework:}

\begin{equation}
A_{\text{skeptical}} = \begin{pmatrix} 0.6 & 0.4 \\ 0.4 & 0.6 \end{pmatrix}
\label{eq:skeptical_framework}
\end{equation}

Lower confidence, requires more evidence before committing to beliefs.

Different epistemic frameworks lead to different cognitive behaviors,
learning speeds, and adaptation patterns---enabling formal analysis of
epistemological questions previously limited to philosophical discourse.
\end{block}

\begin{block}{Recursive Self-Modeling}
\protect\phantomsection\label{recursive-self-modeling}
The framework reveals the recursive relationship between modeler and
modeled system:

\begin{enumerate}
\tightlist
\item
  Modeler uses Active Inference to model cognitive systems
\item
  Insights improve understanding of modeler's own cognition
\item
  Improved self-understanding leads to better models
\item
  Cycle continues with increasing sophistication
\end{enumerate}
\end{block}
\end{block}

\begin{block}{Methodological Advances}
\protect\phantomsection\label{sec:methodological_advances}
\begin{block}{Systematic Analysis Structure}
\protect\phantomsection\label{systematic-analysis-structure}
The quadrant structure provides tools for analyzing meta-level
phenomena: - \textbf{Clear Processing Level Distinctions:} Unambiguous
cognitive operation categories - \textbf{Hierarchical Organization:}
Higher quadrants build on lower ones - \textbf{Multi-Scale Integration:}
Processes at different scales analyzed together
\end{block}

\begin{block}{Research Design Tools}
\protect\phantomsection\label{research-design-tools}
The framework enables researchers to: - Design experiments targeting
specific quadrants - Compare interventions across processing levels -
Develop targeted cognitive enhancement strategies - Bridge biological
and artificial cognition
\end{block}

\begin{block}{Theoretical Integration}
\protect\phantomsection\label{theoretical-integration}
The framework bridges multiple traditions: - \textbf{Active Inference +
Meta-Cognition:} Formalizes self-monitoring within mathematical
structure - \textbf{FEP + Cognitive Architectures:} Shows multi-level
operation of FEP principles - \textbf{Pragmatic + Epistemic Reasoning:}
Unifies value systems and knowledge frameworks
\end{block}
\end{block}

\begin{block}{Broader Implications}
\protect\phantomsection\label{sec:broader_implications}
\begin{block}{Nature of Intelligence}
\protect\phantomsection\label{nature-of-intelligence}
Active Inference suggests intelligence emerges from: - \textbf{Epistemic
Competence:} Constructing accurate world models - \textbf{Pragmatic
Wisdom:} Effective goal-directed behavior - \textbf{Meta-Level
Reflection:} Self-awareness and adaptive control - \textbf{Framework
Flexibility:} Modifying fundamental cognitive structures

Intelligence, in this view, is framework flexibility: the capacity to
modify the structures within which cognition operates.
\end{block}

\begin{block}{Reality and Representation}
\protect\phantomsection\label{reality-and-representation}
The meta-epistemic aspect raises fundamental questions: -
\textbf{Multiple Realities:} Different epistemic frameworks construct
different worlds - \textbf{Framework Relativity:} Cognitive adequacy
depends on framework appropriateness - \textbf{Reality Construction:}
Cognition as active construction, not passive reception
\end{block}

\begin{block}{Consciousness and Self-Awareness}
\protect\phantomsection\label{consciousness-and-self-awareness}
The recursive nature of meta-cognition provides insights into
consciousness: - \textbf{Self-Modeling:} Consciousness as modeling one's
own cognitive processes - \textbf{Hierarchical Self-Awareness:} Multiple
levels of self-reflection - \textbf{Emergent Properties:} Consciousness
arising from meta-level organization
\end{block}
\end{block}

\begin{block}{Limitations}
\protect\phantomsection\label{sec:limitations}
\begin{block}{Currently Acknowledged}
\protect\phantomsection\label{currently-acknowledged}
\textbf{Empirical Validation:} The framework is primarily theoretical;
systematic empirical validation is needed to confirm quadrant
distinctions correspond to measurable processing differences.

\textbf{Computational Complexity:} Higher quadrants involve complex
optimization. Quadrant 4's framework-level optimization requires
searching high-dimensional parameter spaces, which can be
computationally expensive.

\textbf{Measurement Challenges:} Meta-level processes are difficult to
measure directly. Novel measurement techniques combining behavioral,
neural, and computational approaches are needed.

\textbf{Scale Issues:} Scaling to complex real-world systems with
thousands of states requires further development, particularly for
Quadrants 3 and 4.
\end{block}
\end{block}

\begin{block}{Future Directions}
\protect\phantomsection\label{sec:future_directions}
\begin{block}{Empirical Validation}
\protect\phantomsection\label{empirical-validation}
\begin{itemize}
\tightlist
\item
  \textbf{Experimental Paradigms:} Tasks targeting specific quadrants
\item
  \textbf{Measurement Techniques:} Novel meta-cognitive process
  assessment
\item
  \textbf{Longitudinal Studies:} Tracking meta-cognitive development
\item
  \textbf{Cross-Cultural Research:} Comparing frameworks across cultures
\end{itemize}
\end{block}

\begin{block}{Computational Development}
\protect\phantomsection\label{computational-development}
\begin{itemize}
\tightlist
\item
  \textbf{Efficient Algorithms:} Approximate methods for framework
  optimization
\item
  \textbf{Hierarchical Techniques:} Leveraging quadrant structure
\item
  \textbf{Parallel Computation:} Scaling to large systems
\end{itemize}
\end{block}

\begin{block}{Application Domains}
\protect\phantomsection\label{application-domains}
\begin{itemize}
\tightlist
\item
  \textbf{Clinical Interventions:} Therapeutic approaches targeting
  specific quadrants
\item
  \textbf{Educational Technology:} Meta-cognitive training systems
\item
  \textbf{AI Development:} Implementation in artificial cognitive
  systems
\item
  \textbf{Policy Development:} Applications of cognitive security
  insights
\end{itemize}
\end{block}

\begin{block}{Extension Possibilities}
\protect\phantomsection\label{extension-possibilities}
\begin{itemize}
\tightlist
\item
  \textbf{Multi-Agent Systems:} Extension to social cognition
\item
  \textbf{Developmental Psychology:} Cognitive development trajectories
\item
  \textbf{Quantum Extensions:} Quantum information processing
\item
  \textbf{Embodied Cognition:} Sensorimotor integration
\end{itemize}
\end{block}
\end{block}

\begin{block}{Conclusions}
\protect\phantomsection\label{sec:conclusions}
\begin{block}{Summary of Contributions}
\protect\phantomsection\label{summary-of-contributions}
We introduced a systematic \(2 \times 2\) matrix structure for analyzing
Active Inference's meta-level operation:

\begin{enumerate}
\tightlist
\item
  \textbf{Quadrant 1:} Baseline EFE computation with direct sensory
  processing
\item
  \textbf{Quadrant 2:} Extended EFE with meta-data weighting and quality
  integration
\item
  \textbf{Quadrant 3:} Hierarchical EFE with self-assessment and
  adaptive control
\item
  \textbf{Quadrant 4:} Framework-level optimization enabling cognitive
  architecture evolution
\end{enumerate}

This structure provides: - \textbf{Meta-Pragmatic Insights:} Complex
value hierarchies beyond reward functions - \textbf{Meta-Epistemic
Insights:} Epistemology as design parameter - \textbf{Security
Framework:} Systematic analysis of cognitive vulnerabilities -
\textbf{Methodological Tools:} Experimental targeting of specific
processing levels
\end{block}

\begin{block}{Unified Framework}
\protect\phantomsection\label{unified-framework}
Active Inference, through its meta-level operation, provides a unified
framework for understanding: - \textbf{Perception as Inference:}
Bayesian hypothesis testing - \textbf{Action as Free Energy
Minimization:} Goal-directed behavior - \textbf{Learning as Model
Refinement:} Generative model adaptation - \textbf{Meta-Cognition as
Self-Modeling:} Recursive cognitive awareness
\end{block}

\begin{block}{Closing Perspective}
\protect\phantomsection\label{closing-perspective}
The capacity to specify epistemic frameworks (what can be known) and
pragmatic landscapes (what matters) makes Active Inference not merely a
theory of cognition but a \textbf{meta-theory}---a methodology for
understanding how cognitive theories themselves are constructed and
evaluated.

Intelligence, ultimately, is \textbf{framework flexibility}: the
capacity to modify the structures within which cognition operates. The
quadrant structure reveals how this flexibility operates across multiple
levels, from basic data processing to fundamental cognitive architecture
evolution.
\end{block}
\end{block}
\end{frame}

\end{document}
