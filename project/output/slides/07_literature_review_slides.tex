% Options for packages loaded elsewhere
\PassOptionsToPackage{unicode}{hyperref}
\PassOptionsToPackage{hyphens}{url}
\documentclass[
  ignorenonframetext,
]{beamer}
\newif\ifbibliography
\usepackage{pgfpages}
\setbeamertemplate{caption}[numbered]
\setbeamertemplate{caption label separator}{: }
\setbeamercolor{caption name}{fg=normal text.fg}
\beamertemplatenavigationsymbolsempty
% remove section numbering
\setbeamertemplate{part page}{
  \centering
  \begin{beamercolorbox}[sep=16pt,center]{part title}
    \usebeamerfont{part title}\insertpart\par
  \end{beamercolorbox}
}
\setbeamertemplate{section page}{
  \centering
  \begin{beamercolorbox}[sep=12pt,center]{section title}
    \usebeamerfont{section title}\insertsection\par
  \end{beamercolorbox}
}
\setbeamertemplate{subsection page}{
  \centering
  \begin{beamercolorbox}[sep=8pt,center]{subsection title}
    \usebeamerfont{subsection title}\insertsubsection\par
  \end{beamercolorbox}
}
% Prevent slide breaks in the middle of a paragraph
\widowpenalties 1 10000
\raggedbottom
\AtBeginPart{
  \frame{\partpage}
}
\AtBeginSection{
  \ifbibliography
  \else
    \frame{\sectionpage}
  \fi
}
\AtBeginSubsection{
  \frame{\subsectionpage}
}
\usepackage{iftex}
\ifPDFTeX
  \usepackage[T1]{fontenc}
  \usepackage[utf8]{inputenc}
  \usepackage{textcomp} % provide euro and other symbols
\else % if luatex or xetex
  \usepackage{unicode-math} % this also loads fontspec
  \defaultfontfeatures{Scale=MatchLowercase}
  \defaultfontfeatures[\rmfamily]{Ligatures=TeX,Scale=1}
\fi
\usepackage{lmodern}
\ifPDFTeX\else
  % xetex/luatex font selection
\fi
% Use upquote if available, for straight quotes in verbatim environments
\IfFileExists{upquote.sty}{\usepackage{upquote}}{}
\IfFileExists{microtype.sty}{% use microtype if available
  \usepackage[]{microtype}
  \UseMicrotypeSet[protrusion]{basicmath} % disable protrusion for tt fonts
}{}
\makeatletter
\@ifundefined{KOMAClassName}{% if non-KOMA class
  \IfFileExists{parskip.sty}{%
    \usepackage{parskip}
  }{% else
    \setlength{\parindent}{0pt}
    \setlength{\parskip}{6pt plus 2pt minus 1pt}}
}{% if KOMA class
  \KOMAoptions{parskip=half}}
\makeatother
\setlength{\emergencystretch}{3em} % prevent overfull lines
\providecommand{\tightlist}{%
  \setlength{\itemsep}{0pt}\setlength{\parskip}{0pt}}
\usepackage{bookmark}
\IfFileExists{xurl.sty}{\usepackage{xurl}}{} % add URL line breaks if available
\urlstyle{same}
\hypersetup{
  hidelinks,
  pdfcreator={LaTeX via pandoc}}

\author{\texorpdfstring{}{}}
\date{}

\begin{document}

\section{Literature Review}\label{literature-review}

\begin{frame}{Foundational Works}
\protect\phantomsection\label{foundational-works}
\begin{block}{Laws of Form (Spencer-Brown, 1969)}
\protect\phantomsection\label{laws-of-form-spencer-brown-1969}
G. Spencer-Brown's \emph{Laws of Form} \cite{spencerbrown1969}
established the calculus of indications as a minimal foundation for
Boolean algebra. The work introduces the primary distinction---a
boundary separating inside from outside---as the fundamental cognitive
and mathematical primitive.

\textbf{Key contributions}: - Two-axiom system (Calling and Crossing) -
Nine derived consequences (C1-C9) - Imaginary Boolean values for
self-reference - Philosophical framework connecting distinction to
existence

The calculus emerged from Spencer-Brown's work as a consulting engineer,
where he sought minimal representations for switching circuits. The
resulting system transcends engineering to address foundational
questions in logic and epistemology.
\end{block}

\begin{block}{Kauffman's Extensions}
\protect\phantomsection\label{kauffmans-extensions}
Louis H. Kauffman extended boundary logic in multiple directions
\cite{kauffman2001,kauffman2005}:

\textbf{Self-Reference and Imaginary Values}: Kauffman formalized
Spencer-Brown's imaginary values, showing that the equation
\(j = \langle j \rangle\) generates temporal oscillation rather than
contradiction. This provides a constructive treatment of self-reference
unavailable in classical logic.

\textbf{Knot Theory Connections}: Kauffman demonstrated connections
between the calculus of indications and knot invariants, suggesting deep
relationships between boundary logic and topology.

\textbf{Categorical Interpretations}: Work on the categorical semantics
of boundary logic established connections to category theory and type
theory.
\end{block}

\begin{block}{Bricken's Boundary Mathematics}
\protect\phantomsection\label{brickens-boundary-mathematics}
William Bricken developed boundary logic into practical computational
tools \cite{bricken2019,bricken2021}:

\textbf{Iconic Arithmetic}: Bricken extended boundary notation to
represent natural numbers and arithmetic operations, demonstrating that
the iconic approach applies beyond Boolean logic.

\textbf{Educational Applications}: The boundary notation provides
intuitive representations suitable for teaching logic and mathematics at
various levels.

\textbf{Computational Efficiency}: Analysis of boundary representations
for circuit optimization and Boolean reasoning.
\end{block}
\end{frame}

\begin{frame}{Related Formal Systems}
\protect\phantomsection\label{related-formal-systems}
\begin{block}{Classical Set Theory}
\protect\phantomsection\label{classical-set-theory}
Zermelo-Fraenkel Set Theory with Choice (ZFC) remains the standard
foundation for mathematics \cite{kunen1980}. The comparison with
Containment Theory illuminates:

\begin{itemize}
\tightlist
\item
  \textbf{Axiomatic overhead}: ZFC requires 9+ axioms; boundary logic
  requires 2
\item
  \textbf{Self-reference handling}: ZFC restricts comprehension;
  boundary logic incorporates oscillation
\item
  \textbf{Infinity}: ZFC axiomatizes infinity; boundary logic is
  inherently finite
\end{itemize}
\end{block}

\begin{block}{Boolean Algebra}
\protect\phantomsection\label{boolean-algebra}
Boolean algebra \cite{huntington1904,stone1936} provides the standard
treatment of propositional logic. The isomorphism between boundary logic
and Boolean algebra establishes their equivalence while highlighting
representational differences:

\begin{itemize}
\tightlist
\item
  Boolean algebra uses abstract operations (∧, ∨, ¬)
\item
  Boundary logic uses spatial operations (enclosure, juxtaposition)
\item
  Both achieve functional completeness
\end{itemize}
\end{block}

\begin{block}{Category Theory}
\protect\phantomsection\label{category-theory}
Categorical approaches to logic \cite{lambek1986,awodey2010} provide
frameworks for understanding boundary logic:

\begin{itemize}
\tightlist
\item
  Forms as objects in a category
\item
  Reductions as morphisms
\item
  Axioms as natural transformations
\item
  Completeness as universal properties
\end{itemize}
\end{block}

\begin{block}{Type Theory}
\protect\phantomsection\label{type-theory}
Homotopy Type Theory \cite{hottbook} and other type-theoretic approaches
connect to boundary logic through:

\begin{itemize}
\tightlist
\item
  Types as spaces (boundaries create spaces)
\item
  Negation as complement
\item
  Self-reference as recursive types
\item
  The univalence axiom and path equivalence
\end{itemize}
\end{block}
\end{frame}

\begin{frame}{Variational and Inference Frameworks}
\protect\phantomsection\label{variational-and-inference-frameworks}
\begin{block}{Free Energy Principle}
\protect\phantomsection\label{free-energy-principle}
The free energy principle \cite{friston2010,isomura2022experimental}
provides connections to boundary logic through:

\begin{itemize}
\tightlist
\item
  Distinction as minimizing variational free energy
\item
  Boundaries as Markov blankets
\item
  Inference through boundary maintenance
\end{itemize}

Isomura et al.~\cite{isomura2022experimental} experimentally validated
the free energy principle using neural networks, demonstrating that
systems maintaining boundaries exhibit inference-like behavior.
\end{block}

\begin{block}{Active Inference}
\protect\phantomsection\label{active-inference}
Active inference frameworks
\cite{sennesh2022deriving,hinrichs2025geometric} extend the free energy
principle to action:

\begin{itemize}
\tightlist
\item
  Agents maintain boundaries through action
\item
  Perception and action unified through boundary management
\item
  Self-organization through distinction maintenance
\end{itemize}

These connections suggest boundary logic may provide formal tools for
understanding cognitive and biological systems.
\end{block}

\begin{block}{Variational Methods}
\protect\phantomsection\label{variational-methods}
Variational approaches in physics and computation
\cite{valsson2014variational,gaybalmaz2017free} share structural
features with boundary reduction:

\begin{itemize}
\tightlist
\item
  Optimization through functional minimization
\item
  Convergence to canonical states
\item
  Preservation of essential structure
\end{itemize}

The variational principle in boundary logic---reducing to canonical
forms---parallels variational methods in other domains.
\end{block}
\end{frame}

\begin{frame}{Computational Logic}
\protect\phantomsection\label{computational-logic}
\begin{block}{SAT Solving and Boolean Satisfiability}
\protect\phantomsection\label{sat-solving-and-boolean-satisfiability}
Boolean satisfiability (SAT) \cite{biere2009} relates to boundary logic
through:

\begin{itemize}
\tightlist
\item
  Both address Boolean reasoning
\item
  SAT is NP-complete (decision problem)
\item
  Boundary evaluation is polynomial (evaluation problem)
\item
  Different computational contexts
\end{itemize}
\end{block}

\begin{block}{Proof Assistants}
\protect\phantomsection\label{proof-assistants}
Formal verification systems \cite{bertot2004,nipkow2002} provide context
for boundary logic verification:

\begin{itemize}
\tightlist
\item
  Reduction traces as proof certificates
\item
  Canonical forms as normal forms
\item
  Computational verification as proof checking
\end{itemize}
\end{block}

\begin{block}{Circuit Synthesis}
\protect\phantomsection\label{circuit-synthesis}
Digital circuit design \cite{micheli1994} directly applies boundary
logic:

\begin{itemize}
\tightlist
\item
  NAND completeness corresponds to \(\langle ab \rangle\)
\item
  Reduction rules map to circuit optimization
\item
  Geometric visualization aids design
\end{itemize}
\end{block}
\end{frame}

\begin{frame}{Philosophical and Cognitive Connections}
\protect\phantomsection\label{philosophical-and-cognitive-connections}
\begin{block}{Epistemology of Distinction}
\protect\phantomsection\label{epistemology-of-distinction}
Philosophical work on distinction \cite{bateson1972,maturana1980}
connects to boundary logic:

\begin{itemize}
\tightlist
\item
  Distinction as primary cognitive act
\item
  Information as difference that makes a difference
\item
  Self-organization through recursive distinction
\end{itemize}
\end{block}

\begin{block}{Cognitive Science}
\protect\phantomsection\label{cognitive-science}
Cognitive approaches \cite{varela1991,thompson2007} find resonance with
boundary logic:

\begin{itemize}
\tightlist
\item
  Perception as distinction-making
\item
  Categories as boundaries
\item
  Self-reference as consciousness
\end{itemize}
\end{block}

\begin{block}{Cybernetics}
\protect\phantomsection\label{cybernetics}
The cybernetic tradition \cite{wiener1948,vonfoerster1981} anticipated
boundary logic concepts:

\begin{itemize}
\tightlist
\item
  Feedback and self-reference
\item
  Boundaries and systems
\item
  Observation and distinction
\end{itemize}
\end{block}
\end{frame}

\begin{frame}{Open Questions in the Literature}
\protect\phantomsection\label{open-questions-in-the-literature}
\begin{block}{Completeness}
\protect\phantomsection\label{completeness}
Is the consequence system (C1-C9) complete for all Boolean identities?
Spencer-Brown claims completeness but rigorous proofs remain debated.
\end{block}

\begin{block}{Complexity}
\protect\phantomsection\label{complexity}
Tight complexity bounds for boundary reduction and relationship to
circuit complexity classes require further investigation.
\end{block}

\begin{block}{Extensions}
\protect\phantomsection\label{extensions}
Boundary arithmetic (Bricken), predicate boundary logic, and
higher-order extensions remain active research areas.
\end{block}

\begin{block}{Applications}
\protect\phantomsection\label{applications}
Practical applications in circuit design, cognitive modeling, and
educational tools warrant systematic exploration.
\end{block}
\end{frame}

\begin{frame}{Synthesis}
\protect\phantomsection\label{synthesis}
The literature reveals boundary logic as a nexus connecting:

\begin{enumerate}
\tightlist
\item
  \textbf{Foundations}: Alternative to set-theoretic foundations
\item
  \textbf{Computation}: Circuit design and Boolean reasoning
\item
  \textbf{Cognition}: Models of distinction and self-reference
\item
  \textbf{Physics}: Variational principles and free energy
\end{enumerate}

This work contributes computational verification of the foundational
claims, enabling rigorous exploration of these connections.
\end{frame}

\end{document}
