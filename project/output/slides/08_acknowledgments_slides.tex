% Options for packages loaded elsewhere
\PassOptionsToPackage{unicode}{hyperref}
\PassOptionsToPackage{hyphens}{url}
\documentclass[
  ignorenonframetext,
]{beamer}
\newif\ifbibliography
\usepackage{pgfpages}
\setbeamertemplate{caption}[numbered]
\setbeamertemplate{caption label separator}{: }
\setbeamercolor{caption name}{fg=normal text.fg}
\beamertemplatenavigationsymbolsempty
% remove section numbering
\setbeamertemplate{part page}{
  \centering
  \begin{beamercolorbox}[sep=16pt,center]{part title}
    \usebeamerfont{part title}\insertpart\par
  \end{beamercolorbox}
}
\setbeamertemplate{section page}{
  \centering
  \begin{beamercolorbox}[sep=12pt,center]{section title}
    \usebeamerfont{section title}\insertsection\par
  \end{beamercolorbox}
}
\setbeamertemplate{subsection page}{
  \centering
  \begin{beamercolorbox}[sep=8pt,center]{subsection title}
    \usebeamerfont{subsection title}\insertsubsection\par
  \end{beamercolorbox}
}
% Prevent slide breaks in the middle of a paragraph
\widowpenalties 1 10000
\raggedbottom
\AtBeginPart{
  \frame{\partpage}
}
\AtBeginSection{
  \ifbibliography
  \else
    \frame{\sectionpage}
  \fi
}
\AtBeginSubsection{
  \frame{\subsectionpage}
}
\usepackage{iftex}
\ifPDFTeX
  \usepackage[T1]{fontenc}
  \usepackage[utf8]{inputenc}
  \usepackage{textcomp} % provide euro and other symbols
\else % if luatex or xetex
  \usepackage{unicode-math} % this also loads fontspec
  \defaultfontfeatures{Scale=MatchLowercase}
  \defaultfontfeatures[\rmfamily]{Ligatures=TeX,Scale=1}
\fi
\usepackage{lmodern}
\ifPDFTeX\else
  % xetex/luatex font selection
\fi
% Use upquote if available, for straight quotes in verbatim environments
\IfFileExists{upquote.sty}{\usepackage{upquote}}{}
\IfFileExists{microtype.sty}{% use microtype if available
  \usepackage[]{microtype}
  \UseMicrotypeSet[protrusion]{basicmath} % disable protrusion for tt fonts
}{}
\makeatletter
\@ifundefined{KOMAClassName}{% if non-KOMA class
  \IfFileExists{parskip.sty}{%
    \usepackage{parskip}
  }{% else
    \setlength{\parindent}{0pt}
    \setlength{\parskip}{6pt plus 2pt minus 1pt}}
}{% if KOMA class
  \KOMAoptions{parskip=half}}
\makeatother
\setlength{\emergencystretch}{3em} % prevent overfull lines
\providecommand{\tightlist}{%
  \setlength{\itemsep}{0pt}\setlength{\parskip}{0pt}}
\usepackage{bookmark}
\IfFileExists{xurl.sty}{\usepackage{xurl}}{} % add URL line breaks if available
\urlstyle{same}
\hypersetup{
  hidelinks,
  pdfcreator={LaTeX via pandoc}}

\author{\texorpdfstring{}{}}
\date{}

\begin{document}

\begin{frame}{Acknowledgments}
\protect\phantomsection\label{sec:acknowledgments}
This comprehensive review and computational framework synthesizes 4,000+
years of accumulated grafting knowledge, drawing from diverse sources
across agricultural science, plant biology, and horticultural practice.

\begin{block}{Historical Knowledge}
\protect\phantomsection\label{historical-knowledge}
We acknowledge the countless generations of agricultural practitioners,
from ancient Mesopotamian and Chinese grafters to contemporary
horticultural researchers, whose empirical observations and innovations
form the foundation of this work.
\end{block}

\begin{block}{Scientific Literature}
\protect\phantomsection\label{scientific-literature}
This research builds upon foundational works in grafting biology
\cite{melnyk2018, goldschmidt2014}, horticultural practice
\cite{garner2013, hartmann2014}, and rootstock development
\cite{webster2002}, among many others cited throughout this manuscript.
\end{block}

\begin{block}{Computational Infrastructure}
\protect\phantomsection\label{computational-infrastructure}
The computational toolkit was developed using open-source scientific
computing resources:

\begin{itemize}
\tightlist
\item
  Python scientific computing stack (NumPy, SciPy, Matplotlib) for
  numerical analysis and visualization
\item
  LaTeX, Pandoc, and XeLaTeX for professional document preparation
\item
  Research Project Template framework for reproducible research
  workflows
\end{itemize}
\end{block}

\begin{block}{Traditional Knowledge Systems}
\protect\phantomsection\label{traditional-knowledge-systems}
We recognize the importance of traditional grafting knowledge systems
across cultures---Mediterranean, Asian, Indigenous, and others---whose
practices have been refined through millennia of observation and
adaptation. This work attempts to honor these traditions by integrating
them with modern scientific understanding.
\end{block}

\begin{block}{Educational Mission}
\protect\phantomsection\label{educational-mission}
This project is dedicated to making grafting knowledge accessible to
students, practitioners, researchers, and enthusiasts across the
agricultural sciences. The integration of comprehensive documentation
with practical computational tools aims to support both learning and
application.
\end{block}
\end{frame}

\begin{frame}
\emph{All errors and interpretations remain the sole responsibility of
the author. This work represents an ongoing synthesis of grafting
science, and contributions, corrections, and extensions are welcomed.}
\end{frame}

\end{document}
