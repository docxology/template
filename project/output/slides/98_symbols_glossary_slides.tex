% Options for packages loaded elsewhere
\PassOptionsToPackage{unicode}{hyperref}
\PassOptionsToPackage{hyphens}{url}
\documentclass[
  ignorenonframetext,
]{beamer}
\newif\ifbibliography
\usepackage{pgfpages}
\setbeamertemplate{caption}[numbered]
\setbeamertemplate{caption label separator}{: }
\setbeamercolor{caption name}{fg=normal text.fg}
\beamertemplatenavigationsymbolsempty
% remove section numbering
\setbeamertemplate{part page}{
  \centering
  \begin{beamercolorbox}[sep=16pt,center]{part title}
    \usebeamerfont{part title}\insertpart\par
  \end{beamercolorbox}
}
\setbeamertemplate{section page}{
  \centering
  \begin{beamercolorbox}[sep=12pt,center]{section title}
    \usebeamerfont{section title}\insertsection\par
  \end{beamercolorbox}
}
\setbeamertemplate{subsection page}{
  \centering
  \begin{beamercolorbox}[sep=8pt,center]{subsection title}
    \usebeamerfont{subsection title}\insertsubsection\par
  \end{beamercolorbox}
}
% Prevent slide breaks in the middle of a paragraph
\widowpenalties 1 10000
\raggedbottom
\AtBeginPart{
  \frame{\partpage}
}
\AtBeginSection{
  \ifbibliography
  \else
    \frame{\sectionpage}
  \fi
}
\AtBeginSubsection{
  \frame{\subsectionpage}
}
\usepackage{iftex}
\ifPDFTeX
  \usepackage[T1]{fontenc}
  \usepackage[utf8]{inputenc}
  \usepackage{textcomp} % provide euro and other symbols
\else % if luatex or xetex
  \usepackage{unicode-math} % this also loads fontspec
  \defaultfontfeatures{Scale=MatchLowercase}
  \defaultfontfeatures[\rmfamily]{Ligatures=TeX,Scale=1}
\fi
\usepackage{lmodern}
\ifPDFTeX\else
  % xetex/luatex font selection
\fi
% Use upquote if available, for straight quotes in verbatim environments
\IfFileExists{upquote.sty}{\usepackage{upquote}}{}
\IfFileExists{microtype.sty}{% use microtype if available
  \usepackage[]{microtype}
  \UseMicrotypeSet[protrusion]{basicmath} % disable protrusion for tt fonts
}{}
\makeatletter
\@ifundefined{KOMAClassName}{% if non-KOMA class
  \IfFileExists{parskip.sty}{%
    \usepackage{parskip}
  }{% else
    \setlength{\parindent}{0pt}
    \setlength{\parskip}{6pt plus 2pt minus 1pt}}
}{% if KOMA class
  \KOMAoptions{parskip=half}}
\makeatother
\usepackage{longtable,booktabs,array}
\newcounter{none} % for unnumbered tables
\usepackage{calc} % for calculating minipage widths
\usepackage{caption}
% Make caption package work with longtable
\makeatletter
\def\fnum@table{\tablename~\thetable}
\makeatother
\setlength{\emergencystretch}{3em} % prevent overfull lines
\providecommand{\tightlist}{%
  \setlength{\itemsep}{0pt}\setlength{\parskip}{0pt}}
\usepackage{bookmark}
\IfFileExists{xurl.sty}{\usepackage{xurl}}{} % add URL line breaks if available
\urlstyle{same}
\hypersetup{
  hidelinks,
  pdfcreator={LaTeX via pandoc}}

\author{\texorpdfstring{}{}}
\date{}

\begin{document}

\section{Symbols and Glossary}\label{symbols-and-glossary}

\begin{frame}{Primary Symbols}
\protect\phantomsection\label{primary-symbols}
{\def\LTcaptype{none} % do not increment counter
\begin{longtable}[]{@{}
  >{\raggedright\arraybackslash}p{(\linewidth - 4\tabcolsep) * \real{0.2963}}
  >{\raggedright\arraybackslash}p{(\linewidth - 4\tabcolsep) * \real{0.2222}}
  >{\raggedright\arraybackslash}p{(\linewidth - 4\tabcolsep) * \real{0.4815}}@{}}
\toprule\noalign{}
\begin{minipage}[b]{\linewidth}\raggedright
Symbol
\end{minipage} & \begin{minipage}[b]{\linewidth}\raggedright
Name
\end{minipage} & \begin{minipage}[b]{\linewidth}\raggedright
Description
\end{minipage} \\
\midrule\noalign{}
\endhead
\(\langle\ \rangle\) & Mark / Cross & The primary distinction;
represents TRUE \\
\(\emptyset\) & Void & Empty space; represents FALSE \\
\(\langle a \rangle\) & Enclosure & Boundary containing form \(a\);
represents NOT \(a\) \\
\(ab\) & Juxtaposition & Forms side-by-side; represents \(a\) AND
\(b\) \\
\(j\) & Imaginary value & Self-referential form:
\(j = \langle j \rangle\) \\
\bottomrule\noalign{}
\end{longtable}
}
\end{frame}

\begin{frame}{Derived Symbols}
\protect\phantomsection\label{derived-symbols}
{\def\LTcaptype{none} % do not increment counter
\begin{longtable}[]{@{}
  >{\raggedright\arraybackslash}p{(\linewidth - 4\tabcolsep) * \real{0.2051}}
  >{\raggedright\arraybackslash}p{(\linewidth - 4\tabcolsep) * \real{0.3077}}
  >{\raggedright\arraybackslash}p{(\linewidth - 4\tabcolsep) * \real{0.4872}}@{}}
\toprule\noalign{}
\begin{minipage}[b]{\linewidth}\raggedright
Symbol
\end{minipage} & \begin{minipage}[b]{\linewidth}\raggedright
Definition
\end{minipage} & \begin{minipage}[b]{\linewidth}\raggedright
Boolean Equivalent
\end{minipage} \\
\midrule\noalign{}
\endhead
\(\langle\langle a \rangle\langle b \rangle\rangle\) & De Morgan
disjunction & \(a\) OR \(b\) \\
\(\langle a\langle b \rangle\rangle\) & Material implication &
\(a \to b\) \\
\(\langle ab \rangle\) & Sheffer stroke & \(a\) NAND \(b\) \\
\(\langle\langle\langle a \rangle\langle b \rangle\rangle\rangle\) &
Peirce arrow & \(a\) NOR \(b\) \\
\bottomrule\noalign{}
\end{longtable}
}
\end{frame}

\begin{frame}{Meta-Symbols}
\protect\phantomsection\label{meta-symbols}
{\def\LTcaptype{none} % do not increment counter
\begin{longtable}[]{@{}ll@{}}
\toprule\noalign{}
Symbol & Meaning \\
\midrule\noalign{}
\endhead
\(\llbracket f \rrbracket\) & Truth value of form \(f\) \\
\(\equiv\) & Semantic equivalence \\
\(=\) & Syntactic equality after reduction \\
\(\to\) & Reduces to (single step) \\
\(\to^*\) & Reduces to (multiple steps) \\
\bottomrule\noalign{}
\end{longtable}
}
\end{frame}

\begin{frame}{Axiom Labels}
\protect\phantomsection\label{axiom-labels}
{\def\LTcaptype{none} % do not increment counter
\begin{longtable}[]{@{}
  >{\raggedright\arraybackslash}p{(\linewidth - 4\tabcolsep) * \real{0.2917}}
  >{\raggedright\arraybackslash}p{(\linewidth - 4\tabcolsep) * \real{0.2500}}
  >{\raggedright\arraybackslash}p{(\linewidth - 4\tabcolsep) * \real{0.4583}}@{}}
\toprule\noalign{}
\begin{minipage}[b]{\linewidth}\raggedright
Label
\end{minipage} & \begin{minipage}[b]{\linewidth}\raggedright
Name
\end{minipage} & \begin{minipage}[b]{\linewidth}\raggedright
Statement
\end{minipage} \\
\midrule\noalign{}
\endhead
J1 & Calling / Involution & \(\langle\langle a \rangle\rangle = a\) \\
J2 & Crossing / Condensation &
\(\langle\ \rangle\langle\ \rangle = \langle\ \rangle\) \\
\bottomrule\noalign{}
\end{longtable}
}
\end{frame}

\begin{frame}{Consequence Labels (C1-C9)}
\protect\phantomsection\label{consequence-labels-c1-c9}
{\def\LTcaptype{none} % do not increment counter
\begin{longtable}[]{@{}
  >{\raggedright\arraybackslash}p{(\linewidth - 4\tabcolsep) * \real{0.2917}}
  >{\raggedright\arraybackslash}p{(\linewidth - 4\tabcolsep) * \real{0.2500}}
  >{\raggedright\arraybackslash}p{(\linewidth - 4\tabcolsep) * \real{0.4583}}@{}}
\toprule\noalign{}
\begin{minipage}[b]{\linewidth}\raggedright
Label
\end{minipage} & \begin{minipage}[b]{\linewidth}\raggedright
Name
\end{minipage} & \begin{minipage}[b]{\linewidth}\raggedright
Statement
\end{minipage} \\
\midrule\noalign{}
\endhead
C1 & Position & \(\langle\langle a \rangle b \rangle a = a\) \\
C2 & Transposition &
\(\langle\langle a \rangle\langle b \rangle\rangle c = \langle ac \rangle\langle bc \rangle\) \\
C3 & Generation &
\(\langle\langle a \rangle a \rangle = \langle\ \rangle\) \\
C4 & Integration & \(a \lor \text{TRUE} = \text{TRUE}\) \\
C5 & Occultation & \(\langle\langle a \rangle\rangle a = a\) \\
C6 & Iteration & \(aa = a\) \\
C7 & Extension &
\(\langle\langle a \rangle\langle b \rangle\rangle\langle\langle a \rangle b \rangle = a\) \\
C8 & Echelon &
\(\langle\langle ab \rangle c \rangle = \langle ac \rangle\langle bc \rangle\) \\
C9 & Cross-Transposition &
\(\langle\langle ac \rangle\langle bc \rangle\rangle = \langle\langle a \rangle\langle b \rangle\rangle c\) \\
\bottomrule\noalign{}
\end{longtable}
}
\end{frame}

\begin{frame}{Glossary}
\protect\phantomsection\label{glossary}
\begin{block}{Agential Cut}
\protect\phantomsection\label{agential-cut}
(Barad) An enacted boundary that constitutes the entities it separates;
parallels the Spencer-Brown mark as constitutive rather than
representational.
\end{block}

\begin{block}{Boundary}
\protect\phantomsection\label{boundary}
A line of demarcation creating inside and outside; the fundamental
operation in the calculus of indications.
\end{block}

\begin{block}{Boundary Logic}
\protect\phantomsection\label{boundary-logic}
A logical system built from the primitive act of drawing distinctions
(boundaries); synonymous with the calculus of indications and
Containment Theory.
\end{block}

\begin{block}{Calling}
\protect\phantomsection\label{calling}
Axiom J1: Double enclosure returns to the original form. Also known as
involution or double negation elimination.
\end{block}

\begin{block}{Calculus of Indications}
\protect\phantomsection\label{calculus-of-indications}
Spencer-Brown's original name for the formal system of boundary logic;
the calculus built from the primitive notion of distinction.
\end{block}

\begin{block}{Canonical Form}
\protect\phantomsection\label{canonical-form}
The irreducible form of an expression after all reduction rules have
been applied. Only void and mark are canonical.
\end{block}

\begin{block}{Condensation}
\protect\phantomsection\label{condensation}
See Crossing.
\end{block}

\begin{block}{Confluence}
\protect\phantomsection\label{confluence}
The property that all reduction sequences from a given form lead to the
same canonical form (also called the Church-Rosser property).
\end{block}

\begin{block}{Containment Theory}
\protect\phantomsection\label{containment-theory}
The approach to mathematical foundations using spatial containment
(boundaries) rather than set membership.
\end{block}

\begin{block}{Crossing}
\protect\phantomsection\label{crossing}
Axiom J2: Multiple marks in juxtaposition condense to a single mark.
Also known as condensation.
\end{block}

\begin{block}{Distinction}
\protect\phantomsection\label{distinction}
The fundamental act of separating this from that; the primitive notion
in the calculus of indications.
\end{block}

\begin{block}{Enclosure}
\protect\phantomsection\label{enclosure}
The operation of placing a boundary around a form; corresponds to
logical negation.
\end{block}

\begin{block}{Existential Graphs}
\protect\phantomsection\label{existential-graphs}
C.S. Peirce's diagrammatic logic system, a precursor to Spencer-Brown's
calculus.
\end{block}

\begin{block}{Form}
\protect\phantomsection\label{form}
Any well-formed expression in the calculus of indications, built from
void, mark, enclosure, and juxtaposition.
\end{block}

\begin{block}{Ground Form}
\protect\phantomsection\label{ground-form}
A form without variables, where all values are concrete (either void or
mark). Ground forms can be directly evaluated to canonical form.
\end{block}

\begin{block}{Icon}
\protect\phantomsection\label{icon}
(Peirce) A sign that represents by resembling what it signifies; the
mark is iconic of distinction.
\end{block}

\begin{block}{Isomorphism}
\protect\phantomsection\label{isomorphism}
A structure-preserving mapping between two mathematical systems that
shows they are essentially equivalent. Boundary logic is isomorphic to
Boolean algebra.
\end{block}

\begin{block}{Imaginary Value}
\protect\phantomsection\label{imaginary-value}
A self-referential form satisfying \(j = \langle j \rangle\); neither
marked nor void but oscillating between states.
\end{block}

\begin{block}{Indication}
\protect\phantomsection\label{indication}
The act of pointing to or marking; the fundamental operation in Laws of
Form.
\end{block}

\begin{block}{Intra-action}
\protect\phantomsection\label{intra-action}
(Barad) Mutual constitution of entities through their interaction;
parallels how forms co-determine through juxtaposition.
\end{block}

\begin{block}{Juxtaposition}
\protect\phantomsection\label{juxtaposition}
Placing forms side by side; corresponds to logical conjunction (AND).
\end{block}

\begin{block}{Laws of Form}
\protect\phantomsection\label{laws-of-form}
G. Spencer-Brown's 1969 book introducing the calculus of indications.
\end{block}

\begin{block}{Mark}
\protect\phantomsection\label{mark}
The symbol \(\langle\ \rangle\) representing the primary distinction;
corresponds to TRUE.
\end{block}

\begin{block}{Pragmatism}
\protect\phantomsection\label{pragmatism}
North American philosophical tradition emphasizing consequences,
practice, and operational meaning; grounds boundary logic's emphasis on
reduction as meaning.
\end{block}

\begin{block}{Primary Distinction}
\protect\phantomsection\label{primary-distinction}
The fundamental cognitive act of creating a boundary; the primitive of
the calculus.
\end{block}

\begin{block}{Reduction}
\protect\phantomsection\label{reduction}
The process of applying axioms to simplify a form toward its canonical
representation.
\end{block}

\begin{block}{Self-Reference}
\protect\phantomsection\label{self-reference}
A form that contains itself as a subform; leads to imaginary values in
boundary logic.
\end{block}

\begin{block}{Void}
\protect\phantomsection\label{void}
The empty space containing no marks; corresponds to FALSE. Also called
the unmarked state.
\end{block}

\begin{block}{ZFC}
\protect\phantomsection\label{zfc}
Zermelo-Fraenkel Set Theory with Choice; the standard axiomatic
foundation for mathematics, contrasted with Containment Theory.
\end{block}
\end{frame}

\end{document}
