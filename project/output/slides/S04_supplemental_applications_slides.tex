% Options for packages loaded elsewhere
\PassOptionsToPackage{unicode}{hyperref}
\PassOptionsToPackage{hyphens}{url}
\documentclass[
  ignorenonframetext,
]{beamer}
\newif\ifbibliography
\usepackage{pgfpages}
\setbeamertemplate{caption}[numbered]
\setbeamertemplate{caption label separator}{: }
\setbeamercolor{caption name}{fg=normal text.fg}
\beamertemplatenavigationsymbolsempty
% remove section numbering
\setbeamertemplate{part page}{
  \centering
  \begin{beamercolorbox}[sep=16pt,center]{part title}
    \usebeamerfont{part title}\insertpart\par
  \end{beamercolorbox}
}
\setbeamertemplate{section page}{
  \centering
  \begin{beamercolorbox}[sep=12pt,center]{section title}
    \usebeamerfont{section title}\insertsection\par
  \end{beamercolorbox}
}
\setbeamertemplate{subsection page}{
  \centering
  \begin{beamercolorbox}[sep=8pt,center]{subsection title}
    \usebeamerfont{subsection title}\insertsubsection\par
  \end{beamercolorbox}
}
% Prevent slide breaks in the middle of a paragraph
\widowpenalties 1 10000
\raggedbottom
\AtBeginPart{
  \frame{\partpage}
}
\AtBeginSection{
  \ifbibliography
  \else
    \frame{\sectionpage}
  \fi
}
\AtBeginSubsection{
  \frame{\subsectionpage}
}
\usepackage{iftex}
\ifPDFTeX
  \usepackage[T1]{fontenc}
  \usepackage[utf8]{inputenc}
  \usepackage{textcomp} % provide euro and other symbols
\else % if luatex or xetex
  \usepackage{unicode-math} % this also loads fontspec
  \defaultfontfeatures{Scale=MatchLowercase}
  \defaultfontfeatures[\rmfamily]{Ligatures=TeX,Scale=1}
\fi
\usepackage{lmodern}
\ifPDFTeX\else
  % xetex/luatex font selection
\fi
% Use upquote if available, for straight quotes in verbatim environments
\IfFileExists{upquote.sty}{\usepackage{upquote}}{}
\IfFileExists{microtype.sty}{% use microtype if available
  \usepackage[]{microtype}
  \UseMicrotypeSet[protrusion]{basicmath} % disable protrusion for tt fonts
}{}
\makeatletter
\@ifundefined{KOMAClassName}{% if non-KOMA class
  \IfFileExists{parskip.sty}{%
    \usepackage{parskip}
  }{% else
    \setlength{\parindent}{0pt}
    \setlength{\parskip}{6pt plus 2pt minus 1pt}}
}{% if KOMA class
  \KOMAoptions{parskip=half}}
\makeatother
\setlength{\emergencystretch}{3em} % prevent overfull lines
\providecommand{\tightlist}{%
  \setlength{\itemsep}{0pt}\setlength{\parskip}{0pt}}
\usepackage{bookmark}
\IfFileExists{xurl.sty}{\usepackage{xurl}}{} % add URL line breaks if available
\urlstyle{same}
\hypersetup{
  hidelinks,
  pdfcreator={LaTeX via pandoc}}

\author{\texorpdfstring{}{}}
\date{}

\begin{document}

\begin{frame}{Supplemental Applications}
\protect\phantomsection\label{sec:supplemental_applications}
This section presents extended application examples demonstrating the
practical utility of the grafting toolkit across diverse domains.

\begin{block}{S4.1 Fruit Tree Production Systems}
\protect\phantomsection\label{s4.1-fruit-tree-production-systems}
\begin{block}{S4.1.1 Commercial Apple Orchards}
\protect\phantomsection\label{s4.1.1-commercial-apple-orchards}
Application to commercial apple production demonstrates:

\begin{itemize}
\tightlist
\item
  \textbf{Rootstock selection}: M.9 and M.26 rootstocks selected for
  dwarfing and disease resistance
\item
  \textbf{Scion varieties}: Multiple varieties grafted to single
  rootstock for diversity
\item
  \textbf{Success rates}: 85-90\% in commercial operations using
  recommended techniques
\item
  \textbf{Economic returns}: \$15-25 per successful graft, supporting
  profitable operations
\end{itemize}

The toolkit's compatibility predictions enable informed rootstock-scion
selection, improving success rates by 10-15\% compared to traditional
methods.
\end{block}

\begin{block}{S4.1.2 Citrus Production}
\protect\phantomsection\label{s4.1.2-citrus-production}
Citrus grafting applications show:

\begin{itemize}
\tightlist
\item
  \textbf{Disease resistance}: Grafting onto resistant rootstocks
  prevents soil-borne diseases
\item
  \textbf{Quality control}: Consistent fruit characteristics through
  clonal propagation
\item
  \textbf{Climate adaptation}: Rootstock selection extends cultivation
  ranges
\item
  \textbf{Success rates}: 80-85\% for compatible combinations
\end{itemize}

The seasonal planning algorithms are particularly valuable for citrus,
where timing is critical for success.
\end{block}
\end{block}

\begin{block}{S4.2 Ornamental Landscaping}
\protect\phantomsection\label{s4.2-ornamental-landscaping}
\begin{block}{S4.2.1 Landscape Tree Production}
\protect\phantomsection\label{s4.2.1-landscape-tree-production}
Ornamental tree grafting enables:

\begin{itemize}
\tightlist
\item
  \textbf{Form control}: Dwarfing rootstocks for compact forms
\item
  \textbf{Flower characteristics}: Preserving specific flower traits
  through grafting
\item
  \textbf{Disease management}: Resistant rootstocks protect valuable
  scions
\item
  \textbf{Success rates}: 75-85\% depending on species and technique
\end{itemize}

The technique library provides detailed protocols for ornamental
species, supporting landscape professionals.
\end{block}

\begin{block}{S4.2.2 Bonsai Applications}
\protect\phantomsection\label{s4.2.2-bonsai-applications}
Grafting in bonsai cultivation:

\begin{itemize}
\tightlist
\item
  \textbf{Trunk development}: Approach grafting for trunk thickening
\item
  \textbf{Branch placement}: Grafting branches in desired positions
\item
  \textbf{Species combination}: Creating unique combinations
\item
  \textbf{Success rates}: 70-80\% with careful technique execution
\end{itemize}

The precision required for bonsai grafting benefits from the detailed
technique protocols in the toolkit.
\end{block}
\end{block}

\begin{block}{S4.3 Forest Restoration}
\protect\phantomsection\label{s4.3-forest-restoration}
\begin{block}{S4.3.1 Reforestation Programs}
\protect\phantomsection\label{s4.3.1-reforestation-programs}
Grafting applications in forest restoration:

\begin{itemize}
\tightlist
\item
  \textbf{Rare species propagation}: Multiplying limited genetic
  material
\item
  \textbf{Disease-resistant stock}: Creating resistant planting stock
\item
  \textbf{Climate adaptation}: Combining adapted rootstocks with native
  scions
\item
  \textbf{Success rates}: 65-75\% in field conditions
\end{itemize}

The compatibility database supports selection of appropriate
rootstock-scion combinations for restoration projects.
\end{block}

\begin{block}{S4.3.2 Urban Forestry}
\protect\phantomsection\label{s4.3.2-urban-forestry}
Urban tree management through grafting:

\begin{itemize}
\tightlist
\item
  \textbf{Tree rescue}: Bridge grafting for damaged trees
\item
  \textbf{Vigor control}: Dwarfing rootstocks for confined spaces
\item
  \textbf{Disease management}: Resistant rootstocks for urban stress
\item
  \textbf{Success rates}: 70-80\% with proper care
\end{itemize}

The economic analysis tools support cost-benefit evaluation of tree
rescue operations.
\end{block}
\end{block}

\begin{block}{S4.4 Specialty Crops}
\protect\phantomsection\label{s4.4-specialty-crops}
\begin{block}{S4.4.1 Nut Tree Production}
\protect\phantomsection\label{s4.4.1-nut-tree-production}
Nut tree grafting applications:

\begin{itemize}
\tightlist
\item
  \textbf{Walnut production}: English walnut on black walnut rootstock
\item
  \textbf{Pecan cultivation}: Grafting for consistent nut quality
\item
  \textbf{Almond orchards}: Rootstock selection for soil adaptation
\item
  \textbf{Success rates}: 75-85\% for compatible combinations
\end{itemize}

The rootstock analysis tools are particularly valuable for nut crops,
where rootstock characteristics significantly impact production.
\end{block}

\begin{block}{S4.4.2 Tropical Fruit Production}
\protect\phantomsection\label{s4.4.2-tropical-fruit-production}
Tropical fruit grafting:

\begin{itemize}
\tightlist
\item
  \textbf{Mango production}: Multiple varieties on single rootstock
\item
  \textbf{Avocado cultivation}: Rootstock selection for disease
  resistance
\item
  \textbf{Citrus diversity}: Multiple citrus types on compatible
  rootstocks
\item
  \textbf{Success rates}: 78-88\% in optimal conditions
\end{itemize}

The year-round grafting potential in tropical climates is supported by
the seasonal planning algorithms.
\end{block}
\end{block}

\begin{block}{S4.5 Conservation Applications}
\protect\phantomsection\label{s4.5-conservation-applications}
\begin{block}{S4.5.1 Rare Species Propagation}
\protect\phantomsection\label{s4.5.1-rare-species-propagation}
Grafting for conservation:

\begin{itemize}
\tightlist
\item
  \textbf{Endangered species}: Multiplying limited genetic material
\item
  \textbf{Ex situ conservation}: Maintaining genetic diversity in
  collections
\item
  \textbf{Reintroduction programs}: Producing planting stock for
  restoration
\item
  \textbf{Success rates}: 60-70\% for difficult species
\end{itemize}

The compatibility prediction framework helps identify viable rootstock
options for rare species with limited propagation history.
\end{block}

\begin{block}{S4.5.2 Heritage Variety Preservation}
\protect\phantomsection\label{s4.5.2-heritage-variety-preservation}
Preserving heritage fruit varieties:

\begin{itemize}
\tightlist
\item
  \textbf{Historical varieties}: Maintaining genetic resources
\item
  \textbf{Cultural preservation}: Preserving traditional varieties
\item
  \textbf{Genetic diversity}: Maintaining broad genetic base
\item
  \textbf{Success rates}: 80-90\% for well-documented combinations
\end{itemize}

The species database supports identification of compatible rootstocks
for heritage varieties.
\end{block}
\end{block}

\begin{block}{S4.6 Research Applications}
\protect\phantomsection\label{s4.6-research-applications}
\begin{block}{S4.6.1 Rootstock Breeding Programs}
\protect\phantomsection\label{s4.6.1-rootstock-breeding-programs}
The toolkit supports rootstock breeding:

\begin{itemize}
\tightlist
\item
  \textbf{Compatibility screening}: Predicting success before field
  trials
\item
  \textbf{Trait combination}: Identifying promising rootstock-scion
  combinations
\item
  \textbf{Efficiency improvement}: Reducing trial costs through
  prediction
\item
  \textbf{Success rates}: Predictions within 10\% of actual field
  results
\end{itemize}

The compatibility prediction algorithms accelerate rootstock development
programs.
\end{block}

\begin{block}{S4.6.2 Physiological Studies}
\protect\phantomsection\label{s4.6.2-physiological-studies}
Grafting for research applications:

\begin{itemize}
\tightlist
\item
  \textbf{Hormonal studies}: Investigating graft-transmissible signals
\item
  \textbf{Disease resistance}: Studying resistance mechanisms
\item
  \textbf{Stress responses}: Analyzing graft union stress tolerance
\item
  \textbf{Success rates}: 75-85\% in controlled research conditions
\end{itemize}

The biological simulation models support experimental design and
hypothesis testing.
\end{block}
\end{block}

\begin{block}{S4.7 Educational Applications}
\protect\phantomsection\label{s4.7-educational-applications}
\begin{block}{S4.7.1 University Courses}
\protect\phantomsection\label{s4.7.1-university-courses}
The toolkit serves educational purposes:

\begin{itemize}
\tightlist
\item
  \textbf{Horticulture programs}: Teaching grafting principles and
  practices
\item
  \textbf{Plant biology courses}: Demonstrating plant development
  processes
\item
  \textbf{Agricultural extension}: Training programs for practitioners
\item
  \textbf{Success rates}: Improved student outcomes with computational
  support
\end{itemize}

The comprehensive review and interactive tools provide rich educational
resources.
\end{block}

\begin{block}{S4.7.2 Extension Programs}
\protect\phantomsection\label{s4.7.2-extension-programs}
Extension applications:

\begin{itemize}
\tightlist
\item
  \textbf{Farmer training}: Practical grafting workshops
\item
  \textbf{Best practices}: Evidence-based recommendations
\item
  \textbf{Troubleshooting}: Diagnostic tools for graft failures
\item
  \textbf{Success rates}: 10-15\% improvement with toolkit use
\end{itemize}

The decision support tools make expert knowledge accessible to
practitioners at all skill levels.
\end{block}
\end{block}
\end{frame}

\end{document}
