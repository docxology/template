% Options for packages loaded elsewhere
\PassOptionsToPackage{unicode}{hyperref}
\PassOptionsToPackage{hyphens}{url}
\documentclass[
  ignorenonframetext,
]{beamer}
\newif\ifbibliography
\usepackage{pgfpages}
\setbeamertemplate{caption}[numbered]
\setbeamertemplate{caption label separator}{: }
\setbeamercolor{caption name}{fg=normal text.fg}
\beamertemplatenavigationsymbolsempty
% remove section numbering
\setbeamertemplate{part page}{
  \centering
  \begin{beamercolorbox}[sep=16pt,center]{part title}
    \usebeamerfont{part title}\insertpart\par
  \end{beamercolorbox}
}
\setbeamertemplate{section page}{
  \centering
  \begin{beamercolorbox}[sep=12pt,center]{section title}
    \usebeamerfont{section title}\insertsection\par
  \end{beamercolorbox}
}
\setbeamertemplate{subsection page}{
  \centering
  \begin{beamercolorbox}[sep=8pt,center]{subsection title}
    \usebeamerfont{subsection title}\insertsubsection\par
  \end{beamercolorbox}
}
% Prevent slide breaks in the middle of a paragraph
\widowpenalties 1 10000
\raggedbottom
\AtBeginPart{
  \frame{\partpage}
}
\AtBeginSection{
  \ifbibliography
  \else
    \frame{\sectionpage}
  \fi
}
\AtBeginSubsection{
  \frame{\subsectionpage}
}
\usepackage{iftex}
\ifPDFTeX
  \usepackage[T1]{fontenc}
  \usepackage[utf8]{inputenc}
  \usepackage{textcomp} % provide euro and other symbols
\else % if luatex or xetex
  \usepackage{unicode-math} % this also loads fontspec
  \defaultfontfeatures{Scale=MatchLowercase}
  \defaultfontfeatures[\rmfamily]{Ligatures=TeX,Scale=1}
\fi
\usepackage{lmodern}
\ifPDFTeX\else
  % xetex/luatex font selection
\fi
% Use upquote if available, for straight quotes in verbatim environments
\IfFileExists{upquote.sty}{\usepackage{upquote}}{}
\IfFileExists{microtype.sty}{% use microtype if available
  \usepackage[]{microtype}
  \UseMicrotypeSet[protrusion]{basicmath} % disable protrusion for tt fonts
}{}
\makeatletter
\@ifundefined{KOMAClassName}{% if non-KOMA class
  \IfFileExists{parskip.sty}{%
    \usepackage{parskip}
  }{% else
    \setlength{\parindent}{0pt}
    \setlength{\parskip}{6pt plus 2pt minus 1pt}}
}{% if KOMA class
  \KOMAoptions{parskip=half}}
\makeatother
\usepackage{longtable,booktabs,array}
\newcounter{none} % for unnumbered tables
\usepackage{calc} % for calculating minipage widths
\usepackage{caption}
% Make caption package work with longtable
\makeatletter
\def\fnum@table{\tablename~\thetable}
\makeatother
\setlength{\emergencystretch}{3em} % prevent overfull lines
\providecommand{\tightlist}{%
  \setlength{\itemsep}{0pt}\setlength{\parskip}{0pt}}
\usepackage{bookmark}
\IfFileExists{xurl.sty}{\usepackage{xurl}}{} % add URL line breaks if available
\urlstyle{same}
\hypersetup{
  hidelinks,
  pdfcreator={LaTeX via pandoc}}

\author{\texorpdfstring{}{}}
\date{}

\begin{document}

\section{Introduction}\label{introduction}

\begin{frame}{The Foundation Problem}
\protect\phantomsection\label{the-foundation-problem}
Mathematics rests upon foundations, and for over a century, Set Theory
has served as the dominant foundation for mathematical reasoning. The
Zermelo-Fraenkel axioms with Choice (ZFC) provide the standard framework
within which most mathematics is constructed. Yet this foundation
carries significant conceptual weight: nine or more axioms, including
the axiom of infinity, axiom of choice, and carefully crafted
restrictions to avoid paradoxes like Russell's.

In 1969, G. Spencer-Brown proposed a radical alternative in \emph{Laws
of Form}: a calculus requiring only two axioms, built on the primitive
notion of \textbf{distinction} rather than membership. This
calculus---variously called boundary logic, the calculus of indications,
or Containment Theory---offers a foundation of remarkable parsimony
while maintaining complete equivalence to Boolean algebra and
propositional logic.
\end{frame}

\begin{frame}{Historical Context}
\protect\phantomsection\label{historical-context}
\begin{block}{Spencer-Brown's Laws of Form (1969)}
\protect\phantomsection\label{spencer-browns-laws-of-form-1969}
George Spencer-Brown developed the calculus of indications from a simple
observation: the most fundamental cognitive act is \textbf{making a
distinction}---separating inside from outside, this from that. The
\emph{mark} or \emph{cross}, written \(\langle\ \rangle\), represents
this primary distinction: it creates a boundary that distinguishes the
space inside from the space outside.

From this single primitive, Spencer-Brown derived two axioms:

\begin{enumerate}
\tightlist
\item
  \textbf{The Law of Calling} (Involution):
  \(\langle\langle a \rangle\rangle = a\)

  \begin{itemize}
  \tightlist
  \item
    Crossing a boundary twice returns to the original state
  \item
    Equivalent to double negation elimination
  \end{itemize}
\item
  \textbf{The Law of Crossing} (Condensation):
  \(\langle\ \rangle\langle\ \rangle = \langle\ \rangle\)

  \begin{itemize}
  \tightlist
  \item
    Two marks condense to one mark
  \item
    The marked state is idempotent
  \end{itemize}
\end{enumerate}

These axioms generate the complete Boolean algebra, yet their
interpretation is fundamentally spatial rather than membership-based.
\end{block}

\begin{block}{Kauffman's Extensions}
\protect\phantomsection\label{kauffmans-extensions}
Louis H. Kauffman extended Spencer-Brown's work in several directions,
connecting it to knot theory, recursive forms, and category theory.
Kauffman demonstrated that the calculus of indications provides a
natural notation for Boolean algebra and showed how self-referential
forms---equations like \(f = \langle f \rangle\)---lead to ``imaginary''
Boolean values analogous to \(\sqrt{-1}\) in complex numbers.

These imaginary values oscillate between marked and unmarked states,
providing a formal treatment of self-reference that avoids the paradoxes
plaguing naive set theory. Where Russell's paradox forces set theory to
carefully restrict comprehension, boundary logic incorporates
self-reference naturally.
\end{block}

\begin{block}{Bricken's Computational Boundary Mathematics}
\protect\phantomsection\label{brickens-computational-boundary-mathematics}
William Bricken developed boundary logic into a practical computational
framework, demonstrating that forms translate directly to logic circuits
(NAND is universal and corresponds to \(\langle ab \rangle\)) and that
the calculus provides an efficient representation for Boolean reasoning.

Bricken's ``iconic arithmetic'' extends the notation to numerical
computation, suggesting that boundary representations may offer
advantages beyond Boolean logic.
\end{block}
\end{frame}

\begin{frame}{Motivation for This Work}
\protect\phantomsection\label{motivation-for-this-work}
Despite its theoretical elegance, Containment Theory remains
underexplored in mainstream mathematics and computer science. This work
aims to:

\begin{enumerate}
\tightlist
\item
  \textbf{Provide rigorous computational verification} of the
  theoretical claims in Laws of Form
\item
  \textbf{Establish precise correspondence} between boundary logic and
  Boolean algebra
\item
  \textbf{Analyze complexity properties} of the reduction algorithm
\item
  \textbf{Compare foundational properties} with Set Theory
  systematically
\item
  \textbf{Create accessible tools} for exploring and verifying boundary
  logic
\end{enumerate}
\end{frame}

\begin{frame}{Document Structure}
\protect\phantomsection\label{document-structure}
This manuscript presents:

\begin{itemize}
\tightlist
\item
  \textbf{Methodology} (Section 3): Formal definition of the calculus,
  axioms, reduction rules, and Boolean correspondence
\item
  \textbf{Results} (Section 4): Computational verification of theorems,
  complexity analysis, and proof demonstrations
\item
  \textbf{Discussion} (Section 5): Comparison with Set Theory,
  philosophical implications, and applications
\item
  \textbf{Conclusion} (Section 6): Summary of contributions and future
  directions
\end{itemize}

The computational framework accompanying this manuscript provides a
complete implementation of boundary logic with verified test coverage
exceeding 70\%, enabling readers to explore and verify all claims
independently.
\end{frame}

\begin{frame}{Notation}
\protect\phantomsection\label{notation}
Throughout this work, we use the following notation:

{\def\LTcaptype{none} % do not increment counter
\begin{longtable}[]{@{}
  >{\raggedright\arraybackslash}p{(\linewidth - 2\tabcolsep) * \real{0.4706}}
  >{\raggedright\arraybackslash}p{(\linewidth - 2\tabcolsep) * \real{0.5294}}@{}}
\toprule\noalign{}
\begin{minipage}[b]{\linewidth}\raggedright
Symbol
\end{minipage} & \begin{minipage}[b]{\linewidth}\raggedright
Meaning
\end{minipage} \\
\midrule\noalign{}
\endhead
\(\langle\ \rangle\) & The mark (cross), representing TRUE \\
\(\emptyset\) or void & Empty space, representing FALSE \\
\(\langle a \rangle\) & Enclosure of \(a\), representing NOT \(a\) \\
\(ab\) & Juxtaposition, representing \(a\) AND \(b\) \\
\(\langle\langle a \rangle\langle b \rangle\rangle\) & De Morgan form
for \(a\) OR \(b\) \\
\bottomrule\noalign{}
\end{longtable}
}

We write \(\langle\langle a \rangle\rangle\) for double enclosure and
use parentheses \((\ )\), square brackets \([\ ]\), or angle brackets
\(\langle\ \rangle\) interchangeably when clarity permits.
\end{frame}

\end{document}
