% Options for packages loaded elsewhere
\PassOptionsToPackage{unicode}{hyperref}
\PassOptionsToPackage{hyphens}{url}
\documentclass[
  ignorenonframetext,
]{beamer}
\newif\ifbibliography
\usepackage{pgfpages}
\setbeamertemplate{caption}[numbered]
\setbeamertemplate{caption label separator}{: }
\setbeamercolor{caption name}{fg=normal text.fg}
\beamertemplatenavigationsymbolsempty
% remove section numbering
\setbeamertemplate{part page}{
  \centering
  \begin{beamercolorbox}[sep=16pt,center]{part title}
    \usebeamerfont{part title}\insertpart\par
  \end{beamercolorbox}
}
\setbeamertemplate{section page}{
  \centering
  \begin{beamercolorbox}[sep=12pt,center]{section title}
    \usebeamerfont{section title}\insertsection\par
  \end{beamercolorbox}
}
\setbeamertemplate{subsection page}{
  \centering
  \begin{beamercolorbox}[sep=8pt,center]{subsection title}
    \usebeamerfont{subsection title}\insertsubsection\par
  \end{beamercolorbox}
}
% Prevent slide breaks in the middle of a paragraph
\widowpenalties 1 10000
\raggedbottom
\AtBeginPart{
  \frame{\partpage}
}
\AtBeginSection{
  \ifbibliography
  \else
    \frame{\sectionpage}
  \fi
}
\AtBeginSubsection{
  \frame{\subsectionpage}
}
\usepackage{iftex}
\ifPDFTeX
  \usepackage[T1]{fontenc}
  \usepackage[utf8]{inputenc}
  \usepackage{textcomp} % provide euro and other symbols
\else % if luatex or xetex
  \usepackage{unicode-math} % this also loads fontspec
  \defaultfontfeatures{Scale=MatchLowercase}
  \defaultfontfeatures[\rmfamily]{Ligatures=TeX,Scale=1}
\fi
\usepackage{lmodern}
\ifPDFTeX\else
  % xetex/luatex font selection
\fi
% Use upquote if available, for straight quotes in verbatim environments
\IfFileExists{upquote.sty}{\usepackage{upquote}}{}
\IfFileExists{microtype.sty}{% use microtype if available
  \usepackage[]{microtype}
  \UseMicrotypeSet[protrusion]{basicmath} % disable protrusion for tt fonts
}{}
\makeatletter
\@ifundefined{KOMAClassName}{% if non-KOMA class
  \IfFileExists{parskip.sty}{%
    \usepackage{parskip}
  }{% else
    \setlength{\parindent}{0pt}
    \setlength{\parskip}{6pt plus 2pt minus 1pt}}
}{% if KOMA class
  \KOMAoptions{parskip=half}}
\makeatother
\setlength{\emergencystretch}{3em} % prevent overfull lines
\providecommand{\tightlist}{%
  \setlength{\itemsep}{0pt}\setlength{\parskip}{0pt}}
\usepackage{bookmark}
\IfFileExists{xurl.sty}{\usepackage{xurl}}{} % add URL line breaks if available
\urlstyle{same}
\hypersetup{
  hidelinks,
  pdfcreator={LaTeX via pandoc}}

\author{\texorpdfstring{}{}}
\date{}

\begin{document}

\section{Introduction}\label{sec:introduction}

\begin{frame}{Overview}
\protect\phantomsection\label{overview}
This research documents and analyzes Andrius Kulikauskas's comprehensive
framework of ``Ways of Figuring Things Out,'' a systematic collection of
210 documented ways from the database, with connections to the broader
framework of 284 ways described in the source text. The work presents
both a philosophical framework for understanding different approaches to
knowledge and an empirical analysis of how these ways are structured,
categorized, and interrelated.

\begin{block}{Data Summary}
\protect\phantomsection\label{data-summary}
The analysis database contains comprehensive metadata for all documented
ways:

\begin{table}[h]
\centering
\begin{tabular}{|l|c|}
\hline
\textbf{Category} & \textbf{Count} \\
\hline
Total ways (database) & 210 \\
Total ways (documented) & 284 \\
Rooms in House of Knowledge & 24 \\
Distinct dialogue types & 38 \\
Unique dialogue partners & 196 \\
Network nodes & 210 \\
Network edges & 1,290 \\
\hline
\end{tabular}
\caption{Summary statistics for Ways of Figuring Things Out database}
\label{tab:data_summary}
\end{table}

The framework structure is visualized in Figure
\ref{fig:room_hierarchy}, showing the distribution of ways across the 24
rooms. The network of relationships between ways is presented in Figure
\ref{fig:ways_network}, revealing clusters and central connecting ways.
\end{block}
\end{frame}

\begin{frame}{The House of Knowledge Framework}
\protect\phantomsection\label{the-house-of-knowledge-framework}
The Ways framework is organized around a ``House of Knowledge''
containing 24 rooms, each representing a different aspect of how we come
to know and understand. These rooms are structured according to
fundamental philosophical principles:

\begin{itemize}
\tightlist
\item
  \textbf{Believing (1-2-3-4)}: Four levels of believing, from basic
  belief to fostering spirit among us
\item
  \textbf{Caring (1-2-3-4)}: Four levels of caring, from basic openness
  to acknowledging what transcends our limits\\
\item
  \textbf{Relative Learning}: The cycle of taking a stand, following
  through, and reflecting
\item
  \textbf{Dialogue Types}: Absolute, Relative, and Embrace God
  perspectives
\end{itemize}

Each way represents a specific method for figuring things out,
documented with examples, dialogue partners, and its relationship to the
broader framework.
\end{frame}

\begin{frame}{Research Objectives}
\protect\phantomsection\label{research-objectives}
This work aims to:

\begin{enumerate}
\tightlist
\item
  \textbf{Documentation}: Provide complete documentation of all 284 ways
  with their characteristics, examples, and relationships
\item
  \textbf{Categorization}: Systematically categorize ways according to
  dialogue types, rooms, and philosophical structures
\item
  \textbf{Analysis}: Conduct empirical analysis of way distributions,
  patterns, and interrelationships
\item
  \textbf{Visualization}: Create visual representations of the network
  of ways and their connections
\item
  \textbf{Application}: Develop tools and frameworks for applying these
  ways in educational and research contexts
\end{enumerate}
\end{frame}

\begin{frame}[fragile]{Data Sources}
\protect\phantomsection\label{data-sources}
The research draws on two primary data sources:

\begin{itemize}
\tightlist
\item
  \textbf{SQL Database}: A comprehensive SQLite database (converted from
  MySQL) containing 210 ways with complete metadata including dialogue
  types, examples, room assignments (mene), God relationships (Dievas),
  and conversant information
\item
  \textbf{Text Documentation}: Detailed markdown documentation
  (\texttt{ways.md}) providing philosophical context, examples, and
  descriptions for all 284 ways
\end{itemize}
\end{frame}

\begin{frame}{Methodology Overview}
\protect\phantomsection\label{methodology-overview}
Our approach combines:

\begin{itemize}
\tightlist
\item
  \textbf{Database Analysis}: SQLite conversion and querying of the ways
  database to extract patterns and relationships
\item
  \textbf{Network Analysis}: Graph-based analysis of how ways connect
  through dialogue partners and shared characteristics
\item
  \textbf{Statistical Analysis}: Quantitative analysis of distributions
  across categories, dialogue types, and rooms
\item
  \textbf{Text Analysis}: Analysis of way descriptions and examples to
  extract themes and patterns
\item
  \textbf{Visualization}: Creation of network graphs, hierarchical
  visualizations, and statistical plots
\end{itemize}
\end{frame}

\begin{frame}{Key Contributions}
\protect\phantomsection\label{key-contributions}
This research makes several key contributions:

\begin{enumerate}
\tightlist
\item
  \textbf{Complete Documentation}: First comprehensive systematic
  documentation of all 284 ways
\item
  \textbf{Empirical Analysis}: Quantitative analysis revealing patterns
  in way distributions and relationships
\item
  \textbf{Network Mapping}: Visualization of the network structure
  connecting different ways
\item
  \textbf{Categorization System}: Systematic organization within the
  24-room House of Knowledge framework
\item
  \textbf{Practical Tools}: Database and analysis tools for researchers
  and practitioners
\end{enumerate}
\end{frame}

\begin{frame}{Manuscript Organization}
\protect\phantomsection\label{manuscript-organization}
The manuscript is organized as follows:

\begin{enumerate}
\tightlist
\item
  \textbf{Abstract} (Section \ref{sec:abstract}): Overview of the
  research and key findings
\item
  \textbf{Introduction} (Section \ref{sec:introduction}): Framework
  overview and research objectives
\item
  \textbf{Methodology} (Section \ref{sec:methodology}): Database
  structure, analysis methods, and House of Knowledge framework
\item
  \textbf{Experimental Results} (Section
  \ref{sec:experimental_results}): Statistical analysis of ways,
  distributions, and patterns
\item
  \textbf{Discussion} (Section \ref{sec:discussion}): Interpretation of
  findings and philosophical implications
\item
  \textbf{Conclusion} (Section \ref{sec:conclusion}): Summary and future
  directions
\end{enumerate}

Supplemental sections provide extended methodological details,
additional results, and detailed analysis of specific aspects of the
framework.
\end{frame}

\begin{frame}{Philosophical Context}
\protect\phantomsection\label{philosophical-context}
The Ways framework emerges from a deep engagement with questions of
epistemology, learning, and knowledge. It addresses fundamental
questions:

\begin{itemize}
\tightlist
\item
  How do we come to know things?
\item
  What are the different valid approaches to understanding?
\item
  How do belief, care, and learning interact in knowledge acquisition?
\item
  What role does dialogue play in figuring things out?
\item
  How can we systematically organize different approaches to knowledge?
\end{itemize}

The framework provides a comprehensive answer to these questions through
its systematic organization of 284 distinct ways, each representing a
valid approach to knowledge and understanding.
\end{frame}

\begin{frame}{Applications}
\protect\phantomsection\label{applications}
This research has applications across multiple domains:

\begin{itemize}
\tightlist
\item
  \textbf{Education}: Understanding different learning styles and
  approaches
\item
  \textbf{Research Methodology}: Systematic approaches to knowledge
  acquisition
\item
  \textbf{Personal Development}: Tools for understanding one's own ways
  of figuring things out
\item
  \textbf{Philosophy}: Contributions to epistemology and knowledge
  systems theory
\item
  \textbf{Interdisciplinary Studies}: Framework for understanding
  knowledge across domains
\end{itemize}
\end{frame}

\begin{frame}{Structure of This Work}
\protect\phantomsection\label{structure-of-this-work}
The following sections provide detailed analysis of the Ways framework.
Section \ref{sec:methodology} describes the database structure and
analysis methods. Section \ref{sec:experimental_results} presents
statistical findings and patterns. Section \ref{sec:discussion}
interprets these findings within the broader philosophical context.
Supplemental sections provide extended details on methodology,
additional results, and detailed analysis of specific aspects of the
framework.
\end{frame}

\end{document}
