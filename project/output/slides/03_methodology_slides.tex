% Options for packages loaded elsewhere
\PassOptionsToPackage{unicode}{hyperref}
\PassOptionsToPackage{hyphens}{url}
\documentclass[
  ignorenonframetext,
]{beamer}
\newif\ifbibliography
\usepackage{pgfpages}
\setbeamertemplate{caption}[numbered]
\setbeamertemplate{caption label separator}{: }
\setbeamercolor{caption name}{fg=normal text.fg}
\beamertemplatenavigationsymbolsempty
% remove section numbering
\setbeamertemplate{part page}{
  \centering
  \begin{beamercolorbox}[sep=16pt,center]{part title}
    \usebeamerfont{part title}\insertpart\par
  \end{beamercolorbox}
}
\setbeamertemplate{section page}{
  \centering
  \begin{beamercolorbox}[sep=12pt,center]{section title}
    \usebeamerfont{section title}\insertsection\par
  \end{beamercolorbox}
}
\setbeamertemplate{subsection page}{
  \centering
  \begin{beamercolorbox}[sep=8pt,center]{subsection title}
    \usebeamerfont{subsection title}\insertsubsection\par
  \end{beamercolorbox}
}
% Prevent slide breaks in the middle of a paragraph
\widowpenalties 1 10000
\raggedbottom
\AtBeginPart{
  \frame{\partpage}
}
\AtBeginSection{
  \ifbibliography
  \else
    \frame{\sectionpage}
  \fi
}
\AtBeginSubsection{
  \frame{\subsectionpage}
}
\usepackage{iftex}
\ifPDFTeX
  \usepackage[T1]{fontenc}
  \usepackage[utf8]{inputenc}
  \usepackage{textcomp} % provide euro and other symbols
\else % if luatex or xetex
  \usepackage{unicode-math} % this also loads fontspec
  \defaultfontfeatures{Scale=MatchLowercase}
  \defaultfontfeatures[\rmfamily]{Ligatures=TeX,Scale=1}
\fi
\usepackage{lmodern}
\ifPDFTeX\else
  % xetex/luatex font selection
\fi
% Use upquote if available, for straight quotes in verbatim environments
\IfFileExists{upquote.sty}{\usepackage{upquote}}{}
\IfFileExists{microtype.sty}{% use microtype if available
  \usepackage[]{microtype}
  \UseMicrotypeSet[protrusion]{basicmath} % disable protrusion for tt fonts
}{}
\makeatletter
\@ifundefined{KOMAClassName}{% if non-KOMA class
  \IfFileExists{parskip.sty}{%
    \usepackage{parskip}
  }{% else
    \setlength{\parindent}{0pt}
    \setlength{\parskip}{6pt plus 2pt minus 1pt}}
}{% if KOMA class
  \KOMAoptions{parskip=half}}
\makeatother
\usepackage{longtable,booktabs,array}
\newcounter{none} % for unnumbered tables
\usepackage{calc} % for calculating minipage widths
\usepackage{caption}
% Make caption package work with longtable
\makeatletter
\def\fnum@table{\tablename~\thetable}
\makeatother
\setlength{\emergencystretch}{3em} % prevent overfull lines
\providecommand{\tightlist}{%
  \setlength{\itemsep}{0pt}\setlength{\parskip}{0pt}}
\usepackage{bookmark}
\IfFileExists{xurl.sty}{\usepackage{xurl}}{} % add URL line breaks if available
\urlstyle{same}
\hypersetup{
  hidelinks,
  pdfcreator={LaTeX via pandoc}}

\author{\texorpdfstring{}{}}
\date{}

\begin{document}

\section{Methodology}\label{methodology}

\begin{frame}{Formal Definition of the Calculus}
\protect\phantomsection\label{formal-definition-of-the-calculus}
\begin{block}{The Primitive: Distinction}
\protect\phantomsection\label{the-primitive-distinction}
The calculus of indications \cite{spencerbrown1969} begins with a single
primitive: the act of \textbf{distinction}. To distinguish is to create
a boundary that separates two regions---an inside and an outside. This
act is represented by the \textbf{mark} or \textbf{cross}:

\[\langle\ \rangle\] \{\#eq:mark\}

The mark creates a bounded region. Content placed inside the mark is
\textbf{contained} within the boundary; content outside is in the
\textbf{void}.
\end{block}

\begin{block}{Definition 1: Form}
\protect\phantomsection\label{definition-1-form}
A \textbf{form} is defined recursively:

\begin{enumerate}
\tightlist
\item
  The \textbf{void} (empty space) is a form
\item
  The \textbf{mark} \(\langle\ \rangle\) is a form
\item
  If \(a\) is a form, then \(\langle a \rangle\) (enclosure of \(a\)) is
  a form
\item
  If \(a\) and \(b\) are forms, then \(ab\) (juxtaposition of \(a\) and
  \(b\)) is a form
\end{enumerate}

Nothing else is a form.
\end{block}

\begin{block}{Definition 2: Depth and Size}
\protect\phantomsection\label{definition-2-depth-and-size}
For a form \(f\): - \textbf{Depth}: Maximum nesting level of boundaries
(void has depth 0, mark has depth 1) - \textbf{Size}: Total count of
marks (boundaries) in the form
\end{block}
\end{frame}

\begin{frame}{The Two Axioms}
\protect\phantomsection\label{the-two-axioms}
The entire calculus derives from two axioms:

\begin{block}{Axiom J1: Calling (Involution)}
\protect\phantomsection\label{axiom-j1-calling-involution}
\[\langle\langle a \rangle\rangle = a\] \{\#eq:calling\}

\textbf{Interpretation}: Crossing a boundary twice returns to the
original state. This is the spatial analog of double negation: NOT(NOT
\(a\)) = \(a\).

\textbf{Proof sketch}: Consider being inside a region bounded by
\(\langle a \rangle\). The inner boundary places you ``outside of
\(a\)'' relative to \(a\). The outer boundary then places you ``inside''
relative to being ``outside of \(a\)''---returning you to \(a\).
\end{block}

\begin{block}{Axiom J2: Crossing (Condensation)}
\protect\phantomsection\label{axiom-j2-crossing-condensation}
\[\langle\ \rangle\langle\ \rangle = \langle\ \rangle\]
\{\#eq:crossing\}

\textbf{Interpretation}: Multiple marks in juxtaposition condense to a
single mark. The marked state is idempotent.

\textbf{Proof sketch}: Two boundaries side by side both indicate ``the
marked state.'' Indicating the same thing twice does not change what is
indicated.
\end{block}
\end{frame}

\begin{frame}[fragile]{Reduction Algorithm}
\protect\phantomsection\label{reduction-algorithm}
\begin{block}{Definition 3: Canonical Form}
\protect\phantomsection\label{definition-3-canonical-form}
A form is in \textbf{canonical form} if no reduction rule can be
applied. The only canonical forms are: - The void \(\emptyset\) - The
mark \(\langle\ \rangle\)
\end{block}

\begin{block}{Reduction Rules}
\protect\phantomsection\label{reduction-rules}
The reduction engine applies rules in the following priority:

\begin{enumerate}
\item
  \textbf{Calling Reduction}: If a form matches
  \(\langle\langle a \rangle\rangle\) where \(a\) has exactly one
  enclosed child, reduce to \(a\)
\item
  \textbf{Crossing Reduction}: If a form contains multiple simple marks
  \(\langle\ \rangle\) in juxtaposition, condense to single mark
\item
  \textbf{Void Elimination}: Remove void elements from juxtaposition
  (void is the identity for AND)
\item
  \textbf{Recursive Application}: Apply rules to nested subforms
\end{enumerate}
\end{block}

\begin{block}{Algorithm: Reduce to Canonical Form}
\protect\phantomsection\label{algorithm-reduce-to-canonical-form}
\begin{verbatim}
function REDUCE(form):
    while REDUCIBLE(form):
        if CALLING_PATTERN(form):
            form ← APPLY_CALLING(form)
        else if CROSSING_PATTERN(form):
            form ← APPLY_CROSSING(form)
        else if VOID_PATTERN(form):
            form ← REMOVE_VOID(form)
        else:
            form ← REDUCE_SUBFORMS(form)
    return form
\end{verbatim}
\end{block}

\begin{block}{Theorem 1: Termination}
\protect\phantomsection\label{theorem-1-termination}
\textbf{Claim}: The reduction algorithm terminates for all well-formed
inputs.

\textbf{Proof}: Each rule application strictly decreases either: - The
depth of the form (calling), or - The size of the form (crossing, void
elimination)

Since both metrics are non-negative integers, the algorithm must
terminate.
\end{block}

\begin{block}{Theorem 2: Confluence}
\protect\phantomsection\label{theorem-2-confluence}
\textbf{Claim}: All reduction sequences from a given form lead to the
same canonical form.

\textbf{Proof sketch}: The rules are non-overlapping (each pattern is
distinct) and local (applying one rule does not invalidate others). The
Church-Rosser property follows.
\end{block}
\end{frame}

\begin{frame}{Boolean Algebra Correspondence}
\protect\phantomsection\label{boolean-algebra-correspondence}
\begin{block}{The Isomorphism}
\protect\phantomsection\label{the-isomorphism}
Boundary logic is isomorphic to Boolean algebra
\cite{huntington1904,stone1936}:

{\def\LTcaptype{none} % do not increment counter
\begin{longtable}[]{@{}
  >{\raggedright\arraybackslash}p{(\linewidth - 4\tabcolsep) * \real{0.2963}}
  >{\raggedright\arraybackslash}p{(\linewidth - 4\tabcolsep) * \real{0.3148}}
  >{\raggedright\arraybackslash}p{(\linewidth - 4\tabcolsep) * \real{0.3889}}@{}}
\toprule\noalign{}
\begin{minipage}[b]{\linewidth}\raggedright
Boundary Logic
\end{minipage} & \begin{minipage}[b]{\linewidth}\raggedright
Boolean Algebra
\end{minipage} & \begin{minipage}[b]{\linewidth}\raggedright
Propositional Logic
\end{minipage} \\
\midrule\noalign{}
\endhead
\(\langle\ \rangle\) (mark) & TRUE (1) & T \\
void (empty) & FALSE (0) & F \\
\(\langle a \rangle\) & NOT \(a\) & \(\neg a\) \\
\(ab\) & \(a\) AND \(b\) & \(a \land b\) \\
\(\langle\langle a \rangle\langle b \rangle\rangle\) & \(a\) OR \(b\) &
\(a \lor b\) \\
\(\langle a \langle b \rangle\rangle\) & \(a \to b\) &
\(a \rightarrow b\) \\
\bottomrule\noalign{}
\end{longtable}
}
\end{block}

\begin{block}{Derivation of OR}
\protect\phantomsection\label{derivation-of-or}
The De Morgan form for disjunction:
\[a \lor b = \neg(\neg a \land \neg b) = \langle\langle a \rangle\langle b \rangle\rangle\]
\{\#eq:or\}
\end{block}

\begin{block}{Derivation of NAND}
\protect\phantomsection\label{derivation-of-nand}
The NAND gate, functionally complete:
\[a \text{ NAND } b = \neg(a \land b) = \langle ab \rangle\]
\{\#eq:nand\}
\end{block}
\end{frame}

\begin{frame}{Derived Theorems (Consequences)}
\protect\phantomsection\label{derived-theorems-consequences}
Spencer-Brown derives nine consequences (C1-C9) from the axioms. We
verify each computationally:

\begin{block}{C1: Position}
\protect\phantomsection\label{c1-position}
\[\langle\langle a \rangle b \rangle a = a\]
\end{block}

\begin{block}{C2: Transposition}
\protect\phantomsection\label{c2-transposition}
\[\langle\langle a \rangle\langle b \rangle\rangle c = \langle ac \rangle\langle bc \rangle\]
\end{block}

\begin{block}{C3: Generation (Excluded Middle)}
\protect\phantomsection\label{c3-generation-excluded-middle}
\[\langle\langle a \rangle a \rangle = \langle\ \rangle\]

This corresponds to \(a \lor \neg a = \text{TRUE}\).
\end{block}

\begin{block}{C4: Integration}
\protect\phantomsection\label{c4-integration}
\[a \lor \text{TRUE} = \text{TRUE}\]

In boundary notation:
\(\langle\langle a \rangle\langle\ \rangle\rangle = \langle\ \rangle\)
(disjunction with TRUE yields TRUE).
\end{block}

\begin{block}{C5: Occultation}
\protect\phantomsection\label{c5-occultation}
\[\langle\langle a \rangle\rangle a = a\]
\end{block}

\begin{block}{C6: Iteration (Idempotence)}
\protect\phantomsection\label{c6-iteration-idempotence}
\[aa = a\]
\end{block}

\begin{block}{C7: Extension}
\protect\phantomsection\label{c7-extension}
\[\langle\langle a \rangle\langle b \rangle\rangle\langle\langle a \rangle b \rangle = a\]
\end{block}

\begin{block}{C8: Echelon}
\protect\phantomsection\label{c8-echelon}
\[\langle\langle ab \rangle c \rangle = \langle ac \rangle\langle bc \rangle\]
\end{block}

\begin{block}{C9: Cross-Transposition}
\protect\phantomsection\label{c9-cross-transposition}
\[\langle\langle ac \rangle\langle bc \rangle\rangle = \langle\langle a \rangle\langle b \rangle\rangle c\]
\end{block}
\end{frame}

\begin{frame}{Evaluation Semantics}
\protect\phantomsection\label{evaluation-semantics}
\begin{block}{Definition 4: Truth Value}
\protect\phantomsection\label{definition-4-truth-value}
The truth value \(\llbracket f \rrbracket\) of a form \(f\):

\[\llbracket \text{void} \rrbracket = \text{FALSE}\]
\[\llbracket \langle\ \rangle \rrbracket = \text{TRUE}\]
\[\llbracket \langle a \rangle \rrbracket = \neg\llbracket a \rrbracket\]
\[\llbracket ab \rrbracket = \llbracket a \rrbracket \land \llbracket b \rrbracket\]
\{\#eq:semantics\}
\end{block}

\begin{block}{Theorem 3: Soundness}
\protect\phantomsection\label{theorem-3-soundness}
\textbf{Claim}: Equivalent forms evaluate to the same truth value.

\textbf{Proof}: The axioms preserve truth value: - J1:
\(\llbracket\langle\langle a \rangle\rangle\rrbracket = \neg\neg\llbracket a \rrbracket = \llbracket a \rrbracket\)
✓ - J2:
\(\llbracket\langle\ \rangle\langle\ \rangle\rrbracket = \text{TRUE} \land \text{TRUE} = \text{TRUE} = \llbracket\langle\ \rangle\rrbracket\)
✓
\end{block}
\end{frame}

\begin{frame}[fragile]{Implementation}
\protect\phantomsection\label{implementation}
The computational framework implements principles from formal
verification \cite{bertot2004,nipkow2002}:

\begin{enumerate}
\tightlist
\item
  \textbf{Form Construction}: \texttt{Form} class with void, mark,
  enclosure, juxtaposition
\item
  \textbf{Reduction Engine}: \texttt{ReductionEngine} with step-by-step
  traces
\item
  \textbf{Evaluation}: \texttt{FormEvaluator} for truth value extraction
\item
  \textbf{Theorem Verification}: \texttt{Theorem} class with automatic
  proof checking
\item
  \textbf{Visualization}: Nested boundary diagrams for forms
\end{enumerate}

All implementations achieve test coverage exceeding 70\% with real data
verification (no mock testing).
\end{frame}

\end{document}
