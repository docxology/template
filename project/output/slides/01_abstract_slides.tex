% Options for packages loaded elsewhere
\PassOptionsToPackage{unicode}{hyperref}
\PassOptionsToPackage{hyphens}{url}
\documentclass[
  ignorenonframetext,
]{beamer}
\newif\ifbibliography
\usepackage{pgfpages}
\setbeamertemplate{caption}[numbered]
\setbeamertemplate{caption label separator}{: }
\setbeamercolor{caption name}{fg=normal text.fg}
\beamertemplatenavigationsymbolsempty
% remove section numbering
\setbeamertemplate{part page}{
  \centering
  \begin{beamercolorbox}[sep=16pt,center]{part title}
    \usebeamerfont{part title}\insertpart\par
  \end{beamercolorbox}
}
\setbeamertemplate{section page}{
  \centering
  \begin{beamercolorbox}[sep=12pt,center]{section title}
    \usebeamerfont{section title}\insertsection\par
  \end{beamercolorbox}
}
\setbeamertemplate{subsection page}{
  \centering
  \begin{beamercolorbox}[sep=8pt,center]{subsection title}
    \usebeamerfont{subsection title}\insertsubsection\par
  \end{beamercolorbox}
}
% Prevent slide breaks in the middle of a paragraph
\widowpenalties 1 10000
\raggedbottom
\AtBeginPart{
  \frame{\partpage}
}
\AtBeginSection{
  \ifbibliography
  \else
    \frame{\sectionpage}
  \fi
}
\AtBeginSubsection{
  \frame{\subsectionpage}
}
\usepackage{iftex}
\ifPDFTeX
  \usepackage[T1]{fontenc}
  \usepackage[utf8]{inputenc}
  \usepackage{textcomp} % provide euro and other symbols
\else % if luatex or xetex
  \usepackage{unicode-math} % this also loads fontspec
  \defaultfontfeatures{Scale=MatchLowercase}
  \defaultfontfeatures[\rmfamily]{Ligatures=TeX,Scale=1}
\fi
\usepackage{lmodern}
\ifPDFTeX\else
  % xetex/luatex font selection
\fi
% Use upquote if available, for straight quotes in verbatim environments
\IfFileExists{upquote.sty}{\usepackage{upquote}}{}
\IfFileExists{microtype.sty}{% use microtype if available
  \usepackage[]{microtype}
  \UseMicrotypeSet[protrusion]{basicmath} % disable protrusion for tt fonts
}{}
\makeatletter
\@ifundefined{KOMAClassName}{% if non-KOMA class
  \IfFileExists{parskip.sty}{%
    \usepackage{parskip}
  }{% else
    \setlength{\parindent}{0pt}
    \setlength{\parskip}{6pt plus 2pt minus 1pt}}
}{% if KOMA class
  \KOMAoptions{parskip=half}}
\makeatother
\setlength{\emergencystretch}{3em} % prevent overfull lines
\providecommand{\tightlist}{%
  \setlength{\itemsep}{0pt}\setlength{\parskip}{0pt}}
\usepackage{bookmark}
\IfFileExists{xurl.sty}{\usepackage{xurl}}{} % add URL line breaks if available
\urlstyle{same}
\hypersetup{
  hidelinks,
  pdfcreator={LaTeX via pandoc}}

\author{\texorpdfstring{}{}}
\date{}

\begin{document}

\begin{frame}{Abstract}
\protect\phantomsection\label{abstract}
Containment Theory presents an alternative foundation to classical Set
Theory, replacing the primitive notion of membership (\(\in\)) with
spatial containment through boundary distinctions. Building on G.
Spencer-Brown's \emph{Laws of Form} (1969), we develop a complete
computational framework for \textbf{boundary logic} (also called the
\textbf{calculus of indications}) that demonstrates its equivalence to
Boolean algebra while offering distinct advantages in parsimony,
geometric intuition, and handling of self-reference. \textbf{Boundary
logic} is a logical system built from the primitive act of drawing
distinctions (boundaries), while the \textbf{calculus of indications} is
Spencer-Brown's original name for this formal system.

The calculus operates from just two axioms: \textbf{Calling}
(\(\langle\langle a \rangle\rangle = a\), where double enclosure returns
to the original form) and \textbf{Crossing}
(\(\langle\ \rangle\langle\ \rangle = \langle\ \rangle\), where multiple
marks condense to a single mark). From these primitives, we derive the
complete Boolean algebra, establishing that the marked state
\(\langle\ \rangle\) corresponds to TRUE and the unmarked void (empty
space) corresponds to FALSE, with enclosure \(\langle a \rangle\)
representing negation and juxtaposition \(ab\) representing conjunction.

We present a \textbf{reduction engine} that transforms arbitrary
boundary \textbf{forms} (expressions built from marks, enclosures, and
juxtapositions) to \textbf{canonical representations} (either void or
mark---the simplest irreducible forms), prove termination in polynomial
time for \textbf{ground forms} (forms without variables, where all
values are concrete), and verify all derived theorems computationally.
Our implementation achieves formal verification of Spencer-Brown's nine
consequences (C1-C9), De Morgan's laws, and the fundamental Boolean
axioms through systematic reduction to canonical forms.

The comparison with Set Theory reveals that boundary logic achieves
logical completeness with minimal axiomatic commitment (2 vs 9+ axioms
in ZFC), provides native geometric interpretation through nested
boundaries, and naturally handles self-referential structures through
Spencer-Brown's ``imaginary'' Boolean values---constructs that create
paradoxes in classical set theory. These properties suggest applications
in circuit design, cognitive modeling, and foundations of computation.

This work establishes Containment Theory as a viable alternative
foundation for discrete mathematics, with complete computational
verification of its theoretical claims and open-source implementation
for further investigation.

\textbf{Keywords:} containment theory, boundary logic, Laws of Form,
iconic mathematics, Boolean algebra, foundational mathematics, calculus
of indications
\end{frame}

\end{document}
