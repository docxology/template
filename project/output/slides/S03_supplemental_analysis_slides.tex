% Options for packages loaded elsewhere
\PassOptionsToPackage{unicode}{hyperref}
\PassOptionsToPackage{hyphens}{url}
\documentclass[
  ignorenonframetext,
]{beamer}
\newif\ifbibliography
\usepackage{pgfpages}
\setbeamertemplate{caption}[numbered]
\setbeamertemplate{caption label separator}{: }
\setbeamercolor{caption name}{fg=normal text.fg}
\beamertemplatenavigationsymbolsempty
% remove section numbering
\setbeamertemplate{part page}{
  \centering
  \begin{beamercolorbox}[sep=16pt,center]{part title}
    \usebeamerfont{part title}\insertpart\par
  \end{beamercolorbox}
}
\setbeamertemplate{section page}{
  \centering
  \begin{beamercolorbox}[sep=12pt,center]{section title}
    \usebeamerfont{section title}\insertsection\par
  \end{beamercolorbox}
}
\setbeamertemplate{subsection page}{
  \centering
  \begin{beamercolorbox}[sep=8pt,center]{subsection title}
    \usebeamerfont{subsection title}\insertsubsection\par
  \end{beamercolorbox}
}
% Prevent slide breaks in the middle of a paragraph
\widowpenalties 1 10000
\raggedbottom
\AtBeginPart{
  \frame{\partpage}
}
\AtBeginSection{
  \ifbibliography
  \else
    \frame{\sectionpage}
  \fi
}
\AtBeginSubsection{
  \frame{\subsectionpage}
}
\usepackage{iftex}
\ifPDFTeX
  \usepackage[T1]{fontenc}
  \usepackage[utf8]{inputenc}
  \usepackage{textcomp} % provide euro and other symbols
\else % if luatex or xetex
  \usepackage{unicode-math} % this also loads fontspec
  \defaultfontfeatures{Scale=MatchLowercase}
  \defaultfontfeatures[\rmfamily]{Ligatures=TeX,Scale=1}
\fi
\usepackage{lmodern}
\ifPDFTeX\else
  % xetex/luatex font selection
\fi
% Use upquote if available, for straight quotes in verbatim environments
\IfFileExists{upquote.sty}{\usepackage{upquote}}{}
\IfFileExists{microtype.sty}{% use microtype if available
  \usepackage[]{microtype}
  \UseMicrotypeSet[protrusion]{basicmath} % disable protrusion for tt fonts
}{}
\makeatletter
\@ifundefined{KOMAClassName}{% if non-KOMA class
  \IfFileExists{parskip.sty}{%
    \usepackage{parskip}
  }{% else
    \setlength{\parindent}{0pt}
    \setlength{\parskip}{6pt plus 2pt minus 1pt}}
}{% if KOMA class
  \KOMAoptions{parskip=half}}
\makeatother
\usepackage{longtable,booktabs,array}
\newcounter{none} % for unnumbered tables
\usepackage{calc} % for calculating minipage widths
\usepackage{caption}
% Make caption package work with longtable
\makeatletter
\def\fnum@table{\tablename~\thetable}
\makeatother
\setlength{\emergencystretch}{3em} % prevent overfull lines
\providecommand{\tightlist}{%
  \setlength{\itemsep}{0pt}\setlength{\parskip}{0pt}}
\usepackage{bookmark}
\IfFileExists{xurl.sty}{\usepackage{xurl}}{} % add URL line breaks if available
\urlstyle{same}
\hypersetup{
  hidelinks,
  pdfcreator={LaTeX via pandoc}}

\author{\texorpdfstring{}{}}
\date{}

\begin{document}

\section{Supplemental Analysis: Pragmatist and Neo-Materialist
Foundations}\label{supplemental-analysis-pragmatist-and-neo-materialist-foundations}

\subsection{S3.1 North American Pragmatism and the Calculus of
Indications}\label{s3.1-north-american-pragmatism-and-the-calculus-of-indications}

\begin{frame}{The Peircean Heritage}
\protect\phantomsection\label{the-peircean-heritage}
Charles Sanders Peirce (1839-1914) developed \textbf{Existential
Graphs}---a diagrammatic logic that anticipates Spencer-Brown's calculus
in fundamental ways. The connection is not merely superficial but
structural.

\begin{block}{Peirce's Existential Graphs}
\protect\phantomsection\label{peirces-existential-graphs}
Peirce's system employs: - \textbf{Sheet of Assertion}: The blank page
represents truth (cf.~Spencer-Brown's unmarked space) - \textbf{Cuts}:
Closed curves that negate their contents (cf.~enclosure) -
\textbf{Juxtaposition}: Co-presence on the sheet represents conjunction

{\def\LTcaptype{none} % do not increment counter
\begin{longtable}[]{@{}
  >{\raggedright\arraybackslash}p{(\linewidth - 4\tabcolsep) * \real{0.3542}}
  >{\raggedright\arraybackslash}p{(\linewidth - 4\tabcolsep) * \real{0.3125}}
  >{\raggedright\arraybackslash}p{(\linewidth - 4\tabcolsep) * \real{0.3333}}@{}}
\toprule\noalign{}
\begin{minipage}[b]{\linewidth}\raggedright
Peirce's Graphs
\end{minipage} & \begin{minipage}[b]{\linewidth}\raggedright
Spencer-Brown
\end{minipage} & \begin{minipage}[b]{\linewidth}\raggedright
Interpretation
\end{minipage} \\
\midrule\noalign{}
\endhead
Blank sheet & Void & Base state \\
Cut (○) & Mark \(\langle\ \rangle\) & Negation/distinction \\
Double cut & \(\langle\langle\ \rangle\rangle\) & Double negation =
identity \\
Adjacent graphs & Juxtaposition & Conjunction \\
\bottomrule\noalign{}
\end{longtable}
}

Peirce's \textbf{Alpha graphs} (propositional logic) are essentially
isomorphic to the calculus of indications.
\end{block}

\begin{block}{Phaneroscopy and Firstness}
\protect\phantomsection\label{phaneroscopy-and-firstness}
Peirce's categories illuminate the boundary:

\begin{enumerate}
\tightlist
\item
  \textbf{Firstness}: Quality of feeling, pure possibility---\emph{the
  void before distinction}
\item
  \textbf{Secondness}: Reaction, resistance, brute fact---\emph{the act
  of distinction}
\item
  \textbf{Thirdness}: Mediation, law, representation---\emph{the form
  after distinction}
\end{enumerate}

The mark \(\langle\ \rangle\) instantiates the passage from Firstness
(void) through Secondness (drawing) to Thirdness (form).
\end{block}

\begin{block}{Semiotics and the Icon}
\protect\phantomsection\label{semiotics-and-the-icon}
Spencer-Brown's notation is fundamentally \textbf{iconic} in Peirce's
sense: - The mark \emph{looks like} what it represents (a boundary) -
The notation exhibits its meaning rather than merely denoting it -
Reasoning proceeds by manipulation of the icon itself

\begin{quote}
``The icon does not stand for its object by resembling it\ldots{} it is
itself a fragment of that object.'' --- Peirce
\end{quote}
\end{block}
\end{frame}

\begin{frame}{William James: Radical Empiricism}
\protect\phantomsection\label{william-james-radical-empiricism}
James's \textbf{radical empiricism} (1904-1912) insisted that relations
are as real as the things related. This aligns with boundary logic:

{\def\LTcaptype{none} % do not increment counter
\begin{longtable}[]{@{}ll@{}}
\toprule\noalign{}
James & Containment Theory \\
\midrule\noalign{}
\endhead
Relations are real & Boundaries are primitive \\
Conjunctive relations & Juxtaposition \\
Disjunctive relations & Separation by mark \\
Pure experience & Void before distinction \\
\bottomrule\noalign{}
\end{longtable}
}

James's ``stream of consciousness'' fragments through distinction; the
calculus formalizes this fragmentation.

\begin{block}{The Pragmatic Maxim}
\protect\phantomsection\label{the-pragmatic-maxim}
Peirce's pragmatic maxim: ``Consider what effects\ldots{} the object of
our conception has. Then, our conception of these effects is the whole
of our conception of the object.''

For the mark \(\langle\ \rangle\): - \textbf{Effect}: Creates
inside/outside - \textbf{Conception}: The mark \emph{is} distinction
itself - \textbf{Meaning}: Fully contained in operational consequences
\end{block}
\end{frame}

\begin{frame}{John Dewey: Inquiry as Distinction}
\protect\phantomsection\label{john-dewey-inquiry-as-distinction}
Dewey's \textbf{instrumentalism} treats inquiry as the transformation of
indeterminate situations into determinate ones---precisely the function
of distinction.

{\def\LTcaptype{none} % do not increment counter
\begin{longtable}[]{@{}ll@{}}
\toprule\noalign{}
Dewey's Inquiry & Boundary Operation \\
\midrule\noalign{}
\endhead
Indeterminate situation & Void \\
Problematic situation & Recognition of need for distinction \\
Institution of a problem & Drawing the mark \\
Determination & Canonical form \\
\bottomrule\noalign{}
\end{longtable}
}

Dewey's emphasis on \textbf{continuity} (situations flowing into one
another) parallels the recursive structure of nested boundaries.

\begin{block}{Experience and Nature}
\protect\phantomsection\label{experience-and-nature}
\begin{quote}
``To exist is to be in a situation\ldots{}'' --- Dewey
\end{quote}

To be distinguished \emph{is} to exist. The mark creates existence from
the void. Dewey's naturalism grounds this in biological and cultural
practice: organisms survive by making effective distinctions.
\end{block}
\end{frame}

\subsection{S3.2 Process Philosophy and the
Mark}\label{s3.2-process-philosophy-and-the-mark}

\begin{frame}{Alfred North Whitehead}
\protect\phantomsection\label{alfred-north-whitehead}
Whitehead's \textbf{process philosophy} provides metaphysical grounding:

\begin{block}{Actual Entities}
\protect\phantomsection\label{actual-entities}
Whitehead's \textbf{actual entities} are the final real things: - Each
actual entity \emph{becomes} through \textbf{prehension} (grasping
others) - The void corresponds to \textbf{eternal objects} (pure
potentiality) - The mark corresponds to \textbf{actualization} (becoming
definite)

{\def\LTcaptype{none} % do not increment counter
\begin{longtable}[]{@{}ll@{}}
\toprule\noalign{}
Whitehead & Containment Theory \\
\midrule\noalign{}
\endhead
Creativity & The capacity for distinction \\
Eternal objects & Void (potentiality) \\
Actual entities & Marked forms \\
Prehension & Enclosure (taking in) \\
Concrescence & Reduction to canonical form \\
\bottomrule\noalign{}
\end{longtable}
}
\end{block}

\begin{block}{The Category of the Ultimate}
\protect\phantomsection\label{the-category-of-the-ultimate}
Whitehead's three notions: 1. \textbf{Creativity}: The ultimate
principle of novelty 2. \textbf{Many}: The disjunctive diversity of the
universe 3. \textbf{One}: The novel entity synthesizing the many

Distinction (mark-making) \emph{is} creativity instantiated: from the
many (void, undifferentiated), the one (canonical form) emerges.
\end{block}
\end{frame}

\subsection{S3.3 Neo-Materialism and Agential
Realism}\label{s3.3-neo-materialism-and-agential-realism}

\begin{frame}{Karen Barad: Intra-action}
\protect\phantomsection\label{karen-barad-intra-action}
Karen Barad's \textbf{agential realism} reconceives the relationship
between observer, observed, and observation. The boundary is not between
pre-existing entities but constitutive of entities.

\begin{block}{Intra-action vs.~Interaction}
\protect\phantomsection\label{intra-action-vs.-interaction}
{\def\LTcaptype{none} % do not increment counter
\begin{longtable}[]{@{}
  >{\raggedright\arraybackslash}p{(\linewidth - 4\tabcolsep) * \real{0.2857}}
  >{\raggedright\arraybackslash}p{(\linewidth - 4\tabcolsep) * \real{0.4127}}
  >{\raggedright\arraybackslash}p{(\linewidth - 4\tabcolsep) * \real{0.3016}}@{}}
\toprule\noalign{}
\begin{minipage}[b]{\linewidth}\raggedright
Traditional View
\end{minipage} & \begin{minipage}[b]{\linewidth}\raggedright
Barad's Agential Realism
\end{minipage} & \begin{minipage}[b]{\linewidth}\raggedright
Containment Theory
\end{minipage} \\
\midrule\noalign{}
\endhead
Entities interact & Entities intra-act & Forms compose \\
Boundaries pre-exist & Boundaries enacted & Mark creates boundary \\
Observer separate & Observer entangled & Self-reference (imaginary
values) \\
\bottomrule\noalign{}
\end{longtable}
}
\end{block}

\begin{block}{Agential Cuts}
\protect\phantomsection\label{agential-cuts}
Barad's \textbf{agential cuts} determine what becomes determinate:

\begin{quote}
``It is through specific agential intra-actions that the boundaries and
properties of the `components' of phenomena become determinate.'' ---
Barad, \emph{Meeting the Universe Halfway}
\end{quote}

The Spencer-Brown mark \emph{is} an agential cut: it doesn't represent a
pre-existing distinction but enacts one.
\end{block}

\begin{block}{Diffraction}
\protect\phantomsection\label{diffraction}
Barad's \textbf{diffraction} (vs.~reflection) as methodological
approach: - Reflection presupposes fixed identities mirrored -
Diffraction attends to patterns of difference

Reduction in boundary logic is diffractive: it doesn't preserve original
form but produces interference patterns (canonical forms) from
distinctions.
\end{block}
\end{frame}

\begin{frame}{Donna Haraway: Situated Knowledges}
\protect\phantomsection\label{donna-haraway-situated-knowledges}
Haraway's \textbf{situated knowledges} reject the ``god trick'' of
seeing everything from nowhere:

{\def\LTcaptype{none} % do not increment counter
\begin{longtable}[]{@{}lll@{}}
\toprule\noalign{}
God Trick & Situated Knowledge & Boundary Logic \\
\midrule\noalign{}
\endhead
View from nowhere & View from somewhere & View from inside/outside \\
Unmarked observer & Marked observer & Observer as form \\
Neutral & Positioned & Self-referential \\
\bottomrule\noalign{}
\end{longtable}
}

The imaginary value \(j = \langle j \rangle\) formalizes the observer
observing itself---a situated, recursive position.
\end{frame}

\subsection{S3.4 Deleuze and
Immanence}\label{s3.4-deleuze-and-immanence}

\begin{frame}{Difference in Itself}
\protect\phantomsection\label{difference-in-itself}
Gilles Deleuze's \textbf{philosophy of difference} resonates with
distinction-as-primitive:

{\def\LTcaptype{none} % do not increment counter
\begin{longtable}[]{@{}
  >{\raggedright\arraybackslash}p{(\linewidth - 4\tabcolsep) * \real{0.4815}}
  >{\raggedright\arraybackslash}p{(\linewidth - 4\tabcolsep) * \real{0.1667}}
  >{\raggedright\arraybackslash}p{(\linewidth - 4\tabcolsep) * \real{0.3519}}@{}}
\toprule\noalign{}
\begin{minipage}[b]{\linewidth}\raggedright
Representational Thought
\end{minipage} & \begin{minipage}[b]{\linewidth}\raggedright
Deleuze
\end{minipage} & \begin{minipage}[b]{\linewidth}\raggedright
Containment Theory
\end{minipage} \\
\midrule\noalign{}
\endhead
Identity primary & Difference primary & Distinction primary \\
Difference = not-same & Difference in itself & Mark creates
difference \\
Categories fixed & Categories produced & Forms reducible \\
\bottomrule\noalign{}
\end{longtable}
}

\begin{block}{The Virtual and the Actual}
\protect\phantomsection\label{the-virtual-and-the-actual}
Deleuze's \textbf{virtual/actual} distinction maps onto void/mark:

{\def\LTcaptype{none} % do not increment counter
\begin{longtable}[]{@{}lll@{}}
\toprule\noalign{}
Deleuze & Spencer-Brown & Character \\
\midrule\noalign{}
\endhead
Virtual & Void & Real but not actual \\
Actualization & Mark-making & Determination \\
Actual & Canonical form & Fully determined \\
\bottomrule\noalign{}
\end{longtable}
}

The void is \emph{virtual}---it has real effects (as identity for
conjunction) without being actual (marked).
\end{block}
\end{frame}

\begin{frame}{Intensive Differences}
\protect\phantomsection\label{intensive-differences}
Deleuze's \textbf{intensive quantities} (differences that don't divide
without changing nature) relate to depth in boundary logic:

\begin{itemize}
\tightlist
\item
  Depth = intensive magnitude
\item
  Flattening (reduction) changes nature
\item
  \(\langle\langle a \rangle\rangle \neq \langle a \rangle \neq a\)
  intensively
\end{itemize}
\end{frame}

\subsection{S3.5 Brian Massumi and
Affect}\label{s3.5-brian-massumi-and-affect}

\begin{frame}{Affect and the Virtual}
\protect\phantomsection\label{affect-and-the-virtual}
Massumi's \textbf{affect theory} treats intensity as prior to formed
content:

{\def\LTcaptype{none} % do not increment counter
\begin{longtable}[]{@{}ll@{}}
\toprule\noalign{}
Massumi & Containment Theory \\
\midrule\noalign{}
\endhead
Affect (intensity) & Void (potential) \\
Emotion (qualified) & Form (structured) \\
Passage & Reduction \\
Autonomy of affect & Resistance to reduction \\
\bottomrule\noalign{}
\end{longtable}
}

Irreducible forms (already canonical) resist further passage---they are
``stuck'' affects.
\end{frame}

\begin{frame}{Ontopower}
\protect\phantomsection\label{ontopower}
Massumi's \textbf{ontopower}: power operating at the level of emergence.

The capacity to make distinctions \emph{is} ontopower---the capacity to
create realities by differentiating the undifferentiated.
\end{frame}

\subsection{S3.6 New Materialism and Matter's
Agency}\label{s3.6-new-materialism-and-matters-agency}

\begin{frame}{Vibrant Matter (Jane Bennett)}
\protect\phantomsection\label{vibrant-matter-jane-bennett}
Jane Bennett's \textbf{vital materialism} attributes agency to matter
itself:

{\def\LTcaptype{none} % do not increment counter
\begin{longtable}[]{@{}ll@{}}
\toprule\noalign{}
Bennett & Boundary Logic \\
\midrule\noalign{}
\endhead
Actants & Forms as actors \\
Assemblages & Juxtapositions \\
Thing-power & Reduction capacity \\
\bottomrule\noalign{}
\end{longtable}
}

Forms are not passive representations but active participants in
reduction---they \emph{do} things.
\end{frame}

\begin{frame}{Material Semiotics (ANT)}
\protect\phantomsection\label{material-semiotics-ant}
Actor-Network Theory's \textbf{material semiotics}: - Signs and things
are equally actors - Networks are heterogeneous assemblages -
Translation transforms identities

The calculus of indications is maximally material-semiotic: the notation
(material marks) \emph{is} the logic (semiotic structure).
\end{frame}

\subsection{S3.7 Synthesis: Pragmatist-Materialist
Containment}\label{s3.7-synthesis-pragmatist-materialist-containment}

\begin{frame}{Core Commitments}
\protect\phantomsection\label{core-commitments}
From these traditions, Containment Theory inherits:

\begin{enumerate}
\tightlist
\item
  \textbf{Anti-representationalism} (Pragmatism): Forms don't represent;
  they enact
\item
  \textbf{Relational ontology} (Neo-materialism): Boundaries constitute
  entities
\item
  \textbf{Process primacy} (Whitehead): Becoming precedes being
\item
  \textbf{Situatedness} (Haraway): Observer within system
\item
  \textbf{Difference primacy} (Deleuze): Distinction before identity
\end{enumerate}
\end{frame}

\begin{frame}{The Mark as Pragmatic-Materialist Primitive}
\protect\phantomsection\label{the-mark-as-pragmatic-materialist-primitive}
The mark \(\langle\ \rangle\) unifies: - \textbf{Pragmatist}:
Operational definition (effects = meaning) - \textbf{Materialist}:
Physical inscription (matter makes marks) - \textbf{Processual}:
Temporal act (distinction happens) - \textbf{Relational}: Creates
relations (inside/outside)
\end{frame}

\begin{frame}{Research Program}
\protect\phantomsection\label{research-program}
This philosophical grounding suggests:

\begin{enumerate}
\tightlist
\item
  \textbf{Experimental Pragmatism}: Test forms by their consequences
\item
  \textbf{Material Practice}: Implement forms in physical media
\item
  \textbf{Processual Analysis}: Study reduction as temporal unfolding
\item
  \textbf{Ecological Thinking}: Forms in environments of other forms
\end{enumerate}
\end{frame}

\subsection{S3.8 Key Texts and
Lineages}\label{s3.8-key-texts-and-lineages}

\begin{frame}{North American Pragmatism}
\protect\phantomsection\label{north-american-pragmatism}
{\def\LTcaptype{none} % do not increment counter
\begin{longtable}[]{@{}
  >{\raggedright\arraybackslash}p{(\linewidth - 4\tabcolsep) * \real{0.2667}}
  >{\raggedright\arraybackslash}p{(\linewidth - 4\tabcolsep) * \real{0.3333}}
  >{\raggedright\arraybackslash}p{(\linewidth - 4\tabcolsep) * \real{0.4000}}@{}}
\toprule\noalign{}
\begin{minipage}[b]{\linewidth}\raggedright
Author
\end{minipage} & \begin{minipage}[b]{\linewidth}\raggedright
Key Work
\end{minipage} & \begin{minipage}[b]{\linewidth}\raggedright
Connection
\end{minipage} \\
\midrule\noalign{}
\endhead
C.S. Peirce & \emph{Collected Papers} (1931-58) & Existential graphs,
icons \\
William James & \emph{Essays in Radical Empiricism} (1912) & Relations
as real \\
John Dewey & \emph{Logic: The Theory of Inquiry} (1938) & Inquiry as
distinction \\
George Herbert Mead & \emph{Mind, Self, and Society} (1934) &
Self-reference \\
Richard Rorty & \emph{Philosophy and the Mirror of Nature} (1979) &
Anti-representationalism \\
Robert Brandom & \emph{Making It Explicit} (1994) & Inferential
semantics \\
\bottomrule\noalign{}
\end{longtable}
}
\end{frame}

\begin{frame}{Process Philosophy}
\protect\phantomsection\label{process-philosophy}
{\def\LTcaptype{none} % do not increment counter
\begin{longtable}[]{@{}
  >{\raggedright\arraybackslash}p{(\linewidth - 4\tabcolsep) * \real{0.2667}}
  >{\raggedright\arraybackslash}p{(\linewidth - 4\tabcolsep) * \real{0.3333}}
  >{\raggedright\arraybackslash}p{(\linewidth - 4\tabcolsep) * \real{0.4000}}@{}}
\toprule\noalign{}
\begin{minipage}[b]{\linewidth}\raggedright
Author
\end{minipage} & \begin{minipage}[b]{\linewidth}\raggedright
Key Work
\end{minipage} & \begin{minipage}[b]{\linewidth}\raggedright
Connection
\end{minipage} \\
\midrule\noalign{}
\endhead
A.N. Whitehead & \emph{Process and Reality} (1929) & Actual entities,
creativity \\
Charles Hartshorne & \emph{Creative Synthesis} (1970) &
Panexperientialism \\
Isabelle Stengers & \emph{Thinking with Whitehead} (2011) & Speculative
philosophy \\
\bottomrule\noalign{}
\end{longtable}
}
\end{frame}

\begin{frame}{Neo-Materialism}
\protect\phantomsection\label{neo-materialism}
{\def\LTcaptype{none} % do not increment counter
\begin{longtable}[]{@{}
  >{\raggedright\arraybackslash}p{(\linewidth - 4\tabcolsep) * \real{0.2667}}
  >{\raggedright\arraybackslash}p{(\linewidth - 4\tabcolsep) * \real{0.3333}}
  >{\raggedright\arraybackslash}p{(\linewidth - 4\tabcolsep) * \real{0.4000}}@{}}
\toprule\noalign{}
\begin{minipage}[b]{\linewidth}\raggedright
Author
\end{minipage} & \begin{minipage}[b]{\linewidth}\raggedright
Key Work
\end{minipage} & \begin{minipage}[b]{\linewidth}\raggedright
Connection
\end{minipage} \\
\midrule\noalign{}
\endhead
Karen Barad & \emph{Meeting the Universe Halfway} (2007) & Agential
cuts \\
Donna Haraway & \emph{Staying with the Trouble} (2016) & Situated
becoming \\
Jane Bennett & \emph{Vibrant Matter} (2010) & Thing-power \\
Rosi Braidotti & \emph{The Posthuman} (2013) & Affirmative ethics \\
\bottomrule\noalign{}
\end{longtable}
}
\end{frame}

\begin{frame}{Continental Connections}
\protect\phantomsection\label{continental-connections}
{\def\LTcaptype{none} % do not increment counter
\begin{longtable}[]{@{}
  >{\raggedright\arraybackslash}p{(\linewidth - 4\tabcolsep) * \real{0.2667}}
  >{\raggedright\arraybackslash}p{(\linewidth - 4\tabcolsep) * \real{0.3333}}
  >{\raggedright\arraybackslash}p{(\linewidth - 4\tabcolsep) * \real{0.4000}}@{}}
\toprule\noalign{}
\begin{minipage}[b]{\linewidth}\raggedright
Author
\end{minipage} & \begin{minipage}[b]{\linewidth}\raggedright
Key Work
\end{minipage} & \begin{minipage}[b]{\linewidth}\raggedright
Connection
\end{minipage} \\
\midrule\noalign{}
\endhead
Gilles Deleuze & \emph{Difference and Repetition} (1968) & Difference in
itself \\
Brian Massumi & \emph{Parables for the Virtual} (2002) & Affect,
intensity \\
Gilbert Simondon & \emph{Individuation} (1958) & Transduction \\
Bruno Latour & \emph{We Have Never Been Modern} (1991) &
Actor-networks \\
\bottomrule\noalign{}
\end{longtable}
}
\end{frame}

\end{document}
