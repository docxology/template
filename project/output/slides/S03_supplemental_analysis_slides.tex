% Options for packages loaded elsewhere
\PassOptionsToPackage{unicode}{hyperref}
\PassOptionsToPackage{hyphens}{url}
\documentclass[
  ignorenonframetext,
]{beamer}
\newif\ifbibliography
\usepackage{pgfpages}
\setbeamertemplate{caption}[numbered]
\setbeamertemplate{caption label separator}{: }
\setbeamercolor{caption name}{fg=normal text.fg}
\beamertemplatenavigationsymbolsempty
% remove section numbering
\setbeamertemplate{part page}{
  \centering
  \begin{beamercolorbox}[sep=16pt,center]{part title}
    \usebeamerfont{part title}\insertpart\par
  \end{beamercolorbox}
}
\setbeamertemplate{section page}{
  \centering
  \begin{beamercolorbox}[sep=12pt,center]{section title}
    \usebeamerfont{section title}\insertsection\par
  \end{beamercolorbox}
}
\setbeamertemplate{subsection page}{
  \centering
  \begin{beamercolorbox}[sep=8pt,center]{subsection title}
    \usebeamerfont{subsection title}\insertsubsection\par
  \end{beamercolorbox}
}
% Prevent slide breaks in the middle of a paragraph
\widowpenalties 1 10000
\raggedbottom
\AtBeginPart{
  \frame{\partpage}
}
\AtBeginSection{
  \ifbibliography
  \else
    \frame{\sectionpage}
  \fi
}
\AtBeginSubsection{
  \frame{\subsectionpage}
}
\usepackage{iftex}
\ifPDFTeX
  \usepackage[T1]{fontenc}
  \usepackage[utf8]{inputenc}
  \usepackage{textcomp} % provide euro and other symbols
\else % if luatex or xetex
  \usepackage{unicode-math} % this also loads fontspec
  \defaultfontfeatures{Scale=MatchLowercase}
  \defaultfontfeatures[\rmfamily]{Ligatures=TeX,Scale=1}
\fi
\usepackage{lmodern}
\ifPDFTeX\else
  % xetex/luatex font selection
\fi
% Use upquote if available, for straight quotes in verbatim environments
\IfFileExists{upquote.sty}{\usepackage{upquote}}{}
\IfFileExists{microtype.sty}{% use microtype if available
  \usepackage[]{microtype}
  \UseMicrotypeSet[protrusion]{basicmath} % disable protrusion for tt fonts
}{}
\makeatletter
\@ifundefined{KOMAClassName}{% if non-KOMA class
  \IfFileExists{parskip.sty}{%
    \usepackage{parskip}
  }{% else
    \setlength{\parindent}{0pt}
    \setlength{\parskip}{6pt plus 2pt minus 1pt}}
}{% if KOMA class
  \KOMAoptions{parskip=half}}
\makeatother
\setlength{\emergencystretch}{3em} % prevent overfull lines
\providecommand{\tightlist}{%
  \setlength{\itemsep}{0pt}\setlength{\parskip}{0pt}}
\usepackage{bookmark}
\IfFileExists{xurl.sty}{\usepackage{xurl}}{} % add URL line breaks if available
\urlstyle{same}
\hypersetup{
  hidelinks,
  pdfcreator={LaTeX via pandoc}}

\author{\texorpdfstring{}{}}
\date{}

\begin{document}

\begin{frame}{Supplemental Analysis}
\protect\phantomsection\label{sec:supplemental_analysis}
This section provides detailed analytical results and theoretical
extensions that complement the main findings.

\begin{block}{S3.1 Theoretical Framework Extensions}
\protect\phantomsection\label{s3.1-theoretical-framework-extensions}
\begin{block}{S3.1.1 Epistemological Foundations}
\protect\phantomsection\label{s3.1.1-epistemological-foundations}
The Ways framework extends traditional epistemology by:

\begin{enumerate}
\tightlist
\item
  \textbf{Pluralism}: Recognizing multiple valid ways of knowing
\item
  \textbf{Structure}: Providing organization through the House of
  Knowledge
\item
  \textbf{Dialogue}: Emphasizing relational aspects of knowledge
\item
  \textbf{Integration}: Combining belief, care, and learning
\end{enumerate}
\end{block}

\begin{block}{S3.1.2 Learning Theory Integration}
\protect\phantomsection\label{s3.1.2-learning-theory-integration}
The framework integrates with learning theory through:

\begin{itemize}
\tightlist
\item
  \textbf{Believing structures}: Connect to constructivist learning
\item
  \textbf{Caring structures}: Relate to experiential learning
\item
  \textbf{Relative learning}: Maps to iterative/reflective learning
  cycles
\end{itemize}
\end{block}
\end{block}

\begin{block}{S3.2 Network Analysis Extensions}
\protect\phantomsection\label{s3.2-network-analysis-extensions}
\begin{block}{S3.2.1 Advanced Centrality Metrics}
\protect\phantomsection\label{s3.2.1-advanced-centrality-metrics}
Beyond basic centrality, we analyze:

\begin{itemize}
\tightlist
\item
  \textbf{Eigenvector Centrality}: Importance based on connections to
  important nodes
\item
  \textbf{PageRank}: Adapted for way importance
\item
  \textbf{Katz Centrality}: Weighted importance with attenuation
\end{itemize}
\end{block}

\begin{block}{S3.2.2 Network Motifs}
\protect\phantomsection\label{s3.2.2-network-motifs}
Analysis of network motifs (small subgraph patterns) reveals: - Common
3-node patterns - 4-node structures - Recurring motifs that indicate
framework structure
\end{block}

\begin{block}{S3.2.3 Network Resilience}
\protect\phantomsection\label{s3.2.3-network-resilience}
Analysis of network resilience shows: - Critical ways (removal
significantly affects connectivity) - Robustness to way removal -
Network structure stability
\end{block}
\end{block}

\begin{block}{S3.3 Statistical Model Extensions}
\protect\phantomsection\label{s3.3-statistical-model-extensions}
\begin{block}{S3.3.1 Multivariate Analysis}
\protect\phantomsection\label{s3.3.1-multivariate-analysis}
Multivariate analysis examines: - Relationships between multiple
variables simultaneously - Factor analysis of way characteristics -
Principal component analysis of way space
\end{block}

\begin{block}{S3.3.2 Predictive Models}
\protect\phantomsection\label{s3.3.2-predictive-models}
Models for predicting: - Way characteristics from other features - Room
assignments from way descriptions - Dialogue types from way content
\end{block}
\end{block}

\begin{block}{S3.4 Text Analysis Extensions}
\protect\phantomsection\label{s3.4-text-analysis-extensions}
\begin{block}{S3.4.1 Natural Language Processing}
\protect\phantomsection\label{s3.4.1-natural-language-processing}
Advanced NLP techniques: - Named entity recognition - Semantic
similarity analysis - Topic modeling (LDA, etc.) - Sentiment analysis of
way descriptions
\end{block}

\begin{block}{S3.4.2 Cross-Language Analysis}
\protect\phantomsection\label{s3.4.2-cross-language-analysis}
If Lithuanian text is present: - Translation analysis - Cross-language
pattern comparison - Cultural context analysis
\end{block}
\end{block}

\begin{block}{S3.5 Comparative Analysis}
\protect\phantomsection\label{s3.5-comparative-analysis}
\begin{block}{S3.5.1 Framework Comparison}
\protect\phantomsection\label{s3.5.1-framework-comparison}
Comparison with other epistemological frameworks: - Similarities and
differences - Unique contributions of Ways framework - Integration
possibilities
\end{block}

\begin{block}{S3.5.2 Domain-Specific Analysis}
\protect\phantomsection\label{s3.5.2-domain-specific-analysis}
Analysis of ways in specific domains: - Scientific ways - Artistic ways
- Practical ways - Spiritual ways
\end{block}
\end{block}

\begin{block}{S3.6 Computational Complexity}
\protect\phantomsection\label{s3.6-computational-complexity}
\begin{block}{S3.6.1 Analysis Complexity}
\protect\phantomsection\label{s3.6.1-analysis-complexity}
Computational requirements for \(n = 210\) ways:

\textbf{Network Construction:} - Room-based edges: \(O(n^2)\) in worst
case, but typically \(O(n \cdot k)\) where \(k\) is average ways per
room - Partner-based edges: \(O(n^2)\) in worst case - Type-based edges:
\(O(n^2)\) in worst case - Total: \(O(n^2)\) resulting in
\(|E| = 1,290\) edges

\textbf{Centrality Computation:} - Degree centrality:
\(O(|E|) = O(1,290)\) - Betweenness centrality:
\(O(n \cdot |E|) = O(210 \times 1,290) = O(270,900)\) - Closeness
centrality: \(O(n \cdot |E|)\) using BFS - Eigenvector centrality:
\(O(|E| \cdot \text{iterations})\) typically 50-100 iterations

\textbf{Cross-Tabulation:} - Type × Room: \(O(n) = O(210)\) single pass
through ways - Type × Partner: \(O(n) = O(210)\) - Total: \(O(n)\)
linear time

\textbf{Information-Theoretic Metrics:} - Entropy calculation: \(O(k)\)
where \(k\) is number of categories (typically \(k < 50\)) - Mutual
information: \(O(k_1 \cdot k_2)\) for two categorical variables - Total:
\(O(k^2)\) where \(k\) is bounded by number of categories
\end{block}

\begin{block}{S3.6.2 Scalability}
\protect\phantomsection\label{s3.6.2-scalability}
Scalability analysis for: - Large numbers of ways - Extended
relationship networks - Real-time analysis requirements
\end{block}
\end{block}

\begin{block}{S3.7 Validation and Robustness}
\protect\phantomsection\label{s3.7-validation-and-robustness}
\begin{block}{S3.7.1 Cross-Validation}
\protect\phantomsection\label{s3.7.1-cross-validation}
Cross-validation approaches: - K-fold validation of statistical models -
Bootstrap sampling for confidence intervals - Leave-one-out validation
\end{block}

\begin{block}{S3.7.2 Sensitivity Analysis}
\protect\phantomsection\label{s3.7.2-sensitivity-analysis}
Sensitivity to: - Missing data - Data quality variations - Analysis
parameter choices - Network construction methods
\end{block}
\end{block}

\begin{block}{S3.8 Limitations and Assumptions}
\protect\phantomsection\label{s3.8-limitations-and-assumptions}
\begin{block}{S3.8.1 Methodological Limitations}
\protect\phantomsection\label{s3.8.1-methodological-limitations}
\begin{itemize}
\tightlist
\item
  Quantitative analysis may miss qualitative nuances
\item
  Network structure based on explicit database relationships
\item
  Text analysis limited by available descriptions
\item
  Assumptions about way independence
\end{itemize}
\end{block}

\begin{block}{S3.8.2 Data Limitations}
\protect\phantomsection\label{s3.8.2-data-limitations}
\begin{itemize}
\tightlist
\item
  Incomplete metadata for some ways
\item
  Potential biases in way documentation
\item
  Limited temporal information
\item
  Cultural context considerations
\end{itemize}
\end{block}
\end{block}

\begin{block}{S3.9 Future Analytical Directions}
\protect\phantomsection\label{s3.9-future-analytical-directions}
\begin{block}{S3.9.1 Advanced Network Analysis}
\protect\phantomsection\label{s3.9.1-advanced-network-analysis}
\begin{itemize}
\tightlist
\item
  Temporal network analysis (if dating available)
\item
  Multilayer network analysis
\item
  Dynamic network models
\end{itemize}
\end{block}

\begin{block}{S3.9.2 Machine Learning Applications}
\protect\phantomsection\label{s3.9.2-machine-learning-applications}
\begin{itemize}
\tightlist
\item
  Classification of ways
\item
  Clustering analysis
\item
  Recommendation systems
\item
  Predictive modeling
\end{itemize}
\end{block}

\begin{block}{S3.9.3 Interdisciplinary Integration}
\protect\phantomsection\label{s3.9.3-interdisciplinary-integration}
Integration with: - Cognitive science - Educational research -
Philosophy of science - Knowledge management
\end{block}
\end{block}
\end{frame}

\end{document}
