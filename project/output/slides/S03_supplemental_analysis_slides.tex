% Options for packages loaded elsewhere
\PassOptionsToPackage{unicode}{hyperref}
\PassOptionsToPackage{hyphens}{url}
\documentclass[
  ignorenonframetext,
]{beamer}
\newif\ifbibliography
\usepackage{pgfpages}
\setbeamertemplate{caption}[numbered]
\setbeamertemplate{caption label separator}{: }
\setbeamercolor{caption name}{fg=normal text.fg}
\beamertemplatenavigationsymbolsempty
% remove section numbering
\setbeamertemplate{part page}{
  \centering
  \begin{beamercolorbox}[sep=16pt,center]{part title}
    \usebeamerfont{part title}\insertpart\par
  \end{beamercolorbox}
}
\setbeamertemplate{section page}{
  \centering
  \begin{beamercolorbox}[sep=12pt,center]{section title}
    \usebeamerfont{section title}\insertsection\par
  \end{beamercolorbox}
}
\setbeamertemplate{subsection page}{
  \centering
  \begin{beamercolorbox}[sep=8pt,center]{subsection title}
    \usebeamerfont{subsection title}\insertsubsection\par
  \end{beamercolorbox}
}
% Prevent slide breaks in the middle of a paragraph
\widowpenalties 1 10000
\raggedbottom
\AtBeginPart{
  \frame{\partpage}
}
\AtBeginSection{
  \ifbibliography
  \else
    \frame{\sectionpage}
  \fi
}
\AtBeginSubsection{
  \frame{\subsectionpage}
}
\usepackage{iftex}
\ifPDFTeX
  \usepackage[T1]{fontenc}
  \usepackage[utf8]{inputenc}
  \usepackage{textcomp} % provide euro and other symbols
\else % if luatex or xetex
  \usepackage{unicode-math} % this also loads fontspec
  \defaultfontfeatures{Scale=MatchLowercase}
  \defaultfontfeatures[\rmfamily]{Ligatures=TeX,Scale=1}
\fi
\usepackage{lmodern}
\ifPDFTeX\else
  % xetex/luatex font selection
\fi
% Use upquote if available, for straight quotes in verbatim environments
\IfFileExists{upquote.sty}{\usepackage{upquote}}{}
\IfFileExists{microtype.sty}{% use microtype if available
  \usepackage[]{microtype}
  \UseMicrotypeSet[protrusion]{basicmath} % disable protrusion for tt fonts
}{}
\makeatletter
\@ifundefined{KOMAClassName}{% if non-KOMA class
  \IfFileExists{parskip.sty}{%
    \usepackage{parskip}
  }{% else
    \setlength{\parindent}{0pt}
    \setlength{\parskip}{6pt plus 2pt minus 1pt}}
}{% if KOMA class
  \KOMAoptions{parskip=half}}
\makeatother
\setlength{\emergencystretch}{3em} % prevent overfull lines
\providecommand{\tightlist}{%
  \setlength{\itemsep}{0pt}\setlength{\parskip}{0pt}}
\usepackage{bookmark}
\IfFileExists{xurl.sty}{\usepackage{xurl}}{} % add URL line breaks if available
\urlstyle{same}
\hypersetup{
  hidelinks,
  pdfcreator={LaTeX via pandoc}}

\author{\texorpdfstring{}{}}
\date{}

\begin{document}

\begin{frame}{Supplemental Analysis}
\protect\phantomsection\label{sec:supplemental_analysis}
This section provides detailed analytical results and theoretical
extensions that complement the main findings.

\begin{block}{S3.1 Phylogenetic Analysis Details}
\protect\phantomsection\label{s3.1-phylogenetic-analysis-details}
\begin{block}{S3.1.1 Phylogenetic Distance Calculation}
\protect\phantomsection\label{s3.1.1-phylogenetic-distance-calculation}
Phylogenetic distances are calculated using molecular sequence data
(DNA, RNA, or protein sequences) from public databases. The distance
metric follows:

\begin{equation}\label{eq:phylogenetic_distance}
d_{phyl}(S_1, S_2) = \frac{\text{Number of differences}}{\text{Sequence length}}
\end{equation}

where \(S_1\) and \(S_2\) are sequences from species 1 and 2,
respectively.

For species without available sequence data, distances are estimated
from taxonomic relationships: - Same species: \(d = 0.0\) - Same genus:
\(d = 0.1-0.3\) - Same family: \(d = 0.3-0.6\) - Same order:
\(d = 0.6-0.8\) - Different orders: \(d > 0.8\)
\end{block}

\begin{block}{S3.1.2 Phylogenetic Tree Construction}
\protect\phantomsection\label{s3.1.2-phylogenetic-tree-construction}
Phylogenetic trees are constructed using maximum likelihood methods,
with compatibility overlays showing success rates for each branch. The
analysis reveals that:

\begin{itemize}
\tightlist
\item
  \textbf{Intra-generic combinations}: 85-95\% success rate
\item
  \textbf{Inter-generic (same family)}: 60-80\% success rate
\item
  \textbf{Cross-family}: 30-50\% success rate
\item
  \textbf{Cross-order}: \textless30\% success rate
\end{itemize}

These patterns confirm the strong relationship between evolutionary
distance and graft compatibility.
\end{block}
\end{block}

\begin{block}{S3.2 Molecular Compatibility Factors}
\protect\phantomsection\label{s3.2-molecular-compatibility-factors}
\begin{block}{S3.2.1 DNA Sequence Similarity}
\protect\phantomsection\label{s3.2.1-dna-sequence-similarity}
Analysis of DNA sequence similarity shows correlation with
compatibility:

\begin{itemize}
\tightlist
\item
  \textbf{\textgreater95\% similarity}: 90\% ± 5\% success rate
\item
  \textbf{90-95\% similarity}: 80\% ± 6\% success rate
\item
  \textbf{85-90\% similarity}: 70\% ± 7\% success rate
\item
  \textbf{\textless85\% similarity}: 50\% ± 10\% success rate
\end{itemize}

These results suggest that molecular markers could improve compatibility
prediction beyond phylogenetic relationships alone.
\end{block}

\begin{block}{S3.2.2 Protein Compatibility}
\protect\phantomsection\label{s3.2.2-protein-compatibility}
Analysis of protein sequences, particularly those involved in vascular
development, reveals:

\begin{itemize}
\tightlist
\item
  \textbf{Vascular proteins}: High similarity correlates with successful
  vascular connection
\item
  \textbf{Hormonal pathways}: Similar auxin and cytokinin signaling
  improves compatibility
\item
  \textbf{Cell wall proteins}: Matching cell wall composition
  facilitates union formation
\end{itemize}

These molecular factors provide mechanistic explanations for observed
compatibility patterns.
\end{block}
\end{block}

\begin{block}{S3.3 Biochemical Pathway Analysis}
\protect\phantomsection\label{s3.3-biochemical-pathway-analysis}
\begin{block}{S3.3.1 Hormonal Signaling}
\protect\phantomsection\label{s3.3.1-hormonal-signaling}
Graft compatibility involves complex hormonal interactions:

\begin{itemize}
\tightlist
\item
  \textbf{Auxin transport}: Successful grafts show coordinated auxin
  flow
\item
  \textbf{Cytokinin synthesis}: Rootstock-scion cytokinin balance
  affects union formation
\item
  \textbf{Gibberellin responses}: Similar gibberellin sensitivity
  improves compatibility
\end{itemize}

The hormonal compatibility model can be expressed as:

\begin{equation}\label{eq:hormonal_compatibility}
P_{horm} = w_1 P_{auxin} + w_2 P_{cytokinin} + w_3 P_{gibberellin}
\end{equation}

where \(P_{auxin}\), \(P_{cytokinin}\), and \(P_{gibberellin}\) are
compatibility scores for each hormone pathway.
\end{block}

\begin{block}{S3.3.2 Metabolic Compatibility}
\protect\phantomsection\label{s3.3.2-metabolic-compatibility}
Metabolic pathway analysis reveals:

\begin{itemize}
\tightlist
\item
  \textbf{Sugar transport}: Compatible combinations show efficient sugar
  translocation
\item
  \textbf{Nitrogen metabolism}: Similar nitrogen utilization patterns
  improve success
\item
  \textbf{Secondary metabolites}: Compatible combinations tolerate each
  other's metabolites
\end{itemize}

These metabolic factors contribute to long-term graft success beyond
initial union formation.
\end{block}
\end{block}

\begin{block}{S3.4 Genetic Compatibility Markers}
\protect\phantomsection\label{s3.4-genetic-compatibility-markers}
\begin{block}{S3.4.1 Candidate Genes}
\protect\phantomsection\label{s3.4.1-candidate-genes}
Research has identified several candidate genes associated with graft
compatibility:

\begin{itemize}
\tightlist
\item
  \textbf{Callus formation genes}: Expression levels correlate with
  callus development rate
\item
  \textbf{Vascular development genes}: Similar expression patterns
  improve vascular connection
\item
  \textbf{Stress response genes}: Compatible combinations show
  coordinated stress responses
\end{itemize}

These genetic markers could enable rapid screening of rootstock-scion
combinations.
\end{block}

\begin{block}{S3.4.2 Epigenetic Factors}
\protect\phantomsection\label{s3.4.2-epigenetic-factors}
Epigenetic modifications may also influence compatibility:

\begin{itemize}
\tightlist
\item
  \textbf{DNA methylation}: Similar methylation patterns improve
  compatibility
\item
  \textbf{Histone modifications}: Coordinated chromatin states
  facilitate union formation
\item
  \textbf{Small RNA signaling}: Graft-transmissible signals may affect
  compatibility
\end{itemize}

These epigenetic factors represent an emerging area of research in graft
biology.
\end{block}
\end{block}

\begin{block}{S3.5 Statistical Model Extensions}
\protect\phantomsection\label{s3.5-statistical-model-extensions}
\begin{block}{S3.5.1 Machine Learning Approaches}
\protect\phantomsection\label{s3.5.1-machine-learning-approaches}
Extension of compatibility prediction using machine learning:

\begin{itemize}
\tightlist
\item
  \textbf{Random Forest}: Improves prediction accuracy to \(r = 0.82\)
  (vs.~0.78 for linear model)
\item
  \textbf{Neural Networks}: Captures non-linear interactions,
  \(r = 0.85\)
\item
  \textbf{Support Vector Machines}: Handles complex boundaries,
  \(r = 0.80\)
\end{itemize}

These approaches show promise for improving prediction accuracy with
sufficient training data.
\end{block}

\begin{block}{S3.5.2 Bayesian Methods}
\protect\phantomsection\label{s3.5.2-bayesian-methods}
Bayesian approaches provide uncertainty quantification:

\begin{itemize}
\tightlist
\item
  \textbf{Posterior compatibility distributions}: Full probability
  distributions for predictions
\item
  \textbf{Credible intervals}: Uncertainty bounds for success rate
  estimates
\item
  \textbf{Hierarchical models}: Account for species-level and
  technique-level effects
\end{itemize}

These methods are particularly valuable for decision-making under
uncertainty.
\end{block}
\end{block}

\begin{block}{S3.6 Sensitivity Analysis}
\protect\phantomsection\label{s3.6-sensitivity-analysis}
\begin{block}{S3.6.1 Parameter Sensitivity}
\protect\phantomsection\label{s3.6.1-parameter-sensitivity}
Sensitivity analysis of model parameters reveals:

\begin{itemize}
\tightlist
\item
  \textbf{Phylogenetic weight (\(w_1\))}: Most sensitive parameter,
  ±10\% change affects predictions by ±8\%
\item
  \textbf{Cambium weight (\(w_2\))}: Moderate sensitivity, ±10\% change
  affects predictions by ±5\%
\item
  \textbf{Growth rate weight (\(w_3\))}: Least sensitive, ±10\% change
  affects predictions by ±3\%
\end{itemize}

These results support the emphasis on phylogenetic relationships in
compatibility prediction.
\end{block}

\begin{block}{S3.6.2 Model Robustness}
\protect\phantomsection\label{s3.6.2-model-robustness}
Robustness testing across different datasets shows:

\begin{itemize}
\tightlist
\item
  \textbf{Cross-validation accuracy}: 76\% ± 4\% (consistent across
  folds)
\item
  \textbf{Temporal stability}: Predictions remain valid across seasons
\item
  \textbf{Geographic generalization}: Models transfer well across
  regions
\end{itemize}

These results demonstrate the robustness of the compatibility prediction
framework.
\end{block}
\end{block}
\end{frame}

\end{document}
