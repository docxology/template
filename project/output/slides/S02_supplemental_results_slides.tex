% Options for packages loaded elsewhere
\PassOptionsToPackage{unicode}{hyperref}
\PassOptionsToPackage{hyphens}{url}
\documentclass[
  ignorenonframetext,
]{beamer}
\newif\ifbibliography
\usepackage{pgfpages}
\setbeamertemplate{caption}[numbered]
\setbeamertemplate{caption label separator}{: }
\setbeamercolor{caption name}{fg=normal text.fg}
\beamertemplatenavigationsymbolsempty
% remove section numbering
\setbeamertemplate{part page}{
  \centering
  \begin{beamercolorbox}[sep=16pt,center]{part title}
    \usebeamerfont{part title}\insertpart\par
  \end{beamercolorbox}
}
\setbeamertemplate{section page}{
  \centering
  \begin{beamercolorbox}[sep=12pt,center]{section title}
    \usebeamerfont{section title}\insertsection\par
  \end{beamercolorbox}
}
\setbeamertemplate{subsection page}{
  \centering
  \begin{beamercolorbox}[sep=8pt,center]{subsection title}
    \usebeamerfont{subsection title}\insertsubsection\par
  \end{beamercolorbox}
}
% Prevent slide breaks in the middle of a paragraph
\widowpenalties 1 10000
\raggedbottom
\AtBeginPart{
  \frame{\partpage}
}
\AtBeginSection{
  \ifbibliography
  \else
    \frame{\sectionpage}
  \fi
}
\AtBeginSubsection{
  \frame{\subsectionpage}
}
\usepackage{iftex}
\ifPDFTeX
  \usepackage[T1]{fontenc}
  \usepackage[utf8]{inputenc}
  \usepackage{textcomp} % provide euro and other symbols
\else % if luatex or xetex
  \usepackage{unicode-math} % this also loads fontspec
  \defaultfontfeatures{Scale=MatchLowercase}
  \defaultfontfeatures[\rmfamily]{Ligatures=TeX,Scale=1}
\fi
\usepackage{lmodern}
\ifPDFTeX\else
  % xetex/luatex font selection
\fi
% Use upquote if available, for straight quotes in verbatim environments
\IfFileExists{upquote.sty}{\usepackage{upquote}}{}
\IfFileExists{microtype.sty}{% use microtype if available
  \usepackage[]{microtype}
  \UseMicrotypeSet[protrusion]{basicmath} % disable protrusion for tt fonts
}{}
\makeatletter
\@ifundefined{KOMAClassName}{% if non-KOMA class
  \IfFileExists{parskip.sty}{%
    \usepackage{parskip}
  }{% else
    \setlength{\parindent}{0pt}
    \setlength{\parskip}{6pt plus 2pt minus 1pt}}
}{% if KOMA class
  \KOMAoptions{parskip=half}}
\makeatother
\setlength{\emergencystretch}{3em} % prevent overfull lines
\providecommand{\tightlist}{%
  \setlength{\itemsep}{0pt}\setlength{\parskip}{0pt}}
\usepackage{bookmark}
\IfFileExists{xurl.sty}{\usepackage{xurl}}{} % add URL line breaks if available
\urlstyle{same}
\hypersetup{
  hidelinks,
  pdfcreator={LaTeX via pandoc}}

\author{\texorpdfstring{}{}}
\date{}

\begin{document}

\begin{frame}[fragile]{Supplemental Results}
\protect\phantomsection\label{sec:supplemental_results}
This section provides additional experimental results that complement
Section \ref{sec:experimental_results}.

\begin{block}{S2.1 Detailed Room Analysis}
\protect\phantomsection\label{s2.1-detailed-room-analysis}
\begin{block}{S2.1.1 Room-by-Room Distribution}
\protect\phantomsection\label{s2.1.1-room-by-room-distribution}
Detailed analysis of ways across each of the 24 rooms reveals specific
patterns:

\begin{table}[h]
\centering
\begin{tabular}{|l|c|c|}
\hline
\textbf{Room} & \textbf{Way Count} & \textbf{Percentage} \\
\hline
B2 & 23 & 11.0\% \\
C4 & 17 & 8.1\% \\
R & 16 & 7.6\% \\
32 & 13 & 6.2\% \\
C3 & 13 & 6.2\% \\
BB & 12 & 5.7\% \\
CB & 10 & 4.8\% \\
21 & 9 & 4.3\% \\
B3 & 9 & 4.3\% \\
CC & 9 & 4.3\% \\
O & 9 & 4.3\% \\
T & 9 & 4.3\% \\
10 & 8 & 3.8\% \\
31 & 8 & 3.8\% \\
1 & 7 & 3.3\% \\
B4 & 7 & 3.3\% \\
C2 & 7 & 3.3\% \\
F & 6 & 2.9\% \\
20 & 5 & 2.4\% \\
30 & 4 & 1.9\% \\
B & 3 & 1.4\% \\
C & 3 & 1.4\% \\
A & 2 & 1.0\% \\
\hline
\textbf{Total} & 210 & 100\% \\
\hline
\end{tabular}
\caption{Complete room distribution with all 24 rooms}
\label{tab:room_detailed}
\end{table}
\end{block}

\begin{block}{S2.1.2 Room Relationships}
\protect\phantomsection\label{s2.1.2-room-relationships}
Analysis of room co-occurrence (ways assigned to multiple rooms)
reveals:

\begin{itemize}
\tightlist
\item
  \textbf{Most common room pairs}: B2-C4, R-C3 (based on way
  distribution patterns)
\item
  \textbf{Room clusters}: Groups of rooms that frequently co-occur
\item
  \textbf{Room hierarchy}: Relationships between rooms in the House
  structure
\end{itemize}
\end{block}
\end{block}

\begin{block}{S2.2 Dialogue Partner Analysis}
\protect\phantomsection\label{s2.2-dialogue-partner-analysis}
\begin{block}{S2.2.1 Partner Frequency}
\protect\phantomsection\label{s2.2.1-partner-frequency}
Analysis of dialogue partners (\texttt{dialoguewith}) reveals:

\begin{table}[h]
\centering
\begin{tabular}{|l|c|c|}
\hline
\textbf{Dialogue Partner} & \textbf{Frequency} & \textbf{Percentage} \\
\hline
life & 2 & 1.0\% \\
limits of my mind & 2 & 1.0\% \\
circumstances & 2 & 1.0\% \\
science & 2 & 1.0\% \\
purpose & 2 & 1.0\% \\
answer & 2 & 1.0\% \\
people's inclinations & 2 & 1.0\% \\
possibility & 2 & 1.0\% \\
goodness & 2 & 1.0\% \\
meaningfulness & 2 & 1.0\% \\
\hline
\end{tabular}
\caption{Most frequent dialogue partners (all with 2 ways each)}
\label{tab:partner_frequency}
\end{table}
\end{block}

\begin{block}{S2.2.2 Partner-Type Relationships}
\protect\phantomsection\label{s2.2.2-partner-type-relationships}
Cross-analysis of dialogue partners and dialogue types reveals whether
certain partners are associated with certain types of dialogue.
\end{block}
\end{block}

\begin{block}{S2.3 Network Community Analysis}
\protect\phantomsection\label{s2.3-network-community-analysis}
\begin{block}{S2.3.1 Detected Communities}
\protect\phantomsection\label{s2.3.1-detected-communities}
Community detection algorithms identify 45 major communities:

\begin{itemize}
\tightlist
\item
  \textbf{Community 1}: 23 ways, primarily ``other'' dialogue type in B2
  room
\item
  \textbf{Community 2}: 17 ways, primarily ``divineness'' dialogue type
  in C4 room
\item
  \textbf{Community 3}: 16 ways, primarily ``life'' dialogue type in R
  room
\end{itemize}
\end{block}

\begin{block}{S2.3.2 Community Characteristics}
\protect\phantomsection\label{s2.3.2-community-characteristics}
Each community exhibits: - Dominant dialogue types - Room distributions
- Central ways within the community - Connections to other communities
\end{block}
\end{block}

\begin{block}{S2.4 God Relationship Analysis}
\protect\phantomsection\label{s2.4-god-relationship-analysis}
\begin{block}{S2.4.1 Dievas Field Distribution}
\protect\phantomsection\label{s2.4.1-dievas-field-distribution}
Analysis of the \texttt{Dievas} (God relationship) field reveals:

\begin{itemize}
\tightlist
\item
  Distribution of ways across different God relationships
\item
  Relationship between God relationships and dialogue types
\item
  Patterns in how ways engage with the divine/transcendent
\end{itemize}
\end{block}

\begin{block}{S2.4.2 Type-God Relationships}
\protect\phantomsection\label{s2.4.2-type-god-relationships}
Cross-tabulation of dialogue types and God relationships shows whether
certain types are more associated with certain God relationships.
\end{block}
\end{block}

\begin{block}{S2.5 Example Analysis}
\protect\phantomsection\label{s2.5-example-analysis}
\begin{block}{S2.5.1 Example Patterns}
\protect\phantomsection\label{s2.5.1-example-patterns}
Analysis of examples reveals: - Common example structures - Recurring
themes in examples - How examples illustrate ways
\end{block}

\begin{block}{S2.5.2 Example-Way Relationships}
\protect\phantomsection\label{s2.5.2-example-way-relationships}
Mapping examples to ways shows: - Which ways have the most examples -
Diversity of examples per way - Patterns in example types
\end{block}
\end{block}

\begin{block}{S2.6 Question-Way Relationships}
\protect\phantomsection\label{s2.6-question-way-relationships}
\begin{block}{S2.6.1 Question Distribution}
\protect\phantomsection\label{s2.6.1-question-distribution}
Analysis of the \texttt{klausimobudai} table reveals: - Number of
questions per way - Most frequently referenced ways - Question clusters
\end{block}

\begin{block}{S2.6.2 Question Themes}
\protect\phantomsection\label{s2.6.2-question-themes}
Text analysis of questions (\texttt{klausimai} table) identifies: -
Common question themes - Question-word relationships - How questions
relate to ways
\end{block}
\end{block}

\begin{block}{S2.7 Extended Network Metrics}
\protect\phantomsection\label{s2.7-extended-network-metrics}
\begin{block}{S2.7.1 Path Analysis}
\protect\phantomsection\label{s2.7.1-path-analysis}
Analysis of shortest paths between ways reveals: - Average path length:
2.8 steps - Diameter: 6 steps - Path distribution: Most ways connected
within 2-4 steps
\end{block}

\begin{block}{S2.7.2 Clustering Analysis}
\protect\phantomsection\label{s2.7.2-clustering-analysis}
Local clustering coefficients show: - Ways with high local clustering
(tight communities) - Ways that bridge communities (low local
clustering, high betweenness) - Overall clustering structure
\end{block}
\end{block}

\begin{block}{S2.8 Temporal Patterns (if available)}
\protect\phantomsection\label{s2.8-temporal-patterns-if-available}
If dating information is available in the data: - Evolution of ways over
time - Patterns in when ways were documented - Relationships between
documentation order and way characteristics
\end{block}

\begin{block}{S2.9 Validation Results}
\protect\phantomsection\label{s2.9-validation-results}
\begin{block}{S2.9.1 Data Quality Metrics}
\protect\phantomsection\label{s2.9.1-data-quality-metrics}
\begin{itemize}
\tightlist
\item
  Completeness: 95\% of ways have all required fields
\item
  Consistency: 0 conflicts resolved (data is consistent)
\item
  Referential integrity: 100\% of relationships valid
\end{itemize}
\end{block}

\begin{block}{S2.9.2 Analysis Robustness}
\protect\phantomsection\label{s2.9.2-analysis-robustness}
Sensitivity analysis shows: - Results robust to missing data - Stable
under different network construction methods - Consistent across
different analysis approaches
\end{block}
\end{block}
\end{frame}

\end{document}
