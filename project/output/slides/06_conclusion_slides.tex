% Options for packages loaded elsewhere
\PassOptionsToPackage{unicode}{hyperref}
\PassOptionsToPackage{hyphens}{url}
\documentclass[
  ignorenonframetext,
]{beamer}
\newif\ifbibliography
\usepackage{pgfpages}
\setbeamertemplate{caption}[numbered]
\setbeamertemplate{caption label separator}{: }
\setbeamercolor{caption name}{fg=normal text.fg}
\beamertemplatenavigationsymbolsempty
% remove section numbering
\setbeamertemplate{part page}{
  \centering
  \begin{beamercolorbox}[sep=16pt,center]{part title}
    \usebeamerfont{part title}\insertpart\par
  \end{beamercolorbox}
}
\setbeamertemplate{section page}{
  \centering
  \begin{beamercolorbox}[sep=12pt,center]{section title}
    \usebeamerfont{section title}\insertsection\par
  \end{beamercolorbox}
}
\setbeamertemplate{subsection page}{
  \centering
  \begin{beamercolorbox}[sep=8pt,center]{subsection title}
    \usebeamerfont{subsection title}\insertsubsection\par
  \end{beamercolorbox}
}
% Prevent slide breaks in the middle of a paragraph
\widowpenalties 1 10000
\raggedbottom
\AtBeginPart{
  \frame{\partpage}
}
\AtBeginSection{
  \ifbibliography
  \else
    \frame{\sectionpage}
  \fi
}
\AtBeginSubsection{
  \frame{\subsectionpage}
}
\usepackage{iftex}
\ifPDFTeX
  \usepackage[T1]{fontenc}
  \usepackage[utf8]{inputenc}
  \usepackage{textcomp} % provide euro and other symbols
\else % if luatex or xetex
  \usepackage{unicode-math} % this also loads fontspec
  \defaultfontfeatures{Scale=MatchLowercase}
  \defaultfontfeatures[\rmfamily]{Ligatures=TeX,Scale=1}
\fi
\usepackage{lmodern}
\ifPDFTeX\else
  % xetex/luatex font selection
\fi
% Use upquote if available, for straight quotes in verbatim environments
\IfFileExists{upquote.sty}{\usepackage{upquote}}{}
\IfFileExists{microtype.sty}{% use microtype if available
  \usepackage[]{microtype}
  \UseMicrotypeSet[protrusion]{basicmath} % disable protrusion for tt fonts
}{}
\makeatletter
\@ifundefined{KOMAClassName}{% if non-KOMA class
  \IfFileExists{parskip.sty}{%
    \usepackage{parskip}
  }{% else
    \setlength{\parindent}{0pt}
    \setlength{\parskip}{6pt plus 2pt minus 1pt}}
}{% if KOMA class
  \KOMAoptions{parskip=half}}
\makeatother
\setlength{\emergencystretch}{3em} % prevent overfull lines
\providecommand{\tightlist}{%
  \setlength{\itemsep}{0pt}\setlength{\parskip}{0pt}}
\usepackage{bookmark}
\IfFileExists{xurl.sty}{\usepackage{xurl}}{} % add URL line breaks if available
\urlstyle{same}
\hypersetup{
  hidelinks,
  pdfcreator={LaTeX via pandoc}}

\author{\texorpdfstring{}{}}
\date{}

\begin{document}

\section{Conclusion}\label{sec:conclusion}

\begin{frame}{Summary of Contributions}
\protect\phantomsection\label{summary-of-contributions}
This comprehensive transdisciplinary review and computational toolkit
makes several significant contributions to the field of tree grafting:

\begin{enumerate}
\item
  \textbf{Biological Framework}: Comprehensive synthesis of graft
  compatibility mechanisms based on phylogenetic relationships, cambium
  characteristics, and growth rates, expressed through mathematical
  models
  \eqref{eq:phylogenetic_compatibility}-\eqref{eq:combined_compatibility}
\item
  \textbf{Technique Analysis}: Detailed analysis of major grafting
  techniques (whip \& tongue, cleft, bark, bud, approach, bridge,
  inarching) with success rate predictions and application guidelines
\item
  \textbf{Biological Simulation}: Computational models of cambium
  integration, callus formation, and vascular connection
  \eqref{eq:healing_dynamics}-\eqref{eq:vascular_dynamics} that capture
  temporal healing dynamics
\item
  \textbf{Compatibility Prediction}: Algorithms for predicting graft
  success based on multiple factors \eqref{eq:success_probability},
  validated through statistical analysis
\item
  \textbf{Decision Support Tools}: Interactive systems for rootstock
  selection, technique recommendation, seasonal planning, and economic
  analysis
\item
  \textbf{Comprehensive Review}: Transdisciplinary synthesis spanning
  4,000+ years of grafting history, biological mechanisms, technical
  methods, agricultural applications, and economic impacts
\end{enumerate}
\end{frame}

\begin{frame}{Key Findings}
\protect\phantomsection\label{key-findings}
\begin{block}{Biological Insights}
\protect\phantomsection\label{biological-insights}
The analysis confirms that phylogenetic distance is the strongest
predictor of graft compatibility (\(r = -0.75\)), with compatibility
decreasing exponentially as evolutionary relationships become more
distant. This finding supports evidence-based rootstock-scion selection,
prioritizing intra-generic combinations for high success rates.

The healing process follows a sequential pattern: cambium contact
enables callus formation, which facilitates vascular connection.
Environmental conditions (temperature 20-25°C, humidity 70-90\%)
significantly modulate healing rates, with optimal conditions improving
success by 15-20\%.
\end{block}

\begin{block}{Technical Recommendations}
\protect\phantomsection\label{technical-recommendations}
Technique selection should be based on rootstock diameter, species
characteristics, and precision requirements: - \textbf{Whip and tongue}:
Best for similar diameters (5-25 mm), highest success (85\%) -
\textbf{Bud grafting}: Most efficient for mass propagation (80\%
success) - \textbf{Cleft grafting}: Suitable for larger diameters (10-50
mm), moderate success (75\%) - \textbf{Bark grafting}: Useful for mature
trees (20-100 mm), lower success (70\%)
\end{block}

\begin{block}{Economic Viability}
\protect\phantomsection\label{economic-viability}
Grafting operations are highly economically viable, with break-even
success rates (17.5\%) well below typical performance (70-85\%). The
high value of successful grafts relative to costs creates strong
economic incentives for quality execution and optimal technique
selection.
\end{block}
\end{frame}

\begin{frame}{Practical Applications}
\protect\phantomsection\label{practical-applications}
\begin{block}{Commercial Operations}
\protect\phantomsection\label{commercial-operations}
The toolkit provides practical tools for commercial grafting operations:
- Compatibility prediction enables informed rootstock-scion selection -
Technique recommendations optimize success rates - Seasonal planning
identifies optimal timing windows - Economic analysis supports business
decision-making
\end{block}

\begin{block}{Research Applications}
\protect\phantomsection\label{research-applications}
The framework supports research in: - Rootstock breeding programs
through compatibility prediction - Climate adaptation through seasonal
planning algorithms - Technique development through simulation
capabilities - Biological understanding through mechanistic models
\end{block}

\begin{block}{Educational Use}
\protect\phantomsection\label{educational-use}
The comprehensive review and computational tools provide educational
resources for: - University courses in horticulture and arboriculture -
Extension programs for practitioners - Self-directed learning for
students - Professional development for industry workers
\end{block}
\end{frame}

\begin{frame}{Future Research Directions}
\protect\phantomsection\label{future-research-directions}
\begin{block}{Immediate Extensions}
\protect\phantomsection\label{immediate-extensions}
Several promising directions for immediate future work:

\begin{enumerate}
\tightlist
\item
  \textbf{Molecular Markers}: Integration of DNA, protein, and
  metabolite markers for improved compatibility prediction
\item
  \textbf{Long-term Studies}: Extension of models to predict long-term
  graft performance and compatibility
\item
  \textbf{Disease Interactions}: Incorporation of disease transmission
  and resistance factors
\item
  \textbf{Stress Responses}: Modeling of stress-induced incompatibility
  and recovery
\end{enumerate}
\end{block}

\begin{block}{Long-term Vision}
\protect\phantomsection\label{long-term-vision}
The foundation established here opens several long-term research
directions:

\begin{enumerate}
\tightlist
\item
  \textbf{Climate Adaptation}: Development of climate-adapted
  rootstock-scion combinations for changing conditions
\item
  \textbf{Novel Techniques}: Creation of new grafting methods for
  difficult species or challenging environments
\item
  \textbf{Machine Learning}: Integration of ML methods for improved
  prediction accuracy from large-scale data
\item
  \textbf{Global Database}: Development of comprehensive global
  compatibility database with community contributions
\end{enumerate}
\end{block}
\end{frame}

\begin{frame}{Broader Impact}
\protect\phantomsection\label{broader-impact}
\begin{block}{Food Security}
\protect\phantomsection\label{food-security}
Grafting contributes to global food security through efficient
production of high-quality fruits and nuts. The ability to optimize
operations through computational tools can improve productivity and
reduce waste, supporting food security goals in a changing climate.
\end{block}

\begin{block}{Conservation}
\protect\phantomsection\label{conservation}
Grafting enables conservation of rare or endangered species through
propagation when seed production is limited. The framework supports
these efforts by providing compatibility predictions and technique
recommendations for challenging species.
\end{block}

\begin{block}{Cultural Preservation}
\protect\phantomsection\label{cultural-preservation}
The integration of traditional knowledge with modern science preserves
4,000+ years of grafting heritage while making it accessible to
contemporary practitioners. This synthesis honors traditional practices
while advancing scientific understanding.
\end{block}
\end{frame}

\begin{frame}{Final Remarks}
\protect\phantomsection\label{final-remarks}
This work demonstrates that comprehensive synthesis of traditional
knowledge, biological understanding, and computational methods can yield
both theoretical insights and practical tools for tree grafting. The
integration of historical context, biological mechanisms, technical
methods, and economic analysis creates a holistic framework that serves
researchers, practitioners, and students.

The computational toolkit provides accessible tools for decision-making,
while the comprehensive review preserves and synthesizes knowledge
spanning millennia. As climate change, disease pressures, and food
security challenges intensify, the ability to optimize grafting
operations becomes increasingly valuable.

We believe this work represents a significant contribution to
horticultural science, providing both a comprehensive knowledge
synthesis and practical computational tools. The framework's success
across diverse applications---from commercial fruit production to
conservation efforts---demonstrates the broad utility of integrating
traditional knowledge with modern computational methods.

The future of grafting lies in continued integration of scientific
understanding with practical application, building on the foundation
established here to address emerging challenges in agriculture,
conservation, and food security. Through continued research,
development, and application, grafting will remain a vital tool for
humanity's relationship with trees and the ecosystems they support.
\end{frame}

\end{document}
