% Options for packages loaded elsewhere
\PassOptionsToPackage{unicode}{hyperref}
\PassOptionsToPackage{hyphens}{url}
\documentclass[
  ignorenonframetext,
]{beamer}
\newif\ifbibliography
\usepackage{pgfpages}
\setbeamertemplate{caption}[numbered]
\setbeamertemplate{caption label separator}{: }
\setbeamercolor{caption name}{fg=normal text.fg}
\beamertemplatenavigationsymbolsempty
% remove section numbering
\setbeamertemplate{part page}{
  \centering
  \begin{beamercolorbox}[sep=16pt,center]{part title}
    \usebeamerfont{part title}\insertpart\par
  \end{beamercolorbox}
}
\setbeamertemplate{section page}{
  \centering
  \begin{beamercolorbox}[sep=12pt,center]{section title}
    \usebeamerfont{section title}\insertsection\par
  \end{beamercolorbox}
}
\setbeamertemplate{subsection page}{
  \centering
  \begin{beamercolorbox}[sep=8pt,center]{subsection title}
    \usebeamerfont{subsection title}\insertsubsection\par
  \end{beamercolorbox}
}
% Prevent slide breaks in the middle of a paragraph
\widowpenalties 1 10000
\raggedbottom
\AtBeginPart{
  \frame{\partpage}
}
\AtBeginSection{
  \ifbibliography
  \else
    \frame{\sectionpage}
  \fi
}
\AtBeginSubsection{
  \frame{\subsectionpage}
}
\usepackage{iftex}
\ifPDFTeX
  \usepackage[T1]{fontenc}
  \usepackage[utf8]{inputenc}
  \usepackage{textcomp} % provide euro and other symbols
\else % if luatex or xetex
  \usepackage{unicode-math} % this also loads fontspec
  \defaultfontfeatures{Scale=MatchLowercase}
  \defaultfontfeatures[\rmfamily]{Ligatures=TeX,Scale=1}
\fi
\usepackage{lmodern}
\ifPDFTeX\else
  % xetex/luatex font selection
\fi
% Use upquote if available, for straight quotes in verbatim environments
\IfFileExists{upquote.sty}{\usepackage{upquote}}{}
\IfFileExists{microtype.sty}{% use microtype if available
  \usepackage[]{microtype}
  \UseMicrotypeSet[protrusion]{basicmath} % disable protrusion for tt fonts
}{}
\makeatletter
\@ifundefined{KOMAClassName}{% if non-KOMA class
  \IfFileExists{parskip.sty}{%
    \usepackage{parskip}
  }{% else
    \setlength{\parindent}{0pt}
    \setlength{\parskip}{6pt plus 2pt minus 1pt}}
}{% if KOMA class
  \KOMAoptions{parskip=half}}
\makeatother
\setlength{\emergencystretch}{3em} % prevent overfull lines
\providecommand{\tightlist}{%
  \setlength{\itemsep}{0pt}\setlength{\parskip}{0pt}}
\usepackage{bookmark}
\IfFileExists{xurl.sty}{\usepackage{xurl}}{} % add URL line breaks if available
\urlstyle{same}
\hypersetup{
  hidelinks,
  pdfcreator={LaTeX via pandoc}}

\author{\texorpdfstring{}{}}
\date{}

\begin{document}

\section{Conclusion}\label{conclusion}

\begin{frame}{Summary of Contributions}
\protect\phantomsection\label{summary-of-contributions}
This work establishes Containment Theory as a computationally verified
alternative foundation for Boolean reasoning. Our primary contributions
are:

\begin{block}{1. Rigorous Implementation}
\protect\phantomsection\label{rigorous-implementation}
We provide a complete computational framework implementing: -
\textbf{Form construction}: Void, mark, enclosure, and juxtaposition
operations - \textbf{Reduction engine}: Polynomial-time reduction to
canonical forms with step traces - \textbf{Theorem verification}:
Automated checking of all nine Spencer-Brown consequences -
\textbf{Boolean correspondence}: Verified isomorphism to Boolean algebra
- \textbf{Evaluation semantics}: Sound extraction of truth values
\end{block}

\begin{block}{2. Formal Verification}
\protect\phantomsection\label{formal-verification}
All theoretical claims are computationally verified: - Both axioms
(Calling and Crossing) demonstrated - Nine derived consequences (C1-C9)
verified by reduction - De Morgan's laws established - Boolean axiom set
confirmed - Consistency (non-contradiction) proven
\end{block}

\begin{block}{3. Complexity Analysis}
\protect\phantomsection\label{complexity-analysis}
We establish: - Termination guarantee for all well-formed inputs -
Polynomial-time complexity for typical forms - Confluence of reduction
sequences - Explicit complexity scaling analysis
\end{block}

\begin{block}{4. Comparative Analysis}
\protect\phantomsection\label{comparative-analysis}
The comparison with Set Theory reveals: - Radical axiomatic economy (2
axioms vs 9+) - Natural geometric interpretation - Constructive
treatment of self-reference - Direct circuit correspondence
\end{block}
\end{frame}

\begin{frame}{Key Findings}
\protect\phantomsection\label{key-findings}
\begin{block}{The Minimality of Distinction}
\protect\phantomsection\label{the-minimality-of-distinction}
The entire Boolean algebra emerges from a single cognitive primitive:
\textbf{making a distinction}. This suggests that: - Logic is
fundamentally spatial - Boolean reasoning requires minimal axiomatic
commitment - Complexity in formal systems may be reducible
\end{block}

\begin{block}{Self-Reference as Dynamics}
\protect\phantomsection\label{self-reference-as-dynamics}
Rather than generating paradoxes, self-referential forms in boundary
logic produce \textbf{temporal oscillation}. The imaginary value
\(j = \langle j \rangle\) is not contradictory but dynamic---suggesting
that self-reference naturally leads to process rather than paradox.
\end{block}

\begin{block}{Geometric Foundations}
\protect\phantomsection\label{geometric-foundations}
Boundary logic's success demonstrates that geometric intuition can serve
as mathematical foundation. The mark creates inside/outside; enclosure
creates negation; juxtaposition creates conjunction. These spatial
operations suffice for propositional completeness.
\end{block}
\end{frame}

\begin{frame}{Implications}
\protect\phantomsection\label{implications}
\begin{block}{For Foundations of Mathematics}
\protect\phantomsection\label{for-foundations-of-mathematics}
Containment Theory demonstrates that alternative foundations exist with
different trade-offs: - \textbf{Set Theory}: Power and generality at
cost of axiom complexity - \textbf{Boundary Logic}: Minimality and
intuition for finite structures

Neither replaces the other; they serve different purposes.
\end{block}

\begin{block}{For Computer Science}
\protect\phantomsection\label{for-computer-science}
Digital logic gains: - Direct correspondence between forms and circuits
- Reduction-based optimization potential - Geometric visualization of
Boolean functions
\end{block}

\begin{block}{For Cognitive Science}
\protect\phantomsection\label{for-cognitive-science}
The calculus provides formal tools for studying
\cite{varela1991,thompson2007,friston2010}: - Distinction as primitive
cognitive act - Negation as boundary crossing - Self-reference as
oscillation - Attention as juxtaposition
\end{block}
\end{frame}

\begin{frame}{Future Work}
\protect\phantomsection\label{future-work}
\begin{block}{Immediate Extensions}
\protect\phantomsection\label{immediate-extensions}
\begin{enumerate}
\tightlist
\item
  \textbf{Variable quantification}: Extending to predicate logic
\item
  \textbf{Arithmetic integration}: Incorporating Bricken's iconic
  arithmetic
\item
  \textbf{Imaginary value computation}: Full treatment of
  self-referential dynamics
\end{enumerate}
\end{block}

\begin{block}{Long-term Research}
\protect\phantomsection\label{long-term-research}
\begin{enumerate}
\tightlist
\item
  \textbf{Category-theoretic formalization}: Forms as a category with
  natural transformations
\item
  \textbf{Quantum boundary logic}: Superposition in boundary notation
\item
  \textbf{Neural boundary networks}: Boundary-based machine learning
  architectures
\end{enumerate}
\end{block}

\begin{block}{Open Questions}
\protect\phantomsection\label{open-questions}
\begin{enumerate}
\tightlist
\item
  \textbf{Is the consequence system complete?} Do C1-C9 generate all
  Boolean identities?
\item
  \textbf{What are tight complexity bounds?} Optimal reduction
  algorithms
\item
  \textbf{Can boundary logic scale to practical circuits?} Industrial
  applicability
\end{enumerate}
\end{block}
\end{frame}

\begin{frame}[fragile]{Reproducibility}
\protect\phantomsection\label{reproducibility}
All results are reproducible: - Complete source code:
\texttt{project/src/} - Test suite: \texttt{project/tests/} - Scripts:
\texttt{python3\ scripts/02\_run\_analysis.py} - Documentation: This
manuscript and \texttt{AGENTS.md}

The implementation uses only standard Python libraries with no external
dependencies beyond numpy and matplotlib for visualization.
\end{frame}

\begin{frame}{Closing Remarks}
\protect\phantomsection\label{closing-remarks}
G. Spencer-Brown opened \emph{Laws of Form} with:

\begin{quote}
``A universe comes into being when a space is severed or taken apart.''
\end{quote}

Our computational verification confirms that this simple act---making a
distinction---suffices to generate the complete Boolean algebra. The
boundary is both primitive and powerful, creating structure from void
through the minimal commitment of two axioms.

Containment Theory stands as a testament to mathematical minimalism:
that complexity often arises from simplicity, and that the foundations
of logic may be more spatial than symbolic.
\end{frame}

\begin{frame}
\emph{``We take as given the idea of distinction and the idea of
indication, and that we cannot make an indication without drawing a
distinction.''}

--- G. Spencer-Brown, \emph{Laws of Form} (1969)
\end{frame}

\end{document}
