% Options for packages loaded elsewhere
\PassOptionsToPackage{unicode}{hyperref}
\PassOptionsToPackage{hyphens}{url}
\documentclass[
  ignorenonframetext,
]{beamer}
\newif\ifbibliography
\usepackage{pgfpages}
\setbeamertemplate{caption}[numbered]
\setbeamertemplate{caption label separator}{: }
\setbeamercolor{caption name}{fg=normal text.fg}
\beamertemplatenavigationsymbolsempty
% remove section numbering
\setbeamertemplate{part page}{
  \centering
  \begin{beamercolorbox}[sep=16pt,center]{part title}
    \usebeamerfont{part title}\insertpart\par
  \end{beamercolorbox}
}
\setbeamertemplate{section page}{
  \centering
  \begin{beamercolorbox}[sep=12pt,center]{section title}
    \usebeamerfont{section title}\insertsection\par
  \end{beamercolorbox}
}
\setbeamertemplate{subsection page}{
  \centering
  \begin{beamercolorbox}[sep=8pt,center]{subsection title}
    \usebeamerfont{subsection title}\insertsubsection\par
  \end{beamercolorbox}
}
% Prevent slide breaks in the middle of a paragraph
\widowpenalties 1 10000
\raggedbottom
\AtBeginPart{
  \frame{\partpage}
}
\AtBeginSection{
  \ifbibliography
  \else
    \frame{\sectionpage}
  \fi
}
\AtBeginSubsection{
  \frame{\subsectionpage}
}
\usepackage{iftex}
\ifPDFTeX
  \usepackage[T1]{fontenc}
  \usepackage[utf8]{inputenc}
  \usepackage{textcomp} % provide euro and other symbols
\else % if luatex or xetex
  \usepackage{unicode-math} % this also loads fontspec
  \defaultfontfeatures{Scale=MatchLowercase}
  \defaultfontfeatures[\rmfamily]{Ligatures=TeX,Scale=1}
\fi
\usepackage{lmodern}
\ifPDFTeX\else
  % xetex/luatex font selection
\fi
% Use upquote if available, for straight quotes in verbatim environments
\IfFileExists{upquote.sty}{\usepackage{upquote}}{}
\IfFileExists{microtype.sty}{% use microtype if available
  \usepackage[]{microtype}
  \UseMicrotypeSet[protrusion]{basicmath} % disable protrusion for tt fonts
}{}
\makeatletter
\@ifundefined{KOMAClassName}{% if non-KOMA class
  \IfFileExists{parskip.sty}{%
    \usepackage{parskip}
  }{% else
    \setlength{\parindent}{0pt}
    \setlength{\parskip}{6pt plus 2pt minus 1pt}}
}{% if KOMA class
  \KOMAoptions{parskip=half}}
\makeatother
\setlength{\emergencystretch}{3em} % prevent overfull lines
\providecommand{\tightlist}{%
  \setlength{\itemsep}{0pt}\setlength{\parskip}{0pt}}
\usepackage{bookmark}
\IfFileExists{xurl.sty}{\usepackage{xurl}}{} % add URL line breaks if available
\urlstyle{same}
\hypersetup{
  hidelinks,
  pdfcreator={LaTeX via pandoc}}

\author{\texorpdfstring{}{}}
\date{}

\begin{document}

\section{Conclusion}\label{sec:conclusion}

\begin{frame}{Summary of Contributions}
\protect\phantomsection\label{summary-of-contributions}
This research presents a comprehensive systematic analysis of Andrius
Kulikauskas's ``Ways of Figuring Things Out'' framework, documenting and
analyzing 210 ways from the database (with connections to the broader
framework of 284 ways documented in the source text). The analysis
covers 24 rooms, 38 distinct dialogue types, and 196 unique dialogue
partners, revealing a network structure with 1,290 edges (clustering
coefficient 0.886) connecting ways through shared characteristics. The
work makes several key contributions:

\begin{block}{Documentation and Categorization}
\protect\phantomsection\label{documentation-and-categorization}
\begin{enumerate}
\tightlist
\item
  \textbf{Complete Documentation}: Systematic documentation of 210 ways
  from the database with complete metadata including dialogue types,
  room assignments, examples, and relationships
\item
  \textbf{24-Room Framework}: Organization of ways within the House of
  Knowledge structure, mapping ways to their appropriate rooms
\item
  \textbf{Dialogue Type Classification}: Categorization of ways
  according to Absolute, Relative, and Embrace God dialogue types
\item
  \textbf{Relationship Mapping}: Documentation of how ways relate
  through dialogue partners, shared rooms, and question relationships
\end{enumerate}
\end{block}

\begin{block}{Empirical Analysis}
\protect\phantomsection\label{empirical-analysis}
\begin{enumerate}
\tightlist
\item
  \textbf{Distribution Analysis}: Quantitative analysis of way
  distributions across dialogue types, rooms, and categories
\item
  \textbf{Network Analysis}: Graph-based analysis revealing the network
  structure of way relationships
\item
  \textbf{Statistical Patterns}: Identification of patterns in room
  co-occurrence, dialogue type distributions, and central ways
\item
  \textbf{Cross-Tabulation}: Analysis of relationships between different
  dimensions of the framework
\end{enumerate}
\end{block}

\begin{block}{Framework Understanding}
\protect\phantomsection\label{framework-understanding}
\begin{enumerate}
\tightlist
\item
  \textbf{Structural Insights}: Understanding of how the 24-room House
  of Knowledge organizes different aspects of knowledge
\item
  \textbf{Philosophical Integration}: Recognition of how Believing,
  Caring, and Relative Learning structures integrate
\item
  \textbf{Epistemological Pluralism}: Demonstration of multiple valid
  approaches to knowledge
\item
  \textbf{Practical Applications}: Tools and frameworks for applying
  ways in education, research, and personal development
\end{enumerate}
\end{block}
\end{frame}

\begin{frame}{Key Findings}
\protect\phantomsection\label{key-findings}
\begin{block}{Framework Structure}
\protect\phantomsection\label{framework-structure}
The analysis reveals that the Ways framework is not uniform but exhibits
structured patterns: - Ways cluster within certain rooms: B2 (23 ways,
11.0\%), C4 (17 ways, 8.1\%), R (16 ways, 7.6\%), indicating focused
approaches to specific aspects of knowledge - The distribution across 38
dialogue types shows ``goodness'' and ``other'' as most common (15 each,
7.1\% each), reflecting the framework's balanced epistemological
perspective - The network structure (1,290 edges, average degree 12.29,
clustering coefficient 0.886) shows both high local clustering
(room-based) and long-range connections (type and partner-based),
creating a rich, interconnected system with small-world properties
\end{block}

\begin{block}{Central Ways}
\protect\phantomsection\label{central-ways}
Certain ways serve as central nodes in the network, connecting different
parts of the framework. These central ways likely represent: -
Fundamental approaches that bridge different categories - Entry points
to the framework for new learners - Methods that integrate multiple
aspects of knowledge
\end{block}

\begin{block}{Room Relationships}
\protect\phantomsection\label{room-relationships}
Analysis reveals relationships between rooms, showing how different
aspects of knowledge relate: - Some room pairs frequently co-occur,
indicating complementary approaches - The three fundamental structures
(Believing, Caring, Learning) provide organization - The 24-room
structure provides comprehensive coverage of knowledge aspects
\end{block}
\end{frame}

\begin{frame}{Broader Impact}
\protect\phantomsection\label{broader-impact}
\begin{block}{Contribution to Epistemology}
\protect\phantomsection\label{contribution-to-epistemology}
This work contributes to epistemology by:

\begin{itemize}
\tightlist
\item
  Providing a comprehensive catalog of ways of knowing
\item
  Demonstrating the validity of multiple epistemological approaches
\item
  Showing how different ways relate and complement each other
\item
  Integrating belief, care, and learning in knowledge acquisition
\end{itemize}
\end{block}

\begin{block}{Contribution to Education}
\protect\phantomsection\label{contribution-to-education}
The framework contributes to education by:

\begin{itemize}
\tightlist
\item
  Providing a systematic approach to understanding learning
\item
  Recognizing the validity of multiple learning approaches
\item
  Offering structure for curriculum and teaching methods
\item
  Supporting personalized and adaptive education
\end{itemize}
\end{block}

\begin{block}{Contribution to Research}
\protect\phantomsection\label{contribution-to-research}
For researchers, the framework provides:

\begin{itemize}
\tightlist
\item
  A systematic approach to method selection
\item
  Understanding of how different methods relate
\item
  Epistemological awareness in research design
\item
  Support for interdisciplinary research
\end{itemize}
\end{block}
\end{frame}

\begin{frame}{Practical Applications}
\protect\phantomsection\label{practical-applications}
\begin{block}{Educational Tools}
\protect\phantomsection\label{educational-tools}
The framework enables:

\begin{itemize}
\tightlist
\item
  Recognition of different learning styles and approaches
\item
  Adaptation of teaching methods to match different ways
\item
  Curriculum design that exposes students to multiple ways
\item
  Assessment methods appropriate for different ways
\end{itemize}
\end{block}

\begin{block}{Research Methodology}
\protect\phantomsection\label{research-methodology}
Researchers can use the framework for:

\begin{itemize}
\tightlist
\item
  Systematic selection of appropriate research methods
\item
  Understanding how methods complement each other
\item
  Epistemological awareness in research design
\item
  Interdisciplinary bridge-building
\end{itemize}
\end{block}

\begin{block}{Personal Development}
\protect\phantomsection\label{personal-development}
Individuals can use the framework for:

\begin{itemize}
\tightlist
\item
  Understanding their own preferred ways of figuring things out
\item
  Learning new ways to expand capabilities
\item
  Recognizing which ways are appropriate for which situations
\item
  Developing the ability to use multiple ways as needed
\end{itemize}
\end{block}
\end{frame}

\begin{frame}{Future Directions}
\protect\phantomsection\label{future-directions}
\begin{block}{Framework Expansion}
\protect\phantomsection\label{framework-expansion}
Future research can: 1. Document additional ways beyond the current 284
2. Explore ways from other philosophical traditions 3. Investigate ways
in specific domains (science, art, humanities) 4. Develop ways for
emerging contexts (digital, global, interdisciplinary)
\end{block}

\begin{block}{Empirical Validation}
\protect\phantomsection\label{empirical-validation}
Empirical research can: 1. Test the effectiveness of different ways in
different contexts 2. Investigate individual differences in way
preferences 3. Study how ways develop and change over time 4. Examine
relationships between ways and learning outcomes
\end{block}

\begin{block}{Computational Applications}
\protect\phantomsection\label{computational-applications}
Computational research can: 1. Develop AI systems that use different
ways 2. Create recommendation systems for way selection 3. Build tools
for way analysis and visualization 4. Develop educational software based
on the framework
\end{block}

\begin{block}{Interdisciplinary Integration}
\protect\phantomsection\label{interdisciplinary-integration}
The framework can be integrated with: 1. Cognitive science research on
learning and knowledge 2. Educational research on teaching methods and
curriculum 3. Philosophy of science and epistemology 4. Knowledge
management and organizational learning
\end{block}
\end{frame}

\begin{frame}[fragile]{Methodological Contributions}
\protect\phantomsection\label{methodological-contributions}
\begin{block}{Database-Driven Analysis}
\protect\phantomsection\label{database-driven-analysis}
This work demonstrates:

\begin{itemize}
\tightlist
\item
  How philosophical frameworks can be systematically documented in
  databases
\item
  The value of quantitative analysis for understanding qualitative
  frameworks
\item
  How network analysis reveals structure in knowledge systems
\item
  The integration of database analysis with text analysis
\end{itemize}
\end{block}

\begin{block}{Visualization Approaches}
\protect\phantomsection\label{visualization-approaches}
The visualization work shows:

\begin{itemize}
\tightlist
\item
  How network graphs reveal structure in way relationships
\item
  How hierarchical visualizations illustrate the House of Knowledge
\item
  How statistical plots communicate distribution patterns
\item
  How multiple visualization types complement each other
\end{itemize}
\end{block}

\begin{block}{Integration of Quantitative and Qualitative}
\protect\phantomsection\label{integration-of-quantitative-and-qualitative}
The work demonstrates:

\begin{itemize}
\tightlist
\item
  How quantitative analysis complements qualitative understanding
\item
  The value of systematic documentation for philosophical frameworks
\item
  How data-driven insights enhance philosophical interpretation
\item
  The integration of empirical analysis with philosophical analysis
\end{itemize}
\end{block}

\begin{block}{Implementation Modules}
\protect\phantomsection\label{implementation-modules}
The research implements a comprehensive software framework for ways
analysis:

\textbf{Database Layer}: \texttt{database.py}, \texttt{sql\_queries.py},
\texttt{models.py} - ORM models and query interfaces \textbf{Analysis
Layer}: \texttt{ways\_analysis.py}, \texttt{network\_analysis.py},
\texttt{house\_of\_knowledge.py} - Specialized analysis modules
\textbf{Statistics Layer}: \texttt{statistics.py}, \texttt{metrics.py} -
Quantitative analysis functions \textbf{Supporting Modules}: Data
processing, visualization, and reporting utilities

All modules follow the thin orchestrator pattern with business logic in
\texttt{src/} and orchestration in \texttt{scripts/}.
\end{block}
\end{frame}

\begin{frame}{Final Remarks}
\protect\phantomsection\label{final-remarks}
This research provides both a philosophical framework and a practical
system for understanding and applying diverse ways of figuring things
out. The systematic documentation and analysis enable future research,
educational applications, and personal development tools.

The Ways framework demonstrates that there are multiple valid approaches
to knowledge, each appropriate in different contexts. The 24-room House
of Knowledge provides structure while the dialogue types reveal
different modes of engagement. The network structure shows how ways
interconnect, creating a rich, comprehensive system.

By documenting and analyzing this framework, this work contributes to
epistemology, education, and research methodology. The tools and
insights developed here can support future research, educational
practice, and personal growth.

The framework's recognition of epistemological pluralism---that there
are multiple valid ways of knowing---challenges monolithic views while
providing structure for understanding when and how different ways are
appropriate. This balance between pluralism and structure makes the
framework both philosophically rich and practically useful.

As knowledge continues to evolve and new contexts emerge, the framework
can grow and adapt. Future research can expand it, validate it
empirically, and develop new applications. This work provides the
foundation for that future development.

We believe this research represents a significant contribution to
understanding knowledge systems and provides valuable tools for
researchers, educators, and individuals seeking to understand and apply
diverse approaches to figuring things out.
\end{frame}

\end{document}
