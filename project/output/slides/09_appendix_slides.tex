% Options for packages loaded elsewhere
\PassOptionsToPackage{unicode}{hyperref}
\PassOptionsToPackage{hyphens}{url}
\documentclass[
  ignorenonframetext,
]{beamer}
\newif\ifbibliography
\usepackage{pgfpages}
\setbeamertemplate{caption}[numbered]
\setbeamertemplate{caption label separator}{: }
\setbeamercolor{caption name}{fg=normal text.fg}
\beamertemplatenavigationsymbolsempty
% remove section numbering
\setbeamertemplate{part page}{
  \centering
  \begin{beamercolorbox}[sep=16pt,center]{part title}
    \usebeamerfont{part title}\insertpart\par
  \end{beamercolorbox}
}
\setbeamertemplate{section page}{
  \centering
  \begin{beamercolorbox}[sep=12pt,center]{section title}
    \usebeamerfont{section title}\insertsection\par
  \end{beamercolorbox}
}
\setbeamertemplate{subsection page}{
  \centering
  \begin{beamercolorbox}[sep=8pt,center]{subsection title}
    \usebeamerfont{subsection title}\insertsubsection\par
  \end{beamercolorbox}
}
% Prevent slide breaks in the middle of a paragraph
\widowpenalties 1 10000
\raggedbottom
\AtBeginPart{
  \frame{\partpage}
}
\AtBeginSection{
  \ifbibliography
  \else
    \frame{\sectionpage}
  \fi
}
\AtBeginSubsection{
  \frame{\subsectionpage}
}
\usepackage{iftex}
\ifPDFTeX
  \usepackage[T1]{fontenc}
  \usepackage[utf8]{inputenc}
  \usepackage{textcomp} % provide euro and other symbols
\else % if luatex or xetex
  \usepackage{unicode-math} % this also loads fontspec
  \defaultfontfeatures{Scale=MatchLowercase}
  \defaultfontfeatures[\rmfamily]{Ligatures=TeX,Scale=1}
\fi
\usepackage{lmodern}
\ifPDFTeX\else
  % xetex/luatex font selection
\fi
% Use upquote if available, for straight quotes in verbatim environments
\IfFileExists{upquote.sty}{\usepackage{upquote}}{}
\IfFileExists{microtype.sty}{% use microtype if available
  \usepackage[]{microtype}
  \UseMicrotypeSet[protrusion]{basicmath} % disable protrusion for tt fonts
}{}
\makeatletter
\@ifundefined{KOMAClassName}{% if non-KOMA class
  \IfFileExists{parskip.sty}{%
    \usepackage{parskip}
  }{% else
    \setlength{\parindent}{0pt}
    \setlength{\parskip}{6pt plus 2pt minus 1pt}}
}{% if KOMA class
  \KOMAoptions{parskip=half}}
\makeatother
\usepackage{color}
\usepackage{fancyvrb}
\newcommand{\VerbBar}{|}
\newcommand{\VERB}{\Verb[commandchars=\\\{\}]}
\DefineVerbatimEnvironment{Highlighting}{Verbatim}{commandchars=\\\{\}}
% Add ',fontsize=\small' for more characters per line
\newenvironment{Shaded}{}{}
\newcommand{\AlertTok}[1]{\textcolor[rgb]{1.00,0.00,0.00}{\textbf{#1}}}
\newcommand{\AnnotationTok}[1]{\textcolor[rgb]{0.38,0.63,0.69}{\textbf{\textit{#1}}}}
\newcommand{\AttributeTok}[1]{\textcolor[rgb]{0.49,0.56,0.16}{#1}}
\newcommand{\BaseNTok}[1]{\textcolor[rgb]{0.25,0.63,0.44}{#1}}
\newcommand{\BuiltInTok}[1]{\textcolor[rgb]{0.00,0.50,0.00}{#1}}
\newcommand{\CharTok}[1]{\textcolor[rgb]{0.25,0.44,0.63}{#1}}
\newcommand{\CommentTok}[1]{\textcolor[rgb]{0.38,0.63,0.69}{\textit{#1}}}
\newcommand{\CommentVarTok}[1]{\textcolor[rgb]{0.38,0.63,0.69}{\textbf{\textit{#1}}}}
\newcommand{\ConstantTok}[1]{\textcolor[rgb]{0.53,0.00,0.00}{#1}}
\newcommand{\ControlFlowTok}[1]{\textcolor[rgb]{0.00,0.44,0.13}{\textbf{#1}}}
\newcommand{\DataTypeTok}[1]{\textcolor[rgb]{0.56,0.13,0.00}{#1}}
\newcommand{\DecValTok}[1]{\textcolor[rgb]{0.25,0.63,0.44}{#1}}
\newcommand{\DocumentationTok}[1]{\textcolor[rgb]{0.73,0.13,0.13}{\textit{#1}}}
\newcommand{\ErrorTok}[1]{\textcolor[rgb]{1.00,0.00,0.00}{\textbf{#1}}}
\newcommand{\ExtensionTok}[1]{#1}
\newcommand{\FloatTok}[1]{\textcolor[rgb]{0.25,0.63,0.44}{#1}}
\newcommand{\FunctionTok}[1]{\textcolor[rgb]{0.02,0.16,0.49}{#1}}
\newcommand{\ImportTok}[1]{\textcolor[rgb]{0.00,0.50,0.00}{\textbf{#1}}}
\newcommand{\InformationTok}[1]{\textcolor[rgb]{0.38,0.63,0.69}{\textbf{\textit{#1}}}}
\newcommand{\KeywordTok}[1]{\textcolor[rgb]{0.00,0.44,0.13}{\textbf{#1}}}
\newcommand{\NormalTok}[1]{#1}
\newcommand{\OperatorTok}[1]{\textcolor[rgb]{0.40,0.40,0.40}{#1}}
\newcommand{\OtherTok}[1]{\textcolor[rgb]{0.00,0.44,0.13}{#1}}
\newcommand{\PreprocessorTok}[1]{\textcolor[rgb]{0.74,0.48,0.00}{#1}}
\newcommand{\RegionMarkerTok}[1]{#1}
\newcommand{\SpecialCharTok}[1]{\textcolor[rgb]{0.25,0.44,0.63}{#1}}
\newcommand{\SpecialStringTok}[1]{\textcolor[rgb]{0.73,0.40,0.53}{#1}}
\newcommand{\StringTok}[1]{\textcolor[rgb]{0.25,0.44,0.63}{#1}}
\newcommand{\VariableTok}[1]{\textcolor[rgb]{0.10,0.09,0.49}{#1}}
\newcommand{\VerbatimStringTok}[1]{\textcolor[rgb]{0.25,0.44,0.63}{#1}}
\newcommand{\WarningTok}[1]{\textcolor[rgb]{0.38,0.63,0.69}{\textbf{\textit{#1}}}}
\setlength{\emergencystretch}{3em} % prevent overfull lines
\providecommand{\tightlist}{%
  \setlength{\itemsep}{0pt}\setlength{\parskip}{0pt}}
\usepackage{bookmark}
\IfFileExists{xurl.sty}{\usepackage{xurl}}{} % add URL line breaks if available
\urlstyle{same}
\hypersetup{
  hidelinks,
  pdfcreator={LaTeX via pandoc}}

\author{\texorpdfstring{}{}}
\date{}

\begin{document}

\begin{frame}[fragile]{Appendix}
\protect\phantomsection\label{sec:appendix}
This appendix provides additional technical details and derivations that
support the main results.

\begin{block}{A. Detailed Proofs}
\protect\phantomsection\label{a.-detailed-proofs}
\begin{block}{A.1 Proof of Convergence (Theorem 1)}
\protect\phantomsection\label{a.1-proof-of-convergence-theorem-1}
The convergence rate established in \eqref{eq:convergence} follows from
the following detailed analysis.

\textbf{Proof}: Let \(x_k\) be the iterate at step \(k\). From the
update rule \eqref{eq:update}, we have:

\begin{equation}\label{eq:appendix_update}
x_{k+1} = x_k - \alpha_k \nabla f(x_k) + \beta_k (x_k - x_{k-1})
\end{equation}

By the Lipschitz continuity of \(\nabla f\), there exists a constant
\(L > 0\) such that:

\begin{equation}\label{eq:lipschitz}
\|\nabla f(x) - \nabla f(y)\| \leq L \|x - y\|, \quad \forall x, y \in \mathcal{X}
\end{equation}

Using strong convexity with parameter \(\mu > 0\)
\cite{boyd2004, nesterov2018}:

\begin{equation}\label{eq:strong_convexity}
f(y) \geq f(x) + \nabla f(x)^T (y - x) + \frac{\mu}{2} \|y - x\|^2
\end{equation}

Combining these properties with the adaptive step size rule
\eqref{eq:adaptive_step}, following the analysis framework in
\cite{duchi2011, bertsekas2015}, we obtain the linear convergence rate
with \(\rho = \sqrt{1 - \mu/L}\). \(\square\)
\end{block}

\begin{block}{A.2 Complexity Analysis}
\protect\phantomsection\label{a.2-complexity-analysis}
The computational complexity per iteration is derived as follows:

\begin{enumerate}
\tightlist
\item
  \textbf{Gradient computation}: \(O(n)\) for dense problems, \(O(k)\)
  for sparse problems with \(k\) non-zeros
\item
  \textbf{Update rule}: \(O(n)\) for vector operations
\item
  \textbf{Adaptive step size}: \(O(1)\) for the update in
  \eqref{eq:adaptive_step}
\item
  \textbf{Momentum term}: \(O(n)\) for the momentum computation
\end{enumerate}

Total per-iteration complexity: \(O(n)\) for dense problems.

For structured problems, we can exploit the separable structure of
\eqref{eq:objective} to achieve \(O(n \log n)\) complexity using
efficient data structures (see Figure \ref{fig:data_structure}).
\end{block}
\end{block}

\begin{block}{B. Additional Experimental Details}
\protect\phantomsection\label{b.-additional-experimental-details}
\begin{block}{B.1 Hyperparameter Tuning}
\protect\phantomsection\label{b.1-hyperparameter-tuning}
The following hyperparameters were used in our experiments:

\begin{table}[h]
\centering
\begin{tabular}{|l|c|c|c|}
\hline
\textbf{Parameter} & \textbf{Symbol} & \textbf{Value} & \textbf{Range Tested} \\
\hline
Learning rate & $\alpha_0$ & 0.01 & [0.001, 0.1] \\
Momentum & $\beta$ & 0.9 & [0.5, 0.99] \\
Regularization & $\lambda$ & 0.001 & [0, 0.01] \\
Tolerance & $\epsilon$ & $10^{-6}$ & [$10^{-8}$, $10^{-4}$] \\
\hline
\end{tabular}
\caption{Hyperparameter settings used in experiments}
\label{tab:hyperparameters}
\end{table}
\end{block}

\begin{block}{B.2 Computational Environment}
\protect\phantomsection\label{b.2-computational-environment}
All experiments were conducted on: - \textbf{CPU}: Intel Xeon E5-2690 v4
@ 2.60GHz (28 cores) - \textbf{RAM}: 128GB DDR4 - \textbf{GPU}: NVIDIA
Tesla V100 (32GB VRAM) for large-scale experiments - \textbf{OS}: Ubuntu
20.04 LTS - \textbf{Python}: 3.10.12 - \textbf{NumPy}: 1.24.3 -
\textbf{SciPy}: 1.10.1
\end{block}

\begin{block}{B.3 Dataset Preparation}
\protect\phantomsection\label{b.3-dataset-preparation}
Datasets were preprocessed using standard normalization:

\begin{equation}\label{eq:normalization}
\tilde{x}_i = \frac{x_i - \mu}{\sigma}
\end{equation}

where \(\mu\) and \(\sigma\) are the mean and standard deviation
computed from the training set.
\end{block}
\end{block}

\begin{block}{C. Extended Results}
\protect\phantomsection\label{c.-extended-results}
\begin{block}{C.1 Additional Benchmark Comparisons}
\protect\phantomsection\label{c.1-additional-benchmark-comparisons}
Table \ref{tab:extended_comparison} provides detailed performance
comparison across all tested methods.

\begin{table}[h]
\centering
\begin{tabular}{|l|c|c|c|c|}
\hline
\textbf{Method} & \textbf{Time (s)} & \textbf{Iterations} & \textbf{Final Error} & \textbf{Memory (MB)} \\
\hline
Our Method & 12.3 & 245 & $1.2 \times 10^{-6}$ & 156 \\
Gradient Descent & 18.7 & 412 & $1.5 \times 10^{-6}$ & 312 \\
Adam & 15.4 & 358 & $1.4 \times 10^{-6}$ & 298 \\
L-BFGS & 16.2 & 198 & $1.1 \times 10^{-6}$ & 425 \\
\hline
\end{tabular}
\caption{Extended performance comparison with computational details}
\label{tab:extended_comparison}
\end{table}
\end{block}

\begin{block}{C.2 Sensitivity Analysis}
\protect\phantomsection\label{c.2-sensitivity-analysis}
Detailed sensitivity analysis for all hyperparameters shows robust
performance across wide parameter ranges, confirming the theoretical
predictions from Section \ref{sec:methodology}.
\end{block}
\end{block}

\begin{block}{D. Implementation Details}
\protect\phantomsection\label{d.-implementation-details}
\begin{block}{D.1 Pseudocode}
\protect\phantomsection\label{d.1-pseudocode}
\begin{Shaded}
\begin{Highlighting}[]
\KeywordTok{def}\NormalTok{ optimize(f, x0, alpha0, beta, max\_iter, tol):}
    \CommentTok{"""}
\CommentTok{    Optimization algorithm implementation.}
\CommentTok{    }
\CommentTok{    Args:}
\CommentTok{        f: Objective function}
\CommentTok{        x0: Initial point}
\CommentTok{        alpha0: Initial learning rate}
\CommentTok{        beta: Momentum coefficient}
\CommentTok{        max\_iter: Maximum iterations}
\CommentTok{        tol: Convergence tolerance}
\CommentTok{    }
\CommentTok{    Returns:}
\CommentTok{        x\_opt: Optimal solution}
\CommentTok{        history: Convergence history}
\CommentTok{    """}
\NormalTok{    x }\OperatorTok{=}\NormalTok{ x0}
\NormalTok{    x\_prev }\OperatorTok{=}\NormalTok{ x0}
\NormalTok{    history }\OperatorTok{=}\NormalTok{ []}
\NormalTok{    grad\_sum\_sq }\OperatorTok{=} \DecValTok{0}
    
    \ControlFlowTok{for}\NormalTok{ k }\KeywordTok{in} \BuiltInTok{range}\NormalTok{(max\_iter):}
        \CommentTok{\# Compute gradient}
\NormalTok{        grad }\OperatorTok{=}\NormalTok{ compute\_gradient(f, x)}
\NormalTok{        grad\_sum\_sq }\OperatorTok{+=}\NormalTok{ np.linalg.norm(grad)}\OperatorTok{**}\DecValTok{2}
        
        \CommentTok{\# Adaptive step size}
\NormalTok{        alpha }\OperatorTok{=}\NormalTok{ alpha0 }\OperatorTok{/}\NormalTok{ np.sqrt(}\DecValTok{1} \OperatorTok{+}\NormalTok{ grad\_sum\_sq)}
        
        \CommentTok{\# Update with momentum}
\NormalTok{        x\_new }\OperatorTok{=}\NormalTok{ x }\OperatorTok{{-}}\NormalTok{ alpha }\OperatorTok{*}\NormalTok{ grad }\OperatorTok{+}\NormalTok{ beta }\OperatorTok{*}\NormalTok{ (x }\OperatorTok{{-}}\NormalTok{ x\_prev)}
        
        \CommentTok{\# Check convergence}
        \ControlFlowTok{if}\NormalTok{ np.linalg.norm(x\_new }\OperatorTok{{-}}\NormalTok{ x) }\OperatorTok{\textless{}}\NormalTok{ tol:}
            \ControlFlowTok{break}
        
        \CommentTok{\# Update history}
\NormalTok{        history.append(\{}\StringTok{\textquotesingle{}iter\textquotesingle{}}\NormalTok{: k, }\StringTok{\textquotesingle{}error\textquotesingle{}}\NormalTok{: f(x\_new)\})}
        
        \CommentTok{\# Prepare next iteration}
\NormalTok{        x\_prev }\OperatorTok{=}\NormalTok{ x}
\NormalTok{        x }\OperatorTok{=}\NormalTok{ x\_new}
    
    \ControlFlowTok{return}\NormalTok{ x, history}
\end{Highlighting}
\end{Shaded}
\end{block}

\begin{block}{D.2 Performance Optimizations}
\protect\phantomsection\label{d.2-performance-optimizations}
Key performance optimizations implemented: 1. Vectorized operations
using NumPy 2. Sparse matrix representations when applicable 3. In-place
updates to reduce memory allocation 4. Parallel gradient computations
for separable problems
\end{block}
\end{block}
\end{frame}

\end{document}
