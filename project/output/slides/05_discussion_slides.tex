% Options for packages loaded elsewhere
\PassOptionsToPackage{unicode}{hyperref}
\PassOptionsToPackage{hyphens}{url}
\documentclass[
  ignorenonframetext,
]{beamer}
\newif\ifbibliography
\usepackage{pgfpages}
\setbeamertemplate{caption}[numbered]
\setbeamertemplate{caption label separator}{: }
\setbeamercolor{caption name}{fg=normal text.fg}
\beamertemplatenavigationsymbolsempty
% remove section numbering
\setbeamertemplate{part page}{
  \centering
  \begin{beamercolorbox}[sep=16pt,center]{part title}
    \usebeamerfont{part title}\insertpart\par
  \end{beamercolorbox}
}
\setbeamertemplate{section page}{
  \centering
  \begin{beamercolorbox}[sep=12pt,center]{section title}
    \usebeamerfont{section title}\insertsection\par
  \end{beamercolorbox}
}
\setbeamertemplate{subsection page}{
  \centering
  \begin{beamercolorbox}[sep=8pt,center]{subsection title}
    \usebeamerfont{subsection title}\insertsubsection\par
  \end{beamercolorbox}
}
% Prevent slide breaks in the middle of a paragraph
\widowpenalties 1 10000
\raggedbottom
\AtBeginPart{
  \frame{\partpage}
}
\AtBeginSection{
  \ifbibliography
  \else
    \frame{\sectionpage}
  \fi
}
\AtBeginSubsection{
  \frame{\subsectionpage}
}
\usepackage{iftex}
\ifPDFTeX
  \usepackage[T1]{fontenc}
  \usepackage[utf8]{inputenc}
  \usepackage{textcomp} % provide euro and other symbols
\else % if luatex or xetex
  \usepackage{unicode-math} % this also loads fontspec
  \defaultfontfeatures{Scale=MatchLowercase}
  \defaultfontfeatures[\rmfamily]{Ligatures=TeX,Scale=1}
\fi
\usepackage{lmodern}
\ifPDFTeX\else
  % xetex/luatex font selection
\fi
% Use upquote if available, for straight quotes in verbatim environments
\IfFileExists{upquote.sty}{\usepackage{upquote}}{}
\IfFileExists{microtype.sty}{% use microtype if available
  \usepackage[]{microtype}
  \UseMicrotypeSet[protrusion]{basicmath} % disable protrusion for tt fonts
}{}
\makeatletter
\@ifundefined{KOMAClassName}{% if non-KOMA class
  \IfFileExists{parskip.sty}{%
    \usepackage{parskip}
  }{% else
    \setlength{\parindent}{0pt}
    \setlength{\parskip}{6pt plus 2pt minus 1pt}}
}{% if KOMA class
  \KOMAoptions{parskip=half}}
\makeatother
\setlength{\emergencystretch}{3em} % prevent overfull lines
\providecommand{\tightlist}{%
  \setlength{\itemsep}{0pt}\setlength{\parskip}{0pt}}
\usepackage{bookmark}
\IfFileExists{xurl.sty}{\usepackage{xurl}}{} % add URL line breaks if available
\urlstyle{same}
\hypersetup{
  hidelinks,
  pdfcreator={LaTeX via pandoc}}

\author{\texorpdfstring{}{}}
\date{}

\begin{document}

\section{Discussion}\label{sec:discussion}

\begin{frame}{Interpretation of Findings}
\protect\phantomsection\label{interpretation-of-findings}
The systematic analysis of Andrius Kulikauskas's Ways of Figuring Things
Out framework reveals several important patterns and insights into how
different approaches to knowledge are structured and interrelated.

\begin{block}{Framework Structure}
\protect\phantomsection\label{framework-structure}
The 24-room House of Knowledge provides a comprehensive organizational
structure for understanding different ways of figuring things out. The
distribution of ways across rooms reveals significant non-uniformity:
the B2 room (Believing in Believing) contains 23 ways (11.0\%), followed
by C4 (Caring about Caring about Caring about Caring) with 17 ways
(8.1\%), and R (Reflecting) with 16 ways (7.6\%). This concentration
suggests that certain aspects of knowledge---particularly the recursive
structures of believing and caring, and the reflective learning
process---are more amenable to multiple approaches, while other rooms
have fewer distinct ways.

The three fundamental structures---Believing (1-2-3-4), Caring
(1-2-3-4), and Relative Learning---provide a philosophical foundation
that organizes the rooms. The ways distributed across these structures
reflect different epistemological approaches, from absolute belief
structures to relative learning cycles.
\end{block}

\begin{block}{Dialogue Type Patterns}
\protect\phantomsection\label{dialogue-type-patterns}
The distribution of ways across 38 distinct dialogue types reveals
important patterns: ``goodness'' and ``other'' each account for 15 ways
(7.1\% each), followed by ``regularity'' (11 ways, 5.2\%), ``I'' and
``answer'' (9 ways each, 4.3\%). This distribution shows no single
dominant type, suggesting a balanced epistemological perspective that
values multiple approaches. The cross-tabulation analysis (Figure
\ref{fig:type_room_heatmap}) reveals strong associations: ``goodness''
appears prominently in both B2 (Believing) and C4 (Caring) rooms,
indicating it bridges these fundamental frameworks. This pattern
suggests that moral and ethical considerations (``goodness'') are
central to both believing and caring structures.

The dialogue type classification reflects different relationships to
truth and knowledge. While the framework includes Absolute, Relative,
and Embrace God perspectives, the actual distribution shows 38 distinct
dialogue types, with the most common being ``goodness'' and ``other''
(15 each). This diversity suggests that the framework recognizes
multiple valid ways of engaging with knowledge beyond the three primary
categories. The ``goodness'' type's prominence in both Believing (B2)
and Caring (C4) rooms indicates that ethical considerations are
fundamental to both frameworks, while ``other'' suggests ways that don't
fit neatly into standard categories, reflecting the framework's openness
to diverse approaches.
\end{block}

\begin{block}{Network Structure Insights}
\protect\phantomsection\label{network-structure-insights}
The network analysis reveals a highly connected structure with 1,290
edges connecting 210 ways, resulting in an average degree of 12.29
connections per way and a clustering coefficient of 0.886. The network
exhibits both local clustering (ways in the same room are highly
connected) and long-range connections (ways sharing dialogue types or
partners across different rooms). Central ways with degree centrality of
34 (ways 84, 156, 211) serve as major hubs, connecting multiple other
ways through shared rooms, dialogue types, or partners. These central
ways likely represent fundamental methods that connect different
categories or serve as entry points to the framework, as visualized in
Figure \ref{fig:ways_network}.

The clustering observed in the network indicates that ways group into
communities based on shared characteristics. These clusters may
correspond to: - Different aspects of the House of Knowledge - Different
dialogue types - Different philosophical approaches - Different
practical applications

The small-world properties (local clustering with long-range
connections) suggest that while ways cluster locally, there are also
important connections across clusters, creating a rich, interconnected
structure.
\end{block}
\end{frame}

\begin{frame}[fragile]{Philosophical Implications}
\protect\phantomsection\label{philosophical-implications}
\begin{block}{Epistemological Pluralism}
\protect\phantomsection\label{epistemological-pluralism}
The framework demonstrates epistemological pluralism---the recognition
that there are multiple valid ways of knowing and understanding. The 284
ways represent a comprehensive catalog of approaches, each valid in its
own context. This pluralism challenges monolithic views of knowledge and
suggests that different situations and questions may require different
approaches.

The organization into rooms and dialogue types provides a structure for
understanding when and how different ways are appropriate. Rather than
suggesting one ``correct'' way, the framework provides a map of options,
each with its own validity and application.
\end{block}

\begin{block}{Integration of Belief, Care, and Learning}
\protect\phantomsection\label{integration-of-belief-care-and-learning}
The framework integrates three fundamental aspects of knowledge: -
\textbf{Believing}: Reference to absolute structures or truths -
\textbf{Caring}: Openness to what is outside us - \textbf{Learning}: The
cycle of taking a stand, following through, and reflecting

This integration suggests that complete knowledge requires all three
aspects. Ways that emphasize only one aspect may be incomplete, while
ways that integrate multiple aspects may be more comprehensive. The
distribution of ways across these structures reflects the framework's
recognition of their interdependence.
\end{block}

\begin{block}{Dialogue and Knowledge}
\protect\phantomsection\label{dialogue-and-knowledge}
The emphasis on dialogue partners (\texttt{dialoguewith}) suggests that
knowledge is not purely individual but emerges through engagement with
others. Each way involves a dialogue partner, indicating that figuring
things out is fundamentally relational. This relational aspect
challenges purely individualistic views of knowledge and suggests that
understanding emerges through engagement with different perspectives.

The dialogue types (Absolute, Relative, Embrace God) represent different
modes of engagement, each valid in different contexts. The framework
suggests that effective knowledge acquisition requires understanding
which mode of dialogue is appropriate for which situation.
\end{block}
\end{frame}

\begin{frame}{Practical Applications}
\protect\phantomsection\label{practical-applications}
\begin{block}{Educational Contexts}
\protect\phantomsection\label{educational-contexts}
The framework has clear applications in education:

\begin{enumerate}
\tightlist
\item
  \textbf{Learning Style Recognition}: Understanding that different
  students may prefer different ways of figuring things out
\item
  \textbf{Teaching Methods}: Adapting teaching to match different ways
\item
  \textbf{Curriculum Design}: Organizing curriculum to expose students
  to multiple ways
\item
  \textbf{Assessment}: Recognizing that different ways may require
  different assessment methods
\end{enumerate}

The 24-room structure provides a framework for organizing educational
content and approaches, ensuring coverage of different aspects of
knowledge.
\end{block}

\begin{block}{Research Methodology}
\protect\phantomsection\label{research-methodology}
For researchers, the framework provides:

\begin{enumerate}
\tightlist
\item
  \textbf{Method Selection}: A systematic way to choose appropriate
  research methods
\item
  \textbf{Method Integration}: Understanding how different methods
  complement each other
\item
  \textbf{Epistemological Awareness}: Recognition of the epistemological
  assumptions underlying different methods
\item
  \textbf{Interdisciplinary Bridge}: A framework for understanding
  knowledge across disciplines
\end{enumerate}

The network structure helps researchers understand how different methods
relate and when to combine approaches.
\end{block}

\begin{block}{Personal Development}
\protect\phantomsection\label{personal-development}
For individuals, the framework offers:

\begin{enumerate}
\tightlist
\item
  \textbf{Self-Understanding}: Recognizing one's own preferred ways of
  figuring things out
\item
  \textbf{Expansion}: Learning new ways to expand one's capabilities
\item
  \textbf{Context Awareness}: Understanding which ways are appropriate
  for which situations
\item
  \textbf{Integration}: Developing the ability to use multiple ways as
  needed
\end{enumerate}

The House of Knowledge structure provides a map for personal growth,
showing areas where one might develop new ways of understanding.
\end{block}
\end{frame}

\begin{frame}{Limitations and Challenges}
\protect\phantomsection\label{limitations-and-challenges}
\begin{block}{Framework Completeness}
\protect\phantomsection\label{framework-completeness}
While the framework is comprehensive (284 ways), it may not be
exhaustive. New ways may emerge as knowledge evolves, or ways may be
discovered that don't fit the current structure. The framework should be
seen as a living system that can grow and adapt.
\end{block}

\begin{block}{Cultural Context}
\protect\phantomsection\label{cultural-context}
The framework emerges from a specific cultural and philosophical context
(Andrius Kulikauskas's work). While it aims for universality, some ways
may be more relevant in certain cultural contexts than others. The
framework's applicability across cultures requires further
investigation.
\end{block}

\begin{block}{Measurement Challenges}
\protect\phantomsection\label{measurement-challenges}
Quantitative analysis of ways faces challenges: - Ways are qualitative
and may resist precise measurement - Relationships between ways may be
complex and multi-dimensional - The framework's philosophical nature
makes some aspects difficult to quantify

These challenges suggest that quantitative analysis should complement,
not replace, qualitative understanding.
\end{block}
\end{frame}

\begin{frame}{Future Research Directions}
\protect\phantomsection\label{future-research-directions}
\begin{block}{Framework Expansion}
\protect\phantomsection\label{framework-expansion}
Future research could: 1. Document additional ways beyond the current
284 2. Explore ways from other philosophical traditions 3. Investigate
ways in specific domains (science, art, etc.) 4. Develop ways for
emerging contexts (digital, global, etc.)
\end{block}

\begin{block}{Empirical Validation}
\protect\phantomsection\label{empirical-validation}
Empirical research could: 1. Test the effectiveness of different ways in
different contexts 2. Investigate individual differences in way
preferences 3. Study how ways develop and change over time 4. Examine
the relationship between ways and learning outcomes
\end{block}

\begin{block}{Computational Applications}
\protect\phantomsection\label{computational-applications}
Computational research could: 1. Develop AI systems that use different
ways 2. Create recommendation systems for way selection 3. Build tools
for way analysis and visualization 4. Develop educational software based
on the framework
\end{block}

\begin{block}{Interdisciplinary Integration}
\protect\phantomsection\label{interdisciplinary-integration}
The framework could be integrated with: 1. Cognitive science research on
learning 2. Educational research on teaching methods 3. Philosophy of
science and epistemology 4. Knowledge management and organizational
learning
\end{block}
\end{frame}

\begin{frame}{Broader Impact}
\protect\phantomsection\label{broader-impact}
\begin{block}{Contribution to Epistemology}
\protect\phantomsection\label{contribution-to-epistemology}
The framework contributes to epistemology by: 1. Providing a
comprehensive catalog of ways of knowing 2. Showing the relationships
between different approaches 3. Demonstrating the validity of multiple
perspectives 4. Integrating belief, care, and learning in knowledge
acquisition
\end{block}

\begin{block}{Contribution to Education}
\protect\phantomsection\label{contribution-to-education}
The framework contributes to education by: 1. Providing a systematic
approach to understanding learning 2. Recognizing the validity of
multiple learning approaches 3. Offering a structure for curriculum and
teaching 4. Supporting personalized and adaptive education
\end{block}

\begin{block}{Contribution to Research}
\protect\phantomsection\label{contribution-to-research}
The framework contributes to research by: 1. Providing a systematic
approach to method selection 2. Showing how different methods relate and
complement 3. Encouraging epistemological awareness 4. Supporting
interdisciplinary research
\end{block}
\end{frame}

\begin{frame}{Conclusion}
\protect\phantomsection\label{conclusion}
The systematic analysis of the Ways of Figuring Things Out framework
reveals a rich, structured approach to understanding knowledge
acquisition. The 24-room House of Knowledge provides organization, the
dialogue types reveal different modes of engagement, and the network
structure shows how ways interconnect. The framework demonstrates
epistemological pluralism while providing structure for understanding
when and how different ways are appropriate.

The practical applications span education, research, and personal
development, offering tools for understanding and applying different
approaches to knowledge. Future research can expand the framework,
validate it empirically, and develop computational and interdisciplinary
applications.

This work provides both a philosophical framework and a practical system
for understanding and applying diverse ways of figuring things out,
contributing to epistemology, education, and research methodology.
\end{frame}

\end{document}
