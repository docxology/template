% Options for packages loaded elsewhere
\PassOptionsToPackage{unicode}{hyperref}
\PassOptionsToPackage{hyphens}{url}
\documentclass[
  ignorenonframetext,
]{beamer}
\newif\ifbibliography
\usepackage{pgfpages}
\setbeamertemplate{caption}[numbered]
\setbeamertemplate{caption label separator}{: }
\setbeamercolor{caption name}{fg=normal text.fg}
\beamertemplatenavigationsymbolsempty
% remove section numbering
\setbeamertemplate{part page}{
  \centering
  \begin{beamercolorbox}[sep=16pt,center]{part title}
    \usebeamerfont{part title}\insertpart\par
  \end{beamercolorbox}
}
\setbeamertemplate{section page}{
  \centering
  \begin{beamercolorbox}[sep=12pt,center]{section title}
    \usebeamerfont{section title}\insertsection\par
  \end{beamercolorbox}
}
\setbeamertemplate{subsection page}{
  \centering
  \begin{beamercolorbox}[sep=8pt,center]{subsection title}
    \usebeamerfont{subsection title}\insertsubsection\par
  \end{beamercolorbox}
}
% Prevent slide breaks in the middle of a paragraph
\widowpenalties 1 10000
\raggedbottom
\AtBeginPart{
  \frame{\partpage}
}
\AtBeginSection{
  \ifbibliography
  \else
    \frame{\sectionpage}
  \fi
}
\AtBeginSubsection{
  \frame{\subsectionpage}
}
\usepackage{iftex}
\ifPDFTeX
  \usepackage[T1]{fontenc}
  \usepackage[utf8]{inputenc}
  \usepackage{textcomp} % provide euro and other symbols
\else % if luatex or xetex
  \usepackage{unicode-math} % this also loads fontspec
  \defaultfontfeatures{Scale=MatchLowercase}
  \defaultfontfeatures[\rmfamily]{Ligatures=TeX,Scale=1}
\fi
\usepackage{lmodern}
\ifPDFTeX\else
  % xetex/luatex font selection
\fi
% Use upquote if available, for straight quotes in verbatim environments
\IfFileExists{upquote.sty}{\usepackage{upquote}}{}
\IfFileExists{microtype.sty}{% use microtype if available
  \usepackage[]{microtype}
  \UseMicrotypeSet[protrusion]{basicmath} % disable protrusion for tt fonts
}{}
\makeatletter
\@ifundefined{KOMAClassName}{% if non-KOMA class
  \IfFileExists{parskip.sty}{%
    \usepackage{parskip}
  }{% else
    \setlength{\parindent}{0pt}
    \setlength{\parskip}{6pt plus 2pt minus 1pt}}
}{% if KOMA class
  \KOMAoptions{parskip=half}}
\makeatother
\usepackage{longtable,booktabs,array}
\newcounter{none} % for unnumbered tables
\usepackage{calc} % for calculating minipage widths
\usepackage{caption}
% Make caption package work with longtable
\makeatletter
\def\fnum@table{\tablename~\thetable}
\makeatother
\setlength{\emergencystretch}{3em} % prevent overfull lines
\providecommand{\tightlist}{%
  \setlength{\itemsep}{0pt}\setlength{\parskip}{0pt}}
\usepackage{bookmark}
\IfFileExists{xurl.sty}{\usepackage{xurl}}{} % add URL line breaks if available
\urlstyle{same}
\hypersetup{
  hidelinks,
  pdfcreator={LaTeX via pandoc}}

\author{\texorpdfstring{}{}}
\date{}

\begin{document}

\section{Discussion}\label{discussion}

\begin{frame}{Set Theory vs.~Containment Theory}
\protect\phantomsection\label{set-theory-vs.-containment-theory}
The comparison between classical Set Theory (ZFC) and Containment Theory
reveals fundamental differences in approach, axiomatics, and conceptual
structure.

\begin{block}{Axiomatic Economy}
\protect\phantomsection\label{axiomatic-economy}
{\def\LTcaptype{none} % do not increment counter
\begin{longtable}[]{@{}
  >{\raggedright\arraybackslash}p{(\linewidth - 4\tabcolsep) * \real{0.2292}}
  >{\raggedright\arraybackslash}p{(\linewidth - 4\tabcolsep) * \real{0.3750}}
  >{\raggedright\arraybackslash}p{(\linewidth - 4\tabcolsep) * \real{0.3958}}@{}}
\toprule\noalign{}
\begin{minipage}[b]{\linewidth}\raggedright
Criterion
\end{minipage} & \begin{minipage}[b]{\linewidth}\raggedright
Set Theory (ZFC)
\end{minipage} & \begin{minipage}[b]{\linewidth}\raggedright
Containment Theory
\end{minipage} \\
\midrule\noalign{}
\endhead
\textbf{Number of Axioms} & 9 (including Choice) & 2 \\
\textbf{Primitive Notion} & Membership (\(\in\)) & Distinction
(boundary) \\
\textbf{Undefined Terms} & Set, membership & Mark, void \\
\textbf{Infinity Required} & Yes (Axiom of Infinity) & No (finite
calculus) \\
\bottomrule\noalign{}
\end{longtable}
}

Set Theory requires: 1. Extensionality 2. Empty Set 3. Pairing 4. Union
5. Power Set 6. Infinity 7. Separation (schema) 8. Replacement (schema)
9. Foundation (Regularity) 10. Choice (optional)

Containment Theory requires only: 1. Calling:
\(\langle\langle a \rangle\rangle = a\) 2. Crossing:
\(\langle\ \rangle\langle\ \rangle = \langle\ \rangle\)
\end{block}

\begin{block}{Expressiveness Comparison}
\protect\phantomsection\label{expressiveness-comparison}
{\def\LTcaptype{none} % do not increment counter
\begin{longtable}[]{@{}
  >{\raggedright\arraybackslash}p{(\linewidth - 4\tabcolsep) * \real{0.2250}}
  >{\raggedright\arraybackslash}p{(\linewidth - 4\tabcolsep) * \real{0.3000}}
  >{\raggedright\arraybackslash}p{(\linewidth - 4\tabcolsep) * \real{0.4750}}@{}}
\toprule\noalign{}
\begin{minipage}[b]{\linewidth}\raggedright
Concept
\end{minipage} & \begin{minipage}[b]{\linewidth}\raggedright
Set Theory
\end{minipage} & \begin{minipage}[b]{\linewidth}\raggedright
Containment Theory
\end{minipage} \\
\midrule\noalign{}
\endhead
TRUE & \(\{x : x = x\}\) (universe) & \(\langle\ \rangle\) \\
FALSE & \(\emptyset\) (empty set) & void \\
NOT & Complement \(A^c\) & Enclosure \(\langle a \rangle\) \\
AND & Intersection \(A \cap B\) & Juxtaposition \(ab\) \\
OR & Union \(A \cup B\) &
\(\langle\langle a \rangle\langle b \rangle\rangle\) \\
Implication & \(A^c \cup B\) & \(\langle a\langle b \rangle\rangle\) \\
\bottomrule\noalign{}
\end{longtable}
}

Both systems achieve Boolean completeness, but through fundamentally
different primitives.
\end{block}

\begin{block}{Self-Reference and Paradoxes}
\protect\phantomsection\label{self-reference-and-paradoxes}
\textbf{Russell's Paradox in Set Theory}:

The set \(R = \{x : x \notin x\}\) leads to contradiction: - If
\(R \in R\), then \(R \notin R\) - If \(R \notin R\), then \(R \in R\)

Set Theory resolves this by restricting comprehension (no unrestricted
set formation).

\textbf{Self-Reference in Containment Theory}:

The equation \(f = \langle f \rangle\) has no solution among marks and
voids. Spencer-Brown introduces \textbf{imaginary values}---forms that
oscillate between states:

\[j = \langle j \rangle\]

This imaginary value \(j\) is neither marked nor void but alternates
between them over ``time.'' Rather than a paradox, self-reference
becomes a dynamic oscillation.

\textbf{Comparison}:

{\def\LTcaptype{none} % do not increment counter
\begin{longtable}[]{@{}ll@{}}
\toprule\noalign{}
System & Self-Reference Treatment \\
\midrule\noalign{}
\endhead
Set Theory & Paradox → Restriction (Foundation axiom) \\
Containment Theory & Imaginary value → Dynamic oscillation \\
\bottomrule\noalign{}
\end{longtable}
}
\end{block}

\begin{block}{Geometric Intuition}
\protect\phantomsection\label{geometric-intuition}
{\def\LTcaptype{none} % do not increment counter
\begin{longtable}[]{@{}lll@{}}
\toprule\noalign{}
Feature & Set Theory & Containment Theory \\
\midrule\noalign{}
\endhead
\textbf{Visualization} & Venn diagrams (regions) & Nested boundaries \\
\textbf{Primitive Operation} & Collection & Drawing a line \\
\textbf{Spatial Metaphor} & Contains (membership) & Inside/Outside \\
\textbf{Natural Interpretation} & Abstract & Geometric \\
\bottomrule\noalign{}
\end{longtable}
}

Boundary logic's operations map directly to spatial actions: -
\textbf{Making a mark}: Drawing a boundary - \textbf{Enclosure}:
Creating an inside - \textbf{Juxtaposition}: Side-by-side placement -
\textbf{Calling}: Crossing back through a boundary
\end{block}

\begin{block}{Complexity Implications}
\protect\phantomsection\label{complexity-implications}
\textbf{Set-theoretic Boolean operations} require: - Universe definition
- Complement with respect to universe - Intersection defined via
membership

\textbf{Boundary logic Boolean operations}: - Mark is TRUE (primitive) -
Enclosure is NOT (one rule) - Juxtaposition is AND (spatial) -
Everything else derived

The reduction algorithm in Containment Theory operates in polynomial
time for ground forms, while SAT solving (Boolean satisfiability) is
NP-complete. This does not contradict---the boundary calculus solves
\emph{evaluation}, not \emph{satisfiability}.
\end{block}
\end{frame}

\begin{frame}{Theoretical Implications}
\protect\phantomsection\label{theoretical-implications}
\begin{block}{Foundations of Mathematics}
\protect\phantomsection\label{foundations-of-mathematics}
Containment Theory suggests that mathematical foundations need not be as
complex as ZFC. For finite, discrete structures: - Boolean algebra -
Propositional logic - Digital circuits - Finite state machines

The two-axiom system suffices completely.
\end{block}

\begin{block}{Philosophy of Distinction}
\protect\phantomsection\label{philosophy-of-distinction}
Spencer-Brown's system has philosophical implications:

\textbf{Epistemological}: All knowledge begins with
distinction---separating figure from ground, this from that.

\textbf{Ontological}: The void (undistinguished space) may represent
pre-phenomenal reality; distinction creates existence.

\textbf{Self-Reference}: The imaginary values suggest that
self-reference is not paradoxical but generates temporal
dynamics---consciousness observing itself creates oscillation.
\end{block}

\begin{block}{Connections to Other Formalisms}
\protect\phantomsection\label{connections-to-other-formalisms}
\textbf{Category Theory}: Forms can be viewed as morphisms; the axioms
define natural transformations.

\textbf{Type Theory}: The mark/void distinction parallels
inhabited/empty types.

\textbf{Lambda Calculus}: Enclosure resembles abstraction; juxtaposition
resembles application.

\textbf{Homotopy Type Theory}: Boundaries as paths; calling as path
inversion.
\end{block}
\end{frame}

\begin{frame}{Applications}
\protect\phantomsection\label{applications}
\begin{block}{Digital Circuit Design}
\protect\phantomsection\label{digital-circuit-design}
The NAND gate is functionally complete and corresponds directly to
\(\langle ab \rangle\):

{\def\LTcaptype{none} % do not increment counter
\begin{longtable}[]{@{}
  >{\raggedright\arraybackslash}p{(\linewidth - 6\tabcolsep) * \real{0.1111}}
  >{\raggedright\arraybackslash}p{(\linewidth - 6\tabcolsep) * \real{0.1111}}
  >{\raggedright\arraybackslash}p{(\linewidth - 6\tabcolsep) * \real{0.3111}}
  >{\raggedright\arraybackslash}p{(\linewidth - 6\tabcolsep) * \real{0.4667}}@{}}
\toprule\noalign{}
\begin{minipage}[b]{\linewidth}\raggedright
\(a\)
\end{minipage} & \begin{minipage}[b]{\linewidth}\raggedright
\(b\)
\end{minipage} & \begin{minipage}[b]{\linewidth}\raggedright
\(a\) NAND \(b\)
\end{minipage} & \begin{minipage}[b]{\linewidth}\raggedright
\(\langle ab \rangle\)
\end{minipage} \\
\midrule\noalign{}
\endhead
T & T & F &
\(\langle\langle\ \rangle\langle\ \rangle\rangle = \emptyset\) \\
T & F & T &
\(\langle\langle\ \rangle\emptyset\rangle = \langle\ \rangle\) \\
F & T & T &
\(\langle\emptyset\langle\ \rangle\rangle = \langle\ \rangle\) \\
F & F & T & \(\langle\emptyset\rangle = \langle\ \rangle\) \\
\bottomrule\noalign{}
\end{longtable}
}

Circuit optimization can leverage boundary reduction rules.
\end{block}

\begin{block}{Cognitive Modeling}
\protect\phantomsection\label{cognitive-modeling}
The calculus of indications models basic cognitive operations: -
\textbf{Perception}: Making distinctions - \textbf{Negation}: Crossing
boundaries - \textbf{Conjunction}: Simultaneous attention -
\textbf{Oscillation}: Self-reflective awareness

Free energy principles in cognitive science relate to maintaining
distinction boundaries.
\end{block}

\begin{block}{Formal Verification}
\protect\phantomsection\label{formal-verification}
Boundary logic offers potential advantages for verification: - Explicit
reduction traces (proof witnesses) - Polynomial-time evaluation -
Geometric proof visualization
\end{block}
\end{frame}

\begin{frame}{Limitations}
\protect\phantomsection\label{limitations}
\begin{block}{What Containment Theory Does Not Replace}
\protect\phantomsection\label{what-containment-theory-does-not-replace}
\begin{enumerate}
\tightlist
\item
  \textbf{Set Theory for infinite structures}: ZFC handles infinite
  sets, ordinals, cardinals
\item
  \textbf{Numerical computation}: Arithmetic requires additional
  structure
\item
  \textbf{Analysis}: Real numbers, limits, continuity need richer
  foundations
\end{enumerate}
\end{block}

\begin{block}{Current Implementation Limitations}
\protect\phantomsection\label{current-implementation-limitations}
\begin{enumerate}
\tightlist
\item
  \textbf{Variable handling}: Current implementation focuses on ground
  forms
\item
  \textbf{Proof automation}: Limited to reduction-based verification
\item
  \textbf{Visualization}: Nested boundaries become complex at high depth
\end{enumerate}
\end{block}
\end{frame}

\begin{frame}{Future Directions}
\protect\phantomsection\label{future-directions}
\begin{block}{Extensions}
\protect\phantomsection\label{extensions}
\begin{enumerate}
\tightlist
\item
  \textbf{Imaginary values}: Full computational treatment of
  self-referential forms
\item
  \textbf{Arithmetic}: Boundary representations for natural numbers
  (Bricken's iconic arithmetic)
\item
  \textbf{Higher-order logic}: Extending to predicate calculus
\end{enumerate}
\end{block}

\begin{block}{Applications}
\protect\phantomsection\label{applications-1}
\begin{enumerate}
\tightlist
\item
  \textbf{Quantum computing}: Boundary logic for superposition states
\item
  \textbf{Neural networks}: Boundary-based activation functions
\item
  \textbf{Knowledge representation}: Spatial logic for AI systems
\end{enumerate}
\end{block}

\begin{block}{Theoretical Questions}
\protect\phantomsection\label{theoretical-questions}
\begin{enumerate}
\tightlist
\item
  \textbf{Completeness}: Is the consequence system complete for all
  Boolean identities?
\item
  \textbf{Complexity}: Tight bounds on reduction complexity
\item
  \textbf{Categorification}: Full categorical treatment of boundary
  logic
\end{enumerate}
\end{block}
\end{frame}

\end{document}
