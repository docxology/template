% Options for packages loaded elsewhere
\PassOptionsToPackage{unicode}{hyperref}
\PassOptionsToPackage{hyphens}{url}
\documentclass[
  ignorenonframetext,
]{beamer}
\newif\ifbibliography
\usepackage{pgfpages}
\setbeamertemplate{caption}[numbered]
\setbeamertemplate{caption label separator}{: }
\setbeamercolor{caption name}{fg=normal text.fg}
\beamertemplatenavigationsymbolsempty
% remove section numbering
\setbeamertemplate{part page}{
  \centering
  \begin{beamercolorbox}[sep=16pt,center]{part title}
    \usebeamerfont{part title}\insertpart\par
  \end{beamercolorbox}
}
\setbeamertemplate{section page}{
  \centering
  \begin{beamercolorbox}[sep=12pt,center]{section title}
    \usebeamerfont{section title}\insertsection\par
  \end{beamercolorbox}
}
\setbeamertemplate{subsection page}{
  \centering
  \begin{beamercolorbox}[sep=8pt,center]{subsection title}
    \usebeamerfont{subsection title}\insertsubsection\par
  \end{beamercolorbox}
}
% Prevent slide breaks in the middle of a paragraph
\widowpenalties 1 10000
\raggedbottom
\AtBeginPart{
  \frame{\partpage}
}
\AtBeginSection{
  \ifbibliography
  \else
    \frame{\sectionpage}
  \fi
}
\AtBeginSubsection{
  \frame{\subsectionpage}
}
\usepackage{iftex}
\ifPDFTeX
  \usepackage[T1]{fontenc}
  \usepackage[utf8]{inputenc}
  \usepackage{textcomp} % provide euro and other symbols
\else % if luatex or xetex
  \usepackage{unicode-math} % this also loads fontspec
  \defaultfontfeatures{Scale=MatchLowercase}
  \defaultfontfeatures[\rmfamily]{Ligatures=TeX,Scale=1}
\fi
\usepackage{lmodern}
\ifPDFTeX\else
  % xetex/luatex font selection
\fi
% Use upquote if available, for straight quotes in verbatim environments
\IfFileExists{upquote.sty}{\usepackage{upquote}}{}
\IfFileExists{microtype.sty}{% use microtype if available
  \usepackage[]{microtype}
  \UseMicrotypeSet[protrusion]{basicmath} % disable protrusion for tt fonts
}{}
\makeatletter
\@ifundefined{KOMAClassName}{% if non-KOMA class
  \IfFileExists{parskip.sty}{%
    \usepackage{parskip}
  }{% else
    \setlength{\parindent}{0pt}
    \setlength{\parskip}{6pt plus 2pt minus 1pt}}
}{% if KOMA class
  \KOMAoptions{parskip=half}}
\makeatother
\setlength{\emergencystretch}{3em} % prevent overfull lines
\providecommand{\tightlist}{%
  \setlength{\itemsep}{0pt}\setlength{\parskip}{0pt}}
\usepackage{bookmark}
\IfFileExists{xurl.sty}{\usepackage{xurl}}{} % add URL line breaks if available
\urlstyle{same}
\hypersetup{
  hidelinks,
  pdfcreator={LaTeX via pandoc}}

\author{\texorpdfstring{}{}}
\date{}

\begin{document}

\section{Discussion}\label{sec:discussion}

\subsection{Biological Insights}\label{biological-insights}

\begin{frame}{Compatibility Mechanisms}
\protect\phantomsection\label{compatibility-mechanisms}
The strong correlation between phylogenetic distance and graft
compatibility (\(r = -0.75\)) confirms that evolutionary relationships
are the primary determinant of successful graft unions. This
relationship reflects shared anatomical structures, biochemical
pathways, and growth patterns that enable vascular integration. Closely
related species share similar cambium characteristics, vascular anatomy,
and hormonal signaling systems, facilitating successful union formation
\cite{melnyk2018, goldschmidt2014}.

The exponential decay model \eqref{eq:phylogenetic_compatibility} with
decay constant \(k \approx 2.0\) suggests that compatibility decreases
rapidly beyond genus-level relationships. This finding has practical
implications for rootstock-scion selection, indicating that
intra-generic combinations should be prioritized when high success rates
are required.
\end{frame}

\begin{frame}{Healing Process Dynamics}
\protect\phantomsection\label{healing-process-dynamics}
Our simulation models
\eqref{eq:healing_dynamics}-\eqref{eq:vascular_dynamics} capture the
sequential nature of graft healing: cambium contact must precede callus
formation, which in turn enables vascular connection. This temporal
sequence reflects the biological reality that each stage provides the
foundation for the next, with environmental conditions modulating the
rate of progression at each stage.

The model predictions align well with literature-reported healing
timelines, validating our understanding of the biological processes. The
exponential growth patterns observed in callus formation and vascular
connection reflect the self-reinforcing nature of tissue development,
where established connections facilitate further growth.
\end{frame}

\subsection{Technical Implications}\label{technical-implications}

\begin{frame}{Technique Selection Guidelines}
\protect\phantomsection\label{technique-selection-guidelines}
The comparative analysis of grafting techniques reveals clear guidelines
for technique selection:

\begin{itemize}
\tightlist
\item
  \textbf{Diameter matching}: Whip and tongue requires precise diameter
  matching (within 10\%), while cleft and bark grafting tolerate larger
  mismatches
\item
  \textbf{Rootstock size}: Large diameter rootstock (\textgreater20 mm)
  favors cleft or bark grafting
\item
  \textbf{Mass propagation}: Bud grafting offers highest efficiency for
  commercial operations
\item
  \textbf{Precision requirement}: Whip and tongue demands highest skill
  level but offers best success rates
\end{itemize}

These findings support evidence-based technique selection, moving beyond
traditional rules of thumb to data-driven recommendations.
\end{frame}

\begin{frame}{Environmental Management}
\protect\phantomsection\label{environmental-management}
The environmental analysis demonstrates the critical importance of
post-grafting care. Optimal conditions (temperature 20-25°C, humidity
70-90\%) can improve success rates by 15-20\% compared to suboptimal
conditions. This finding emphasizes the need for controlled environments
in commercial grafting operations, particularly for high-value species
or difficult combinations.

The environmental suitability model \eqref{eq:environmental_score}
provides a quantitative framework for assessing grafting conditions,
enabling practitioners to optimize their operations through
environmental control.
\end{frame}

\subsection{Agricultural Applications}\label{agricultural-applications}

\begin{frame}{Commercial Fruit Production}
\protect\phantomsection\label{commercial-fruit-production}
The economic analysis reveals that grafting operations are highly
viable, with break-even success rates (17.5\%) well below typical
performance (70-85\%). This economic margin provides flexibility for
experimentation and optimization, supporting innovation in
rootstock-scion combinations.

The productivity metrics (8,750-17,000 successful grafts per year per
worker) demonstrate the scalability of commercial grafting operations.
Combined with the economic viability, these figures support the
continued importance of grafting in modern fruit production.
\end{frame}

\begin{frame}{Rootstock Breeding Programs}
\protect\phantomsection\label{rootstock-breeding-programs}
The compatibility prediction framework enables more efficient rootstock
breeding programs by identifying promising combinations before extensive
field trials. The ability to predict compatibility from phylogenetic
relationships and biological characteristics reduces the time and cost
of rootstock development, accelerating the introduction of improved
rootstocks for disease resistance, vigor control, and climate
adaptation.
\end{frame}

\begin{frame}{Climate Adaptation}
\protect\phantomsection\label{climate-adaptation}
As climate change alters growing conditions, grafting provides a
mechanism for rapid adaptation. The ability to combine climate-adapted
rootstocks with desirable scion characteristics enables extension of
cultivation ranges and maintenance of production under changing
conditions. Our seasonal planning algorithms support this adaptation by
identifying optimal timing windows across different climate zones.
\end{frame}

\subsection{Economic Considerations}\label{economic-considerations}

\begin{frame}{Cost-Benefit Analysis}
\protect\phantomsection\label{cost-benefit-analysis}
The economic analysis demonstrates that grafting operations are
economically viable across a wide range of success rates. With
break-even rates around 17.5\% and typical success rates of 70-85\%,
grafting operations generate substantial economic returns. The high
value of successful grafts (\$20 per graft) relative to costs (\$3.50
per attempt) creates strong economic incentives for quality execution
and optimal technique selection.
\end{frame}

\begin{frame}{Market Dynamics}
\protect\phantomsection\label{market-dynamics}
The economic viability of grafting supports a robust market for grafted
plants, with commercial nurseries producing millions of grafted trees
annually. The ability to predict success rates and optimize operations
through our computational toolkit can improve profitability and reduce
waste, benefiting both producers and consumers.
\end{frame}

\subsection{Cultural and Historical
Perspectives}\label{cultural-and-historical-perspectives}

\begin{frame}{Traditional Knowledge Integration}
\protect\phantomsection\label{traditional-knowledge-integration}
The 4,000+ year history of grafting represents a rich repository of
traditional knowledge that has been refined through generations of
practice. Our computational framework synthesizes this traditional
knowledge with modern scientific understanding, creating a bridge
between empirical practice and theoretical analysis.

The technique library documents methods that have been passed down
through generations, preserving this knowledge while making it
accessible to modern practitioners. This integration of traditional and
scientific knowledge represents a valuable contribution to agricultural
science.
\end{frame}

\begin{frame}{Regional Variations}
\protect\phantomsection\label{regional-variations}
Grafting techniques have evolved differently across regions, reflecting
local conditions, available species, and cultural practices. Our
framework accommodates these variations through parameterized models
that can be adjusted for different contexts, supporting both
preservation of traditional methods and adaptation to new conditions.
\end{frame}

\subsection{Limitations and
Challenges}\label{limitations-and-challenges}

\begin{frame}{Model Limitations}
\protect\phantomsection\label{model-limitations}
While our compatibility prediction model shows good accuracy
(\(r = 0.78\)), several limitations remain:

\begin{enumerate}
\tightlist
\item
  \textbf{Molecular factors}: Current models do not incorporate
  molecular compatibility markers (DNA, proteins)
\item
  \textbf{Long-term performance}: Predictions focus on initial union
  formation, not long-term compatibility
\item
  \textbf{Disease interactions}: Models do not account for disease
  transmission through grafts
\item
  \textbf{Stress responses}: Limited incorporation of stress-induced
  incompatibility
\end{enumerate}

These limitations represent opportunities for future research and model
refinement.
\end{frame}

\begin{frame}{Data Availability}
\protect\phantomsection\label{data-availability}
The synthetic nature of our trial data, while realistic and based on
literature parameters, represents a limitation. Validation with
real-world field trial data would strengthen the model predictions and
provide more accurate success rate estimates for specific species
combinations.
\end{frame}

\begin{frame}{Computational Complexity}
\protect\phantomsection\label{computational-complexity}
While our simulation models provide valuable insights, they simplify the
complex biological processes involved in graft healing. More
sophisticated models incorporating molecular-level interactions,
hormonal signaling, and stress responses could provide deeper
understanding but would require significantly more computational
resources.
\end{frame}

\subsection{Future Research
Directions}\label{future-research-directions}

\begin{frame}{Molecular Compatibility Markers}
\protect\phantomsection\label{molecular-compatibility-markers}
Future research should explore molecular markers for compatibility
prediction, potentially enabling rapid screening of rootstock-scion
combinations without extensive field trials. DNA sequencing, proteomic
analysis, and metabolomic profiling could identify compatibility markers
that improve prediction accuracy beyond phylogenetic relationships.
\end{frame}

\begin{frame}{Climate Adaptation Strategies}
\protect\phantomsection\label{climate-adaptation-strategies}
As climate change accelerates, research into climate-adapted
rootstock-scion combinations becomes increasingly important. Our
framework provides a foundation for this research, but extension to
incorporate climate projections and adaptation strategies would enhance
its utility.
\end{frame}

\begin{frame}{Novel Grafting Techniques}
\protect\phantomsection\label{novel-grafting-techniques}
Development of new grafting techniques for difficult species or
challenging conditions represents an important research direction. Our
framework can support this development by providing simulation
capabilities for testing hypothetical techniques before field trials.
\end{frame}

\begin{frame}{Machine Learning Integration}
\protect\phantomsection\label{machine-learning-integration}
Integration of machine learning methods could improve prediction
accuracy by identifying complex patterns in compatibility data that are
not captured by our current models. Large-scale data collection from
commercial operations could support this development.
\end{frame}

\subsection{Broader Impact}\label{broader-impact}

\begin{frame}{Food Security}
\protect\phantomsection\label{food-security}
Grafting contributes to global food security by enabling efficient
production of high-quality fruits and nuts. The ability to optimize
grafting operations through our computational toolkit can improve
productivity and reduce waste, supporting food security goals.
\end{frame}

\begin{frame}{Conservation Applications}
\protect\phantomsection\label{conservation-applications}
Grafting enables conservation of rare or endangered species by allowing
propagation when seed production is limited. Our framework supports
these conservation efforts by providing compatibility predictions and
technique recommendations for challenging species.
\end{frame}

\begin{frame}{Educational Value}
\protect\phantomsection\label{educational-value}
The comprehensive review and computational toolkit provide educational
resources for students and practitioners. The integration of biological
mechanisms, historical context, and practical applications creates a
rich learning environment that supports skill development in
horticulture and arboriculture.
\end{frame}

\end{document}
