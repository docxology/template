% Options for packages loaded elsewhere
\PassOptionsToPackage{unicode}{hyperref}
\PassOptionsToPackage{hyphens}{url}
\documentclass[
  ignorenonframetext,
]{beamer}
\newif\ifbibliography
\usepackage{pgfpages}
\setbeamertemplate{caption}[numbered]
\setbeamertemplate{caption label separator}{: }
\setbeamercolor{caption name}{fg=normal text.fg}
\beamertemplatenavigationsymbolsempty
% remove section numbering
\setbeamertemplate{part page}{
  \centering
  \begin{beamercolorbox}[sep=16pt,center]{part title}
    \usebeamerfont{part title}\insertpart\par
  \end{beamercolorbox}
}
\setbeamertemplate{section page}{
  \centering
  \begin{beamercolorbox}[sep=12pt,center]{section title}
    \usebeamerfont{section title}\insertsection\par
  \end{beamercolorbox}
}
\setbeamertemplate{subsection page}{
  \centering
  \begin{beamercolorbox}[sep=8pt,center]{subsection title}
    \usebeamerfont{subsection title}\insertsubsection\par
  \end{beamercolorbox}
}
% Prevent slide breaks in the middle of a paragraph
\widowpenalties 1 10000
\raggedbottom
\AtBeginPart{
  \frame{\partpage}
}
\AtBeginSection{
  \ifbibliography
  \else
    \frame{\sectionpage}
  \fi
}
\AtBeginSubsection{
  \frame{\subsectionpage}
}
\usepackage{iftex}
\ifPDFTeX
  \usepackage[T1]{fontenc}
  \usepackage[utf8]{inputenc}
  \usepackage{textcomp} % provide euro and other symbols
\else % if luatex or xetex
  \usepackage{unicode-math} % this also loads fontspec
  \defaultfontfeatures{Scale=MatchLowercase}
  \defaultfontfeatures[\rmfamily]{Ligatures=TeX,Scale=1}
\fi
\usepackage{lmodern}
\ifPDFTeX\else
  % xetex/luatex font selection
\fi
% Use upquote if available, for straight quotes in verbatim environments
\IfFileExists{upquote.sty}{\usepackage{upquote}}{}
\IfFileExists{microtype.sty}{% use microtype if available
  \usepackage[]{microtype}
  \UseMicrotypeSet[protrusion]{basicmath} % disable protrusion for tt fonts
}{}
\makeatletter
\@ifundefined{KOMAClassName}{% if non-KOMA class
  \IfFileExists{parskip.sty}{%
    \usepackage{parskip}
  }{% else
    \setlength{\parindent}{0pt}
    \setlength{\parskip}{6pt plus 2pt minus 1pt}}
}{% if KOMA class
  \KOMAoptions{parskip=half}}
\makeatother
\usepackage{longtable,booktabs,array}
\newcounter{none} % for unnumbered tables
\usepackage{calc} % for calculating minipage widths
\usepackage{caption}
% Make caption package work with longtable
\makeatletter
\def\fnum@table{\tablename~\thetable}
\makeatother
\setlength{\emergencystretch}{3em} % prevent overfull lines
\providecommand{\tightlist}{%
  \setlength{\itemsep}{0pt}\setlength{\parskip}{0pt}}
\usepackage{bookmark}
\IfFileExists{xurl.sty}{\usepackage{xurl}}{} % add URL line breaks if available
\urlstyle{same}
\hypersetup{
  hidelinks,
  pdfcreator={LaTeX via pandoc}}

\author{\texorpdfstring{}{}}
\date{}

\begin{document}

\begin{frame}{Results}
\protect\phantomsection\label{results}
This section presents the theoretical results and mathematical
derivations obtained through our methodological approach.

\begin{block}{Theoretical Results}
\protect\phantomsection\label{theoretical-results}
The main theoretical contribution is encapsulated in the following
proposition, building on established optimization theory
\cite{bertsekas1999nonlinear, boyd2004convex}. This prose-focused
project demonstrates mathematical exposition without requiring figure
generation, highlighting the template's flexibility for different
research approaches.

\textbf{Proposition 1.} For any continuously differentiable function
\(f: \mathbb{R}^n \rightarrow \mathbb{R}\), the gradient descent
algorithm with appropriate step sizes converges to a stationary point.
\end{block}

\begin{block}{Mathematical Derivations}
\protect\phantomsection\label{mathematical-derivations}
Consider the Taylor expansion of \(f\) around point \(x\) (see
\eqref{eq:taylor_series} for the general form):

\begin{equation}
\label{eq:taylor_expansion}
f(x + h) = f(x) + \nabla f(x)^T h + \frac{1}{2} h^T \nabla^2 f(x) h + O(\|h\|^3)
\end{equation}

For small \(h\), the dominant term is the linear term
\(\nabla f(x)^T h\).

\begin{block}{Advanced Convergence Analysis}
\protect\phantomsection\label{advanced-convergence-analysis}
The convergence rate for Newton's method is quadratic:

\begin{equation}
\label{eq:newton_convergence}
\|x_{k+1} - x^*\| \leq C \|x_k - x^*\|^2
\end{equation}

where \(C\) depends on the Lipschitz constant of the Hessian.
\end{block}

\begin{block}{Eigenvalue Analysis}
\protect\phantomsection\label{eigenvalue-analysis}
For quadratic forms, the condition number is crucial:

\begin{equation}
\label{eq:condition_number}
\kappa(A) = \frac{\lambda_{\max}}{\lambda_{\min}}
\end{equation}

The convergence factor becomes:

\begin{equation}
\label{eq:spectral_radius}
\rho = \frac{\kappa - 1}{\kappa + 1}
\end{equation}
\end{block}

\begin{block}{Fourier Analysis}
\protect\phantomsection\label{fourier-analysis}
The Fourier transform of a function \(f(t)\) is:

\begin{equation}
\label{eq:fourier_transform}
\hat{f}(\omega) = \int_{-\infty}^{\infty} f(t) e^{-i\omega t} \, dt
\end{equation}

Parseval's theorem states:

\begin{equation}
\label{eq:parseval}
\int_{-\infty}^{\infty} |f(t)|^2 \, dt = \frac{1}{2\pi} \int_{-\infty}^{\infty} |\hat{f}(\omega)|^2 \, d\omega
\end{equation}
\end{block}

\begin{block}{Differential Equations}
\protect\phantomsection\label{differential-equations}
The solution to the first-order linear ODE:

\begin{equation}
\label{eq:differential_equation}
\frac{dy}{dx} + P(x)y = Q(x)
\end{equation}

is given by:

\begin{equation}
\label{eq:differential_solution}
y = e^{-\int P(x) \, dx} \left( \int Q(x) e^{\int P(x) \, dx} \, dx + C \right)
\end{equation}
\end{block}

\begin{block}{Vector Calculus Identities}
\protect\phantomsection\label{vector-calculus-identities}
The divergence theorem (Gauss's theorem):

\begin{equation}
\label{eq:divergence_theorem}
\iiint_V (\nabla \cdot \mathbf{F}) \, dV = \iint_S \mathbf{F} \cdot d\mathbf{S}
\end{equation}

Stokes' theorem:

\begin{equation}
\label{eq:stokes_theorem}
\iint_S (\nabla \times \mathbf{F}) \cdot d\mathbf{S} = \oint_C \mathbf{F} \cdot d\mathbf{r}
\end{equation}
\end{block}

\begin{block}{Complex Analysis}
\protect\phantomsection\label{complex-analysis}
Cauchy's integral theorem states that for analytic function \(f\):

\begin{equation}
\label{eq:cauchy_integral_theorem}
\oint_C f(z) \, dz = 0
\end{equation}

The residue theorem:

\begin{equation}
\label{eq:residue_theorem}
\oint_C f(z) \, dz = 2\pi i \sum \text{Res}(f, a_k)
\end{equation}
\end{block}
\end{block}

\begin{block}{Algorithm Convergence}
\protect\phantomsection\label{algorithm-convergence}
The convergence rate analysis yields:

\begin{equation}
\label{eq:convergence_condition}
\lim_{k \rightarrow \infty} \|\nabla f(x_k)\| = 0 \quad \text{(where $\nabla f$ is defined in \eqref{eq:gradient_definition})}
\end{equation}

with convergence rate depending on the condition number of the Hessian
matrix.
\end{block}

\begin{block}{Key Findings}
\protect\phantomsection\label{key-findings}
Our theoretical analysis reveals several important findings:

\begin{enumerate}
\tightlist
\item
  \textbf{Convergence Properties}

  \begin{itemize}
  \tightlist
  \item
    Linear convergence for strongly convex functions (see Theorem 5,
    \eqref{eq:linear_convergence})
  \item
    Sublinear convergence for general convex functions
  \item
    No convergence guarantee for non-convex functions
  \end{itemize}
\item
  \textbf{Optimal Step Sizes}

  \begin{itemize}
  \tightlist
  \item
    Constant step size:
    \(\alpha = \frac{2}{\lambda_{\min} + \lambda_{\max}}\)
  \item
    Diminishing step size: \(\alpha_k = \frac{\alpha}{k+1}\)
  \item
    Adaptive step size based on function properties
  \end{itemize}
\item
  \textbf{Numerical Stability}

  \begin{itemize}
  \tightlist
  \item
    Condition number affects convergence speed
  \item
    Ill-conditioned problems require preconditioning
  \item
    Gradient computation accuracy impacts final precision
  \end{itemize}
\end{enumerate}
\end{block}

\begin{block}{Comparative Analysis}
\protect\phantomsection\label{comparative-analysis}
{\def\LTcaptype{none} % do not increment counter
\begin{longtable}[]{@{}llll@{}}
\toprule\noalign{}
Method & Convergence Rate & Memory Usage & Implementation Complexity \\
\midrule\noalign{}
\endhead
Gradient Descent & Linear & O(n) & Low \\
Newton Method & Quadratic & O(n²) & High \\
Conjugate Gradient & Superlinear & O(n) & Medium \\
BFGS & Superlinear & O(n²) & High \\
\bottomrule\noalign{}
\end{longtable}
}

Table 1: Comparison of optimization methods showing trade-offs between
convergence speed, memory requirements, and implementation complexity.
\end{block}

\begin{block}{Analysis Results}
\protect\phantomsection\label{analysis-results}
Our mathematical analysis demonstrates the effectiveness of structured
computational approaches to mathematical problems. The analysis pipeline
successfully validates mathematical functions and provides performance
metrics for computational operations.

The results show that:

\begin{enumerate}
\tightlist
\item
  \textbf{Mathematical functions}: Well-designed functions exhibit
  predictable behavior across different input ranges
\item
  \textbf{Computational performance}: Efficient algorithms can process
  mathematical operations with consistent performance characteristics
\item
  \textbf{Numerical stability}: Proper implementation ensures reliable
  results across various computational scenarios
\item
  \textbf{Validation frameworks}: Comprehensive testing validates both
  correctness and performance of mathematical implementations
\end{enumerate}

The analysis demonstrates the importance of rigorous mathematical
validation in computational research.
\end{block}

\begin{block}{Discussion}
\protect\phantomsection\label{discussion}
The results demonstrate that:

\begin{itemize}
\tightlist
\item
  \textbf{Theoretical guarantees} exist for convex optimization problems
\item
  \textbf{Practical performance} depends on problem conditioning
\item
  \textbf{Algorithm selection} should balance convergence speed with
  computational cost, including consideration of interior-point methods
  \cite{nesterov1994interior} and linear programming techniques
  \cite{luenberger1984linear}
\item
  \textbf{Numerical considerations} are crucial for reliable
  implementation
\item
  \textbf{Mathematical visualization} provides valuable insights into
  algorithmic behavior
\end{itemize}
\end{block}

\begin{block}{Future Directions}
\protect\phantomsection\label{future-directions}
Several avenues for future research include:

\begin{itemize}
\tightlist
\item
  Extension to constrained optimization problems
\item
  Development of adaptive step size strategies
\item
  Analysis of stochastic gradient variants
\item
  Application to large-scale machine learning problems
\end{itemize}
\end{block}
\end{frame}

\end{document}
