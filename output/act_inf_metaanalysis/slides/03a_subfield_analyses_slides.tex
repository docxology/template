% Options for packages loaded elsewhere
\PassOptionsToPackage{unicode}{hyperref}
\PassOptionsToPackage{hyphens}{url}
\documentclass[
  ignorenonframetext,
]{beamer}
\newif\ifbibliography
\usepackage{pgfpages}
\setbeamertemplate{caption}[numbered]
\setbeamertemplate{caption label separator}{: }
\setbeamercolor{caption name}{fg=normal text.fg}
\beamertemplatenavigationsymbolsempty
% remove section numbering
\setbeamertemplate{part page}{
  \centering
  \begin{beamercolorbox}[sep=16pt,center]{part title}
    \usebeamerfont{part title}\insertpart\par
  \end{beamercolorbox}
}
\setbeamertemplate{section page}{
  \centering
  \begin{beamercolorbox}[sep=12pt,center]{section title}
    \usebeamerfont{section title}\insertsection\par
  \end{beamercolorbox}
}
\setbeamertemplate{subsection page}{
  \centering
  \begin{beamercolorbox}[sep=8pt,center]{subsection title}
    \usebeamerfont{subsection title}\insertsubsection\par
  \end{beamercolorbox}
}
% Prevent slide breaks in the middle of a paragraph
\widowpenalties 1 10000
\raggedbottom
\AtBeginPart{
  \frame{\partpage}
}
\AtBeginSection{
  \ifbibliography
  \else
    \frame{\sectionpage}
  \fi
}
\AtBeginSubsection{
  \frame{\subsectionpage}
}
\usepackage{iftex}
\ifPDFTeX
  \usepackage[T1]{fontenc}
  \usepackage[utf8]{inputenc}
  \usepackage{textcomp} % provide euro and other symbols
\else % if luatex or xetex
  \usepackage{unicode-math} % this also loads fontspec
  \defaultfontfeatures{Scale=MatchLowercase}
  \defaultfontfeatures[\rmfamily]{Ligatures=TeX,Scale=1}
\fi
\usepackage{lmodern}
\ifPDFTeX\else
  % xetex/luatex font selection
\fi
% Use upquote if available, for straight quotes in verbatim environments
\IfFileExists{upquote.sty}{\usepackage{upquote}}{}
\IfFileExists{microtype.sty}{% use microtype if available
  \usepackage[]{microtype}
  \UseMicrotypeSet[protrusion]{basicmath} % disable protrusion for tt fonts
}{}
\makeatletter
\@ifundefined{KOMAClassName}{% if non-KOMA class
  \IfFileExists{parskip.sty}{%
    \usepackage{parskip}
  }{% else
    \setlength{\parindent}{0pt}
    \setlength{\parskip}{6pt plus 2pt minus 1pt}}
}{% if KOMA class
  \KOMAoptions{parskip=half}}
\makeatother
\usepackage{longtable,booktabs,array}
\newcounter{none} % for unnumbered tables
\usepackage{calc} % for calculating minipage widths
\usepackage{caption}
% Make caption package work with longtable
\makeatletter
\def\fnum@table{\tablename~\thetable}
\makeatother
\setlength{\emergencystretch}{3em} % prevent overfull lines
\providecommand{\tightlist}{%
  \setlength{\itemsep}{0pt}\setlength{\parskip}{0pt}}
\usepackage{bookmark}
\IfFileExists{xurl.sty}{\usepackage{xurl}}{} % add URL line breaks if available
\urlstyle{same}
\hypersetup{
  hidelinks,
  pdfcreator={LaTeX via pandoc}}

\author{\texorpdfstring{}{}}
\date{}

\begin{document}

\begin{frame}{Domain Analyses: Growth Trajectories and Open Problems
\label{sec:subfield_analyses}}
\protect\phantomsection\label{domain-analyses-growth-trajectories-and-open-problems}
\emph{This supplementary section provides detailed characterizations of
each of the eight tracked Active Inference domains, organized under
three tiers: A (Core Theory), B (Tools \& Translation), and C
(Application Domains).}

\begin{block}{Domain A: Core Theory}
\protect\phantomsection\label{domain-a-core-theory}
\begin{block}{A1 --- Quantitative \& Formal Theory (\(n = 120\), 9.9\%)}
\protect\phantomsection\label{a1-quantitative-formal-theory-n-120-9.9}
The A1 domain develops the mathematical foundations underpinning the
Free Energy Principle: information geometry, category-theoretic
formulations of Markov blankets, path integral formulations of free
energy minimization, and gauge-theoretic perspectives on
self-organization. A central debate concerns the ontological status of
Markov blankets---whether they correspond to real physical boundaries or
are merely useful statistical constructs \citep{bruineberg2022emperor}.
Recent work on Bayesian mechanics \citep{sakthivadivel2023bayesian} aims
to place the FEP on firmer mathematical footing. With 120 papers, A1
captures nearly 10\% of the corpus, reflecting the improved classifier's
ability to route papers with mathematical formalism (theorems, proofs,
convergence, posterior distributions, Fokker--Planck equations) into
this domain rather than the qualitative philosophy catch-all.
\end{block}

\begin{block}{A2 --- Qualitative Philosophy \& General Theory
(\(n = 154\), 12.7\%)}
\protect\phantomsection\label{a2-qualitative-philosophy-general-theory-n-154-12.7}
The A2 domain encompasses papers that develop, extend, or review the
core Free Energy Principle and Active Inference framework without
restricting attention to a specific application domain. This includes
Friston's foundational work on variational free energy minimization
\citep{friston2010free}, the textbook treatment by Parr, Pezzulo, and
Friston \citep{parr2022active}, and numerous tutorial and review papers.
The priority-based classifier mitigates over-assignment to A2 by routing
papers with mathematical formalism to A1 and papers with domain-specific
vocabulary to C1--C5 or B before the A2 catch-all is reached.
Nevertheless, the count likely still conceals meaningful internal
structure: papers addressing embodied cognition, Bayesian brain theory,
and philosophical implications of the FEP are all subsumed under this
heading. Key ongoing debates concern the explanatory scope of the
FEP---whether it is a principle of physics, biology, or cognition---and
the relationship between active inference and competing frameworks such
as reinforcement learning and optimal control theory.
\end{block}
\end{block}

\begin{block}{Domain B: Tools \& Translation Methods}
\protect\phantomsection\label{domain-b-tools-translation-methods}
\begin{block}{B --- Algorithms, Scaling, and Software (\(n = 267\),
22.1\%)}
\protect\phantomsection\label{b-algorithms-scaling-and-software-n-267-22.1}
Domain B addresses the computational challenge of making active
inference practical in complex, high-dimensional environments. Early
implementations relied on small discrete state spaces amenable to exact
message passing. Recent work has introduced deep active inference using
neural networks to amortize inference \citep{fountas2020deep}, Monte
Carlo tree search for planning \citep{champion2021realizing}, and hybrid
architectures combining model-based planning with model-free components.
The central open question is whether active inference agents can match
deep reinforcement learning performance on standard benchmarks while
retaining interpretability and sample efficiency. The availability of
the pymdp library \citep{heins2022pymdp} has lowered implementation
barriers, contributing to this domain's growth. The recent establishment
of the Pymdp Fellowship program in 2025 and the release of real-time
stream processing tools like RxInfer.jl v4.0.0 \citep{rxinfer2025}
indicate a vibrant and maturing software ecosystem.
\end{block}
\end{block}

\begin{block}{Domain C: Application Domains}
\protect\phantomsection\label{domain-c-application-domains}
\begin{block}{C1 --- Neuroscience (\(n = 206\), 17.1\%)}
\protect\phantomsection\label{c1-neuroscience-n-206-17.1}
Neuroscience represents the historical core of the Active Inference
research program. The predictive processing account---in which cortical
hierarchies minimize prediction errors through both perceptual inference
and active sampling---remains one of the most empirically tested aspects
of the framework \citep{friston2010free, clark2013whatever}. The broader
neuroscience literature on Dynamic Causal Modeling and predictive coding
is extensive; the relatively modest count here likely reflects the
keyword classifier's inability to distinguish neuroscience-specific
applications from general FEP theory. Bridging the gap between
computational models and empirical neuroimaging data remains the
domain's primary challenge.
\end{block}

\begin{block}{C2 --- Robotics (\(n = 170\), 14.1\%)}
\protect\phantomsection\label{c2-robotics-n-170-14.1}
Robotics applications treat embodied agents as free energy minimizing
systems that unify perception and action through proprioceptive and
exteroceptive prediction errors \citep{lanillos2021active}. Applications
include robotic arm control, mobile navigation, manipulation, and
multi-robot coordination. Active inference offers roboticists a
principled framework for integrating sensory processing, motor planning,
and adaptive behavior without separate perception and control modules.
Key challenges include real-time computational feasibility on embedded
hardware, continuous high-dimensional action spaces, and sim-to-real
transfer.
\end{block}

\begin{block}{C3 --- Language Processing (\(n = 57\), 4.7\%)}
\protect\phantomsection\label{c3-language-processing-n-57-4.7}
The C3 domain formally conceptualizes linguistic processes---speech
perception, sentence comprehension, sequential dialogue, and
reading---as active inference operating over deep hierarchical
generative models of linguistic structure \citep{friston2020generative}.
Active inference models of reading have deterministically accounted for
saccadic eye-movement patterns, while models of speech perception
mathematically explain how human listeners integrate topological prior
expectations with continuous acoustic evidence. Recent breakthroughs
tightly couple active inference to large language models, pragmatics,
and multi-agent communication. Notably, recent literature has
conceptualized LLMs themselves as atypical active inference agents,
introducing frameworks that deploy active inference as a metacognitive
governor to enable adaptive, self-evolving LLM behavior
\citep{heins2024active}.
\end{block}

\begin{block}{C4 --- Computational Psychiatry (\(n = 34\), 2.8\%)}
\protect\phantomsection\label{c4-computational-psychiatry-n-34-2.8}
Computational psychiatry aggressively leverages active inference to
natively model psychiatric conditions as structural aberrations in
belief updating, precision weighting, or prior expectation rigidity
\citep{smith2021computational}. Schizophrenia has been modeled as a
critical failure of precision weighting on bottom-up prediction errors;
clinical depression corresponds to excessively precise, inescapable
negative priors; and autism spectrum profiles as atypical sensory
precision allocation. The domain continues to expand rapidly: 2025
frameworks such as Active Intersubjective Inference (AISI) seamlessly
integrate psychodynamic theory (e.g., self-identity formation via
embodied interactions) with predictive processing algorithms to
mathematically unify the environmental and biological factors underlying
stress disorders \citep{smith2025active}. Translating these expanding
computational models into scalable diagnostic markers and therapeutic
real-world protocols remains an urgent, ongoing objective.
\end{block}

\begin{block}{C5 --- Biology \& Morphogenesis (\(n = 200\), 16.6\%)}
\protect\phantomsection\label{c5-biology-morphogenesis-n-200-16.6}
The C5 domain applies active inference and the FEP to biological systems
beyond the brain: cellular behavior, morphogenesis, evolutionary
dynamics, and the origins of life. Morphogenetic processes have been
modeled as collective active inference, where groups of cells coordinate
to minimize a shared free energy functional
\citep{kuchling2020morphogenesis, levin2022technological}. Recent models
(e.g., MorphoNAS) demonstrate how simple rules derived from the FEP
drive ``neuromorphic development,'' steering systems with morphological
degrees of freedom to independently self-organize the complex neural
computing topologies fundamental to bioengineering
\citep{levin2025morphonas}. As the second-largest domain, C5 reflects
growing interest in extending the FEP to encompass all living systems,
though the ratio of theoretical proposals to empirical validation
remains high.
\end{block}
\end{block}

\begin{block}{Comparative Synthesis}
\protect\phantomsection\label{comparative-synthesis}
Taken together, the three domains reveal a field in transition from a
focused neuroscience program to a broad interdisciplinary framework. The
core--periphery structure is clear: Domain A provides the theoretical
and mathematical substrate, Domain B pursues engineering viability
through scalable algorithms and software, and Domain C tests the
framework's generality across neuroscience (C1), robotics (C2), language
(C3), psychiatry (C4), and biology (C5). The consistent pattern across
applied domains---strong theoretical motivation paired with limited
empirical validation---suggests that the field's next phase of growth
will be determined less by new theory than by the accumulation of
decisive experimental evidence.

In direct response to \textbf{RQ1} (How is the Active Inference field
structured?), the domain taxonomy reveals an asymmetric three-tier
architecture: a dominant theoretical core (A), a growing translational
layer (B), and an expanding but empirically sparse application periphery
(C). The keyword classifier's heavy A2 concentration likely masks
genuine diversity within the theoretical core, but the architecture
itself---theory → tools → applications---is robust across classification
approaches.

\begin{block}{Domain--Hypothesis Cross-Reference}
\protect\phantomsection\label{domainhypothesis-cross-reference}
Each domain has a primary hypothesis linkage (see the detailed
hypothesis evidence analysis in
\hyperref[sec:hypothesis_results]{Section 4b}):

{\def\LTcaptype{none} % do not increment counter
\begin{longtable}[]{@{}lllll@{}}
\toprule\noalign{}
Domain & Category & \(n\) & Primary Hypothesis & Evidence Direction \\
\midrule\noalign{}
\endhead
A1 & Formal & 120 & H3 Markov Blanket Realism & Contested \\
A2 & Philosophy & 154 & H1 FEP Universality & Strongly supporting \\
B & Tools & 267 & H5 Scalability & Mixed \\
C1 & Neuroscience & 206 & H4 Predictive Coding & Supporting \\
C2 & Robotics & 170 & H2 AIF Optimality, H5 Scalability & Mixed \\
C3 & Language & 57 & H8 Language AIF & Emerging \\
C4 & Psychiatry & 34 & H6 Clinical Utility & Supporting \\
C5 & Biology & 200 & H7 Morphogenesis & Supporting \\
\bottomrule\noalign{}
\end{longtable}
}

The evidence directions summarized above are elaborated
quantitatively---with citation-weighted scores, temporal trends, and
three-tier evidence profiling---in the hypothesis results section (see
\hyperref[sec:hypothesis_results]{Section 4b}).
\end{block}
\end{block}
\end{frame}

\end{document}
