% Options for packages loaded elsewhere
\PassOptionsToPackage{unicode}{hyperref}
\PassOptionsToPackage{hyphens}{url}
\documentclass[
  ignorenonframetext,
]{beamer}
\newif\ifbibliography
\usepackage{pgfpages}
\setbeamertemplate{caption}[numbered]
\setbeamertemplate{caption label separator}{: }
\setbeamercolor{caption name}{fg=normal text.fg}
\beamertemplatenavigationsymbolsempty
% remove section numbering
\setbeamertemplate{part page}{
  \centering
  \begin{beamercolorbox}[sep=16pt,center]{part title}
    \usebeamerfont{part title}\insertpart\par
  \end{beamercolorbox}
}
\setbeamertemplate{section page}{
  \centering
  \begin{beamercolorbox}[sep=12pt,center]{section title}
    \usebeamerfont{section title}\insertsection\par
  \end{beamercolorbox}
}
\setbeamertemplate{subsection page}{
  \centering
  \begin{beamercolorbox}[sep=8pt,center]{subsection title}
    \usebeamerfont{subsection title}\insertsubsection\par
  \end{beamercolorbox}
}
% Prevent slide breaks in the middle of a paragraph
\widowpenalties 1 10000
\raggedbottom
\AtBeginPart{
  \frame{\partpage}
}
\AtBeginSection{
  \ifbibliography
  \else
    \frame{\sectionpage}
  \fi
}
\AtBeginSubsection{
  \frame{\subsectionpage}
}
\usepackage{iftex}
\ifPDFTeX
  \usepackage[T1]{fontenc}
  \usepackage[utf8]{inputenc}
  \usepackage{textcomp} % provide euro and other symbols
\else % if luatex or xetex
  \usepackage{unicode-math} % this also loads fontspec
  \defaultfontfeatures{Scale=MatchLowercase}
  \defaultfontfeatures[\rmfamily]{Ligatures=TeX,Scale=1}
\fi
\usepackage{lmodern}
\ifPDFTeX\else
  % xetex/luatex font selection
\fi
% Use upquote if available, for straight quotes in verbatim environments
\IfFileExists{upquote.sty}{\usepackage{upquote}}{}
\IfFileExists{microtype.sty}{% use microtype if available
  \usepackage[]{microtype}
  \UseMicrotypeSet[protrusion]{basicmath} % disable protrusion for tt fonts
}{}
\makeatletter
\@ifundefined{KOMAClassName}{% if non-KOMA class
  \IfFileExists{parskip.sty}{%
    \usepackage{parskip}
  }{% else
    \setlength{\parindent}{0pt}
    \setlength{\parskip}{6pt plus 2pt minus 1pt}}
}{% if KOMA class
  \KOMAoptions{parskip=half}}
\makeatother
\setlength{\emergencystretch}{3em} % prevent overfull lines
\providecommand{\tightlist}{%
  \setlength{\itemsep}{0pt}\setlength{\parskip}{0pt}}
\usepackage{bookmark}
\IfFileExists{xurl.sty}{\usepackage{xurl}}{} % add URL line breaks if available
\urlstyle{same}
\hypersetup{
  hidelinks,
  pdfcreator={LaTeX via pandoc}}

\author{\texorpdfstring{}{}}
\date{}

\begin{document}

\begin{frame}{Abstract}
\protect\phantomsection\label{abstract}
The Free Energy Principle (FEP) and Active Inference have expanded
rapidly across neuroscience, robotics, biology, and formal mathematics.
However, the field lacks systematic methods for tracking which of its
central theoretical claims are well-supported, contested, or merely
assumed. Building on the systematic literature analysis of Knight,
Cordes, and Friedman \citep{knight2022fep}---which pioneered manual
annotation paired with ontology-based analysis at the scale of hundreds
of papers---we present a computational meta-analysis framework that
automates and scales this approach. Our pipeline retrieves literature
from arXiv, Semantic Scholar, and OpenAlex, deduplicating records via a
canonical identifier hierarchy. It classifies papers into a three-tier
taxonomy spanning eight categories: A (Core Theory), B (Tools \&
Translation), and C (Application Domains). To transcend keyword
matching, an LLM-powered extraction system evaluates each abstract
against eight core hypotheses, producing structured nanopublications
with directionality, confidence scores, and natural-language reasoning.
These nanopublications populate an RDF-compatible knowledge graph
evaluated by a citation-weighted evidence scoring function.

Applied to a corpus of \(N = 1208\) papers (spanning 1972--2026), the
framework details a field dominated by core theory (Domain A) but
actively diversifying into tools development (Domain B) and specific
applications (Domain C), notably neuroscience, robotics, and
computational psychiatry. Non-negative matrix factorization identifies
five latent topics that cross-cut the keyword domain taxonomy, while
citation network analysis reveals a sparse yet structured graph
(2\{,\}780 intra-corpus edges, 6.1\% reference resolution) anchored by
pronounced hub papers. By demonstrating that automated LLM-driven
assertion extraction can generate scalable, queryable representations of
scientific evidence, this work provides a robust architectural
foundation for \emph{living literature reviews}---continuously updated
knowledge graphs that track the trajectory of theoretical consensus
across rapidly evolving fields, within Active Inference and beyond.

\textbf{Keywords:} Active Inference, Free Energy Principle,
meta-analysis, knowledge graph, nanopublications, bibliometrics,
hypothesis scoring, LLM extraction, computational neuroscience
\end{frame}

\end{document}
