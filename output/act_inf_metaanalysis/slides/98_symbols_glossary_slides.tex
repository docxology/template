% Options for packages loaded elsewhere
\PassOptionsToPackage{unicode}{hyperref}
\PassOptionsToPackage{hyphens}{url}
\documentclass[
  ignorenonframetext,
]{beamer}
\newif\ifbibliography
\usepackage{pgfpages}
\setbeamertemplate{caption}[numbered]
\setbeamertemplate{caption label separator}{: }
\setbeamercolor{caption name}{fg=normal text.fg}
\beamertemplatenavigationsymbolsempty
% remove section numbering
\setbeamertemplate{part page}{
  \centering
  \begin{beamercolorbox}[sep=16pt,center]{part title}
    \usebeamerfont{part title}\insertpart\par
  \end{beamercolorbox}
}
\setbeamertemplate{section page}{
  \centering
  \begin{beamercolorbox}[sep=12pt,center]{section title}
    \usebeamerfont{section title}\insertsection\par
  \end{beamercolorbox}
}
\setbeamertemplate{subsection page}{
  \centering
  \begin{beamercolorbox}[sep=8pt,center]{subsection title}
    \usebeamerfont{subsection title}\insertsubsection\par
  \end{beamercolorbox}
}
% Prevent slide breaks in the middle of a paragraph
\widowpenalties 1 10000
\raggedbottom
\AtBeginPart{
  \frame{\partpage}
}
\AtBeginSection{
  \ifbibliography
  \else
    \frame{\sectionpage}
  \fi
}
\AtBeginSubsection{
  \frame{\subsectionpage}
}
\usepackage{iftex}
\ifPDFTeX
  \usepackage[T1]{fontenc}
  \usepackage[utf8]{inputenc}
  \usepackage{textcomp} % provide euro and other symbols
\else % if luatex or xetex
  \usepackage{unicode-math} % this also loads fontspec
  \defaultfontfeatures{Scale=MatchLowercase}
  \defaultfontfeatures[\rmfamily]{Ligatures=TeX,Scale=1}
\fi
\usepackage{lmodern}
\ifPDFTeX\else
  % xetex/luatex font selection
\fi
% Use upquote if available, for straight quotes in verbatim environments
\IfFileExists{upquote.sty}{\usepackage{upquote}}{}
\IfFileExists{microtype.sty}{% use microtype if available
  \usepackage[]{microtype}
  \UseMicrotypeSet[protrusion]{basicmath} % disable protrusion for tt fonts
}{}
\makeatletter
\@ifundefined{KOMAClassName}{% if non-KOMA class
  \IfFileExists{parskip.sty}{%
    \usepackage{parskip}
  }{% else
    \setlength{\parindent}{0pt}
    \setlength{\parskip}{6pt plus 2pt minus 1pt}}
}{% if KOMA class
  \KOMAoptions{parskip=half}}
\makeatother
\usepackage{longtable,booktabs,array}
\newcounter{none} % for unnumbered tables
\usepackage{calc} % for calculating minipage widths
\usepackage{caption}
% Make caption package work with longtable
\makeatletter
\def\fnum@table{\tablename~\thetable}
\makeatother
\setlength{\emergencystretch}{3em} % prevent overfull lines
\providecommand{\tightlist}{%
  \setlength{\itemsep}{0pt}\setlength{\parskip}{0pt}}
\usepackage{bookmark}
\IfFileExists{xurl.sty}{\usepackage{xurl}}{} % add URL line breaks if available
\urlstyle{same}
\hypersetup{
  hidelinks,
  pdfcreator={LaTeX via pandoc}}

\author{\texorpdfstring{}{}}
\date{}

\begin{document}

\section{Notation, Abbreviations, and Hypothesis
Definitions}\label{notation-abbreviations-and-hypothesis-definitions}

\begin{frame}{Mathematical Symbols and Notation}
\protect\phantomsection\label{mathematical-symbols-and-notation}
{\def\LTcaptype{none} % do not increment counter
\begin{longtable}[]{@{}
  >{\raggedright\arraybackslash}p{(\linewidth - 2\tabcolsep) * \real{0.5000}}
  >{\raggedright\arraybackslash}p{(\linewidth - 2\tabcolsep) * \real{0.5000}}@{}}
\toprule\noalign{}
\begin{minipage}[b]{\linewidth}\raggedright
Symbol
\end{minipage} & \begin{minipage}[b]{\linewidth}\raggedright
Description
\end{minipage} \\
\midrule\noalign{}
\endhead
\(\mathcal{F}\) & Variational free energy \\
\(\mathbf{F}\) & Expected free energy (for policy selection) \\
\(D_{\mathrm{KL}}\) & Kullback--Leibler divergence \\
\(q(\cdot)\) & Approximate posterior (recognition density) \\
\(p(\cdot)\) & Generative model (prior and likelihood) \\
\(\mathbf{s}\) & Hidden states \\
\(\mathbf{o}\) & Observations \\
\(\pi\) & Policy (sequence of actions) \\
\(\mathbf{A}\) & Likelihood mapping (observation model) \\
\(\mathbf{B}\) & Transition model (state dynamics) \\
\(\mathbf{C}\) & Prior preferences over observations \\
\(\mathbf{D}\) & Prior over initial states \\
\(N\) & Corpus size (total deduplicated papers) \\
\(n\) & Subfield paper count \\
\(T\) & Time span in years (for CAGR computation) \\
\(N_{\text{start}}\) & Publication count in the first year of the
corpus \\
\(N_{\text{end}}\) & Publication count in the last year of the corpus \\
\(w(a)\) & Citation-weighted assertion score:
\(\log(1 + \text{citations}) \cdot \text{confidence}\) \\
\(\text{score}(H)\) & Aggregate evidence score for hypothesis \(H\),
range \([-1, 1]\) \\
\(S(H)\) & Set of supporting assertions for hypothesis \(H\) \\
\(C(H)\) & Set of contradicting assertions for hypothesis \(H\) \\
\(A(H)\) & Set of all assertions for hypothesis \(H\) \\
\(c\) & Assertion confidence, range \([0, 1]\) \\
\(d\) & Assertion direction: supports, contradicts, or neutral \\
\(\mathbf{V}\) & Document-term matrix (NMF input) \\
\(\mathbf{W}\) & Document-topic matrix (NMF factor) \\
\(\mathbf{H}\) & Topic-term matrix (NMF factor) \\
\(k\) & Number of latent topics \\
\(\epsilon\) & Numerical stability constant (\(10^{-10}\)) \\
\(\text{CAGR}\) & Compound annual growth rate \\
\(t_d\) & Publication doubling time \\
\(\bar{g}\) & Mean annual year-over-year growth rate \\
\(\kappa\) & Cohen's kappa (inter-annotator agreement) \\
\bottomrule\noalign{}
\end{longtable}
}
\end{frame}

\begin{frame}{Abbreviations and Acronyms Used}
\protect\phantomsection\label{abbreviations-and-acronyms-used}
{\def\LTcaptype{none} % do not increment counter
\begin{longtable}[]{@{}ll@{}}
\toprule\noalign{}
Abbreviation & Definition \\
\midrule\noalign{}
\endhead
AIF & Active Inference \\
API & Application Programming Interface \\
CAGR & Compound Annual Growth Rate \\
CI & Confidence Interval \\
DCM & Dynamic Causal Modelling \\
DOI & Digital Object Identifier \\
DPI & Dots Per Inch (figure resolution) \\
EEG & Electroencephalography \\
EFE & Expected Free Energy \\
ERP & Event-Related Potential \\
FEP & Free Energy Principle \\
fMRI & Functional Magnetic Resonance Imaging \\
GML & Graph Modelling Language (network serialization format) \\
JSON & JavaScript Object Notation \\
JSONL & JSON Lines (newline-delimited JSON) \\
KG & Knowledge Graph \\
KL & Kullback--Leibler (divergence) \\
LLM & Large Language Model \\
NMF & Non-negative Matrix Factorization \\
NLP & Natural Language Processing \\
ORCID & Open Researcher and Contributor ID \\
OWL & Web Ontology Language \\
PCA & Principal Component Analysis \\
POMDP & Partially Observable Markov Decision Process \\
RDF & Resource Description Framework \\
RL & Reinforcement Learning \\
RNG & Random Number Generator \\
SPARQL & SPARQL Protocol and RDF Query Language \\
SPM & Statistical Parametric Mapping \\
TF-IDF & Term Frequency--Inverse Document Frequency \\
URI & Uniform Resource Identifier \\
YAML & YAML Ain't Markup Language (configuration format) \\
YoY & Year-over-Year \\
\bottomrule\noalign{}
\end{longtable}
}
\end{frame}

\begin{frame}{Standard Hypothesis Definitions and Identifiers}
\protect\phantomsection\label{standard-hypothesis-definitions-and-identifiers}
{\def\LTcaptype{none} % do not increment counter
\begin{longtable}[]{@{}
  >{\raggedright\arraybackslash}p{(\linewidth - 4\tabcolsep) * \real{0.3333}}
  >{\raggedright\arraybackslash}p{(\linewidth - 4\tabcolsep) * \real{0.3333}}
  >{\raggedright\arraybackslash}p{(\linewidth - 4\tabcolsep) * \real{0.3333}}@{}}
\toprule\noalign{}
\begin{minipage}[b]{\linewidth}\raggedright
ID
\end{minipage} & \begin{minipage}[b]{\linewidth}\raggedright
Hypothesis
\end{minipage} & \begin{minipage}[b]{\linewidth}\raggedright
Scope
\end{minipage} \\
\midrule\noalign{}
\endhead
H1 & FEP Universality: The Free Energy Principle applies universally to
all self-organizing systems & A (Core Theory) \\
H2 & AIF Optimality: Active Inference agents achieve optimal
decision-making under uncertainty & B (Tools) \\
H3 & Markov Blanket Realism: Markov blankets correspond to real physical
boundaries & A (Core Theory) \\
H4 & Predictive Coding: Cortical hierarchies minimize prediction errors
via predictive coding & C1 (Neuroscience) \\
H5 & Scalability: Active Inference scales to complex, high-dimensional
environments & B (Tools) \\
H6 & Clinical Utility: Active Inference provides clinically useful
models of psychiatric conditions & C4 (Psychiatry) \\
H7 & Morphogenesis: The FEP explains morphogenetic and developmental
processes & C5 (Biology) \\
H8 & Language AIF: Active Inference provides a viable framework for
language processing & C3 (Language) \\
\bottomrule\noalign{}
\end{longtable}
}
\end{frame}

\begin{frame}[fragile]{Glossary of Key Terms}
\protect\phantomsection\label{glossary-of-key-terms}
{\def\LTcaptype{none} % do not increment counter
\begin{longtable}[]{@{}
  >{\raggedright\arraybackslash}p{(\linewidth - 2\tabcolsep) * \real{0.5000}}
  >{\raggedright\arraybackslash}p{(\linewidth - 2\tabcolsep) * \real{0.5000}}@{}}
\toprule\noalign{}
\begin{minipage}[b]{\linewidth}\raggedright
Term
\end{minipage} & \begin{minipage}[b]{\linewidth}\raggedright
Definition
\end{minipage} \\
\midrule\noalign{}
\endhead
\textbf{Active Inference} & A framework in which agents minimize
expected free energy to select actions, unifying perception, learning,
and decision-making under the Free Energy Principle. \\
\textbf{Assertion} & A directed, confidence-scored claim linking a paper
to a hypothesis (supports, contradicts, or neutral). The basic unit of
evidence in the knowledge graph. \\
\textbf{Canonical ID} & The unique identifier assigned to each paper
during deduplication, following the priority scheme: DOI \textgreater{}
arXiv ID \textgreater{} Semantic Scholar ID \textgreater{} OpenAlex ID
\textgreater{} title hash. \\
\textbf{Expected Free Energy} & A quantity combining epistemic value
(information gain) and pragmatic value (goal achievement) that active
inference agents minimize over policies. \\
\textbf{Free Energy Principle} & The principle that self-organizing
systems minimize variational free energy, an upper bound on surprise, to
maintain their structural integrity. \\
\textbf{Generative Model} & A probabilistic model specifying the joint
distribution over hidden states and observations, encoding an agent's
beliefs about how observations are generated. \\
\textbf{Knowledge Graph} & A directed graph encoding papers, assertions,
hypotheses, and their relationships, serialized in an RDF-compatible
format. \\
\textbf{Markov Blanket} & A statistical boundary separating internal
states from external states, defined as the set of nodes that renders a
system conditionally independent of its environment. \\
\textbf{Nanopublication} & A minimal, self-contained unit of publishable
knowledge consisting of an assertion, provenance metadata, and
publication context. \\
\textbf{Precision} & The inverse variance of a probability distribution;
in active inference, precision weighting determines the influence of
prediction errors at different levels of a hierarchy. \\
\textbf{Variational Free Energy} & An upper bound on surprise (negative
log-evidence) that can be decomposed into complexity (KL divergence from
prior) and accuracy (expected log-likelihood). \\
\textbf{Louvain Algorithm} & A greedy modularity-maximization algorithm
for community detection in networks. Applied to the citation graph to
identify clusters of densely interconnected papers. \\
\textbf{PageRank} & A centrality metric originally designed for web page
ranking. In citation networks, PageRank identifies highly influential
papers that serve as hubs connecting otherwise disconnected
subgraphs. \\
\textbf{Ward Linkage} & A hierarchical clustering method that minimizes
the total within-cluster variance at each merge step. Used to compute
dendrograms of domain centroids from mean TF-IDF vectors. \\
\textbf{Checkpoint} & A JSON Lines snapshot of LLM extraction progress,
recording which papers have been processed and the resulting assertions,
enabling incremental resume after interruption. \\
\textbf{Incremental Resume} & The pipeline's ability to continue from
where a previous run stopped, loading existing corpus/assertions and
processing only new papers, controlled by \texttt{-\/-clear-corpus} and
\texttt{-\/-clear-assertions} CLI flags. \\
\textbf{LLM Config} & A configuration object specifying the Ollama model
name, API URL, temperature, maximum retries, and retry delay for
LLM-based assertion extraction. \\
\textbf{Domain Timeline} & Per-domain yearly publication counts showing
temporal evolution of research activity across the eight tracked
categories (A1--A2, B, C1--C5). \\
\textbf{Progressive Parsing} & The pipeline's multi-stage JSON recovery
strategy for handling malformed LLM output: direct parse → strip code
fences → extract first JSON array → individual element recovery. \\
\textbf{Wong Palette} & The colorblind-safe 8-color palette from Wong
(2011), used as the standard visualization palette throughout all
pipeline-generated figures. \\
\bottomrule\noalign{}
\end{longtable}
}
\end{frame}

\end{document}
