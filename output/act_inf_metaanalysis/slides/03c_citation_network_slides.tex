% Options for packages loaded elsewhere
\PassOptionsToPackage{unicode}{hyperref}
\PassOptionsToPackage{hyphens}{url}
\documentclass[
  ignorenonframetext,
]{beamer}
\newif\ifbibliography
\usepackage{pgfpages}
\setbeamertemplate{caption}[numbered]
\setbeamertemplate{caption label separator}{: }
\setbeamercolor{caption name}{fg=normal text.fg}
\beamertemplatenavigationsymbolsempty
% remove section numbering
\setbeamertemplate{part page}{
  \centering
  \begin{beamercolorbox}[sep=16pt,center]{part title}
    \usebeamerfont{part title}\insertpart\par
  \end{beamercolorbox}
}
\setbeamertemplate{section page}{
  \centering
  \begin{beamercolorbox}[sep=12pt,center]{section title}
    \usebeamerfont{section title}\insertsection\par
  \end{beamercolorbox}
}
\setbeamertemplate{subsection page}{
  \centering
  \begin{beamercolorbox}[sep=8pt,center]{subsection title}
    \usebeamerfont{subsection title}\insertsubsection\par
  \end{beamercolorbox}
}
% Prevent slide breaks in the middle of a paragraph
\widowpenalties 1 10000
\raggedbottom
\AtBeginPart{
  \frame{\partpage}
}
\AtBeginSection{
  \ifbibliography
  \else
    \frame{\sectionpage}
  \fi
}
\AtBeginSubsection{
  \frame{\subsectionpage}
}
\usepackage{iftex}
\ifPDFTeX
  \usepackage[T1]{fontenc}
  \usepackage[utf8]{inputenc}
  \usepackage{textcomp} % provide euro and other symbols
\else % if luatex or xetex
  \usepackage{unicode-math} % this also loads fontspec
  \defaultfontfeatures{Scale=MatchLowercase}
  \defaultfontfeatures[\rmfamily]{Ligatures=TeX,Scale=1}
\fi
\usepackage{lmodern}
\ifPDFTeX\else
  % xetex/luatex font selection
\fi
% Use upquote if available, for straight quotes in verbatim environments
\IfFileExists{upquote.sty}{\usepackage{upquote}}{}
\IfFileExists{microtype.sty}{% use microtype if available
  \usepackage[]{microtype}
  \UseMicrotypeSet[protrusion]{basicmath} % disable protrusion for tt fonts
}{}
\makeatletter
\@ifundefined{KOMAClassName}{% if non-KOMA class
  \IfFileExists{parskip.sty}{%
    \usepackage{parskip}
  }{% else
    \setlength{\parindent}{0pt}
    \setlength{\parskip}{6pt plus 2pt minus 1pt}}
}{% if KOMA class
  \KOMAoptions{parskip=half}}
\makeatother
\usepackage{longtable,booktabs,array}
\newcounter{none} % for unnumbered tables
\usepackage{calc} % for calculating minipage widths
\usepackage{caption}
% Make caption package work with longtable
\makeatletter
\def\fnum@table{\tablename~\thetable}
\makeatother
\setlength{\emergencystretch}{3em} % prevent overfull lines
\providecommand{\tightlist}{%
  \setlength{\itemsep}{0pt}\setlength{\parskip}{0pt}}
\usepackage{bookmark}
\IfFileExists{xurl.sty}{\usepackage{xurl}}{} % add URL line breaks if available
\urlstyle{same}
\hypersetup{
  hidelinks,
  pdfcreator={LaTeX via pandoc}}

\author{\texorpdfstring{}{}}
\date{}

\begin{document}

\begin{frame}{Citation Network Topology \label{sec:citation_network}}
\protect\phantomsection\label{citation-network-topology}
The intra-corpus citation network provides a structural view of how
Active Inference research is organized, identifying influential hub
papers, community structure, and patterns of citation isolation.

\begin{figure}[htbp]
\centering
\includegraphics[width=0.9\textwidth]{../figures/citation_network.png}
\caption{Intra-corpus citation network ($N = 1208$ nodes, 2{,}780 edges). Node size reflects PageRank and HITS centrality scores \citep{kleinberg1999authoritative}; highly cited foundational papers serve as nexus points connecting sub-domains.}
\label{fig:citation_network}
\end{figure}

\begin{block}{Network Density and Degree Distribution}
\protect\phantomsection\label{network-density-and-degree-distribution}
The intra-corpus citation network contains 1208 nodes and 2\{,\}780
edges, with a density of 0.19\% and 700 connected components. The
average in-degree of \(\approx 2.3\) indicates that most papers receive
few intra-corpus citations, consistent with the field's rapid expansion:
the majority of recent papers have not yet accumulated citations within
the corpus. Only 6.1\% of all references (2\{,\}780 of 45\{,\}716)
resolve to other papers within the corpus, reflecting cross-source
identifier mismatches and the field's engagement with a broad external
literature base. Community detection identifies clusters via the Louvain
algorithm \citep{blondel2008louvain}.

\begin{figure}[htbp]
\centering
\includegraphics[width=0.7\textwidth]{../figures/degree_distribution.png}
\caption{In-degree distribution of the citation network. The power-law tail is characteristic of citation networks, with a small number of highly cited hubs.}
\label{fig:degree_distribution}
\end{figure}
\end{block}

\begin{block}{Connected Components and Citation Isolation}
\protect\phantomsection\label{connected-components-and-citation-isolation}
The high number of connected components (700 out of 1208 nodes) reveals
that much of the corpus consists of citation-isolated papers---works
that neither cite nor are cited by other papers in the collection. This
is partially an artifact of cross-source identifier mismatches, but it
also reflects the field's pattern of papers engaging with the FEP
literature conceptually without building explicit citation chains.
PageRank analysis identifies highly influential papers, predominantly
Friston's foundational work \citep{friston2010free} and the AIF textbook
\citep{parr2022active}, which serve as nexus points linking otherwise
disconnected subgraphs.
\end{block}

\begin{block}{Network Summary}
\protect\phantomsection\label{network-summary}
{\def\LTcaptype{none} % do not increment counter
\begin{longtable}[]{@{}ll@{}}
\toprule\noalign{}
Metric & Value \\
\midrule\noalign{}
\endhead
Nodes & 1208 \\
Edges & 2\{,\}780 \\
Reference resolution rate & 6.1\% (2\{,\}780 / 45\{,\}716) \\
Connected components & 700 \\
Network density & 0.19\% \\
Mean in-degree & \(\approx\) 2.3 \\
\bottomrule\noalign{}
\end{longtable}
}
\end{block}
\end{frame}

\end{document}
