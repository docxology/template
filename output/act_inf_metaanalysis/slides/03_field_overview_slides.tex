% Options for packages loaded elsewhere
\PassOptionsToPackage{unicode}{hyperref}
\PassOptionsToPackage{hyphens}{url}
\documentclass[
  ignorenonframetext,
]{beamer}
\newif\ifbibliography
\usepackage{pgfpages}
\setbeamertemplate{caption}[numbered]
\setbeamertemplate{caption label separator}{: }
\setbeamercolor{caption name}{fg=normal text.fg}
\beamertemplatenavigationsymbolsempty
% remove section numbering
\setbeamertemplate{part page}{
  \centering
  \begin{beamercolorbox}[sep=16pt,center]{part title}
    \usebeamerfont{part title}\insertpart\par
  \end{beamercolorbox}
}
\setbeamertemplate{section page}{
  \centering
  \begin{beamercolorbox}[sep=12pt,center]{section title}
    \usebeamerfont{section title}\insertsection\par
  \end{beamercolorbox}
}
\setbeamertemplate{subsection page}{
  \centering
  \begin{beamercolorbox}[sep=8pt,center]{subsection title}
    \usebeamerfont{subsection title}\insertsubsection\par
  \end{beamercolorbox}
}
% Prevent slide breaks in the middle of a paragraph
\widowpenalties 1 10000
\raggedbottom
\AtBeginPart{
  \frame{\partpage}
}
\AtBeginSection{
  \ifbibliography
  \else
    \frame{\sectionpage}
  \fi
}
\AtBeginSubsection{
  \frame{\subsectionpage}
}
\usepackage{iftex}
\ifPDFTeX
  \usepackage[T1]{fontenc}
  \usepackage[utf8]{inputenc}
  \usepackage{textcomp} % provide euro and other symbols
\else % if luatex or xetex
  \usepackage{unicode-math} % this also loads fontspec
  \defaultfontfeatures{Scale=MatchLowercase}
  \defaultfontfeatures[\rmfamily]{Ligatures=TeX,Scale=1}
\fi
\usepackage{lmodern}
\ifPDFTeX\else
  % xetex/luatex font selection
\fi
% Use upquote if available, for straight quotes in verbatim environments
\IfFileExists{upquote.sty}{\usepackage{upquote}}{}
\IfFileExists{microtype.sty}{% use microtype if available
  \usepackage[]{microtype}
  \UseMicrotypeSet[protrusion]{basicmath} % disable protrusion for tt fonts
}{}
\makeatletter
\@ifundefined{KOMAClassName}{% if non-KOMA class
  \IfFileExists{parskip.sty}{%
    \usepackage{parskip}
  }{% else
    \setlength{\parindent}{0pt}
    \setlength{\parskip}{6pt plus 2pt minus 1pt}}
}{% if KOMA class
  \KOMAoptions{parskip=half}}
\makeatother
\usepackage{longtable,booktabs,array}
\newcounter{none} % for unnumbered tables
\usepackage{calc} % for calculating minipage widths
\usepackage{caption}
% Make caption package work with longtable
\makeatletter
\def\fnum@table{\tablename~\thetable}
\makeatother
\setlength{\emergencystretch}{3em} % prevent overfull lines
\providecommand{\tightlist}{%
  \setlength{\itemsep}{0pt}\setlength{\parskip}{0pt}}
\usepackage{bookmark}
\IfFileExists{xurl.sty}{\usepackage{xurl}}{} % add URL line breaks if available
\urlstyle{same}
\hypersetup{
  hidelinks,
  pdfcreator={LaTeX via pandoc}}

\author{\texorpdfstring{}{}}
\date{}

\begin{document}

\begin{frame}{Field Overview: Disciplinary Structure and Growth Dynamics
\label{sec:field_overview}}
\protect\phantomsection\label{field-overview-disciplinary-structure-and-growth-dynamics}
The Active Inference literature has undergone a profound phase
transition. What originated in the late 2000s as a densely clustered
niche within theoretical neuroscience has explosively expanded into a
multi-disciplinary research program spanning three primary domains and
eight strictly tracked categories. Our corpus, extracted from arXiv,
Semantic Scholar, and OpenAlex and rigorously deduplicated to
\(N = 1208\) papers (1972--2026), captures the breadth, tempo, and
internal architecture of this expansion.

\begin{figure}[htbp]
\centering
\includegraphics[width=0.9\textwidth]{../figures/field_summary.png}
\caption{Publication counts by domain ($N = 1208$). Domain A (Core Theory) dominates, with Domains B (Tools) and C (Applications) forming growing tiers.}
\label{fig:field_summary}
\end{figure}

\begin{block}{Corpus-Level Summary}
\protect\phantomsection\label{corpus-level-summary}
{\def\LTcaptype{none} % do not increment counter
\begin{longtable}[]{@{}ll@{}}
\toprule\noalign{}
Metric & Value \\
\midrule\noalign{}
\endhead
Total papers & 1208 \\
Year range & 1972--2026 \\
Peak year & 2025 \\
CAGR & 6.63\% \\
Active domains & 8 of 8 tracked (A1--A2, B, C1--C5) \\
\bottomrule\noalign{}
\end{longtable}
}

The CAGR of 6.63\% reflects the corpus's long temporal span from 1972 to
2026; the field's actual rapid growth phase began around 2013, with
annual output accelerating substantially. The fact that sustained high
output persists into subsequent years suggests the field has reached a
mature production phase rather than experiencing a transient spike.
Citation network metrics are detailed in the dedicated citation network
analysis (see \hyperref[sec:citation_network]{Section 3c}).

\begin{figure}[htbp]
\centering
\includegraphics[width=0.8\textwidth]{../figures/growth_curve.png}
\caption{Annual and cumulative publication counts, 1972--2026. The inflection around 2013 marks the onset of rapid growth, sustained by a steady moving average (dashed line) reflecting the field's matured production phase.}
\label{fig:growth_curve}
\end{figure}
\end{block}

\begin{block}{Domain Distribution}
\protect\phantomsection\label{domain-distribution}
Keyword-based classification assigns each paper to one of eight
categories across three domains:

{\def\LTcaptype{none} % do not increment counter
\begin{longtable}[]{@{}llll@{}}
\toprule\noalign{}
Domain & Category & Papers & Percentage \\
\midrule\noalign{}
\endhead
\textbf{A -- Core Theory} & A1: Formal Theory & 120 & 9.9\% \\
& A2: Qualitative Philosophy & 154 & 12.7\% \\
\textbf{B -- Tools} & B: Tools \& Translation & 267 & 22.1\% \\
\textbf{C -- Applications} & C1: Neuroscience & 206 & 17.1\% \\
& C2: Robotics & 170 & 14.1\% \\
& C3: Language & 57 & 4.7\% \\
& C4: Psychiatry & 34 & 2.8\% \\
& C5: Biology & 200 & 16.6\% \\
\bottomrule\noalign{}
\end{longtable}
}

The concentration of papers in A2 (qualitative philosophy and general
theory) reflects the broad scope of foundational FEP work. The
priority-based classifier mitigates over-assignment by routing papers
with mathematical indicators (theorems, proofs, equations, statistical
formalism) to A1 before falling back to A2, and by preferring specific
application domains (C1--C5) and tools (B) over both core-theory
categories. Nevertheless, papers that discuss FEP/AIF conceptually
without mathematical formalism or domain-specific vocabulary are
legitimately assigned to A2. This figure should be read as a
\emph{ceiling} on theoretical generality rather than a literal measure
of research focus---embedding-based classification would likely
redistribute a further fraction into more specific categories. That all
eight categories are populated, including computational psychiatry (C4)
and formal theory (A1), indicates genuine diversification beyond the
field's neuroscience origins.

\begin{figure}[htbp]
\centering
\includegraphics[width=0.8\textwidth]{../figures/subfield_distribution.png}
\caption{Domain distribution ($N = 1208$). Classification uses hierarchical keyword matching against curated lists applied to titles and abstracts, capturing distinct methodological and domain-specific groupings.}
\label{fig:subfield_distribution}
\end{figure}

Detailed characterizations of each domain---including historical
context, growth trends, and open problems---are provided in the
supplementary domain analyses (see
\hyperref[sec:subfield_analyses]{Section~3a}). Latent topic structure,
vocabulary analysis, and document embeddings are presented in the text
analytics section (see \hyperref[sec:text_analytics]{Section~3b}).
\end{block}

\begin{block}{Cross-Domain Comparison}
\protect\phantomsection\label{cross-domain-comparison}
{\def\LTcaptype{none} % do not increment counter
\begin{longtable}[]{@{}
  >{\raggedright\arraybackslash}p{(\linewidth - 10\tabcolsep) * \real{0.1667}}
  >{\raggedright\arraybackslash}p{(\linewidth - 10\tabcolsep) * \real{0.1667}}
  >{\raggedright\arraybackslash}p{(\linewidth - 10\tabcolsep) * \real{0.1667}}
  >{\raggedright\arraybackslash}p{(\linewidth - 10\tabcolsep) * \real{0.1667}}
  >{\raggedright\arraybackslash}p{(\linewidth - 10\tabcolsep) * \real{0.1667}}
  >{\raggedright\arraybackslash}p{(\linewidth - 10\tabcolsep) * \real{0.1667}}@{}}
\toprule\noalign{}
\begin{minipage}[b]{\linewidth}\raggedright
Domain
\end{minipage} & \begin{minipage}[b]{\linewidth}\raggedright
Category
\end{minipage} & \begin{minipage}[b]{\linewidth}\raggedright
Papers
\end{minipage} & \begin{minipage}[b]{\linewidth}\raggedright
Growth Trend
\end{minipage} & \begin{minipage}[b]{\linewidth}\raggedright
Key Challenge
\end{minipage} & \begin{minipage}[b]{\linewidth}\raggedright
Representative Work
\end{minipage} \\
\midrule\noalign{}
\endhead
A & A1: Formal & 120 (9.9\%) & Growing & Mathematical accessibility for
broader field & \citep{sakthivadivel2023bayesian} \\
A & A2: Philosophy & 154 (12.7\%) & Stable & Residual catch-all; absorbs
FEP prose papers & \citep{friston2010free} \\
B & B: Tools & 267 (22.1\%) & Rapid & Matching deep RL benchmark
performance & \citep{fountas2020deep} \\
C & C1: Neuroscience & 206 (17.1\%) & Stable & Bridging theory and
empirical neuroimaging & \citep{clark2013whatever} \\
C & C2: Robotics & 170 (14.1\%) & Growing & Real-time feasibility on
embedded hardware & \citep{lanillos2021active} \\
C & C3: Language & 57 (4.7\%) & Emerging & Demonstrating gains over
existing NLP models & \citep{friston2020generative} \\
C & C4: Psychiatry & 34 (2.8\%) & Emerging & Translating models to
clinical practice & \citep{smith2021computational} \\
C & C5: Biology & 200 (16.6\%) & Rapid & Empirical validation of
theoretical proposals & \citep{kuchling2020morphogenesis} \\
\bottomrule\noalign{}
\end{longtable}
}

The distribution definitively reveals a diversified topology rather than
concentrated isolation in a single legacy domain. Domain B (Tools \&
Translation) has surged to constitute the largest single category at
22.1\%, immediately followed by the empirical applications of C1
(Neuroscience) at 17.1\% and C2 (Robotics) at 14.1\%. Domain A (Core
Theory) aggregates 22.7\% collectively (A1 + A2), while the emergent
application frontiers (C3--C5) exhibit accelerating growth. Crucially,
A1's measured 120 papers deliberately belie its overarching intellectual
gravity---the mathematical formalisms refined in A1 fundamentally
constrain and enable architectural implementations across all
operational domains.

\begin{figure}[htbp]
\centering
\includegraphics[width=0.9\textwidth]{../figures/subfield_timeline.png}
\caption{Temporal evolution of publication counts by domain. Domain A (Core Theory) dominates throughout; the other domains show varying growth trajectories.}
\label{fig:subfield_timeline}
\end{figure}
\end{block}
\end{frame}

\end{document}
