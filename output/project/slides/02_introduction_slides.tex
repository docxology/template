% Options for packages loaded elsewhere
\PassOptionsToPackage{unicode}{hyperref}
\PassOptionsToPackage{hyphens}{url}
\documentclass[
  ignorenonframetext,
]{beamer}
\newif\ifbibliography
\usepackage{pgfpages}
\setbeamertemplate{caption}[numbered]
\setbeamertemplate{caption label separator}{: }
\setbeamercolor{caption name}{fg=normal text.fg}
\beamertemplatenavigationsymbolsempty
% remove section numbering
\setbeamertemplate{part page}{
  \centering
  \begin{beamercolorbox}[sep=16pt,center]{part title}
    \usebeamerfont{part title}\insertpart\par
  \end{beamercolorbox}
}
\setbeamertemplate{section page}{
  \centering
  \begin{beamercolorbox}[sep=12pt,center]{section title}
    \usebeamerfont{section title}\insertsection\par
  \end{beamercolorbox}
}
\setbeamertemplate{subsection page}{
  \centering
  \begin{beamercolorbox}[sep=8pt,center]{subsection title}
    \usebeamerfont{subsection title}\insertsubsection\par
  \end{beamercolorbox}
}
% Prevent slide breaks in the middle of a paragraph
\widowpenalties 1 10000
\raggedbottom
\AtBeginPart{
  \frame{\partpage}
}
\AtBeginSection{
  \ifbibliography
  \else
    \frame{\sectionpage}
  \fi
}
\AtBeginSubsection{
  \frame{\subsectionpage}
}
\usepackage{iftex}
\ifPDFTeX
  \usepackage[T1]{fontenc}
  \usepackage[utf8]{inputenc}
  \usepackage{textcomp} % provide euro and other symbols
\else % if luatex or xetex
  \usepackage{unicode-math} % this also loads fontspec
  \defaultfontfeatures{Scale=MatchLowercase}
  \defaultfontfeatures[\rmfamily]{Ligatures=TeX,Scale=1}
\fi
\usepackage{lmodern}
\ifPDFTeX\else
  % xetex/luatex font selection
\fi
% Use upquote if available, for straight quotes in verbatim environments
\IfFileExists{upquote.sty}{\usepackage{upquote}}{}
\IfFileExists{microtype.sty}{% use microtype if available
  \usepackage[]{microtype}
  \UseMicrotypeSet[protrusion]{basicmath} % disable protrusion for tt fonts
}{}
\makeatletter
\@ifundefined{KOMAClassName}{% if non-KOMA class
  \IfFileExists{parskip.sty}{%
    \usepackage{parskip}
  }{% else
    \setlength{\parindent}{0pt}
    \setlength{\parskip}{6pt plus 2pt minus 1pt}}
}{% if KOMA class
  \KOMAoptions{parskip=half}}
\makeatother
\setlength{\emergencystretch}{3em} % prevent overfull lines
\providecommand{\tightlist}{%
  \setlength{\itemsep}{0pt}\setlength{\parskip}{0pt}}
\usepackage{bookmark}
\IfFileExists{xurl.sty}{\usepackage{xurl}}{} % add URL line breaks if available
\urlstyle{same}
\hypersetup{
  hidelinks,
  pdfcreator={LaTeX via pandoc}}

\author{\texorpdfstring{}{}}
\date{}

\begin{document}

\section{Introduction}\label{sec:introduction}

\begin{frame}{Overview}
\protect\phantomsection\label{overview}
This is an example project that demonstrates the generic repository
structure for tested code, manuscript editing, and PDF rendering. The
work presents a novel optimization framework with comprehensive
theoretical analysis and experimental validation, building upon
foundational optimization theory \cite{boyd2004, nesterov2018} and
recent advances in adaptive methods \cite{kingma2014, duchi2011}.
\end{frame}

\begin{frame}[fragile]{Project Structure}
\protect\phantomsection\label{project-structure}
The project follows a standardized structure:

\begin{itemize}
\tightlist
\item
  \textbf{\texttt{src/}} - Source code with comprehensive test coverage
\item
  \textbf{\texttt{tests/}} - Test files ensuring 100\% coverage
\item
  \textbf{\texttt{scripts/}} - Project-specific scripts for generating
  figures and data
\item
  \textbf{\texttt{manuscript/}} - Markdown source files for the
  manuscript
\item
  \textbf{\texttt{output/}} - Generated outputs (PDFs, figures, data)
\item
  \textbf{\texttt{repo\_utilities/}} - Generic utility scripts for any
  project
\end{itemize}
\end{frame}

\begin{frame}[fragile]{Key Features}
\protect\phantomsection\label{key-features}
\begin{block}{Test-Driven Development}
\protect\phantomsection\label{test-driven-development}
All source code must have 100\% test coverage before PDF generation
proceeds, as enforced by the build system.
\end{block}

\begin{block}{Automated Script Execution}
\protect\phantomsection\label{automated-script-execution}
Project-specific scripts in the \texttt{scripts/} directory are
automatically executed to generate figures and data, ensuring
reproducibility.
\end{block}

\begin{block}{Markdown to PDF Pipeline}
\protect\phantomsection\label{markdown-to-pdf-pipeline}
Individual markdown modules are converted to PDFs, and a combined
document is generated with proper cross-referencing.
\end{block}

\begin{block}{Generic and Reusable}
\protect\phantomsection\label{generic-and-reusable}
The utility scripts can be used with any project that follows this
structure, making it easy to adopt for new research projects.
\end{block}
\end{frame}

\begin{frame}{Manuscript Organization}
\protect\phantomsection\label{manuscript-organization}
The manuscript is organized into several key sections:

\begin{enumerate}
\tightlist
\item
  \textbf{Abstract} (Section \ref{sec:abstract}): Research overview and
  key contributions
\item
  \textbf{Introduction} (Section \ref{sec:introduction}): Overview and
  project structure
\item
  \textbf{Methodology} (Section \ref{sec:methodology}): Mathematical
  framework and algorithms
\item
  \textbf{Experimental Results} (Section
  \ref{sec:experimental_results}): Performance evaluation and validation
\item
  \textbf{Discussion} (Section \ref{sec:discussion}): Theoretical
  implications and comparisons
\item
  \textbf{Conclusion} (Section \ref{sec:conclusion}): Summary and future
  directions
\item
  \textbf{References} (Section \ref{sec:references}): Bibliography and
  cited works
\end{enumerate}
\end{frame}

\begin{frame}{Example Figure}
\protect\phantomsection\label{example-figure}
The following figure was generated by the example script:

\begin{figure}[h]
\centering
\includegraphics[width=0.8\textwidth]{../output/figures/example_figure.png}
\caption{Example project figure showing a mathematical function}
\label{fig:example_figure}
\end{figure}

This demonstrates how figures are automatically integrated into the
manuscript with proper cross-referencing capabilities. The figure shows
a mathematical function that demonstrates the project's capabilities. As
shown in Figure \ref{fig:example_figure}, the system generates
high-quality visualizations that are automatically integrated into the
manuscript.
\end{frame}

\begin{frame}[fragile]{Data Availability}
\protect\phantomsection\label{data-availability}
All generated data is saved alongside figures for reproducibility:

\begin{itemize}
\tightlist
\item
  \textbf{Figures}: PNG format in \texttt{output/figures/}
\item
  \textbf{Data}: NPZ and CSV formats in \texttt{output/data/}
\item
  \textbf{PDFs}: Individual and combined documents in
  \texttt{output/pdf/}
\item
  \textbf{LaTeX}: Source files in \texttt{output/tex/}
\end{itemize}
\end{frame}

\begin{frame}[fragile]{Usage}
\protect\phantomsection\label{usage}
To generate the complete manuscript:

\begin{verbatim}
# Clean previous outputs
./repo_utilities/clean_output.sh

# Generate everything (tests + scripts + PDFs)
./repo_utilities/render_pdf.sh
\end{verbatim}

The system will automatically: 1. Run all tests with 100\% coverage
requirement 2. Execute project-specific scripts to generate figures and
data 3. Validate markdown references and images 4. Generate individual
and combined PDFs 5. Export LaTeX source files
\end{frame}

\begin{frame}[fragile]{Customization}
\protect\phantomsection\label{customization}
This template can be customized for any project by:

\begin{enumerate}
\tightlist
\item
  Adding project-specific scripts to \texttt{scripts/}
\item
  Modifying markdown files in \texttt{markdown/}
\item
  Setting environment variables for author information
\item
  Adjusting LaTeX preamble in \texttt{preamble.md}
\item
  Adding new sections with proper cross-references
\end{enumerate}
\end{frame}

\begin{frame}[fragile]{Cross-Referencing System}
\protect\phantomsection\label{cross-referencing-system}
The manuscript demonstrates comprehensive cross-referencing:

\begin{itemize}
\tightlist
\item
  \textbf{Section References}: Use the ref command with \texttt{sec:}
  prefix for sections
\item
  \textbf{Equation References}: Use the eqref command with \texttt{eq:}
  prefix for equations (see Section \ref{sec:methodology})
\item
  \textbf{Figure References}: Use the ref command with figure labels
\item
  \textbf{Table References}: Use the ref command with \texttt{tab:}
  prefix for tables
\end{itemize}

All references are automatically numbered and updated when the document
is regenerated. For example, the main objective function
\eqref{eq:objective} is defined in the methodology section.
\end{frame}

\end{document}
