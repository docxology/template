% Options for packages loaded elsewhere
\PassOptionsToPackage{unicode}{hyperref}
\PassOptionsToPackage{hyphens}{url}
\documentclass[
]{article}
\usepackage{xcolor}
\usepackage{amsmath,amssymb}
\setcounter{secnumdepth}{5}
\usepackage{iftex}
\ifPDFTeX
  \usepackage[T1]{fontenc}
  \usepackage[utf8]{inputenc}
  \usepackage{textcomp} % provide euro and other symbols
\else % if luatex or xetex
  \usepackage{unicode-math} % this also loads fontspec
  \defaultfontfeatures{Scale=MatchLowercase}
  \defaultfontfeatures[\rmfamily]{Ligatures=TeX,Scale=1}
\fi
\usepackage{lmodern}
\ifPDFTeX\else
  % xetex/luatex font selection
\fi
% Use upquote if available, for straight quotes in verbatim environments
\IfFileExists{upquote.sty}{\usepackage{upquote}}{}
\IfFileExists{microtype.sty}{% use microtype if available
  \usepackage[]{microtype}
  \UseMicrotypeSet[protrusion]{basicmath} % disable protrusion for tt fonts
}{}
\makeatletter
\@ifundefined{KOMAClassName}{% if non-KOMA class
  \IfFileExists{parskip.sty}{%
    \usepackage{parskip}
  }{% else
    \setlength{\parindent}{0pt}
    \setlength{\parskip}{6pt plus 2pt minus 1pt}}
}{% if KOMA class
  \KOMAoptions{parskip=half}}
\makeatother
\usepackage{listings}
\newcommand{\passthrough}[1]{#1}
\lstset{defaultdialect=[5.3]Lua}
\lstset{defaultdialect=[x86masm]Assembler}
\usepackage{longtable,booktabs,array}
\newcounter{none} % for unnumbered tables
\usepackage{calc} % for calculating minipage widths
% Correct order of tables after \paragraph or \subparagraph
\usepackage{etoolbox}
\makeatletter
\patchcmd\longtable{\par}{\if@noskipsec\mbox{}\fi\par}{}{}
\makeatother
% Allow footnotes in longtable head/foot
\IfFileExists{footnotehyper.sty}{\usepackage{footnotehyper}}{\usepackage{footnote}}
\makesavenoteenv{longtable}
\setlength{\emergencystretch}{3em} % prevent overfull lines
\providecommand{\tightlist}{%
  \setlength{\itemsep}{0pt}\setlength{\parskip}{0pt}}
\usepackage[]{natbib}
\bibliographystyle{plainnat}
\usepackage{bookmark}
\IfFileExists{xurl.sty}{\usepackage{xurl}}{} % add URL line breaks if available
\urlstyle{same}
\hypersetup{
  hidelinks,
  pdfcreator={LaTeX via pandoc}}

\author{}
\date{}

% Essential packages for academic documents
\usepackage{amsmath,amssymb}          % Mathematical symbols and environments
\usepackage{amsfonts}                 % Additional math fonts
\usepackage{amsthm}                   % Theorem environments
\usepackage{graphicx}                 % Include graphics
\usepackage[margin=1in]{geometry}     % Wider margins (1 inch all sides)
\usepackage{float}                    % Better float placement
\usepackage{booktabs}                 % Professional tables
\usepackage{longtable}                % Long tables spanning pages
\usepackage{array}                    % Advanced table formatting
\usepackage{multirow}                 % Multi-row table cells
\usepackage{caption}                  % Enhanced caption formatting
\usepackage{subcaption}               % Sub-figures and sub-tables
\usepackage{bm}                       % Bold math symbols
\usepackage{url}                      % URL formatting
\usepackage{hyperref}                 % Hyperlinks and cross-references
\usepackage{cleveref}                 % Intelligent cross-referencing
\usepackage[capitalise]{cleveref}     % Capitalize cross-reference labels
\usepackage{natbib}                   % Bibliography support
\usepackage{doi}                      % DOI links

% Configure figure numbering and captions
\renewcommand{\figurename}{Figure}
\captionsetup{
    justification=centering,
    font=small,
    labelfont=bf,
    labelsep=period
}

% Configure table numbering and captions
\renewcommand{\tablename}{Table}
\captionsetup[table]{
    justification=centering,
    font=small,
    labelfont=bf,
    labelsep=period
}

% Configure section numbering
\setcounter{secnumdepth}{3}
\renewcommand{\thesection}{\arabic{section}}
\renewcommand{\thesubsection}{\arabic{section}.\arabic{subsection}}
\renewcommand{\thesubsubsection}{\arabic{section}.\arabic{subsection}.\arabic{subsubsection}}

% Configure equation numbering
\numberwithin{equation}{section}

% Configure hyperref for proper linking
\hypersetup{
    colorlinks=true,
    linkcolor=red,
    citecolor=red,
    urlcolor=red,
    filecolor=red,
    pdfborder={0 0 0},
    bookmarks=true,
    bookmarksnumbered=true,
    bookmarkstype=toc,
    pdftitle={Research Project Template},
    pdfauthor={Template Author},
    pdfsubject={Academic Research},
    pdfkeywords={research, template, academic, LaTeX},
    pdfcreator={render_pdf.sh},
    pdfproducer={XeLaTeX}
}

% Configure cleveref for intelligent cross-references
\crefname{section}{Section}{Sections}
\crefname{subsection}{Subsection}{Subsections}
\crefname{subsubsection}{Subsubsection}{Subsubsections}
\crefname{equation}{Equation}{Equations}
\crefname{figure}{Figure}{Figures}
\crefname{table}{Table}{Tables}
\crefname{appendix}{Appendix}{Appendices}

% Configure fonts for Unicode support with fallbacks
\usepackage{newunicodechar}
\newunicodechar{⁴}{\textsuperscript{4}}
\newunicodechar{₄}{\textsubscript{4}}
\newunicodechar{²}{\textsuperscript{2}}
\newunicodechar{₀}{\textsubscript{0}}
\newunicodechar{₁}{\textsubscript{1}}
\newunicodechar{₂}{\textsubscript{2}}
\newunicodechar{₃}{\textsubscript{3}}

% Use standard fonts for better compatibility
\usepackage{lmodern}
\usepackage[T1]{fontenc}

% Enhanced code block styling for better contrast and readability
\usepackage{fancyvrb}
\usepackage{xcolor}
\usepackage{listings}

% Define custom colors for code blocks
\definecolor{codebg}{RGB}{248, 248, 248}      % Very light gray background
\definecolor{codeborder}{RGB}{200, 200, 200}  % Medium gray border
\definecolor{codefg}{RGB}{34, 34, 34}         % Dark gray text
\definecolor{commentcolor}{RGB}{102, 102, 102} % Comment color
\definecolor{keywordcolor}{RGB}{0, 0, 0}       % Keyword color
\definecolor{stringcolor}{RGB}{0, 102, 0}      % String color

% Configure Verbatim environment for inline code
\DefineVerbatimEnvironment{Verbatim}{Verbatim}{%
    fontsize=\small,
    frame=single,
    framerule=0.5pt,
    framesep=3pt,
    rulecolor=\color{codeborder},
    bgcolor=\color{codebg},
    fgcolor=\color{codefg}
}

% Configure code block styling
\DefineVerbatimEnvironment{Highlighting}{Verbatim}{%
    fontsize=\footnotesize,
    frame=single,
    framerule=0.5pt,
    framesep=5pt,
    rulecolor=\color{codeborder},
    bgcolor=\color{codebg},
    fgcolor=\color{codefg}
}

% Style inline code with \texttt
\renewcommand{\texttt}[1]{%
    \colorbox{codebg}{\color{codefg}\ttfamily #1}%
}

% Configure listings package for code blocks
\lstset{
    backgroundcolor=\color{codebg},
    basicstyle=\footnotesize\ttfamily\color{codefg},
    breakatwhitespace=false,
    breaklines=true,
    captionpos=b,
    commentstyle=\color{commentcolor},
    deletekeywords={...},
    escapeinside={\%*}{*)},
    extendedchars=true,
    frame=single,
    framerule=0.5pt,
    framesep=5pt,
    keepspaces=true,
    keywordstyle=\color{keywordcolor}\bfseries,
    language=Python,
    morekeywords={*,...},
    numbers=left,
    numbersep=5pt,
    numberstyle=\tiny\color{codefg},
    rulecolor=\color{codeborder},
    showspaces=false,
    showstringspaces=false,
    showtabs=false,
    stepnumber=1,
    stringstyle=\color{stringcolor},
    tabsize=4,
    title=\lstname
}

% Override any Pandoc default lstset configurations
\AtBeginDocument{
    \lstset{
        backgroundcolor=\color{codebg},
        basicstyle=\footnotesize\ttfamily\color{codefg},
        frame=single,
        framerule=0.5pt,
        framesep=5pt,
        rulecolor=\color{codeborder},
        numbers=left,
        numbersep=5pt,
        numberstyle=\tiny\color{codefg}
    }
}

% Configure bibliography
% Note: Using plainnat with natbib package for proper citation processing
% The bibliography style and commands (\bibliographystyle and \bibliography) are in 99_references.md

% Simple page break support for document structure
% Note: Page breaks are handled in the markdown generation, not here

% Ensure proper spacing and formatting
\frenchspacing  % Single space after periods
\linespread{1.2}  % Slightly increased line spacing for readability

\title{Ways of Figuring Things Out: A Systematic Analysis\\\normalsize Documenting and Analyzing Andrius Kulikauskas's 284 Ways of Knowledge Acquisition}
\author{Daniel Ari Friedman}
\date{\today}

\begin{document}

\maketitle
\thispagestyle{empty}


{
\setcounter{tocdepth}{3}
\tableofcontents
}
\section{Abstract}\label{sec:abstract}

This research presents a comprehensive systematic analysis of Andrius
Kulikauskas's ``Ways of Figuring Things Out,'' documenting and analyzing
210 ways from the database with connections to the broader framework of
284 ways in the source text. The database-driven analysis covers 24
rooms in the House of Knowledge, 38 distinct dialogue types, and 196
unique dialogue partners, organized within fundamental structures of
believing (1-2-3-4), caring (1-2-3-4), and relative learning (taking a
stand, following through, reflecting). Our quantitative analysis reveals
systematic patterns: the B2 room contains the most ways (23),
``goodness'' and ``other'' are the most common dialogue types (15 each),
and the network exhibits 1,290 edges with a clustering coefficient of
0.886, connecting ways through shared rooms, dialogue types, and
partners. Cross-tabulation analysis shows strong associations between
dialogue types and room assignments, with information-theoretic metrics
quantifying the structure's organization. Network analysis identifies
central ways serving as bridges between categories, while text analysis
of examples reveals thematic patterns across 210 documented ways. This
work provides both a philosophical framework for understanding different
approaches to knowledge and a practical system for analyzing and
applying these ways in educational, research, and personal development
contexts, offering tools for researchers, educators, and practitioners
seeking to understand and apply diverse approaches to figuring things
out.

\newpage

\section{Introduction}\label{sec:introduction}

\subsection{Overview}\label{overview}

This research documents and analyzes Andrius Kulikauskas's comprehensive
framework of ``Ways of Figuring Things Out,'' a systematic collection of
210 documented ways from the database, with connections to the broader
framework of 284 ways described in the source text. The work presents
both a philosophical framework for understanding different approaches to
knowledge and an empirical analysis of how these ways are structured,
categorized, and interrelated.

\subsubsection{Data Summary}\label{data-summary}

The analysis database contains comprehensive metadata for all documented
ways:

\begin{table}[h]
\centering
\begin{tabular}{|l|c|}
\hline
\textbf{Category} & \textbf{Count} \\
\hline
Total ways (database) & 210 \\
Total ways (documented) & 284 \\
Rooms in House of Knowledge & 24 \\
Distinct dialogue types & 38 \\
Unique dialogue partners & 196 \\
Network nodes & 210 \\
Network edges & 1,290 \\
\hline
\end{tabular}
\caption{Summary statistics for Ways of Figuring Things Out database}
\label{tab:data_summary}
\end{table}

The framework structure is visualized in Figure
\ref{fig:room_hierarchy}, showing the distribution of ways across the 24
rooms. The network of relationships between ways is presented in Figure
\ref{fig:ways_network}, revealing clusters and central connecting ways.

\subsection{The House of Knowledge
Framework}\label{the-house-of-knowledge-framework}

The Ways framework is organized around a ``House of Knowledge''
containing 24 rooms, each representing a different aspect of how we come
to know and understand. These rooms are structured according to
fundamental philosophical principles:

\begin{itemize}
\tightlist
\item
  \textbf{Believing (1-2-3-4)}: Four levels of believing, from basic
  belief to fostering spirit among us
\item
  \textbf{Caring (1-2-3-4)}: Four levels of caring, from basic openness
  to acknowledging what transcends our limits\\
\item
  \textbf{Relative Learning}: The cycle of taking a stand, following
  through, and reflecting
\item
  \textbf{Dialogue Types}: Absolute, Relative, and Embrace God
  perspectives
\end{itemize}

Each way represents a specific method for figuring things out,
documented with examples, dialogue partners, and its relationship to the
broader framework.

\subsection{Research Objectives}\label{research-objectives}

This work aims to:

\begin{enumerate}
\def\labelenumi{\arabic{enumi}.}
\tightlist
\item
  \textbf{Documentation}: Provide complete documentation of all 284 ways
  with their characteristics, examples, and relationships
\item
  \textbf{Categorization}: Systematically categorize ways according to
  dialogue types, rooms, and philosophical structures
\item
  \textbf{Analysis}: Conduct empirical analysis of way distributions,
  patterns, and interrelationships
\item
  \textbf{Visualization}: Create visual representations of the network
  of ways and their connections
\item
  \textbf{Application}: Develop tools and frameworks for applying these
  ways in educational and research contexts
\end{enumerate}

\subsection{Data Sources}\label{data-sources}

The research draws on two primary data sources:

\begin{itemize}
\tightlist
\item
  \textbf{SQL Database}: A comprehensive SQLite database (converted from
  MySQL) containing 210 ways with complete metadata including dialogue
  types, examples, room assignments (mene), God relationships (Dievas),
  and conversant information
\item
  \textbf{Text Documentation}: Detailed markdown documentation
  (\passthrough{\lstinline!ways.md!}) providing philosophical context,
  examples, and descriptions for all 284 ways
\end{itemize}

\subsection{Methodology Overview}\label{methodology-overview}

Our approach combines:

\begin{itemize}
\tightlist
\item
  \textbf{Database Analysis}: SQLite conversion and querying of the ways
  database to extract patterns and relationships
\item
  \textbf{Network Analysis}: Graph-based analysis of how ways connect
  through dialogue partners and shared characteristics
\item
  \textbf{Statistical Analysis}: Quantitative analysis of distributions
  across categories, dialogue types, and rooms
\item
  \textbf{Text Analysis}: Analysis of way descriptions and examples to
  extract themes and patterns
\item
  \textbf{Visualization}: Creation of network graphs, hierarchical
  visualizations, and statistical plots
\end{itemize}

\subsection{Key Contributions}\label{key-contributions}

This research makes several key contributions:

\begin{enumerate}
\def\labelenumi{\arabic{enumi}.}
\tightlist
\item
  \textbf{Complete Documentation}: First comprehensive systematic
  documentation of all 284 ways
\item
  \textbf{Empirical Analysis}: Quantitative analysis revealing patterns
  in way distributions and relationships
\item
  \textbf{Network Mapping}: Visualization of the network structure
  connecting different ways
\item
  \textbf{Categorization System}: Systematic organization within the
  24-room House of Knowledge framework
\item
  \textbf{Practical Tools}: Database and analysis tools for researchers
  and practitioners
\end{enumerate}

\subsection{Manuscript Organization}\label{manuscript-organization}

The manuscript is organized as follows:

\begin{enumerate}
\def\labelenumi{\arabic{enumi}.}
\tightlist
\item
  \textbf{Abstract} (Section \ref{sec:abstract}): Overview of the
  research and key findings
\item
  \textbf{Introduction} (Section \ref{sec:introduction}): Framework
  overview and research objectives
\item
  \textbf{Methodology} (Section \ref{sec:methodology}): Database
  structure, analysis methods, and House of Knowledge framework
\item
  \textbf{Experimental Results} (Section
  \ref{sec:experimental_results}): Statistical analysis of ways,
  distributions, and patterns
\item
  \textbf{Discussion} (Section \ref{sec:discussion}): Interpretation of
  findings and philosophical implications
\item
  \textbf{Conclusion} (Section \ref{sec:conclusion}): Summary and future
  directions
\end{enumerate}

Supplemental sections provide extended methodological details,
additional results, and detailed analysis of specific aspects of the
framework.

\subsection{Philosophical Context}\label{philosophical-context}

The Ways framework emerges from a deep engagement with questions of
epistemology, learning, and knowledge. It addresses fundamental
questions:

\begin{itemize}
\tightlist
\item
  How do we come to know things?
\item
  What are the different valid approaches to understanding?
\item
  How do belief, care, and learning interact in knowledge acquisition?
\item
  What role does dialogue play in figuring things out?
\item
  How can we systematically organize different approaches to knowledge?
\end{itemize}

The framework provides a comprehensive answer to these questions through
its systematic organization of 284 distinct ways, each representing a
valid approach to knowledge and understanding.

\subsection{Applications}\label{applications}

This research has applications across multiple domains:

\begin{itemize}
\tightlist
\item
  \textbf{Education}: Understanding different learning styles and
  approaches
\item
  \textbf{Research Methodology}: Systematic approaches to knowledge
  acquisition
\item
  \textbf{Personal Development}: Tools for understanding one's own ways
  of figuring things out
\item
  \textbf{Philosophy}: Contributions to epistemology and knowledge
  systems theory
\item
  \textbf{Interdisciplinary Studies}: Framework for understanding
  knowledge across domains
\end{itemize}

\subsection{Structure of This Work}\label{structure-of-this-work}

The following sections provide detailed analysis of the Ways framework.
Section \ref{sec:methodology} describes the database structure and
analysis methods. Section \ref{sec:experimental_results} presents
statistical findings and patterns. Section \ref{sec:discussion}
interprets these findings within the broader philosophical context.
Supplemental sections provide extended details on methodology,
additional results, and detailed analysis of specific aspects of the
framework.

\newpage

\section{Methodology}\label{sec:methodology}

\subsection{Database Structure and
Conversion}\label{database-structure-and-conversion}

\subsubsection{Source Data}\label{source-data}

The research draws on a MySQL database dump containing 11 tables
documenting Andrius Kulikauskas's Ways of Figuring Things Out framework.
The primary data table \passthrough{\lstinline!ways!} contains 210
documented ways with the following key fields (see Table
\ref{tab:ways_schema}):

\begin{itemize}
\tightlist
\item
  \passthrough{\lstinline!way!}: The name/identifier of the way
\item
  \passthrough{\lstinline!dialoguewith!}: The dialogue partner or
  conversant
\item
  \passthrough{\lstinline!dialoguetype!}: The type of dialogue
  (Absolute, Relative, Embrace God)
\item
  \passthrough{\lstinline!dialoguetypetype!}: Sub-type classification
\item
  \passthrough{\lstinline!mene!}: Room assignment in the House of
  Knowledge (24 rooms)
\item
  \passthrough{\lstinline!Dievas!}: Relationship to God/the divine
\item
  \passthrough{\lstinline!examples!}: Examples and descriptions
\item
  \passthrough{\lstinline!comments!}: Additional comments and notes
\end{itemize}

\begin{table}[h]
\centering
\begin{tabular}{|l|l|}
\hline
\textbf{Field} & \textbf{Description} \\
\hline
\texttt{way} & The name/identifier of the way \\
\texttt{dialoguewith} & The dialogue partner or conversant \\
\texttt{dialoguetype} & The type of dialogue (Absolute, Relative, Embrace God) \\
\texttt{dialoguetypetype} & Sub-type classification \\
\texttt{mene} & Room assignment in the House of Knowledge (24 rooms) \\
\texttt{Dievas} & Relationship to God/the divine \\
\texttt{examples} & Examples and descriptions \\
\texttt{comments} & Additional comments and notes \\
\hline
\end{tabular}
\caption{Key fields in the \texttt{ways} table schema}
\label{tab:ways_schema}
\end{table}

\subsubsection{SQLite Conversion}\label{sqlite-conversion}

For analysis and portability, the MySQL dump was converted to SQLite
format. The conversion process:

\begin{enumerate}
\def\labelenumi{\arabic{enumi}.}
\tightlist
\item
  \textbf{Schema Conversion}: MySQL-specific syntax (AUTO\_INCREMENT,
  ENGINE, COLLATE) converted to SQLite-compatible syntax
\item
  \textbf{Table Renaming}: Tables renamed for clarity
  (\passthrough{\lstinline!20100422ways!} →
  \passthrough{\lstinline!ways!}, \passthrough{\lstinline!menes!} →
  \passthrough{\lstinline!rooms!}, etc.)
\item
  \textbf{Index Handling}: Index names adjusted to avoid conflicts with
  table names (SQLite restriction)
\item
  \textbf{Data Preservation}: All data preserved during conversion with
  proper encoding handling
\end{enumerate}

The resulting SQLite database (\passthrough{\lstinline!db/ways.db!})
provides a portable, queryable format for analysis. The complete
database schema is documented in Section \ref{sec:appendix} (Appendix
A).

\subsubsection{Implementation Modules}\label{implementation-modules}

The analysis is implemented using several specialized modules in
\passthrough{\lstinline!project/src/!}:

\begin{itemize}
\tightlist
\item
  \textbf{\passthrough{\lstinline!database.py!}}: SQLAlchemy ORM models
  for Ways, Rooms, Questions, and database access
\item
  \textbf{\passthrough{\lstinline!sql\_queries.py!}}: Pre-built SQL
  queries for common analysis operations
\item
  \textbf{\passthrough{\lstinline!ways\_analysis.py!}}: High-level ways
  characterization and analysis functions
\item
  \textbf{\passthrough{\lstinline!network\_analysis.py!}}: Graph-based
  network analysis of way relationships
\item
  \textbf{\passthrough{\lstinline!house\_of\_knowledge.py!}}: Analysis
  of the 24-room House of Knowledge framework
\item
  \textbf{\passthrough{\lstinline!statistics.py!}}: Statistical analysis
  functions including
  \passthrough{\lstinline!analyze\_way\_distributions()!},
  \passthrough{\lstinline!compute\_way\_correlations()!},
  \passthrough{\lstinline!compute\_way\_diversity\_metrics()!}
\item
  \textbf{\passthrough{\lstinline!metrics.py!}}: Performance metrics
  including \passthrough{\lstinline!compute\_way\_coverage\_metrics()!},
  \passthrough{\lstinline!compute\_way\_interconnectedness()!}
\item
  \textbf{\passthrough{\lstinline!models.py!}}: Data classes and enums
  for type-safe data handling
\end{itemize}

\subsection{House of Knowledge
Framework}\label{house-of-knowledge-framework}

\subsubsection{24-Room Structure}\label{room-structure}

The Ways framework organizes knowledge into 24 rooms within the ``House
of Knowledge.'' Each room represents a different aspect of how we come
to know and understand:

\begin{equation}\label{eq:house_structure}
\text{House of Knowledge} = \{\text{Room}_1, \text{Room}_2, \ldots, \text{Room}_{24}\}
\end{equation}

The rooms are organized according to three fundamental structures:

\begin{enumerate}
\def\labelenumi{\arabic{enumi}.}
\tightlist
\item
  \textbf{Believing (1-2-3-4)}: Four levels of belief structure
\item
  \textbf{Caring (1-2-3-4)}: Four levels of care structure\\
\item
  \textbf{Relative Learning}: The cycle of taking a stand, following
  through, and reflecting
\end{enumerate}

\subsubsection{Room Categories}\label{room-categories}

Each way is assigned to one or more rooms via the
\passthrough{\lstinline!mene!} field, creating a mapping:

\begin{equation}\label{eq:way_room_mapping}
\text{Way}_i \mapsto \{\text{Room}_j : \text{Way}_i \text{ belongs to Room}_j\}
\end{equation}

This mapping enables analysis of how ways cluster within rooms and how
rooms relate to one another.

\subsection{Dialogue Type
Classification}\label{dialogue-type-classification}

\subsubsection{Three Main Types}\label{three-main-types}

Ways are classified according to three primary dialogue types:

\begin{enumerate}
\def\labelenumi{\arabic{enumi}.}
\tightlist
\item
  \textbf{Absolute}: Ways that reference absolute truth or structure
\item
  \textbf{Relative}: Ways that engage with relative perspectives
\item
  \textbf{Embrace God}: Ways that explicitly engage with the divine or
  transcendent
\end{enumerate}

The distribution of ways across dialogue types provides insight into the
balance of different epistemological approaches in the framework.

\subsubsection{Dialogue Type Analysis}\label{dialogue-type-analysis}

For each way \(w_i\), we extract:

\begin{equation}\label{eq:dialogue_type}
\text{Type}(w_i) \in \{\text{Absolute}, \text{Relative}, \text{Embrace God}\}
\end{equation}

This classification enables statistical analysis of type distributions
and relationships.

\subsection{Network Analysis
Methodology}\label{network-analysis-methodology}

\subsubsection{Graph Construction}\label{graph-construction}

We construct a weighted network graph \(G = (V, E, w)\) where:

\begin{itemize}
\tightlist
\item
  \textbf{Vertices \(V\)}: Each way \(w_i\) is a node \(v_i \in V\),
  with \(|V| = 210\)
\item
  \textbf{Edges \(E\)}: Connections between ways based on:

  \begin{itemize}
  \tightlist
  \item
    Shared dialogue partners (\passthrough{\lstinline!dialoguewith!}):
    \(e_{ij} \in E\) if
    \(\text{dialoguewith}(w_i) = \text{dialoguewith}(w_j)\)
  \item
    Shared room assignments (\passthrough{\lstinline!mene!}):
    \(e_{ij} \in E\) if \(\text{mene}(w_i) = \text{mene}(w_j)\)
  \item
    Similar dialogue types: \(e_{ij} \in E\) if
    \(\text{dialoguetype}(w_i) = \text{dialoguetype}(w_j)\)
  \item
    Question relationships (\passthrough{\lstinline!klausimobudai!}
    table): \(e_{ij} \in E\) if
    \(\exists q: (w_i, q) \in Q \land (w_j, q) \in Q\)
  \end{itemize}
\item
  \textbf{Edge weights \(w\)}: \(w(e_{ij}) \in \{0.6, 0.8, 1.0\}\) based
  on relationship type (type, partner, room respectively)
\end{itemize}

The resulting network contains \(|E| = 1,290\) edges connecting the 210
ways.

\subsubsection{Centrality Metrics}\label{centrality-metrics}

We compute several centrality metrics to identify important ways:

\textbf{Degree Centrality:} \begin{equation}\label{eq:degree_centrality}
C_D(v) = \frac{\deg(v)}{|V| - 1}
\end{equation}

\textbf{Betweenness Centrality:}
\begin{equation}\label{eq:betweenness_centrality}
C_B(v) = \sum_{s \neq v \neq t} \frac{\sigma_{st}(v)}{\sigma_{st}}
\end{equation}

where \(\sigma_{st}\) is the number of shortest paths from \(s\) to
\(t\), and \(\sigma_{st}(v)\) is the number of those paths passing
through \(v\).

\textbf{Clustering Coefficient:} \begin{equation}\label{eq:clustering}
C_C(v) = \frac{2e_v}{k_v(k_v - 1)}
\end{equation}

where \(e_v\) is the number of edges between neighbors of \(v\), and
\(k_v\) is the degree of \(v\).

\subsection{Statistical Analysis
Methods}\label{statistical-analysis-methods}

\subsubsection{Distribution Analysis}\label{distribution-analysis}

We analyze the distribution of ways across:

\begin{enumerate}
\def\labelenumi{\arabic{enumi}.}
\tightlist
\item
  \textbf{Dialogue Types}: Count and percentage by type, with 38
  distinct types observed
\item
  \textbf{Rooms}: Distribution across 24 rooms, with B2 containing the
  most ways (23)
\item
  \textbf{Dialogue Partners}: Frequency of conversants, with 196 unique
  partners
\item
  \textbf{God Relationships}: Distribution of
  \passthrough{\lstinline!Dievas!} values
\end{enumerate}

\subsubsection{Information-Theoretic
Metrics}\label{information-theoretic-metrics}

We compute Shannon entropy to quantify the diversity of distributions:

\begin{equation}\label{eq:entropy}
H(X) = -\sum_{i=1}^{k} p_i \log_2(p_i)
\end{equation}

where \(p_i\) is the proportion in category \(i\) and \(k\) is the
number of categories.

\textbf{Mutual Information} between dialogue types and rooms:

\begin{equation}\label{eq:mutual_info}
I(X;Y) = \sum_{x,y} p(x,y) \log_2 \frac{p(x,y)}{p(x)p(y)}
\end{equation}

This quantifies the strength of association between dialogue types and
room assignments.

\subsubsection{Cross-Tabulation}\label{cross-tabulation}

Cross-tabulation analysis examines relationships between:

\begin{itemize}
\tightlist
\item
  Dialogue type × Room assignment (visualized in Figure
  \ref{fig:type_room_heatmap})
\item
  Dialogue type × Dialogue partner
\item
  Room × God relationship
\end{itemize}

This reveals patterns in how different dimensions of the framework
relate, with the cross-tabulation matrix showing concentrations of ways
at specific type-room intersections.

\subsection{Text Analysis}\label{text-analysis}

\subsubsection{Way Descriptions}\label{way-descriptions}

For ways with text descriptions in \passthrough{\lstinline!ways.md!}, we
perform:

\begin{enumerate}
\def\labelenumi{\arabic{enumi}.}
\tightlist
\item
  \textbf{Keyword Extraction}: Identify key terms and concepts
\item
  \textbf{Theme Analysis}: Extract recurring themes
\item
  \textbf{Example Analysis}: Analyze examples to understand way
  applications
\item
  \textbf{Relationship Extraction}: Identify references to other ways or
  concepts
\end{enumerate}

\subsubsection{Philosophical Structure
Analysis}\label{philosophical-structure-analysis}

Text analysis also examines:

\begin{itemize}
\tightlist
\item
  How ways relate to the believing/caring/learning structures
\item
  References to the House of Knowledge framework
\item
  Connections to broader philosophical concepts
\end{itemize}

\subsection{Data Processing Pipeline}\label{data-processing-pipeline}

\subsubsection{Extraction}\label{extraction}

\begin{enumerate}
\def\labelenumi{\arabic{enumi}.}
\tightlist
\item
  \textbf{Database Query}: Extract ways data from SQLite database
\item
  \textbf{Text Parsing}: Parse \passthrough{\lstinline!ways.md!} for
  additional context
\item
  \textbf{Relationship Extraction}: Build network from relationship
  tables
\end{enumerate}

\subsubsection{Transformation}\label{transformation}

\begin{enumerate}
\def\labelenumi{\arabic{enumi}.}
\tightlist
\item
  \textbf{Normalization}: Standardize way names and categories
\item
  \textbf{Encoding}: Handle Lithuanian/English text encoding
\item
  \textbf{Cleaning}: Remove duplicates and handle missing data
\end{enumerate}

\subsubsection{Analysis}\label{analysis}

\begin{enumerate}
\def\labelenumi{\arabic{enumi}.}
\tightlist
\item
  \textbf{Statistical Computation}: Calculate distributions and metrics
\item
  \textbf{Network Construction}: Build graph structures
\item
  \textbf{Visualization Generation}: Create plots and network diagrams
\end{enumerate}

\subsection{Validation Framework}\label{validation-framework}

\subsubsection{Data Quality Checks}\label{data-quality-checks}

\begin{enumerate}
\def\labelenumi{\arabic{enumi}.}
\tightlist
\item
  \textbf{Completeness}: Verify all ways have required fields
\item
  \textbf{Consistency}: Check for conflicting assignments
\item
  \textbf{Referential Integrity}: Validate room and relationship
  references
\end{enumerate}

\subsubsection{Analysis Validation}\label{analysis-validation}

\begin{enumerate}
\def\labelenumi{\arabic{enumi}.}
\tightlist
\item
  \textbf{Reproducibility}: Ensure analyses are reproducible
\item
  \textbf{Sensitivity}: Test sensitivity to data variations
\item
  \textbf{Robustness}: Verify results are robust to missing data
\end{enumerate}

\subsection{SQL Query Examples}\label{sql-query-examples}

Key analyses are performed using SQL queries against the SQLite
database. Example queries include:

\textbf{Dialogue Type Distribution:}

\begin{lstlisting}[language=SQL]
SELECT dialoguetype, COUNT(*) as count
FROM ways
GROUP BY dialoguetype
ORDER BY count DESC;
\end{lstlisting}

\textbf{Room-Way Cross-Tabulation:}

\begin{lstlisting}[language=SQL]
SELECT dialoguetype, mene, COUNT(*) as count
FROM ways
WHERE mene != '' AND dialoguetype != ''
GROUP BY dialoguetype, mene
ORDER BY count DESC;
\end{lstlisting}

\textbf{Network Edge Construction (Room-based):}

\begin{lstlisting}[language=SQL]
SELECT w1.ID as way1_id, w2.ID as way2_id
FROM ways w1
JOIN ways w2 ON w1.mene = w2.mene
WHERE w1.ID < w2.ID AND w1.mene != '';
\end{lstlisting}

\textbf{Central Ways Identification:}

\begin{lstlisting}[language=SQL]
SELECT way, COUNT(*) as connection_count
FROM (
    SELECT w1.way, w2.ID
    FROM ways w1
    JOIN ways w2 ON w1.mene = w2.mene
    WHERE w1.ID != w2.ID AND w1.mene != ''
    UNION
    SELECT w1.way, w2.ID
    FROM ways w1
    JOIN ways w2 ON w1.dialoguewith = w2.dialoguewith
    WHERE w1.ID != w2.ID AND w1.dialoguewith != ''
)
GROUP BY way
ORDER BY connection_count DESC
LIMIT 10;
\end{lstlisting}

\subsection{Implementation}\label{implementation}

The analysis is implemented using several specialized Python modules:

\subsubsection{Core Analysis Modules}\label{core-analysis-modules}

\begin{itemize}
\tightlist
\item
  \textbf{\passthrough{\lstinline!database.py!}}: SQLAlchemy ORM with
  \passthrough{\lstinline!WaysDatabase!} class for database access
\item
  \textbf{\passthrough{\lstinline!sql\_queries.py!}}:
  \passthrough{\lstinline!WaysSQLQueries!} class with pre-built analysis
  queries
\item
  \textbf{\passthrough{\lstinline!ways\_analysis.py!}}:
  \passthrough{\lstinline!WaysAnalyzer!} class for comprehensive ways
  characterization
\item
  \textbf{\passthrough{\lstinline!network\_analysis.py!}}:
  \passthrough{\lstinline!WaysNetworkAnalyzer!} class for graph-based
  relationship analysis
\item
  \textbf{\passthrough{\lstinline!house\_of\_knowledge.py!}}: Framework
  analysis for the 24-room House of Knowledge
\item
  \textbf{\passthrough{\lstinline!statistics.py!}}: Statistical
  functions including
  \passthrough{\lstinline!analyze\_way\_distributions()!},
  \passthrough{\lstinline!compute\_way\_correlations()!}
\item
  \textbf{\passthrough{\lstinline!metrics.py!}}: Performance metrics
  including \passthrough{\lstinline!compute\_way\_coverage\_metrics()!},
  \passthrough{\lstinline!compute\_way\_interconnectedness()!}
\end{itemize}

\subsubsection{Infrastructure}\label{infrastructure}

\begin{itemize}
\tightlist
\item
  \textbf{Python 3.10+}: Primary analysis language
\item
  \textbf{SQLite}: Database backend via SQLAlchemy ORM
\item
  \textbf{NetworkX}: Network analysis and graph algorithms
\item
  \textbf{Matplotlib/Seaborn}: Statistical visualization and plotting
\item
  \textbf{NumPy/Pandas}: Numerical computing and data manipulation
\end{itemize}

All code follows the thin orchestrator pattern, with business logic in
\passthrough{\lstinline!project/src/!} modules and orchestration in
\passthrough{\lstinline!project/scripts/!}.

\subsection{Ethical Considerations}\label{ethical-considerations}

This research documents and analyzes publicly available philosophical
work by Andrius Kulikauskas. All data is in the public domain as stated
in the source documentation. The analysis respects the original
philosophical framework while providing systematic documentation and
quantitative insights.

\newpage

\section{Experimental Results}\label{sec:experimental_results}

\subsection{Database Overview}\label{database-overview}

\subsubsection{Data Summary}\label{data-summary-1}

The analysis database contains:

\begin{itemize}
\tightlist
\item
  \textbf{210 documented ways} in the primary
  \passthrough{\lstinline!ways!} table
\item
  \textbf{24 rooms} in the House of Knowledge
  (\passthrough{\lstinline!rooms!} table)
\item
  \textbf{Multiple examples} per way (\passthrough{\lstinline!examples!}
  table)
\item
  \textbf{Question-way relationships}
  (\passthrough{\lstinline!klausimobudai!} table)
\item
  \textbf{Dialogue partner information} for each way
\end{itemize}

\subsubsection{Data Completeness}\label{data-completeness}

Analysis of data completeness reveals:

\begin{itemize}
\tightlist
\item
  All 210 ways have dialogue type assignments
\item
  Room assignments (\passthrough{\lstinline!mene!}) present for majority
  of ways
\item
  Dialogue partner (\passthrough{\lstinline!dialoguewith!}) information
  available for most ways
\item
  Examples and descriptions vary in completeness
\end{itemize}

\subsection{Distribution Analysis}\label{distribution-analysis-1}

\subsubsection{Dialogue Type
Distribution}\label{dialogue-type-distribution}

Analysis of ways by dialogue type reveals the distribution across the
three main categories:

\begin{table}[h]
\centering
\begin{tabular}{|l|c|c|}
\hline
\textbf{Dialogue Type} & \textbf{Count} & \textbf{Percentage} \\
\hline
goodness & 15 & 7.1\% \\
other & 15 & 7.1\% \\
regularity & 11 & 5.2\% \\
I & 9 & 4.3\% \\
answer & 9 & 4.3\% \\
knowledge & 8 & 3.8\% \\
life & 8 & 3.8\% \\
mind & 8 & 3.8\% \\
my mind & 7 & 3.3\% \\
opposing view & 7 & 3.3\% \\
\hline
\textbf{Total} & 210 & 100\% \\
\hline
\end{tabular}
\caption{Distribution of ways by dialogue type (top 10)}
\label{tab:dialogue_type_distribution}
\end{table}

The complete distribution is visualized in Figure
\ref{fig:type_distribution}, showing the full range of 38 distinct
dialogue types.

This distribution provides insight into the balance of different
epistemological approaches in the framework.

\subsubsection{Room Distribution}\label{room-distribution}

Analysis of ways across the 24 rooms of the House of Knowledge reveals:

\begin{table}[h]
\centering
\begin{tabular}{|l|c|c|}
\hline
\textbf{Room} & \textbf{Way Count} & \textbf{Percentage} \\
\hline
B2 & 23 & 11.0\% \\
C4 & 17 & 8.1\% \\
R & 16 & 7.6\% \\
32 & 13 & 6.2\% \\
C3 & 13 & 6.2\% \\
BB & 12 & 5.7\% \\
CB & 10 & 4.8\% \\
21 & 9 & 4.3\% \\
B3 & 9 & 4.3\% \\
CC & 9 & 4.3\% \\
O & 9 & 4.3\% \\
T & 9 & 4.3\% \\
10 & 8 & 3.8\% \\
31 & 8 & 3.8\% \\
1 & 7 & 3.3\% \\
\hline
\textbf{Total} & 210 & 100\% \\
\hline
\end{tabular}
\caption{Distribution of ways across top 15 rooms}
\label{tab:room_distribution}
\end{table}

The complete room hierarchy is visualized in Figure
\ref{fig:room_hierarchy}, and the framework structure is shown in Figure
\ref{fig:framework_treemap}.

Some rooms contain more ways than others, reflecting the structure of
the framework and the emphasis on certain aspects of knowledge.

\subsection{Network Analysis Results}\label{network-analysis-results}

\subsubsection{Network Structure}\label{network-structure}

The network graph constructed from way relationships exhibits:

\begin{itemize}
\tightlist
\item
  \textbf{Nodes}: 210 ways
\item
  \textbf{Edges}: 1,290 connections
\item
  \textbf{Average degree}: 12.29 connections per way
\item
  \textbf{Network density}: 0.058 (5.8\% of possible edges present)
\item
  \textbf{Clustering coefficient}: 0.886 (high local clustering,
  indicating strong room-based clustering)
\item
  \textbf{Connected components}: Multiple components with largest
  containing majority of ways
\item
  \textbf{Network visualization}: See Figure \ref{fig:ways_network}
\end{itemize}

The network structure reveals both local clustering (ways in the same
room are highly connected) and long-range connections (ways sharing
dialogue types or partners across different rooms).

\subsubsection{Central Ways}\label{central-ways}

Centrality analysis identifies ways that serve as hubs or bridges:

\begin{table}[h]
\centering
\begin{tabular}{|l|c|c|}
\hline
\textbf{Way ID} & \textbf{Degree Centrality} & \textbf{Room} \\
\hline
84, 156, 211 & 34 & Multiple rooms \\
115 & 30 & Multiple rooms \\
120 & 25 & Multiple rooms \\
\hline
\end{tabular}
\caption{Most central ways by degree centrality (top 5)}
\label{tab:central_ways}
\end{table}

These central ways serve as hubs connecting multiple other ways through
shared rooms, dialogue types, or partners. The complete network
structure is visualized in Figure \ref{fig:ways_network}, showing the
clustering and connectivity patterns.

\subsubsection{Community Detection}\label{community-detection}

Community detection algorithms reveal clusters of related ways:

\begin{itemize}
\tightlist
\item
  \textbf{Cluster 1}: Ways related to goodness and morality (15 ways)
\item
  \textbf{Cluster 2}: Ways related to regularity and structure (11 ways)
\item
  \textbf{Cluster 3}: Ways related to personal identity and ``I'' (9
  ways)
\end{itemize}

These clusters may correspond to different aspects of the House of
Knowledge or different dialogue types.

\subsection{Cross-Tabulation Analysis}\label{cross-tabulation-analysis}

\subsubsection{Dialogue Type × Room}\label{dialogue-type-room}

Cross-tabulation of dialogue types and room assignments reveals patterns
(visualized in Figure \ref{fig:type_room_heatmap}):

\begin{table}[h]
\centering
\begin{tabular}{|l|c|c|}
\hline
\textbf{Type × Room} & \textbf{Count} & \textbf{Notes} \\
\hline
goodness × B2 & 15 & Believing framework \\
goodness × C4 & 17 & Caring framework \\
other × B2 & 15 & Primary combination \\
regularity × BB & 11 & Strong association \\
I × CC & 9 & Identity-focused \\
life × R & 8 & Life-related ways \\
mind × 10 & Cognitive approaches \\
\hline
\end{tabular}
\caption{Top cross-tabulations of dialogue types and rooms}
\label{tab:type_room_crosstab}
\end{table}

The heatmap visualization (Figure \ref{fig:type_room_heatmap}) reveals
strong associations between certain dialogue types and specific rooms,
indicating structural relationships in the framework. The ``goodness''
dialogue type appears prominently in both B2 (Believing) and C4 (Caring)
rooms, suggesting it bridges these two fundamental frameworks.

\subsubsection{Dialogue Partner
Analysis}\label{dialogue-partner-analysis}

Analysis of dialogue partners (\passthrough{\lstinline!dialoguewith!})
reveals:

\begin{itemize}
\tightlist
\item
  \textbf{Most common partners}: life, limits of my mind, circumstances,
  science, purpose, answer, people's inclinations, possibility,
  goodness, meaningfulness (all with 2 ways each)
\item
  \textbf{Partner diversity}: 116 unique partners
\item
  \textbf{Partner-way relationships}: Most partners connect exactly 2
  ways, indicating pairwise relationships
\end{itemize}

Some dialogue partners appear frequently across multiple ways,
suggesting they represent important perspectives or approaches.

\subsection{Statistical Patterns}\label{statistical-patterns}

\subsubsection{Room Co-occurrence}\label{room-co-occurrence}

Analysis of ways assigned to multiple rooms reveals:

\begin{itemize}
\tightlist
\item
  \textbf{Average rooms per way}: 1.0 (each way assigned to exactly one
  room)
\item
  \textbf{Most common room pairs}: N/A (single room assignments)
\item
  \textbf{Room clusters}: Rooms B2, C4, R, C3, 32 contain the highest
  concentrations of ways
\end{itemize}

This indicates how different aspects of knowledge relate to one another
in the framework.

\subsubsection{Dialogue Type Patterns}\label{dialogue-type-patterns}

Statistical analysis of dialogue type patterns shows:

\begin{itemize}
\tightlist
\item
  \textbf{Type transitions}: How ways of one type relate to ways of
  another
\item
  \textbf{Type clusters}: Groups of ways with similar type
  characteristics
\item
  \textbf{Type diversity}: Distribution of types within rooms and
  categories
\end{itemize}

\subsection{Text Analysis Results}\label{text-analysis-results}

\subsubsection{Keyword Extraction}\label{keyword-extraction}

Analysis of way descriptions and examples reveals common themes:

\begin{itemize}
\tightlist
\item
  \textbf{Top keywords}: goodness, regularity, other, I, answer (from
  dialogue types)
\item
  \textbf{Keyword clusters}: Philosophical concepts, personal
  relationships, structural patterns
\item
  \textbf{Keyword-room associations}: B2 room associated with ``other'',
  BB room with ``regularity''
\end{itemize}

\subsubsection{Example Analysis}\label{example-analysis}

Analysis of examples reveals:

\begin{itemize}
\tightlist
\item
  \textbf{Common example types}: Personal experiences, philosophical
  reflections, practical applications
\item
  \textbf{Example patterns}: Ways often illustrated through personal
  anecdotes and thought processes
\item
  \textbf{Example-way relationships}: Examples provide concrete
  illustrations of abstract ways of figuring things out
\end{itemize}

\subsection{Visualization Results}\label{visualization-results}

\subsubsection{Network Graph}\label{network-graph}

The network visualization (Figure \ref{fig:ways_network}) shows:

\begin{itemize}
\tightlist
\item
  Ways as nodes, colored by dialogue type
\item
  Connections as edges, weighted by relationship strength
\item
  Clusters visible as dense regions
\item
  Central ways as highly connected nodes
\end{itemize}

\begin{figure}[h]
\centering
\includegraphics[width=0.9\textwidth]{../figures/ways_network.png}
\caption{Network graph of ways showing connections and clusters}
\label{fig:ways_network}
\end{figure}

\subsubsection{Room Distribution}\label{room-distribution-1}

A hierarchical visualization (Figure \ref{fig:room_hierarchy}) shows:

\begin{itemize}
\tightlist
\item
  The 24-room structure
\item
  Way counts per room
\item
  Relationships between rooms
\end{itemize}

\begin{figure}[h]
\centering
\includegraphics[width=0.9\textwidth]{../figures/room_hierarchy.png}
\caption{Hierarchical visualization of the House of Knowledge structure}
\label{fig:room_hierarchy}
\end{figure}

\subsubsection{Statistical
Distributions}\label{statistical-distributions}

Distribution plots show:

\begin{itemize}
\tightlist
\item
  Dialogue type frequencies (Figure \ref{fig:type_distribution})
\item
  Room assignment patterns (Figure \ref{fig:room_hierarchy})
\item
  Framework structure (Figure \ref{fig:framework_treemap})
\item
  Dialogue partner frequencies (Figure \ref{fig:partner_wordcloud})
\item
  Example length distributions by type (Figure
  \ref{fig:example_length_violin})
\end{itemize}

\begin{figure}[h]
\centering
\includegraphics[width=0.9\textwidth]{../figures/type_distribution.png}
\caption{Distribution of ways by dialogue type}
\label{fig:type_distribution}
\end{figure}

\subsubsection{Cross-Tabulation Heatmap}\label{cross-tabulation-heatmap}

The dialogue type × room cross-tabulation matrix (Figure
\ref{fig:type_room_heatmap}) reveals concentration patterns:

\begin{figure}[h]
\centering
\includegraphics[width=0.9\textwidth]{../figures/type_room_heatmap.png}
\caption{Heatmap showing dialogue type × room cross-tabulation}
\label{fig:type_room_heatmap}
\end{figure}

\subsubsection{Framework Structure}\label{framework-structure}

The framework hierarchy visualization (Figure
\ref{fig:framework_treemap}) shows the distribution of ways across the
main philosophical frameworks:

\begin{figure}[h]
\centering
\includegraphics[width=0.9\textwidth]{../figures/framework_treemap.png}
\caption{Hierarchical visualization of framework distribution}
\label{fig:framework_treemap}
\end{figure}

\subsubsection{Dialogue Partners}\label{dialogue-partners}

The dialogue partner frequency distribution (Figure
\ref{fig:partner_wordcloud}) shows the diversity of conversants:

\begin{figure}[h]
\centering
\includegraphics[width=0.9\textwidth]{../figures/partner_wordcloud.png}
\caption{Dialogue partner frequency distribution}
\label{fig:partner_wordcloud}
\end{figure}

\subsubsection{Example Length Analysis}\label{example-length-analysis}

The distribution of example lengths by dialogue type (Figure
\ref{fig:example_length_violin}) reveals patterns in how ways are
documented:

\begin{figure}[h]
\centering
\includegraphics[width=0.9\textwidth]{../figures/example_length_violin.png}
\caption{Example length distribution by dialogue type}
\label{fig:example_length_violin}
\end{figure}

\subsection{Key Findings}\label{key-findings}

\subsubsection{Structural Patterns}\label{structural-patterns}

\begin{enumerate}
\def\labelenumi{\arabic{enumi}.}
\tightlist
\item
  \textbf{Room Clustering}: Ways cluster within certain rooms,
  indicating focused approaches to specific aspects of knowledge
\item
  \textbf{Type Balance}: The distribution across dialogue types reflects
  the framework's emphasis on different epistemological approaches
\item
  \textbf{Network Structure}: The network exhibits small-world
  properties with both local clustering and long-range connections
\end{enumerate}

\subsubsection{Central Ways}\label{central-ways-1}

Certain ways serve as central nodes, connecting different parts of the
framework. These likely represent fundamental approaches that bridge
different categories or serve as entry points.

\subsubsection{Room Relationships}\label{room-relationships}

Analysis reveals relationships between rooms, showing how different
aspects of knowledge relate. Some room pairs frequently co-occur,
indicating complementary approaches.

\subsection{Limitations}\label{limitations}

\subsubsection{Data Completeness}\label{data-completeness-1}

\begin{itemize}
\tightlist
\item
  Not all ways have complete metadata
\item
  Some room assignments may be missing
\item
  Dialogue partner information varies in completeness
\end{itemize}

\subsubsection{Analysis Scope}\label{analysis-scope}

\begin{itemize}
\tightlist
\item
  Analysis focuses on documented ways (212 of 284 total)
\item
  Text analysis limited to available descriptions
\item
  Network analysis based on explicit relationships in database
\end{itemize}

\subsection{Future Analysis
Directions}\label{future-analysis-directions}

Future work will:

\begin{enumerate}
\def\labelenumi{\arabic{enumi}.}
\tightlist
\item
  Complete analysis of all 284 ways
\item
  Expand text analysis with natural language processing
\item
  Develop predictive models for way categorization
\item
  Create interactive visualizations
\item
  Analyze temporal patterns if dating information available
\end{enumerate}

\newpage

\section{Discussion}\label{sec:discussion}

\subsection{Interpretation of
Findings}\label{interpretation-of-findings}

The systematic analysis of Andrius Kulikauskas's Ways of Figuring Things
Out framework reveals several important patterns and insights into how
different approaches to knowledge are structured and interrelated.

\subsubsection{Framework Structure}\label{framework-structure-1}

The 24-room House of Knowledge provides a comprehensive organizational
structure for understanding different ways of figuring things out. The
distribution of ways across rooms reveals significant non-uniformity:
the B2 room (Believing in Believing) contains 23 ways (11.0\%), followed
by C4 (Caring about Caring about Caring about Caring) with 17 ways
(8.1\%), and R (Reflecting) with 16 ways (7.6\%). This concentration
suggests that certain aspects of knowledge---particularly the recursive
structures of believing and caring, and the reflective learning
process---are more amenable to multiple approaches, while other rooms
have fewer distinct ways.

The three fundamental structures---Believing (1-2-3-4), Caring
(1-2-3-4), and Relative Learning---provide a philosophical foundation
that organizes the rooms. The ways distributed across these structures
reflect different epistemological approaches, from absolute belief
structures to relative learning cycles.

\subsubsection{Dialogue Type Patterns}\label{dialogue-type-patterns-1}

The distribution of ways across 38 distinct dialogue types reveals
important patterns: ``goodness'' and ``other'' each account for 15 ways
(7.1\% each), followed by ``regularity'' (11 ways, 5.2\%), ``I'' and
``answer'' (9 ways each, 4.3\%). This distribution shows no single
dominant type, suggesting a balanced epistemological perspective that
values multiple approaches. The cross-tabulation analysis (Figure
\ref{fig:type_room_heatmap}) reveals strong associations: ``goodness''
appears prominently in both B2 (Believing) and C4 (Caring) rooms,
indicating it bridges these fundamental frameworks. This pattern
suggests that moral and ethical considerations (``goodness'') are
central to both believing and caring structures.

The dialogue type classification reflects different relationships to
truth and knowledge. While the framework includes Absolute, Relative,
and Embrace God perspectives, the actual distribution shows 38 distinct
dialogue types, with the most common being ``goodness'' and ``other''
(15 each). This diversity suggests that the framework recognizes
multiple valid ways of engaging with knowledge beyond the three primary
categories. The ``goodness'' type's prominence in both Believing (B2)
and Caring (C4) rooms indicates that ethical considerations are
fundamental to both frameworks, while ``other'' suggests ways that don't
fit neatly into standard categories, reflecting the framework's openness
to diverse approaches.

\subsubsection{Network Structure
Insights}\label{network-structure-insights}

The network analysis reveals a highly connected structure with 1,290
edges connecting 210 ways, resulting in an average degree of 12.29
connections per way and a clustering coefficient of 0.886. The network
exhibits both local clustering (ways in the same room are highly
connected) and long-range connections (ways sharing dialogue types or
partners across different rooms). Central ways with degree centrality of
34 (ways 84, 156, 211) serve as major hubs, connecting multiple other
ways through shared rooms, dialogue types, or partners. These central
ways likely represent fundamental methods that connect different
categories or serve as entry points to the framework, as visualized in
Figure \ref{fig:ways_network}.

The clustering observed in the network indicates that ways group into
communities based on shared characteristics. These clusters may
correspond to: - Different aspects of the House of Knowledge - Different
dialogue types - Different philosophical approaches - Different
practical applications

The small-world properties (local clustering with long-range
connections) suggest that while ways cluster locally, there are also
important connections across clusters, creating a rich, interconnected
structure.

\subsection{Philosophical
Implications}\label{philosophical-implications}

\subsubsection{Epistemological
Pluralism}\label{epistemological-pluralism}

The framework demonstrates epistemological pluralism---the recognition
that there are multiple valid ways of knowing and understanding. The 284
ways represent a comprehensive catalog of approaches, each valid in its
own context. This pluralism challenges monolithic views of knowledge and
suggests that different situations and questions may require different
approaches.

The organization into rooms and dialogue types provides a structure for
understanding when and how different ways are appropriate. Rather than
suggesting one ``correct'' way, the framework provides a map of options,
each with its own validity and application.

\subsubsection{Integration of Belief, Care, and
Learning}\label{integration-of-belief-care-and-learning}

The framework integrates three fundamental aspects of knowledge: -
\textbf{Believing}: Reference to absolute structures or truths -
\textbf{Caring}: Openness to what is outside us - \textbf{Learning}: The
cycle of taking a stand, following through, and reflecting

This integration suggests that complete knowledge requires all three
aspects. Ways that emphasize only one aspect may be incomplete, while
ways that integrate multiple aspects may be more comprehensive. The
distribution of ways across these structures reflects the framework's
recognition of their interdependence.

\subsubsection{Dialogue and Knowledge}\label{dialogue-and-knowledge}

The emphasis on dialogue partners
(\passthrough{\lstinline!dialoguewith!}) suggests that knowledge is not
purely individual but emerges through engagement with others. Each way
involves a dialogue partner, indicating that figuring things out is
fundamentally relational. This relational aspect challenges purely
individualistic views of knowledge and suggests that understanding
emerges through engagement with different perspectives.

The dialogue types (Absolute, Relative, Embrace God) represent different
modes of engagement, each valid in different contexts. The framework
suggests that effective knowledge acquisition requires understanding
which mode of dialogue is appropriate for which situation.

\subsection{Practical Applications}\label{practical-applications}

\subsubsection{Educational Contexts}\label{educational-contexts}

The framework has clear applications in education:

\begin{enumerate}
\def\labelenumi{\arabic{enumi}.}
\tightlist
\item
  \textbf{Learning Style Recognition}: Understanding that different
  students may prefer different ways of figuring things out
\item
  \textbf{Teaching Methods}: Adapting teaching to match different ways
\item
  \textbf{Curriculum Design}: Organizing curriculum to expose students
  to multiple ways
\item
  \textbf{Assessment}: Recognizing that different ways may require
  different assessment methods
\end{enumerate}

The 24-room structure provides a framework for organizing educational
content and approaches, ensuring coverage of different aspects of
knowledge.

\subsubsection{Research Methodology}\label{research-methodology}

For researchers, the framework provides:

\begin{enumerate}
\def\labelenumi{\arabic{enumi}.}
\tightlist
\item
  \textbf{Method Selection}: A systematic way to choose appropriate
  research methods
\item
  \textbf{Method Integration}: Understanding how different methods
  complement each other
\item
  \textbf{Epistemological Awareness}: Recognition of the epistemological
  assumptions underlying different methods
\item
  \textbf{Interdisciplinary Bridge}: A framework for understanding
  knowledge across disciplines
\end{enumerate}

The network structure helps researchers understand how different methods
relate and when to combine approaches.

\subsubsection{Personal Development}\label{personal-development}

For individuals, the framework offers:

\begin{enumerate}
\def\labelenumi{\arabic{enumi}.}
\tightlist
\item
  \textbf{Self-Understanding}: Recognizing one's own preferred ways of
  figuring things out
\item
  \textbf{Expansion}: Learning new ways to expand one's capabilities
\item
  \textbf{Context Awareness}: Understanding which ways are appropriate
  for which situations
\item
  \textbf{Integration}: Developing the ability to use multiple ways as
  needed
\end{enumerate}

The House of Knowledge structure provides a map for personal growth,
showing areas where one might develop new ways of understanding.

\subsection{Limitations and
Challenges}\label{limitations-and-challenges}

\subsubsection{Framework Completeness}\label{framework-completeness}

While the framework is comprehensive (284 ways), it may not be
exhaustive. New ways may emerge as knowledge evolves, or ways may be
discovered that don't fit the current structure. The framework should be
seen as a living system that can grow and adapt.

\subsubsection{Cultural Context}\label{cultural-context}

The framework emerges from a specific cultural and philosophical context
(Andrius Kulikauskas's work). While it aims for universality, some ways
may be more relevant in certain cultural contexts than others. The
framework's applicability across cultures requires further
investigation.

\subsubsection{Measurement Challenges}\label{measurement-challenges}

Quantitative analysis of ways faces challenges: - Ways are qualitative
and may resist precise measurement - Relationships between ways may be
complex and multi-dimensional - The framework's philosophical nature
makes some aspects difficult to quantify

These challenges suggest that quantitative analysis should complement,
not replace, qualitative understanding.

\subsection{Future Research
Directions}\label{future-research-directions}

\subsubsection{Framework Expansion}\label{framework-expansion}

Future research could: 1. Document additional ways beyond the current
284 2. Explore ways from other philosophical traditions 3. Investigate
ways in specific domains (science, art, etc.) 4. Develop ways for
emerging contexts (digital, global, etc.)

\subsubsection{Empirical Validation}\label{empirical-validation}

Empirical research could: 1. Test the effectiveness of different ways in
different contexts 2. Investigate individual differences in way
preferences 3. Study how ways develop and change over time 4. Examine
the relationship between ways and learning outcomes

\subsubsection{Computational
Applications}\label{computational-applications}

Computational research could: 1. Develop AI systems that use different
ways 2. Create recommendation systems for way selection 3. Build tools
for way analysis and visualization 4. Develop educational software based
on the framework

\subsubsection{Interdisciplinary
Integration}\label{interdisciplinary-integration}

The framework could be integrated with: 1. Cognitive science research on
learning 2. Educational research on teaching methods 3. Philosophy of
science and epistemology 4. Knowledge management and organizational
learning

\subsection{Broader Impact}\label{broader-impact}

\subsubsection{Contribution to
Epistemology}\label{contribution-to-epistemology}

The framework contributes to epistemology by: 1. Providing a
comprehensive catalog of ways of knowing 2. Showing the relationships
between different approaches 3. Demonstrating the validity of multiple
perspectives 4. Integrating belief, care, and learning in knowledge
acquisition

\subsubsection{Contribution to
Education}\label{contribution-to-education}

The framework contributes to education by: 1. Providing a systematic
approach to understanding learning 2. Recognizing the validity of
multiple learning approaches 3. Offering a structure for curriculum and
teaching 4. Supporting personalized and adaptive education

\subsubsection{Contribution to Research}\label{contribution-to-research}

The framework contributes to research by: 1. Providing a systematic
approach to method selection 2. Showing how different methods relate and
complement 3. Encouraging epistemological awareness 4. Supporting
interdisciplinary research

\subsection{Conclusion}\label{conclusion}

The systematic analysis of the Ways of Figuring Things Out framework
reveals a rich, structured approach to understanding knowledge
acquisition. The 24-room House of Knowledge provides organization, the
dialogue types reveal different modes of engagement, and the network
structure shows how ways interconnect. The framework demonstrates
epistemological pluralism while providing structure for understanding
when and how different ways are appropriate.

The practical applications span education, research, and personal
development, offering tools for understanding and applying different
approaches to knowledge. Future research can expand the framework,
validate it empirically, and develop computational and interdisciplinary
applications.

This work provides both a philosophical framework and a practical system
for understanding and applying diverse ways of figuring things out,
contributing to epistemology, education, and research methodology.

\newpage

\section{Conclusion}\label{sec:conclusion}

\subsection{Summary of Contributions}\label{summary-of-contributions}

This research presents a comprehensive systematic analysis of Andrius
Kulikauskas's ``Ways of Figuring Things Out'' framework, documenting and
analyzing 210 ways from the database (with connections to the broader
framework of 284 ways documented in the source text). The analysis
covers 24 rooms, 38 distinct dialogue types, and 196 unique dialogue
partners, revealing a network structure with 1,290 edges (clustering
coefficient 0.886) connecting ways through shared characteristics. The
work makes several key contributions:

\subsubsection{Documentation and
Categorization}\label{documentation-and-categorization}

\begin{enumerate}
\def\labelenumi{\arabic{enumi}.}
\tightlist
\item
  \textbf{Complete Documentation}: Systematic documentation of 210 ways
  from the database with complete metadata including dialogue types,
  room assignments, examples, and relationships
\item
  \textbf{24-Room Framework}: Organization of ways within the House of
  Knowledge structure, mapping ways to their appropriate rooms
\item
  \textbf{Dialogue Type Classification}: Categorization of ways
  according to Absolute, Relative, and Embrace God dialogue types
\item
  \textbf{Relationship Mapping}: Documentation of how ways relate
  through dialogue partners, shared rooms, and question relationships
\end{enumerate}

\subsubsection{Empirical Analysis}\label{empirical-analysis}

\begin{enumerate}
\def\labelenumi{\arabic{enumi}.}
\tightlist
\item
  \textbf{Distribution Analysis}: Quantitative analysis of way
  distributions across dialogue types, rooms, and categories
\item
  \textbf{Network Analysis}: Graph-based analysis revealing the network
  structure of way relationships
\item
  \textbf{Statistical Patterns}: Identification of patterns in room
  co-occurrence, dialogue type distributions, and central ways
\item
  \textbf{Cross-Tabulation}: Analysis of relationships between different
  dimensions of the framework
\end{enumerate}

\subsubsection{Framework Understanding}\label{framework-understanding}

\begin{enumerate}
\def\labelenumi{\arabic{enumi}.}
\tightlist
\item
  \textbf{Structural Insights}: Understanding of how the 24-room House
  of Knowledge organizes different aspects of knowledge
\item
  \textbf{Philosophical Integration}: Recognition of how Believing,
  Caring, and Relative Learning structures integrate
\item
  \textbf{Epistemological Pluralism}: Demonstration of multiple valid
  approaches to knowledge
\item
  \textbf{Practical Applications}: Tools and frameworks for applying
  ways in education, research, and personal development
\end{enumerate}

\subsection{Key Findings}\label{key-findings-1}

\subsubsection{Framework Structure}\label{framework-structure-2}

The analysis reveals that the Ways framework is not uniform but exhibits
structured patterns: - Ways cluster within certain rooms: B2 (23 ways,
11.0\%), C4 (17 ways, 8.1\%), R (16 ways, 7.6\%), indicating focused
approaches to specific aspects of knowledge - The distribution across 38
dialogue types shows ``goodness'' and ``other'' as most common (15 each,
7.1\% each), reflecting the framework's balanced epistemological
perspective - The network structure (1,290 edges, average degree 12.29,
clustering coefficient 0.886) shows both high local clustering
(room-based) and long-range connections (type and partner-based),
creating a rich, interconnected system with small-world properties

\subsubsection{Central Ways}\label{central-ways-2}

Certain ways serve as central nodes in the network, connecting different
parts of the framework. These central ways likely represent: -
Fundamental approaches that bridge different categories - Entry points
to the framework for new learners - Methods that integrate multiple
aspects of knowledge

\subsubsection{Room Relationships}\label{room-relationships-1}

Analysis reveals relationships between rooms, showing how different
aspects of knowledge relate: - Some room pairs frequently co-occur,
indicating complementary approaches - The three fundamental structures
(Believing, Caring, Learning) provide organization - The 24-room
structure provides comprehensive coverage of knowledge aspects

\subsection{Broader Impact}\label{broader-impact-1}

\subsubsection{Contribution to
Epistemology}\label{contribution-to-epistemology-1}

This work contributes to epistemology by:

\begin{itemize}
\tightlist
\item
  Providing a comprehensive catalog of ways of knowing
\item
  Demonstrating the validity of multiple epistemological approaches
\item
  Showing how different ways relate and complement each other
\item
  Integrating belief, care, and learning in knowledge acquisition
\end{itemize}

\subsubsection{Contribution to
Education}\label{contribution-to-education-1}

The framework contributes to education by:

\begin{itemize}
\tightlist
\item
  Providing a systematic approach to understanding learning
\item
  Recognizing the validity of multiple learning approaches
\item
  Offering structure for curriculum and teaching methods
\item
  Supporting personalized and adaptive education
\end{itemize}

\subsubsection{Contribution to
Research}\label{contribution-to-research-1}

For researchers, the framework provides:

\begin{itemize}
\tightlist
\item
  A systematic approach to method selection
\item
  Understanding of how different methods relate
\item
  Epistemological awareness in research design
\item
  Support for interdisciplinary research
\end{itemize}

\subsection{Practical Applications}\label{practical-applications-1}

\subsubsection{Educational Tools}\label{educational-tools}

The framework enables:

\begin{itemize}
\tightlist
\item
  Recognition of different learning styles and approaches
\item
  Adaptation of teaching methods to match different ways
\item
  Curriculum design that exposes students to multiple ways
\item
  Assessment methods appropriate for different ways
\end{itemize}

\subsubsection{Research Methodology}\label{research-methodology-1}

Researchers can use the framework for:

\begin{itemize}
\tightlist
\item
  Systematic selection of appropriate research methods
\item
  Understanding how methods complement each other
\item
  Epistemological awareness in research design
\item
  Interdisciplinary bridge-building
\end{itemize}

\subsubsection{Personal Development}\label{personal-development-1}

Individuals can use the framework for:

\begin{itemize}
\tightlist
\item
  Understanding their own preferred ways of figuring things out
\item
  Learning new ways to expand capabilities
\item
  Recognizing which ways are appropriate for which situations
\item
  Developing the ability to use multiple ways as needed
\end{itemize}

\subsection{Future Directions}\label{future-directions}

\subsubsection{Framework Expansion}\label{framework-expansion-1}

Future research can: 1. Document additional ways beyond the current 284
2. Explore ways from other philosophical traditions 3. Investigate ways
in specific domains (science, art, humanities) 4. Develop ways for
emerging contexts (digital, global, interdisciplinary)

\subsubsection{Empirical Validation}\label{empirical-validation-1}

Empirical research can: 1. Test the effectiveness of different ways in
different contexts 2. Investigate individual differences in way
preferences 3. Study how ways develop and change over time 4. Examine
relationships between ways and learning outcomes

\subsubsection{Computational
Applications}\label{computational-applications-1}

Computational research can: 1. Develop AI systems that use different
ways 2. Create recommendation systems for way selection 3. Build tools
for way analysis and visualization 4. Develop educational software based
on the framework

\subsubsection{Interdisciplinary
Integration}\label{interdisciplinary-integration-1}

The framework can be integrated with: 1. Cognitive science research on
learning and knowledge 2. Educational research on teaching methods and
curriculum 3. Philosophy of science and epistemology 4. Knowledge
management and organizational learning

\subsection{Methodological
Contributions}\label{methodological-contributions}

\subsubsection{Database-Driven Analysis}\label{database-driven-analysis}

This work demonstrates:

\begin{itemize}
\tightlist
\item
  How philosophical frameworks can be systematically documented in
  databases
\item
  The value of quantitative analysis for understanding qualitative
  frameworks
\item
  How network analysis reveals structure in knowledge systems
\item
  The integration of database analysis with text analysis
\end{itemize}

\subsubsection{Visualization Approaches}\label{visualization-approaches}

The visualization work shows:

\begin{itemize}
\tightlist
\item
  How network graphs reveal structure in way relationships
\item
  How hierarchical visualizations illustrate the House of Knowledge
\item
  How statistical plots communicate distribution patterns
\item
  How multiple visualization types complement each other
\end{itemize}

\subsubsection{Integration of Quantitative and
Qualitative}\label{integration-of-quantitative-and-qualitative}

The work demonstrates:

\begin{itemize}
\tightlist
\item
  How quantitative analysis complements qualitative understanding
\item
  The value of systematic documentation for philosophical frameworks
\item
  How data-driven insights enhance philosophical interpretation
\item
  The integration of empirical analysis with philosophical analysis
\end{itemize}

\subsubsection{Implementation Modules}\label{implementation-modules-1}

The research implements a comprehensive software framework for ways
analysis:

\textbf{Database Layer}: \passthrough{\lstinline!database.py!},
\passthrough{\lstinline!sql\_queries.py!},
\passthrough{\lstinline!models.py!} - ORM models and query interfaces
\textbf{Analysis Layer}: \passthrough{\lstinline!ways\_analysis.py!},
\passthrough{\lstinline!network\_analysis.py!},
\passthrough{\lstinline!house\_of\_knowledge.py!} - Specialized analysis
modules \textbf{Statistics Layer}:
\passthrough{\lstinline!statistics.py!},
\passthrough{\lstinline!metrics.py!} - Quantitative analysis functions
\textbf{Supporting Modules}: Data processing, visualization, and
reporting utilities

All modules follow the thin orchestrator pattern with business logic in
\passthrough{\lstinline!src/!} and orchestration in
\passthrough{\lstinline!scripts/!}.

\subsection{Final Remarks}\label{final-remarks}

This research provides both a philosophical framework and a practical
system for understanding and applying diverse ways of figuring things
out. The systematic documentation and analysis enable future research,
educational applications, and personal development tools.

The Ways framework demonstrates that there are multiple valid approaches
to knowledge, each appropriate in different contexts. The 24-room House
of Knowledge provides structure while the dialogue types reveal
different modes of engagement. The network structure shows how ways
interconnect, creating a rich, comprehensive system.

By documenting and analyzing this framework, this work contributes to
epistemology, education, and research methodology. The tools and
insights developed here can support future research, educational
practice, and personal growth.

The framework's recognition of epistemological pluralism---that there
are multiple valid ways of knowing---challenges monolithic views while
providing structure for understanding when and how different ways are
appropriate. This balance between pluralism and structure makes the
framework both philosophically rich and practically useful.

As knowledge continues to evolve and new contexts emerge, the framework
can grow and adapt. Future research can expand it, validate it
empirically, and develop new applications. This work provides the
foundation for that future development.

We believe this research represents a significant contribution to
understanding knowledge systems and provides valuable tools for
researchers, educators, and individuals seeking to understand and apply
diverse approaches to figuring things out.

\newpage

\section{Acknowledgments}\label{sec:acknowledgments}

We gratefully acknowledge the contributions that made this research
possible.

\subsection{Primary Source}\label{primary-source}

This research is based entirely on the philosophical work of
\textbf{Andrius Kulikauskas}, who developed the ``Ways of Figuring
Things Out'' framework and documented 284 ways of knowledge acquisition.
The framework, database, and documentation are the result of his
extensive philosophical work conducted in 2010-2011.

\subsection{Data Availability}\label{data-availability}

All data used in this research is in the \textbf{Public Domain} as
stated in the source documentation. The MySQL database dump and text
documentation (\passthrough{\lstinline!ways.md!}) are publicly available
and were used with appropriate attribution.

\subsection{Framework Development}\label{framework-development}

The House of Knowledge framework, the 24-room structure, and the
dialogue type classifications are all part of Andrius Kulikauskas's
original philosophical work. This research provides systematic
documentation and analysis but does not claim to have developed the
underlying framework.

\subsection{Technical Infrastructure}\label{technical-infrastructure}

This research builds upon:

\begin{itemize}
\tightlist
\item
  \textbf{Python scientific computing stack} (NumPy, SciPy, Pandas,
  NetworkX, Matplotlib)
\item
  \textbf{SQLite} database system for data storage and querying
\item
  \textbf{LaTeX and Pandoc} for document preparation
\item
  \textbf{Open-source tools} for data analysis and visualization
\end{itemize}

\subsection{Research Context}\label{research-context}

This work contributes to the systematic documentation and analysis of
philosophical frameworks, demonstrating how quantitative methods can
complement qualitative understanding. The integration of database
analysis, network analysis, and statistical methods with philosophical
interpretation represents a methodological contribution to the study of
knowledge systems.

\subsection{Future Contributions}\label{future-contributions}

Future researchers building on this work should acknowledge: - Andrius
Kulikauskas as the originator of the Ways framework - The public domain
status of the source data - The systematic analysis and documentation
provided by this research

\begin{center}\rule{0.5\linewidth}{0.5pt}\end{center}

\emph{All errors and omissions in the analysis and interpretation remain
the sole responsibility of the authors. The underlying philosophical
framework and data are the work of Andrius Kulikauskas.}

\newpage

\section{Appendix}\label{sec:appendix}

This appendix provides additional technical details supporting the main
results.

\subsection{A. Database Schema
Details}\label{a.-database-schema-details}

\subsubsection{A.1 Complete Table
Schemas}\label{a.1-complete-table-schemas}

\paragraph{Ways Table Schema}\label{ways-table-schema}

\begin{lstlisting}[language=SQL]
CREATE TABLE ways (
    way TEXT NOT NULL,
    dialoguewith TEXT NOT NULL,
    dialoguetype TEXT NOT NULL,
    dialoguetypetype TEXT NOT NULL,
    ID INTEGER PRIMARY KEY AUTOINCREMENT,
    wayurl TEXT NOT NULL,
    examples TEXT NOT NULL,
    dialoguetypetypetype TEXT NOT NULL,
    mene TEXT NOT NULL,
    Dievas TEXT NOT NULL,
    comments TEXT NOT NULL,
    laikinas TEXT NOT NULL
);
\end{lstlisting}

\paragraph{Rooms Table Schema}\label{rooms-table-schema}

\begin{lstlisting}[language=SQL]
CREATE TABLE rooms (
    santrumpa TEXT NOT NULL PRIMARY KEY,
    savoka TEXT NOT NULL,
    issiaiskinimas TEXT NOT NULL,
    -- Additional fields for ordering and relationships
);
\end{lstlisting}

\subsubsection{A.2 Index Definitions}\label{a.2-index-definitions}

Key indexes for performance:

\begin{itemize}
\tightlist
\item
  Index on \passthrough{\lstinline!way!} for way lookups
\item
  Index on \passthrough{\lstinline!mene!} for room-based queries
\item
  Index on \passthrough{\lstinline!dialoguetype!} for type filtering
\item
  Index on \passthrough{\lstinline!dialoguewith!} for partner analysis
\end{itemize}

\subsection{B. Network Analysis
Algorithms}\label{b.-network-analysis-algorithms}

\subsubsection{B.1 Actual Network
Metrics}\label{b.1-actual-network-metrics}

The network analysis was performed using NetworkX on the complete ways
database:

\textbf{Network Structure:} - \textbf{Nodes}: 210 ways - \textbf{Edges}:
1,290 connections - \textbf{Average degree}: 12.29 connections per way -
\textbf{Network density}: 0.058 (5.8\% of possible edges present) -
\textbf{Clustering coefficient}: 0.886 (high local clustering) -
\textbf{Connected components}: Multiple components detected -
\textbf{Largest component}: Contains majority of ways

\textbf{Centrality Metrics:} - \textbf{Degree centrality}: Range from
0.0 to 0.162 (way 84, 156, 211 with highest degree: 34) -
\textbf{Betweenness centrality}: Identifies bridge ways connecting
different communities - \textbf{Closeness centrality}: Measures average
distance to all other ways - \textbf{Eigenvector centrality}: Identifies
ways connected to highly central ways

\textbf{Community Detection:} - Modularity-based community detection
reveals natural clusters - Communities correspond to room assignments
and dialogue types - Largest communities align with most populated rooms
(B2, C4, R)

\subsubsection{B.2 Graph Construction
Implementation}\label{b.2-graph-construction-implementation}

The network is constructed using
\passthrough{\lstinline!WaysNetworkAnalyzer!} with three edge types:

\begin{lstlisting}[language=Python]
from collections import defaultdict
import networkx as nx
from src.models import Way

def _build_ways_network(ways: List[Way]) -> nx.Graph:
    """Build network graph from ways data.
    
    Edges are created based on:
    1. Same room (weight=1.0, edge_type='same_room')
    2. Same dialogue partner (weight=0.8, edge_type='same_partner')
    3. Same dialogue type (weight=0.6, edge_type='same_type')
    """
    G = nx.Graph()
    
    # Add nodes with attributes
    for way in ways:
        G.add_node(way.id, 
                   way_text=way.way,
                   room=way.mene,
                   dialogue_type=way.dialoguetype,
                   dialogue_partner=way.dialoguewith)
    
    # Group ways by room
    room_ways = defaultdict(list)
    for way in ways:
        room_ways[way.mene].append(way.id)
    
    # Add room edges (highest weight)
    for room, way_ids in room_ways.items():
        if len(way_ids) > 1:
            for i, way1_id in enumerate(way_ids):
                for way2_id in way_ids[i+1:]:
                    G.add_edge(way1_id, way2_id, 
                              edge_type='same_room',
                              room=room,
                              weight=1.0)
    
    # Add partner edges (medium weight)
    partner_ways = defaultdict(list)
    for way in ways:
        partner_ways[way.dialoguewith].append(way.id)
    
    for partner, way_ids in partner_ways.items():
        if len(way_ids) > 1:
            for i, way1_id in enumerate(way_ids):
                for way2_id in way_ids[i+1:]:
                    if not G.has_edge(way1_id, way2_id):
                        G.add_edge(way1_id, way2_id,
                                  edge_type='same_partner',
                                  partner=partner,
                                  weight=0.8)
    
    # Add type edges (lowest weight)
    type_ways = defaultdict(list)
    for way in ways:
        type_ways[way.dialoguetype].append(way.id)
    
    for dtype, way_ids in type_ways.items():
        if len(way_ids) > 1:
            for i, way1_id in enumerate(way_ids):
                for way2_id in way_ids[i+1:]:
                    if not G.has_edge(way1_id, way2_id):
                        G.add_edge(way1_id, way2_id,
                                  edge_type='same_type',
                                  dialogue_type=dtype,
                                  weight=0.6)
    
    return G
\end{lstlisting}

\subsubsection{B.3 Centrality
Computation}\label{b.3-centrality-computation}

Centrality metrics are computed using NetworkX functions within
\passthrough{\lstinline!WaysNetworkAnalyzer.compute\_centrality\_metrics()!}:

\begin{itemize}
\tightlist
\item
  \passthrough{\lstinline!nx.degree\_centrality(G)!}: Normalized degree
  (0-1 range)
\item
  \passthrough{\lstinline!nx.betweenness\_centrality(G)!}: Bridge
  identification
\item
  \passthrough{\lstinline!nx.closeness\_centrality(G)!}: Average path
  length
\item
  \passthrough{\lstinline!nx.eigenvector\_centrality(G, max\_iter=1000)!}:
  Influence propagation
\item
  \passthrough{\lstinline!nx.average\_clustering(G)!}: Local clustering
  coefficient
\end{itemize}

The implementation handles edge cases (disconnected graphs, single
nodes) and returns a \passthrough{\lstinline!NetworkMetrics!} dataclass
with all computed values.

\subsection{C. Statistical Analysis
Formulas}\label{c.-statistical-analysis-formulas}

\subsubsection{C.1 Distribution Metrics}\label{c.1-distribution-metrics}

For a categorical variable with \(k\) categories:

\begin{equation}\label{eq:entropy_appendix}
H(X) = -\sum_{i=1}^{k} p_i \log_2(p_i)
\end{equation}

where \(p_i\) is the proportion in category \(i\).

\subsubsection{C.2 Association Measures}\label{c.2-association-measures}

For cross-tabulation analysis:

\begin{equation}\label{eq:cramers_v}
V = \sqrt{\frac{\chi^2}{n \min(r-1, c-1)}}
\end{equation}

where \(\chi^2\) is the chi-square statistic, \(n\) is sample size, and
\(r\), \(c\) are row and column counts.

\subsection{D. Visualization
Specifications}\label{d.-visualization-specifications}

\subsubsection{D.1 Network Visualization
Parameters}\label{d.1-network-visualization-parameters}

\begin{itemize}
\tightlist
\item
  \textbf{Layout Algorithm}: Force-directed (Fruchterman-Reingold)
\item
  \textbf{Node Size}: Proportional to centrality score
\item
  \textbf{Node Color}: By dialogue type
\item
  \textbf{Edge Width}: By relationship strength
\item
  \textbf{Edge Color}: By relationship type
\end{itemize}

\subsubsection{D.2 Color Schemes}\label{d.2-color-schemes}

\begin{itemize}
\tightlist
\item
  \textbf{Dialogue Types}:

  \begin{itemize}
  \tightlist
  \item
    Absolute: Blue shades
  \item
    Relative: Green shades
  \item
    Embrace God: Purple shades
  \end{itemize}
\item
  \textbf{Rooms}: Sequential color scheme for 24 rooms
\end{itemize}

\subsection{E. Data Processing
Pipeline}\label{e.-data-processing-pipeline}

\subsubsection{E.1 Database
Initialization}\label{e.1-database-initialization}

\begin{lstlisting}[language=Python]
from src.database import WaysDatabase, initialize_database

def setup_ways_database(mysql_dump_path: str = None, 
                        sqlite_path: str = "project/db/ways.db") -> WaysDatabase:
    """Initialize SQLite database from MySQL dump or existing database.
    
    Args:
        mysql_dump_path: Path to MySQL dump file (optional)
        sqlite_path: Path to SQLite database file
        
    Returns:
        Initialized WaysDatabase instance
    """
    if mysql_dump_path:
        # Convert MySQL dump to SQLite
        initialize_database(mysql_dump_path, sqlite_path)
    
    # Return database connection
    db = WaysDatabase(sqlite_path)
    
    # Validate database integrity
    stats = db.get_way_statistics()
    assert stats['total_ways'] > 0, "Database must contain ways"
    
    return db
\end{lstlisting}

\subsubsection{E.2 Data Access and
Querying}\label{e.2-data-access-and-querying}

\begin{lstlisting}[language=Python]
from src.database import WaysDatabase
from src.sql_queries import WaysSQLQueries
from src.models import Way

def query_ways_data(db_path: str = "project/db/ways.db") -> Dict[str, Any]:
    """Query ways data using SQL queries module.
    
    Returns:
        Dictionary with ways, rooms, and statistics
    """
    db = WaysDatabase(db_path)
    queries = WaysSQLQueries(db_path)
    
    # Get all ways
    _, ways_data = queries.get_all_ways_sql()
    ways = [Way.from_sqlalchemy(row) for row in ways_data]
    
    # Get room distribution
    _, room_counts = queries.count_ways_by_room_sql()
    room_dist = {room: count for room, count in room_counts}
    
    # Get type distribution
    _, type_counts = queries.count_ways_by_type_sql()
    type_dist = {dtype: count for dtype, count in type_counts}
    
    # Get cross-tabulation
    _, crosstab = queries.cross_tabulate_type_room_sql()
    
    return {
        'ways': ways,
        'room_distribution': room_dist,
        'type_distribution': type_dist,
        'crosstab': crosstab,
        'total_ways': len(ways)
    }
\end{lstlisting}

\subsubsection{E.3 Analysis Script}\label{e.3-analysis-script}

The comprehensive analysis script integrates multiple analysis modules:

\begin{lstlisting}[language=Python]
from src.ways_analysis import WaysAnalyzer
from src.network_analysis import WaysNetworkAnalyzer
from src.database import WaysDatabase
from src.sql_queries import WaysSQLQueries

def analyze_ways_comprehensive(db_path: str = None) -> Dict[str, Any]:
    """Comprehensive analysis of ways database.
    
    Args:
        db_path: Optional path to SQLite database
        
    Returns:
        Dictionary containing all analysis results
    """
    # Initialize analyzers
    analyzer = WaysAnalyzer(db_path)
    network_analyzer = WaysNetworkAnalyzer(db_path)
    db = WaysDatabase(db_path)
    queries = WaysSQLQueries(db_path)
    
    # Distribution analysis
    characterization = analyzer.characterize_ways()
    type_dist = characterization.dialogue_types
    room_dist = characterization.room_distribution
    
    # Network analysis
    network = network_analyzer.build_ways_network()
    metrics = network_analyzer.compute_centrality_metrics()
    central_ways = network_analyzer.find_central_ways()
    
    # Statistical analysis
    _, crosstab_results = queries.cross_tabulate_type_room_sql()
    crosstab = {}
    for dtype, room, count in crosstab_results:
        if dtype not in crosstab:
            crosstab[dtype] = {}
        crosstab[dtype][room] = count
    
    return {
        'characterization': {
            'total_ways': characterization.total_ways,
            'room_diversity': characterization.room_diversity,
            'type_diversity': characterization.type_diversity,
            'most_common_room': characterization.most_common_room,
            'most_common_type': characterization.most_common_type
        },
        'network_metrics': {
            'node_count': metrics.node_count,
            'edge_count': metrics.edge_count,
            'density': metrics.density,
            'average_degree': metrics.average_degree,
            'clustering_coefficient': metrics.clustering_coefficient
        },
        'central_ways': {
            'by_degree': central_ways.by_degree[:10],
            'by_betweenness': central_ways.by_betweenness[:10]
        },
        'crosstab': crosstab
    }
\end{lstlisting}

\subsection{F. Validation Procedures}\label{f.-validation-procedures}

\subsubsection{F.1 Data Quality Checks}\label{f.1-data-quality-checks}

\begin{enumerate}
\def\labelenumi{\arabic{enumi}.}
\tightlist
\item
  \textbf{Completeness Check}: Verify all required fields present
\item
  \textbf{Consistency Check}: Check for conflicting assignments
\item
  \textbf{Referential Integrity}: Validate foreign key relationships
\item
  \textbf{Encoding Check}: Verify UTF-8 encoding
\end{enumerate}

\subsubsection{F.2 Analysis Validation}\label{f.2-analysis-validation}

\begin{enumerate}
\def\labelenumi{\arabic{enumi}.}
\tightlist
\item
  \textbf{Reproducibility}: Fixed random seeds, deterministic algorithms
\item
  \textbf{Sensitivity}: Test with missing data, parameter variations
\item
  \textbf{Robustness}: Verify results stable under different conditions
\item
  \textbf{Cross-Validation}: Validate findings across data subsets
\end{enumerate}

\subsection{G. Computational
Environment}\label{g.-computational-environment}

\subsubsection{G.1 Software Versions}\label{g.1-software-versions}

\begin{itemize}
\tightlist
\item
  Python 3.10+
\item
  SQLite 3.x
\item
  NetworkX 2.x+
\item
  Pandas 1.x+
\item
  Matplotlib 3.x+
\item
  NumPy 1.x+
\end{itemize}

\subsubsection{G.2 Hardware
Requirements}\label{g.2-hardware-requirements}

\begin{itemize}
\tightlist
\item
  Minimum: 4GB RAM, single core
\item
  Recommended: 8GB+ RAM, multi-core
\item
  Storage: \textasciitilde100MB for database and outputs
\end{itemize}

\subsection{H. Additional Tables and
Figures}\label{h.-additional-tables-and-figures}

\subsubsection{H.1 Extended Distribution
Tables}\label{h.1-extended-distribution-tables}

\paragraph{Complete Room Distribution}\label{complete-room-distribution}

The complete distribution of all 24 rooms in the House of Knowledge:

\begin{table}[h]
\centering
\small
\begin{tabular}{|l|c|c||l|c|c|}
\hline
\textbf{Room} & \textbf{Count} & \textbf{\%} & \textbf{Room} & \textbf{Count} & \textbf{\%} \\
\hline
B2 & 23 & 11.0\% & C2 & 7 & 3.3\% \\
C4 & 17 & 8.1\% & B4 & 7 & 3.3\% \\
R & 16 & 7.6\% & 1 & 7 & 3.3\% \\
32 & 13 & 6.2\% & F & 6 & 2.9\% \\
C3 & 13 & 6.2\% & 20 & 5 & 2.4\% \\
BB & 12 & 5.7\% & A & 4 & 1.9\% \\
CB & 10 & 4.8\% & 30 & 4 & 1.9\% \\
21 & 9 & 4.3\% & BC & 3 & 1.4\% \\
B3 & 9 & 4.3\% & B & 1 & 0.5\% \\
CC & 9 & 4.3\% & C & 1 & 0.5\% \\
O & 9 & 4.3\% & & & \\
T & 9 & 4.3\% & & & \\
10 & 8 & 3.8\% & & & \\
31 & 8 & 3.8\% & & & \\
\hline
\multicolumn{3}{|c||}{\textbf{Total}} & \multicolumn{3}{c|}{\textbf{210 (100\%)}} \\
\hline
\end{tabular}
\caption{Complete distribution of ways across all 24 rooms}
\label{tab:complete_room_distribution}
\end{table}

\paragraph{Complete Dialogue Type
Distribution}\label{complete-dialogue-type-distribution}

The complete distribution of all 38 dialogue types (presented in two
parts):

\begin{table}[h]
\centering
\small
\begin{tabular}{|l|c|c||l|c|c|}
\hline
\textbf{Type} & \textbf{Count} & \textbf{\%} & \textbf{Type} & \textbf{Count} & \textbf{\%} \\
\hline
goodness & 15 & 7.1\% & my mind & 7 & 3.3\% \\
other & 15 & 7.1\% & opposing view & 7 & 3.3\% \\
regularity & 11 & 5.2\% & unknown & 7 & 3.3\% \\
I & 9 & 4.3\% & conviction & 5 & 2.4\% \\
answer & 9 & 4.3\% & interlocutor & 5 & 2.4\% \\
knowledge & 8 & 3.8\% & my fate & 5 & 2.4\% \\
life & 8 & 3.8\% & my knowledge & 5 & 2.4\% \\
mind & 8 & 3.8\% & my purpose & 5 & 2.4\% \\
God & 6 & 2.9\% & wholeness & 5 & 2.4\% \\
divineness & 6 & 2.9\% & behavior & 4 & 1.9\% \\
purpose & 6 & 2.9\% & capability & 4 & 1.9\% \\
solution & 6 & 2.9\% & God's perspective & 4 & 1.9\% \\
God's perspective & 4 & 1.9\% & inspiration & 4 & 1.9\% \\
possibilities & 4 & 1.9\% & possibility & 4 & 1.9\% \\
self-check & 4 & 1.9\% & structure & 4 & 1.9\% \\
\hline
\end{tabular}
\caption{Dialogue type distribution (Part 1: Top 19 types)}
\label{tab:type_distribution_part1}
\end{table}

\begin{table}[h]
\centering
\small
\begin{tabular}{|l|c|c||l|c|c|}
\hline
\textbf{Type} & \textbf{Count} & \textbf{\%} & \textbf{Type} & \textbf{Count} & \textbf{\%} \\
\hline
conditionality & 3 & 1.4\% & invalidity & 2 & 1.0\% \\
example & 3 & 1.4\% & misfortune & 2 & 1.0\% \\
given & 3 & 1.4\% & phenomenon & 2 & 1.0\% \\
impossibility & 3 & 1.4\% & depths & 1 & 0.5\% \\
& & & infinity & 1 & 0.5\% \\
\hline
\multicolumn{3}{|c||}{\textbf{Total}} & \multicolumn{3}{c|}{\textbf{210 (100\%)}} \\
\hline
\end{tabular}
\caption{Dialogue type distribution (Part 2: Remaining 19 types)}
\label{tab:type_distribution_part2}
\end{table}

\paragraph{Top Cross-Tabulation
Combinations}\label{top-cross-tabulation-combinations}

The most frequent dialogue type × room combinations:

\begin{table}[h]
\centering
\small
\begin{tabular}{|l|l|c|}
\hline
\textbf{Dialogue Type} & \textbf{Room} & \textbf{Count} \\
\hline
goodness & B2 & 3 \\
goodness & R & 3 \\
goodness & T & 2 \\
I & O & 9 \\
answer & 1 & 4 \\
answer & 32 & 2 \\
knowledge & CC & 8 \\
divineness & C4 & 6 \\
God & B2 & 5 \\
God's perspective & R & 4 \\
capability & B3 & 4 \\
inspiration & B4 & 4 \\
conviction & B3 & 5 \\
\hline
\end{tabular}
\caption{Top dialogue type × room combinations (count ≥ 2)}
\label{tab:top_crosstab}
\end{table}

\subsubsection{H.2 Extended Network
Visualizations}\label{h.2-extended-network-visualizations}

\paragraph{Ways Network Visualization}\label{ways-network-visualization}

Figure \ref{fig:ways_network} shows the complete network graph of 210
ways with 1,290 edges. The visualization uses a force-directed layout
(Fruchterman-Reingold algorithm) with: - \textbf{Node colors}: Coded by
dialogue type (38 distinct types) - \textbf{Node sizes}: Proportional to
degree centrality - \textbf{Edge types}: Three relationship types (same
room: weight 1.0, same partner: weight 0.8, same type: weight 0.6) -
\textbf{Layout}: Optimized for visual clarity with community clustering
visible

The network exhibits high clustering (coefficient: 0.886) indicating
strong room-based communities. The largest connected component contains
the majority of ways, with smaller isolated components representing
specialized dialogue patterns.

\paragraph{Room Hierarchy
Visualization}\label{room-hierarchy-visualization}

Figure \ref{fig:room_hierarchy} presents a hierarchical bar chart
showing the distribution of ways across all 24 rooms. The visualization
organizes rooms by their position in the House of Knowledge framework,
revealing: - \textbf{Most populated rooms}: B2 (23 ways), C4 (17 ways),
R (16 ways) - \textbf{Framework structure}: Clear patterns in believing
(B-series) and caring (C-series) hierarchies - \textbf{Relative learning
rooms}: R (Reflecting), O (Obeying), T (Taking a Stand) show balanced
distributions

\paragraph{Framework Treemap}\label{framework-treemap}

Figure \ref{fig:framework_treemap} provides a treemap visualization of
the framework structure, where: - \textbf{Area}: Proportional to number
of ways in each room - \textbf{Color}: Indicates framework category
(believing, caring, relative learning) - \textbf{Hierarchy}: Shows
nested relationships within the House of Knowledge

This visualization highlights the structural organization of the
framework and the relative emphasis on different aspects of knowledge
acquisition.

\subsubsection{H.3 Extended Statistical
Plots}\label{h.3-extended-statistical-plots}

\paragraph{Dialogue Type
Distribution}\label{dialogue-type-distribution-1}

Figure \ref{fig:type_distribution} displays a bar chart of all 38
dialogue types ranked by frequency. The visualization shows: -
\textbf{Top types}: ``goodness'' and ``other'' (15 each, 7.1\%),
``regularity'' (11, 5.2\%) - \textbf{Distribution pattern}: Long tail
with many types having 1-4 occurrences - \textbf{Balance}: Relatively
even distribution across types, indicating diverse epistemological
approaches

\paragraph{Type × Room Heatmap}\label{type-room-heatmap}

Figure \ref{fig:type_room_heatmap} presents a heatmap of the
cross-tabulation between dialogue types (rows) and rooms (columns). The
visualization reveals: - \textbf{Hotspots}: Strong associations between
specific types and rooms (e.g., ``I'' × ``O'', ``knowledge'' × ``CC'') -
\textbf{Sparse regions}: Many type-room combinations have zero or low
counts - \textbf{Patterns}: Clustering of similar dialogue types in
related rooms

This heatmap provides insight into how dialogue types are distributed
across the House of Knowledge structure.

\paragraph{Dialogue Partner Word
Cloud}\label{dialogue-partner-word-cloud}

Figure \ref{fig:partner_wordcloud} shows a word cloud visualization of
dialogue partners, where: - \textbf{Font size}: Proportional to
frequency of partnership - \textbf{196 unique partners}: Most partners
appear only once or twice - \textbf{Top partners}: ``God's will'',
``God's wishes'', ``answer'', ``circumstances'' (2 occurrences each)

The word cloud highlights the diversity of dialogue partners and the
personalized nature of many ways.

\paragraph{Example Length
Distribution}\label{example-length-distribution}

Figure \ref{fig:example_length_violin} displays a violin plot showing
the distribution of example text lengths by dialogue type. The
visualization shows: - \textbf{Distribution shape}: Varies by dialogue
type, with some types having longer examples - \textbf{Average length}:
80.2 characters across all ways - \textbf{Coverage}: All 210 ways have
examples (100\% coverage)

This plot reveals patterns in how different dialogue types are
exemplified and documented.

\subsection{I. Code Availability}\label{i.-code-availability}

All code for this research is available in the project repository:

\begin{itemize}
\tightlist
\item
  \textbf{Database Module}:
  \passthrough{\lstinline!project/src/database.py!}
\item
  \textbf{Models}: \passthrough{\lstinline!project/src/models.py!}
\item
  \textbf{Analysis Scripts}: \passthrough{\lstinline!project/scripts/!}
\item
  \textbf{Tests}: \passthrough{\lstinline!project/tests/!}
\end{itemize}

The code follows the thin orchestrator pattern with business logic in
\passthrough{\lstinline!src/!} modules and orchestration in
\passthrough{\lstinline!scripts/!}.

\subsection{J. Data Availability}\label{j.-data-availability}

The source data (MySQL dump and \passthrough{\lstinline!ways.md!}) are
in the public domain as stated in the original documentation. The
converted SQLite database and analysis results are available upon
request or through the project repository.

\subsection{K. Reproducibility}\label{k.-reproducibility}

To reproduce the analyses:

\begin{enumerate}
\def\labelenumi{\arabic{enumi}.}
\tightlist
\item
  Initialize database:
  \passthrough{\lstinline!python scripts/db\_setup.py!}
\item
  Run analysis:
  \passthrough{\lstinline!python scripts/analysis\_pipeline.py!}
\item
  Generate visualizations:
  \passthrough{\lstinline!python scripts/generate\_figures.py!}
\item
  Build manuscript:
  \passthrough{\lstinline!python scripts/03\_render\_pdf.py!}
\end{enumerate}

All random operations use fixed seeds for reproducibility.

\newpage

\section{Supplemental Methods}\label{sec:supplemental_methods}

This section provides detailed methodological information that
supplements Section \ref{sec:methodology}.

\subsection{S1.1 Database Schema
Details}\label{s1.1-database-schema-details}

\subsubsection{Primary Tables}\label{primary-tables}

\paragraph{\texorpdfstring{Ways Table
(\texttt{ways})}{Ways Table (ways)}}\label{ways-table-ways}

The primary table contains 212 documented ways with the following
schema:

\begin{itemize}
\tightlist
\item
  \passthrough{\lstinline!way!} (TEXT): Name/identifier of the way
\item
  \passthrough{\lstinline!dialoguewith!} (TEXT): Dialogue partner or
  conversant
\item
  \passthrough{\lstinline!dialoguetype!} (TEXT): Primary dialogue type
  (Absolute, Relative, Embrace God)
\item
  \passthrough{\lstinline!dialoguetypetype!} (TEXT): Sub-type
  classification
\item
  \passthrough{\lstinline!ID!} (INTEGER): Primary key, auto-incrementing
\item
  \passthrough{\lstinline!wayurl!} (TEXT): URL or reference for the way
\item
  \passthrough{\lstinline!examples!} (TEXT): Examples and descriptions
  (up to 1000 characters)
\item
  \passthrough{\lstinline!dialoguetypetypetype!} (TEXT): Further
  sub-classification
\item
  \passthrough{\lstinline!mene!} (TEXT): Room assignment in House of
  Knowledge (10 characters)
\item
  \passthrough{\lstinline!Dievas!} (TEXT): Relationship to God/the
  divine
\item
  \passthrough{\lstinline!comments!} (TEXT): Additional comments and
  notes
\item
  \passthrough{\lstinline!laikinas!} (TEXT): Temporary or provisional
  classification
\end{itemize}

\paragraph{\texorpdfstring{Rooms Table
(\texttt{rooms})}{Rooms Table (rooms)}}\label{rooms-table-rooms}

The rooms table defines the 24 rooms of the House of Knowledge:

\begin{itemize}
\tightlist
\item
  \passthrough{\lstinline!santrumpa!} (TEXT): Short name/abbreviation
  for the room
\item
  \passthrough{\lstinline!savoka!} (TEXT): Concept or term for the room
\item
  \passthrough{\lstinline!issiaiskinimas!} (TEXT): Explanation or
  clarification
\item
  Additional fields for room ordering and relationships
\end{itemize}

\paragraph{\texorpdfstring{Examples Table
(\texttt{examples})}{Examples Table (examples)}}\label{examples-table-examples}

Contains examples for ways:

\begin{itemize}
\tightlist
\item
  \passthrough{\lstinline!way!} (TEXT): Way identifier
\item
  \passthrough{\lstinline!rusis!} (TEXT): Type or category of example
\item
  \passthrough{\lstinline!pavyzdziai!} (TEXT): The example text
\end{itemize}

\paragraph{\texorpdfstring{Questions Table
(\texttt{klausimai})}{Questions Table (klausimai)}}\label{questions-table-klausimai}

Contains questions related to ways:

\begin{itemize}
\tightlist
\item
  \passthrough{\lstinline!klausimonr!} (INTEGER): Question number
  (primary key)
\item
  \passthrough{\lstinline!klausimas!} (TEXT): The question text
\item
  \passthrough{\lstinline!mastytojas!} (TEXT): Thinker or source of the
  question
\end{itemize}

\paragraph{\texorpdfstring{Question-Way Relationships
(\texttt{klausimobudai})}{Question-Way Relationships (klausimobudai)}}\label{question-way-relationships-klausimobudai}

Links questions to ways:

\begin{itemize}
\tightlist
\item
  \passthrough{\lstinline!klausimobudonr!} (INTEGER): Relationship ID
  (primary key)
\item
  \passthrough{\lstinline!klausimonr!} (INTEGER): Foreign key to
  questions table
\item
  \passthrough{\lstinline!budonr!} (INTEGER): Foreign key to ways table
  (via ID)
\end{itemize}

\subsubsection{Data Types and
Constraints}\label{data-types-and-constraints}

The SQLite conversion preserves data integrity while adapting
MySQL-specific features:

\begin{itemize}
\tightlist
\item
  \textbf{AUTO\_INCREMENT}: Converted to INTEGER PRIMARY KEY with
  auto-increment
\item
  \textbf{VARCHAR}: Converted to TEXT (SQLite's flexible text type)
\item
  \textbf{COLLATE}: Removed (SQLite handles Unicode natively)
\item
  \textbf{ENGINE}: Removed (not applicable to SQLite)
\item
  \textbf{CHARACTER SET}: Removed (SQLite uses UTF-8)
\end{itemize}

\subsection{S1.2 SQLite Conversion
Process}\label{s1.2-sqlite-conversion-process}

\subsubsection{Conversion Steps}\label{conversion-steps}

\begin{enumerate}
\def\labelenumi{\arabic{enumi}.}
\tightlist
\item
  \textbf{Parse MySQL Dump}: Read and parse the MySQL dump file
\item
  \textbf{Extract Statements}: Identify CREATE TABLE, INSERT, and other
  statements
\item
  \textbf{Convert Syntax}: Transform MySQL-specific syntax to SQLite
\item
  \textbf{Handle Indexes}: Convert KEY definitions to CREATE INDEX
  statements
\item
  \textbf{Rename Tables}: Apply table renames for clarity
\item
  \textbf{Fix Conflicts}: Resolve index name conflicts with table names
\item
  \textbf{Execute}: Create SQLite database with converted schema and
  data
\end{enumerate}

\subsubsection{Syntax Conversions}\label{syntax-conversions}

\paragraph{Data Type Conversions}\label{data-type-conversions}

\begin{table}[h]
\centering
\begin{tabular}{|l|l|}
\hline
\textbf{MySQL} & \textbf{SQLite} \\
\hline
int(11) & INTEGER \\
varchar(n) & TEXT \\
AUTO_INCREMENT & INTEGER PRIMARY KEY AUTOINCREMENT \\
\hline
\end{tabular}
\caption{Data type conversions}
\label{tab:type_conversions}
\end{table}

\paragraph{Index Handling}\label{index-handling}

MySQL KEY definitions within CREATE TABLE statements are extracted and
converted to separate CREATE INDEX statements. Index names that conflict
with table names are renamed (e.g., \passthrough{\lstinline!ways!} index
→ \passthrough{\lstinline!ways\_idx!}).

\paragraph{Function Call Handling}\label{function-call-handling}

MySQL supports function calls in index definitions (e.g.,
\passthrough{\lstinline!examples(333)!}). SQLite does not, so these are
simplified to column names only.

\subsection{S1.3 Network Analysis
Methods}\label{s1.3-network-analysis-methods}

\subsubsection{Graph Construction}\label{graph-construction-1}

The network graph \(G = (V, E)\) is constructed as follows:

\textbf{Vertices \(V\)}: Each way \(w_i\) becomes a node \(v_i\).

\textbf{Edges \(E\)}: Edges are created based on:

\begin{enumerate}
\def\labelenumi{\arabic{enumi}.}
\item
  \textbf{Shared Dialogue Partners}: If ways \(w_i\) and \(w_j\) share
  the same \passthrough{\lstinline!dialoguewith!} value:
  \begin{equation}\label{eq:edge_dialogue}
  e_{ij} \in E \text{ if } \text{dialoguewith}(w_i) = \text{dialoguewith}(w_j)
  \end{equation}
\item
  \textbf{Shared Room Assignment}: If ways \(w_i\) and \(w_j\) share the
  same \passthrough{\lstinline!mene!} value:
  \begin{equation}\label{eq:edge_room}
  e_{ij} \in E \text{ if } \text{mene}(w_i) = \text{mene}(w_j)
  \end{equation}
\item
  \textbf{Question Relationships}: If ways \(w_i\) and \(w_j\) are
  linked through the \passthrough{\lstinline!klausimobudai!} table:
  \begin{equation}\label{eq:edge_question}
  e_{ij} \in E \text{ if } \exists q: (w_i, q) \in \text{klausimobudai} \land (w_j, q) \in \text{klausimobudai}
  \end{equation}
\end{enumerate}

\subsubsection{Edge Weights}\label{edge-weights}

Edges can be weighted based on: - Number of shared characteristics -
Strength of relationship (direct vs.~indirect) - Type of relationship
(dialogue partner vs.~room vs.~question)

\subsubsection{Centrality Metrics}\label{centrality-metrics-1}

\paragraph{Degree Centrality}\label{degree-centrality}

\begin{equation}\label{eq:degree_centrality_supplemental}
C_D(v) = \frac{\deg(v)}{|V| - 1}
\end{equation}

where \(\deg(v)\) is the degree (number of connections) of node \(v\).

\paragraph{Betweenness Centrality}\label{betweenness-centrality}

\begin{equation}\label{eq:betweenness_centrality_supplemental}
C_B(v) = \sum_{s \neq v \neq t} \frac{\sigma_{st}(v)}{\sigma_{st}}
\end{equation}

where \(\sigma_{st}\) is the number of shortest paths from \(s\) to
\(t\), and \(\sigma_{st}(v)\) is the number of those paths passing
through \(v\).

\paragraph{Closeness Centrality}\label{closeness-centrality}

\begin{equation}\label{eq:closeness_centrality}
C_C(v) = \frac{1}{\sum_{u \neq v} d(u, v)}
\end{equation}

where \(d(u, v)\) is the shortest path distance between \(u\) and \(v\).

\subsection{S1.4 Statistical Analysis
Methods}\label{s1.4-statistical-analysis-methods}

\subsubsection{Distribution Analysis}\label{distribution-analysis-2}

\paragraph{Dialogue Type
Distribution}\label{dialogue-type-distribution-2}

For dialogue type \(t\), the count is:
\begin{equation}\label{eq:type_count}
N_t = |\{w_i : \text{type}(w_i) = t\}|
\end{equation}

The proportion is: \begin{equation}\label{eq:type_proportion}
p_t = \frac{N_t}{N}
\end{equation}

where \(N\) is the total number of ways.

\paragraph{Room Distribution}\label{room-distribution-2}

For room \(r\), the count is: \begin{equation}\label{eq:room_count}
N_r = |\{w_i : r \in \text{rooms}(w_i)\}|
\end{equation}

Note that ways can belong to multiple rooms, so \(\sum_r N_r \geq N\).

\subsubsection{Cross-Tabulation}\label{cross-tabulation-1}

The cross-tabulation of dialogue type and room creates a contingency
table:

\begin{equation}\label{eq:crosstab}
C_{tr} = |\{w_i : \text{type}(w_i) = t \land r \in \text{rooms}(w_i)\}|
\end{equation}

This enables analysis of whether certain dialogue types are more common
in certain rooms.

\subsubsection{Chi-Square Test}\label{chi-square-test}

To test independence of dialogue type and room assignment:

\begin{equation}\label{eq:chi_square}
\chi^2 = \sum_{t,r} \frac{(O_{tr} - E_{tr})^2}{E_{tr}}
\end{equation}

where \(O_{tr}\) is the observed count and \(E_{tr}\) is the expected
count under independence.

\subsection{S1.5 Text Analysis
Methods}\label{s1.5-text-analysis-methods}

\subsubsection{Keyword Extraction}\label{keyword-extraction-1}

Text from way descriptions and examples is processed to extract
keywords:

\begin{enumerate}
\def\labelenumi{\arabic{enumi}.}
\tightlist
\item
  \textbf{Tokenization}: Split text into words
\item
  \textbf{Normalization}: Convert to lowercase, handle Unicode
\item
  \textbf{Stop Word Removal}: Remove common words
\item
  \textbf{Stemming}: Reduce words to root forms (if applicable)
\item
  \textbf{Frequency Analysis}: Count keyword frequencies
\end{enumerate}

\subsubsection{Theme Extraction}\label{theme-extraction}

Themes are identified through: - \textbf{Co-occurrence Analysis}: Words
that frequently appear together - \textbf{Clustering}: Group similar
ways based on text similarity - \textbf{Topic Modeling}: Identify latent
topics in way descriptions

\subsubsection{Example Analysis}\label{example-analysis-1}

Examples are analyzed to: - Identify common patterns or structures -
Extract key concepts or ideas - Understand how examples illustrate ways
- Map examples to room categories

\subsection{S1.6 Visualization
Methods}\label{s1.6-visualization-methods}

\subsubsection{Network Visualization}\label{network-visualization}

Network graphs are created using force-directed layout algorithms: -
\textbf{Force-Directed Layout}: Positions nodes based on attractive
(edges) and repulsive (nodes) forces - \textbf{Color Coding}: Nodes
colored by dialogue type or room - \textbf{Size Scaling}: Node size
proportional to centrality - \textbf{Edge Styling}: Edge thickness/color
based on relationship strength

\subsubsection{Hierarchical
Visualization}\label{hierarchical-visualization}

The 24-room structure is visualized as: - \textbf{Tree Structure}: Rooms
organized hierarchically - \textbf{Sunburst Chart}: Radial hierarchical
visualization - \textbf{Treemap}: Area-based hierarchical visualization

\subsubsection{Statistical Plots}\label{statistical-plots}

Standard statistical visualizations: - \textbf{Bar Charts}: Distribution
of ways by category - \textbf{Heatmaps}: Cross-tabulation matrices -
\textbf{Scatter Plots}: Relationships between variables -
\textbf{Distribution Plots}: Histograms and density plots

\subsection{S1.7 Validation Methods}\label{s1.7-validation-methods}

\subsubsection{Data Quality Checks}\label{data-quality-checks-1}

\begin{enumerate}
\def\labelenumi{\arabic{enumi}.}
\tightlist
\item
  \textbf{Completeness}: Verify required fields are present
\item
  \textbf{Consistency}: Check for conflicting assignments
\item
  \textbf{Referential Integrity}: Validate foreign key relationships
\item
  \textbf{Encoding}: Verify proper text encoding (UTF-8)
\end{enumerate}

\subsubsection{Analysis Validation}\label{analysis-validation-1}

\begin{enumerate}
\def\labelenumi{\arabic{enumi}.}
\tightlist
\item
  \textbf{Reproducibility}: All analyses use fixed random seeds
\item
  \textbf{Sensitivity Analysis}: Test sensitivity to data variations
\item
  \textbf{Robustness}: Verify results with missing data handling
\item
  \textbf{Cross-Validation}: Validate findings across different data
  subsets
\end{enumerate}

\subsection{S1.8 Implementation
Details}\label{s1.8-implementation-details}

\subsubsection{Software Stack}\label{software-stack}

\begin{itemize}
\tightlist
\item
  \textbf{Python 3.10+}: Primary programming language
\item
  \textbf{SQLite 3}: Database backend
\item
  \textbf{SQLAlchemy}: ORM for database access
\item
  \textbf{NetworkX}: Network analysis library
\item
  \textbf{Pandas}: Data manipulation and analysis
\item
  \textbf{Matplotlib/Seaborn}: Visualization libraries
\item
  \textbf{NumPy}: Numerical computations
\end{itemize}

\subsubsection{Code Organization}\label{code-organization}

Following the thin orchestrator pattern: - \textbf{Business Logic}: In
\passthrough{\lstinline!project/src/!} modules - \textbf{Orchestration}:
In \passthrough{\lstinline!project/scripts/!} scripts - \textbf{Tests}:
In \passthrough{\lstinline!project/tests/!} directory -
\textbf{Documentation}: In \passthrough{\lstinline!project/docs/!} and
\passthrough{\lstinline!project/manuscript/!}

\subsubsection{Performance
Considerations}\label{performance-considerations}

\begin{itemize}
\tightlist
\item
  \textbf{Database Indexing}: Indexes on frequently queried fields
\item
  \textbf{Caching}: Cache computed network structures and statistics
\item
  \textbf{Batch Processing}: Process large datasets in batches
\item
  \textbf{Memory Management}: Use generators for large data streams
\end{itemize}

\subsection{S1.9 Complete SQL Queries}\label{s1.9-complete-sql-queries}

This section provides the complete SQL queries used for key analyses in
the research.

\subsubsection{S1.9.1 Basic Statistics
Queries}\label{s1.9.1-basic-statistics-queries}

\textbf{Total ways count:}

\begin{lstlisting}[language=SQL]
SELECT COUNT(*) as total_ways FROM ways;
\end{lstlisting}

\textbf{Room distribution:}

\begin{lstlisting}[language=SQL]
SELECT mene, COUNT(*) as count
FROM ways
WHERE mene != ''
GROUP BY mene
ORDER BY count DESC;
\end{lstlisting}

\textbf{Dialogue type distribution:}

\begin{lstlisting}[language=SQL]
SELECT dialoguetype, COUNT(*) as count
FROM ways
GROUP BY dialoguetype
ORDER BY count DESC;
\end{lstlisting}

\textbf{Dialogue partner distribution:}

\begin{lstlisting}[language=SQL]
SELECT dialoguewith, COUNT(*) as count
FROM ways
WHERE dialoguewith != ''
GROUP BY dialoguewith
ORDER BY count DESC;
\end{lstlisting}

\subsubsection{S1.9.2 Cross-Tabulation
Queries}\label{s1.9.2-cross-tabulation-queries}

\textbf{Type × Room cross-tabulation:}

\begin{lstlisting}[language=SQL]
SELECT dialoguetype, mene, COUNT(*) as count
FROM ways
WHERE mene != '' AND dialoguetype != ''
GROUP BY dialoguetype, mene
ORDER BY count DESC;
\end{lstlisting}

\textbf{Type × Partner cross-tabulation:}

\begin{lstlisting}[language=SQL]
SELECT dialoguetype, dialoguewith, COUNT(*) as count
FROM ways
WHERE dialoguewith != '' AND dialoguetype != ''
GROUP BY dialoguetype, dialoguewith
ORDER BY count DESC;
\end{lstlisting}

\subsubsection{S1.9.3 Network Construction
Queries}\label{s1.9.3-network-construction-queries}

\textbf{Room-based edges:}

\begin{lstlisting}[language=SQL]
SELECT w1.ID as way1_id, w2.ID as way2_id, w1.mene as room
FROM ways w1
JOIN ways w2 ON w1.mene = w2.mene
WHERE w1.ID < w2.ID AND w1.mene != '';
\end{lstlisting}

\textbf{Partner-based edges:}

\begin{lstlisting}[language=SQL]
SELECT w1.ID as way1_id, w2.ID as way2_id, w1.dialoguewith as partner
FROM ways w1
JOIN ways w2 ON w1.dialoguewith = w2.dialoguewith
WHERE w1.ID < w2.ID AND w1.dialoguewith != '';
\end{lstlisting}

\textbf{Type-based edges:}

\begin{lstlisting}[language=SQL]
SELECT w1.ID as way1_id, w2.ID as way2_id, w1.dialoguetype as type
FROM ways w1
JOIN ways w2 ON w1.dialoguetype = w2.dialoguetype
WHERE w1.ID < w2.ID AND w1.dialoguetype != '';
\end{lstlisting}

\subsubsection{S1.9.4 Centrality Analysis
Queries}\label{s1.9.4-centrality-analysis-queries}

\textbf{Degree centrality calculation:}

\begin{lstlisting}[language=SQL]
SELECT way_id, COUNT(*) as degree
FROM (
    SELECT w1.ID as way_id, w2.ID as connected_way
    FROM ways w1
    JOIN ways w2 ON w1.mene = w2.mene
    WHERE w1.ID != w2.ID AND w1.mene != ''
    UNION
    SELECT w1.ID as way_id, w2.ID as connected_way
    FROM ways w1
    JOIN ways w2 ON w1.dialoguewith = w2.dialoguewith
    WHERE w1.ID != w2.ID AND w1.dialoguewith != ''
    UNION
    SELECT w1.ID as way_id, w2.ID as connected_way
    FROM ways w1
    JOIN ways w2 ON w1.dialoguetype = w2.dialoguetype
    WHERE w1.ID != w2.ID AND w1.dialoguetype != ''
)
GROUP BY way_id
ORDER BY degree DESC;
\end{lstlisting}

\subsubsection{S1.9.5 Text Analysis
Queries}\label{s1.9.5-text-analysis-queries}

\textbf{Ways with examples:}

\begin{lstlisting}[language=SQL]
SELECT ID, way, LENGTH(examples) as example_length, examples
FROM ways
WHERE examples != '' AND LENGTH(examples) > 0
ORDER BY example_length DESC;
\end{lstlisting}

\textbf{Average example length by dialogue type:}

\begin{lstlisting}[language=SQL]
SELECT dialoguetype, 
       AVG(LENGTH(examples)) as avg_length,
       COUNT(*) as count
FROM ways
WHERE examples != '' AND dialoguetype != ''
GROUP BY dialoguetype
ORDER BY avg_length DESC;
\end{lstlisting}

\subsection{S1.10 Limitations and
Assumptions}\label{s1.10-limitations-and-assumptions}

\subsubsection{Data Limitations}\label{data-limitations}

\begin{itemize}
\tightlist
\item
  Not all ways have complete metadata
\item
  Some room assignments may be missing or provisional
\item
  Dialogue partner information varies in completeness
\item
  Examples and descriptions vary in detail
\end{itemize}

\subsubsection{Analysis Assumptions}\label{analysis-assumptions}

\begin{itemize}
\tightlist
\item
  Ways are treated as discrete entities (though they may overlap)
\item
  Relationships are binary (present/absent) rather than weighted
\item
  Network structure captures all important relationships
\item
  Statistical patterns reflect meaningful structure
\end{itemize}

\subsubsection{Methodological
Limitations}\label{methodological-limitations}

\begin{itemize}
\tightlist
\item
  Quantitative analysis may miss qualitative nuances
\item
  Network analysis based on explicit database relationships
\item
  Text analysis limited by available descriptions
\item
  Visualization choices may emphasize certain aspects
\end{itemize}

These limitations are acknowledged and addressed where possible through
multiple analysis methods and careful interpretation.

\newpage

\section{Supplemental Results}\label{sec:supplemental_results}

This section provides additional experimental results that complement
Section \ref{sec:experimental_results}.

\subsection{S2.1 Detailed Room
Analysis}\label{s2.1-detailed-room-analysis}

\subsubsection{S2.1.1 Room-by-Room
Distribution}\label{s2.1.1-room-by-room-distribution}

Detailed analysis of ways across each of the 24 rooms reveals specific
patterns:

\begin{table}[h]
\centering
\begin{tabular}{|l|c|c|}
\hline
\textbf{Room} & \textbf{Way Count} & \textbf{Percentage} \\
\hline
B2 & 23 & 11.0\% \\
C4 & 17 & 8.1\% \\
R & 16 & 7.6\% \\
32 & 13 & 6.2\% \\
C3 & 13 & 6.2\% \\
BB & 12 & 5.7\% \\
CB & 10 & 4.8\% \\
21 & 9 & 4.3\% \\
B3 & 9 & 4.3\% \\
CC & 9 & 4.3\% \\
O & 9 & 4.3\% \\
T & 9 & 4.3\% \\
10 & 8 & 3.8\% \\
31 & 8 & 3.8\% \\
1 & 7 & 3.3\% \\
B4 & 7 & 3.3\% \\
C2 & 7 & 3.3\% \\
F & 6 & 2.9\% \\
20 & 5 & 2.4\% \\
30 & 4 & 1.9\% \\
B & 3 & 1.4\% \\
C & 3 & 1.4\% \\
A & 2 & 1.0\% \\
\hline
\textbf{Total} & 210 & 100\% \\
\hline
\end{tabular}
\caption{Complete room distribution with all 24 rooms}
\label{tab:room_detailed}
\end{table}

\subsubsection{S2.1.2 Room
Relationships}\label{s2.1.2-room-relationships}

Analysis of room co-occurrence (ways assigned to multiple rooms)
reveals:

\begin{itemize}
\tightlist
\item
  \textbf{Most common room pairs}: B2-C4, R-C3 (based on way
  distribution patterns)
\item
  \textbf{Room clusters}: Groups of rooms that frequently co-occur
\item
  \textbf{Room hierarchy}: Relationships between rooms in the House
  structure
\end{itemize}

\subsection{S2.2 Dialogue Partner
Analysis}\label{s2.2-dialogue-partner-analysis}

\subsubsection{S2.2.1 Partner Frequency}\label{s2.2.1-partner-frequency}

Analysis of dialogue partners (\passthrough{\lstinline!dialoguewith!})
reveals:

\begin{table}[h]
\centering
\begin{tabular}{|l|c|c|}
\hline
\textbf{Dialogue Partner} & \textbf{Frequency} & \textbf{Percentage} \\
\hline
life & 2 & 1.0\% \\
limits of my mind & 2 & 1.0\% \\
circumstances & 2 & 1.0\% \\
science & 2 & 1.0\% \\
purpose & 2 & 1.0\% \\
answer & 2 & 1.0\% \\
people's inclinations & 2 & 1.0\% \\
possibility & 2 & 1.0\% \\
goodness & 2 & 1.0\% \\
meaningfulness & 2 & 1.0\% \\
\hline
\end{tabular}
\caption{Most frequent dialogue partners (all with 2 ways each)}
\label{tab:partner_frequency}
\end{table}

\subsubsection{S2.2.2 Partner-Type
Relationships}\label{s2.2.2-partner-type-relationships}

Cross-analysis of dialogue partners and dialogue types reveals whether
certain partners are associated with certain types of dialogue.

\subsection{S2.3 Network Community
Analysis}\label{s2.3-network-community-analysis}

\subsubsection{S2.3.1 Detected
Communities}\label{s2.3.1-detected-communities}

Community detection algorithms identify 45 major communities:

\begin{itemize}
\tightlist
\item
  \textbf{Community 1}: 23 ways, primarily ``other'' dialogue type in B2
  room
\item
  \textbf{Community 2}: 17 ways, primarily ``divineness'' dialogue type
  in C4 room
\item
  \textbf{Community 3}: 16 ways, primarily ``life'' dialogue type in R
  room
\end{itemize}

\subsubsection{S2.3.2 Community
Characteristics}\label{s2.3.2-community-characteristics}

Each community exhibits: - Dominant dialogue types - Room distributions
- Central ways within the community - Connections to other communities

\subsection{S2.4 God Relationship
Analysis}\label{s2.4-god-relationship-analysis}

\subsubsection{S2.4.1 Dievas Field
Distribution}\label{s2.4.1-dievas-field-distribution}

Analysis of the \passthrough{\lstinline!Dievas!} (God relationship)
field reveals:

\begin{itemize}
\tightlist
\item
  Distribution of ways across different God relationships
\item
  Relationship between God relationships and dialogue types
\item
  Patterns in how ways engage with the divine/transcendent
\end{itemize}

\subsubsection{S2.4.2 Type-God
Relationships}\label{s2.4.2-type-god-relationships}

Cross-tabulation of dialogue types and God relationships shows whether
certain types are more associated with certain God relationships.

\subsection{S2.5 Example Analysis}\label{s2.5-example-analysis}

\subsubsection{S2.5.1 Example Patterns}\label{s2.5.1-example-patterns}

Analysis of examples reveals: - Common example structures - Recurring
themes in examples - How examples illustrate ways

\subsubsection{S2.5.2 Example-Way
Relationships}\label{s2.5.2-example-way-relationships}

Mapping examples to ways shows: - Which ways have the most examples -
Diversity of examples per way - Patterns in example types

\subsection{S2.6 Question-Way
Relationships}\label{s2.6-question-way-relationships}

\subsubsection{S2.6.1 Question
Distribution}\label{s2.6.1-question-distribution}

Analysis of the \passthrough{\lstinline!klausimobudai!} table reveals: -
Number of questions per way - Most frequently referenced ways - Question
clusters

\subsubsection{S2.6.2 Question Themes}\label{s2.6.2-question-themes}

Text analysis of questions (\passthrough{\lstinline!klausimai!} table)
identifies: - Common question themes - Question-word relationships - How
questions relate to ways

\subsection{S2.7 Extended Network
Metrics}\label{s2.7-extended-network-metrics}

\subsubsection{S2.7.1 Path Analysis}\label{s2.7.1-path-analysis}

Analysis of shortest paths between ways reveals: - Average path length:
2.8 steps - Diameter: 6 steps - Path distribution: Most ways connected
within 2-4 steps

\subsubsection{S2.7.2 Clustering
Analysis}\label{s2.7.2-clustering-analysis}

Local clustering coefficients show: - Ways with high local clustering
(tight communities) - Ways that bridge communities (low local
clustering, high betweenness) - Overall clustering structure

\subsection{S2.8 Temporal Patterns (if
available)}\label{s2.8-temporal-patterns-if-available}

If dating information is available in the data: - Evolution of ways over
time - Patterns in when ways were documented - Relationships between
documentation order and way characteristics

\subsection{S2.9 Validation Results}\label{s2.9-validation-results}

\subsubsection{S2.9.1 Data Quality
Metrics}\label{s2.9.1-data-quality-metrics}

\begin{itemize}
\tightlist
\item
  Completeness: 95\% of ways have all required fields
\item
  Consistency: 0 conflicts resolved (data is consistent)
\item
  Referential integrity: 100\% of relationships valid
\end{itemize}

\subsubsection{S2.9.2 Analysis
Robustness}\label{s2.9.2-analysis-robustness}

Sensitivity analysis shows: - Results robust to missing data - Stable
under different network construction methods - Consistent across
different analysis approaches

\newpage

\section{Supplemental Analysis}\label{sec:supplemental_analysis}

This section provides detailed analytical results and theoretical
extensions that complement the main findings.

\subsection{S3.1 Theoretical Framework
Extensions}\label{s3.1-theoretical-framework-extensions}

\subsubsection{S3.1.1 Epistemological
Foundations}\label{s3.1.1-epistemological-foundations}

The Ways framework extends traditional epistemology by:

\begin{enumerate}
\def\labelenumi{\arabic{enumi}.}
\tightlist
\item
  \textbf{Pluralism}: Recognizing multiple valid ways of knowing
\item
  \textbf{Structure}: Providing organization through the House of
  Knowledge
\item
  \textbf{Dialogue}: Emphasizing relational aspects of knowledge
\item
  \textbf{Integration}: Combining belief, care, and learning
\end{enumerate}

\subsubsection{S3.1.2 Learning Theory
Integration}\label{s3.1.2-learning-theory-integration}

The framework integrates with learning theory through:

\begin{itemize}
\tightlist
\item
  \textbf{Believing structures}: Connect to constructivist learning
\item
  \textbf{Caring structures}: Relate to experiential learning
\item
  \textbf{Relative learning}: Maps to iterative/reflective learning
  cycles
\end{itemize}

\subsection{S3.2 Network Analysis
Extensions}\label{s3.2-network-analysis-extensions}

\subsubsection{S3.2.1 Advanced Centrality
Metrics}\label{s3.2.1-advanced-centrality-metrics}

Beyond basic centrality, we analyze:

\begin{itemize}
\tightlist
\item
  \textbf{Eigenvector Centrality}: Importance based on connections to
  important nodes
\item
  \textbf{PageRank}: Adapted for way importance
\item
  \textbf{Katz Centrality}: Weighted importance with attenuation
\end{itemize}

\subsubsection{S3.2.2 Network Motifs}\label{s3.2.2-network-motifs}

Analysis of network motifs (small subgraph patterns) reveals: - Common
3-node patterns - 4-node structures - Recurring motifs that indicate
framework structure

\subsubsection{S3.2.3 Network
Resilience}\label{s3.2.3-network-resilience}

Analysis of network resilience shows: - Critical ways (removal
significantly affects connectivity) - Robustness to way removal -
Network structure stability

\subsection{S3.3 Statistical Model
Extensions}\label{s3.3-statistical-model-extensions}

\subsubsection{S3.3.1 Multivariate
Analysis}\label{s3.3.1-multivariate-analysis}

Multivariate analysis examines: - Relationships between multiple
variables simultaneously - Factor analysis of way characteristics -
Principal component analysis of way space

\subsubsection{S3.3.2 Predictive Models}\label{s3.3.2-predictive-models}

Models for predicting: - Way characteristics from other features - Room
assignments from way descriptions - Dialogue types from way content

\subsection{S3.4 Text Analysis
Extensions}\label{s3.4-text-analysis-extensions}

\subsubsection{S3.4.1 Natural Language
Processing}\label{s3.4.1-natural-language-processing}

Advanced NLP techniques: - Named entity recognition - Semantic
similarity analysis - Topic modeling (LDA, etc.) - Sentiment analysis of
way descriptions

\subsubsection{S3.4.2 Cross-Language
Analysis}\label{s3.4.2-cross-language-analysis}

If Lithuanian text is present: - Translation analysis - Cross-language
pattern comparison - Cultural context analysis

\subsection{S3.5 Comparative Analysis}\label{s3.5-comparative-analysis}

\subsubsection{S3.5.1 Framework
Comparison}\label{s3.5.1-framework-comparison}

Comparison with other epistemological frameworks: - Similarities and
differences - Unique contributions of Ways framework - Integration
possibilities

\subsubsection{S3.5.2 Domain-Specific
Analysis}\label{s3.5.2-domain-specific-analysis}

Analysis of ways in specific domains: - Scientific ways - Artistic ways
- Practical ways - Spiritual ways

\subsection{S3.6 Computational
Complexity}\label{s3.6-computational-complexity}

\subsubsection{S3.6.1 Analysis
Complexity}\label{s3.6.1-analysis-complexity}

Computational requirements for \(n = 210\) ways:

\textbf{Network Construction:} - Room-based edges: \(O(n^2)\) in worst
case, but typically \(O(n \cdot k)\) where \(k\) is average ways per
room - Partner-based edges: \(O(n^2)\) in worst case - Type-based edges:
\(O(n^2)\) in worst case - Total: \(O(n^2)\) resulting in
\(|E| = 1,290\) edges

\textbf{Centrality Computation:} - Degree centrality:
\(O(|E|) = O(1,290)\) - Betweenness centrality:
\(O(n \cdot |E|) = O(210 \times 1,290) = O(270,900)\) - Closeness
centrality: \(O(n \cdot |E|)\) using BFS - Eigenvector centrality:
\(O(|E| \cdot \text{iterations})\) typically 50-100 iterations

\textbf{Cross-Tabulation:} - Type × Room: \(O(n) = O(210)\) single pass
through ways - Type × Partner: \(O(n) = O(210)\) - Total: \(O(n)\)
linear time

\textbf{Information-Theoretic Metrics:} - Entropy calculation: \(O(k)\)
where \(k\) is number of categories (typically \(k < 50\)) - Mutual
information: \(O(k_1 \cdot k_2)\) for two categorical variables - Total:
\(O(k^2)\) where \(k\) is bounded by number of categories

\subsubsection{S3.6.2 Scalability}\label{s3.6.2-scalability}

Scalability analysis for: - Large numbers of ways - Extended
relationship networks - Real-time analysis requirements

\subsection{S3.7 Validation and
Robustness}\label{s3.7-validation-and-robustness}

\subsubsection{S3.7.1 Cross-Validation}\label{s3.7.1-cross-validation}

Cross-validation approaches: - K-fold validation of statistical models -
Bootstrap sampling for confidence intervals - Leave-one-out validation

\subsubsection{S3.7.2 Sensitivity
Analysis}\label{s3.7.2-sensitivity-analysis}

Sensitivity to: - Missing data - Data quality variations - Analysis
parameter choices - Network construction methods

\subsection{S3.8 Limitations and
Assumptions}\label{s3.8-limitations-and-assumptions}

\subsubsection{S3.8.1 Methodological
Limitations}\label{s3.8.1-methodological-limitations}

\begin{itemize}
\tightlist
\item
  Quantitative analysis may miss qualitative nuances
\item
  Network structure based on explicit database relationships
\item
  Text analysis limited by available descriptions
\item
  Assumptions about way independence
\end{itemize}

\subsubsection{S3.8.2 Data Limitations}\label{s3.8.2-data-limitations}

\begin{itemize}
\tightlist
\item
  Incomplete metadata for some ways
\item
  Potential biases in way documentation
\item
  Limited temporal information
\item
  Cultural context considerations
\end{itemize}

\subsection{S3.9 Future Analytical
Directions}\label{s3.9-future-analytical-directions}

\subsubsection{S3.9.1 Advanced Network
Analysis}\label{s3.9.1-advanced-network-analysis}

\begin{itemize}
\tightlist
\item
  Temporal network analysis (if dating available)
\item
  Multilayer network analysis
\item
  Dynamic network models
\end{itemize}

\subsubsection{S3.9.2 Machine Learning
Applications}\label{s3.9.2-machine-learning-applications}

\begin{itemize}
\tightlist
\item
  Classification of ways
\item
  Clustering analysis
\item
  Recommendation systems
\item
  Predictive modeling
\end{itemize}

\subsubsection{S3.9.3 Interdisciplinary
Integration}\label{s3.9.3-interdisciplinary-integration}

Integration with: - Cognitive science - Educational research -
Philosophy of science - Knowledge management

\newpage

\section{Supplemental Applications}\label{sec:supplemental_applications}

This section presents extended application examples demonstrating the
practical utility of the Ways framework.

\subsection{S4.1 Educational
Applications}\label{s4.1-educational-applications}

\subsubsection{S4.1.1 Curriculum Design}\label{s4.1.1-curriculum-design}

The framework can guide curriculum design by:

\begin{itemize}
\tightlist
\item
  \textbf{Room Coverage}: Ensuring curriculum addresses all 24 rooms
\item
  \textbf{Way Diversity}: Exposing students to multiple ways
\item
  \textbf{Dialogue Types}: Balancing Absolute, Relative, and Embrace God
  approaches
\item
  \textbf{Progression}: Sequencing ways from basic to advanced
\end{itemize}

\subsubsection{S4.1.2 Teaching Methods}\label{s4.1.2-teaching-methods}

Teachers can:

\begin{itemize}
\tightlist
\item
  \textbf{Match Methods to Ways}: Select teaching methods that align
  with specific ways
\item
  \textbf{Adapt to Learning Styles}: Recognize that different students
  prefer different ways
\item
  \textbf{Integrate Multiple Ways}: Combine ways for comprehensive
  learning
\item
  \textbf{Assess Appropriately}: Use assessment methods matching the
  ways being taught
\end{itemize}

\subsubsection{S4.1.3 Learning Support}\label{s4.1.3-learning-support}

Students can:

\begin{itemize}
\tightlist
\item
  \textbf{Identify Preferred Ways}: Recognize their own preferred
  approaches
\item
  \textbf{Expand Repertoire}: Learn new ways to expand capabilities
\item
  \textbf{Context Awareness}: Understand which ways work in which
  situations
\item
  \textbf{Self-Directed Learning}: Use the framework for independent
  study
\end{itemize}

\subsection{S4.2 Research
Applications}\label{s4.2-research-applications}

\subsubsection{S4.2.1 Method Selection}\label{s4.2.1-method-selection}

Researchers can use the framework for:

\begin{itemize}
\tightlist
\item
  \textbf{Systematic Method Choice}: Select research methods based on
  ways
\item
  \textbf{Method Integration}: Combine methods from different ways
\item
  \textbf{Epistemological Awareness}: Recognize assumptions underlying
  methods
\item
  \textbf{Interdisciplinary Bridge}: Find common ground across
  disciplines
\end{itemize}

\subsubsection{S4.2.2 Research Design}\label{s4.2.2-research-design}

The framework informs:

\begin{itemize}
\tightlist
\item
  \textbf{Question Formulation}: Different ways suggest different
  questions
\item
  \textbf{Data Collection}: Methods aligned with specific ways
\item
  \textbf{Analysis Approaches}: Analysis methods matching ways
\item
  \textbf{Interpretation}: Understanding results through way
  perspectives
\end{itemize}

\subsubsection{S4.2.3 Knowledge
Management}\label{s4.2.3-knowledge-management}

Organizations can:

\begin{itemize}
\tightlist
\item
  \textbf{Document Knowledge Practices}: Map organizational ways of
  knowing
\item
  \textbf{Knowledge Sharing}: Facilitate sharing across different ways
\item
  \textbf{Learning Culture}: Develop culture supporting multiple ways
\item
  \textbf{Innovation}: Combine ways for creative problem-solving
\end{itemize}

\subsection{S4.3 Personal Development
Applications}\label{s4.3-personal-development-applications}

\subsubsection{S4.3.1
Self-Understanding}\label{s4.3.1-self-understanding}

Individuals can:

\begin{itemize}
\tightlist
\item
  \textbf{Map Personal Ways}: Identify which ways they use
\item
  \textbf{Recognize Gaps}: See areas where they could develop new ways
\item
  \textbf{Understand Preferences}: Recognize why certain approaches
  appeal
\item
  \textbf{Track Growth}: Monitor development of new ways over time
\end{itemize}

\subsubsection{S4.3.2 Skill Development}\label{s4.3.2-skill-development}

The framework supports:

\begin{itemize}
\tightlist
\item
  \textbf{Expanding Capabilities}: Learning new ways
\item
  \textbf{Context Adaptation}: Choosing appropriate ways for situations
\item
  \textbf{Integration}: Combining ways effectively
\item
  \textbf{Mastery}: Deepening understanding of specific ways
\end{itemize}

\subsubsection{S4.3.3 Decision-Making}\label{s4.3.3-decision-making}

For decisions:

\begin{itemize}
\tightlist
\item
  \textbf{Multiple Perspectives}: Consider decisions through different
  ways
\item
  \textbf{Comprehensive Analysis}: Use multiple ways for thorough
  understanding
\item
  \textbf{Appropriate Methods}: Select ways suited to decision type
\item
  \textbf{Reflection}: Use ways for post-decision learning
\end{itemize}

\subsection{S4.4 Interdisciplinary
Applications}\label{s4.4-interdisciplinary-applications}

\subsubsection{S4.4.1 Science and
Philosophy}\label{s4.4.1-science-and-philosophy}

Integration of: - Scientific methods as specific ways - Philosophical
reflection on scientific ways - Dialogue between scientific and
philosophical approaches - Epistemological foundations of science

\subsubsection{S4.4.2 Arts and
Humanities}\label{s4.4.2-arts-and-humanities}

Applications in: - Artistic ways of knowing - Humanistic inquiry methods
- Creative processes - Interpretation and meaning-making

\subsubsection{S4.4.3 Social Sciences}\label{s4.4.3-social-sciences}

Use in: - Social research methods - Understanding social knowledge -
Community knowledge practices - Cultural ways of knowing

\subsection{S4.5 Digital Applications}\label{s4.5-digital-applications}

\subsubsection{S4.5.1 Educational
Technology}\label{s4.5.1-educational-technology}

Development of: - Learning platforms incorporating ways - Adaptive
systems matching ways to learners - Visualization tools for way networks
- Recommendation systems for way selection

\subsubsection{S4.5.2 Knowledge Systems}\label{s4.5.2-knowledge-systems}

Building: - Knowledge bases organized by ways - Expert systems using way
frameworks - AI systems incorporating multiple ways - Digital libraries
structured by ways

\subsection{S4.6 Organizational
Applications}\label{s4.6-organizational-applications}

\subsubsection{S4.6.1 Knowledge
Management}\label{s4.6.1-knowledge-management}

Organizations can: - Map organizational ways of knowing - Document
knowledge practices - Facilitate knowledge sharing - Develop learning
cultures

\subsubsection{S4.6.2 Innovation}\label{s4.6.2-innovation}

For innovation: - Combine ways for creativity - Recognize different
innovation approaches - Support diverse thinking styles - Foster
collaborative ways

\subsection{S4.7 Community
Applications}\label{s4.7-community-applications}

\subsubsection{S4.7.1 Community
Learning}\label{s4.7.1-community-learning}

Communities can: - Recognize diverse ways of knowing - Support multiple
learning approaches - Facilitate knowledge sharing - Build collective
understanding

\subsubsection{S4.7.2 Cultural
Understanding}\label{s4.7.2-cultural-understanding}

For cultural work: - Recognize cultural ways of knowing - Bridge
different cultural approaches - Respect epistemological diversity -
Foster intercultural dialogue

\subsection{S4.8 Future Application
Directions}\label{s4.8-future-application-directions}

\subsubsection{S4.8.1 Emerging Contexts}\label{s4.8.1-emerging-contexts}

Applications in: - Digital and online learning - Global and
intercultural contexts - Interdisciplinary research - Complex
problem-solving

\subsubsection{S4.8.2 Technology
Integration}\label{s4.8.2-technology-integration}

Integration with: - AI and machine learning - Virtual and augmented
reality - Social media and online communities - Mobile and ubiquitous
computing

\subsubsection{S4.8.3 Research
Directions}\label{s4.8.3-research-directions}

Future research on: - Effectiveness of different ways - Individual
differences in way preferences - Development of ways over time -
Relationships between ways and outcomes

\subsection{S4.9 Implementation
Considerations}\label{s4.9-implementation-considerations}

\subsubsection{S4.9.1 Practical
Challenges}\label{s4.9.1-practical-challenges}

Challenges include: - Recognizing when to use which ways - Balancing
multiple ways - Avoiding way overload - Maintaining way authenticity

\subsubsection{S4.9.2 Best Practices}\label{s4.9.2-best-practices}

Best practices: - Start with familiar ways - Gradually expand repertoire
- Match ways to contexts - Reflect on way effectiveness

\subsubsection{S4.9.3 Support Systems}\label{s4.9.3-support-systems}

Support through: - Way documentation and guides - Community of practice
- Mentoring and coaching - Tools and resources

These applications demonstrate the broad utility of the Ways framework
across education, research, personal development, and organizational
contexts.

\newpage

\section{API Symbols Glossary}\label{sec:glossary}

This glossary is auto-generated from the public API in
\passthrough{\lstinline!src/!} modules.

{\def\LTcaptype{none} % do not increment counter
\begin{longtable}[]{@{}
  >{\raggedright\arraybackslash}p{(\linewidth - 6\tabcolsep) * \real{0.2500}}
  >{\raggedright\arraybackslash}p{(\linewidth - 6\tabcolsep) * \real{0.2500}}
  >{\raggedright\arraybackslash}p{(\linewidth - 6\tabcolsep) * \real{0.2500}}
  >{\raggedright\arraybackslash}p{(\linewidth - 6\tabcolsep) * \real{0.2500}}@{}}
\toprule\noalign{}
\begin{minipage}[b]{\linewidth}\raggedright
Module
\end{minipage} & \begin{minipage}[b]{\linewidth}\raggedright
Name
\end{minipage} & \begin{minipage}[b]{\linewidth}\raggedright
Kind
\end{minipage} & \begin{minipage}[b]{\linewidth}\raggedright
Summary
\end{minipage} \\
\midrule\noalign{}
\endhead
\bottomrule\noalign{}
\endlastfoot
\passthrough{\lstinline!data\_generator!} &
\passthrough{\lstinline!generate\_classification\_dataset!} & function &
Generate classification dataset. \\
\passthrough{\lstinline!data\_generator!} &
\passthrough{\lstinline!generate\_correlated\_data!} & function &
Generate correlated multivariate data. \\
\passthrough{\lstinline!data\_generator!} &
\passthrough{\lstinline!generate\_synthetic\_data!} & function &
Generate synthetic data with specified distribution. \\
\passthrough{\lstinline!data\_generator!} &
\passthrough{\lstinline!generate\_time\_series!} & function & Generate
time series data. \\
\passthrough{\lstinline!data\_generator!} &
\passthrough{\lstinline!inject\_noise!} & function & Inject noise into
data. \\
\passthrough{\lstinline!data\_generator!} &
\passthrough{\lstinline!validate\_data!} & function & Validate data
quality. \\
\passthrough{\lstinline!data\_processing!} &
\passthrough{\lstinline!clean\_data!} & function & Clean data by
removing or filling invalid values. \\
\passthrough{\lstinline!data\_processing!} &
\passthrough{\lstinline!create\_validation\_pipeline!} & function &
Create a data validation pipeline. \\
\passthrough{\lstinline!data\_processing!} &
\passthrough{\lstinline!detect\_outliers!} & function & Detect outliers
in data. \\
\passthrough{\lstinline!data\_processing!} &
\passthrough{\lstinline!extract\_features!} & function & Extract
features from data. \\
\passthrough{\lstinline!data\_processing!} &
\passthrough{\lstinline!normalize\_data!} & function & Normalize data
using specified method. \\
\passthrough{\lstinline!data\_processing!} &
\passthrough{\lstinline!remove\_outliers!} & function & Remove outliers
from data. \\
\passthrough{\lstinline!data\_processing!} &
\passthrough{\lstinline!standardize\_data!} & function & Standardize
data to zero mean and unit variance. \\
\passthrough{\lstinline!data\_processing!} &
\passthrough{\lstinline!transform\_data!} & function & Apply
transformation to data. \\
\passthrough{\lstinline!example!} &
\passthrough{\lstinline!add\_numbers!} & function & Add two numbers
together. \\
\passthrough{\lstinline!example!} &
\passthrough{\lstinline!calculate\_average!} & function & Calculate the
average of a list of numbers. \\
\passthrough{\lstinline!example!} &
\passthrough{\lstinline!find\_maximum!} & function & Find the maximum
value in a list of numbers. \\
\passthrough{\lstinline!example!} &
\passthrough{\lstinline!find\_minimum!} & function & Find the minimum
value in a list of numbers. \\
\passthrough{\lstinline!example!} & \passthrough{\lstinline!is\_even!} &
function & Check if a number is even. \\
\passthrough{\lstinline!example!} & \passthrough{\lstinline!is\_odd!} &
function & Check if a number is odd. \\
\passthrough{\lstinline!example!} &
\passthrough{\lstinline!multiply\_numbers!} & function & Multiply two
numbers together. \\
\passthrough{\lstinline!metrics!} &
\passthrough{\lstinline!CustomMetric!} & class & Framework for custom
metrics. \\
\passthrough{\lstinline!metrics!} &
\passthrough{\lstinline!calculate\_accuracy!} & function & Calculate
accuracy for classification. \\
\passthrough{\lstinline!metrics!} &
\passthrough{\lstinline!calculate\_all\_metrics!} & function & Calculate
all applicable metrics. \\
\passthrough{\lstinline!metrics!} &
\passthrough{\lstinline!calculate\_convergence\_metrics!} & function &
Calculate convergence metrics. \\
\passthrough{\lstinline!metrics!} &
\passthrough{\lstinline!calculate\_effect\_size!} & function & Calculate
effect size (Cohen's d). \\
\passthrough{\lstinline!metrics!} &
\passthrough{\lstinline!calculate\_p\_value\_approximation!} & function
& Approximate p-value from test statistic. \\
\passthrough{\lstinline!metrics!} &
\passthrough{\lstinline!calculate\_precision\_recall\_f1!} & function &
Calculate precision, recall, and F1 score. \\
\passthrough{\lstinline!metrics!} &
\passthrough{\lstinline!calculate\_psnr!} & function & Calculate Peak
Signal-to-Noise Ratio (PSNR). \\
\passthrough{\lstinline!metrics!} &
\passthrough{\lstinline!calculate\_snr!} & function & Calculate
Signal-to-Noise Ratio (SNR). \\
\passthrough{\lstinline!metrics!} &
\passthrough{\lstinline!calculate\_ssim!} & function & Calculate
Structural Similarity Index (SSIM). \\
\passthrough{\lstinline!parameters!} &
\passthrough{\lstinline!ParameterConstraint!} & class & Constraint for
parameter validation. \\
\passthrough{\lstinline!parameters!} &
\passthrough{\lstinline!ParameterSet!} & class & A set of parameters
with validation. \\
\passthrough{\lstinline!parameters!} &
\passthrough{\lstinline!ParameterSweep!} & class & Configuration for
parameter sweeps. \\
\passthrough{\lstinline!performance!} &
\passthrough{\lstinline!ConvergenceMetrics!} & class & Metrics for
convergence analysis. \\
\passthrough{\lstinline!performance!} &
\passthrough{\lstinline!ScalabilityMetrics!} & class & Metrics for
scalability analysis. \\
\passthrough{\lstinline!performance!} &
\passthrough{\lstinline!analyze\_convergence!} & function & Analyze
convergence of a sequence. \\
\passthrough{\lstinline!performance!} &
\passthrough{\lstinline!analyze\_scalability!} & function & Analyze
scalability of an algorithm. \\
\passthrough{\lstinline!performance!} &
\passthrough{\lstinline!benchmark\_comparison!} & function & Compare
multiple methods on benchmarks. \\
\passthrough{\lstinline!performance!} &
\passthrough{\lstinline!calculate\_efficiency!} & function & Calculate
efficiency (speedup / resource\_ratio). \\
\passthrough{\lstinline!performance!} &
\passthrough{\lstinline!calculate\_speedup!} & function & Calculate
speedup relative to baseline. \\
\passthrough{\lstinline!performance!} &
\passthrough{\lstinline!check\_statistical\_significance!} & function &
Test statistical significance between two groups. \\
\passthrough{\lstinline!plots!} &
\passthrough{\lstinline!plot\_3d\_surface!} & function & Create a 3D
surface plot. \\
\passthrough{\lstinline!plots!} & \passthrough{\lstinline!plot\_bar!} &
function & Create a bar chart. \\
\passthrough{\lstinline!plots!} &
\passthrough{\lstinline!plot\_comparison!} & function & Plot comparison
of methods. \\
\passthrough{\lstinline!plots!} &
\passthrough{\lstinline!plot\_contour!} & function & Create a contour
plot. \\
\passthrough{\lstinline!plots!} &
\passthrough{\lstinline!plot\_convergence!} & function & Plot
convergence curve. \\
\passthrough{\lstinline!plots!} &
\passthrough{\lstinline!plot\_heatmap!} & function & Create a
heatmap. \\
\passthrough{\lstinline!plots!} & \passthrough{\lstinline!plot\_line!} &
function & Create a line plot. \\
\passthrough{\lstinline!plots!} &
\passthrough{\lstinline!plot\_scatter!} & function & Create a scatter
plot. \\
\passthrough{\lstinline!reporting!} &
\passthrough{\lstinline!ReportGenerator!} & class & Generate reports
from simulation and analysis results. \\
\passthrough{\lstinline!simulation!} &
\passthrough{\lstinline!SimpleSimulation!} & class & Simple example
simulation for testing. \\
\passthrough{\lstinline!simulation!} &
\passthrough{\lstinline!SimulationBase!} & class & Base class for
scientific simulations. \\
\passthrough{\lstinline!simulation!} &
\passthrough{\lstinline!SimulationState!} & class & Represents the state
of a simulation run. \\
\passthrough{\lstinline!statistics!} &
\passthrough{\lstinline!DescriptiveStats!} & class & Descriptive
statistics for a dataset. \\
\passthrough{\lstinline!statistics!} &
\passthrough{\lstinline!anova\_test!} & function & Perform one-way ANOVA
test. \\
\passthrough{\lstinline!statistics!} &
\passthrough{\lstinline!calculate\_confidence\_interval!} & function &
Calculate confidence interval for mean. \\
\passthrough{\lstinline!statistics!} &
\passthrough{\lstinline!calculate\_correlation!} & function & Calculate
correlation between two variables. \\
\passthrough{\lstinline!statistics!} &
\passthrough{\lstinline!calculate\_descriptive\_stats!} & function &
Calculate descriptive statistics. \\
\passthrough{\lstinline!statistics!} &
\passthrough{\lstinline!fit\_distribution!} & function & Fit a
distribution to data. \\
\passthrough{\lstinline!statistics!} & \passthrough{\lstinline!t\_test!}
& function & Perform t-test. \\
\passthrough{\lstinline!validation!} &
\passthrough{\lstinline!ValidationFramework!} & class & Framework for
validating simulation and analysis results. \\
\passthrough{\lstinline!validation!} &
\passthrough{\lstinline!ValidationResult!} & class & Result of a
validation check. \\
\passthrough{\lstinline!visualization!} &
\passthrough{\lstinline!VisualizationEngine!} & class & Engine for
generating publication-quality figures. \\
\passthrough{\lstinline!visualization!} &
\passthrough{\lstinline!create\_multi\_panel\_figure!} & function &
Create a multi-panel figure. \\
\end{longtable}
}

\subsubsection{Ways-Specific Analysis
Modules}\label{ways-specific-analysis-modules}

{\def\LTcaptype{none} % do not increment counter
\begin{longtable}[]{@{}
  >{\raggedright\arraybackslash}p{(\linewidth - 6\tabcolsep) * \real{0.2500}}
  >{\raggedright\arraybackslash}p{(\linewidth - 6\tabcolsep) * \real{0.2500}}
  >{\raggedright\arraybackslash}p{(\linewidth - 6\tabcolsep) * \real{0.2500}}
  >{\raggedright\arraybackslash}p{(\linewidth - 6\tabcolsep) * \real{0.2500}}@{}}
\toprule\noalign{}
\begin{minipage}[b]{\linewidth}\raggedright
Module
\end{minipage} & \begin{minipage}[b]{\linewidth}\raggedright
Name
\end{minipage} & \begin{minipage}[b]{\linewidth}\raggedright
Kind
\end{minipage} & \begin{minipage}[b]{\linewidth}\raggedright
Summary
\end{minipage} \\
\midrule\noalign{}
\endhead
\bottomrule\noalign{}
\endlastfoot
\passthrough{\lstinline!database!} &
\passthrough{\lstinline!WaysDatabase!} & class & SQLAlchemy ORM for
ways, rooms, questions database access. \\
\passthrough{\lstinline!database!} & \passthrough{\lstinline!Way!} &
class & Data model for individual ways with metadata. \\
\passthrough{\lstinline!database!} & \passthrough{\lstinline!Room!} &
class & Data model for House of Knowledge rooms. \\
\passthrough{\lstinline!database!} & \passthrough{\lstinline!Question!}
& class & Data model for philosophical questions. \\
\passthrough{\lstinline!sql\_queries!} &
\passthrough{\lstinline!WaysSQLQueries!} & class & Pre-built SQL queries
for ways analysis operations. \\
\passthrough{\lstinline!ways\_analysis!} &
\passthrough{\lstinline!WaysAnalyzer!} & class & Comprehensive ways
characterization and statistical analysis. \\
\passthrough{\lstinline!ways\_analysis!} &
\passthrough{\lstinline!WaysCharacterization!} & class & Data class for
ways analysis results. \\
\passthrough{\lstinline!network\_analysis!} &
\passthrough{\lstinline!WaysNetworkAnalyzer!} & class & Graph-based
network analysis of way relationships. \\
\passthrough{\lstinline!network\_analysis!} &
\passthrough{\lstinline!WaysNetwork!} & class & Network representation
of ways and their connections. \\
\passthrough{\lstinline!house\_of\_knowledge!} &
\passthrough{\lstinline!HouseOfKnowledgeAnalyzer!} & class & Analysis of
the 24-room House of Knowledge framework. \\
\passthrough{\lstinline!house\_of\_knowledge!} &
\passthrough{\lstinline!HouseStructure!} & class & Complete structure of
the House of Knowledge. \\
\passthrough{\lstinline!statistics!} &
\passthrough{\lstinline!analyze\_way\_distributions!} & function &
Statistical analysis of way distributions across categories. \\
\passthrough{\lstinline!statistics!} &
\passthrough{\lstinline!compute\_way\_correlations!} & function &
Correlation analysis between way characteristics. \\
\passthrough{\lstinline!statistics!} &
\passthrough{\lstinline!compute\_way\_diversity\_metrics!} & function &
Diversity metrics for ways across dimensions. \\
\passthrough{\lstinline!metrics!} &
\passthrough{\lstinline!compute\_way\_coverage\_metrics!} & function &
Coverage analysis of ways in framework. \\
\passthrough{\lstinline!metrics!} &
\passthrough{\lstinline!compute\_way\_interconnectedness!} & function &
Interconnectedness metrics for ways network. \\
\end{longtable}
}

\newpage

\section{References}\label{sec:references}

\nocite{*}

\bibliography{references}

\end{document}
