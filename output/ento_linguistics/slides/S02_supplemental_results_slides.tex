% Options for packages loaded elsewhere
\PassOptionsToPackage{unicode}{hyperref}
\PassOptionsToPackage{hyphens}{url}
\documentclass[
  ignorenonframetext,
]{beamer}
\newif\ifbibliography
\usepackage{pgfpages}
\setbeamertemplate{caption}[numbered]
\setbeamertemplate{caption label separator}{: }
\setbeamercolor{caption name}{fg=normal text.fg}
\beamertemplatenavigationsymbolsempty
% remove section numbering
\setbeamertemplate{part page}{
  \centering
  \begin{beamercolorbox}[sep=16pt,center]{part title}
    \usebeamerfont{part title}\insertpart\par
  \end{beamercolorbox}
}
\setbeamertemplate{section page}{
  \centering
  \begin{beamercolorbox}[sep=12pt,center]{section title}
    \usebeamerfont{section title}\insertsection\par
  \end{beamercolorbox}
}
\setbeamertemplate{subsection page}{
  \centering
  \begin{beamercolorbox}[sep=8pt,center]{subsection title}
    \usebeamerfont{subsection title}\insertsubsection\par
  \end{beamercolorbox}
}
% Prevent slide breaks in the middle of a paragraph
\widowpenalties 1 10000
\raggedbottom
\AtBeginPart{
  \frame{\partpage}
}
\AtBeginSection{
  \ifbibliography
  \else
    \frame{\sectionpage}
  \fi
}
\AtBeginSubsection{
  \frame{\subsectionpage}
}
\usepackage{iftex}
\ifPDFTeX
  \usepackage[T1]{fontenc}
  \usepackage[utf8]{inputenc}
  \usepackage{textcomp} % provide euro and other symbols
\else % if luatex or xetex
  \usepackage{unicode-math} % this also loads fontspec
  \defaultfontfeatures{Scale=MatchLowercase}
  \defaultfontfeatures[\rmfamily]{Ligatures=TeX,Scale=1}
\fi
\usepackage{lmodern}
\ifPDFTeX\else
  % xetex/luatex font selection
\fi
% Use upquote if available, for straight quotes in verbatim environments
\IfFileExists{upquote.sty}{\usepackage{upquote}}{}
\IfFileExists{microtype.sty}{% use microtype if available
  \usepackage[]{microtype}
  \UseMicrotypeSet[protrusion]{basicmath} % disable protrusion for tt fonts
}{}
\makeatletter
\@ifundefined{KOMAClassName}{% if non-KOMA class
  \IfFileExists{parskip.sty}{%
    \usepackage{parskip}
  }{% else
    \setlength{\parindent}{0pt}
    \setlength{\parskip}{6pt plus 2pt minus 1pt}}
}{% if KOMA class
  \KOMAoptions{parskip=half}}
\makeatother
\setlength{\emergencystretch}{3em} % prevent overfull lines
\providecommand{\tightlist}{%
  \setlength{\itemsep}{0pt}\setlength{\parskip}{0pt}}
\usepackage{bookmark}
\IfFileExists{xurl.sty}{\usepackage{xurl}}{} % add URL line breaks if available
\urlstyle{same}
\hypersetup{
  hidelinks,
  pdfcreator={LaTeX via pandoc}}

\author{\texorpdfstring{}{}}
\date{}

\begin{document}

\begin{frame}{Supplemental Results}
\protect\phantomsection\label{sec:supplemental_results}
This section provides additional experimental results that complement
the computational analysis presented in Section
\ref{sec:experimental_results}.

\begin{block}{S2.1 Extended Domain-Specific Analyses}
\protect\phantomsection\label{s2.1-extended-domain-specific-analyses}
\begin{block}{S2.1.1 Additional Terminology Extraction Results}
\protect\phantomsection\label{s2.1.1-additional-terminology-extraction-results}
Our analysis identified additional terminology patterns across the six
Ento-Linguistic domains:

\begin{table}[h]
\centering
\begin{tabular}{|l|c|c|c|c|c|}
\hline
\textbf{Domain} & \textbf{Additional Terms} & \textbf{Sub-domains} & \textbf{Cross-domain Links} & \textbf{Ambiguity Patterns} \\
\hline
Unit of Individuality & 89 & 4 & 156 & Scale transitions \\
Behavior and Identity & 134 & 6 & 203 & Context-dependent roles \\
Power & Labor & 98 & 3 & 187 & Authority structures \\
Sex & Reproduction & 67 & 2 & 98 & Binary assumptions \\
Kin & Relatedness & 76 & 5 & 145 & Relationship complexity \\
Economics & 45 & 2 & 67 & Resource metaphors \\
\hline
\end{tabular}
\caption{Extended terminology extraction results showing sub-domains and cross-domain relationships}
\label{tab:extended_terminology}
\end{table}
\end{block}

\begin{block}{S2.1.2 Sub-Domain Analysis}
\protect\phantomsection\label{s2.1.2-sub-domain-analysis}
Each major domain contains distinct sub-domains with characteristic
terminology patterns:

\textbf{Unit of Individuality Sub-domains}: - Colony-level concepts
(superorganism, eusociality) - Individual-level concepts (nestmate
recognition, division of labor) - Scale transitions (colony → individual
→ genome)

\textbf{Behavior and Identity Sub-domains}: - Task specialization
(foraging, nursing, defense) - Age-related roles (temporal polyethism) -
Context-dependent flexibility (task switching)
\end{block}
\end{block}

\begin{block}{S2.2 Extended Network Analysis Results}
\protect\phantomsection\label{s2.2-extended-network-analysis-results}
\begin{block}{S2.2.1 Network Structural Properties}
\protect\phantomsection\label{s2.2.1-network-structural-properties}
Extended analysis of terminology networks reveals additional structural
patterns:

\begin{table}[h]
\centering
\begin{tabular}{|l|c|c|c|c|c|}
\hline
\textbf{Network Property} & \textbf{Unit} & \textbf{Behavior} & \textbf{Power} & \textbf{Sex} & \textbf{Economics} \\
\hline
Betweenness Centrality & 0.23 & 0.31 & 0.18 & 0.12 & 0.09 \\
Clustering Coefficient & 0.67 & 0.71 & 0.58 & 0.62 & 0.55 \\
Average Path Length & 3.2 & 2.8 & 3.7 & 4.1 & 3.9 \\
Network Diameter & 8 & 7 & 9 & 10 & 8 \\
Small World Coefficient & 2.1 & 2.3 & 1.8 & 1.9 & 1.7 \\
\hline
\end{tabular}
\caption{Extended network structural properties across all Ento-Linguistic domains}
\label{tab:extended_network_properties}
\end{table}
\end{block}

\begin{block}{S2.2.2 Cross-Domain Relationship Analysis}
\protect\phantomsection\label{s2.2.2-cross-domain-relationship-analysis}
Analysis of relationships between domains reveals conceptual bridges:

\begin{figure}[h]
\centering
\includegraphics[width=0.9\textwidth]{../figures/domain_comparison.png}
\caption{Cross-domain relationship analysis showing conceptual bridges between Ento-Linguistic domains}
\label{fig:cross_domain_relationships}
\end{figure}

\textbf{Key Cross-Domain Bridges}: - Power \& Labor ↔ Behavior and
Identity (role assignment mechanisms) - Unit of Individuality ↔ Kin \&
Relatedness (social structure foundations) - Economics ↔ Power \& Labor
(resource distribution hierarchies)
\end{block}
\end{block}

\begin{block}{S2.3 Extended Framing Analysis}
\protect\phantomsection\label{s2.3-extended-framing-analysis}
\begin{block}{S2.3.1 Framing Prevalence Across Domains}
\protect\phantomsection\label{s2.3.1-framing-prevalence-across-domains}
Extended analysis of framing assumptions reveals domain-specific
patterns:

\begin{table}[h]
\centering
\begin{tabular}{|l|c|c|c|c|c|}
\hline
\textbf{Framing Type} & \textbf{Unit (\%)} & \textbf{Behavior (\%)} & \textbf{Power (\%)} & \textbf{Sex (\%)} & \textbf{Economics (\%)} \\
\hline
Anthropomorphic & 68.3 & 71.2 & 45.8 & 23.1 & 34.7 \\
Hierarchical & 45.8 & 32.4 & 89.2 & 12.3 & 67.8 \\
Economic & 23.1 & 18.9 & 34.5 & 8.7 & 91.3 \\
Kinship-based & 34.7 & 41.2 & 23.4 & 76.5 & 28.9 \\
Technological & 12.4 & 28.7 & 15.6 & 9.8 & 45.2 \\
Biological & 87.6 & 93.1 & 78.9 & 95.4 & 72.3 \\
\hline
\end{tabular}
\caption{Framing prevalence across individual Ento-Linguistic domains}
\label{tab:domain_framing_prevalence}
\end{table}
\end{block}

\begin{block}{S2.3.2 Framing Evolution Over Time}
\protect\phantomsection\label{s2.3.2-framing-evolution-over-time}
Analysis of framing patterns across publication decades:

\begin{figure}[h]
\centering
\includegraphics[width=0.9\textwidth]{../figures/domain_comparison.png}
\caption{Evolution of framing assumptions in entomological literature over time}
\label{fig:framing_evolution}
\end{figure}

\textbf{Temporal Trends}: - Anthropomorphic framing decreased from 75\%
(1970s) to 45\% (2020s) - Economic framing increased from 15\% (1970s)
to 65\% (2020s) - Hierarchical framing remained stable at
\textasciitilde50\% across decades
\end{block}
\end{block}

\begin{block}{S2.4 Extended Case Studies}
\protect\phantomsection\label{s2.4-extended-case-studies}
\begin{block}{S2.4.1 Caste Terminology Evolution: 1970-2024}
\protect\phantomsection\label{s2.4.1-caste-terminology-evolution-1970-2024}
Longitudinal analysis reveals changing conceptual frameworks:

\begin{figure}[h]
\centering
\includegraphics[width=0.9\textwidth]{../figures/domain_comparison.png}
\caption{Longitudinal evolution of caste terminology usage patterns}
\label{fig:caste_evolution_extended}
\end{figure}

\textbf{Decadal Shifts}: - \textbf{1970s-1980s}: Rigid caste categories
dominant (92\% usage) - \textbf{1990s-2000s}: Transition to task-based
understanding (67\% traditional caste) - \textbf{2010s-2024}:
Recognition of plasticity and individual variation (34\% traditional
caste)
\end{block}

\begin{block}{S2.4.2 Superorganism Debate: Conceptual Evolution}
\protect\phantomsection\label{s2.4.2-superorganism-debate-conceptual-evolution}
Extended analysis of superorganism terminology evolution:

\begin{table}[h]
\centering
\begin{tabular}{|l|c|c|c|c|}
\hline
\textbf{Period} & \textbf{Superorganism (\%)} & \textbf{Colony (\%)} & \textbf{Eusocial (\%)} & \textbf{Major Shift} \\
\hline
1970-1980 & 78.3 & 12.4 & 9.3 & Emergence of superorganism concept \\
1980-1990 & 65.7 & 23.1 & 11.2 & Introduction of colony-level analysis \\
1990-2000 & 43.2 & 38.9 & 17.9 & Recognition of individual variation \\
2000-2010 & 28.7 & 52.1 & 19.2 & Integration of genomic perspectives \\
2010-2024 & 18.3 & 61.5 & 20.2 & Multi-scale individuality frameworks \\
\hline
\end{tabular}
\caption{Evolution of superorganism debate terminology across decades}
\label{tab:superorganism_evolution}
\end{table}
\end{block}
\end{block}

\begin{block}{S2.5 Extended Statistical Validation}
\protect\phantomsection\label{s2.5-extended-statistical-validation}
\begin{block}{S2.5.1 Inter-annotator Agreement Results}
\protect\phantomsection\label{s2.5.1-inter-annotator-agreement-results}
Comprehensive validation across multiple annotators:

\begin{table}[h]
\centering
\begin{tabular}{|l|c|c|c|}
\hline
\textbf{Agreement Metric} & \textbf{Term Classification} & \textbf{Framing Identification} & \textbf{Ambiguity Detection} \\
\hline
Cohen's Kappa & 0.87 & 0.82 & 0.79 \\
Fleiss' Kappa & 0.85 & 0.80 & 0.76 \\
Percentage Agreement & 91.3\% & 87.6\% & 84.2\% \\
\hline
\end{tabular}
\caption{Inter-annotator agreement results for key analysis components}
\label{tab:inter_annotator_agreement}
\end{table}
\end{block}

\begin{block}{S2.5.2 Bootstrap Validation Results}
\protect\phantomsection\label{s2.5.2-bootstrap-validation-results}
Stability analysis across 1000 bootstrap samples:

\begin{itemize}
\tightlist
\item
  \textbf{Terminology extraction}: 94.3\% stability (SD = 2.1\%)
\item
  \textbf{Domain classification}: 91.7\% stability (SD = 3.2\%)
\item
  \textbf{Network structure}: 88.9\% stability (SD = 4.1\%)
\item
  \textbf{Framing identification}: 86.4\% stability (SD = 4.8\%)
\end{itemize}
\end{block}
\end{block}

\begin{block}{S2.6 Additional Domain-Specific Figures}
\protect\phantomsection\label{s2.6-additional-domain-specific-figures}
\begin{block}{S2.6.1 Domain-Specific Visualizations}
\protect\phantomsection\label{s2.6.1-domain-specific-visualizations}
Extended visualizations for each domain provide deeper insights:

\textbf{Unit of Individuality Domain}:

\begin{figure}[h]
\centering
\includegraphics[width=0.9\textwidth]{../figures/unit_of_individuality_term_frequencies.png}
\caption{Term frequency distribution in Unit of Individuality domain}
\label{fig:unit_individuality_frequencies}
\end{figure}

\begin{figure}[h]
\centering
\includegraphics[width=0.9\textwidth]{../figures/unit_of_individuality_ambiguities.png}
\caption{Ambiguity patterns in Unit of Individuality terminology}
\label{fig:unit_individuality_ambiguities}
\end{figure}

\textbf{Behavior and Identity Domain}:

\begin{figure}[h]
\centering
\includegraphics[width=0.9\textwidth]{../figures/behavior_and_identity_term_frequencies.png}
\caption{Behavioral terminology frequency distribution}
\label{fig:behavior_identity_frequencies}
\end{figure}

\begin{figure}[h]
\centering
\includegraphics[width=0.9\textwidth]{../figures/behavior_and_identity_ambiguities.png}
\caption{Identity-related ambiguity patterns}
\label{fig:behavior_identity_ambiguities}
\end{figure}

\textbf{Power \& Labor Domain}:

\begin{figure}[h]
\centering
\includegraphics[width=0.9\textwidth]{../figures/power_labor_term_frequencies.png}
\caption{Hierarchical terminology frequency distribution}
\label{fig:power_labor_frequencies}
\end{figure}

\begin{figure}[h]
\centering
\includegraphics[width=0.9\textwidth]{../figures/power_labor_ambiguities.png}
\caption{Power and labor related ambiguity patterns}
\label{fig:power_labor_ambiguities}
\end{figure}

\textbf{Sex \& Reproduction Domain}:

\begin{figure}[h]
\centering
\includegraphics[width=0.9\textwidth]{../figures/sex_and_reproduction_term_frequencies.png}
\caption{Reproductive terminology frequency distribution}
\label{fig:sex_reproduction_frequencies}
\end{figure}

\begin{figure}[h]
\centering
\includegraphics[width=0.9\textwidth]{../figures/sex_and_reproduction_ambiguities.png}
\caption{Reproductive terminology ambiguity patterns}
\label{fig:sex_reproduction_ambiguities}
\end{figure}

\textbf{Kin \& Relatedness Domain}:

\begin{figure}[h]
\centering
\includegraphics[width=0.9\textwidth]{../figures/kin_and_relatedness_term_frequencies.png}
\caption{Kinship terminology frequency distribution}
\label{fig:kin_relatedness_frequencies}
\end{figure}

\begin{figure}[h]
\centering
\includegraphics[width=0.9\textwidth]{../figures/kin_and_relatedness_ambiguities.png}
\caption{Kinship terminology ambiguity patterns}
\label{fig:kin_relatedness_ambiguities}
\end{figure}

These extended results provide comprehensive coverage of the
Ento-Linguistic domains, revealing complex patterns of terminology use,
framing assumptions, and conceptual evolution in entomological research.
\end{block}
\end{block}
\end{frame}

\end{document}
