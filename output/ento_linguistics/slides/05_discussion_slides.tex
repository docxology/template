% Options for packages loaded elsewhere
\PassOptionsToPackage{unicode}{hyperref}
\PassOptionsToPackage{hyphens}{url}
\documentclass[
  ignorenonframetext,
]{beamer}
\newif\ifbibliography
\usepackage{pgfpages}
\setbeamertemplate{caption}[numbered]
\setbeamertemplate{caption label separator}{: }
\setbeamercolor{caption name}{fg=normal text.fg}
\beamertemplatenavigationsymbolsempty
% remove section numbering
\setbeamertemplate{part page}{
  \centering
  \begin{beamercolorbox}[sep=16pt,center]{part title}
    \usebeamerfont{part title}\insertpart\par
  \end{beamercolorbox}
}
\setbeamertemplate{section page}{
  \centering
  \begin{beamercolorbox}[sep=12pt,center]{section title}
    \usebeamerfont{section title}\insertsection\par
  \end{beamercolorbox}
}
\setbeamertemplate{subsection page}{
  \centering
  \begin{beamercolorbox}[sep=8pt,center]{subsection title}
    \usebeamerfont{subsection title}\insertsubsection\par
  \end{beamercolorbox}
}
% Prevent slide breaks in the middle of a paragraph
\widowpenalties 1 10000
\raggedbottom
\AtBeginPart{
  \frame{\partpage}
}
\AtBeginSection{
  \ifbibliography
  \else
    \frame{\sectionpage}
  \fi
}
\AtBeginSubsection{
  \frame{\subsectionpage}
}
\usepackage{iftex}
\ifPDFTeX
  \usepackage[T1]{fontenc}
  \usepackage[utf8]{inputenc}
  \usepackage{textcomp} % provide euro and other symbols
\else % if luatex or xetex
  \usepackage{unicode-math} % this also loads fontspec
  \defaultfontfeatures{Scale=MatchLowercase}
  \defaultfontfeatures[\rmfamily]{Ligatures=TeX,Scale=1}
\fi
\usepackage{lmodern}
\ifPDFTeX\else
  % xetex/luatex font selection
\fi
% Use upquote if available, for straight quotes in verbatim environments
\IfFileExists{upquote.sty}{\usepackage{upquote}}{}
\IfFileExists{microtype.sty}{% use microtype if available
  \usepackage[]{microtype}
  \UseMicrotypeSet[protrusion]{basicmath} % disable protrusion for tt fonts
}{}
\makeatletter
\@ifundefined{KOMAClassName}{% if non-KOMA class
  \IfFileExists{parskip.sty}{%
    \usepackage{parskip}
  }{% else
    \setlength{\parindent}{0pt}
    \setlength{\parskip}{6pt plus 2pt minus 1pt}}
}{% if KOMA class
  \KOMAoptions{parskip=half}}
\makeatother
\setlength{\emergencystretch}{3em} % prevent overfull lines
\providecommand{\tightlist}{%
  \setlength{\itemsep}{0pt}\setlength{\parskip}{0pt}}
\usepackage{bookmark}
\IfFileExists{xurl.sty}{\usepackage{xurl}}{} % add URL line breaks if available
\urlstyle{same}
\hypersetup{
  hidelinks,
  pdfcreator={LaTeX via pandoc}}

\author{\texorpdfstring{}{}}
\date{}

\begin{document}

\section{Discussion}\label{sec:discussion}

\begin{frame}{Theoretical Implications of Language as Constitutive in
Scientific Practice}
\protect\phantomsection\label{theoretical-implications-of-language-as-constitutive-in-scientific-practice}
The computational analysis presented in Section
\ref{sec:experimental_results} reveals profound theoretical implications
for understanding how language actively constitutes scientific knowledge
rather than merely representing it. Our findings demonstrate that
terminology networks in entomology are not neutral descriptive tools,
but active frameworks that shape research questions, methodological
choices, and interpretive possibilities.

\begin{block}{The Constitutive Role of Scientific Language}
\protect\phantomsection\label{the-constitutive-role-of-scientific-language}
Our analysis of Ento-Linguistic domains reveals systematic patterns
where terminology imposes conceptual structures on biological phenomena:

\textbf{Hierarchical Imposition}: The Power \& Labor domain demonstrates
how terms like ``caste,'' ``queen,'' and ``worker'' import human social
hierarchies into ant biology, creating analytical frameworks that may
not reflect biological reality.

\textbf{Scale Construction}: The Unit of Individuality domain shows how
terminology creates artificial boundaries between biological scales,
with ``colony'' and ``superorganism'' concepts shaping debates about
biological individuality.

\textbf{Identity Formation}: Behavioral descriptions in the Behavior and
Identity domain transform fluid biological processes into categorical
identities, influencing how researchers perceive and study ant social
organization.
\end{block}

\begin{block}{Network Theory and Scientific Discourse}
\protect\phantomsection\label{network-theory-and-scientific-discourse}
The terminology networks we constructed reveal structural properties of
scientific language that have implications for knowledge production:

\begin{equation}\label{eq:discourse_network_impact}
I(\text{discourse}) = \sum_{d \in D} w_d \cdot C_d \cdot A_d
\end{equation}

where \(I(\text{discourse})\) represents the impact of discourse
structure on knowledge production, \(w_d\) is domain weight, \(C_d\) is
conceptual clustering, and \(A_d\) is ambiguity density.

\textbf{Clustering Effects}: High clustering coefficients in domain
networks suggest that scientific communities develop specialized
terminological dialects that may inhibit interdisciplinary
communication.

\textbf{Bridging Terms}: Low-degree terms that connect multiple domains
represent potential points of conceptual integration or confusion.
\end{block}
\end{frame}

\begin{frame}{Comparison with Existing Discourse Analysis Frameworks}
\protect\phantomsection\label{comparison-with-existing-discourse-analysis-frameworks}
\begin{block}{Scientific Discourse Analysis Traditions}
\protect\phantomsection\label{scientific-discourse-analysis-traditions}
Our work extends several established frameworks for analyzing scientific
language:

\textbf{Sociology of Scientific Knowledge (SSK)}: Our findings support
SSK arguments that scientific facts are socially constructed,
demonstrating how terminology networks embody social negotiations about
biological reality \cite{latour1987}.

\textbf{Feminist Epistemology}: The pervasive anthropomorphic framing we
identified aligns with feminist critiques of androcentric science, where
human social categories are projected onto nature \cite{haraway1991}.

\textbf{Philosophy of Language in Science}: Our context-dependent
analysis supports arguments that scientific terms gain meaning through
use within communities, rather than possessing fixed,
context-independent definitions \cite{kuhn1996}.
\end{block}

\begin{block}{Linguistic Anthropology Approaches}
\protect\phantomsection\label{linguistic-anthropology-approaches}
\textbf{Ethnoscience and Folk Taxonomies}: The categorical structures
imposed by entomological terminology parallel ethnoscientific
classifications, where cultural categories shape perception of natural
phenomena \cite{berlin1992}.

\textbf{Language Ideology}: Our analysis of framing assumptions reveals
how language ideologies in science privilege certain ways of knowing
while marginalizing others.
\end{block}
\end{frame}

\begin{frame}{Implications for Scientific Communication}
\protect\phantomsection\label{implications-for-scientific-communication}
\begin{block}{Language as Research Constraint}
\protect\phantomsection\label{language-as-research-constraint}
Our findings demonstrate how terminology networks create invisible
constraints on scientific inquiry:

\textbf{Question Formulation}: Researchers working within established
terminological frameworks may fail to ask questions that fall outside
those frameworks.

\textbf{Methodological Choices}: Terminological assumptions influence
which methods are considered appropriate or ``natural'' for studying
phenomena.

\textbf{Interpretive Frameworks}: Established terminology provides
ready-made interpretive categories that may not fit complex biological
realities.
\end{block}

\begin{block}{The Ethics of Scientific Language}
\protect\phantomsection\label{the-ethics-of-scientific-language}
The entanglement of speech and thought in scientific practice raises
ethical questions about responsibility for language use:

\textbf{Communicative Clarity}: In value-aligned scientific communities,
researchers have an ethical obligation to use language that maximizes
clarity and minimizes unnecessary confusion.

\textbf{Terminological Stewardship}: Scientific communities should
actively curate their terminology to ensure it serves research goals
rather than perpetuating historical accidents.

\textbf{Inclusive Language}: Recognition of anthropomorphic and
hierarchical framings calls for more inclusive terminological practices
that avoid inappropriate projections of human social structures.
\end{block}

\begin{block}{Practical Recommendations for Researchers}
\protect\phantomsection\label{practical-recommendations-for-researchers}
Based on our analysis, we offer concrete recommendations for improving
terminological practices in entomological research:

\textbf{1. Terminological Awareness}: Researchers should maintain
conscious awareness of the conceptual frameworks embedded in scientific
terminology, particularly when terms carry implicit assumptions about
social structure or individuality.

\textbf{2. Alternative Terminology}: When established terms create
confusion or inappropriate framings, researchers should consider
developing or adopting clearer alternatives. For example, replacing
``slave'' with ``worker'' in ant literature represents an improvement in
communicative clarity.

\textbf{3. Cross-Domain Translation}: Researchers working across
disciplines should be prepared to translate concepts between different
terminological frameworks, recognizing that terms may carry different
meanings in different contexts.

\textbf{4. Critical Language Analysis}: Scientific training should
include instruction in analyzing how language shapes research questions
and interpretations, preparing researchers to critically examine their
terminological choices.
\end{block}
\end{frame}

\begin{frame}{Broader Implications for Scientific Practice}
\protect\phantomsection\label{broader-implications-for-scientific-practice}
\begin{block}{Interdisciplinarity and Communication}
\protect\phantomsection\label{interdisciplinarity-and-communication}
The structural properties of terminology networks have implications for
interdisciplinary research:

\textbf{Dialect Formation}: Specialized domains develop terminological
dialects that create communication barriers between subdisciplines.

\textbf{Conceptual Translation}: Moving between domains requires not
just linguistic translation, but conceptual reframing.

\textbf{Knowledge Integration}: Effective integration of findings across
domains requires attention to terminological differences.
\end{block}

\begin{block}{Research Evaluation and Peer Review}
\protect\phantomsection\label{research-evaluation-and-peer-review}
Our analysis suggests that language use should be considered in research
evaluation:

\textbf{Clarity as Quality Metric}: The clarity and appropriateness of
terminology should be evaluated alongside methodological rigor.

\textbf{Terminological Innovation}: Research that successfully addresses
terminological limitations should be valued.

\textbf{Communication Standards}: Scientific communities should develop
standards for terminological clarity and appropriateness.
\end{block}
\end{frame}

\begin{frame}{Limitations and Methodological Considerations}
\protect\phantomsection\label{limitations-and-methodological-considerations}
\begin{block}{Scope Limitations}
\protect\phantomsection\label{scope-limitations}
\begin{enumerate}
\tightlist
\item
  \textbf{Corpus Boundaries}: Our analysis is limited to
  English-language entomological literature; multilingual patterns
  unexplored
\item
  \textbf{Temporal Scope}: Cross-sectional analysis cannot capture
  terminological evolution
\item
  \textbf{Domain Coverage}: While comprehensive within entomology,
  patterns may differ in other biological disciplines
\item
  \textbf{Context Window Constraints}: 50-word co-occurrence windows may
  miss long-range conceptual relationships
\end{enumerate}
\end{block}

\begin{block}{Methodological Challenges}
\protect\phantomsection\label{methodological-challenges}
\begin{enumerate}
\tightlist
\item
  \textbf{Ambiguity Detection}: Automated ambiguity detection relies on
  statistical patterns that may miss subtle conceptual distinctions
\item
  \textbf{Context Classification}: Determining appropriate contexts for
  term usage remains partly interpretive
\item
  \textbf{Framing Identification}: Anthropomorphic and hierarchical
  framings are identified statistically but require theoretical
  interpretation
\item
  \textbf{Network Construction}: Edge weight calculations balance
  sensitivity and specificity but remain approximations
\end{enumerate}
\end{block}
\end{frame}

\begin{frame}{Future Research Directions}
\protect\phantomsection\label{future-research-directions}
\begin{block}{Theoretical Developments}
\protect\phantomsection\label{theoretical-developments}
\textbf{Extended Discourse Analysis}: Develop more sophisticated
frameworks for analyzing how language constitutes scientific objects and
relationships.

\textbf{Longitudinal Studies}: Track terminological evolution over time
to understand how scientific language changes with theoretical
developments.

\textbf{Comparative Analysis}: Compare terminological patterns across
biological disciplines to identify general principles of scientific
language use.
\end{block}

\begin{block}{Methodological Advancements}
\protect\phantomsection\label{methodological-advancements}
\textbf{Multilingual Analysis}: Extend analysis to non-English
scientific literature to identify cross-cultural terminological
patterns.

\textbf{Semantic Network Analysis}: Incorporate semantic analysis
techniques to better capture conceptual relationships.

\textbf{Interactive Terminology Tools}: Develop tools that help
researchers navigate terminological complexity and identify appropriate
language use.
\end{block}

\begin{block}{Practical Applications}
\protect\phantomsection\label{practical-applications}
\textbf{Terminology Guidelines}: Develop evidence-based guidelines for
clear scientific communication in biology.

\textbf{Educational Tools}: Create training materials that help
researchers understand how language shapes their work.

\textbf{Peer Review Frameworks}: Integrate language analysis into peer
review processes to improve scientific communication quality.
\end{block}
\end{frame}

\begin{frame}{Meta-Standards for Scientific Communication}
\protect\phantomsection\label{meta-standards-for-scientific-communication}
Our work establishes foundations for meta-standards that scientific
communities can use to evaluate and improve their communication
practices:

\textbf{Clarity Standards}: Terminology should maximize understanding
while minimizing unnecessary ambiguity.

\textbf{Appropriateness Standards}: Language should be appropriate to
the phenomena being described, avoiding inappropriate projections of
human social structures.

\textbf{Consistency Standards}: Within research communities, terminology
should be used consistently to facilitate communication.

\textbf{Evolution Standards}: Communities should have mechanisms for
terminological evolution as understanding develops.
\end{frame}

\begin{frame}{Conclusion}
\protect\phantomsection\label{conclusion}
The Ento-Linguistic analysis reveals that scientific language is not a
transparent medium for representing biological reality, but an active
constituent of scientific knowledge. Terminology networks shape research
questions, methodological choices, and interpretive frameworks in ways
that are often invisible to practitioners. By making these constitutive
effects visible, our work provides a foundation for more conscious and
responsible scientific communication practices. The ethical imperative
for clear communication in value-aligned scientific communities calls
for active terminological stewardship and the development of
meta-standards for evaluating language use in research. Future work
should extend these insights across disciplines while developing
practical tools for improving scientific discourse.
\end{frame}

\begin{frame}{Limitations and Future Directions}
\protect\phantomsection\label{limitations-and-future-directions}
\begin{block}{Methodological Limitations}
\protect\phantomsection\label{methodological-limitations}
While our Ento-Linguistic analysis provides comprehensive insights into
terminology use in entomology, several methodological constraints
warrant consideration:

\begin{enumerate}
\tightlist
\item
  \textbf{Corpus Scope}: Analysis limited to English-language
  entomological literature; multilingual patterns unexplored
\item
  \textbf{Temporal Range}: Cross-sectional analysis cannot fully capture
  terminological evolution over time
\item
  \textbf{Context Window Size}: 50-word co-occurrence windows may miss
  long-range conceptual relationships
\item
  \textbf{Domain Boundaries}: Some terms span multiple domains, creating
  classification challenges
\end{enumerate}
\end{block}

\begin{block}{Theoretical Scope Considerations}
\protect\phantomsection\label{theoretical-scope-considerations}
Our framework successfully identifies framing assumptions and contextual
variation in scientific language, but faces inherent challenges in
discourse analysis:

\begin{enumerate}
\tightlist
\item
  \textbf{Ambiguity Detection}: Automated ambiguity detection relies on
  statistical patterns that may miss subtle conceptual distinctions
\item
  \textbf{Context Classification}: Determining appropriate contexts for
  term usage remains partly interpretive
\item
  \textbf{Framing Identification}: Anthropomorphic and hierarchical
  framings are identified statistically but require theoretical
  interpretation
\item
  \textbf{Network Construction}: Edge weight calculations balance
  sensitivity and specificity but remain approximations
\end{enumerate}
\end{block}

\begin{block}{Future Research Directions}
\protect\phantomsection\label{future-research-directions-1}
\begin{block}{Extended Methodological Development}
\protect\phantomsection\label{extended-methodological-development}
\textbf{Multilingual Analysis}: Extend Ento-Linguistic analysis to
non-English scientific literature to identify cross-cultural
terminological patterns. For example, comparing German ``Staaten''
vs.~English ``colony'' terminology in social insect research.

\textbf{Longitudinal Studies}: Track terminological evolution over time
to understand how scientific language changes with theoretical
developments. This could reveal how the shift from ``superorganism'' to
``colonial'' perspectives altered research questions in entomology.

\textbf{Advanced Semantic Analysis}: Integrate transformer-based
embeddings and advanced semantic analysis techniques to better capture
conceptual relationships in scientific terminology.
\end{block}

\begin{block}{Theoretical Advancements}
\protect\phantomsection\label{theoretical-advancements}
\textbf{Extended Discourse Frameworks}: Develop more sophisticated
theories of how scientific language constitutes research objects and
relationships beyond the six domains analyzed here.

\textbf{Cross-Disciplinary Applications}: Apply Ento-Linguistic methods
to other scientific disciplines to identify general principles of
scientific communication. Compare terminological patterns in
evolutionary biology, neuroscience, and ecology.

\textbf{Interactive Terminology Tools}: Develop software tools that help
researchers navigate terminological complexity and identify appropriate
language use in real-time.
\end{block}

\begin{block}{Practical Applications}
\protect\phantomsection\label{practical-applications-1}
\textbf{Terminology Guidelines}: Create evidence-based guidelines for
clear scientific communication across biological disciplines, building
on the meta-standards developed in this work.

\textbf{Educational Interventions}: Develop training programs that help
researchers understand how language shapes their work and establish
conscious practices for terminological stewardship.

\textbf{Peer Review Integration}: Incorporate language clarity
assessment into scientific peer review processes to improve
communication quality across disciplines.
\end{block}
\end{block}
\end{frame}

\end{document}
