% Options for packages loaded elsewhere
\PassOptionsToPackage{unicode}{hyperref}
\PassOptionsToPackage{hyphens}{url}
\documentclass[
  ignorenonframetext,
]{beamer}
\newif\ifbibliography
\usepackage{pgfpages}
\setbeamertemplate{caption}[numbered]
\setbeamertemplate{caption label separator}{: }
\setbeamercolor{caption name}{fg=normal text.fg}
\beamertemplatenavigationsymbolsempty
% remove section numbering
\setbeamertemplate{part page}{
  \centering
  \begin{beamercolorbox}[sep=16pt,center]{part title}
    \usebeamerfont{part title}\insertpart\par
  \end{beamercolorbox}
}
\setbeamertemplate{section page}{
  \centering
  \begin{beamercolorbox}[sep=12pt,center]{section title}
    \usebeamerfont{section title}\insertsection\par
  \end{beamercolorbox}
}
\setbeamertemplate{subsection page}{
  \centering
  \begin{beamercolorbox}[sep=8pt,center]{subsection title}
    \usebeamerfont{subsection title}\insertsubsection\par
  \end{beamercolorbox}
}
% Prevent slide breaks in the middle of a paragraph
\widowpenalties 1 10000
\raggedbottom
\AtBeginPart{
  \frame{\partpage}
}
\AtBeginSection{
  \ifbibliography
  \else
    \frame{\sectionpage}
  \fi
}
\AtBeginSubsection{
  \frame{\subsectionpage}
}
\usepackage{iftex}
\ifPDFTeX
  \usepackage[T1]{fontenc}
  \usepackage[utf8]{inputenc}
  \usepackage{textcomp} % provide euro and other symbols
\else % if luatex or xetex
  \usepackage{unicode-math} % this also loads fontspec
  \defaultfontfeatures{Scale=MatchLowercase}
  \defaultfontfeatures[\rmfamily]{Ligatures=TeX,Scale=1}
\fi
\usepackage{lmodern}
\ifPDFTeX\else
  % xetex/luatex font selection
\fi
% Use upquote if available, for straight quotes in verbatim environments
\IfFileExists{upquote.sty}{\usepackage{upquote}}{}
\IfFileExists{microtype.sty}{% use microtype if available
  \usepackage[]{microtype}
  \UseMicrotypeSet[protrusion]{basicmath} % disable protrusion for tt fonts
}{}
\makeatletter
\@ifundefined{KOMAClassName}{% if non-KOMA class
  \IfFileExists{parskip.sty}{%
    \usepackage{parskip}
  }{% else
    \setlength{\parindent}{0pt}
    \setlength{\parskip}{6pt plus 2pt minus 1pt}}
}{% if KOMA class
  \KOMAoptions{parskip=half}}
\makeatother
\setlength{\emergencystretch}{3em} % prevent overfull lines
\providecommand{\tightlist}{%
  \setlength{\itemsep}{0pt}\setlength{\parskip}{0pt}}
\usepackage{bookmark}
\IfFileExists{xurl.sty}{\usepackage{xurl}}{} % add URL line breaks if available
\urlstyle{same}
\hypersetup{
  hidelinks,
  pdfcreator={LaTeX via pandoc}}

\author{\texorpdfstring{}{}}
\date{}

\begin{document}

\begin{frame}{Supplemental Analysis}
\protect\phantomsection\label{sec:supplemental_analysis}
This section provides detailed analytical results and theoretical
extensions that complement the main findings presented in Sections
\ref{sec:methodology} and \ref{sec:experimental_results}.

\begin{block}{S3.1 Theoretical Extensions}
\protect\phantomsection\label{s3.1-theoretical-extensions}
\begin{block}{S3.1.1 Extended Discourse Analysis Frameworks}
\protect\phantomsection\label{s3.1.1-extended-discourse-analysis-frameworks}
Building on our mixed-methodology approach, we extend the theoretical
framework for analyzing scientific discourse beyond the six
Ento-Linguistic domains. Our analysis reveals that terminology networks
serve as both descriptive tools and constitutive elements of scientific
knowledge production.

\textbf{Extended Constitutive Framework}:

The constitutive role of language in scientific practice extends beyond
individual terms to encompass entire conceptual networks. We formalize
this through the concept of \textbf{discursive framing networks}:

\begin{equation}\label{eq:discursive_framing}
F(D, T) = \sum_{t \in T} w_t \cdot f_t(D) \cdot c_t
\end{equation}

where \(D\) represents a domain, \(T\) is the terminology set, \(w_t\)
are term weights, \(f_t(D)\) is the framing function for domain \(D\),
and \(c_t\) represents contextual factors.
\end{block}

\begin{block}{S3.1.2 Advanced Ambiguity Classification Systems}
\protect\phantomsection\label{s3.1.2-advanced-ambiguity-classification-systems}
Our ambiguity detection framework extends beyond simple polysemy to
include context-dependent meaning shifts that are characteristic of
scientific terminology evolution:

\textbf{Multi-Level Ambiguity Classification}:

\begin{enumerate}
\item **Lexical Ambiguity**: Multiple dictionary meanings (e.g., "individual" in biological vs. psychological contexts)
\item **Contextual Ambiguity**: Meaning shifts based on research tradition (e.g., "caste" in classical vs. modern entomology)
\item **Scale Ambiguity**: Meaning variations across biological scales (e.g., "behavior" at individual vs. colony levels)
\item **Temporal Ambiguity**: Historical meaning evolution (e.g., "superorganism" from 1970s to present)
\end{enumerate}
\end{block}

\begin{block}{S3.1.3 Cross-Domain Conceptual Mapping}
\protect\phantomsection\label{s3.1.3-cross-domain-conceptual-mapping}
We develop advanced conceptual mapping techniques that reveal
relationships between domains:

\begin{equation}\label{eq:cross_domain_mapping}
M_{ij} = \frac{1}{|T_i \cap T_j|} \sum_{t \in T_i \cap T_j} s(t, D_i, D_j)
\end{equation}

where \(M_{ij}\) is the mapping strength between domains \(D_i\) and
\(D_j\), and \(s(t, D_i, D_j)\) measures semantic similarity of term
\(t\) across domains.
\end{block}
\end{block}

\begin{block}{S3.2 Extended Framing Analysis Methods}
\protect\phantomsection\label{s3.2-extended-framing-analysis-methods}
\begin{block}{S3.2.1 Anthropomorphic Framing Detection}
\protect\phantomsection\label{s3.2.1-anthropomorphic-framing-detection}
Advanced anthropomorphic framing detection incorporates linguistic and
conceptual indicators:

\textbf{Linguistic Indicators}: - Pronominalization (use of ``it''
vs.~``she/he'' for colonies) - Agency attribution (active vs.~passive
voice patterns) - Intentionality markers (words implying purpose or
planning)

\textbf{Conceptual Indicators}: - Social structure projections (human
hierarchies onto insect societies) - Emotional attribution
(anthropomorphic emotional terms) - Cultural bias patterns (Western
social norms in biological descriptions)
\end{block}

\begin{block}{S3.2.2 Hierarchical Framing Analysis}
\protect\phantomsection\label{s3.2.2-hierarchical-framing-analysis}
Extended analysis of hierarchical framing reveals nested levels of
social structure imposition:

\textbf{Macro-Level Hierarchies}: Colony-level social organization
(queen → workers → males)

\textbf{Micro-Level Hierarchies}: Individual-level interactions
(dominant → subordinate nestmates)

\textbf{Inter-Colony Hierarchies}: Population-level relationships
(territorial dominance, resource competition)
\end{block}
\end{block}

\begin{block}{S3.3 Advanced Network Analysis Techniques}
\protect\phantomsection\label{s3.3-advanced-network-analysis-techniques}
\begin{block}{S3.3.1 Temporal Network Evolution}
\protect\phantomsection\label{s3.3.1-temporal-network-evolution}
Analysis of how terminology networks evolve over time reveals conceptual
shifts:

\begin{equation}\label{eq:temporal_network_evolution}
\Delta G(t) = G(t+1) - G(t) = \sum_{e \in E} \delta_e(t) + \sum_{v \in V} \delta_v(t)
\end{equation}

where \(\delta_e(t)\) and \(\delta_v(t)\) represent edge and vertex
changes over time periods.

\textbf{Key Evolutionary Patterns}: - \textbf{Network Growth}: Addition
of new terms and relationships - \textbf{Structural Rearrangements}:
Changes in network topology - \textbf{Conceptual Consolidation}:
Strengthening of established relationships - \textbf{Paradigm Shifts}:
Major restructuring events
\end{block}

\begin{block}{S3.3.2 Multi-Scale Network Analysis}
\protect\phantomsection\label{s3.3.2-multi-scale-network-analysis}
Extending network analysis to multiple scales reveals hierarchical
organization:

\textbf{Local Scale}: Individual term relationships within domains
\textbf{Domain Scale}: Inter-term relationships within domains
\textbf{Cross-Domain Scale}: Relationships between domains \textbf{Field
Scale}: Relationships across the entire entomological terminology
network
\end{block}
\end{block}

\begin{block}{S3.4 Extended Validation Frameworks}
\protect\phantomsection\label{s3.4-extended-validation-frameworks}
\begin{block}{S3.4.1 Inter-Subjectivity Validation}
\protect\phantomsection\label{s3.4.1-inter-subjectivity-validation}
Advanced validation incorporates multiple perspectives:

\textbf{Expert Validation}: Entomological domain experts review
classifications \textbf{Peer Validation}: Interdisciplinary researchers
assess cross-domain mappings \textbf{Historical Validation}: Analysis of
terminology evolution against known conceptual shifts
\textbf{Cross-Cultural Validation}: Comparison with non-English
entomological literature
\end{block}

\begin{block}{S3.4.2 Robustness Testing}
\protect\phantomsection\label{s3.4.2-robustness-testing}
Comprehensive robustness analysis ensures result stability:

\textbf{Subsampling Stability}: Performance across different corpus
subsets \textbf{Parameter Sensitivity}: Robustness to algorithmic
parameter variations \textbf{Annotation Consistency}: Agreement across
multiple human annotators \textbf{Temporal Stability}: Consistency
across publication periods
\end{block}
\end{block}

\begin{block}{S3.5 Advanced Case Study Analysis}
\protect\phantomsection\label{s3.5-advanced-case-study-analysis}
\begin{block}{S3.5.1 Caste Terminology Evolution: 1850-2024}
\protect\phantomsection\label{s3.5.1-caste-terminology-evolution-1850-2024}
Ultra-longitudinal analysis reveals century-scale conceptual evolution:

\textbf{Pre-Darwinian Period (1850-1859)}: Essentialist caste categories
based on morphological differences

\textbf{Darwinian Synthesis (1860-1899)}: Evolutionary explanations for
caste differences

\textbf{Genetic Revolution (1900-1949)}: Chromosomal mechanisms
underlying caste determination

\textbf{Molecular Biology Era (1950-1999)}: Gene expression and hormonal
control of caste differentiation

\textbf{Genomic Era (2000-2024)}: Epigenetic and transcriptomic
regulation of caste phenotypes
\end{block}

\begin{block}{S3.5.2 Superorganism Concept Evolution}
\protect\phantomsection\label{s3.5.2-superorganism-concept-evolution}
Detailed analysis of the superorganism concept across seven decades:

\begin{table}[h]
\centering
\begin{tabular}{|l|c|c|c|c|}
\hline
\textbf{Era} & \textbf{Dominant Metaphor} & \textbf{Key Evidence} & \textbf{Critiques} & \textbf{Legacy} \\
\hline
1960s & Organismic & Division of labor analogies & Ignores individual variation & Established field \\
1970s & Cybernetic & Communication networks & Mechanistic reductionism & Systems thinking \\
1980s & Genetic & Kin selection theory & Haplodiploidy focus & Evolutionary framework \\
1990s & Neuroendocrine & Pheromonal control & Colony complexity & Regulatory mechanisms \\
2000s & Epigenetic & DNA methylation & Environmental effects & Developmental plasticity \\
2010s & Microbiome & Symbiont communities & Host-symbiont dynamics & Extended organism concept \\
\hline
\end{tabular}
\caption{Evolution of superorganism concept across research eras}
\label{tab:superorganism_concept_evolution}
\end{table}
\end{block}
\end{block}

\begin{block}{S3.6 Methodological Reflections}
\protect\phantomsection\label{s3.6-methodological-reflections}
\begin{block}{S3.6.1 Mixed-Methodology Integration}
\protect\phantomsection\label{s3.6.1-mixed-methodology-integration}
Our approach successfully integrates qualitative and quantitative
methods:

\textbf{Qualitative Contributions}: - Theoretical framework development
- Conceptual category identification - Historical context analysis -
Cross-domain relationship mapping

\textbf{Quantitative Contributions}: - Statistical pattern
identification - Network structure analysis - Temporal trend
quantification - Validation metric development
\end{block}

\begin{block}{S3.6.2 Limitations and Scope Considerations}
\protect\phantomsection\label{s3.6.2-limitations-and-scope-considerations}
\textbf{Methodological Limitations}: 1. \textbf{Corpus Scope}: Limited
to English-language publications 2. \textbf{Temporal Resolution}:
Decade-level rather than year-level analysis 3. \textbf{Domain
Boundaries}: Some concepts span multiple domains 4. \textbf{Annotation
Scalability}: Human validation limits analysis scope

\textbf{Theoretical Scope}: 1. \textbf{Cultural Bias}: Western
scientific traditions dominate the corpus 2. \textbf{Disciplinary
Boundaries}: Entomological focus may miss broader patterns 3.
\textbf{Historical Context}: Analysis reflects current perspectives on
past work 4. \textbf{Paradigm Dependence}: Results may vary across
research traditions
\end{block}
\end{block}

\begin{block}{S3.7 Future Theoretical Directions}
\protect\phantomsection\label{s3.7-future-theoretical-directions}
\begin{block}{S3.7.1 Advanced Semantic Analysis}
\protect\phantomsection\label{s3.7.1-advanced-semantic-analysis}
Future work will incorporate advanced semantic techniques:

\textbf{Transformer-Based Embeddings}: Contextual language models for
more sophisticated semantic analysis

\textbf{Multilingual Extensions}: Cross-language terminology mapping and
comparison

\textbf{Dynamic Semantic Networks}: Temporal evolution of term meanings
and relationships
\end{block}

\begin{block}{S3.7.2 Extended Conceptual Frameworks}
\protect\phantomsection\label{s3.7.2-extended-conceptual-frameworks}
Theoretical extensions will address broader questions:

\textbf{Constitutive Linguistics}: How scientific language creates
research objects and relationships

\textbf{Interdisciplinary Translation}: Mechanisms for translating
concepts across disciplinary boundaries

\textbf{Knowledge Representation}: Formal ontologies for scientific
terminology networks

\textbf{Cultural Epistemology}: How cultural contexts shape scientific
language and concepts
\end{block}

\begin{block}{S3.7.3 Practical Applications}
\protect\phantomsection\label{s3.7.3-practical-applications}
Extended applications will include:

\textbf{Terminology Standards}: Development of evidence-based guidelines
for scientific communication

\textbf{Educational Interventions}: Training programs for researchers on
terminological awareness

\textbf{Peer Review Tools}: Automated assistance for evaluating
terminological clarity

\textbf{Cross-Disciplinary Bridges}: Tools for facilitating
interdisciplinary communication

This extended analytical framework provides comprehensive theoretical
and methodological foundations for understanding the constitutive role
of language in scientific practice, with particular focus on the complex
interplay between speech and thought in entomological research.
\end{block}
\end{block}
\end{frame}

\end{document}
