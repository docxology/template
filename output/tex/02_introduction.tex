% Options for packages loaded elsewhere
\PassOptionsToPackage{unicode}{hyperref}
\PassOptionsToPackage{hyphens}{url}
\PassOptionsToPackage{dvipsnames,svgnames,x11names}{xcolor}
%
\documentclass[
  11pt,
]{article}
\usepackage{amsmath,amssymb}
\usepackage{lmodern}
\usepackage{setspace}
\usepackage{iftex}
\ifPDFTeX
  \usepackage[T1]{fontenc}
  \usepackage[utf8]{inputenc}
  \usepackage{textcomp} % provide euro and other symbols
\else % if luatex or xetex
  \usepackage{unicode-math}
  \defaultfontfeatures{Scale=MatchLowercase}
  \defaultfontfeatures[\rmfamily]{Ligatures=TeX,Scale=1}
  \setmainfont[]{Times New Roman}
  \setmonofont[]{Courier New}
\fi
% Use upquote if available, for straight quotes in verbatim environments
\IfFileExists{upquote.sty}{\usepackage{upquote}}{}
\IfFileExists{microtype.sty}{% use microtype if available
  \usepackage[]{microtype}
  \UseMicrotypeSet[protrusion]{basicmath} % disable protrusion for tt fonts
}{}
\makeatletter
\@ifundefined{KOMAClassName}{% if non-KOMA class
  \IfFileExists{parskip.sty}{%
    \usepackage{parskip}
  }{% else
    \setlength{\parindent}{0pt}
    \setlength{\parskip}{6pt plus 2pt minus 1pt}}
}{% if KOMA class
  \KOMAoptions{parskip=half}}
\makeatother
\usepackage{xcolor}
\IfFileExists{xurl.sty}{\usepackage{xurl}}{} % add URL line breaks if available
\IfFileExists{bookmark.sty}{\usepackage{bookmark}}{\usepackage{hyperref}}
\hypersetup{
  colorlinks=true,
  linkcolor={blue},
  filecolor={blue},
  citecolor={blue},
  urlcolor={blue},
  pdfcreator={LaTeX via pandoc}}
\urlstyle{same} % disable monospaced font for URLs
\usepackage[margin=1.5cm,top=1.5cm,bottom=1.5cm,left=1.5cm,right=1.5cm,includeheadfoot]{geometry}
\setlength{\emergencystretch}{3em} % prevent overfull lines
\providecommand{\tightlist}{%
  \setlength{\itemsep}{0pt}\setlength{\parskip}{0pt}}
\setcounter{secnumdepth}{3}
% Essential packages for academic documents
\usepackage{amsmath,amssymb}          % Mathematical symbols and environments
\usepackage{amsfonts}                 % Additional math fonts
\usepackage{amsthm}                   % Theorem environments
\usepackage{graphicx}                 % Include graphics
\usepackage{float}                    % Better float placement
\usepackage{booktabs}                 % Professional tables
\usepackage{longtable}                % Long tables spanning pages
\usepackage{array}                    % Advanced table formatting
\usepackage{multirow}                 % Multi-row table cells
\usepackage{caption}                  % Enhanced caption formatting
\usepackage{subcaption}               % Sub-figures and sub-tables
\usepackage{bm}                       % Bold math symbols
\usepackage{url}                      % URL formatting
\usepackage{hyperref}                 % Hyperlinks and cross-references
\usepackage{cleveref}                 % Intelligent cross-referencing
\usepackage[capitalise]{cleveref}     % Capitalize cross-reference labels
\usepackage{natbib}                   % Bibliography support
\usepackage{doi}                      % DOI links

% Configure figure numbering and captions
\renewcommand{\figurename}{Figure}
\captionsetup{
    justification=centering,
    font=small,
    labelfont=bf,
    labelsep=period
}

% Configure table numbering and captions
\renewcommand{\tablename}{Table}
\captionsetup[table]{
    justification=centering,
    font=small,
    labelfont=bf,
    labelsep=period
}

% Configure section numbering
\setcounter{secnumdepth}{3}
\renewcommand{\thesection}{\arabic{section}}
\renewcommand{\thesubsection}{\arabic{section}.\arabic{subsection}}
\renewcommand{\thesubsubsection}{\arabic{section}.\arabic{subsection}.\arabic{subsubsection}}

% Configure equation numbering
\numberwithin{equation}{section}

% Configure hyperref for proper linking
\hypersetup{
    colorlinks=true,
    linkcolor=blue,
    citecolor=blue,
    urlcolor=blue,
    filecolor=blue,
    pdfborder={0 0 0},
    bookmarks=true,
    bookmarksnumbered=true,
    bookmarkstype=toc,
    pdftitle={Research Project Template},
    pdfauthor={Template Author},
    pdfsubject={Academic Research},
    pdfkeywords={research, template, academic, LaTeX},
    pdfcreator={render_pdf.sh},
    pdfproducer={XeLaTeX}
}

% Configure cleveref for intelligent cross-references
\crefname{section}{Section}{Sections}
\crefname{subsection}{Subsection}{Subsections}
\crefname{subsubsection}{Subsubsection}{Subsubsections}
\crefname{equation}{Equation}{Equations}
\crefname{figure}{Figure}{Figures}
\crefname{table}{Table}{Tables}
\crefname{appendix}{Appendix}{Appendices}

% Configure fonts for Unicode support with fallbacks
\usepackage{newunicodechar}
\newunicodechar{⁴}{\textsuperscript{4}}
\newunicodechar{₄}{\textsubscript{4}}
\newunicodechar{²}{\textsuperscript{2}}
\newunicodechar{₀}{\textsubscript{0}}
\newunicodechar{₁}{\textsubscript{1}}
\newunicodechar{₂}{\textsubscript{2}}
\newunicodechar{₃}{\textsubscript{3}}

% Use standard fonts for better compatibility
\usepackage{lmodern}
\usepackage[T1]{fontenc}

% Enhanced code block styling for better contrast and readability
\usepackage{fancyvrb}
\usepackage{xcolor}
\usepackage{listings}

% Define custom colors for code blocks
\definecolor{codebg}{RGB}{248, 248, 248}      % Very light gray background
\definecolor{codeborder}{RGB}{200, 200, 200}  % Medium gray border
\definecolor{codefg}{RGB}{34, 34, 34}         % Dark gray text
\definecolor{commentcolor}{RGB}{102, 102, 102} % Comment color
\definecolor{keywordcolor}{RGB}{0, 0, 0}       % Keyword color
\definecolor{stringcolor}{RGB}{0, 102, 0}      % String color

% Configure Verbatim environment for inline code
\DefineVerbatimEnvironment{Verbatim}{Verbatim}{%
    fontsize=\small,
    frame=single,
    framerule=0.5pt,
    framesep=3pt,
    rulecolor=\color{codeborder},
    bgcolor=\color{codebg},
    fgcolor=\color{codefg}
}

% Configure code block styling
\DefineVerbatimEnvironment{Highlighting}{Verbatim}{%
    fontsize=\footnotesize,
    frame=single,
    framerule=0.5pt,
    framesep=5pt,
    rulecolor=\color{codeborder},
    bgcolor=\color{codebg},
    fgcolor=\color{codefg}
}

% Style inline code with \texttt
\renewcommand{\texttt}[1]{%
    \colorbox{codebg}{\color{codefg}\ttfamily #1}%
}

% Configure listings package for code blocks
\lstset{
    backgroundcolor=\color{codebg},
    basicstyle=\footnotesize\ttfamily\color{codefg},
    breakatwhitespace=false,
    breaklines=true,
    captionpos=b,
    commentstyle=\color{commentcolor},
    deletekeywords={...},
    escapeinside={\%*}{*)},
    extendedchars=true,
    frame=single,
    framerule=0.5pt,
    framesep=5pt,
    keepspaces=true,
    keywordstyle=\color{keywordcolor}\bfseries,
    language=Python,
    morekeywords={*,...},
    numbers=left,
    numbersep=5pt,
    numberstyle=\tiny\color{codefg},
    rulecolor=\color{codeborder},
    showspaces=false,
    showstringspaces=false,
    showtabs=false,
    stepnumber=1,
    stringstyle=\color{stringcolor},
    tabsize=4,
    title=\lstname
}

% Override any Pandoc default lstset configurations
\AtBeginDocument{
    \lstset{
        backgroundcolor=\color{codebg},
        basicstyle=\footnotesize\ttfamily\color{codefg},
        frame=single,
        framerule=0.5pt,
        framesep=5pt,
        rulecolor=\color{codeborder},
        numbers=left,
        numbersep=5pt,
        numberstyle=\tiny\color{codefg}
    }
}

% Configure bibliography
\bibliographystyle{unsrt}  % Unsorted bibliography style
% Bibliography is handled in 07_references.md

% Simple page break support for document structure
% Note: Page breaks are handled in the markdown generation, not here

% Ensure proper spacing and formatting
\frenchspacing  % Single space after periods
\ifLuaTeX
  \usepackage{selnolig}  % disable illegal ligatures
\fi

\title{02 introduction}
\author{ORCID: 0000-0000-0000-1234\ Email: author@example.com\ DOI: 10.5281/zenodo.12345678}
\date{November 13, 2025}

\begin{document}
\maketitle

{
\hypersetup{linkcolor=black}
\setcounter{tocdepth}{3}
\tableofcontents
}
\setstretch{1.2}
\hypertarget{sec:introduction}{%
\section{Introduction}\label{sec:introduction}}

\hypertarget{overview}{%
\subsection{Overview}\label{overview}}

This is an example project that demonstrates the generic repository
structure for tested code, manuscript editing, and PDF rendering. The
work presents a novel optimization framework with comprehensive
theoretical analysis and experimental validation, building upon
foundational optimization theory \cite{boyd2004, nesterov2018} and
recent advances in adaptive methods \cite{kingma2014, duchi2011}.

\hypertarget{project-structure}{%
\subsection{Project Structure}\label{project-structure}}

The project follows a standardized structure:

\begin{itemize}
\tightlist
\item
  \textbf{\texttt{src/}} - Source code with comprehensive test coverage
\item
  \textbf{\texttt{tests/}} - Test files ensuring 100\% coverage
\item
  \textbf{\texttt{scripts/}} - Project-specific scripts for generating
  figures and data
\item
  \textbf{\texttt{markdown/}} - Source markdown files for the manuscript
\item
  \textbf{\texttt{output/}} - Generated outputs (PDFs, figures, data)
\item
  \textbf{\texttt{repo\_utilities/}} - Generic utility scripts for any
  project
\end{itemize}

\hypertarget{key-features}{%
\subsection{Key Features}\label{key-features}}

\hypertarget{test-driven-development}{%
\subsubsection{Test-Driven Development}\label{test-driven-development}}

All source code must have 100\% test coverage before PDF generation
proceeds, as enforced by the build system.

\hypertarget{automated-script-execution}{%
\subsubsection{Automated Script
Execution}\label{automated-script-execution}}

Project-specific scripts in the \texttt{scripts/} directory are
automatically executed to generate figures and data, ensuring
reproducibility.

\hypertarget{markdown-to-pdf-pipeline}{%
\subsubsection{Markdown to PDF
Pipeline}\label{markdown-to-pdf-pipeline}}

Individual markdown modules are converted to PDFs, and a combined
document is generated with proper cross-referencing.

\hypertarget{generic-and-reusable}{%
\subsubsection{Generic and Reusable}\label{generic-and-reusable}}

The utility scripts can be used with any project that follows this
structure, making it easy to adopt for new research projects.

\hypertarget{manuscript-organization}{%
\subsection{Manuscript Organization}\label{manuscript-organization}}

The manuscript is organized into several key sections:

\begin{enumerate}
\def\labelenumi{\arabic{enumi}.}
\tightlist
\item
  \textbf{Abstract} (Section \ref{sec:abstract}): Research overview and
  key contributions
\item
  \textbf{Introduction} (Section \ref{sec:introduction}): Overview and
  project structure
\item
  \textbf{Methodology} (Section \ref{sec:methodology}): Mathematical
  framework and algorithms
\item
  \textbf{Experimental Results} (Section
  \ref{sec:experimental_results}): Performance evaluation and validation
\item
  \textbf{Discussion} (Section \ref{sec:discussion}): Theoretical
  implications and comparisons
\item
  \textbf{Conclusion} (Section \ref{sec:conclusion}): Summary and future
  directions
\item
  \textbf{References} (Section \ref{sec:references}): Bibliography and
  cited works
\end{enumerate}

\hypertarget{example-figure}{%
\subsection{Example Figure}\label{example-figure}}

The following figure was generated by the example script:

\begin{figure}[h]
\centering
\includegraphics[width=0.8\textwidth]{../output/figures/example_figure.png}
\caption{Example project figure showing a mathematical function}
\label{fig:example_figure}
\end{figure}

This demonstrates how figures are automatically integrated into the
manuscript with proper cross-referencing capabilities. The figure shows
a mathematical function that demonstrates the project's capabilities. As
shown in Figure \ref{fig:example_figure}, the system generates
high-quality visualizations that are automatically integrated into the
manuscript.

\hypertarget{data-availability}{%
\subsection{Data Availability}\label{data-availability}}

All generated data is saved alongside figures for reproducibility:

\begin{itemize}
\tightlist
\item
  \textbf{Figures}: PNG format in \texttt{output/figures/}
\item
  \textbf{Data}: NPZ and CSV formats in \texttt{output/data/}
\item
  \textbf{PDFs}: Individual and combined documents in
  \texttt{output/pdf/}
\item
  \textbf{LaTeX}: Source files in \texttt{output/tex/}
\end{itemize}

\hypertarget{usage}{%
\subsection{Usage}\label{usage}}

To generate the complete manuscript:

\begin{verbatim}
# Clean previous outputs
./repo_utilities/clean_output.sh

# Generate everything (tests + scripts + PDFs)
./repo_utilities/render_pdf.sh
\end{verbatim}

The system will automatically: 1. Run all tests with 100\% coverage
requirement 2. Execute project-specific scripts to generate figures and
data 3. Validate markdown references and images 4. Generate individual
and combined PDFs 5. Export LaTeX source files

\hypertarget{customization}{%
\subsection{Customization}\label{customization}}

This template can be customized for any project by:

\begin{enumerate}
\def\labelenumi{\arabic{enumi}.}
\tightlist
\item
  Adding project-specific scripts to \texttt{scripts/}
\item
  Modifying markdown files in \texttt{markdown/}
\item
  Setting environment variables for author information
\item
  Adjusting LaTeX preamble in \texttt{preamble.md}
\item
  Adding new sections with proper cross-references
\end{enumerate}

\hypertarget{cross-referencing-system}{%
\subsection{Cross-Referencing System}\label{cross-referencing-system}}

The manuscript demonstrates comprehensive cross-referencing:

\begin{itemize}
\tightlist
\item
  \textbf{Section References}: Use
  \texttt{\textbackslash{}ref\{sec:section\_name\}} to reference
  sections
\item
  \textbf{Equation References}: Use
  \texttt{\textbackslash{}eqref\{eq:objective\}} to reference equations
  (see Section \ref{sec:methodology})
\item
  \textbf{Figure References}: Use
  \texttt{\textbackslash{}ref\{fig:figure\_name\}} to reference figures
\item
  \textbf{Table References}: Use
  \texttt{\textbackslash{}ref\{tab:table\_name\}} to reference tables
\end{itemize}

All references are automatically numbered and updated when the document
is regenerated. For example, the main objective function
\eqref{eq:objective} is defined in the methodology section.

\end{document}
