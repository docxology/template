% Options for packages loaded elsewhere
\PassOptionsToPackage{unicode}{hyperref}
\PassOptionsToPackage{hyphens}{url}
\PassOptionsToPackage{dvipsnames,svgnames,x11names}{xcolor}
%
\documentclass[
  11pt,
]{article}
\usepackage{amsmath,amssymb}
\usepackage{lmodern}
\usepackage{setspace}
\usepackage{iftex}
\ifPDFTeX
  \usepackage[T1]{fontenc}
  \usepackage[utf8]{inputenc}
  \usepackage{textcomp} % provide euro and other symbols
\else % if luatex or xetex
  \usepackage{unicode-math}
  \defaultfontfeatures{Scale=MatchLowercase}
  \defaultfontfeatures[\rmfamily]{Ligatures=TeX,Scale=1}
  \setmainfont[]{Times New Roman}
  \setmonofont[]{Courier New}
\fi
% Use upquote if available, for straight quotes in verbatim environments
\IfFileExists{upquote.sty}{\usepackage{upquote}}{}
\IfFileExists{microtype.sty}{% use microtype if available
  \usepackage[]{microtype}
  \UseMicrotypeSet[protrusion]{basicmath} % disable protrusion for tt fonts
}{}
\makeatletter
\@ifundefined{KOMAClassName}{% if non-KOMA class
  \IfFileExists{parskip.sty}{%
    \usepackage{parskip}
  }{% else
    \setlength{\parindent}{0pt}
    \setlength{\parskip}{6pt plus 2pt minus 1pt}}
}{% if KOMA class
  \KOMAoptions{parskip=half}}
\makeatother
\usepackage{xcolor}
\IfFileExists{xurl.sty}{\usepackage{xurl}}{} % add URL line breaks if available
\IfFileExists{bookmark.sty}{\usepackage{bookmark}}{\usepackage{hyperref}}
\hypersetup{
  colorlinks=true,
  linkcolor={blue},
  filecolor={blue},
  citecolor={blue},
  urlcolor={blue},
  pdfcreator={LaTeX via pandoc}}
\urlstyle{same} % disable monospaced font for URLs
\usepackage[margin=1.5cm,top=1.5cm,bottom=1.5cm,left=1.5cm,right=1.5cm,includeheadfoot]{geometry}
\setlength{\emergencystretch}{3em} % prevent overfull lines
\providecommand{\tightlist}{%
  \setlength{\itemsep}{0pt}\setlength{\parskip}{0pt}}
\setcounter{secnumdepth}{3}
% Essential packages for academic documents
\usepackage{amsmath,amssymb}          % Mathematical symbols and environments
\usepackage{amsfonts}                 % Additional math fonts
\usepackage{amsthm}                   % Theorem environments
\usepackage{graphicx}                 % Include graphics
\usepackage{float}                    % Better float placement
\usepackage{booktabs}                 % Professional tables
\usepackage{longtable}                % Long tables spanning pages
\usepackage{array}                    % Advanced table formatting
\usepackage{multirow}                 % Multi-row table cells
\usepackage{caption}                  % Enhanced caption formatting
\usepackage{subcaption}               % Sub-figures and sub-tables
\usepackage{bm}                       % Bold math symbols
\usepackage{url}                      % URL formatting
\usepackage{hyperref}                 % Hyperlinks and cross-references
\usepackage{cleveref}                 % Intelligent cross-referencing
\usepackage[capitalise]{cleveref}     % Capitalize cross-reference labels
\usepackage{natbib}                   % Bibliography support
\usepackage{doi}                      % DOI links

% Configure figure numbering and captions
\renewcommand{\figurename}{Figure}
\captionsetup{
    justification=centering,
    font=small,
    labelfont=bf,
    labelsep=period
}

% Configure table numbering and captions
\renewcommand{\tablename}{Table}
\captionsetup[table]{
    justification=centering,
    font=small,
    labelfont=bf,
    labelsep=period
}

% Configure section numbering
\setcounter{secnumdepth}{3}
\renewcommand{\thesection}{\arabic{section}}
\renewcommand{\thesubsection}{\arabic{section}.\arabic{subsection}}
\renewcommand{\thesubsubsection}{\arabic{section}.\arabic{subsection}.\arabic{subsubsection}}

% Configure equation numbering
\numberwithin{equation}{section}

% Configure hyperref for proper linking
\hypersetup{
    colorlinks=true,
    linkcolor=blue,
    citecolor=blue,
    urlcolor=blue,
    filecolor=blue,
    pdfborder={0 0 0},
    bookmarks=true,
    bookmarksnumbered=true,
    bookmarkstype=toc,
    pdftitle={Research Project Template},
    pdfauthor={Template Author},
    pdfsubject={Academic Research},
    pdfkeywords={research, template, academic, LaTeX},
    pdfcreator={render_pdf.sh},
    pdfproducer={XeLaTeX}
}

% Configure cleveref for intelligent cross-references
\crefname{section}{Section}{Sections}
\crefname{subsection}{Subsection}{Subsections}
\crefname{subsubsection}{Subsubsection}{Subsubsections}
\crefname{equation}{Equation}{Equations}
\crefname{figure}{Figure}{Figures}
\crefname{table}{Table}{Tables}
\crefname{appendix}{Appendix}{Appendices}

% Configure fonts for Unicode support with fallbacks
\usepackage{newunicodechar}
\newunicodechar{⁴}{\textsuperscript{4}}
\newunicodechar{₄}{\textsubscript{4}}
\newunicodechar{²}{\textsuperscript{2}}
\newunicodechar{₀}{\textsubscript{0}}
\newunicodechar{₁}{\textsubscript{1}}
\newunicodechar{₂}{\textsubscript{2}}
\newunicodechar{₃}{\textsubscript{3}}

% Use standard fonts for better compatibility
\usepackage{lmodern}
\usepackage[T1]{fontenc}

% Enhanced code block styling for better contrast and readability
\usepackage{fancyvrb}
\usepackage{xcolor}
\usepackage{listings}

% Define custom colors for code blocks
\definecolor{codebg}{RGB}{248, 248, 248}      % Very light gray background
\definecolor{codeborder}{RGB}{200, 200, 200}  % Medium gray border
\definecolor{codefg}{RGB}{34, 34, 34}         % Dark gray text
\definecolor{commentcolor}{RGB}{102, 102, 102} % Comment color
\definecolor{keywordcolor}{RGB}{0, 0, 0}       % Keyword color
\definecolor{stringcolor}{RGB}{0, 102, 0}      % String color

% Configure Verbatim environment for inline code
\DefineVerbatimEnvironment{Verbatim}{Verbatim}{%
    fontsize=\small,
    frame=single,
    framerule=0.5pt,
    framesep=3pt,
    rulecolor=\color{codeborder},
    bgcolor=\color{codebg},
    fgcolor=\color{codefg}
}

% Configure code block styling
\DefineVerbatimEnvironment{Highlighting}{Verbatim}{%
    fontsize=\footnotesize,
    frame=single,
    framerule=0.5pt,
    framesep=5pt,
    rulecolor=\color{codeborder},
    bgcolor=\color{codebg},
    fgcolor=\color{codefg}
}

% Style inline code with \texttt
\renewcommand{\texttt}[1]{%
    \colorbox{codebg}{\color{codefg}\ttfamily #1}%
}

% Configure listings package for code blocks
\lstset{
    backgroundcolor=\color{codebg},
    basicstyle=\footnotesize\ttfamily\color{codefg},
    breakatwhitespace=false,
    breaklines=true,
    captionpos=b,
    commentstyle=\color{commentcolor},
    deletekeywords={...},
    escapeinside={\%*}{*)},
    extendedchars=true,
    frame=single,
    framerule=0.5pt,
    framesep=5pt,
    keepspaces=true,
    keywordstyle=\color{keywordcolor}\bfseries,
    language=Python,
    morekeywords={*,...},
    numbers=left,
    numbersep=5pt,
    numberstyle=\tiny\color{codefg},
    rulecolor=\color{codeborder},
    showspaces=false,
    showstringspaces=false,
    showtabs=false,
    stepnumber=1,
    stringstyle=\color{stringcolor},
    tabsize=4,
    title=\lstname
}

% Override any Pandoc default lstset configurations
\AtBeginDocument{
    \lstset{
        backgroundcolor=\color{codebg},
        basicstyle=\footnotesize\ttfamily\color{codefg},
        frame=single,
        framerule=0.5pt,
        framesep=5pt,
        rulecolor=\color{codeborder},
        numbers=left,
        numbersep=5pt,
        numberstyle=\tiny\color{codefg}
    }
}

% Configure bibliography
\bibliographystyle{unsrt}  % Unsorted bibliography style
% Bibliography is handled in 07_references.md

% Simple page break support for document structure
% Note: Page breaks are handled in the markdown generation, not here

% Ensure proper spacing and formatting
\frenchspacing  % Single space after periods
\ifLuaTeX
  \usepackage{selnolig}  % disable illegal ligatures
\fi

\title{S02 supplemental results}
\author{ORCID: 0000-0000-0000-1234\ Email: author@example.com\ DOI: 10.5281/zenodo.12345678}
\date{November 13, 2025}

\begin{document}
\maketitle

{
\hypersetup{linkcolor=black}
\setcounter{tocdepth}{3}
\tableofcontents
}
\setstretch{1.2}
\hypertarget{sec:supplemental_results}{%
\section{Supplemental Results}\label{sec:supplemental_results}}

This section provides additional experimental results that complement
Section \ref{sec:experimental_results}.

\hypertarget{s2.1-extended-benchmark-results}{%
\subsection{S2.1 Extended Benchmark
Results}\label{s2.1-extended-benchmark-results}}

\hypertarget{s2.1.1-additional-datasets}{%
\subsubsection{S2.1.1 Additional
Datasets}\label{s2.1.1-additional-datasets}}

We evaluated our method on 15 additional benchmark datasets beyond those
reported in Section \ref{sec:experimental_results}:

\begin{table}[h]
\centering
\begin{tabular}{|l|c|c|c|c|}
\hline
\textbf{Dataset} & \textbf{Size} & \textbf{Dimensions} & \textbf{Type} & \textbf{Source} \\
\hline
UCI-1 & 1,000 & 20 & Regression & UCI ML Repository \\
UCI-2 & 5,000 & 50 & Classification & UCI ML Repository \\
UCI-3 & 10,000 & 100 & Multi-class & UCI ML Repository \\
Synthetic-1 & 50,000 & 500 & Convex & Generated \\
Synthetic-2 & 100,000 & 1000 & Non-convex & Generated \\
LibSVM-1 & 20,000 & 150 & Binary & LIBSVM \\
LibSVM-2 & 30,000 & 300 & Multi-class & LIBSVM \\
OpenML-1 & 15,000 & 80 & Regression & OpenML \\
OpenML-2 & 25,000 & 120 & Classification & OpenML \\
Real-world-1 & 8,000 & 40 & Time-series & Industrial \\
Real-world-2 & 12,000 & 60 & Sensor data & Industrial \\
Medical-1 & 3,000 & 25 & Diagnosis & Medical DB \\
Medical-2 & 5,000 & 35 & Prognosis & Medical DB \\
Finance-1 & 10,000 & 50 & Stock prediction & Financial \\
Finance-2 & 15,000 & 75 & Risk assessment & Financial \\
\hline
\end{tabular}
\caption{Additional benchmark datasets used in extended evaluation}
\label{tab:extended_datasets}
\end{table}

\hypertarget{s2.1.2-performance-across-all-datasets}{%
\subsubsection{S2.1.2 Performance Across All
Datasets}\label{s2.1.2-performance-across-all-datasets}}

\begin{table}[h]
\centering
\begin{tabular}{|l|c|c|c|c|}
\hline
\textbf{Method} & \textbf{Avg. Accuracy} & \textbf{Avg. Time (s)} & \textbf{Avg. Iterations} & \textbf{Success Rate} \\
\hline
Our Method & 0.943 & 18.7 & 287 & 96.2\% \\
Gradient Descent & 0.901 & 24.3 & 421 & 85.0\% \\
Adam & 0.915 & 21.2 & 378 & 88.5\% \\
L-BFGS & 0.928 & 22.8 & 245 & 91.3\% \\
RMSProp & 0.908 & 20.5 & 395 & 86.7\% \\
Adagrad & 0.895 & 23.1 & 412 & 83.8\% \\
\hline
\end{tabular}
\caption{Comprehensive performance comparison across all 20 benchmark datasets}
\label{tab:comprehensive_comparison}
\end{table}

\hypertarget{s2.2-convergence-behavior-analysis}{%
\subsection{S2.2 Convergence Behavior
Analysis}\label{s2.2-convergence-behavior-analysis}}

\hypertarget{s2.2.1-problem-specific-convergence-patterns}{%
\subsubsection{S2.2.1 Problem-Specific Convergence
Patterns}\label{s2.2.1-problem-specific-convergence-patterns}}

Different problem types exhibit distinct convergence patterns:

\textbf{Convex Problems}: Exponential convergence as predicted by theory
\eqref{eq:convergence} \cite{nesterov2018, boyd2004}, with empirical
rate matching theoretical bounds within 5\%.

\textbf{Non-Convex Problems}: Initial phase shows rapid descent followed
by slower convergence near local minima. Our adaptive strategy maintains
stability throughout.

\textbf{High-Dimensional Problems}: Memory-efficient implementation
enables scaling to \(n > 10^6\) dimensions with linear memory growth.

\hypertarget{s2.2.2-iteration-wise-progress}{%
\subsubsection{S2.2.2 Iteration-wise
Progress}\label{s2.2.2-iteration-wise-progress}}

\begin{table}[h]
\centering
\begin{tabular}{|l|c|c|c|c|c|}
\hline
\textbf{Iteration} & \textbf{Objective Value} & \textbf{Gradient Norm} & \textbf{Step Size} & \textbf{Momentum} & \textbf{Time (s)} \\
\hline
1 & 125.3 & 18.7 & 0.0100 & 0.000 & 0.12 \\
10 & 42.1 & 8.3 & 0.0095 & 0.900 & 1.18 \\
50 & 8.7 & 2.1 & 0.0082 & 0.900 & 5.92 \\
100 & 2.3 & 0.6 & 0.0071 & 0.900 & 11.84 \\
200 & 0.4 & 0.1 & 0.0058 & 0.900 & 23.67 \\
287 & 0.0012 & 0.00005 & 0.0045 & 0.900 & 33.95 \\
\hline
\end{tabular}
\caption{Typical iteration-wise progress on medium-scale problem}
\label{tab:iteration_progress}
\end{table}

\hypertarget{s2.3-scalability-analysis}{%
\subsection{S2.3 Scalability Analysis}\label{s2.3-scalability-analysis}}

\hypertarget{s2.3.1-performance-vs.-problem-size}{%
\subsubsection{S2.3.1 Performance vs.~Problem
Size}\label{s2.3.1-performance-vs.-problem-size}}

\begin{table}[h]
\centering
\begin{tabular}{|c|c|c|c|c|}
\hline
\textbf{Problem Size ($n$)} & \textbf{Time (s)} & \textbf{Memory (MB)} & \textbf{Iterations} & \textbf{Scaling} \\
\hline
$10^2$ & 0.08 & 2.3 & 145 & $O(n)$ \\
$10^3$ & 0.82 & 23.1 & 198 & $O(n \log n)$ \\
$10^4$ & 9.45 & 231.5 & 247 & $O(n \log n)$ \\
$10^5$ & 118.7 & 2315.2 & 298 & $O(n \log n)$ \\
$10^6$ & 1523.4 & 23152.8 & 356 & $O(n \log n)$ \\
\hline
\end{tabular}
\caption{Scalability analysis confirming theoretical complexity bounds}
\label{tab:scalability_detailed}
\end{table}

The empirical scaling confirms our theoretical \(O(n \log n)\)
per-iteration complexity from Section \ref{sec:methodology}.

\hypertarget{s2.4-robustness-analysis}{%
\subsection{S2.4 Robustness Analysis}\label{s2.4-robustness-analysis}}

\hypertarget{s2.4.1-performance-under-noise}{%
\subsubsection{S2.4.1 Performance Under
Noise}\label{s2.4.1-performance-under-noise}}

We evaluated robustness under various noise conditions:

\begin{table}[h]
\centering
\begin{tabular}{|l|c|c|c|}
\hline
\textbf{Noise Type} & \textbf{Noise Level} & \textbf{Success Rate} & \textbf{Avg. Degradation} \\
\hline
Gaussian & $\sigma = 0.01$ & 95.8\% & 2.3\% \\
Gaussian & $\sigma = 0.05$ & 93.2\% & 6.7\% \\
Gaussian & $\sigma = 0.10$ & 89.5\% & 12.4\% \\
Uniform & $U(-0.05, 0.05)$ & 94.1\% & 5.2\% \\
Salt-and-Pepper & $p = 0.05$ & 92.7\% & 7.8\% \\
Outliers & 5\% corrupted & 91.3\% & 8.9\% \\
\hline
\end{tabular}
\caption{Robustness under different noise conditions}
\label{tab:robustness_noise}
\end{table}

\hypertarget{s2.4.2-initialization-sensitivity}{%
\subsubsection{S2.4.2 Initialization
Sensitivity}\label{s2.4.2-initialization-sensitivity}}

Algorithm performance across 1000 random initializations:

\begin{itemize}
\tightlist
\item
  \textbf{Mean convergence time}: 18.7 ± 3.2 seconds
\item
  \textbf{Median iterations}: 287 (IQR: 265-312)
\item
  \textbf{Success rate}: 96.2\% (38 failures out of 1000 runs)
\item
  \textbf{Final error}: \((1.2 ± 0.3) \times 10^{-6}\)
\end{itemize}

The low variance confirms robustness to initialization.

\hypertarget{s2.5-comparison-with-domain-specific-methods}{%
\subsection{S2.5 Comparison with Domain-Specific
Methods}\label{s2.5-comparison-with-domain-specific-methods}}

\hypertarget{s2.5.1-machine-learning-applications}{%
\subsubsection{S2.5.1 Machine Learning
Applications}\label{s2.5.1-machine-learning-applications}}

\begin{table}[h]
\centering
\begin{tabular}{|l|c|c|c|}
\hline
\textbf{Method} & \textbf{Training Accuracy} & \textbf{Test Accuracy} & \textbf{Training Time (s)} \\
\hline
Our Method & 0.987 & 0.942 & 245 \\
SGD & 0.975 & 0.935 & 312 \\
Adam & 0.982 & 0.938 & 278 \\
RMSProp & 0.978 & 0.936 & 295 \\
AdamW & 0.983 & 0.940 & 283 \\
\hline
\end{tabular}
\caption{Performance on neural network training tasks}
\label{tab:ml_applications}
\end{table}

\hypertarget{s2.5.2-signal-processing-applications}{%
\subsubsection{S2.5.2 Signal Processing
Applications}\label{s2.5.2-signal-processing-applications}}

For sparse signal reconstruction problems, our method outperforms
specialized algorithms:

\begin{itemize}
\tightlist
\item
  \textbf{Recovery rate}: 98.7\% vs.~94.2\% (ISTA) and 96.5\% (FISTA)
\item
  \textbf{Computation time}: 45\% faster than iterative thresholding
  methods
\item
  \textbf{Memory usage}: 60\% lower than quasi-Newton methods
\end{itemize}

\hypertarget{s2.6-ablation-study-details}{%
\subsection{S2.6 Ablation Study
Details}\label{s2.6-ablation-study-details}}

\hypertarget{s2.6.1-component-contribution-analysis}{%
\subsubsection{S2.6.1 Component Contribution
Analysis}\label{s2.6.1-component-contribution-analysis}}

\begin{table}[h]
\centering
\begin{tabular}{|l|c|c|c|}
\hline
\textbf{Configuration} & \textbf{Convergence Rate} & \textbf{Iterations} & \textbf{Success Rate} \\
\hline
Full method & 0.85 & 287 & 96.2\% \\
No momentum & 0.91 & 412 & 91.5\% \\
No adaptive step & 0.89 & 385 & 89.8\% \\
No regularization & 0.87 & 325 & 88.3\% \\
Fixed step size & 0.93 & 478 & 85.7\% \\
\hline
\end{tabular}
\caption{Detailed ablation study showing contribution of each component}
\label{tab:ablation_detailed}
\end{table}

Each component contributes significantly to overall performance, with
momentum providing the largest individual benefit.

\hypertarget{s2.7-real-world-case-studies}{%
\subsection{S2.7 Real-World Case
Studies}\label{s2.7-real-world-case-studies}}

\hypertarget{s2.7.1-industrial-application-manufacturing-optimization}{%
\subsubsection{S2.7.1 Industrial Application: Manufacturing
Optimization}\label{s2.7.1-industrial-application-manufacturing-optimization}}

Applied to production line optimization: - \textbf{Problem size}: 50,000
parameters - \textbf{Constraints}: 2,500 inequality constraints -
\textbf{Solution time}: 3.2 hours vs.~8.5 hours (baseline) -
\textbf{Cost reduction}: 12.3\% improvement in operational efficiency

\hypertarget{s2.7.2-scientific-application-climate-modeling}{%
\subsubsection{S2.7.2 Scientific Application: Climate
Modeling}\label{s2.7.2-scientific-application-climate-modeling}}

Applied to parameter estimation in climate models: - \textbf{Model
complexity}: 1,000,000+ parameters - \textbf{Computational savings}:
65\% reduction in simulation time - \textbf{Accuracy}: Matches or
exceeds traditional methods - \textbf{Scalability}: Enables ensemble
runs previously infeasible

These real-world applications demonstrate the practical value and
scalability of our approach beyond academic benchmarks.

\end{document}
