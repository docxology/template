% Options for packages loaded elsewhere
\PassOptionsToPackage{unicode}{hyperref}
\PassOptionsToPackage{hyphens}{url}
\PassOptionsToPackage{dvipsnames,svgnames,x11names}{xcolor}
%
\documentclass[
  11pt,
]{article}
\usepackage{amsmath,amssymb}
\usepackage{lmodern}
\usepackage{setspace}
\usepackage{iftex}
\ifPDFTeX
  \usepackage[T1]{fontenc}
  \usepackage[utf8]{inputenc}
  \usepackage{textcomp} % provide euro and other symbols
\else % if luatex or xetex
  \usepackage{unicode-math}
  \defaultfontfeatures{Scale=MatchLowercase}
  \defaultfontfeatures[\rmfamily]{Ligatures=TeX,Scale=1}
  \setmainfont[]{Times New Roman}
  \setmonofont[]{Courier New}
\fi
% Use upquote if available, for straight quotes in verbatim environments
\IfFileExists{upquote.sty}{\usepackage{upquote}}{}
\IfFileExists{microtype.sty}{% use microtype if available
  \usepackage[]{microtype}
  \UseMicrotypeSet[protrusion]{basicmath} % disable protrusion for tt fonts
}{}
\makeatletter
\@ifundefined{KOMAClassName}{% if non-KOMA class
  \IfFileExists{parskip.sty}{%
    \usepackage{parskip}
  }{% else
    \setlength{\parindent}{0pt}
    \setlength{\parskip}{6pt plus 2pt minus 1pt}}
}{% if KOMA class
  \KOMAoptions{parskip=half}}
\makeatother
\usepackage{xcolor}
\IfFileExists{xurl.sty}{\usepackage{xurl}}{} % add URL line breaks if available
\IfFileExists{bookmark.sty}{\usepackage{bookmark}}{\usepackage{hyperref}}
\hypersetup{
  colorlinks=true,
  linkcolor={blue},
  filecolor={blue},
  citecolor={blue},
  urlcolor={blue},
  pdfcreator={LaTeX via pandoc}}
\urlstyle{same} % disable monospaced font for URLs
\usepackage[margin=1.5cm,top=1.5cm,bottom=1.5cm,left=1.5cm,right=1.5cm,includeheadfoot]{geometry}
\setlength{\emergencystretch}{3em} % prevent overfull lines
\providecommand{\tightlist}{%
  \setlength{\itemsep}{0pt}\setlength{\parskip}{0pt}}
\setcounter{secnumdepth}{3}
% Essential packages for academic documents
\usepackage{amsmath,amssymb}          % Mathematical symbols and environments
\usepackage{amsfonts}                 % Additional math fonts
\usepackage{amsthm}                   % Theorem environments
\usepackage{graphicx}                 % Include graphics
\usepackage{float}                    % Better float placement
\usepackage{booktabs}                 % Professional tables
\usepackage{longtable}                % Long tables spanning pages
\usepackage{array}                    % Advanced table formatting
\usepackage{multirow}                 % Multi-row table cells
\usepackage{caption}                  % Enhanced caption formatting
\usepackage{subcaption}               % Sub-figures and sub-tables
\usepackage{bm}                       % Bold math symbols
\usepackage{url}                      % URL formatting
\usepackage{hyperref}                 % Hyperlinks and cross-references
\usepackage{cleveref}                 % Intelligent cross-referencing
\usepackage[capitalise]{cleveref}     % Capitalize cross-reference labels
\usepackage{natbib}                   % Bibliography support
\usepackage{doi}                      % DOI links

% Configure figure numbering and captions
\renewcommand{\figurename}{Figure}
\captionsetup{
    justification=centering,
    font=small,
    labelfont=bf,
    labelsep=period
}

% Configure table numbering and captions
\renewcommand{\tablename}{Table}
\captionsetup[table]{
    justification=centering,
    font=small,
    labelfont=bf,
    labelsep=period
}

% Configure section numbering
\setcounter{secnumdepth}{3}
\renewcommand{\thesection}{\arabic{section}}
\renewcommand{\thesubsection}{\arabic{section}.\arabic{subsection}}
\renewcommand{\thesubsubsection}{\arabic{section}.\arabic{subsection}.\arabic{subsubsection}}

% Configure equation numbering
\numberwithin{equation}{section}

% Configure hyperref for proper linking
\hypersetup{
    colorlinks=true,
    linkcolor=blue,
    citecolor=blue,
    urlcolor=blue,
    filecolor=blue,
    pdfborder={0 0 0},
    bookmarks=true,
    bookmarksnumbered=true,
    bookmarkstype=toc,
    pdftitle={Research Project Template},
    pdfauthor={Template Author},
    pdfsubject={Academic Research},
    pdfkeywords={research, template, academic, LaTeX},
    pdfcreator={render_pdf.sh},
    pdfproducer={XeLaTeX}
}

% Configure cleveref for intelligent cross-references
\crefname{section}{Section}{Sections}
\crefname{subsection}{Subsection}{Subsections}
\crefname{subsubsection}{Subsubsection}{Subsubsections}
\crefname{equation}{Equation}{Equations}
\crefname{figure}{Figure}{Figures}
\crefname{table}{Table}{Tables}
\crefname{appendix}{Appendix}{Appendices}

% Configure fonts for Unicode support with fallbacks
\usepackage{newunicodechar}
\newunicodechar{⁴}{\textsuperscript{4}}
\newunicodechar{₄}{\textsubscript{4}}
\newunicodechar{²}{\textsuperscript{2}}
\newunicodechar{₀}{\textsubscript{0}}
\newunicodechar{₁}{\textsubscript{1}}
\newunicodechar{₂}{\textsubscript{2}}
\newunicodechar{₃}{\textsubscript{3}}

% Use standard fonts for better compatibility
\usepackage{lmodern}
\usepackage[T1]{fontenc}

% Enhanced code block styling for better contrast and readability
\usepackage{fancyvrb}
\usepackage{xcolor}
\usepackage{listings}

% Define custom colors for code blocks
\definecolor{codebg}{RGB}{248, 248, 248}      % Very light gray background
\definecolor{codeborder}{RGB}{200, 200, 200}  % Medium gray border
\definecolor{codefg}{RGB}{34, 34, 34}         % Dark gray text
\definecolor{commentcolor}{RGB}{102, 102, 102} % Comment color
\definecolor{keywordcolor}{RGB}{0, 0, 0}       % Keyword color
\definecolor{stringcolor}{RGB}{0, 102, 0}      % String color

% Configure Verbatim environment for inline code
\DefineVerbatimEnvironment{Verbatim}{Verbatim}{%
    fontsize=\small,
    frame=single,
    framerule=0.5pt,
    framesep=3pt,
    rulecolor=\color{codeborder},
    bgcolor=\color{codebg},
    fgcolor=\color{codefg}
}

% Configure code block styling
\DefineVerbatimEnvironment{Highlighting}{Verbatim}{%
    fontsize=\footnotesize,
    frame=single,
    framerule=0.5pt,
    framesep=5pt,
    rulecolor=\color{codeborder},
    bgcolor=\color{codebg},
    fgcolor=\color{codefg}
}

% Style inline code with \texttt
\renewcommand{\texttt}[1]{%
    \colorbox{codebg}{\color{codefg}\ttfamily #1}%
}

% Configure listings package for code blocks
\lstset{
    backgroundcolor=\color{codebg},
    basicstyle=\footnotesize\ttfamily\color{codefg},
    breakatwhitespace=false,
    breaklines=true,
    captionpos=b,
    commentstyle=\color{commentcolor},
    deletekeywords={...},
    escapeinside={\%*}{*)},
    extendedchars=true,
    frame=single,
    framerule=0.5pt,
    framesep=5pt,
    keepspaces=true,
    keywordstyle=\color{keywordcolor}\bfseries,
    language=Python,
    morekeywords={*,...},
    numbers=left,
    numbersep=5pt,
    numberstyle=\tiny\color{codefg},
    rulecolor=\color{codeborder},
    showspaces=false,
    showstringspaces=false,
    showtabs=false,
    stepnumber=1,
    stringstyle=\color{stringcolor},
    tabsize=4,
    title=\lstname
}

% Override any Pandoc default lstset configurations
\AtBeginDocument{
    \lstset{
        backgroundcolor=\color{codebg},
        basicstyle=\footnotesize\ttfamily\color{codefg},
        frame=single,
        framerule=0.5pt,
        framesep=5pt,
        rulecolor=\color{codeborder},
        numbers=left,
        numbersep=5pt,
        numberstyle=\tiny\color{codefg}
    }
}

% Configure bibliography
\bibliographystyle{unsrt}  % Unsorted bibliography style
% Bibliography is handled in 07_references.md

% Simple page break support for document structure
% Note: Page breaks are handled in the markdown generation, not here

% Ensure proper spacing and formatting
\frenchspacing  % Single space after periods
\ifLuaTeX
  \usepackage{selnolig}  % disable illegal ligatures
\fi

\title{03 methodology}
\author{ORCID: 0000-0000-0000-0000\\ Email: author@example.com}
\date{November 13, 2025}

\begin{document}
\maketitle

{
\hypersetup{linkcolor=black}
\setcounter{tocdepth}{3}
\tableofcontents
}
\setstretch{1.2}
\hypertarget{sec:methodology}{%
\section{Methodology}\label{sec:methodology}}

\hypertarget{mathematical-framework}{%
\subsection{Mathematical Framework}\label{mathematical-framework}}

Our approach is based on a novel optimization framework that combines
multiple mathematical techniques, extending classical convex
optimization methods \cite{boyd2004, nesterov2018} with modern adaptive
strategies \cite{kingma2014, duchi2011}. The core algorithm can be
expressed as follows:

\begin{equation}\label{eq:objective}
f(x) = \sum_{i=1}^{n} w_i \phi_i(x) + \lambda R(x)
\end{equation}

where \(x \in \mathbb{R}^d\) is the optimization variable, \(w_i\) are
learned weights, \(\phi_i\) are basis functions, and \(R(x)\) is a
regularization term with strength \(\lambda\).

The optimization problem we solve is:

\begin{equation}\label{eq:optimization}
\min_{x \in \mathcal{X}} f(x) \quad \text{subject to} \quad g_i(x) \leq 0, \quad i = 1, \ldots, m
\end{equation}

where \(\mathcal{X}\) is the feasible set and \(g_i(x)\) are constraint
functions.

\hypertarget{algorithm-description}{%
\subsection{Algorithm Description}\label{algorithm-description}}

Our iterative algorithm updates the solution according to:

\begin{equation}\label{eq:update}
x_{k+1} = x_k - \alpha_k \nabla f(x_k) + \beta_k (x_k - x_{k-1})
\end{equation}

where \(\alpha_k\) is the learning rate and \(\beta_k\) is the momentum
coefficient. The convergence rate is characterized by:

\begin{equation}\label{eq:convergence}
\|x_k - x^*\| \leq C \rho^k
\end{equation}

where \(x^*\) is the optimal solution, \(C > 0\) is a constant, and
\(\rho \in (0,1)\) is the convergence rate.

\hypertarget{implementation-details}{%
\subsection{Implementation Details}\label{implementation-details}}

The algorithm implementation follows the pseudocode shown in Figure
\ref{fig:experimental_setup}. The key insight is that we can decompose
the objective function \eqref{eq:objective} into separable components,
allowing for efficient parallel computation. This approach builds upon
proximal optimization techniques \cite{beck2009, parikh2014} and recent
advances in large-scale optimization \cite{schmidt2017, wright2010}.

\begin{figure}[h]
\centering
\includegraphics[width=0.9\textwidth]{../output/figures/experimental_setup.png}
\caption{Experimental pipeline showing the complete workflow}
\label{fig:experimental_setup}
\end{figure}

For numerical stability, we use the following adaptive step size rule:

\begin{equation}\label{eq:adaptive_step}
\alpha_k = \frac{\alpha_0}{\sqrt{1 + \sum_{i=1}^{k} \|\nabla f(x_i)\|^2}}
\end{equation}

This ensures that the algorithm converges even when the gradient varies
significantly across iterations.

\hypertarget{performance-analysis}{%
\subsection{Performance Analysis}\label{performance-analysis}}

The computational complexity of our approach is \(O(n \log n)\) per
iteration, where \(n\) is the problem dimension. This is achieved
through the efficient data structures shown in Figure
\ref{fig:data_structure}.

\begin{figure}[h]
\centering
\includegraphics[width=0.9\textwidth]{../output/figures/data_structure.png}
\caption{Efficient data structures used in our implementation}
\label{fig:data_structure}
\end{figure}

The memory requirements scale as:

\begin{equation}\label{eq:memory}
M(n) = O(n) + O(\log n) \cdot \text{number of iterations}
\end{equation}

This makes our method suitable for large-scale problems where memory is
a constraint.

\hypertarget{validation-framework}{%
\subsection{Validation Framework}\label{validation-framework}}

To validate our theoretical results, we use the experimental setup
illustrated in Figure \ref{fig:experimental_setup}. The performance
metrics are computed using:

\begin{equation}\label{eq:accuracy}
\text{Accuracy} = \frac{1}{N} \sum_{i=1}^{N} \mathbb{I}[f(x_i) \leq f(x^*) + \epsilon]
\end{equation}

where \(\mathbb{I}[\cdot]\) is the indicator function and \(\epsilon\)
is the tolerance threshold.

The convergence analysis results are summarized in Figure
\ref{fig:convergence_plot}, which shows the empirical convergence rates
compared to the theoretical bound \eqref{eq:convergence}.

\end{document}
