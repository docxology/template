% Options for packages loaded elsewhere
\PassOptionsToPackage{unicode}{hyperref}
\PassOptionsToPackage{hyphens}{url}
\PassOptionsToPackage{dvipsnames,svgnames,x11names}{xcolor}
%
\documentclass[
  11pt,
]{article}
\usepackage{amsmath,amssymb}
\usepackage{lmodern}
\usepackage{setspace}
\usepackage{iftex}
\ifPDFTeX
  \usepackage[T1]{fontenc}
  \usepackage[utf8]{inputenc}
  \usepackage{textcomp} % provide euro and other symbols
\else % if luatex or xetex
  \usepackage{unicode-math}
  \defaultfontfeatures{Scale=MatchLowercase}
  \defaultfontfeatures[\rmfamily]{Ligatures=TeX,Scale=1}
  \setmainfont[]{Times New Roman}
  \setmonofont[]{Courier New}
\fi
% Use upquote if available, for straight quotes in verbatim environments
\IfFileExists{upquote.sty}{\usepackage{upquote}}{}
\IfFileExists{microtype.sty}{% use microtype if available
  \usepackage[]{microtype}
  \UseMicrotypeSet[protrusion]{basicmath} % disable protrusion for tt fonts
}{}
\makeatletter
\@ifundefined{KOMAClassName}{% if non-KOMA class
  \IfFileExists{parskip.sty}{%
    \usepackage{parskip}
  }{% else
    \setlength{\parindent}{0pt}
    \setlength{\parskip}{6pt plus 2pt minus 1pt}}
}{% if KOMA class
  \KOMAoptions{parskip=half}}
\makeatother
\usepackage{xcolor}
\IfFileExists{xurl.sty}{\usepackage{xurl}}{} % add URL line breaks if available
\IfFileExists{bookmark.sty}{\usepackage{bookmark}}{\usepackage{hyperref}}
\hypersetup{
  colorlinks=true,
  linkcolor={blue},
  filecolor={blue},
  citecolor={blue},
  urlcolor={blue},
  pdfcreator={LaTeX via pandoc}}
\urlstyle{same} % disable monospaced font for URLs
\usepackage[margin=1.5cm,top=1.5cm,bottom=1.5cm,left=1.5cm,right=1.5cm,includeheadfoot]{geometry}
\setlength{\emergencystretch}{3em} % prevent overfull lines
\providecommand{\tightlist}{%
  \setlength{\itemsep}{0pt}\setlength{\parskip}{0pt}}
\setcounter{secnumdepth}{3}
% Essential packages for academic documents
\usepackage{amsmath,amssymb}          % Mathematical symbols and environments
\usepackage{amsfonts}                 % Additional math fonts
\usepackage{amsthm}                   % Theorem environments
\usepackage{graphicx}                 % Include graphics
\usepackage{float}                    % Better float placement
\usepackage{booktabs}                 % Professional tables
\usepackage{longtable}                % Long tables spanning pages
\usepackage{array}                    % Advanced table formatting
\usepackage{multirow}                 % Multi-row table cells
\usepackage{caption}                  % Enhanced caption formatting
\usepackage{subcaption}               % Sub-figures and sub-tables
\usepackage{bm}                       % Bold math symbols
\usepackage{url}                      % URL formatting
\usepackage{hyperref}                 % Hyperlinks and cross-references
\usepackage{cleveref}                 % Intelligent cross-referencing
\usepackage[capitalise]{cleveref}     % Capitalize cross-reference labels
\usepackage{natbib}                   % Bibliography support
\usepackage{doi}                      % DOI links

% Configure figure numbering and captions
\renewcommand{\figurename}{Figure}
\captionsetup{
    justification=centering,
    font=small,
    labelfont=bf,
    labelsep=period
}

% Configure table numbering and captions
\renewcommand{\tablename}{Table}
\captionsetup[table]{
    justification=centering,
    font=small,
    labelfont=bf,
    labelsep=period
}

% Configure section numbering
\setcounter{secnumdepth}{3}
\renewcommand{\thesection}{\arabic{section}}
\renewcommand{\thesubsection}{\arabic{section}.\arabic{subsection}}
\renewcommand{\thesubsubsection}{\arabic{section}.\arabic{subsection}.\arabic{subsubsection}}

% Configure equation numbering
\numberwithin{equation}{section}

% Configure hyperref for proper linking
\hypersetup{
    colorlinks=true,
    linkcolor=blue,
    citecolor=blue,
    urlcolor=blue,
    filecolor=blue,
    pdfborder={0 0 0},
    bookmarks=true,
    bookmarksnumbered=true,
    bookmarkstype=toc,
    pdftitle={Research Project Template},
    pdfauthor={Template Author},
    pdfsubject={Academic Research},
    pdfkeywords={research, template, academic, LaTeX},
    pdfcreator={render_pdf.sh},
    pdfproducer={XeLaTeX}
}

% Configure cleveref for intelligent cross-references
\crefname{section}{Section}{Sections}
\crefname{subsection}{Subsection}{Subsections}
\crefname{subsubsection}{Subsubsection}{Subsubsections}
\crefname{equation}{Equation}{Equations}
\crefname{figure}{Figure}{Figures}
\crefname{table}{Table}{Tables}
\crefname{appendix}{Appendix}{Appendices}

% Configure fonts for Unicode support with fallbacks
\usepackage{newunicodechar}
\newunicodechar{⁴}{\textsuperscript{4}}
\newunicodechar{₄}{\textsubscript{4}}
\newunicodechar{²}{\textsuperscript{2}}
\newunicodechar{₀}{\textsubscript{0}}
\newunicodechar{₁}{\textsubscript{1}}
\newunicodechar{₂}{\textsubscript{2}}
\newunicodechar{₃}{\textsubscript{3}}

% Use standard fonts for better compatibility
\usepackage{lmodern}
\usepackage[T1]{fontenc}

% Enhanced code block styling for better contrast and readability
\usepackage{fancyvrb}
\usepackage{xcolor}
\usepackage{listings}

% Define custom colors for code blocks
\definecolor{codebg}{RGB}{248, 248, 248}      % Very light gray background
\definecolor{codeborder}{RGB}{200, 200, 200}  % Medium gray border
\definecolor{codefg}{RGB}{34, 34, 34}         % Dark gray text
\definecolor{commentcolor}{RGB}{102, 102, 102} % Comment color
\definecolor{keywordcolor}{RGB}{0, 0, 0}       % Keyword color
\definecolor{stringcolor}{RGB}{0, 102, 0}      % String color

% Configure Verbatim environment for inline code
\DefineVerbatimEnvironment{Verbatim}{Verbatim}{%
    fontsize=\small,
    frame=single,
    framerule=0.5pt,
    framesep=3pt,
    rulecolor=\color{codeborder},
    bgcolor=\color{codebg},
    fgcolor=\color{codefg}
}

% Configure code block styling
\DefineVerbatimEnvironment{Highlighting}{Verbatim}{%
    fontsize=\footnotesize,
    frame=single,
    framerule=0.5pt,
    framesep=5pt,
    rulecolor=\color{codeborder},
    bgcolor=\color{codebg},
    fgcolor=\color{codefg}
}

% Style inline code with \texttt
\renewcommand{\texttt}[1]{%
    \colorbox{codebg}{\color{codefg}\ttfamily #1}%
}

% Configure listings package for code blocks
\lstset{
    backgroundcolor=\color{codebg},
    basicstyle=\footnotesize\ttfamily\color{codefg},
    breakatwhitespace=false,
    breaklines=true,
    captionpos=b,
    commentstyle=\color{commentcolor},
    deletekeywords={...},
    escapeinside={\%*}{*)},
    extendedchars=true,
    frame=single,
    framerule=0.5pt,
    framesep=5pt,
    keepspaces=true,
    keywordstyle=\color{keywordcolor}\bfseries,
    language=Python,
    morekeywords={*,...},
    numbers=left,
    numbersep=5pt,
    numberstyle=\tiny\color{codefg},
    rulecolor=\color{codeborder},
    showspaces=false,
    showstringspaces=false,
    showtabs=false,
    stepnumber=1,
    stringstyle=\color{stringcolor},
    tabsize=4,
    title=\lstname
}

% Override any Pandoc default lstset configurations
\AtBeginDocument{
    \lstset{
        backgroundcolor=\color{codebg},
        basicstyle=\footnotesize\ttfamily\color{codefg},
        frame=single,
        framerule=0.5pt,
        framesep=5pt,
        rulecolor=\color{codeborder},
        numbers=left,
        numbersep=5pt,
        numberstyle=\tiny\color{codefg}
    }
}

% Configure bibliography
\bibliographystyle{unsrt}  % Unsorted bibliography style
% Bibliography is handled in 07_references.md

% Simple page break support for document structure
% Note: Page breaks are handled in the markdown generation, not here

% Ensure proper spacing and formatting
\frenchspacing  % Single space after periods
\ifLuaTeX
  \usepackage{selnolig}  % disable illegal ligatures
\fi

\title{06 conclusion}
\author{ORCID: 0000-0000-0000-0000\\ Email: author@example.com}
\date{November 13, 2025}

\begin{document}
\maketitle

{
\hypersetup{linkcolor=black}
\setcounter{tocdepth}{3}
\tableofcontents
}
\setstretch{1.2}
\hypertarget{sec:conclusion}{%
\section{Conclusion}\label{sec:conclusion}}

\hypertarget{summary-of-contributions}{%
\subsection{Summary of Contributions}\label{summary-of-contributions}}

This work presents a novel optimization framework that achieves both
theoretical guarantees and practical performance. Our main contributions
are:

\begin{enumerate}
\def\labelenumi{\arabic{enumi}.}
\tightlist
\item
  \textbf{Theoretical Framework}: A comprehensive mathematical framework
  expressed in equations \eqref{eq:objective} through
  \eqref{eq:complexity_bound}
\item
  \textbf{Efficient Algorithm}: An iterative optimization algorithm with
  proven convergence rate \eqref{eq:convergence}
\item
  \textbf{Adaptive Strategy}: A novel adaptive step size rule
  \eqref{eq:adaptive_step} that ensures numerical stability
\item
  \textbf{Scalable Implementation}: An \(O(n \log n)\) complexity
  implementation validated by experimental results
\end{enumerate}

\hypertarget{key-results}{%
\subsection{Key Results}\label{key-results}}

\hypertarget{theoretical-achievements}{%
\subsubsection{Theoretical
Achievements}\label{theoretical-achievements}}

The theoretical analysis presented in Section \ref{sec:methodology}
establishes several important results:

\begin{itemize}
\tightlist
\item
  \textbf{Convergence Guarantee}: Linear convergence with rate
  \(\rho \in (0,1)\) as shown in \eqref{eq:convergence}
\item
  \textbf{Complexity Bound}: Optimal \(O(n \log n)\) per-iteration
  complexity
\item
  \textbf{Memory Scaling}: Linear memory requirements \eqref{eq:memory}
  suitable for large-scale problems
\end{itemize}

\hypertarget{experimental-validation}{%
\subsubsection{Experimental Validation}\label{experimental-validation}}

The experimental results from Section \ref{sec:experimental_results}
confirm our theoretical predictions:

\begin{itemize}
\tightlist
\item
  \textbf{Convergence Rate}: Empirical constants \(C \approx 1.2\) and
  \(\rho \approx 0.85\) match theoretical bounds, as demonstrated in
  Figure \ref{fig:convergence_plot}
\item
  \textbf{Scalability}: Performance scales as predicted by our
  complexity analysis
\item
  \textbf{Robustness}: 94.3\% success rate across diverse problem
  instances
\end{itemize}

\hypertarget{performance-improvements}{%
\subsubsection{Performance
Improvements}\label{performance-improvements}}

Our method demonstrates significant improvements over state-of-the-art
approaches:

\begin{equation}\label{eq:final_improvement}
\text{Overall Improvement} = \frac{\text{Performance}_{\text{ours}} - \text{Performance}_{\text{best}}}{\text{Performance}_{\text{best}}} \times 100\% = 23.7\%
\end{equation}

\hypertarget{broader-impact}{%
\subsection{Broader Impact}\label{broader-impact}}

\hypertarget{scientific-applications}{%
\subsubsection{Scientific Applications}\label{scientific-applications}}

The optimization framework developed here has applications across
multiple domains:

\begin{enumerate}
\def\labelenumi{\arabic{enumi}.}
\tightlist
\item
  \textbf{Machine Learning}: Efficient training of large-scale neural
  networks \cite{kingma2014, wright2010}
\item
  \textbf{Signal Processing}: Sparse signal reconstruction and denoising
  \cite{beck2009}
\item
  \textbf{Computational Biology}: Protein structure prediction and
  molecular dynamics
\item
  \textbf{Climate Modeling}: Parameter estimation in complex
  environmental systems \cite{polak1997}
\end{enumerate}

\hypertarget{industry-relevance}{%
\subsubsection{Industry Relevance}\label{industry-relevance}}

The practical benefits demonstrated in our experiments translate to
real-world impact:

\begin{itemize}
\tightlist
\item
  \textbf{Computational Efficiency}: 30\% reduction in iteration count
\item
  \textbf{Scalability}: Linear memory scaling enables larger problem
  sizes
\item
  \textbf{Reliability}: High success rates reduce operational costs
\end{itemize}

\hypertarget{future-directions}{%
\subsection{Future Directions}\label{future-directions}}

\hypertarget{immediate-extensions}{%
\subsubsection{Immediate Extensions}\label{immediate-extensions}}

Several promising directions for immediate future work emerged from our
analysis:

\begin{enumerate}
\def\labelenumi{\arabic{enumi}.}
\tightlist
\item
  \textbf{Non-convex Problems}: Extending theoretical guarantees beyond
  convexity
\item
  \textbf{Stochastic Variants}: Developing versions for noisy gradient
  estimates
\item
  \textbf{Multi-objective Optimization}: Handling conflicting objectives
  simultaneously
\end{enumerate}

\hypertarget{long-term-vision}{%
\subsubsection{Long-term Vision}\label{long-term-vision}}

The theoretical foundation established here opens several long-term
research directions:

\begin{enumerate}
\def\labelenumi{\arabic{enumi}.}
\tightlist
\item
  \textbf{Theoretical Advances}: Improving complexity bounds through
  more sophisticated analysis (see Section
  \ref{sec:supplemental_analysis})
\item
  \textbf{Algorithmic Innovation}: Developing variants for specific
  application domains (see Section \ref{sec:supplemental_applications})
\item
  \textbf{Software Ecosystem}: Building comprehensive optimization
  libraries
\end{enumerate}

\hypertarget{final-remarks}{%
\subsection{Final Remarks}\label{final-remarks}}

This work demonstrates that careful theoretical analysis combined with
practical implementation can yield optimization methods that are both
theoretically sound and practically effective. The convergence
guarantees, complexity analysis, and experimental validation provide a
solid foundation for future developments in optimization theory and
practice.

The framework's success across diverse problem domains suggests that the
principles developed here have broader applicability than initially
envisioned. As optimization problems become increasingly complex and
large-scale, the efficiency and reliability demonstrated by our approach
will become increasingly valuable.

We believe this work represents a significant step forward in the field
of optimization, providing both theoretical insights and practical tools
for researchers and practitioners alike.

\end{document}
