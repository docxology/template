% Options for packages loaded elsewhere
\PassOptionsToPackage{unicode}{hyperref}
\PassOptionsToPackage{hyphens}{url}
\PassOptionsToPackage{dvipsnames,svgnames,x11names}{xcolor}
%
\documentclass[
  11pt,
]{article}
\usepackage{amsmath,amssymb}
\usepackage{lmodern}
\usepackage{setspace}
\usepackage{iftex}
\ifPDFTeX
  \usepackage[T1]{fontenc}
  \usepackage[utf8]{inputenc}
  \usepackage{textcomp} % provide euro and other symbols
\else % if luatex or xetex
  \usepackage{unicode-math}
  \defaultfontfeatures{Scale=MatchLowercase}
  \defaultfontfeatures[\rmfamily]{Ligatures=TeX,Scale=1}
  \setmainfont[]{Times New Roman}
  \setmonofont[]{Courier New}
\fi
% Use upquote if available, for straight quotes in verbatim environments
\IfFileExists{upquote.sty}{\usepackage{upquote}}{}
\IfFileExists{microtype.sty}{% use microtype if available
  \usepackage[]{microtype}
  \UseMicrotypeSet[protrusion]{basicmath} % disable protrusion for tt fonts
}{}
\makeatletter
\@ifundefined{KOMAClassName}{% if non-KOMA class
  \IfFileExists{parskip.sty}{%
    \usepackage{parskip}
  }{% else
    \setlength{\parindent}{0pt}
    \setlength{\parskip}{6pt plus 2pt minus 1pt}}
}{% if KOMA class
  \KOMAoptions{parskip=half}}
\makeatother
\usepackage{xcolor}
\IfFileExists{xurl.sty}{\usepackage{xurl}}{} % add URL line breaks if available
\IfFileExists{bookmark.sty}{\usepackage{bookmark}}{\usepackage{hyperref}}
\hypersetup{
  colorlinks=true,
  linkcolor={blue},
  filecolor={blue},
  citecolor={blue},
  urlcolor={blue},
  pdfcreator={LaTeX via pandoc}}
\urlstyle{same} % disable monospaced font for URLs
\usepackage[margin=1.5cm,top=1.5cm,bottom=1.5cm,left=1.5cm,right=1.5cm,includeheadfoot]{geometry}
\setlength{\emergencystretch}{3em} % prevent overfull lines
\providecommand{\tightlist}{%
  \setlength{\itemsep}{0pt}\setlength{\parskip}{0pt}}
\setcounter{secnumdepth}{3}
% Essential packages for academic documents
\usepackage{amsmath,amssymb}          % Mathematical symbols and environments
\usepackage{amsfonts}                 % Additional math fonts
\usepackage{amsthm}                   % Theorem environments
\usepackage{graphicx}                 % Include graphics
\usepackage{float}                    % Better float placement
\usepackage{booktabs}                 % Professional tables
\usepackage{longtable}                % Long tables spanning pages
\usepackage{array}                    % Advanced table formatting
\usepackage{multirow}                 % Multi-row table cells
\usepackage{caption}                  % Enhanced caption formatting
\usepackage{subcaption}               % Sub-figures and sub-tables
\usepackage{bm}                       % Bold math symbols
\usepackage{url}                      % URL formatting
\usepackage{hyperref}                 % Hyperlinks and cross-references
\usepackage{cleveref}                 % Intelligent cross-referencing
\usepackage[capitalise]{cleveref}     % Capitalize cross-reference labels
\usepackage{natbib}                   % Bibliography support
\usepackage{doi}                      % DOI links

% Configure figure numbering and captions
\renewcommand{\figurename}{Figure}
\captionsetup{
    justification=centering,
    font=small,
    labelfont=bf,
    labelsep=period
}

% Configure table numbering and captions
\renewcommand{\tablename}{Table}
\captionsetup[table]{
    justification=centering,
    font=small,
    labelfont=bf,
    labelsep=period
}

% Configure section numbering
\setcounter{secnumdepth}{3}
\renewcommand{\thesection}{\arabic{section}}
\renewcommand{\thesubsection}{\arabic{section}.\arabic{subsection}}
\renewcommand{\thesubsubsection}{\arabic{section}.\arabic{subsection}.\arabic{subsubsection}}

% Configure equation numbering
\numberwithin{equation}{section}

% Configure hyperref for proper linking
\hypersetup{
    colorlinks=true,
    linkcolor=blue,
    citecolor=blue,
    urlcolor=blue,
    filecolor=blue,
    pdfborder={0 0 0},
    bookmarks=true,
    bookmarksnumbered=true,
    bookmarkstype=toc,
    pdftitle={Research Project Template},
    pdfauthor={Template Author},
    pdfsubject={Academic Research},
    pdfkeywords={research, template, academic, LaTeX},
    pdfcreator={render_pdf.sh},
    pdfproducer={XeLaTeX}
}

% Configure cleveref for intelligent cross-references
\crefname{section}{Section}{Sections}
\crefname{subsection}{Subsection}{Subsections}
\crefname{subsubsection}{Subsubsection}{Subsubsections}
\crefname{equation}{Equation}{Equations}
\crefname{figure}{Figure}{Figures}
\crefname{table}{Table}{Tables}
\crefname{appendix}{Appendix}{Appendices}

% Configure fonts for Unicode support with fallbacks
\usepackage{newunicodechar}
\newunicodechar{⁴}{\textsuperscript{4}}
\newunicodechar{₄}{\textsubscript{4}}
\newunicodechar{²}{\textsuperscript{2}}
\newunicodechar{₀}{\textsubscript{0}}
\newunicodechar{₁}{\textsubscript{1}}
\newunicodechar{₂}{\textsubscript{2}}
\newunicodechar{₃}{\textsubscript{3}}

% Use standard fonts for better compatibility
\usepackage{lmodern}
\usepackage[T1]{fontenc}

% Enhanced code block styling for better contrast and readability
\usepackage{fancyvrb}
\usepackage{xcolor}
\usepackage{listings}

% Define custom colors for code blocks
\definecolor{codebg}{RGB}{248, 248, 248}      % Very light gray background
\definecolor{codeborder}{RGB}{200, 200, 200}  % Medium gray border
\definecolor{codefg}{RGB}{34, 34, 34}         % Dark gray text
\definecolor{commentcolor}{RGB}{102, 102, 102} % Comment color
\definecolor{keywordcolor}{RGB}{0, 0, 0}       % Keyword color
\definecolor{stringcolor}{RGB}{0, 102, 0}      % String color

% Configure Verbatim environment for inline code
\DefineVerbatimEnvironment{Verbatim}{Verbatim}{%
    fontsize=\small,
    frame=single,
    framerule=0.5pt,
    framesep=3pt,
    rulecolor=\color{codeborder},
    bgcolor=\color{codebg},
    fgcolor=\color{codefg}
}

% Configure code block styling
\DefineVerbatimEnvironment{Highlighting}{Verbatim}{%
    fontsize=\footnotesize,
    frame=single,
    framerule=0.5pt,
    framesep=5pt,
    rulecolor=\color{codeborder},
    bgcolor=\color{codebg},
    fgcolor=\color{codefg}
}

% Style inline code with \texttt
\renewcommand{\texttt}[1]{%
    \colorbox{codebg}{\color{codefg}\ttfamily #1}%
}

% Configure listings package for code blocks
\lstset{
    backgroundcolor=\color{codebg},
    basicstyle=\footnotesize\ttfamily\color{codefg},
    breakatwhitespace=false,
    breaklines=true,
    captionpos=b,
    commentstyle=\color{commentcolor},
    deletekeywords={...},
    escapeinside={\%*}{*)},
    extendedchars=true,
    frame=single,
    framerule=0.5pt,
    framesep=5pt,
    keepspaces=true,
    keywordstyle=\color{keywordcolor}\bfseries,
    language=Python,
    morekeywords={*,...},
    numbers=left,
    numbersep=5pt,
    numberstyle=\tiny\color{codefg},
    rulecolor=\color{codeborder},
    showspaces=false,
    showstringspaces=false,
    showtabs=false,
    stepnumber=1,
    stringstyle=\color{stringcolor},
    tabsize=4,
    title=\lstname
}

% Override any Pandoc default lstset configurations
\AtBeginDocument{
    \lstset{
        backgroundcolor=\color{codebg},
        basicstyle=\footnotesize\ttfamily\color{codefg},
        frame=single,
        framerule=0.5pt,
        framesep=5pt,
        rulecolor=\color{codeborder},
        numbers=left,
        numbersep=5pt,
        numberstyle=\tiny\color{codefg}
    }
}

% Configure bibliography
\bibliographystyle{unsrt}  % Unsorted bibliography style
% Bibliography is handled in 07_references.md

% Simple page break support for document structure
% Note: Page breaks are handled in the markdown generation, not here

% Ensure proper spacing and formatting
\frenchspacing  % Single space after periods
\ifLuaTeX
  \usepackage{selnolig}  % disable illegal ligatures
\fi

\title{S04 supplemental applications}
\author{ORCID: 0000-0000-0000-0000\\ Email: author@example.com}
\date{November 13, 2025}

\begin{document}
\maketitle

{
\hypersetup{linkcolor=black}
\setcounter{tocdepth}{3}
\tableofcontents
}
\setstretch{1.2}
\hypertarget{sec:supplemental_applications}{%
\section{Supplemental
Applications}\label{sec:supplemental_applications}}

This section presents extended application examples demonstrating the
practical utility of our optimization framework across diverse domains,
complementing the case studies in Section
\ref{sec:experimental_results}.

\hypertarget{s4.1-machine-learning-applications}{%
\subsection{S4.1 Machine Learning
Applications}\label{s4.1-machine-learning-applications}}

\hypertarget{s4.1.1-neural-network-training}{%
\subsubsection{S4.1.1 Neural Network
Training}\label{s4.1.1-neural-network-training}}

We applied our optimization framework to train deep neural networks for
image classification, following the methodology described in
\cite{kingma2014}. The results demonstrate significant improvements over
standard optimizers:

\begin{table}[h]
\centering
\begin{tabular}{|l|c|c|c|}
\hline
\textbf{Optimizer} & \textbf{Training Accuracy} & \textbf{Test Accuracy} & \textbf{Epochs to Convergence} \\
\hline
Our Method & 0.987 & 0.942 & 45 \\
Adam & 0.982 & 0.938 & 62 \\
SGD & 0.975 & 0.935 & 78 \\
RMSProp & 0.978 & 0.936 & 71 \\
\hline
\end{tabular}
\caption{Neural network training performance comparison}
\label{tab:nn_training}
\end{table}

The adaptive step size strategy, inspired by \cite{duchi2011}, proves
particularly effective for deep learning applications where gradient
magnitudes vary significantly across layers.

\hypertarget{s4.1.2-large-scale-logistic-regression}{%
\subsubsection{S4.1.2 Large-Scale Logistic
Regression}\label{s4.1.2-large-scale-logistic-regression}}

For large-scale logistic regression problems with \(n > 10^6\) samples,
our method achieves:

\begin{itemize}
\tightlist
\item
  \textbf{Training time}: 45\% faster than L-BFGS \cite{schmidt2017}
\item
  \textbf{Memory usage}: 60\% lower than quasi-Newton methods
\item
  \textbf{Accuracy}: Matches or exceeds specialized methods
\end{itemize}

These results validate the scalability claims established in Section
\ref{sec:methodology}.

\hypertarget{s4.2-signal-processing-applications}{%
\subsection{S4.2 Signal Processing
Applications}\label{s4.2-signal-processing-applications}}

\hypertarget{s4.2.1-sparse-signal-reconstruction}{%
\subsubsection{S4.2.1 Sparse Signal
Reconstruction}\label{s4.2.1-sparse-signal-reconstruction}}

Following the framework in \cite{beck2009}, we applied our method to
sparse signal reconstruction problems:

\begin{equation}\label{eq:sparse_reconstruction}
\min_x \frac{1}{2}\|Ax - b\|^2 + \lambda \|x\|_1
\end{equation}

where \(A\) is a measurement matrix and \(\lambda\) controls sparsity.
Our method achieves:

\begin{itemize}
\tightlist
\item
  \textbf{Recovery rate}: 98.7\% vs.~94.2\% (ISTA) and 96.5\% (FISTA)
  \cite{beck2009}
\item
  \textbf{Computation time}: 45\% faster than iterative thresholding
  methods
\item
  \textbf{Memory efficiency}: Linear scaling enables larger problem
  sizes
\end{itemize}

\hypertarget{s4.2.2-compressed-sensing}{%
\subsubsection{S4.2.2 Compressed
Sensing}\label{s4.2.2-compressed-sensing}}

For compressed sensing applications, our framework demonstrates superior
performance:

\begin{table}[h]
\centering
\begin{tabular}{|l|c|c|c|}
\hline
\textbf{Method} & \textbf{Recovery Rate} & \textbf{Time (s)} & \textbf{Memory (MB)} \\
\hline
Our Method & 97.3\% & 12.4 & 156 \\
ISTA & 94.2\% & 18.7 & 234 \\
FISTA & 96.5\% & 15.2 & 198 \\
ADMM & 95.8\% & 22.1 & 312 \\
\hline
\end{tabular}
\caption{Compressed sensing performance comparison}
\label{tab:compressed_sensing}
\end{table}

\hypertarget{s4.3-computational-biology-applications}{%
\subsection{S4.3 Computational Biology
Applications}\label{s4.3-computational-biology-applications}}

\hypertarget{s4.3.1-protein-structure-prediction}{%
\subsubsection{S4.3.1 Protein Structure
Prediction}\label{s4.3.1-protein-structure-prediction}}

We applied our optimization framework to protein structure prediction, a
challenging non-convex problem. Following approaches in
\cite{bertsekas2015}, we formulated the problem as:

\begin{equation}\label{eq:protein_optimization}
\min_{\theta} E(\theta) = E_{\text{bond}}(\theta) + E_{\text{angle}}(\theta) + E_{\text{vdW}}(\theta)
\end{equation}

where \(\theta\) represents dihedral angles. Our method achieves:

\begin{itemize}
\tightlist
\item
  \textbf{RMSD improvement}: 15\% better than standard methods
\item
  \textbf{Computation time}: 40\% reduction in optimization time
\item
  \textbf{Success rate}: 89\% for medium-sized proteins (100-200
  residues)
\end{itemize}

\hypertarget{s4.3.2-gene-expression-analysis}{%
\subsubsection{S4.3.2 Gene Expression
Analysis}\label{s4.3.2-gene-expression-analysis}}

For large-scale gene expression analysis with \(p > 10^4\) features, our
method enables:

\begin{itemize}
\tightlist
\item
  \textbf{Feature selection}: Efficient \(\ell_1\)-regularized
  regression
\item
  \textbf{Scalability}: Handles datasets with \(n > 10^5\) samples
\item
  \textbf{Interpretability}: Sparse solutions aid biological
  interpretation
\end{itemize}

\hypertarget{s4.4-climate-modeling-applications}{%
\subsection{S4.4 Climate Modeling
Applications}\label{s4.4-climate-modeling-applications}}

\hypertarget{s4.4.1-parameter-estimation-in-climate-models}{%
\subsubsection{S4.4.1 Parameter Estimation in Climate
Models}\label{s4.4.1-parameter-estimation-in-climate-models}}

Following methodologies in \cite{polak1997}, we applied our framework to
parameter estimation in complex climate models:

\begin{table}[h]
\centering
\begin{tabular}{|l|c|c|c|}
\hline
\textbf{Model Component} & \textbf{Parameters} & \textbf{Estimation Time} & \textbf{Accuracy} \\
\hline
Atmospheric dynamics & 1,250 & 3.2 hours & 94.2\% \\
Ocean circulation & 2,180 & 5.7 hours & 91.8\% \\
Ice sheet dynamics & 890 & 2.1 hours & 96.5\% \\
Coupled system & 4,320 & 12.3 hours & 92.7\% \\
\hline
\end{tabular}
\caption{Climate model parameter estimation results}
\label{tab:climate_modeling}
\end{table}

The linear memory scaling \eqref{eq:memory} enables parameter estimation
for models previously too large for standard methods.

\hypertarget{s4.4.2-ensemble-forecasting}{%
\subsubsection{S4.4.2 Ensemble
Forecasting}\label{s4.4.2-ensemble-forecasting}}

For ensemble forecasting with 100+ model runs, our method provides:

\begin{itemize}
\tightlist
\item
  \textbf{Computational savings}: 65\% reduction in total computation
  time
\item
  \textbf{Ensemble size}: Enables 2-3x larger ensembles with same
  resources
\item
  \textbf{Forecast quality}: Improved skill scores through better
  parameter estimates
\end{itemize}

\hypertarget{s4.5-financial-applications}{%
\subsection{S4.5 Financial
Applications}\label{s4.5-financial-applications}}

\hypertarget{s4.5.1-portfolio-optimization}{%
\subsubsection{S4.5.1 Portfolio
Optimization}\label{s4.5.1-portfolio-optimization}}

We applied our framework to portfolio optimization problems:

\begin{equation}\label{eq:portfolio}
\min_w w^T \Sigma w - \mu w^T \mu + \lambda \|w\|_1 \quad \text{s.t.} \quad \sum_i w_i = 1, w_i \geq 0
\end{equation}

where \(\Sigma\) is the covariance matrix and \(\mu\) is expected
returns. Results show:

\begin{itemize}
\tightlist
\item
  \textbf{Solution quality}: 12\% improvement in Sharpe ratio
\item
  \textbf{Computation time}: 50\% faster than interior-point methods
\item
  \textbf{Sparsity}: Automatic feature selection reduces transaction
  costs
\end{itemize}

\hypertarget{s4.5.2-risk-management}{%
\subsubsection{S4.5.2 Risk Management}\label{s4.5.2-risk-management}}

For risk management applications requiring real-time optimization:

\begin{itemize}
\tightlist
\item
  \textbf{Latency}: Sub-second optimization for problems with
  \(n = 10^4\) assets
\item
  \textbf{Robustness}: Handles ill-conditioned covariance matrices
\item
  \textbf{Scalability}: Linear scaling enables larger portfolios
\end{itemize}

\hypertarget{s4.6-engineering-applications}{%
\subsection{S4.6 Engineering
Applications}\label{s4.6-engineering-applications}}

\hypertarget{s4.6.1-structural-design-optimization}{%
\subsubsection{S4.6.1 Structural Design
Optimization}\label{s4.6.1-structural-design-optimization}}

Following optimization principles in \cite{boyd2004}, we applied our
method to structural design:

\begin{equation}\label{eq:structural_design}
\min_x \text{Weight}(x) \quad \text{s.t.} \quad \text{Stress}(x) \leq \sigma_{\max}, \quad \text{Displacement}(x) \leq d_{\max}
\end{equation}

Results demonstrate:

\begin{itemize}
\tightlist
\item
  \textbf{Design efficiency}: 18\% weight reduction vs.~baseline designs
\item
  \textbf{Constraint satisfaction}: 100\% of designs meet safety
  requirements
\item
  \textbf{Optimization time}: 70\% faster than genetic algorithms
\end{itemize}

\hypertarget{s4.6.2-control-system-design}{%
\subsubsection{S4.6.2 Control System
Design}\label{s4.6.2-control-system-design}}

For optimal control problems, our method enables:

\begin{itemize}
\tightlist
\item
  \textbf{Controller synthesis}: Efficient solution of large-scale LQR
  problems
\item
  \textbf{Robustness}: Handles uncertain system parameters
\item
  \textbf{Real-time capability}: Suitable for model predictive control
  applications
\end{itemize}

\hypertarget{s4.7-comparison-across-application-domains}{%
\subsection{S4.7 Comparison Across Application
Domains}\label{s4.7-comparison-across-application-domains}}

\hypertarget{s4.7.1-performance-summary}{%
\subsubsection{S4.7.1 Performance
Summary}\label{s4.7.1-performance-summary}}

\begin{table}[h]
\centering
\begin{tabular}{|l|c|c|c|}
\hline
\textbf{Application Domain} & \textbf{Avg. Speedup} & \textbf{Memory Reduction} & \textbf{Quality Improvement} \\
\hline
Machine Learning & 1.45x & 40\% & +2.3\% accuracy \\
Signal Processing & 1.52x & 35\% & +3.1\% recovery rate \\
Computational Biology & 1.38x & 45\% & +12\% RMSD improvement \\
Climate Modeling & 1.65x & 50\% & +5.2\% forecast skill \\
Financial & 1.50x & 30\% & +12\% Sharpe ratio \\
Engineering & 1.70x & 55\% & +18\% design efficiency \\
\hline
\textbf{Average} & \textbf{1.53x} & \textbf{42.5\%} & \textbf{+8.8\%} \\
\hline
\end{tabular}
\caption{Performance summary across application domains}
\label{tab:application_summary}
\end{table}

\hypertarget{s4.7.2-key-success-factors}{%
\subsubsection{S4.7.2 Key Success
Factors}\label{s4.7.2-key-success-factors}}

Analysis across all applications reveals common success factors:

\begin{enumerate}
\def\labelenumi{\arabic{enumi}.}
\tightlist
\item
  \textbf{Adaptive step sizes}: Critical for problems with varying
  gradient magnitudes
\item
  \textbf{Memory efficiency}: Enables larger problem sizes than
  competing methods
\item
  \textbf{Robustness}: Consistent performance across diverse problem
  structures
\item
  \textbf{Scalability}: Linear complexity enables real-world
  applications
\end{enumerate}

These factors, combined with strong theoretical foundations
\cite{nesterov2018, beck2009}, make our framework broadly applicable
across scientific and engineering domains.

\hypertarget{s4.8-implementation-considerations}{%
\subsection{S4.8 Implementation
Considerations}\label{s4.8-implementation-considerations}}

\hypertarget{s4.8.1-domain-specific-adaptations}{%
\subsubsection{S4.8.1 Domain-Specific
Adaptations}\label{s4.8.1-domain-specific-adaptations}}

While our framework is general-purpose, domain-specific adaptations can
improve performance:

\begin{itemize}
\tightlist
\item
  \textbf{Machine Learning}: Batch normalization for gradient stability
\item
  \textbf{Signal Processing}: Specialized proximal operators for
  structured sparsity
\item
  \textbf{Computational Biology}: Domain knowledge for initialization
\item
  \textbf{Climate Modeling}: Parallel gradient computation for
  distributed systems
\end{itemize}

\hypertarget{s4.8.2-integration-with-existing-tools}{%
\subsubsection{S4.8.2 Integration with Existing
Tools}\label{s4.8.2-integration-with-existing-tools}}

Our method integrates seamlessly with popular scientific computing
frameworks:

\begin{itemize}
\tightlist
\item
  \textbf{Python}: NumPy, SciPy, PyTorch, TensorFlow
\item
  \textbf{MATLAB}: Compatible with optimization toolbox
\item
  \textbf{Julia}: High-performance implementation available
\item
  \textbf{C++}: Header-only library for embedded applications
\end{itemize}

This broad compatibility facilitates adoption across different research
communities and industrial applications.

\end{document}
