% Options for packages loaded elsewhere
\PassOptionsToPackage{unicode}{hyperref}
\PassOptionsToPackage{hyphens}{url}
\documentclass[
  ignorenonframetext,
]{beamer}
\newif\ifbibliography
\usepackage{pgfpages}
\setbeamertemplate{caption}[numbered]
\setbeamertemplate{caption label separator}{: }
\setbeamercolor{caption name}{fg=normal text.fg}
\beamertemplatenavigationsymbolsempty
% remove section numbering
\setbeamertemplate{part page}{
  \centering
  \begin{beamercolorbox}[sep=16pt,center]{part title}
    \usebeamerfont{part title}\insertpart\par
  \end{beamercolorbox}
}
\setbeamertemplate{section page}{
  \centering
  \begin{beamercolorbox}[sep=12pt,center]{section title}
    \usebeamerfont{section title}\insertsection\par
  \end{beamercolorbox}
}
\setbeamertemplate{subsection page}{
  \centering
  \begin{beamercolorbox}[sep=8pt,center]{subsection title}
    \usebeamerfont{subsection title}\insertsubsection\par
  \end{beamercolorbox}
}
% Prevent slide breaks in the middle of a paragraph
\widowpenalties 1 10000
\raggedbottom
\AtBeginPart{
  \frame{\partpage}
}
\AtBeginSection{
  \ifbibliography
  \else
    \frame{\sectionpage}
  \fi
}
\AtBeginSubsection{
  \frame{\subsectionpage}
}
\usepackage{iftex}
\ifPDFTeX
  \usepackage[T1]{fontenc}
  \usepackage[utf8]{inputenc}
  \usepackage{textcomp} % provide euro and other symbols
\else % if luatex or xetex
  \usepackage{unicode-math} % this also loads fontspec
  \defaultfontfeatures{Scale=MatchLowercase}
  \defaultfontfeatures[\rmfamily]{Ligatures=TeX,Scale=1}
\fi
\usepackage{lmodern}
\ifPDFTeX\else
  % xetex/luatex font selection
\fi
% Use upquote if available, for straight quotes in verbatim environments
\IfFileExists{upquote.sty}{\usepackage{upquote}}{}
\IfFileExists{microtype.sty}{% use microtype if available
  \usepackage[]{microtype}
  \UseMicrotypeSet[protrusion]{basicmath} % disable protrusion for tt fonts
}{}
\makeatletter
\@ifundefined{KOMAClassName}{% if non-KOMA class
  \IfFileExists{parskip.sty}{%
    \usepackage{parskip}
  }{% else
    \setlength{\parindent}{0pt}
    \setlength{\parskip}{6pt plus 2pt minus 1pt}}
}{% if KOMA class
  \KOMAoptions{parskip=half}}
\makeatother
\setlength{\emergencystretch}{3em} % prevent overfull lines
\providecommand{\tightlist}{%
  \setlength{\itemsep}{0pt}\setlength{\parskip}{0pt}}
\usepackage{bookmark}
\IfFileExists{xurl.sty}{\usepackage{xurl}}{} % add URL line breaks if available
\urlstyle{same}
\hypersetup{
  hidelinks,
  pdfcreator={LaTeX via pandoc}}

\author{\texorpdfstring{}{}}
\date{}

\begin{document}

\begin{frame}{Discussion}
\protect\phantomsection\label{sec:discussion}
The (2 \times 2) matrix framework reveals Active Inference as a
fundamentally meta-level methodology with profound implications for
cognitive science, artificial intelligence, and our understanding of
intelligence itself. This section explores the theoretical implications
of viewing Active Inference through the lens of meta-pragmatic and
meta-epistemic operation.

\begin{block}{Meta-Pragmatic Implications}
\protect\phantomsection\label{sec:meta_pragmatic_implications}
Active Inference's meta-pragmatic nature transcends traditional
approaches to goal-directed behavior by enabling modelers to specify
pragmatic frameworks rather than simple reward functions.

\begin{block}{Beyond Reward Functions}
\protect\phantomsection\label{beyond-reward-functions}
Traditional reinforcement learning specifies rewards as scalar values:

\[R(s,a) \in \mathbb{R}\label{eq:traditional_reward}\]

Active Inference, however, enables specification of preference
landscapes through matrix (C):

\[C(o) \in \mathbb{R}^{|\mathcal{O}|}\label{eq:active_inference_preferences}\]

This enables modeling of: - \textbf{Complex Value Structures:}
Multi-dimensional preferences with trade-offs - \textbf{Ethical
Considerations:} Incorporation of moral and social values -
\textbf{Contextual Goals:} Situation-dependent value hierarchies -
\textbf{Meta-Preferences:} Preferences about preference structures
themselves
\end{block}

\begin{block}{Pragmatic Framework Design}
\protect\phantomsection\label{pragmatic-framework-design}
The meta-pragmatic power enables researchers to explore: - How different
societies develop different value systems - How individual development
shapes personal pragmatic frameworks - How cultural evolution influences
collective goal structures - How artificial agents might develop their
own pragmatic frameworks
\end{block}
\end{block}

\begin{block}{Meta-Epistemic Implications}
\protect\phantomsection\label{sec:meta_epistemic_implications}
Active Inference enables specification of epistemic frameworks, allowing
modelers to define not just what agents believe, but how they come to
know the world.

\begin{block}{Epistemological Pluralism}
\protect\phantomsection\label{epistemological-pluralism}
Different epistemic frameworks can be specified through generative model
parameters:

\textbf{Empirical Framework:}

\[A_{empirical} = \begin{pmatrix} 0.95 & 0.05 \\ 0.05 & 0.95 \end{pmatrix}\label{eq:empirical_framework}\]

High confidence in sensory observations, low uncertainty.

\textbf{Skeptical Framework:}

\[A_{skeptical} = \begin{pmatrix} 0.6 & 0.4 \\ 0.4 & 0.6 \end{pmatrix}\label{eq:skeptical_framework}\]

Lower confidence, higher epistemic caution.

\textbf{Dogmatic Framework:}

\[A_{dogmatic} = \begin{pmatrix} 1.0 & 0.0 \\ 0.0 & 1.0 \end{pmatrix}\label{eq:dogmatic_framework}\]

Absolute certainty, no epistemic doubt.
\end{block}

\begin{block}{Knowledge Architecture Design}
\protect\phantomsection\label{knowledge-architecture-design}
Active Inference enables design of knowledge acquisition systems:

\begin{itemize}
\tightlist
\item
  \textbf{Learning Mechanisms:} How beliefs update over time
\item
  \textbf{Uncertainty Handling:} Approaches to ambiguous information
\item
  \textbf{Evidence Integration:} How multiple sources combine
\item
  \textbf{Hypothesis Testing:} Frameworks for belief validation
\end{itemize}
\end{block}
\end{block}

\begin{block}{The Modeler's Dual Role}
\protect\phantomsection\label{sec:modeler_dual_role}
The framework reveals the recursive relationship between modeler and
modeled system.

\begin{block}{As Architect}
\protect\phantomsection\label{as-architect}
The modeler specifies the boundaries of cognition: - \textbf{Epistemic
Boundaries:} What can be known (matrix (A)) - \textbf{Pragmatic
Landscape:} What matters (matrix (C)) - \textbf{Causal Structure:} What
can be controlled (matrix (B)) - \textbf{Initial Assumptions:} What is
taken for granted (matrix (D))
\end{block}

\begin{block}{As Subject}
\protect\phantomsection\label{as-subject}
The modeler applies Active Inference to their own cognition: - Uses
meta-epistemic principles to design research methodologies - Employs
meta-pragmatic frameworks for scientific decision-making - Engages in
recursive self-modeling of cognitive processes
\end{block}

\begin{block}{Recursive Self-Understanding}
\protect\phantomsection\label{recursive-self-understanding}
This creates a recursive loop of understanding: 1. Modeler uses Active
Inference to model cognitive systems 2. Insights from modeling improve
understanding of modeler's own cognition 3. Improved self-understanding
leads to better models 4. Cycle continues with increasing sophistication
\end{block}
\end{block}

\begin{block}{Cognitive Security Implications}
\protect\phantomsection\label{sec:cognitive_security_implications}
The meta-level framework has significant implications for cognitive
security and the robustness of belief systems.

\begin{block}{Meta-Cognitive Vulnerabilities}
\protect\phantomsection\label{meta-cognitive-vulnerabilities}
Understanding meta-cognitive processing reveals potential
vulnerabilities:

\textbf{Quadrant 3 Attacks:} Manipulation of confidence assessment
mechanisms - False confidence calibration - Induced
over/under-confidence - Meta-cognitive hijacking

\textbf{Quadrant 4 Attacks:} Framework-level manipulation - Epistemic
framework subversion - Pragmatic landscape alteration - Higher-order
reasoning corruption
\end{block}

\begin{block}{Defense Strategies}
\protect\phantomsection\label{defense-strategies}
The framework suggests defense approaches:

\textbf{Meta-Cognitive Monitoring:} Continuous validation of confidence
assessments \textbf{Framework Integrity Checks:} Verification of
epistemic and pragmatic consistency \textbf{Recursive Validation:}
Higher-order checking of meta-level processes
\end{block}

\begin{block}{Societal Implications}
\protect\phantomsection\label{societal-implications}
These insights extend to societal cognitive security:

\begin{itemize}
\tightlist
\item
  \textbf{Information Warfare:} Meta-level manipulation of public belief
  systems
\item
  \textbf{AI Safety:} Ensuring artificial agents maintain meta-cognitive
  frameworks
\item
  \textbf{Educational Systems:} Developing curricula that build
  meta-cognitive resilience
\end{itemize}
\end{block}
\end{block}

\begin{block}{Free Energy Principle Integration}
\protect\phantomsection\label{sec:fep_integration}
The framework integrates seamlessly with the Free Energy Principle,
providing a concrete realization of FEP's abstract principles.

\begin{block}{What Is a Thing?}
\protect\phantomsection\label{what-is-a-thing}
The FEP defines a ``thing'' as a system that maintains its structure
over time through free energy minimization. Our framework shows how this
operates across multiple levels:

\textbf{Physical Level:} Boundary maintenance through Markov blankets
\textbf{Cognitive Level:} Belief updating through EFE minimization
\textbf{Meta-Cognitive Level:} Framework adaptation through higher-order
reasoning \textbf{Meta-Theoretical Level:} Scientific understanding
through recursive modeling
\end{block}

\begin{block}{Unification Across Domains}
\protect\phantomsection\label{unification-across-domains}
The framework provides a unified approach to diverse phenomena:

\textbf{Biological Systems:} Organisms maintaining homeostasis
\textbf{Artificial Agents:} AI systems with meta-learning capabilities
\textbf{Social Systems:} Groups maintaining collective identity
\textbf{Scientific Communities:} Knowledge accumulation through paradigm
shifts
\end{block}
\end{block}

\begin{block}{Methodological Contributions}
\protect\phantomsection\label{sec:methodological_contributions}
The framework advances Active Inference methodology in several ways:

\begin{block}{Systematic Analysis Framework}
\protect\phantomsection\label{systematic-analysis-framework}
Provides a systematic way to analyze meta-level phenomena: - Clear
distinctions between processing levels - Hierarchical organization of
cognitive processes - Integration of multiple abstraction levels
\end{block}

\begin{block}{Research Design Tool}
\protect\phantomsection\label{research-design-tool}
Enables researchers to: - Design experiments targeting specific
quadrants - Compare interventions across processing levels - Develop
targeted cognitive enhancement strategies
\end{block}

\begin{block}{Theoretical Integration}
\protect\phantomsection\label{theoretical-integration}
Bridges multiple theoretical traditions: - Active Inference with
meta-cognition research - Free Energy Principle with cognitive
architectures - Pragmatic reasoning with epistemic logic
\end{block}
\end{block}

\begin{block}{Limitations and Future Directions}
\protect\phantomsection\label{sec:limitations_future}
\begin{block}{Current Limitations}
\protect\phantomsection\label{current-limitations}
\textbf{Empirical Validation:} Framework is primarily theoretical;
empirical validation needed \textbf{Computational Complexity:} Higher
quadrants involve complex optimization \textbf{Measurement Challenges:}
Meta-level processes are difficult to measure directly \textbf{Scale
Issues:} Framework scaling to complex real-world systems
\end{block}

\begin{block}{Future Research Directions}
\protect\phantomsection\label{future-research-directions}
\textbf{Empirical Studies:} Develop experimental paradigms for each
quadrant \textbf{Computational Methods:} Efficient algorithms for
meta-level optimization \textbf{Measurement Techniques:} Novel
approaches to meta-cognitive process measurement \textbf{Applications:}
Deployment in AI systems and cognitive enhancement
\end{block}

\begin{block}{Extension Possibilities}
\protect\phantomsection\label{extension-possibilities}
\textbf{Multi-Agent Systems:} Framework extension to social cognition
\textbf{Developmental Psychology:} Application to cognitive development
\textbf{Clinical Applications:} Therapeutic interventions targeting
specific quadrants \textbf{Educational Technology:} Meta-cognitive
training systems
\end{block}
\end{block}

\begin{block}{Broader Philosophical Implications}
\protect\phantomsection\label{sec:philosophical_implications}
The framework touches on fundamental questions about cognition and
reality.

\begin{block}{Nature of Intelligence}
\protect\phantomsection\label{nature-of-intelligence}
Active Inference suggests intelligence emerges from: - \textbf{Epistemic
Competence:} Ability to construct accurate world models -
\textbf{Pragmatic Wisdom:} Capacity for effective goal-directed behavior
- \textbf{Meta-Level Reflection:} Self-awareness and adaptive control -
\textbf{Framework Flexibility:} Ability to modify fundamental cognitive
structures
\end{block}

\begin{block}{Reality and Representation}
\protect\phantomsection\label{reality-and-representation}
The meta-epistemic aspect raises questions about: - \textbf{Multiple
Realities:} Different epistemic frameworks construct different worlds -
\textbf{Framework Relativity:} Cognitive adequacy depends on framework
appropriateness - \textbf{Reality Construction:} Cognition as active
construction, not passive reception
\end{block}

\begin{block}{Consciousness and Self-Awareness}
\protect\phantomsection\label{consciousness-and-self-awareness}
The recursive nature of meta-cognition suggests: -
\textbf{Self-Modeling:} Consciousness as modeling one's own cognitive
processes - \textbf{Hierarchical Self-Awareness:} Three levels of
self-reflection - \textbf{Emergent Properties:} Consciousness emerging
from meta-level cognitive organization
\end{block}
\end{block}
\end{frame}

\end{document}
