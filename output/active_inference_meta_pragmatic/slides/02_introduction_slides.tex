% Options for packages loaded elsewhere
\PassOptionsToPackage{unicode}{hyperref}
\PassOptionsToPackage{hyphens}{url}
\documentclass[
  ignorenonframetext,
]{beamer}
\newif\ifbibliography
\usepackage{pgfpages}
\setbeamertemplate{caption}[numbered]
\setbeamertemplate{caption label separator}{: }
\setbeamercolor{caption name}{fg=normal text.fg}
\beamertemplatenavigationsymbolsempty
% remove section numbering
\setbeamertemplate{part page}{
  \centering
  \begin{beamercolorbox}[sep=16pt,center]{part title}
    \usebeamerfont{part title}\insertpart\par
  \end{beamercolorbox}
}
\setbeamertemplate{section page}{
  \centering
  \begin{beamercolorbox}[sep=12pt,center]{section title}
    \usebeamerfont{section title}\insertsection\par
  \end{beamercolorbox}
}
\setbeamertemplate{subsection page}{
  \centering
  \begin{beamercolorbox}[sep=8pt,center]{subsection title}
    \usebeamerfont{subsection title}\insertsubsection\par
  \end{beamercolorbox}
}
% Prevent slide breaks in the middle of a paragraph
\widowpenalties 1 10000
\raggedbottom
\AtBeginPart{
  \frame{\partpage}
}
\AtBeginSection{
  \ifbibliography
  \else
    \frame{\sectionpage}
  \fi
}
\AtBeginSubsection{
  \frame{\subsectionpage}
}
\usepackage{iftex}
\ifPDFTeX
  \usepackage[T1]{fontenc}
  \usepackage[utf8]{inputenc}
  \usepackage{textcomp} % provide euro and other symbols
\else % if luatex or xetex
  \usepackage{unicode-math} % this also loads fontspec
  \defaultfontfeatures{Scale=MatchLowercase}
  \defaultfontfeatures[\rmfamily]{Ligatures=TeX,Scale=1}
\fi
\usepackage{lmodern}
\ifPDFTeX\else
  % xetex/luatex font selection
\fi
% Use upquote if available, for straight quotes in verbatim environments
\IfFileExists{upquote.sty}{\usepackage{upquote}}{}
\IfFileExists{microtype.sty}{% use microtype if available
  \usepackage[]{microtype}
  \UseMicrotypeSet[protrusion]{basicmath} % disable protrusion for tt fonts
}{}
\makeatletter
\@ifundefined{KOMAClassName}{% if non-KOMA class
  \IfFileExists{parskip.sty}{%
    \usepackage{parskip}
  }{% else
    \setlength{\parindent}{0pt}
    \setlength{\parskip}{6pt plus 2pt minus 1pt}}
}{% if KOMA class
  \KOMAoptions{parskip=half}}
\makeatother
\setlength{\emergencystretch}{3em} % prevent overfull lines
\providecommand{\tightlist}{%
  \setlength{\itemsep}{0pt}\setlength{\parskip}{0pt}}
\usepackage{bookmark}
\IfFileExists{xurl.sty}{\usepackage{xurl}}{} % add URL line breaks if available
\urlstyle{same}
\hypersetup{
  hidelinks,
  pdfcreator={LaTeX via pandoc}}

\author{\texorpdfstring{}{}}
\date{}

\begin{document}

\begin{frame}{Introduction}
\protect\phantomsection\label{sec:introduction}
Active Inference represents a paradigm shift in our understanding of
cognition, perception, and action. Originating from the Free Energy
Principle {[}@friston2010free{]}, Active Inference provides a unified
mathematical formalism for understanding biological agents as systems
that minimize variational free energy through perception and action.
While Active Inference has been successfully applied to diverse domains
including neuroscience {[}@friston2012prediction{]}, psychiatry
{[}@friston2014active{]}, and artificial intelligence
{[}@tani2016exploring{]}, its fundamental nature as a meta-theoretical
methodology---enabling specification of the frameworks within which
cognition occurs---has remained underexplored.

\begin{block}{The Traditional View: Active Inference as Free Energy
Minimization}
\protect\phantomsection\label{the-traditional-view-active-inference-as-free-energy-minimization}
Conventionally, Active Inference is understood as a process where agents
act to fulfill prior preferences while gathering information about their
environment. The Expected Free Energy (EFE) formulation combines
epistemic and pragmatic terms:

\[\mathcal{F}(\pi) = \mathbb{E}_{q(s_\tau)}[\log q(s_\tau) - \log p(s_\tau \mid \pi)] + \mathbb{E}_{q(o_\tau)}[\log p(o_\tau \mid s_\tau) + \log p(s_\tau) - \log q(s_\tau)]\label{eq:efe}\]

The first term in Equation \eqref{eq:efe},
(\mathbb{E}\emph{\{q(s}\tau)\}{[}\log q(s\_\tau) -
\log p(s\_\tau \mid \pi){]}), represents \emph{epistemic value}:
information gain about hidden states through policy execution. The
second term,
(\mathbb{E}\emph{\{q(o}\tau)\}{[}\log p(o\_\tau \mid s\_\tau) +
\log p(s\_\tau) - \log q(s\_\tau){]}), represents \emph{pragmatic
value}: goal achievement through preferred observations. Action
selection minimizes EFE, balancing exploration (epistemic) and
exploitation (pragmatic) in a principled manner. Critically, both terms
depend on generative model specifications (matrices (A), (B), (C), and
(D)) that the modeler defines, revealing the meta-level nature of the
framework.
\end{block}

\begin{block}{Beyond the Traditional View: Active Inference as
Meta-Methodology}
\protect\phantomsection\label{beyond-the-traditional-view-active-inference-as-meta-methodology}
Active Inference operates at a fundamentally meta-level that
distinguishes it from traditional decision-making algorithms. Rather
than simply providing another method for selecting actions given fixed
observation models and reward functions, Active Inference allows
researchers to specify the very frameworks within which cognition
occurs. This meta-level operation manifests in two key dimensions that
together create a meta-methodology for cognitive science:

\begin{block}{Meta-Epistemic Aspect}
\protect\phantomsection\label{meta-epistemic-aspect}
Active Inference allows modelers to specify epistemic frameworks through
generative model matrices (A), (B), and (D). Matrix (A) defines
observation likelihoods (P(o \mid s)) (see Equation
\eqref{eq:matrix_a}), establishing what can be known about the world and
how reliably observations indicate underlying states. Matrix (D) defines
prior beliefs (P(s)) (see Equation \eqref{eq:matrix_d}), setting initial
assumptions about the world's structure. Matrix (B) defines state
transitions (P(s' \mid s, a)) (see Equation \eqref{eq:matrix_b}),
specifying causal relationships and how actions influence state changes.
Through these specifications, researchers define not just current
beliefs, but the epistemological boundaries of cognition
itself---determining what knowledge is possible, how evidence
accumulates, and what causal structures are assumed. This specification
power transforms framework design from an external constraint into an
internal research question.
\end{block}

\begin{block}{Meta-Pragmatic Aspect}
\protect\phantomsection\label{meta-pragmatic-aspect}
Beyond epistemic specification, Active Inference supports meta-pragmatic
modeling through matrix (C) (see Equation \eqref{eq:matrix_c}), which
defines preference priors. Unlike traditional reinforcement learning
where rewards are externally specified, Active Inference allows modelers
to specify pragmatic landscapes. The modeler specifies what constitutes
``value'' for the agent, creating opportunities to explore how different
value systems shape cognition and behavior. This specification power
extends beyond simple reward functions to enable complex value
hierarchies, trade-offs, and ethical considerations.
\end{block}
\end{block}

\begin{block}{The (2 \times 2) Framework: Data/Meta-Data (\times)
Cognitive/Meta-Cognitive}
\protect\phantomsection\label{the-2-framework-datameta-data-cognitivemeta-cognitive}
To systematically analyze Active Inference's meta-level contributions,
we introduce a (2 \times 2) matrix framework (Figure
\ref{fig:quadrant_matrix}) with axes of Data/Meta-Data and
Cognitive/Meta-Cognitive processing.

\textbf{Quadrant 1 - Data Processing (Cognitive Level):} Basic cognitive
processing of raw sensory data, implementing baseline pragmatic and
epistemic functionality through EFE minimization. This quadrant
represents the fundamental Active Inference mechanism where agents
process observations, update beliefs, and select actions to minimize
expected free energy. It provides the foundation for all higher-level
cognitive operations.

\textbf{Quadrant 2 - Meta-Data Processing (Cognitive Level):} Processing
that incorporates meta-information (confidence scores, timestamps,
reliability metrics) to enhance primary data processing. This quadrant
extends Quadrant 1 by weighting observations and inferences based on
quality information, improving decision reliability in uncertain
conditions. Meta-data integration allows systems to adapt their
processing based on information quality rather than treating all data
equally.

\textbf{Quadrant 3 - Data Processing (Meta-Cognitive Level):} Reflective
processing where agents evaluate their own cognitive processes,
implementing self-monitoring and adaptive control. This quadrant allows
systems to assess their own inference quality and adjust processing
strategies accordingly, demonstrating meta-cognitive self-regulation.
The hierarchical structure enables systems to monitor and regulate their
own cognitive operations.

\textbf{Quadrant 4 - Meta-Data Processing (Meta-Cognitive Level):}
Higher-order reasoning involving meta-data about meta-cognition,
supporting framework-level adaptation and meta-theoretical analysis.
This quadrant represents the highest level of cognitive abstraction,
where systems analyze patterns in their own meta-cognitive performance
to optimize fundamental framework parameters. Recursive self-analysis
enables continuous improvement of cognitive architectures.

\begin{figure}[h]
\centering
\includegraphics[width=0.8\textwidth]{../figures/quadrant_matrix.png}
\caption{\(2 \times 2\) Quadrant Structure: Data/Meta-Data \(\times\) Cognitive/Meta-Cognitive processing levels in Active Inference. The structure organizes cognitive processing along two dimensions: (1) Data vs Meta-Data (X-axis), distinguishing raw sensory inputs from information about data quality, reliability, and provenance; (2) Cognitive vs Meta-Cognitive (Y-axis), distinguishing direct information transformation from self-reflective monitoring and control of cognitive processes. Each quadrant represents a distinct mode of cognitive operation: Quadrant 1 (Data, Cognitive) provides fundamental EFE computation (Equation \eqref{eq:efe_simple}); Quadrant 2 (Meta-Data, Cognitive) enhances processing through quality weighting (Equation \eqref{eq:efe_metadata}); Quadrant 3 (Data, Meta-Cognitive) enables self-monitoring and adaptive control (Equation \eqref{eq:efe_hierarchical}); Quadrant 4 (Meta-Data, Meta-Cognitive) supports framework-level optimization (Equation \eqref{eq:framework_optimization}). Each quadrant has specific mathematical formulations and practical applications, creating a comprehensive framework for understanding multi-level cognitive operation.}
\label{fig:quadrant_matrix}
\end{figure}
\end{block}

\begin{block}{Contributions and Implications}
\protect\phantomsection\label{contributions-and-implications}
This structure reveals Active Inference as a meta-methodology that
transcends traditional approaches to cognition. By allowing meta-level
specification of epistemic and pragmatic frameworks, Active Inference
provides systematic tools for understanding:

\begin{enumerate}
\item
  \textbf{Cognitive Architecture Design:} How different epistemic
  frameworks (specified through matrices (A), (B), (D)) and pragmatic
  frameworks (specified through matrix (C)) shape cognition,
  decision-making, and behavior. Framework design itself becomes a
  research question.
\item
  \textbf{Meta-Cognitive Processing:} Self-reflective cognitive
  mechanisms operating across Quadrants 3 and 4, enabling systems to
  monitor, regulate, and evolve their own cognitive processes. These
  mechanisms have profound implications for understanding consciousness,
  self-awareness, and adaptive intelligence.
\item
  \textbf{Cognitive Security:} Vulnerabilities arising from meta-level
  cognitive manipulation, where attackers target confidence assessment
  (Quadrant 3) or framework parameters (Quadrant 4) rather than just
  beliefs or actions. The structure provides systematic defense
  strategies operating at multiple levels, enabling protection against
  sophisticated attacks that exploit meta-cognitive processes.
\item
  \textbf{Unification of Cognitive Science:} Bridging biological and
  artificial cognition through shared principles of free energy
  minimization operating across multiple scales, from physical systems
  to cognitive agents to scientific communities.
\end{enumerate}
\end{block}

\begin{block}{Paper Structure}
\protect\phantomsection\label{paper-structure}
Section \hyperlink{sec:methodology}{3} introduces the (2 \times 2)
matrix framework and demonstrates how Active Inference operates across
all four quadrants. Section \hyperlink{sec:experimental_results}{4}
provides conceptual demonstrations of each quadrant with mathematical
examples. Section \hyperlink{sec:discussion}{5} explores theoretical
implications and meta-level interpretations. Section
\hyperlink{sec:conclusion}{6} summarizes contributions and future
directions.

Supplemental materials provide mathematical derivations, additional
examples, and implementation details for the framework.
\end{block}
\end{frame}

\end{document}
