% Options for packages loaded elsewhere
\PassOptionsToPackage{unicode}{hyperref}
\PassOptionsToPackage{hyphens}{url}
\documentclass[
  ignorenonframetext,
]{beamer}
\newif\ifbibliography
\usepackage{pgfpages}
\setbeamertemplate{caption}[numbered]
\setbeamertemplate{caption label separator}{: }
\setbeamercolor{caption name}{fg=normal text.fg}
\beamertemplatenavigationsymbolsempty
% remove section numbering
\setbeamertemplate{part page}{
  \centering
  \begin{beamercolorbox}[sep=16pt,center]{part title}
    \usebeamerfont{part title}\insertpart\par
  \end{beamercolorbox}
}
\setbeamertemplate{section page}{
  \centering
  \begin{beamercolorbox}[sep=12pt,center]{section title}
    \usebeamerfont{section title}\insertsection\par
  \end{beamercolorbox}
}
\setbeamertemplate{subsection page}{
  \centering
  \begin{beamercolorbox}[sep=8pt,center]{subsection title}
    \usebeamerfont{subsection title}\insertsubsection\par
  \end{beamercolorbox}
}
% Prevent slide breaks in the middle of a paragraph
\widowpenalties 1 10000
\raggedbottom
\AtBeginPart{
  \frame{\partpage}
}
\AtBeginSection{
  \ifbibliography
  \else
    \frame{\sectionpage}
  \fi
}
\AtBeginSubsection{
  \frame{\subsectionpage}
}
\usepackage{iftex}
\ifPDFTeX
  \usepackage[T1]{fontenc}
  \usepackage[utf8]{inputenc}
  \usepackage{textcomp} % provide euro and other symbols
\else % if luatex or xetex
  \usepackage{unicode-math} % this also loads fontspec
  \defaultfontfeatures{Scale=MatchLowercase}
  \defaultfontfeatures[\rmfamily]{Ligatures=TeX,Scale=1}
\fi
\usepackage{lmodern}
\ifPDFTeX\else
  % xetex/luatex font selection
\fi
% Use upquote if available, for straight quotes in verbatim environments
\IfFileExists{upquote.sty}{\usepackage{upquote}}{}
\IfFileExists{microtype.sty}{% use microtype if available
  \usepackage[]{microtype}
  \UseMicrotypeSet[protrusion]{basicmath} % disable protrusion for tt fonts
}{}
\makeatletter
\@ifundefined{KOMAClassName}{% if non-KOMA class
  \IfFileExists{parskip.sty}{%
    \usepackage{parskip}
  }{% else
    \setlength{\parindent}{0pt}
    \setlength{\parskip}{6pt plus 2pt minus 1pt}}
}{% if KOMA class
  \KOMAoptions{parskip=half}}
\makeatother
\setlength{\emergencystretch}{3em} % prevent overfull lines
\providecommand{\tightlist}{%
  \setlength{\itemsep}{0pt}\setlength{\parskip}{0pt}}
\usepackage{bookmark}
\IfFileExists{xurl.sty}{\usepackage{xurl}}{} % add URL line breaks if available
\urlstyle{same}
\hypersetup{
  hidelinks,
  pdfcreator={LaTeX via pandoc}}

\author{\texorpdfstring{}{}}
\date{}

\begin{document}

\begin{frame}{Introduction}
\protect\phantomsection\label{sec:introduction}
Active Inference represents a paradigm shift in our understanding of
cognition, perception, and action. Originating from the Free Energy
Principle {[}@friston2010free{]}, Active Inference provides a unified
mathematical framework for understanding biological agents as systems
that minimize variational free energy through perception and action.
While the framework has been successfully applied to diverse domains
including neuroscience {[}@friston2012prediction{]}, psychiatry
{[}@friston2014active{]}, and artificial intelligence
{[}@tani2016exploring{]}, its fundamental nature as a meta-theoretical
methodology has remained underexplored.

\begin{block}{The Traditional View: Active Inference as Free Energy
Minimization}
\protect\phantomsection\label{the-traditional-view-active-inference-as-free-energy-minimization}
Conventionally, Active Inference is understood as a process where agents
act to fulfill prior preferences while gathering information about their
environment. The Expected Free Energy (EFE) formulation combines
epistemic and pragmatic terms:

\[\mathcal{F}(\pi) = \mathbb{E}_{q(s_\tau)}[\log q(s_\tau) - \log p(s_\tau \mid \pi)] + \mathbb{E}_{q(o_\tau)}[\log p(o_\tau \mid s_\tau) + \log p(s_\tau) - \log q(s_\tau)]\label{eq:efe}\]

The first term represents \emph{epistemic value} (information gain),
while the second represents \emph{pragmatic value} (goal achievement).
Action selection minimizes EFE, balancing exploration and exploitation.
\end{block}

\begin{block}{Beyond the Traditional View: Active Inference as
Meta-Methodology}
\protect\phantomsection\label{beyond-the-traditional-view-active-inference-as-meta-methodology}
Active Inference operates at a fundamentally meta-level. Rather than
simply providing another algorithm for decision-making, Active Inference
enables researchers to specify the very frameworks within which
cognition occurs. This meta-level operation manifests in two key
dimensions:

\begin{block}{Meta-Epistemic Aspect}
\protect\phantomsection\label{meta-epistemic-aspect}
Active Inference enables modelers to specify epistemic frameworks
through generative model matrices (A), (B), (C), and (D). Matrix (A)
defines observation likelihoods (P(o \mid s)), establishing what can be
known about the world. Matrix (D) defines prior beliefs (P(s)), setting
initial assumptions. Matrix (B) defines state transitions (P(s' \mid s,
a)), specifying causal relationships. Through these specifications,
researchers define not just current beliefs, but the epistemological
boundaries of cognition itself.
\end{block}

\begin{block}{Meta-Pragmatic Aspect}
\protect\phantomsection\label{meta-pragmatic-aspect}
Beyond epistemic specification, Active Inference enables meta-pragmatic
modeling through matrix (C), which defines preference priors. Unlike
traditional reinforcement learning where rewards are externally
specified, Active Inference enables modelers to specify pragmatic
landscapes. The modeler specifies what constitutes ``value'' for the
agent, enabling exploration of how different value systems shape
cognition and behavior.
\end{block}
\end{block}

\begin{block}{The (2 \times 2) Framework: Data/Meta-Data (\times)
Cognitive/Meta-Cognitive}
\protect\phantomsection\label{the-2-framework-datameta-data-cognitivemeta-cognitive}
To systematically analyze Active Inference's meta-level contributions,
we introduce a (2 \times 2) matrix framework (Figure
\ref{fig:quadrant_matrix}) with axes of Data/Meta-Data and
Cognitive/Meta-Cognitive processing.

\textbf{Data Processing (Cognitive Level):} Basic cognitive processing
of raw sensory data, implementing baseline pragmatic and epistemic
functionality through EFE minimization.

\textbf{Meta-Data Processing (Cognitive Level):} Processing that
incorporates meta-information (confidence scores, timestamps,
reliability metrics) to improve cognitive performance.

\textbf{Data Processing (Meta-Cognitive Level):} Reflective processing
where agents evaluate their own cognitive processes, implementing
self-monitoring and adaptive control.

\textbf{Meta-Data Processing (Meta-Cognitive Level):} Higher-order
reasoning involving meta-data about meta-cognition, enabling
framework-level adaptation and meta-theoretical analysis.
\end{block}

\begin{block}{Contributions and Implications}
\protect\phantomsection\label{contributions-and-implications}
This framework reveals Active Inference as a methodology that transcends
traditional approaches to cognition. By enabling meta-level
specification of epistemic and pragmatic frameworks, Active Inference
provides tools for understanding:

\begin{enumerate}
\tightlist
\item
  \textbf{Cognitive Architecture Design:} How different epistemic and
  pragmatic frameworks shape cognition
\item
  \textbf{Meta-Cognitive Processing:} Self-reflective cognitive
  mechanisms and their societal implications
\item
  \textbf{Cognitive Security:} Vulnerabilities arising from meta-level
  cognitive manipulation
\item
  \textbf{Unification of Cognitive Science:} Bridging biological and
  artificial cognition through shared principles
\end{enumerate}
\end{block}

\begin{block}{Paper Structure}
\protect\phantomsection\label{paper-structure}
Section \hyperlink{sec:methodology}{3} introduces the (2 \times 2)
matrix framework and demonstrates how Active Inference operates across
all four quadrants. Section \hyperlink{sec:experimental_results}{4}
provides conceptual demonstrations of each quadrant with mathematical
examples. Section \hyperlink{sec:discussion}{5} explores theoretical
implications and meta-level interpretations. Section
\hyperlink{sec:conclusion}{6} summarizes contributions and future
directions.

Supplemental materials provide mathematical derivations, additional
examples, and implementation details for the framework.
\end{block}
\end{frame}

\end{document}
