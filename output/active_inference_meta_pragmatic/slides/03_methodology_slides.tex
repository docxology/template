% Options for packages loaded elsewhere
\PassOptionsToPackage{unicode}{hyperref}
\PassOptionsToPackage{hyphens}{url}
\documentclass[
  ignorenonframetext,
]{beamer}
\newif\ifbibliography
\usepackage{pgfpages}
\setbeamertemplate{caption}[numbered]
\setbeamertemplate{caption label separator}{: }
\setbeamercolor{caption name}{fg=normal text.fg}
\beamertemplatenavigationsymbolsempty
% remove section numbering
\setbeamertemplate{part page}{
  \centering
  \begin{beamercolorbox}[sep=16pt,center]{part title}
    \usebeamerfont{part title}\insertpart\par
  \end{beamercolorbox}
}
\setbeamertemplate{section page}{
  \centering
  \begin{beamercolorbox}[sep=12pt,center]{section title}
    \usebeamerfont{section title}\insertsection\par
  \end{beamercolorbox}
}
\setbeamertemplate{subsection page}{
  \centering
  \begin{beamercolorbox}[sep=8pt,center]{subsection title}
    \usebeamerfont{subsection title}\insertsubsection\par
  \end{beamercolorbox}
}
% Prevent slide breaks in the middle of a paragraph
\widowpenalties 1 10000
\raggedbottom
\AtBeginPart{
  \frame{\partpage}
}
\AtBeginSection{
  \ifbibliography
  \else
    \frame{\sectionpage}
  \fi
}
\AtBeginSubsection{
  \frame{\subsectionpage}
}
\usepackage{iftex}
\ifPDFTeX
  \usepackage[T1]{fontenc}
  \usepackage[utf8]{inputenc}
  \usepackage{textcomp} % provide euro and other symbols
\else % if luatex or xetex
  \usepackage{unicode-math} % this also loads fontspec
  \defaultfontfeatures{Scale=MatchLowercase}
  \defaultfontfeatures[\rmfamily]{Ligatures=TeX,Scale=1}
\fi
\usepackage{lmodern}
\ifPDFTeX\else
  % xetex/luatex font selection
\fi
% Use upquote if available, for straight quotes in verbatim environments
\IfFileExists{upquote.sty}{\usepackage{upquote}}{}
\IfFileExists{microtype.sty}{% use microtype if available
  \usepackage[]{microtype}
  \UseMicrotypeSet[protrusion]{basicmath} % disable protrusion for tt fonts
}{}
\makeatletter
\@ifundefined{KOMAClassName}{% if non-KOMA class
  \IfFileExists{parskip.sty}{%
    \usepackage{parskip}
  }{% else
    \setlength{\parindent}{0pt}
    \setlength{\parskip}{6pt plus 2pt minus 1pt}}
}{% if KOMA class
  \KOMAoptions{parskip=half}}
\makeatother
\setlength{\emergencystretch}{3em} % prevent overfull lines
\providecommand{\tightlist}{%
  \setlength{\itemsep}{0pt}\setlength{\parskip}{0pt}}
\usepackage{bookmark}
\IfFileExists{xurl.sty}{\usepackage{xurl}}{} % add URL line breaks if available
\urlstyle{same}
\hypersetup{
  hidelinks,
  pdfcreator={LaTeX via pandoc}}

\author{\texorpdfstring{}{}}
\date{}

\begin{document}

\begin{frame}{Methodology}
\protect\phantomsection\label{sec:methodology}
This section presents the core methodological contribution: a (2
\times 2) matrix structure for understanding Active Inference as a
meta-pragmatic and meta-epistemic methodology. The structure organizes
cognitive processing along two dimensions: Data/Meta-Data and
Cognitive/Meta-Cognitive, revealing four distinct quadrants of cognitive
operation. Each quadrant represents a different combination of
processing level (cognitive vs.~meta-cognitive) and data type (data
vs.~meta-data), enabling systematic analysis of how Active Inference
operates across multiple levels of abstraction.

\begin{block}{The (2 \times 2) Matrix Framework}
\protect\phantomsection\label{the-2-matrix-framework}
Active Inference's meta-level operation becomes apparent when analyzed
through a structure that distinguishes between data processing and
meta-data processing, as well as cognitive and meta-cognitive levels of
operation (Figure \ref{fig:quadrant_matrix}). This (2 \times 2)
organization reveals how Active Inference operates simultaneously across
multiple levels, from basic data processing to framework-level
reasoning.

\begin{block}{Framework Dimensions}
\protect\phantomsection\label{framework-dimensions}
\textbf{Data vs Meta-Data (X-axis):} - \textbf{Data:} Raw sensory inputs
and immediate cognitive processing - \textbf{Meta-Data:} Information
about data processing (confidence scores, timestamps, reliability
metrics, processing provenance)

\textbf{Cognitive vs Meta-Cognitive (Y-axis):} - \textbf{Cognitive:}
Direct processing and transformation of information -
\textbf{Meta-Cognitive:} Processing about processing; self-reflection,
monitoring, and control of cognitive processes
\end{block}

\begin{block}{Quadrant Definitions}
\protect\phantomsection\label{quadrant-definitions}
\begin{block}{Quadrant 1: Data Processing (Cognitive)}
\protect\phantomsection\label{sec:q1_definition}
\textbf{Definition:} Basic cognitive processing of raw sensory data at
the fundamental level of cognition, where agents directly process
observations without incorporating quality information or
self-reflection.

\textbf{Active Inference Role:} Baseline pragmatic and epistemic
processing through Expected Free Energy minimization, providing the
foundation upon which all other quadrants build. This quadrant
implements the core Active Inference mechanism in its simplest form.

\textbf{Mathematical Formulation:}

\[\mathcal{F}(\pi) = G(\pi) + H[Q(\pi)]\label{eq:efe_simple}\]

Where (G(\pi)) represents pragmatic value (goal achievement) and
(H{[}Q(\pi){]}) represents epistemic affordance (information gain). This
formulation is shown in Equation \eqref{eq:efe_simple}.

\textbf{Example:} A thermostat maintaining temperature through direct
sensor readings and immediate action selection. The thermostat processes
temperature data at the cognitive level, selecting heating or cooling
actions to minimize EFE. The pragmatic component (G(\pi)) reflects the
preference for comfortable temperature (minimizing deviation from
setpoint), while the epistemic component (H{[}Q(\pi){]}) captures
information gain about environmental conditions (learning whether the
environment is heating or cooling). This basic operation demonstrates
the fundamental balance between goal achievement and information
gathering that characterizes all Active Inference systems.
\end{block}

\begin{block}{Quadrant 2: Meta-Data Organization (Cognitive)}
\protect\phantomsection\label{sec:q2_definition}
\textbf{Definition:} Cognitive processing that incorporates meta-data
(information about data quality, reliability, and provenance) to enhance
primary data processing, improving decision reliability beyond what
basic data processing can achieve.

\textbf{Active Inference Role:} Enhanced epistemic and pragmatic
processing through meta-data integration, extending Quadrant 1
operations by weighting observations and inferences based on quality
information.

\textbf{Mathematical Formulation:} Extended EFE with meta-data
weighting:

\[\mathcal{F}(\pi) = w_e \cdot H[Q(\pi)] + w_p \cdot G(\pi) + w_m \cdot M(\pi)\label{eq:efe_metadata}\]

Where (M(\pi)) represents meta-data derived utility, (w\_e) is the
epistemic weight, (w\_p) is the pragmatic weight, and (w\_m) is the
meta-data weight. These weights adapt based on context and processing
requirements. This extended formulation is shown in Equation
\eqref{eq:efe_metadata}.

\textbf{Example:} Processing sensory data with associated confidence
scores and temporal metadata to improve decision reliability. For
instance, a navigation system might weight GPS readings by signal
strength (confidence (c(t))) and temporal consistency (meta-data
(\tau(t))), giving less weight to readings that conflict with recent
measurements or have low signal quality. The meta-data weight (w\_m)
adapts dynamically: when confidence is high and temporal consistency is
good, (w\_m) increases, allowing the system to rely more heavily on the
meta-data enhanced inference. This adaptive weighting demonstrates how
meta-data integration improves cognitive performance beyond basic data
processing.
\end{block}

\begin{block}{Quadrant 3: Reflective Processing (Meta-Cognitive)}
\protect\phantomsection\label{sec:q3_definition}
\textbf{Definition:} Meta-cognitive evaluation and control of data
processing, where agents reflect on their own cognitive processes,
assess inference quality, and adaptively adjust processing strategies
based on self-assessment.

\textbf{Active Inference Role:} Self-monitoring and adaptive cognitive
control through hierarchical EFE evaluation, enabling systems to
regulate their own cognitive operations based on confidence and
performance assessment.

\textbf{Mathematical Formulation:} Hierarchical EFE with
self-assessment:

\[\mathcal{F}(\pi) = \mathcal{F}_{primary}(\pi) + \lambda \cdot \mathcal{F}_{meta}(\pi)\label{eq:efe_hierarchical}\]

Where (\mathcal{F}\_\{meta\}) evaluates the quality of primary
processing and (\lambda) controls meta-cognitive influence. The
hierarchical structure enables self-monitoring by evaluating primary
cognitive processes through meta-level assessment. This hierarchical
formulation is shown in Equation \eqref{eq:efe_hierarchical}.

\textbf{Example:} An agent assessing its confidence in inferences and
adjusting processing strategies accordingly. When confidence drops below
threshold (\gamma), the meta-cognitive control parameter (\lambda)
increases, amplifying the influence of (\mathcal{F}\emph{\{meta\}(\pi))
in policy selection. This triggers adaptive responses: allocating more
computational resources to inference, switching to more conservative
decision-making strategies, or seeking additional information before
acting. The hierarchical structure (\mathcal{F}(\pi) =
\mathcal{F}}\{primary\}(\pi) + \lambda \cdot \mathcal{F}\_\{meta\}(\pi))
enables the system to monitor and regulate its own cognitive processes,
demonstrating self-awareness and adaptive control.
\end{block}

\begin{block}{Quadrant 4: Higher-Order Reasoning (Meta-Cognitive)}
\protect\phantomsection\label{sec:q4_definition}
\textbf{Definition:} Meta-cognitive processing of meta-data about
cognition itself, where systems analyze patterns in their own
meta-cognitive performance to optimize fundamental framework parameters,
enabling recursive self-analysis at the highest level of cognitive
abstraction.

\textbf{Active Inference Role:} Framework-level reasoning and
meta-theoretical analysis through parameter optimization, allowing
systems to evolve their cognitive architectures based on higher-order
performance analysis.

\textbf{Mathematical Formulation:} Multi-level hierarchical
optimization:

\[\min_{\Theta} \mathcal{F}(\pi; \Theta) + \mathcal{R}(\Theta)\label{eq:framework_optimization}\]

Where (\Theta) represents framework parameters and (\mathcal{R}(\Theta))
is a regularization term ensuring framework coherence. This optimization
enables the system to adapt its fundamental cognitive architecture
through recursive self-analysis. The framework optimization is shown in
Equation \eqref{eq:framework_optimization}.

\textbf{Example:} Analyzing patterns in meta-cognitive performance to
adapt fundamental processing frameworks. A system might observe that its
confidence assessments are consistently miscalibrated (average
confidence (\bar\{c\}) deviates from actual accuracy), leading it to
optimize framework parameters (\Theta) including confidence thresholds
(\theta\emph{c), adaptation rates (\alpha), and strategy selection
parameters (\beta). The optimization (\min}\{\Theta\} \mathcal{F}(\pi;
\Theta) + \mathcal{R}(\Theta)) balances performance improvement with
framework coherence, ensuring that parameter changes maintain system
stability. This recursive self-analysis enables the system to evolve its
fundamental cognitive architecture, representing the highest level of
meta-cognitive operation.
\end{block}
\end{block}
\end{block}

\begin{block}{Active Inference as Meta-Epistemic}
\protect\phantomsection\label{active-inference-as-meta-epistemic}
Active Inference supports meta-epistemic modeling by allowing
researchers to specify the epistemological frameworks within which
agents operate. This specification power transforms epistemology from an
external constraint into an internal design parameter, enabling
systematic exploration of how different ways of knowing shape cognition.

\begin{block}{Epistemic Framework Specification}
\protect\phantomsection\label{epistemic-framework-specification}
Through the generative model matrices, researchers define:

\textbf{Observation Model (Matrix (A)):} What can be known about the
world

\[A = [a_{ij}] \quad a_{ij} = P(o_i \mid s_j)\label{eq:matrix_a}\]

\textbf{Prior Knowledge (Matrix (D)):} Initial assumptions about the
world

\[D = [d_i] \quad d_i = P(s_i)\label{eq:matrix_d}\]

\textbf{Causal Structure (Matrix (B)):} How actions influence the world

\[B = [b_{ijk}] \quad b_{ijk} = P(s_j \mid s_i, a_k)\label{eq:matrix_b}\]
\end{block}

\begin{block}{Meta-Epistemic Implications}
\protect\phantomsection\label{meta-epistemic-implications}
By specifying these matrices, researchers define not just current
beliefs, but the fundamental structure of knowledge acquisition and
representation. This meta-epistemic power supports systematic
exploration of epistemological questions that were previously difficult
to formalize:

\begin{enumerate}
\item
  \textbf{Framework Comparison:} Epistemic frameworks can be compared by
  varying (A), (B), (D) specifications. For example, comparing empirical
  frameworks (high diagonal values in (A), indicating strong trust in
  observations) versus skeptical frameworks (lower diagonal values,
  indicating greater uncertainty) reveals how different assumptions
  about observation reliability shape knowledge acquisition strategies.
  Researchers can systematically explore how different epistemic
  assumptions lead to different cognitive behaviors, learning speeds,
  and adaptation patterns. This comparative approach enables formal
  analysis of epistemological questions that were previously limited to
  philosophical discourse.
\item
  \textbf{Knowledge Architecture Design:} The structure of cognition
  itself becomes a design parameter. Researchers can design knowledge
  architectures optimized for specific tasks, environments, or
  constraints, exploring how different epistemic structures enable or
  constrain cognitive capabilities. For instance, designing an (A)
  matrix with high off-diagonal values creates an epistemic framework
  that maintains uncertainty longer, requiring more evidence before
  committing to beliefs---useful for environments with high observation
  noise.
\item
  \textbf{Epistemological Pluralism:} Different ways of knowing can be
  modeled and compared within the same mathematical structure. This
  supports systematic analysis of how different epistemic approaches
  (empirical, theoretical, intuitive) lead to different cognitive
  outcomes, providing a formal basis for epistemological analysis. The
  structure allows researchers to explore questions like: How does an
  empirical epistemic framework (high observation reliability) compare
  to a theoretical framework (strong prior structure in (D)) in terms of
  learning speed, robustness, and adaptability? This pluralistic
  approach enables formal comparison of epistemological traditions that
  were previously considered incommensurable.
\end{enumerate}
\end{block}
\end{block}

\begin{block}{Active Inference as Meta-Pragmatic}
\protect\phantomsection\label{active-inference-as-meta-pragmatic}
Active Inference supports meta-pragmatic modeling by allowing
specification of pragmatic frameworks beyond simple reward functions.
This specification power transforms value system design from an external
constraint into an internal research question, enabling systematic
exploration of how different value structures shape cognition and
behavior.

\begin{block}{Pragmatic Framework Specification}
\protect\phantomsection\label{pragmatic-framework-specification}
\textbf{Preference Structure (Matrix (C)):} What matters to the agent

\[C = [c_i] \quad c_i = \log P(o_i)\label{eq:matrix_c}\]

This specification goes beyond traditional reinforcement learning by
enabling researchers to specify value landscapes.
\end{block}

\begin{block}{Meta-Pragmatic Implications}
\protect\phantomsection\label{meta-pragmatic-implications}
The meta-pragmatic aspect supports systematic exploration of value
systems and their cognitive consequences, making value system design a
research question:

\begin{enumerate}
\item
  \textbf{Value System Design:} Specification of what constitutes
  ``good'' outcomes through matrix (C) enables researchers to design
  value systems optimized for specific goals, constraints, or ethical
  principles. This goes beyond simple reward functions to enable complex
  value hierarchies with trade-offs. For example, a (C) matrix can
  encode preferences that balance individual benefit with collective
  welfare, or that prioritize long-term sustainability over short-term
  gains, revealing how different value structures shape decision-making
  patterns.
\item
  \textbf{Pragmatic Pluralism:} Different pragmatic frameworks can be
  explored and compared. Researchers can model how different value
  systems (utilitarian, deontological, virtue-based) shape
  decision-making and behavior, revealing the cognitive consequences of
  different ethical frameworks. A utilitarian (C) matrix might
  prioritize outcomes that maximize aggregate utility, while a
  deontological matrix might encode categorical imperatives, enabling
  systematic comparison of how these different ethical approaches lead
  to different cognitive and behavioral patterns.
\item
  \textbf{Value Learning:} How value systems themselves evolve and adapt
  can be modeled through learning mechanisms that update matrix (C)
  based on experience, feedback, or reflection. This supports
  exploration of how agents develop their own value systems over time,
  modeling processes like moral development, cultural value acquisition,
  and personal preference formation. The structure allows researchers to
  study how value systems change in response to experience, social
  influence, or self-reflection, providing formal models of value system
  dynamics that were previously difficult to formalize.
\item
  \textbf{Ethical Framework Integration:} Incorporation of ethical
  considerations directly into the preference landscape enables agents
  to reason about moral implications of actions. This provides a formal
  framework for exploring how ethical principles shape cognition and
  behavior, enabling systematic analysis of questions like: How do
  different ethical frameworks (encoded in (C)) influence
  decision-making under uncertainty? How do value conflicts (competing
  preferences in (C)) get resolved?
\end{enumerate}
\end{block}
\end{block}

\begin{block}{Integration Across Quadrants}
\protect\phantomsection\label{integration-across-quadrants}
Active Inference operates across all four quadrants simultaneously, with
different aspects of the structure contributing to each quadrant. This
integration creates a hierarchical cognitive architecture where lower
quadrants provide foundations for higher quadrants:

\begin{itemize}
\tightlist
\item
  \textbf{Quadrant 1 (Foundation):} Core EFE computation with basic (A),
  (B), (C), (D) specifications provides the fundamental cognitive
  processing layer
\item
  \textbf{Quadrant 2 (Enhancement):} Meta-data weighted EFE with
  confidence-weighted processing enhances Quadrant 1 operations by
  incorporating quality information
\item
  \textbf{Quadrant 3 (Reflection):} Self-reflective EFE evaluation and
  meta-cognitive control monitors and regulates Quadrants 1 and 2,
  enabling adaptive strategy selection
\item
  \textbf{Quadrant 4 (Evolution):} Framework-level EFE optimization and
  meta-theoretical reasoning analyzes patterns across all lower
  quadrants to evolve the cognitive architecture itself
\end{itemize}

This hierarchical structure enables systems to operate at multiple
levels simultaneously: processing data (Q1), incorporating meta-data
(Q2), reflecting on processing quality (Q3), and evolving framework
parameters (Q4). The relative influence of each quadrant adapts
dynamically based on context, uncertainty, and performance requirements.
\end{block}

\begin{block}{The Modeler's Dual Role}
\protect\phantomsection\label{the-modelers-dual-role}
The structure reveals the dual role of the Active Inference modeler, who
operates at both the cognitive and meta-cognitive levels:

\begin{block}{As Architect}
\protect\phantomsection\label{as-architect}
\begin{itemize}
\tightlist
\item
  Specifies epistemic frameworks ((A), (B), (D) matrices)
\item
  Defines pragmatic landscapes ((C) matrix)
\item
  Designs cognitive architectures
\item
  Establishes boundary conditions for cognition
\end{itemize}
\end{block}

\begin{block}{As Subject}
\protect\phantomsection\label{as-subject}
\begin{itemize}
\tightlist
\item
  Uses Active Inference to understand their own cognition
\item
  Applies meta-epistemic principles to knowledge acquisition
\item
  Employs meta-pragmatic frameworks for decision-making
\item
  Engages in recursive self-modeling
\end{itemize}

This dual role creates a recursive relationship where the tools used to
model others become tools for self-understanding. The modeler, in
specifying frameworks for studied systems, implicitly reveals their own
epistemic and pragmatic assumptions. This recursive self-modeling
enables researchers to apply Active Inference principles to understand
their own cognitive processes, creating a virtuous cycle where modeling
improves self-understanding, which in turn improves modeling
capabilities.
\end{block}
\end{block}

\begin{block}{Validation Approach}
\protect\phantomsection\label{validation-approach}
The structure's validity is demonstrated through multiple complementary
approaches that together provide strong theoretical and practical
support:

\begin{enumerate}
\item
  \textbf{Theoretical Consistency:} Alignment with Free Energy Principle
  foundations ensures that all quadrant formulations minimize
  variational free energy at their respective levels, maintaining
  theoretical coherence with established Active Inference principles.
\item
  \textbf{Mathematical Rigor:} Proper formulation of EFE across all
  quadrants, with each quadrant's mathematical structure building
  systematically on previous quadrants. All formulations are grounded in
  established Active Inference theory and maintain probabilistic
  consistency.
\item
  \textbf{Conceptual Clarity:} Clear distinction between quadrants and
  processing levels enables systematic analysis. The (2 \times 2)
  structure provides unambiguous categorization of cognitive operations,
  facilitating both theoretical analysis and experimental design.
\item
  \textbf{Practical Applicability:} Framework enables systematic
  analysis of meta-level phenomena that were previously difficult to
  formalize. The quadrant structure provides a practical tool for
  researchers to target specific processing levels in experimental
  design and theoretical development.
\end{enumerate}

The following sections provide concrete demonstrations of each quadrant
with mathematical examples and conceptual analysis, showing how the
structure applies to real cognitive scenarios. These demonstrations
reveal the practical utility of the quadrant organization for
understanding, designing, and improving cognitive systems.

The (2 \times 2) matrix structure, illustrated in Figure
\ref{fig:quadrant_matrix}, organizes our analysis of Active Inference as
a meta-pragmatic and meta-epistemic methodology. This structure reveals
four distinct quadrants of cognitive operation, each representing
different combinations of data/meta-data processing and
cognitive/meta-cognitive levels, enabling systematic exploration of how
Active Inference operates across multiple scales of cognitive
abstraction.

The Expected Free Energy (EFE) formulation (Equation \eqref{eq:efe})
combines epistemic and pragmatic components, as shown in Figure
\ref{fig:efe_decomposition}. This decomposition reveals how Active
Inference balances information gathering (epistemic value, Equation
\eqref{eq:epistemic_component}) with goal achievement (pragmatic value,
Equation \eqref{eq:pragmatic_component}) in a principled mathematical
framework.

\begin{figure}[h]
\centering
\includegraphics[width=0.8\textwidth]{../figures/efe_decomposition.png}
\caption{Expected Free Energy (EFE) decomposition into epistemic and pragmatic components (Equation \eqref{eq:efe}). The EFE \(\mathcal{F}(\pi)\) combines two fundamental terms: (1) Epistemic affordance \(H[Q(\pi)]\) (Equation \eqref{eq:epistemic_component}), measuring information gain about hidden states through policy execution; (2) Pragmatic value \(G(\pi)\) (Equation \eqref{eq:pragmatic_component}), measuring goal achievement through preferred observations. This decomposition enables systematic analysis of how agents balance exploration (epistemic) and exploitation (pragmatic) in decision-making.}
\label{fig:efe_decomposition}
\end{figure}

The perception-action loop in Active Inference, illustrated in Figure
\ref{fig:perception_action_loop}, demonstrates how agents continuously
update beliefs and select actions to minimize expected free energy.

\begin{figure}[h]
\centering
\includegraphics[width=0.8\textwidth]{../figures/perception_action_loop.png}
\caption{Active Inference perception-action loop showing how perception drives action through EFE minimization (Equation \eqref{eq:efe}). The cycle consists of: (1) Observation of sensory data; (2) Bayesian inference updating posterior beliefs \(q(s)\) about hidden states; (3) Policy evaluation computing EFE \(\mathcal{F}(\pi)\) for candidate actions; (4) Action selection minimizing EFE; (5) Action execution generating new observations. This closed loop enables agents to actively shape their sensory input while maintaining accurate world models.}
\label{fig:perception_action_loop}
\end{figure}

The generative model structure, shown in Figure
\ref{fig:generative_model_structure}, illustrates how the four core
matrices ((A), (B), (C), (D)) define the epistemic and pragmatic
frameworks within which agents operate. These matrices are defined in
Equations \eqref{eq:matrix_a}, \eqref{eq:matrix_b}, \eqref{eq:matrix_c},
and \eqref{eq:matrix_d}.

\begin{figure}[h]
\centering
\includegraphics[width=0.8\textwidth]{../figures/generative_model_structure.png}
\caption{Structure of generative models in Active Inference showing \(A\), \(B\), \(C\), \(D\) matrices and their relationships. Matrix \(A\) (Equation \eqref{eq:matrix_a}) defines observation likelihoods \(P(o \mid s)\), establishing what can be known. Matrix \(B\) (Equation \eqref{eq:matrix_b}) defines state transitions \(P(s' \mid s, a)\), specifying causal structure. Matrix \(C\) (Equation \eqref{eq:matrix_c}) defines preferences over observations, establishing pragmatic goals. Matrix \(D\) (Equation \eqref{eq:matrix_d}) defines prior beliefs \(P(s)\), setting initial assumptions. Together, these matrices enable modelers to specify the fundamental frameworks of cognition.}
\label{fig:generative_model_structure}
\end{figure}

The meta-level aspects of Active Inference, demonstrated in Figure
\ref{fig:meta_level_concepts}, reveal how modelers specify both
epistemic and pragmatic frameworks, transcending traditional approaches
to cognition.

\begin{figure}[h]
\centering
\includegraphics[width=0.8\textwidth]{../figures/meta_level_concepts.png}
\caption{Meta-pragmatic and meta-epistemic aspects showing modeler specification power. The meta-epistemic dimension enables specification of knowledge acquisition frameworks through matrices \(A\) (Equation \eqref{eq:matrix_a}), \(B\) (Equation \eqref{eq:matrix_b}), and \(D\) (Equation \eqref{eq:matrix_d}), defining what can be known and how beliefs update. The meta-pragmatic dimension enables specification of value landscapes through matrix \(C\) (Equation \eqref{eq:matrix_c}), defining what matters to the agent. This dual specification power makes Active Inference a meta-methodology for cognitive science, enabling exploration of how different frameworks shape cognition.}
\label{fig:meta_level_concepts}
\end{figure}

The Free Energy Principle provides the theoretical foundation for Active
Inference, as illustrated in Figure \ref{fig:fep_system_boundaries},
showing how systems maintain their structure through boundary
maintenance.

\begin{figure}[h]
\centering
\includegraphics[width=0.8\textwidth]{../figures/fep_system_boundaries.png}
\caption{Free Energy Principle system boundaries showing Markov blanket separating internal and external states. The Markov blanket defines the boundary between a system (internal states) and its environment (external states) through sensory and active states. Systems maintain their structure by minimizing variational free energy \(\mathcal{F}[q]\), which bounds surprise. This principle applies across multiple scales: physical systems maintain boundaries through thermodynamic processes, cognitive systems maintain beliefs through inference, and meta-cognitive systems maintain frameworks through adaptation.}
\label{fig:fep_system_boundaries}
\end{figure}

Free energy minimization dynamics, shown in Figure
\ref{fig:free_energy_dynamics}, demonstrate how systems converge toward
stable states through continuous optimization.

\begin{figure}[h]
\centering
\includegraphics[width=0.8\textwidth]{../figures/free_energy_dynamics.png}
\caption{Free energy minimization dynamics showing convergence over time and epistemic/pragmatic components. The trajectory shows how variational free energy \(\mathcal{F}[q]\) decreases over time as the system updates its beliefs and actions. The decomposition reveals the relative contributions of epistemic (information gain) and pragmatic (goal achievement) components. Convergence indicates successful model fitting and goal achievement, while divergence may signal model inadequacy or goal conflict.}
\label{fig:free_energy_dynamics}
\end{figure}

Structure preservation, illustrated in Figure
\ref{fig:structure_preservation}, shows how systems maintain their
internal organization despite environmental perturbations.

\begin{figure}[h]
\centering
\includegraphics[width=0.8\textwidth]{../figures/structure_preservation.png}
\caption{Structure preservation dynamics showing how systems maintain internal organization through free energy minimization. Despite external perturbations and environmental changes, systems maintain stable internal states through active inference. The Markov blanket enables selective coupling with the environment, allowing systems to resist entropy while remaining responsive to relevant information. This principle explains how biological systems, cognitive agents, and even social structures maintain their identity over time.}
\label{fig:structure_preservation}
\end{figure}

The enhanced quadrant matrix, shown in Figure
\ref{fig:quadrant_matrix_enhanced}, provides detailed descriptions and
examples for each quadrant, facilitating systematic analysis of
cognitive processes.

\begin{figure}[h]
\centering
\includegraphics[width=0.8\textwidth]{../figures/quadrant_matrix_enhanced.png}
\caption{\(2 \times 2\) Quadrant Structure with detailed descriptions and examples for each quadrant. Quadrant 1 (Data, Cognitive): Basic EFE computation (Equation \eqref{eq:efe_simple}) with direct sensory processing, providing the foundation for all cognitive operations. Quadrant 2 (Meta-Data, Cognitive): Extended EFE with meta-data weighting (Equation \eqref{eq:efe_metadata}), enhancing processing through confidence scores and reliability metrics. Quadrant 3 (Data, Meta-Cognitive): Hierarchical EFE with self-assessment (Equation \eqref{eq:efe_hierarchical}), enabling self-reflective processing with confidence assessment and adaptive control. Quadrant 4 (Meta-Data, Meta-Cognitive): Framework-level optimization (Equation \eqref{eq:framework_optimization}), supporting reasoning about meta-cognitive processes with parameter optimization. Each quadrant includes mathematical formulations, practical examples (thermostat, navigation system, adaptive agent, framework-evolving system), and connections to Active Inference theory, demonstrating the hierarchical relationship between quadrants.}
\label{fig:quadrant_matrix_enhanced}
\end{figure}

The Free Energy Principle serves as a unifying bridge between physics
and cognition, as demonstrated in Figure
\ref{fig:physics_cognition_bridge}, revealing deep connections across
domains.

\begin{figure}[h]
\centering
\includegraphics[width=0.8\textwidth]{../figures/physics_cognition_bridge.png}
\caption{Free Energy Principle as the bridge between physics and cognition domains. The same mathematical principle—variational free energy minimization—applies across multiple scales: (1) Physical systems minimize thermodynamic free energy, maintaining structure through energy flows; (2) Biological systems minimize variational free energy, maintaining organization through metabolism and behavior; (3) Cognitive systems minimize expected free energy, maintaining accurate world models through perception and action; (4) Meta-cognitive systems minimize framework-level free energy, maintaining adaptive cognitive architectures through self-reflection. This unification enables understanding of intelligence as a natural extension of physical principles.}
\label{fig:physics_cognition_bridge}
\end{figure}
\end{block}
\end{frame}

\end{document}
