% Options for packages loaded elsewhere
\PassOptionsToPackage{unicode}{hyperref}
\PassOptionsToPackage{hyphens}{url}
\documentclass[
  ignorenonframetext,
]{beamer}
\newif\ifbibliography
\usepackage{pgfpages}
\setbeamertemplate{caption}[numbered]
\setbeamertemplate{caption label separator}{: }
\setbeamercolor{caption name}{fg=normal text.fg}
\beamertemplatenavigationsymbolsempty
% remove section numbering
\setbeamertemplate{part page}{
  \centering
  \begin{beamercolorbox}[sep=16pt,center]{part title}
    \usebeamerfont{part title}\insertpart\par
  \end{beamercolorbox}
}
\setbeamertemplate{section page}{
  \centering
  \begin{beamercolorbox}[sep=12pt,center]{section title}
    \usebeamerfont{section title}\insertsection\par
  \end{beamercolorbox}
}
\setbeamertemplate{subsection page}{
  \centering
  \begin{beamercolorbox}[sep=8pt,center]{subsection title}
    \usebeamerfont{subsection title}\insertsubsection\par
  \end{beamercolorbox}
}
% Prevent slide breaks in the middle of a paragraph
\widowpenalties 1 10000
\raggedbottom
\AtBeginPart{
  \frame{\partpage}
}
\AtBeginSection{
  \ifbibliography
  \else
    \frame{\sectionpage}
  \fi
}
\AtBeginSubsection{
  \frame{\subsectionpage}
}
\usepackage{iftex}
\ifPDFTeX
  \usepackage[T1]{fontenc}
  \usepackage[utf8]{inputenc}
  \usepackage{textcomp} % provide euro and other symbols
\else % if luatex or xetex
  \usepackage{unicode-math} % this also loads fontspec
  \defaultfontfeatures{Scale=MatchLowercase}
  \defaultfontfeatures[\rmfamily]{Ligatures=TeX,Scale=1}
\fi
\usepackage{lmodern}
\ifPDFTeX\else
  % xetex/luatex font selection
\fi
% Use upquote if available, for straight quotes in verbatim environments
\IfFileExists{upquote.sty}{\usepackage{upquote}}{}
\IfFileExists{microtype.sty}{% use microtype if available
  \usepackage[]{microtype}
  \UseMicrotypeSet[protrusion]{basicmath} % disable protrusion for tt fonts
}{}
\makeatletter
\@ifundefined{KOMAClassName}{% if non-KOMA class
  \IfFileExists{parskip.sty}{%
    \usepackage{parskip}
  }{% else
    \setlength{\parindent}{0pt}
    \setlength{\parskip}{6pt plus 2pt minus 1pt}}
}{% if KOMA class
  \KOMAoptions{parskip=half}}
\makeatother
\setlength{\emergencystretch}{3em} % prevent overfull lines
\providecommand{\tightlist}{%
  \setlength{\itemsep}{0pt}\setlength{\parskip}{0pt}}
\usepackage{bookmark}
\IfFileExists{xurl.sty}{\usepackage{xurl}}{} % add URL line breaks if available
\urlstyle{same}
\hypersetup{
  hidelinks,
  pdfcreator={LaTeX via pandoc}}

\author{\texorpdfstring{}{}}
\date{}

\begin{document}

\begin{frame}{Methodology}
\protect\phantomsection\label{sec:methodology}
This section presents the core methodological contribution: a (2
\times 2) matrix framework for understanding Active Inference as a
meta-pragmatic and meta-epistemic methodology. The framework structures
cognitive processing along two dimensions: Data/Meta-Data and
Cognitive/Meta-Cognitive, revealing four distinct quadrants of cognitive
operation.

\begin{block}{The (2 \times 2) Matrix Framework}
\protect\phantomsection\label{the-2-matrix-framework}
Active Inference's meta-level operation becomes apparent when analyzed
through a framework that distinguishes between data processing and
meta-data processing, as well as cognitive and meta-cognitive levels of
operation (Figure \ref{fig:quadrant_matrix}).

\begin{block}{Framework Dimensions}
\protect\phantomsection\label{framework-dimensions}
\textbf{Data vs Meta-Data (X-axis):} - \textbf{Data:} Raw sensory inputs
and immediate cognitive processing - \textbf{Meta-Data:} Information
about data processing (confidence scores, timestamps, reliability
metrics, processing provenance)

\textbf{Cognitive vs Meta-Cognitive (Y-axis):} - \textbf{Cognitive:}
Direct processing and transformation of information -
\textbf{Meta-Cognitive:} Processing about processing; self-reflection,
monitoring, and control of cognitive processes
\end{block}

\begin{block}{Quadrant Definitions}
\protect\phantomsection\label{quadrant-definitions}
\begin{block}{Quadrant 1: Data Processing (Cognitive)}
\protect\phantomsection\label{sec:q1_definition}
\textbf{Definition:} Basic cognitive processing of raw sensory data at
the fundamental level of cognition.

\textbf{Active Inference Role:} Baseline pragmatic and epistemic
processing through Expected Free Energy minimization.

\textbf{Mathematical Formulation:}

\[\mathcal{F}(\pi) = G(\pi) + H[Q(\pi)]\label{eq:efe_simple}\]

Where (G(\pi)) represents pragmatic value (goal achievement) and
(H{[}Q(\pi){]}) represents epistemic affordance (information gain).

\textbf{Example:} A thermostat maintaining temperature through direct
sensor readings and immediate action selection.
\end{block}

\begin{block}{Quadrant 2: Meta-Data Organization (Cognitive)}
\protect\phantomsection\label{sec:q2_definition}
\textbf{Definition:} Cognitive processing that incorporates meta-data to
enhance primary processing.

\textbf{Active Inference Role:} Epistemic processing through meta-data
integration.

\textbf{Mathematical Formulation:} Extended EFE with meta-data
weighting:

\[\mathcal{F}(\pi) = w_e \cdot H[Q(\pi)] + w_p \cdot G(\pi) + w_m \cdot M(\pi)\label{eq:efe_metadata}\]

Where (M(\pi)) represents meta-data derived utility and (w) terms are
adaptive weights.

\textbf{Example:} Processing sensory data with associated confidence
scores and temporal metadata to improve decision reliability.
\end{block}

\begin{block}{Quadrant 3: Reflective Processing (Meta-Cognitive)}
\protect\phantomsection\label{sec:q3_definition}
\textbf{Definition:} Meta-cognitive evaluation and control of data
processing.

\textbf{Active Inference Role:} Self-monitoring and adaptive cognitive
control.

\textbf{Mathematical Formulation:} Hierarchical EFE with
self-assessment:

\[\mathcal{F}(\pi) = \mathcal{F}_{primary}(\pi) + \lambda \cdot \mathcal{F}_{meta}(\pi)\label{eq:efe_hierarchical}\]

Where (\mathcal{F}\_\{meta\}) evaluates the quality of primary
processing and (\lambda) controls meta-cognitive influence.

\textbf{Example:} An agent assessing its confidence in inferences and
adjusting processing strategies accordingly.
\end{block}

\begin{block}{Quadrant 4: Higher-Order Reasoning (Meta-Cognitive)}
\protect\phantomsection\label{sec:q4_definition}
\textbf{Definition:} Meta-cognitive processing of meta-data about
cognition.

\textbf{Active Inference Role:} Framework-level reasoning and
meta-theoretical analysis.

\textbf{Mathematical Formulation:} Multi-level hierarchical
optimization:

\[\min_{\Theta} \mathcal{F}(\pi; \Theta) + \mathcal{R}(\Theta)\label{eq:framework_optimization}\]

Where (\Theta) represents framework parameters and (\mathcal{R}) is a
regularization term ensuring framework coherence.

\textbf{Example:} Analyzing patterns in meta-cognitive performance to
adapt fundamental processing frameworks.
\end{block}
\end{block}
\end{block}

\begin{block}{Active Inference as Meta-Epistemic}
\protect\phantomsection\label{active-inference-as-meta-epistemic}
Active Inference enables meta-epistemic modeling by allowing researchers
to specify the epistemological frameworks within which agents operate.

\begin{block}{Epistemic Framework Specification}
\protect\phantomsection\label{epistemic-framework-specification}
Through the generative model matrices, researchers define:

\textbf{Observation Model (Matrix (A)):} What can be known about the
world

\[A = [a_{ij}] \quad a_{ij} = P(o_i \mid s_j)\label{eq:matrix_a}\]

\textbf{Prior Knowledge (Matrix (D)):} Initial assumptions about the
world

\[D = [d_i] \quad d_i = P(s_i)\label{eq:matrix_d}\]

\textbf{Causal Structure (Matrix (B)):} How actions influence the world

\[B = [b_{ijk}] \quad b_{ijk} = P(s_j \mid s_i, a_k)\label{eq:matrix_b}\]
\end{block}

\begin{block}{Meta-Epistemic Implications}
\protect\phantomsection\label{meta-epistemic-implications}
By specifying these matrices, researchers define not just current
beliefs, but the fundamental structure of knowledge acquisition and
representation. This meta-epistemic power enables:

\begin{enumerate}
\tightlist
\item
  \textbf{Framework Comparison:} Epistemic frameworks can be compared by
  varying (A), (B), (D) specifications
\item
  \textbf{Knowledge Architecture Design:} The structure of cognition
  itself becomes a design parameter
\item
  \textbf{Epistemological Pluralism:} Different ways of knowing can be
  modeled and compared
\end{enumerate}
\end{block}
\end{block}

\begin{block}{Active Inference as Meta-Pragmatic}
\protect\phantomsection\label{active-inference-as-meta-pragmatic}
Active Inference enables meta-pragmatic modeling by allowing
specification of pragmatic frameworks beyond simple reward functions.

\begin{block}{Pragmatic Framework Specification}
\protect\phantomsection\label{pragmatic-framework-specification}
\textbf{Preference Structure (Matrix (C)):} What matters to the agent

\[C = [c_i] \quad c_i = \log P(o_i)\label{eq:matrix_c}\]

This specification goes beyond traditional reinforcement learning by
enabling researchers to specify value landscapes.
\end{block}

\begin{block}{Meta-Pragmatic Implications}
\protect\phantomsection\label{meta-pragmatic-implications}
The meta-pragmatic aspect enables:

\begin{enumerate}
\tightlist
\item
  \textbf{Value System Design:} Specification of what constitutes
  ``good'' outcomes
\item
  \textbf{Pragmatic Pluralism:} Different pragmatic frameworks can be
  explored
\item
  \textbf{Value Learning:} How value systems themselves evolve and adapt
\item
  \textbf{Ethical Framework Integration:} Incorporation of ethical
  considerations
\end{enumerate}
\end{block}
\end{block}

\begin{block}{Integration Across Quadrants}
\protect\phantomsection\label{integration-across-quadrants}
Active Inference operates across all four quadrants simultaneously, with
different aspects of the framework contributing to each quadrant:

\begin{itemize}
\tightlist
\item
  \textbf{Quadrant 1:} Core EFE computation with basic (A), (B), (C),
  (D) specifications
\item
  \textbf{Quadrant 2:} Meta-data weighted EFE with confidence-weighted
  processing
\item
  \textbf{Quadrant 3:} Self-reflective EFE evaluation and meta-cognitive
  control
\item
  \textbf{Quadrant 4:} Framework-level EFE optimization and
  meta-theoretical reasoning
\end{itemize}
\end{block}

\begin{block}{The Modeler's Dual Role}
\protect\phantomsection\label{the-modelers-dual-role}
The framework reveals the dual role of the Active Inference modeler:

\begin{block}{As Architect}
\protect\phantomsection\label{as-architect}
\begin{itemize}
\tightlist
\item
  Specifies epistemic frameworks ((A), (B), (D) matrices)
\item
  Defines pragmatic landscapes ((C) matrix)
\item
  Designs cognitive architectures
\item
  Establishes boundary conditions for cognition
\end{itemize}
\end{block}

\begin{block}{As Subject}
\protect\phantomsection\label{as-subject}
\begin{itemize}
\tightlist
\item
  Uses Active Inference to understand their own cognition
\item
  Applies meta-epistemic principles to knowledge acquisition
\item
  Employs meta-pragmatic frameworks for decision-making
\item
  Engages in recursive self-modeling
\end{itemize}

This dual role creates a recursive relationship where the tools used to
model others become tools for self-understanding.
\end{block}
\end{block}

\begin{block}{Validation Approach}
\protect\phantomsection\label{validation-approach}
The framework's validity is demonstrated through:

\begin{enumerate}
\tightlist
\item
  \textbf{Theoretical Consistency:} Alignment with Free Energy Principle
  foundations
\item
  \textbf{Mathematical Rigor:} Proper formulation of EFE across all
  quadrants
\item
  \textbf{Conceptual Clarity:} Clear distinction between quadrants and
  processing levels
\item
  \textbf{Practical Applicability:} Framework enables systematic
  analysis of meta-level phenomena
\end{enumerate}

The following sections provide concrete demonstrations of each quadrant
with mathematical examples and conceptual analysis.

\begin{figure}[h]
\centering
\includegraphics[width=0.8\textwidth]{../figures/quadrant_matrix.png}
\caption{\(2 \times 2\) Quadrant Framework: Data/Meta-Data \(\times\) Cognitive/Meta-Cognitive processing levels in Active Inference}
\label{fig:quadrant_matrix}
\end{figure}

\begin{figure}[h]
\centering
\includegraphics[width=0.8\textwidth]{../figures/efe_decomposition.png}
\caption{Expected Free Energy decomposition into epistemic and pragmatic components}
\label{fig:efe_decomposition}
\end{figure}

\begin{figure}[h]
\centering
\includegraphics[width=0.8\textwidth]{../figures/perception_action_loop.png}
\caption{Active Inference perception-action loop showing how perception drives action through EFE minimization}
\label{fig:perception_action_loop}
\end{figure}

\begin{figure}[h]
\centering
\includegraphics[width=0.8\textwidth]{../figures/generative_model_structure.png}
\caption{Structure of generative models in Active Inference showing A, B, C, D matrices}
\label{fig:generative_model_structure}
\end{figure}

\begin{figure}[h]
\centering
\includegraphics[width=0.8\textwidth]{../figures/meta_level_concepts.png}
\caption{Meta-pragmatic and meta-epistemic aspects showing modeler specification power}
\label{fig:meta_level_concepts}
\end{figure}

\begin{figure}[h]
\centering
\includegraphics[width=0.8\textwidth]{../figures/fep_system_boundaries.png}
\caption{Free Energy Principle system boundaries showing Markov blanket separating internal and external states}
\label{fig:fep_system_boundaries}
\end{figure}

\begin{figure}[h]
\centering
\includegraphics[width=0.8\textwidth]{../figures/free_energy_dynamics.png}
\caption{Free energy minimization dynamics showing convergence over time and epistemic/pragmatic components}
\label{fig:free_energy_dynamics}
\end{figure}

\begin{figure}[h]
\centering
\includegraphics[width=0.8\textwidth]{../figures/structure_preservation.png}
\caption{Structure preservation dynamics showing how systems maintain internal organization through free energy minimization}
\label{fig:structure_preservation}
\end{figure}

\begin{figure}[h]
\centering
\includegraphics[width=0.8\textwidth]{../figures/quadrant_matrix_enhanced.png}
\caption{\(2 \times 2\) Quadrant Framework with detailed descriptions and examples}
\label{fig:quadrant_matrix_enhanced}
\end{figure}

\begin{figure}[h]
\centering
\includegraphics[width=0.8\textwidth]{../figures/physics_cognition_bridge.png}
\caption{Free Energy Principle as the bridge between physics and cognition domains}
\label{fig:physics_cognition_bridge}
\end{figure}
\end{block}
\end{frame}

\end{document}
