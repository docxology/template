% Options for packages loaded elsewhere
\PassOptionsToPackage{unicode}{hyperref}
\PassOptionsToPackage{hyphens}{url}
\documentclass[
  ignorenonframetext,
]{beamer}
\newif\ifbibliography
\usepackage{pgfpages}
\setbeamertemplate{caption}[numbered]
\setbeamertemplate{caption label separator}{: }
\setbeamercolor{caption name}{fg=normal text.fg}
\beamertemplatenavigationsymbolsempty
% remove section numbering
\setbeamertemplate{part page}{
  \centering
  \begin{beamercolorbox}[sep=16pt,center]{part title}
    \usebeamerfont{part title}\insertpart\par
  \end{beamercolorbox}
}
\setbeamertemplate{section page}{
  \centering
  \begin{beamercolorbox}[sep=12pt,center]{section title}
    \usebeamerfont{section title}\insertsection\par
  \end{beamercolorbox}
}
\setbeamertemplate{subsection page}{
  \centering
  \begin{beamercolorbox}[sep=8pt,center]{subsection title}
    \usebeamerfont{subsection title}\insertsubsection\par
  \end{beamercolorbox}
}
% Prevent slide breaks in the middle of a paragraph
\widowpenalties 1 10000
\raggedbottom
\AtBeginPart{
  \frame{\partpage}
}
\AtBeginSection{
  \ifbibliography
  \else
    \frame{\sectionpage}
  \fi
}
\AtBeginSubsection{
  \frame{\subsectionpage}
}
\usepackage{iftex}
\ifPDFTeX
  \usepackage[T1]{fontenc}
  \usepackage[utf8]{inputenc}
  \usepackage{textcomp} % provide euro and other symbols
\else % if luatex or xetex
  \usepackage{unicode-math} % this also loads fontspec
  \defaultfontfeatures{Scale=MatchLowercase}
  \defaultfontfeatures[\rmfamily]{Ligatures=TeX,Scale=1}
\fi
\usepackage{lmodern}
\ifPDFTeX\else
  % xetex/luatex font selection
\fi
% Use upquote if available, for straight quotes in verbatim environments
\IfFileExists{upquote.sty}{\usepackage{upquote}}{}
\IfFileExists{microtype.sty}{% use microtype if available
  \usepackage[]{microtype}
  \UseMicrotypeSet[protrusion]{basicmath} % disable protrusion for tt fonts
}{}
\makeatletter
\@ifundefined{KOMAClassName}{% if non-KOMA class
  \IfFileExists{parskip.sty}{%
    \usepackage{parskip}
  }{% else
    \setlength{\parindent}{0pt}
    \setlength{\parskip}{6pt plus 2pt minus 1pt}}
}{% if KOMA class
  \KOMAoptions{parskip=half}}
\makeatother
\setlength{\emergencystretch}{3em} % prevent overfull lines
\providecommand{\tightlist}{%
  \setlength{\itemsep}{0pt}\setlength{\parskip}{0pt}}
\usepackage{bookmark}
\IfFileExists{xurl.sty}{\usepackage{xurl}}{} % add URL line breaks if available
\urlstyle{same}
\hypersetup{
  hidelinks,
  pdfcreator={LaTeX via pandoc}}

\author{\texorpdfstring{}{}}
\date{}

\begin{document}

\begin{frame}{The 2×2 Quadrant Model}
\protect\phantomsection\label{sec:quadrant_model}
The \(2 \times 2\) matrix structure organizes Active Inference as a
meta-pragmatic and meta-epistemic methodology. Cognitive processing
varies along two dimensions: Data/Meta-Data and
Cognitive/Meta-Cognitive, yielding four quadrants. Each quadrant
represents a distinct combination of processing level and data type and
employs specific mathematical formulations.

\begin{block}{Quadrant Structure Overview}
\protect\phantomsection\label{sec:quadrant_overview}
To systematically analyze Active Inference's meta-level contributions,
we introduce a framework with axes of Data/Meta-Data and
Cognitive/Meta-Cognitive processing.

\textbf{Data vs Meta-Data (X-axis):} - \textbf{Data:} Raw sensory inputs
and immediate cognitive processing - \textbf{Meta-Data:} Information
about data processing (confidence scores, timestamps, reliability
metrics, processing provenance)

\textbf{Cognitive vs Meta-Cognitive (Y-axis):} - \textbf{Cognitive:}
Direct processing and transformation of information -
\textbf{Meta-Cognitive:} Processing about processing; self-reflection,
monitoring, and control of cognitive processes

\begin{figure}[h]
\centering
\includegraphics[width=0.8\textwidth]{../figures/quadrant_matrix.png}
\caption{$2 \times 2$ Quadrant Structure: Data/Meta-Data $\times$ Cognitive/Meta-Cognitive processing levels in Active Inference. The structure organizes cognitive processing along two dimensions: (1) Data vs Meta-Data (X-axis), distinguishing raw sensory inputs from information about data quality; (2) Cognitive vs Meta-Cognitive (Y-axis), distinguishing direct information transformation from self-reflective monitoring. Each quadrant represents a distinct mode of cognitive operation with specific mathematical formulations.}
\label{fig:quadrant_matrix}
\end{figure}

\begin{figure}[h]
\centering
\includegraphics[width=0.8\textwidth]{../figures/quadrant_matrix_enhanced.png}
\caption{Enhanced $2 \times 2$ Quadrant Structure with detailed descriptions and examples for each quadrant. Q1 provides basic EFE computation; Q2 enhances processing through quality weighting; Q3 enables self-monitoring and adaptive control; Q4 supports framework-level optimization. Each quadrant includes mathematical formulations and practical examples demonstrating the hierarchical relationship between quadrants.}
\label{fig:quadrant_matrix_enhanced}
\end{figure}
\end{block}
\end{frame}

\begin{frame}
\begin{block}{Quadrant 1: Data Processing (Cognitive)}
\protect\phantomsection\label{sec:quadrant_1}
\textbf{Definition:} Basic cognitive processing of raw sensory data at
the fundamental level of cognition, where agents directly process
observations without incorporating quality information or
self-reflection.

\textbf{Active Inference Role:} Baseline pragmatic and epistemic
processing through Expected Free Energy minimization, providing the
foundation upon which all other quadrants build.

\begin{block}{Mathematical Formulation}
\protect\phantomsection\label{mathematical-formulation}
\[\mathcal{F}(\pi) = G(\pi) + H[Q(\pi)]\label{eq:efe_simple}\]

Where \(G(\pi)\) represents pragmatic value (goal achievement) and
\(H[Q(\pi)]\) represents epistemic affordance (information gain).
\end{block}

\begin{block}{Demonstration: Temperature Regulation}
\protect\phantomsection\label{demonstration-temperature-regulation}
Consider a simple agent navigating a two-state environment:

\textbf{Generative Model Specification:} - States: \(s_1\) = ``too
cold'', \(s_2\) = ``too hot'' - Observations: \(o_1\) = ``cold sensor'',
\(o_2\) = ``hot sensor'' - Actions: \(a_1\) = ``heat'', \(a_2\) =
``cool''

\textbf{Matrix Specifications:}

\[A = \begin{pmatrix} 0.9 & 0.1 \\ 0.1 & 0.9 \end{pmatrix} \quad C = \begin{pmatrix} 2.0 \\ -2.0 \end{pmatrix} \quad D = \begin{pmatrix} 0.5 \\ 0.5 \end{pmatrix}\label{eq:q1_matrices}\]

\textbf{EFE Calculation:} For current observation \(o_1\) (cold sensor):

\textbf{Posterior Inference:}

\[q(s) \propto A[:,o_1] \odot D = \begin{pmatrix} 0.45 \\ 0.05 \end{pmatrix}\label{eq:posterior_inference}\]

\textbf{Policy Evaluation:} - Policy \(\pi_1\) (heat):
\(\mathcal{F}(\pi_1) = 0.23\) - Policy \(\pi_2\) (cool):
\(\mathcal{F}(\pi_2) = 1.45\)

\textbf{Result:} Agent selects heating action (lower EFE), demonstrating
basic pragmatic-epistemic balance.

\begin{figure}[h]
\centering
\includegraphics[width=0.8\textwidth]{../figures/quadrant_1_data_cognitive.png}
\caption{Quadrant 1: Basic data processing showing EFE minimization for policy selection. The visualization demonstrates how an agent processes raw sensory data (temperature readings) and selects actions (heating/cooling) by minimizing Expected Free Energy $\mathcal{F}(\pi)$ (Equation \eqref{eq:efe_simple}). Policy $\pi_1$ (heat) achieves lower EFE (0.23) than $\pi_2$ (cool) (1.45), demonstrating principled exploration-exploitation balance.}
\label{fig:quadrant_1_data_cognitive}
\end{figure}
\end{block}
\end{block}
\end{frame}

\begin{frame}
\begin{block}{Quadrant 2: Meta-Data Organization (Cognitive)}
\protect\phantomsection\label{sec:quadrant_2}
\textbf{Definition:} Cognitive processing that incorporates meta-data
(information about data quality, reliability, and provenance) to enhance
primary data processing, improving decision reliability beyond basic
data processing.

\textbf{Active Inference Role:} Enhanced epistemic and pragmatic
processing through meta-data integration, extending Quadrant 1
operations by weighting observations and inferences based on quality
information.

\begin{block}{Mathematical Formulation}
\protect\phantomsection\label{mathematical-formulation-1}
Extended EFE with meta-data weighting:

\[\mathcal{F}(\pi) = w_e \cdot H[Q(\pi)] + w_p \cdot G(\pi) + w_m \cdot M(\pi)\label{eq:efe_metadata}\]

Where: - \(M(\pi)\) represents meta-data derived utility - \(w_e\) is
the epistemic weight - \(w_p\) is the pragmatic weight - \(w_m\) is the
meta-data weight
\end{block}

\begin{block}{Demonstration: Navigation with Confidence Scores}
\protect\phantomsection\label{demonstration-navigation-with-confidence-scores}
Extend Quadrant 1 with confidence scores and temporal meta-data:

\textbf{Meta-Data Structure:} - Confidence scores: \(c(t) \in [0,1]\)
for each observation - Temporal stamps: \(\tau(t)\) for sequencing -
Reliability metrics: \(r(t)\) based on sensor quality

\textbf{Confidence-Weighted Inference:}

\[q(s \mid t) = \frac{c(t) \cdot A[:,o_t] \odot q(s \mid t-1)}{Z}\label{eq:confidence_weighted_inference}\]

Where \(Z\) is a normalization constant. When \(c(t)\) is high, the
observation strongly influences beliefs; when \(c(t)\) is low, previous
beliefs \(q(s \mid t-1)\) are weighted more heavily.

\textbf{Result:} Agent adapts processing based on meta-data quality,
improving decision reliability from 85\% (raw data) to 94\% (meta-data
weighted) in uncertain conditions.

\textbackslash begin\{figure\}{[}h{]} \centering
\includegraphics[width=0.8\textwidth]{../figures/quadrant_2_metadata_cognitive.png}
\textbackslash caption\{Quadrant 2: Meta-data organization showing
quality-weighted processing with confidence scores. Confidence scores
\(c(t)\), temporal stamps \(\tau(t)\), and reliability metrics \(r(t)\)
are integrated into EFE calculation (Equation \eqref{eq:efe_metadata}).
When confidence is low, epistemic weighting increases to gather more
information. This adaptive behavior improves decision reliability from
85\% to 94\%.\} \label{fig:quadrant_2_metadata_cognitive}
\textbackslash end\{figure\}
\end{block}
\end{block}
\end{frame}

\begin{frame}[fragile]
\begin{block}{Quadrant 3: Reflective Processing (Meta-Cognitive)}
\protect\phantomsection\label{sec:quadrant_3}
\textbf{Definition:} Meta-cognitive evaluation and control of data
processing, where agents reflect on their own cognitive processes,
assess inference quality, and adaptively adjust processing strategies.

\textbf{Active Inference Role:} Self-monitoring and adaptive cognitive
control through hierarchical EFE evaluation, enabling systems to
regulate their own cognitive operations based on confidence and
performance assessment.

\begin{block}{Mathematical Formulation}
\protect\phantomsection\label{mathematical-formulation-2}
Hierarchical EFE with self-assessment:

\[\mathcal{F}(\pi) = \mathcal{F}_{primary}(\pi) + \lambda \cdot \mathcal{F}_{meta}(\pi)\label{eq:efe_hierarchical}\]

Where \(\mathcal{F}_{meta}\) evaluates the quality of primary processing
and \(\lambda\) controls meta-cognitive influence.

\textbf{Confidence Assessment Function:}

\[confidence(q, o) = \frac{1}{1 + \exp(-\alpha \cdot (H[q] - H_{expected}))}\label{eq:confidence_assessment}\]

\textbf{Adaptive Strategy Selection:}

\[\pi^*(o, c) = \arg\min_{\pi \in \Pi} \mathcal{F}(\pi) + \lambda(c) \cdot \mathcal{R}(\pi)\label{eq:adaptive_strategy_selection}\]

Where: - \(\lambda(c)\) increases with low confidence -
\(\mathcal{R}(\pi)\) penalizes complex strategies when confidence is low
\end{block}

\begin{block}{Demonstration: Adaptive Strategy Selection}
\protect\phantomsection\label{demonstration-adaptive-strategy-selection}
\textbf{Confidence Trajectory Example:}

\begin{verbatim}
Time:     0    1    2    3    4    5
Conf:   0.9  0.8  0.3  0.2  0.7  0.9
Strat:  Std  Std  Cons Cons Std  Std
EFE:   0.23 0.28 0.45 0.52 0.25 0.22
\end{verbatim}

At times 0-1, high confidence allows standard processing. At times 2-3,
confidence drops, triggering conservative strategies. At times 4-5,
confidence recovers, allowing efficient standard processing.

\begin{figure}[h]
\centering
\includegraphics[width=0.8\textwidth]{../figures/quadrant_3_data_metacognitive.png}
\caption{Quadrant 3: Meta-cognitive reflective processing showing confidence assessment and adaptive attention. The agent monitors inference quality through confidence assessment (Equation \eqref{eq:confidence_assessment}). When confidence drops below threshold $\gamma$, the agent adapts processing strategies (Equation \eqref{eq:adaptive_strategy_selection}), switching to conservative strategies during uncertainty and returning to efficient processing when confidence recovers.}
\label{fig:quadrant_3_data_metacognitive}
\end{figure}
\end{block}
\end{block}
\end{frame}

\begin{frame}[fragile]
\begin{block}{Quadrant 4: Higher-Order Reasoning (Meta-Cognitive)}
\protect\phantomsection\label{sec:quadrant_4}
\textbf{Definition:} Meta-cognitive processing of meta-data about
cognition itself, where systems analyze patterns in their own
meta-cognitive performance to optimize fundamental framework parameters,
enabling recursive self-analysis at the highest level of cognitive
abstraction.

\textbf{Active Inference Role:} Framework-level reasoning and
meta-theoretical analysis through parameter optimization, allowing
systems to evolve their cognitive architectures.

\begin{block}{Mathematical Formulation}
\protect\phantomsection\label{mathematical-formulation-3}
Multi-level hierarchical optimization:

\[\min_{\Theta} \mathcal{F}(\pi; \Theta) + \mathcal{R}(\Theta)\label{eq:framework_optimization}\]

Where \(\Theta\) represents framework parameters and
\(\mathcal{R}(\Theta)\) is a regularization term ensuring framework
coherence.

\textbf{Higher-Order Optimization:}

\[\Theta^* = \arg\max_{\Theta} \mathbb{E}[U(c, e, \kappa \mid \Theta)]\label{eq:higher_order_optimization}\]

Where: - \(\bar{c}\) = average confidence - \(e(\sigma)\) = strategy
effectiveness - \(\kappa\) = framework coherence
\end{block}

\begin{block}{Demonstration: Framework Parameter Optimization}
\protect\phantomsection\label{demonstration-framework-parameter-optimization}
\textbf{Performance Analysis:}

\begin{verbatim}
Framework Parameter | Current | Optimized | Improvement
Confidence Threshold | 0.7    | 0.65     | +12%
Adaptation Rate     | 0.1    | 0.15     | +8%
Strategy Diversity  | 3      | 5        | +15%
Overall Performance | 78%    | 96%      | +23%
\end{verbatim}

Lowering the confidence threshold (0.7 → 0.65) enables earlier
uncertainty detection. Increasing adaptation rate (0.1 → 0.15) allows
faster response. Expanding strategy diversity (3 → 5) provides more
options. Combined effect: +23\% overall improvement.

\textbackslash begin\{figure\}{[}h{]} \centering
\includegraphics[width=0.8\textwidth]{../figures/quadrant_4_metadata_metacognitive.png}
\textbackslash caption\{Quadrant 4: Higher-order reasoning showing
framework-level meta-cognitive processing. The system analyzes patterns
in meta-cognitive performance to optimize framework parameters (Equation
\eqref{eq:higher_order_optimization}). Framework evolution from initial
(\(\theta_c=0.7\), \(\alpha=0.1\), \(d=3\)) to optimized
(\(\theta_c=0.65\), \(\alpha=0.15\), \(d=5\)) achieves +23\% performance
improvement through recursive self-analysis.\}
\label{fig:quadrant_4_metadata_metacognitive}
\textbackslash end\{figure\}
\end{block}
\end{block}
\end{frame}

\begin{frame}
\begin{block}{Cross-Quadrant Integration}
\protect\phantomsection\label{sec:cross_quadrant_integration}
All quadrants operate simultaneously in Active Inference systems,
creating a multi-layered cognitive architecture:

\begin{block}{Simultaneous Operation}
\protect\phantomsection\label{simultaneous-operation}
\textbf{Quadrant 1 (Foundation):} Basic EFE computation provides
fundamental cognitive processing using Equation \eqref{eq:efe_simple}.

\textbf{Quadrant 2 (Enhancement):} Meta-data integration improves
processing reliability using Equation \eqref{eq:efe_metadata}.

\textbf{Quadrant 3 (Reflection):} Self-monitoring enables adaptive
control using Equation \eqref{eq:efe_hierarchical}.

\textbf{Quadrant 4 (Evolution):} Framework-level reasoning drives system
improvement using Equation \eqref{eq:framework_optimization}.
\end{block}

\begin{block}{Dynamic Balance}
\protect\phantomsection\label{dynamic-balance}
The relative influence of each quadrant adapts based on context: -
\textbf{Routine Conditions:} Quadrant 1 dominates with efficient
processing - \textbf{Uncertainty:} Quadrant 2 increases meta-data
weighting - \textbf{Errors:} Quadrant 3 triggers self-reflection and
strategy adjustment - \textbf{Novelty:} Quadrant 4 enables framework
adaptation
\end{block}

\begin{block}{Emergent Properties}
\protect\phantomsection\label{emergent-properties}
The integration produces meta-level cognitive capabilities: 1.
\textbf{Self-Awareness:} Quadrant 3 enables monitoring of cognitive
processes 2. \textbf{Adaptability:} Quadrant 4 allows framework
evolution 3. \textbf{Robustness:} Multiple processing levels provide
failure resilience 4. \textbf{Learning:} Framework adaptation enables
cumulative improvement
\end{block}
\end{block}
\end{frame}

\begin{frame}
\begin{block}{Framework Validation}
\protect\phantomsection\label{sec:framework_validation}
\begin{block}{Theoretical Consistency}
\protect\phantomsection\label{theoretical-consistency}
The quadrant structure maintains consistency with Active Inference
principles: - \textbf{Free Energy Principle:} All quadrants minimize
variational free energy at their respective levels - \textbf{Generative
Models:} Each quadrant utilizes \(A\), \(B\), \(C\), \(D\) matrices
appropriately - \textbf{Hierarchical Processing:} Quadrants represent
increasing levels of abstraction
\end{block}

\begin{block}{Mathematical Rigor}
\protect\phantomsection\label{mathematical-rigor}
All formulations are grounded in established Active Inference theory: -
EFE formulations follow standard derivations - Meta-data integration
uses probabilistic weighting - Meta-cognitive control employs
hierarchical optimization - Framework adaptation uses evolutionary
principles
\end{block}

\begin{block}{Conceptual Clarity}
\protect\phantomsection\label{conceptual-clarity}
The structure provides clear distinctions: - \textbf{Data vs Meta-Data:}
Raw inputs vs quality information - \textbf{Cognitive vs
Meta-Cognitive:} Direct processing vs self-reflection - \textbf{Quadrant
Boundaries:} Clear categorization enabling systematic analysis
\end{block}
\end{block}
\end{frame}

\end{document}
