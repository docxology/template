% Options for packages loaded elsewhere
\PassOptionsToPackage{unicode}{hyperref}
\PassOptionsToPackage{hyphens}{url}
\documentclass[
  ignorenonframetext,
]{beamer}
\newif\ifbibliography
\usepackage{pgfpages}
\setbeamertemplate{caption}[numbered]
\setbeamertemplate{caption label separator}{: }
\setbeamercolor{caption name}{fg=normal text.fg}
\beamertemplatenavigationsymbolsempty
% remove section numbering
\setbeamertemplate{part page}{
  \centering
  \begin{beamercolorbox}[sep=16pt,center]{part title}
    \usebeamerfont{part title}\insertpart\par
  \end{beamercolorbox}
}
\setbeamertemplate{section page}{
  \centering
  \begin{beamercolorbox}[sep=12pt,center]{section title}
    \usebeamerfont{section title}\insertsection\par
  \end{beamercolorbox}
}
\setbeamertemplate{subsection page}{
  \centering
  \begin{beamercolorbox}[sep=8pt,center]{subsection title}
    \usebeamerfont{subsection title}\insertsubsection\par
  \end{beamercolorbox}
}
% Prevent slide breaks in the middle of a paragraph
\widowpenalties 1 10000
\raggedbottom
\AtBeginPart{
  \frame{\partpage}
}
\AtBeginSection{
  \ifbibliography
  \else
    \frame{\sectionpage}
  \fi
}
\AtBeginSubsection{
  \frame{\subsectionpage}
}
\usepackage{iftex}
\ifPDFTeX
  \usepackage[T1]{fontenc}
  \usepackage[utf8]{inputenc}
  \usepackage{textcomp} % provide euro and other symbols
\else % if luatex or xetex
  \usepackage{unicode-math} % this also loads fontspec
  \defaultfontfeatures{Scale=MatchLowercase}
  \defaultfontfeatures[\rmfamily]{Ligatures=TeX,Scale=1}
\fi
\usepackage{lmodern}
\ifPDFTeX\else
  % xetex/luatex font selection
\fi
% Use upquote if available, for straight quotes in verbatim environments
\IfFileExists{upquote.sty}{\usepackage{upquote}}{}
\IfFileExists{microtype.sty}{% use microtype if available
  \usepackage[]{microtype}
  \UseMicrotypeSet[protrusion]{basicmath} % disable protrusion for tt fonts
}{}
\makeatletter
\@ifundefined{KOMAClassName}{% if non-KOMA class
  \IfFileExists{parskip.sty}{%
    \usepackage{parskip}
  }{% else
    \setlength{\parindent}{0pt}
    \setlength{\parskip}{6pt plus 2pt minus 1pt}}
}{% if KOMA class
  \KOMAoptions{parskip=half}}
\makeatother
\setlength{\emergencystretch}{3em} % prevent overfull lines
\providecommand{\tightlist}{%
  \setlength{\itemsep}{0pt}\setlength{\parskip}{0pt}}
\usepackage{bookmark}
\IfFileExists{xurl.sty}{\usepackage{xurl}}{} % add URL line breaks if available
\urlstyle{same}
\hypersetup{
  hidelinks,
  pdfcreator={LaTeX via pandoc}}

\author{\texorpdfstring{}{}}
\date{}

\begin{document}

\begin{frame}{Conclusion}
\protect\phantomsection\label{conclusion}
This small code project successfully demonstrated a complete research
pipeline from algorithm implementation through testing, analysis, and
manuscript generation.

\begin{block}{Project Achievements}
\protect\phantomsection\label{project-achievements}
The implementation achieved all major objectives:

\begin{enumerate}
\tightlist
\item
  \textbf{Clean Codebase}: Well-structured, documented, and testable
  code
\item
  \textbf{Comprehensive Testing}: 100\% test coverage with meaningful
  assertions
\item
  \textbf{Automated Analysis}: Scripts that generate figures and data
  automatically
\item
  \textbf{Manuscript Integration}: Research write-up referencing
  generated outputs
\item
  \textbf{Pipeline Compatibility}: Full integration with the research
  template system
\end{enumerate}
\end{block}

\begin{block}{Technical Contributions}
\protect\phantomsection\label{technical-contributions}
\begin{block}{Algorithm Implementation}
\protect\phantomsection\label{algorithm-implementation}
\begin{itemize}
\tightlist
\item
  Correct gradient descent implementation with convergence detection
\item
  Robust numerical computations using NumPy
\item
  Flexible parameter configuration
\end{itemize}
\end{block}

\begin{block}{Testing Strategy}
\protect\phantomsection\label{testing-strategy}
\begin{itemize}
\tightlist
\item
  Unit tests for all core functions
\item
  Integration tests for algorithm convergence
\item
  Edge case coverage for robustness
\item
  Numerical accuracy validation
\end{itemize}
\end{block}

\begin{block}{Analysis Capabilities}
\protect\phantomsection\label{analysis-capabilities}
\begin{itemize}
\tightlist
\item
  Automated experiment execution
\item
  Publication-quality figure generation
\item
  Structured data output in CSV format
\item
  Figure registration for manuscript integration
\end{itemize}
\end{block}
\end{block}

\begin{block}{Research Pipeline Validation}
\protect\phantomsection\label{research-pipeline-validation}
The project validates the research template's ability to handle:

\begin{itemize}
\tightlist
\item
  \textbf{Code projects}: From implementation to publication
\item
  \textbf{Automated analysis}: Reproducible result generation
\item
  \textbf{Figure integration}: Seamless manuscript-visualization linkage
\item
  \textbf{Testing requirements}: Maintaining quality standards
\item
  \textbf{Multi-project support}: Running multiple independent research
  projects
\item
  \textbf{LLM integration}: Automated scientific review and manuscript
  analysis
\item
  \textbf{Executive reporting}: Cross-project metrics and comprehensive
  dashboards
\item
  \textbf{Multi-format output}: PDF, HTML, and presentation generation
\end{itemize}
\end{block}

\begin{block}{Key Insights}
\protect\phantomsection\label{key-insights}
\begin{enumerate}
\tightlist
\item
  \textbf{Step Size Selection}: Critical for convergence speed and
  stability
\item
  \textbf{Testing Importance}: Comprehensive tests catch numerical
  issues early
\item
  \textbf{Automation Benefits}: Scripts ensure reproducible analysis
\item
  \textbf{Documentation Value}: Clear code and manuscripts improve
  research quality
\end{enumerate}
\end{block}

\begin{block}{Future Extensions}
\protect\phantomsection\label{future-extensions}
This foundation could be extended to:

\begin{itemize}
\tightlist
\item
  \textbf{Advanced algorithms}: Newton methods, quasi-Newton approaches
\item
  \textbf{Constrained optimization}: Handling inequality constraints
\item
  \textbf{Stochastic methods}: Mini-batch and online learning variants,
  including adaptive optimization algorithms such as Adam
  \cite{kingma2014adam}
\item
  \textbf{Parallel computing}: Distributed optimization algorithms
\end{itemize}
\end{block}

\begin{block}{Final Assessment}
\protect\phantomsection\label{final-assessment}
The small code project successfully demonstrates that the research
template can support projects ranging from prose-focused manuscripts to
fully-tested algorithmic implementations. The combination of rigorous
testing, automated analysis, and integrated documentation provides a
solid foundation for reproducible computational research.

This work contributes to the broader goal of improving research software
quality and reproducibility through standardized development practices
and comprehensive testing strategies.
\end{block}
\end{frame}

\end{document}
