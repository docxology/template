% Options for packages loaded elsewhere
\PassOptionsToPackage{unicode}{hyperref}
\PassOptionsToPackage{hyphens}{url}
\documentclass[
  ignorenonframetext,
]{beamer}
\newif\ifbibliography
\usepackage{pgfpages}
\setbeamertemplate{caption}[numbered]
\setbeamertemplate{caption label separator}{: }
\setbeamercolor{caption name}{fg=normal text.fg}
\beamertemplatenavigationsymbolsempty
% remove section numbering
\setbeamertemplate{part page}{
  \centering
  \begin{beamercolorbox}[sep=16pt,center]{part title}
    \usebeamerfont{part title}\insertpart\par
  \end{beamercolorbox}
}
\setbeamertemplate{section page}{
  \centering
  \begin{beamercolorbox}[sep=12pt,center]{section title}
    \usebeamerfont{section title}\insertsection\par
  \end{beamercolorbox}
}
\setbeamertemplate{subsection page}{
  \centering
  \begin{beamercolorbox}[sep=8pt,center]{subsection title}
    \usebeamerfont{subsection title}\insertsubsection\par
  \end{beamercolorbox}
}
% Prevent slide breaks in the middle of a paragraph
\widowpenalties 1 10000
\raggedbottom
\AtBeginPart{
  \frame{\partpage}
}
\AtBeginSection{
  \ifbibliography
  \else
    \frame{\sectionpage}
  \fi
}
\AtBeginSubsection{
  \frame{\subsectionpage}
}
\usepackage{iftex}
\ifPDFTeX
  \usepackage[T1]{fontenc}
  \usepackage[utf8]{inputenc}
  \usepackage{textcomp} % provide euro and other symbols
\else % if luatex or xetex
  \usepackage{unicode-math} % this also loads fontspec
  \defaultfontfeatures{Scale=MatchLowercase}
  \defaultfontfeatures[\rmfamily]{Ligatures=TeX,Scale=1}
\fi
\usepackage{lmodern}
\ifPDFTeX\else
  % xetex/luatex font selection
\fi
% Use upquote if available, for straight quotes in verbatim environments
\IfFileExists{upquote.sty}{\usepackage{upquote}}{}
\IfFileExists{microtype.sty}{% use microtype if available
  \usepackage[]{microtype}
  \UseMicrotypeSet[protrusion]{basicmath} % disable protrusion for tt fonts
}{}
\makeatletter
\@ifundefined{KOMAClassName}{% if non-KOMA class
  \IfFileExists{parskip.sty}{%
    \usepackage{parskip}
  }{% else
    \setlength{\parindent}{0pt}
    \setlength{\parskip}{6pt plus 2pt minus 1pt}}
}{% if KOMA class
  \KOMAoptions{parskip=half}}
\makeatother
\setlength{\emergencystretch}{3em} % prevent overfull lines
\providecommand{\tightlist}{%
  \setlength{\itemsep}{0pt}\setlength{\parskip}{0pt}}
\usepackage{bookmark}
\IfFileExists{xurl.sty}{\usepackage{xurl}}{} % add URL line breaks if available
\urlstyle{same}
\hypersetup{
  hidelinks,
  pdfcreator={LaTeX via pandoc}}

\author{\texorpdfstring{}{}}
\date{}

\begin{document}

\begin{frame}
\newpage
\end{frame}

\section{Conclusion: Summary and Actionable
Recommendations}\label{conclusion-summary-and-actionable-recommendations}

\begin{frame}{Summary}
\protect\phantomsection\label{sec:summary}
We presented the \textbf{Cognitive Integrity Framework (CIF)}, a formal
foundation for securing multiagent AI operators against cognitive
manipulation attacks. As AI deployment shifts from single-model
inference to autonomous agent orchestration, the attack surface expands
from input/output filtering to encompass beliefs, goals, trust
relationships, and inter-agent coordination. CIF addresses this expanded
surface through formal mechanisms with provable guarantees.

\begin{block}{Formal Contributions}
\protect\phantomsection\label{formal-contributions}
\begin{table}[htbp]
\centering
\caption{Summary of formal contributions.}
\label{tab:formal-contributions}
\begin{tabular}{@{}lp{8cm}@{}}
\toprule
Contribution & Significance \\
\midrule
Trust Calculus & Bounded delegation with $O(\delta^d)$ decay guarantee prevents trust laundering and amplification---a \textit{structural} property independent of adversary sophistication \\
Defense Composition Algebra & Formal rules enabling predictable reasoning about layered defense effectiveness \\
Information-Theoretic Bounds & Fundamental limits on stealth-impact tradeoff constraining adversary capabilities \\
Integrity Properties & Belief consistency, goal alignment, provenance verifiability as verifiable properties \\
\bottomrule
\end{tabular}
\end{table}
\end{block}

\begin{block}{Conceptual Contributions}
\protect\phantomsection\label{conceptual-contributions}
\begin{enumerate}
\item \textbf{Cognitive Security Operator Posture}: A defensive stance for systems where the attack surface is the reasoning process itself---distinct from traditional perimeter security
\item \textbf{The 2026 Multiagent Landscape}: Characterization of contemporary agentic AI as cyberphysical cognitive operators with persistent agency, active world modification, hierarchical delegation, and cross-modality operation
\item \textbf{Cross-Modality Trust}: Extension of trust calculus to handle heterogeneous modalities with modality-adjusted reliability factors
\item \textbf{Federated Trust}: Framework for reasoning about trust across organizational boundaries
\end{enumerate}
\end{block}

\begin{block}{Core Insights}
\protect\phantomsection\label{core-insights}
\begin{enumerate}
\item \textbf{Multiagent systems require multiagent security}: Single-agent defenses miss inter-agent attack vectors entirely. The trust relationship between agents is itself an attack surface.

\item \textbf{Trust must be bounded}: Without $\delta^d$ decay, delegation chains enable trust laundering where adversarial content acquires trusted-source status through intermediaries. This is a \textit{structural} vulnerability requiring \textit{structural} mitigation.

\item \textbf{Defenses compose predictably}: The defense composition algebra enables formal reasoning about layered security. Orthogonal defenses compose multiplicatively, explaining why full CIF substantially outperforms any single mechanism.

\item \textbf{Information-theoretic limits constrain adversaries}: The stealth-impact tradeoff theorem establishes that high-impact attacks cannot remain completely undetectable. This provides theoretical grounding for defense strategies.

\item \textbf{Cognitive security requires continuous verification}: Unlike perimeter security that trusts internals after boundary checks, cognitive security requires continuous verification of beliefs, goals, and trust relationships.
\end{enumerate}
\end{block}
\end{frame}

\begin{frame}{Actionable Recommendations}
\protect\phantomsection\label{sec:recommendations}
\begin{block}{For Practitioners}
\protect\phantomsection\label{for-practitioners}
\textbf{Immediate priorities}:

\begin{enumerate}
\item Implement trust decay in all delegation chains ($\delta \leq 0.9$)
\item Deploy cognitive tripwires for identity and boundary beliefs
\item Establish belief provenance tracking for high-stakes decisions
\item Define escalation procedures for cognitive security alerts
\end{enumerate}

\textbf{Architecture selection}: Match security posture to threat model.
Hierarchical architectures with Byzantine-tolerant orchestrators suit
high-security contexts; peer-to-peer topologies with trust decay may
suffice for collaborative environments.
\end{block}

\begin{block}{For Researchers}
\protect\phantomsection\label{for-researchers}
\textbf{Open Questions} with significant impact potential:

\textbf{Theoretical Foundations}

\begin{itemize}
\item
  \textbf{Q1: Optimal trust decay functions.} Under what conditions is
  exponential decay (\(\delta^d\)) optimal? Are there task distributions
  or adversary models where alternative decay functions (e.g.,
  polynomial, threshold-based) provide better security-utility
  tradeoffs?
\item
  \textbf{Q2: Tight detection bounds.} Can the stealth-impact bounds in
  Theorem 6.2 be tightened? What adversary adaptations most effectively
  approach the theoretical limit, and what detection enhancements can
  push the bound further?
\item
  \textbf{Q3: Belief consistency under partial observability.} How
  should agents maintain belief integrity when they cannot observe the
  full system state? What guarantees remain achievable with bounded
  observation horizons?
\end{itemize}

\textbf{Defense Mechanisms}

\begin{itemize}
\item
  \textbf{Q4: Adaptive defense evolution.} How can defense mechanisms
  learn from detected attacks without creating new vulnerabilities? Can
  we formalize safe online learning for cognitive defenses?
\item
  \textbf{Q5: Semantic equivalence detection.} What architectures best
  detect semantically equivalent attacks that evade syntactic pattern
  matching? How do we balance detection sensitivity against
  computational overhead?
\item
  \textbf{Q6: Orchestrator hardening.} Given that orchestrator
  compromise bypasses downstream defenses, what architectural patterns
  minimize single-point-of-failure risk while maintaining coordination
  efficiency?
\end{itemize}

\textbf{Scalability and Performance}

\begin{itemize}
\item
  \textbf{Q7: Large-scale consensus.} How can Byzantine-tolerant
  consensus scale beyond \(O(n^2)\) message complexity for agent
  populations \(>1000\)? Are hierarchical or probabilistic approaches
  sufficient for CIF guarantees?
\item
  \textbf{Q8: Real-time defense overhead.} What is the fundamental
  latency-security tradeoff for cognitive firewalls? Can streaming
  classifiers achieve comparable accuracy to batch models?
\end{itemize}

\textbf{Evaluation and Benchmarking}

\begin{itemize}
\item
  \textbf{Q9: Adversarial benchmark construction.} How should we
  construct attack corpora that remain challenging despite model
  improvements? Can we formalize attack diversity and coverage metrics?
\item
  \textbf{Q10: Colony CogSec evaluation.} What benchmarks capture
  stigmergic attack surfaces---shared state manipulation, environmental
  signaling, emergent coordination failures?
  (\cref{sec:eusocial-cogsec})
\end{itemize}

\textbf{Cross-Organizational Deployment}

\begin{itemize}
\item
  \textbf{Q11: Federated trust interoperability.} How can organizations
  with different trust semantics, decay parameters, and risk tolerances
  federate securely? What minimal protocol guarantees enable safe
  cross-boundary delegation?
\item
  \textbf{Q12: Trust portability.} When agents migrate between
  organizations or contexts, how should accumulated trust transfer? What
  prevents trust-laundering through organizational hops?
\end{itemize}

\textbf{Governance and Long-term Safety}

\begin{itemize}
\item
  \textbf{Q13: Liability attribution.} When a delegated agent causes
  harm through a multi-hop chain, how should responsibility distribute?
  What logging and provenance mechanisms support post-hoc attribution?
\item
  \textbf{Q14: Emergent goal stability.} As agent populations grow and
  interact, what formal guarantees prevent collective goal drift toward
  unintended attractors? How do we verify alignment preservation at
  scale?
\end{itemize}

\textbf{Cognitive Science and First Principles of Intelligence}

\begin{itemize}
\item
  \textbf{Q15: Cognitive security as predictive processing.} How do CIF
  defense mechanisms map onto predictive coding architectures? Can
  belief sandboxing be understood as precision-weighted prediction error
  gating?
\item
  \textbf{Q16: Collective intelligence foundations.} What principles
  from swarm cognition, distributed problem-solving, and stigmergic
  coordination inform robust multiagent security? How do honeybee quorum
  sensing and ant colony consensus differ from Byzantine fault
  tolerance?
\item
  \textbf{Q17: Metacognitive integrity.} How should agents reason about
  their own cognitive security status? What introspective mechanisms
  enable agents to detect when their own belief-formation processes may
  be compromised?
\end{itemize}

\textbf{Active Inference and Free Energy Principle}

\begin{itemize}
\item
  \textbf{Q18: CIF as active inference.} Can the Cognitive Integrity
  Framework be reformulated within the Free Energy Principle? Do trust
  dynamics correspond to precision estimation, and attacks to artificial
  inflation of prediction errors? (\cref{sec:eusocial-cogsec})
\item
  \textbf{Q19: Expected free energy for defense selection.} How can
  agents use expected free energy to select among available defense
  mechanisms? What priors over attack distributions optimize epistemic
  and pragmatic value?
\item
  \textbf{Q20: Allostatic cognitive security.} How should agents
  maintain cognitive homeostasis under adversarial conditions? What are
  the analogs of interoceptive inference for detecting internal state
  manipulation?
\end{itemize}

\textbf{Systems Neuroscience and Neural Computation}

\begin{itemize}
\item
  \textbf{Q21: Neuromodulatory trust dynamics.} How do biological
  neuromodulatory systems (dopamine, acetylcholine, norepinephrine)
  implement trust and uncertainty estimation? What computational
  principles transfer to artificial cognitive security?
\item
  \textbf{Q22: Hierarchical predictive security.} How should defense
  mechanisms be organized across cortical-like processing hierarchies?
  Can top-down predictions provide robustness against bottom-up
  adversarial inputs?
\item
  \textbf{Q23: Attentional gating for cognitive firewalls.} What can
  selective attention mechanisms teach us about efficient input
  filtering? How do biological systems achieve low-latency threat
  detection without exhaustive content analysis?
\end{itemize}

\textbf{Cyberphysical Cybernetics and Embodied AI}

\begin{itemize}
\item
  \textbf{Q24: Sensorimotor cognitive security.} How do embodied agents
  maintain belief integrity when sensory and motor channels are attack
  surfaces? What closed-loop control principles apply to cognitive
  defense?
\item
  \textbf{Q25: Wearable and IoT agent security.} How should
  resource-constrained edge agents implement CIF mechanisms? What
  minimal trust infrastructure enables secure coordination among
  heterogeneous IoT devices?
\item
  \textbf{Q26: Biomimetic defense architectures.} What can immune system
  principles (self/non-self discrimination, clonal selection, immune
  memory) contribute to cognitive attack detection and response?
\item
  \textbf{Q27: Multi-scale temporal integration.} How should cognitive
  security mechanisms integrate across millisecond (reflexive), second
  (deliberative), and hour/day (adaptive) timescales? What corresponds
  to habit formation in defense automation?
\end{itemize}
\end{block}

\begin{block}{For Policymakers}
\protect\phantomsection\label{for-policymakers}
\textbf{Governance priorities}:

\begin{enumerate}
\item Establish cognitive security audit requirements for high-risk deployments
\item Require transparency on inter-agent trust mechanisms and delegation policies
\item Create incident reporting frameworks for cognitive attacks
\item Fund research on adaptive defenses and standardization efforts
\item Address liability allocation for delegated agent actions
\end{enumerate}
\end{block}
\end{frame}

\begin{frame}{Closing Statement}
\protect\phantomsection\label{sec:closing}
The shift from single-model inference to multiagent operators is not
merely an engineering evolution---it introduces fundamentally new
security challenges that require fundamentally new approaches.
Traditional security focuses on perimeters and access control; cognitive
security must address the integrity of reasoning processes themselves.

CIF provides both theoretical foundations and practical mechanisms for
this challenge. The trust calculus offers provable guarantees against
amplification attacks. The defense composition algebra enables
principled reasoning about layered security. The information-theoretic
bounds establish fundamental limits on adversary capabilities. Together,
these formal contributions move cognitive security from ad-hoc defenses
to principled engineering.

\textbf{Part 2} of this series provides empirical validation
demonstrating that these formal mechanisms translate to practical
protection across diverse production architectures. \textbf{Part 3}
offers actionable deployment guidance for practitioners and AI agents.
Together, the three papers provide a comprehensive framework for
understanding, implementing, and operating cognitive security in
multiagent AI systems.

The formal gaps identified in this work---semantic equivalence attacks,
progressive drift, orchestrator compromise---define the frontier for
future research, while the provable guarantees (bounded trust,
composable defenses, information-theoretic limits) provide the stable
theoretical foundation on which that research can build.

\textbf{As autonomous AI agents increasingly operate in high-stakes contexts---executing code, modifying infrastructure, controlling resources, and making decisions with lasting consequences---the formal foundations established here become not merely useful but essential infrastructure for secure deployment.}

Cognitive security is not optional for the multiagent future. It is
foundational.
\end{frame}

\end{document}
