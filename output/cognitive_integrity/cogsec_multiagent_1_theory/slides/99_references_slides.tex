% Options for packages loaded elsewhere
\PassOptionsToPackage{unicode}{hyperref}
\PassOptionsToPackage{hyphens}{url}
\documentclass[
  ignorenonframetext,
]{beamer}
\newif\ifbibliography
\usepackage{pgfpages}
\setbeamertemplate{caption}[numbered]
\setbeamertemplate{caption label separator}{: }
\setbeamercolor{caption name}{fg=normal text.fg}
\beamertemplatenavigationsymbolsempty
% remove section numbering
\setbeamertemplate{part page}{
  \centering
  \begin{beamercolorbox}[sep=16pt,center]{part title}
    \usebeamerfont{part title}\insertpart\par
  \end{beamercolorbox}
}
\setbeamertemplate{section page}{
  \centering
  \begin{beamercolorbox}[sep=12pt,center]{section title}
    \usebeamerfont{section title}\insertsection\par
  \end{beamercolorbox}
}
\setbeamertemplate{subsection page}{
  \centering
  \begin{beamercolorbox}[sep=8pt,center]{subsection title}
    \usebeamerfont{subsection title}\insertsubsection\par
  \end{beamercolorbox}
}
% Prevent slide breaks in the middle of a paragraph
\widowpenalties 1 10000
\raggedbottom
\AtBeginPart{
  \frame{\partpage}
}
\AtBeginSection{
  \ifbibliography
  \else
    \frame{\sectionpage}
  \fi
}
\AtBeginSubsection{
  \frame{\subsectionpage}
}
\usepackage{iftex}
\ifPDFTeX
  \usepackage[T1]{fontenc}
  \usepackage[utf8]{inputenc}
  \usepackage{textcomp} % provide euro and other symbols
\else % if luatex or xetex
  \usepackage{unicode-math} % this also loads fontspec
  \defaultfontfeatures{Scale=MatchLowercase}
  \defaultfontfeatures[\rmfamily]{Ligatures=TeX,Scale=1}
\fi
\usepackage{lmodern}
\ifPDFTeX\else
  % xetex/luatex font selection
\fi
% Use upquote if available, for straight quotes in verbatim environments
\IfFileExists{upquote.sty}{\usepackage{upquote}}{}
\IfFileExists{microtype.sty}{% use microtype if available
  \usepackage[]{microtype}
  \UseMicrotypeSet[protrusion]{basicmath} % disable protrusion for tt fonts
}{}
\makeatletter
\@ifundefined{KOMAClassName}{% if non-KOMA class
  \IfFileExists{parskip.sty}{%
    \usepackage{parskip}
  }{% else
    \setlength{\parindent}{0pt}
    \setlength{\parskip}{6pt plus 2pt minus 1pt}}
}{% if KOMA class
  \KOMAoptions{parskip=half}}
\makeatother
\setlength{\emergencystretch}{3em} % prevent overfull lines
\providecommand{\tightlist}{%
  \setlength{\itemsep}{0pt}\setlength{\parskip}{0pt}}
\usepackage{bookmark}
\IfFileExists{xurl.sty}{\usepackage{xurl}}{} % add URL line breaks if available
\urlstyle{same}
\hypersetup{
  hidelinks,
  pdfcreator={LaTeX via pandoc}}

\author{\texorpdfstring{}{}}
\date{}

\begin{document}

\begin{frame}
\newpage
\end{frame}

\section{References}\label{sec:references}

\begin{frame}{Foundational Works}
\protect\phantomsection\label{foundational-works}
\begin{enumerate}
\item
  Lamport, L., Shostak, R., \& Pease, M. (1982). The Byzantine Generals
  Problem. \emph{ACM Transactions on Programming Languages and Systems},
  4(3), 382-401.
\item
  Dwork, C., Lynch, N., \& Stockmeyer, L. (1988). Consensus in the
  Presence of Partial Synchrony. \emph{Journal of the ACM}, 35(2),
  288-323.
\item
  Josang, A., Ismail, R., \& Boyd, C. (2007). A Survey of Trust and
  Reputation Systems for Online Service Provision. \emph{Decision
  Support Systems}, 43(2), 618-644.
\end{enumerate}
\end{frame}

\begin{frame}{Prompt Injection and LLM Security}
\protect\phantomsection\label{prompt-injection-and-llm-security}
\begin{enumerate}
\item
  Qi, X., et al.~(2024). Visual Adversarial Examples Jailbreak Aligned
  Large Language Models. \emph{AAAI 2024}.
\item
  Perez, F., \& Ribeiro, I. (2023). Ignore This Title and HackAPrompt:
  Exposing Systemic Vulnerabilities of LLMs. \emph{EMNLP 2023}.
\item
  Greshake, K., et al.~(2023). Not What You've Signed Up For:
  Compromising Real-World LLM-Integrated Applications with Indirect
  Prompt Injection. \emph{ACM AISec 2023}.
\item
  Liu, Y., et al.~(2023). Prompt Injection Attack Against LLM-Integrated
  Applications. \emph{arXiv:2306.05499}.
\end{enumerate}
\end{frame}

\begin{frame}{Constitutional AI and Alignment}
\protect\phantomsection\label{constitutional-ai-and-alignment}
\begin{enumerate}
\item
  Bai, Y., et al.~(2022). Constitutional AI: Harmlessness from AI
  Feedback. \emph{arXiv:2212.08073}.
\item
  Askell, A., et al.~(2021). A General Language Assistant as a
  Laboratory for Alignment. \emph{arXiv:2112.00861}.
\end{enumerate}
\end{frame}

\begin{frame}{Multiagent Systems}
\protect\phantomsection\label{multiagent-systems}
\begin{enumerate}
\item
  Wooldridge, M. (2009). \emph{An Introduction to Multiagent Systems}.
  John Wiley \& Sons.
\item
  Shoham, Y., \& Leyton-Brown, K. (2008). \emph{Multiagent Systems:
  Algorithmic, Game-Theoretic, and Logical Foundations}. Cambridge
  University Press.
\item
  Hong, S., et al.~(2023). MetaGPT: Meta Programming for Multi-Agent
  Collaborative Framework. \emph{arXiv:2308.00352}.
\item
  Wu, Q., et al.~(2023). AutoGen: Enabling Next-Gen LLM Applications via
  Multi-Agent Conversation. \emph{arXiv:2308.08155}.
\end{enumerate}
\end{frame}

\begin{frame}{Trust in Distributed Systems}
\protect\phantomsection\label{trust-in-distributed-systems}
\begin{enumerate}
\item
  Marsh, S. P. (1994). Formalising Trust as a Computational Concept.
  \emph{PhD Thesis, University of Stirling}.
\item
  Gambetta, D. (1988). Can We Trust Trust? In \emph{Trust: Making and
  Breaking Cooperative Relations}, 213-237.
\item
  Sabater, J., \& Sierra, C. (2005). Review on Computational Trust and
  Reputation Models. \emph{Artificial Intelligence Review}, 24(1),
  33-60.
\end{enumerate}
\end{frame}

\begin{frame}{Adversarial ML}
\protect\phantomsection\label{adversarial-ml}
\begin{enumerate}
\item
  Goodfellow, I. J., Shlens, J., \& Szegedy, C. (2015). Explaining and
  Harnessing Adversarial Examples. \emph{ICLR 2015}.
\item
  Carlini, N., \& Wagner, D. (2017). Towards Evaluating the Robustness
  of Neural Networks. \emph{IEEE S\&P 2017}.
\end{enumerate}
\end{frame}

\begin{frame}{Formal Verification}
\protect\phantomsection\label{formal-verification}
\begin{enumerate}
\item
  Clarke, E. M., Grumberg, O., \& Peled, D. A. (1999). \emph{Model
  Checking}. MIT Press.
\item
  Alur, R. (2015). \emph{Principles of Cyber-Physical Systems}. MIT
  Press.
\end{enumerate}
\end{frame}

\begin{frame}{Cognitive Security}
\protect\phantomsection\label{cognitive-security}
\begin{enumerate}
\item
  Waltzman, R. (2017). The Weaponization of Information: The Need for
  Cognitive Security. \emph{RAND Corporation}.
\item
  Beskow, D. M., \& Carley, K. M. (2019). Social Cybersecurity: An
  Emerging National Security Requirement. \emph{Military Review}, 99(2),
  117.
\end{enumerate}
\end{frame}

\begin{frame}{Agent Frameworks}
\protect\phantomsection\label{agent-frameworks}
\begin{enumerate}
\item
  LangChain. (2023). LangGraph: Build Stateful Multi-Actor Applications.
  \emph{Documentation}.
\item
  CrewAI. (2024). Framework for Orchestrating Role-Playing, Autonomous
  AI Agents.
\item
  Anthropic. (2024). Claude Code: AI-Powered Software Engineering.
\end{enumerate}
\end{frame}

\begin{frame}{2025 Agentic AI Security}
\protect\phantomsection\label{agentic-ai-security}
\begin{enumerate}
\item
  OWASP Foundation. (2025). OWASP Top 10 for LLM Applications 2025.
\item
  OWASP GenAI Security Project. (2025). OWASP Top 10 for Agentic
  Applications 2026.
\item
  Chen, W., Zhang, Y., \& Liu, J. (2025). A Multi-Agent LLM Defense
  Pipeline Against Prompt Injection Attacks. \emph{arXiv:2509.14285}.
\item
  Jo, Y., Kim, S., \& Park, J. (2025). Byzantine-Robust Decentralized
  Coordination of LLM Agents. \emph{arXiv:2507.14928}.
\item
  Wang, H., Li, X., \& Chen, Y. (2025). Rethinking the Reliability of
  Multi-agent System: A Perspective from Byzantine Fault Tolerance.
  \emph{arXiv:2511.10400}.
\item
  Debenedetti, E., Zhang, J., \& Carlini, N. (2025). Adaptive Attacks
  Break Defenses Against Indirect Prompt Injection Attacks on LLM
  Agents. \emph{NAACL 2025 Findings}.
\item
  Li, Z., Wang, T., \& Zhang, L. (2025). Prompt Injection Attack to Tool
  Selection in LLM Agents. \emph{arXiv:2504.19793}.
\item
  Cloud Security Alliance. (2025). Cognitive Degradation Resilience for
  Agentic AI.
\item
  Chen, X., Liu, Y., \& Wang, Z. (2025). AI Agents Under Threat: A
  Survey of Key Security Challenges and Future Pathways. \emph{ACM
  Computing Surveys}.
\end{enumerate}
\end{frame}

\begin{frame}{Eusocial Intelligence and Swarm Systems}
\protect\phantomsection\label{eusocial-intelligence-and-swarm-systems}
\begin{enumerate}
\item
  Wilson, E. O. (1971). \emph{The Insect Societies}. Belknap Press of
  Harvard University Press.
\item
  Hölldobler, B., \& Wilson, E. O. (1990). \emph{The Ants}. Belknap
  Press of Harvard University Press.
\item
  Bonabeau, E., Dorigo, M., \& Theraulaz, G. (1999). \emph{Swarm
  Intelligence: From Natural to Artificial Systems}. Oxford University
  Press.
\item
  Seeley, T. D. (2010). \emph{Honeybee Democracy}. Princeton University
  Press.
\item
  Grassé, P.-P. (1959). La reconstruction du nid et les coordinations
  interindividuelles chez Bellicositermes natalensis et Cubitermes sp.
  La théorie de la stigmergie. \emph{Insectes Sociaux}, 6(1), 41-80.
\item
  Lenoir, A., D'Ettorre, P., Errard, C., \& Hefetz, A. (2001). Chemical
  Ecology and Social Parasitism in Ants. \emph{Annual Review of
  Entomology}, 46, 573-599.
\item
  Kilner, R. M., \& Langmore, N. E. (2011). Cuckoos Versus Hosts in
  Insects and Birds: Adaptations, Counter-adaptations and Outcomes.
  \emph{Biological Reviews}, 86, 836-852.
\item
  Couzin, I. D. (2009). Collective Cognition in Animal Groups.
  \emph{Trends in Cognitive Sciences}, 13(1), 36-43.
\item
  Detrain, C., \& Deneubourg, J.-L. (2006). Self-Organized Structures in
  a Superorganism: Do Ants ``Behave'' Like Molecules? \emph{Physics of
  Life Reviews}, 3(3), 162-187.
\item
  Pratt, S. C. (2005). Quorum Sensing by Encounter Rates in the Ant
  Temnothorax albipennis. \emph{Behavioral Ecology}, 16(2), 488-496.
\end{enumerate}
\end{frame}

\end{document}
