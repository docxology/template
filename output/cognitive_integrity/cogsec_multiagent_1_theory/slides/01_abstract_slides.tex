% Options for packages loaded elsewhere
\PassOptionsToPackage{unicode}{hyperref}
\PassOptionsToPackage{hyphens}{url}
\documentclass[
  ignorenonframetext,
]{beamer}
\newif\ifbibliography
\usepackage{pgfpages}
\setbeamertemplate{caption}[numbered]
\setbeamertemplate{caption label separator}{: }
\setbeamercolor{caption name}{fg=normal text.fg}
\beamertemplatenavigationsymbolsempty
% remove section numbering
\setbeamertemplate{part page}{
  \centering
  \begin{beamercolorbox}[sep=16pt,center]{part title}
    \usebeamerfont{part title}\insertpart\par
  \end{beamercolorbox}
}
\setbeamertemplate{section page}{
  \centering
  \begin{beamercolorbox}[sep=12pt,center]{section title}
    \usebeamerfont{section title}\insertsection\par
  \end{beamercolorbox}
}
\setbeamertemplate{subsection page}{
  \centering
  \begin{beamercolorbox}[sep=8pt,center]{subsection title}
    \usebeamerfont{subsection title}\insertsubsection\par
  \end{beamercolorbox}
}
% Prevent slide breaks in the middle of a paragraph
\widowpenalties 1 10000
\raggedbottom
\AtBeginPart{
  \frame{\partpage}
}
\AtBeginSection{
  \ifbibliography
  \else
    \frame{\sectionpage}
  \fi
}
\AtBeginSubsection{
  \frame{\subsectionpage}
}
\usepackage{iftex}
\ifPDFTeX
  \usepackage[T1]{fontenc}
  \usepackage[utf8]{inputenc}
  \usepackage{textcomp} % provide euro and other symbols
\else % if luatex or xetex
  \usepackage{unicode-math} % this also loads fontspec
  \defaultfontfeatures{Scale=MatchLowercase}
  \defaultfontfeatures[\rmfamily]{Ligatures=TeX,Scale=1}
\fi
\usepackage{lmodern}
\ifPDFTeX\else
  % xetex/luatex font selection
\fi
% Use upquote if available, for straight quotes in verbatim environments
\IfFileExists{upquote.sty}{\usepackage{upquote}}{}
\IfFileExists{microtype.sty}{% use microtype if available
  \usepackage[]{microtype}
  \UseMicrotypeSet[protrusion]{basicmath} % disable protrusion for tt fonts
}{}
\makeatletter
\@ifundefined{KOMAClassName}{% if non-KOMA class
  \IfFileExists{parskip.sty}{%
    \usepackage{parskip}
  }{% else
    \setlength{\parindent}{0pt}
    \setlength{\parskip}{6pt plus 2pt minus 1pt}}
}{% if KOMA class
  \KOMAoptions{parskip=half}}
\makeatother
\setlength{\emergencystretch}{3em} % prevent overfull lines
\providecommand{\tightlist}{%
  \setlength{\itemsep}{0pt}\setlength{\parskip}{0pt}}
\usepackage{bookmark}
\IfFileExists{xurl.sty}{\usepackage{xurl}}{} % add URL line breaks if available
\urlstyle{same}
\hypersetup{
  hidelinks,
  pdfcreator={LaTeX via pandoc}}

\author{\texorpdfstring{}{}}
\date{}

\begin{document}

\begin{frame}{Abstract}
\protect\phantomsection\label{abstract}
Multiagent AI systems introduce cognitive attack surfaces absent in
single-model inference. When agents delegate to agents, forming beliefs
about beliefs through recursive trust hierarchies, manipulation of
reasoning processes---rather than mere data corruption---becomes a
primary security concern. This paper presents the Cognitive Integrity
Framework (CIF), providing formal foundations for cognitive security in
multiagent operators. We develop four interconnected theoretical
contributions: a Trust Calculus with bounded delegation (exponential
\(\delta^d\) decay) that prevents trust amplification through delegation
chains; a Defense Composition Algebra with series and parallel
composition theorems establishing multiplicative detection bounds;
Information-Theoretic Limits relating stealth constraints to maximum
attack impact through a fundamental stealth-impact tradeoff; and a
formal Adversary Hierarchy (\(\Omega_1\)--\(\Omega_5\)) characterizing
external, peripheral, agent-level, coordination, and systemic threats
with increasing capability and decreasing detectability. The framework
provides complete coverage of the OWASP Top 10 for Agentic Applications
through formal threat models grounded in cognitive state manipulation
rather than traditional input/output filtering.

CIF bridges classical security concepts with the cognitive requirements
of agentic systems. We extend Byzantine fault tolerance to cognitive
manipulation---agents that appear functional but hold corrupted
beliefs---and adapt trust management systems to continuous trust
evolution with provable decay bounds. The framework formalizes five
architectural defense mechanisms (cognitive firewalls, belief
sandboxing, behavioral tripwires, provenance tracking, Byzantine
consensus) with composition rules enabling formal reasoning about
layered security. Technical foundations include: operational semantics
for message passing and trust updates; invariants for belief integrity,
goal preservation, and trust boundedness; model checking configurations
for safety property verification; and a complete notation system for
attack parameterization, defense specification, and cognitive state
representation. This is Part 1 of a three-part series: Part 1 (this
paper, DOI: 10.5281/zenodo.18364119) presents formal foundations and
theoretical analysis; Part 2 (DOI: 10.5281/zenodo.18364128) provides
computational validation and implementation; Part 3 (DOI:
10.5281/zenodo.18364130) offers practical deployment guidance. The
framework will continue to be developed and versioned at
\url{https://github.com/docxology/cognitive_integrity/}.
\end{frame}

\end{document}
