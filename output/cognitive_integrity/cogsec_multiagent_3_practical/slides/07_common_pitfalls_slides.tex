% Options for packages loaded elsewhere
\PassOptionsToPackage{unicode}{hyperref}
\PassOptionsToPackage{hyphens}{url}
\documentclass[
  ignorenonframetext,
]{beamer}
\newif\ifbibliography
\usepackage{pgfpages}
\setbeamertemplate{caption}[numbered]
\setbeamertemplate{caption label separator}{: }
\setbeamercolor{caption name}{fg=normal text.fg}
\beamertemplatenavigationsymbolsempty
% remove section numbering
\setbeamertemplate{part page}{
  \centering
  \begin{beamercolorbox}[sep=16pt,center]{part title}
    \usebeamerfont{part title}\insertpart\par
  \end{beamercolorbox}
}
\setbeamertemplate{section page}{
  \centering
  \begin{beamercolorbox}[sep=12pt,center]{section title}
    \usebeamerfont{section title}\insertsection\par
  \end{beamercolorbox}
}
\setbeamertemplate{subsection page}{
  \centering
  \begin{beamercolorbox}[sep=8pt,center]{subsection title}
    \usebeamerfont{subsection title}\insertsubsection\par
  \end{beamercolorbox}
}
% Prevent slide breaks in the middle of a paragraph
\widowpenalties 1 10000
\raggedbottom
\AtBeginPart{
  \frame{\partpage}
}
\AtBeginSection{
  \ifbibliography
  \else
    \frame{\sectionpage}
  \fi
}
\AtBeginSubsection{
  \frame{\subsectionpage}
}
\usepackage{iftex}
\ifPDFTeX
  \usepackage[T1]{fontenc}
  \usepackage[utf8]{inputenc}
  \usepackage{textcomp} % provide euro and other symbols
\else % if luatex or xetex
  \usepackage{unicode-math} % this also loads fontspec
  \defaultfontfeatures{Scale=MatchLowercase}
  \defaultfontfeatures[\rmfamily]{Ligatures=TeX,Scale=1}
\fi
\usepackage{lmodern}
\ifPDFTeX\else
  % xetex/luatex font selection
\fi
% Use upquote if available, for straight quotes in verbatim environments
\IfFileExists{upquote.sty}{\usepackage{upquote}}{}
\IfFileExists{microtype.sty}{% use microtype if available
  \usepackage[]{microtype}
  \UseMicrotypeSet[protrusion]{basicmath} % disable protrusion for tt fonts
}{}
\makeatletter
\@ifundefined{KOMAClassName}{% if non-KOMA class
  \IfFileExists{parskip.sty}{%
    \usepackage{parskip}
  }{% else
    \setlength{\parindent}{0pt}
    \setlength{\parskip}{6pt plus 2pt minus 1pt}}
}{% if KOMA class
  \KOMAoptions{parskip=half}}
\makeatother
\usepackage{longtable,booktabs,array}
\newcounter{none} % for unnumbered tables
\usepackage{calc} % for calculating minipage widths
\usepackage{caption}
% Make caption package work with longtable
\makeatletter
\def\fnum@table{\tablename~\thetable}
\makeatother
\usepackage{graphicx}
\makeatletter
\newsavebox\pandoc@box
\newcommand*\pandocbounded[1]{% scales image to fit in text height/width
  \sbox\pandoc@box{#1}%
  \Gscale@div\@tempa{\textheight}{\dimexpr\ht\pandoc@box+\dp\pandoc@box\relax}%
  \Gscale@div\@tempb{\linewidth}{\wd\pandoc@box}%
  \ifdim\@tempb\p@<\@tempa\p@\let\@tempa\@tempb\fi% select the smaller of both
  \ifdim\@tempa\p@<\p@\scalebox{\@tempa}{\usebox\pandoc@box}%
  \else\usebox{\pandoc@box}%
  \fi%
}
% Set default figure placement to htbp
\def\fps@figure{htbp}
\makeatother
\setlength{\emergencystretch}{3em} % prevent overfull lines
\providecommand{\tightlist}{%
  \setlength{\itemsep}{0pt}\setlength{\parskip}{0pt}}
\usepackage{bookmark}
\IfFileExists{xurl.sty}{\usepackage{xurl}}{} % add URL line breaks if available
\urlstyle{same}
\hypersetup{
  hidelinks,
  pdfcreator={LaTeX via pandoc}}

\author{\texorpdfstring{}{}}
\date{}

\begin{document}

\begin{frame}
\newpage
\end{frame}

\begin{frame}{Common Pitfalls}
\protect\phantomsection\label{sec:common-pitfalls}
This section documents anti-patterns observed in multiagent deployments.
Each entry describes the pattern, its consequences, and specific
mitigations. Figure \ref{fig:pitfall-severity} ranks these pitfalls by
severity to guide remediation prioritization.

\begin{figure}
\centering
\includegraphics[width=0.9\linewidth,height=\textheight,keepaspectratio,alt={Common Deployment Pitfalls by Severity. This chart ranks the eight most common cognitive security anti-patterns by severity (scale 1-5). The two most critical pitfalls---Implicit Trust in Outputs and Missing Input Validation---both relate to failing to treat agent communications and external content as potentially adversarial. Colors indicate pitfall category: red (security), orange (operational), blue (design). Address critical (5) and high (4) severity items before production deployment.}]{../figures/pitfall_severity.pdf}
\caption{Common Deployment Pitfalls by Severity. This chart ranks the
eight most common cognitive security anti-patterns by severity (scale
1-5). The two most critical pitfalls---Implicit Trust in Outputs and
Missing Input Validation---both relate to failing to treat agent
communications and external content as potentially adversarial. Colors
indicate pitfall category: red (security), orange (operational), blue
(design). Address critical (5) and high (4) severity items before
production deployment.}\label{fig:pitfall-severity}
\end{figure}
\end{frame}

\begin{frame}
\begin{block}{Pitfall 1: Implicit Trust}
\protect\phantomsection\label{pitfall-1-implicit-trust}
\textbf{Pattern}: Treating all inter-agent communication as trusted by
default.

\textbf{Indicators}: - No source verification on agent messages - All
agents have equal authority regardless of role - Delegation without
bounds or decay

\textbf{Consequences}: - Single compromised agent influences entire
system - Trust amplification attacks succeed (see Part 1, Section 3.4) -
No containment of adversarial content

\textbf{Mitigation}: 1. Implement explicit trust scoring on inter-agent
channels 2. Require minimum trust thresholds for consequential actions
3. Apply delegation decay (δ \textless{} 1 per hop) 4. Verify source on
every inter-agent message
\end{block}
\end{frame}

\begin{frame}
\begin{block}{Pitfall 2: Security as Afterthought}
\protect\phantomsection\label{pitfall-2-security-as-afterthought}
\textbf{Pattern}: Adding cognitive security after architecture is
finalized.

\textbf{Indicators}: - Security checks only at external interfaces -
Core agent logic has no security awareness - Belief provenance untracked

\textbf{Consequences}: - Bypass opportunities at integration points -
Performance overhead from external security layers - Incomplete attack
surface coverage

\textbf{Mitigation}: 1. Design cognitive security into architecture from
the start 2. Embed trust checks in delegation logic 3. Build provenance
tracking into belief management 4. Include security constraints in agent
system prompts
\end{block}
\end{frame}

\begin{frame}
\begin{block}{Pitfall 3: Uncalibrated Thresholds}
\protect\phantomsection\label{pitfall-3-uncalibrated-thresholds}
\textbf{Pattern}: Setting security thresholds without understanding
tradeoffs.

\textbf{Indicators}: - Thresholds copied from examples without
adjustment - Same thresholds for all contexts - No testing against
representative attacks

\textbf{Consequences}: - Too strict: high false positive rate, user
friction - Too permissive: attacks succeed undetected - Settings
mismatched to actual risk profile

\textbf{Mitigation}: 1. Assess risk profile before configuring (see
Section 6) 2. Test thresholds against representative attack samples
(Part 2 corpus) 3. Monitor false positive/negative rates in production
4. Adjust based on operational feedback
\end{block}
\end{frame}

\begin{frame}
\begin{block}{Pitfall 4: Individual-Only Security}
\protect\phantomsection\label{pitfall-4-individual-only-security}
\textbf{Pattern}: Focusing on single-agent security while ignoring
multi-agent attack surfaces.

\textbf{Indicators}: - No consensus mechanism for critical decisions -
Agent count changes without verification - No Sybil resistance

\textbf{Consequences}: - Fake agents influence collective decisions -
Consensus manipulation - Coordination failures masked as normal
disagreement

\textbf{Mitigation}: 1. Implement Byzantine consensus for critical
collective decisions 2. Require agent authentication before vote
counting 3. Monitor for unusual coordination patterns 4. Apply quorum
requirements assuming adversarial presence

Part 1, Section 4.5 formalizes Byzantine consensus requirements.
\end{block}
\end{frame}

\begin{frame}
\begin{block}{Pitfall 5: Static Tripwires}
\protect\phantomsection\label{pitfall-5-static-tripwires}
\textbf{Pattern}: Deploying canary tripwires once without rotation.

\textbf{Indicators}: - Same canary values since deployment - No rotation
schedule - Predictable canary locations

\textbf{Consequences}: - Sophisticated adversaries learn to avoid
canaries - Effectiveness degrades over time - False confidence in
detection coverage

\textbf{Mitigation}: 1. Implement automated canary rotation 2. Vary
placement across agents and belief categories 3. Monitor canary check
patterns, not just modifications 4. Include non-obvious canaries
\end{block}
\end{frame}

\begin{frame}
\begin{block}{Pitfall 6: Ignoring Progressive Drift}
\protect\phantomsection\label{pitfall-6-ignoring-progressive-drift}
\textbf{Pattern}: Only alerting on large, sudden belief changes.

\textbf{Indicators}: - High threshold for drift alerts - No long-term
drift tracking - Static baseline

\textbf{Consequences}: - Sub-threshold changes accumulate undetected -
Beliefs slowly corrupted - Significant deviation without alert

\textbf{Mitigation}: 1. Use sliding window drift detection 2. Track
cumulative drift, not just per-update delta 3. Periodic baseline
comparison 4. Alert on trend as well as absolute magnitude

Part 1, Definition 5.1 formalizes drift scoring.
\end{block}
\end{frame}

\begin{frame}
\begin{block}{Pitfall 7: Insufficient Logging}
\protect\phantomsection\label{pitfall-7-insufficient-logging}
\textbf{Pattern}: Retaining insufficient information for post-incident
analysis.

\textbf{Indicators}: - Only final decisions logged - No belief state
history - Inter-agent messages disposed after processing

\textbf{Consequences}: - Cannot reconstruct attack path - Cannot
identify injection point - Cannot assess full impact scope

\textbf{Mitigation}: 1. Log all belief updates with provenance tags 2.
Retain inter-agent message history 3. Periodic cognitive state snapshots
4. Structured logging for causal analysis
\end{block}
\end{frame}

\begin{frame}
\begin{block}{Pitfall 8: Single-Orchestrator Reliance}
\protect\phantomsection\label{pitfall-8-single-orchestrator-reliance}
\textbf{Pattern}: Relying entirely on orchestrator integrity without
backup.

\textbf{Indicators}: - Single orchestrator for entire system - No
orchestrator monitoring - Workers unconditionally trust orchestrator

\textbf{Consequences}: - Orchestrator compromise = total system
compromise - No recovery path - Complete trust inversion attack possible
(see Part 1, Section 2.3)

\textbf{Mitigation}: 1. Consider multi-orchestrator architectures for
critical decisions 2. Monitor orchestrator behavior with same rigor as
agents 3. Workers verify orchestrator identity on critical commands 4.
Implement orchestrator-specific tripwires
\end{block}
\end{frame}

\begin{frame}
\begin{block}{Summary Checklist}
\protect\phantomsection\label{summary-checklist}
{\def\LTcaptype{none} % do not increment counter
\begin{longtable}[]{@{}lll@{}}
\toprule\noalign{}
Pitfall & Assessment & Status \\
\midrule\noalign{}
\endhead
Implicit trust & Trust scoring implemented? & ☐ \\
Security afterthought & Security in initial architecture? & ☐ \\
Uncalibrated thresholds & Thresholds tested against attacks? & ☐ \\
Individual-only security & Byzantine consensus deployed? & ☐ \\
Static tripwires & Canary rotation scheduled? & ☐ \\
Ignoring drift & Progressive drift monitoring? & ☐ \\
Insufficient logging & Full belief history retained? & ☐ \\
Single orchestrator & Orchestrator monitored? & ☐ \\
\bottomrule\noalign{}
\end{longtable}
}

Address unchecked items before production deployment.
\end{block}
\end{frame}

\end{document}
