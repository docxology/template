% Options for packages loaded elsewhere
\PassOptionsToPackage{unicode}{hyperref}
\PassOptionsToPackage{hyphens}{url}
\documentclass[
  ignorenonframetext,
]{beamer}
\newif\ifbibliography
\usepackage{pgfpages}
\setbeamertemplate{caption}[numbered]
\setbeamertemplate{caption label separator}{: }
\setbeamercolor{caption name}{fg=normal text.fg}
\beamertemplatenavigationsymbolsempty
% remove section numbering
\setbeamertemplate{part page}{
  \centering
  \begin{beamercolorbox}[sep=16pt,center]{part title}
    \usebeamerfont{part title}\insertpart\par
  \end{beamercolorbox}
}
\setbeamertemplate{section page}{
  \centering
  \begin{beamercolorbox}[sep=12pt,center]{section title}
    \usebeamerfont{section title}\insertsection\par
  \end{beamercolorbox}
}
\setbeamertemplate{subsection page}{
  \centering
  \begin{beamercolorbox}[sep=8pt,center]{subsection title}
    \usebeamerfont{subsection title}\insertsubsection\par
  \end{beamercolorbox}
}
% Prevent slide breaks in the middle of a paragraph
\widowpenalties 1 10000
\raggedbottom
\AtBeginPart{
  \frame{\partpage}
}
\AtBeginSection{
  \ifbibliography
  \else
    \frame{\sectionpage}
  \fi
}
\AtBeginSubsection{
  \frame{\subsectionpage}
}
\usepackage{iftex}
\ifPDFTeX
  \usepackage[T1]{fontenc}
  \usepackage[utf8]{inputenc}
  \usepackage{textcomp} % provide euro and other symbols
\else % if luatex or xetex
  \usepackage{unicode-math} % this also loads fontspec
  \defaultfontfeatures{Scale=MatchLowercase}
  \defaultfontfeatures[\rmfamily]{Ligatures=TeX,Scale=1}
\fi
\usepackage{lmodern}
\ifPDFTeX\else
  % xetex/luatex font selection
\fi
% Use upquote if available, for straight quotes in verbatim environments
\IfFileExists{upquote.sty}{\usepackage{upquote}}{}
\IfFileExists{microtype.sty}{% use microtype if available
  \usepackage[]{microtype}
  \UseMicrotypeSet[protrusion]{basicmath} % disable protrusion for tt fonts
}{}
\makeatletter
\@ifundefined{KOMAClassName}{% if non-KOMA class
  \IfFileExists{parskip.sty}{%
    \usepackage{parskip}
  }{% else
    \setlength{\parindent}{0pt}
    \setlength{\parskip}{6pt plus 2pt minus 1pt}}
}{% if KOMA class
  \KOMAoptions{parskip=half}}
\makeatother
\setlength{\emergencystretch}{3em} % prevent overfull lines
\providecommand{\tightlist}{%
  \setlength{\itemsep}{0pt}\setlength{\parskip}{0pt}}
\usepackage{bookmark}
\IfFileExists{xurl.sty}{\usepackage{xurl}}{} % add URL line breaks if available
\urlstyle{same}
\hypersetup{
  hidelinks,
  pdfcreator={LaTeX via pandoc}}

\author{\texorpdfstring{}{}}
\date{}

\begin{document}

\begin{frame}
\newpage
\end{frame}

\section{Introduction}\label{sec:intro}

\begin{frame}{The Emergence of Cognitive Security}
\protect\phantomsection\label{the-emergence-of-cognitive-security}
The deployment of multiagent AI systems in production environments
represents a fundamental shift in how we must conceptualize security.
Traditional cybersecurity focuses on protecting data integrity, ensuring
authorized access, and maintaining system availability. However, when AI
agents possess beliefs, pursue goals, and reason about the world, a new
attack surface emerges: the cognitive processes themselves
\cite{waltzman2017weaponization,friedman2024cogsec}.

Cognitive security, as articulated by Friedman and the COGSEC
collaborative \cite{cogsec2024atlas}, addresses the protection of
cognitive processes---the beliefs, goals, reasoning patterns, and
decision-making capabilities---of intelligent agents, whether human or
artificial. In multiagent systems, this concern is amplified: agents
must trust information from other agents, delegate tasks across trust
boundaries, and maintain coherent beliefs despite potentially
adversarial inputs. An attacker who can manipulate what an agent
\emph{believes} may achieve far more damage than one who merely corrupts
stored data.

The 2024-2025 period has witnessed an explosion of agentic AI
deployments: autonomous coding assistants, research agents, customer
service orchestrators, and multi-model reasoning systems have moved from
research prototypes to production infrastructure
\cite{wu2023autogen,hong2023metagpt}. With this deployment comes an
urgent need for practical guidance on securing these systems against
cognitive attacks---prompt injections that propagate through agent
networks, trust exploitation that enables privilege escalation, and
belief manipulation that corrupts organizational knowledge bases.
\end{frame}

\begin{frame}{Why This Paper Matters}
\protect\phantomsection\label{why-this-paper-matters}
Part 1 of this series establishes the theoretical foundations of
cognitive security for multiagent operators, formalizing trust calculus,
defense composition algebras, and integrity properties. Part 2
demonstrates that these theoretical constructs translate to measurable
protection in empirical evaluations. This paper---Part 3---addresses the
question that practitioners ask most urgently: \emph{how do I actually
deploy and operate a multiagent system with cognitive security in mind?}

The gap between theoretical security guarantees and practical
implementation is substantial. Formal verification proves that certain
attack patterns cannot succeed under specified conditions, but
real-world deployments face operational pressures, resource constraints,
and adversaries who adapt to defensive measures. Security teams need
checklists, configuration guidance, and operational procedures that can
be implemented without requiring expertise in formal methods.

Moreover, the threat landscape for agentic AI is evolving rapidly. The
OWASP Top 10 for Agentic Applications (2026) \cite{owasp2025agentic}
documents attack patterns that did not exist two years prior. Adaptive
adversaries have demonstrated that static defenses fail against
iterative attacks \cite{adaptive2025attacks}, requiring organizations to
adopt defense-in-depth postures that compose multiple protective
mechanisms. This paper translates the defense composition theorems of
Part 1 into practical deployment patterns.
\end{frame}

\begin{frame}{Scope and Audience}
\protect\phantomsection\label{scope-and-audience}
We focus on actionable guidance rather than theoretical completeness.
Readers seeking formal foundations should consult Part 1 (DOI:
10.5281/zenodo.18364119). Those interested in empirical
validation---detection rates, false positive analysis, performance
overhead measurements---should refer to Part 2 (DOI:
10.5281/zenodo.18364128).

This guidance serves:

\begin{itemize}
\tightlist
\item
  \textbf{Security practitioners} evaluating multiagent deployments
  against cognitive attack surfaces, who need to understand how
  traditional security controls map to AI-specific threats
\item
  \textbf{Developers} building agentic applications who want security
  integrated from the start, avoiding the technical debt of retrofitting
  defenses to production systems
\item
  \textbf{Operations teams} managing production multiagent systems who
  need monitoring, alerting, and incident response procedures adapted to
  cognitive attacks
\item
  \textbf{Compliance and risk teams} mapping cognitive security to
  existing risk frameworks such as NIST AI RMF, ISO 42001, and
  sector-specific regulations
\end{itemize}
\end{frame}

\begin{frame}{What You Will Learn}
\protect\phantomsection\label{what-you-will-learn}
This paper provides:

\begin{enumerate}
\item
  \textbf{Operator Posture Assessment} (Section 2): A framework for
  evaluating your organization's cognitive security readiness across
  five dimensions---visibility, trust management, defense layering,
  incident response, and continuous improvement. Most organizations
  operate at Level 1 (Reactive) or Level 2 (Basic); this section
  provides a roadmap to Level 4 (Proactive) operation.
\item
  \textbf{Human-Actionable Checklist} (Section 3): Step-by-step
  deployment guidance organized by phase (pre-deployment, deployment,
  post-deployment). Each item links to the formal justification in Part
  1 for readers who want theoretical grounding.
\item
  \textbf{Agent Self-Monitoring Guidelines} (Section 4):
  Machine-readable rules that agents can apply during operation to
  detect signs of cognitive compromise. These guidelines can be
  incorporated into system prompts or reasoning frameworks.
\item
  \textbf{Deployment Configuration} (Section 5): Parameter
  recommendations calibrated to different risk profiles---from
  development sandboxes to high-stakes production environments handling
  sensitive operations.
\item
  \textbf{Risk Assessment Methodology} (Section 6): A structured
  approach to threat modeling for multiagent systems, adapted from
  established frameworks but tailored to cognitive attack vectors.
\item
  \textbf{Common Pitfalls and Mitigations} (Section 7): Observed
  anti-patterns from real-world deployments, with specific
  recommendations for avoiding or correcting them.
\end{enumerate}
\end{frame}

\begin{frame}{The Cognitive Security Mindset}
\protect\phantomsection\label{the-cognitive-security-mindset}
Securing multiagent systems requires a shift in perspective. Traditional
security asks: ``Who has access to this data? Is this request
authorized? Is this input sanitized?'' Cognitive security adds: ``What
does this agent believe? Who influenced those beliefs? Are those beliefs
consistent with verified ground truth?''

This perspective reveals attack surfaces invisible to traditional
security tools. A prompt injection that passes input validation and
executes within authorized permissions may still corrupt an agent's
beliefs about what actions are appropriate. Trust exploitation that
operates entirely within the formal permission model may enable
unauthorized capability escalation through the social layer of agent
interaction.

The practical guidance in this paper operationalizes the theoretical
insight from Part 1: that cognitive security requires protecting not
just the inputs and outputs of AI systems, but the \emph{reasoning
processes} that connect them. We provide concrete tools for achieving
this protection in production environments.

Throughout, we emphasize that cognitive security is not a one-time
implementation but an ongoing operational practice. Adversaries adapt,
systems evolve, and the threat landscape shifts. The goal is to
establish procedures and capabilities that enable organizations to
maintain cognitive integrity despite this dynamism.
\end{frame}

\end{document}
