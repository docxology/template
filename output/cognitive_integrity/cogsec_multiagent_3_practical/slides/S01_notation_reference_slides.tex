% Options for packages loaded elsewhere
\PassOptionsToPackage{unicode}{hyperref}
\PassOptionsToPackage{hyphens}{url}
\documentclass[
  ignorenonframetext,
]{beamer}
\newif\ifbibliography
\usepackage{pgfpages}
\setbeamertemplate{caption}[numbered]
\setbeamertemplate{caption label separator}{: }
\setbeamercolor{caption name}{fg=normal text.fg}
\beamertemplatenavigationsymbolsempty
% remove section numbering
\setbeamertemplate{part page}{
  \centering
  \begin{beamercolorbox}[sep=16pt,center]{part title}
    \usebeamerfont{part title}\insertpart\par
  \end{beamercolorbox}
}
\setbeamertemplate{section page}{
  \centering
  \begin{beamercolorbox}[sep=12pt,center]{section title}
    \usebeamerfont{section title}\insertsection\par
  \end{beamercolorbox}
}
\setbeamertemplate{subsection page}{
  \centering
  \begin{beamercolorbox}[sep=8pt,center]{subsection title}
    \usebeamerfont{subsection title}\insertsubsection\par
  \end{beamercolorbox}
}
% Prevent slide breaks in the middle of a paragraph
\widowpenalties 1 10000
\raggedbottom
\AtBeginPart{
  \frame{\partpage}
}
\AtBeginSection{
  \ifbibliography
  \else
    \frame{\sectionpage}
  \fi
}
\AtBeginSubsection{
  \frame{\subsectionpage}
}
\usepackage{iftex}
\ifPDFTeX
  \usepackage[T1]{fontenc}
  \usepackage[utf8]{inputenc}
  \usepackage{textcomp} % provide euro and other symbols
\else % if luatex or xetex
  \usepackage{unicode-math} % this also loads fontspec
  \defaultfontfeatures{Scale=MatchLowercase}
  \defaultfontfeatures[\rmfamily]{Ligatures=TeX,Scale=1}
\fi
\usepackage{lmodern}
\ifPDFTeX\else
  % xetex/luatex font selection
\fi
% Use upquote if available, for straight quotes in verbatim environments
\IfFileExists{upquote.sty}{\usepackage{upquote}}{}
\IfFileExists{microtype.sty}{% use microtype if available
  \usepackage[]{microtype}
  \UseMicrotypeSet[protrusion]{basicmath} % disable protrusion for tt fonts
}{}
\makeatletter
\@ifundefined{KOMAClassName}{% if non-KOMA class
  \IfFileExists{parskip.sty}{%
    \usepackage{parskip}
  }{% else
    \setlength{\parindent}{0pt}
    \setlength{\parskip}{6pt plus 2pt minus 1pt}}
}{% if KOMA class
  \KOMAoptions{parskip=half}}
\makeatother
\usepackage{longtable,booktabs,array}
\newcounter{none} % for unnumbered tables
\usepackage{calc} % for calculating minipage widths
\usepackage{caption}
% Make caption package work with longtable
\makeatletter
\def\fnum@table{\tablename~\thetable}
\makeatother
\setlength{\emergencystretch}{3em} % prevent overfull lines
\providecommand{\tightlist}{%
  \setlength{\itemsep}{0pt}\setlength{\parskip}{0pt}}
\usepackage{bookmark}
\IfFileExists{xurl.sty}{\usepackage{xurl}}{} % add URL line breaks if available
\urlstyle{same}
\hypersetup{
  hidelinks,
  pdfcreator={LaTeX via pandoc}}

\author{\texorpdfstring{}{}}
\date{}

\begin{document}

\begin{frame}
\newpage
\end{frame}

\begin{frame}{Notation Reference}
\protect\phantomsection\label{sec:notation-reference}
This paper intentionally minimizes mathematical notation to maximize
accessibility. Where notation is used, it follows the Cognitive
Integrity Framework (CIF) formal specification defined in Part 1 of this
series.

\begin{block}{Minimal Notation Used}
\protect\phantomsection\label{minimal-notation-used}
{\def\LTcaptype{none} % do not increment counter
\begin{longtable}[]{@{}
  >{\raggedright\arraybackslash}p{(\linewidth - 4\tabcolsep) * \real{0.2424}}
  >{\raggedright\arraybackslash}p{(\linewidth - 4\tabcolsep) * \real{0.2727}}
  >{\raggedright\arraybackslash}p{(\linewidth - 4\tabcolsep) * \real{0.4848}}@{}}
\toprule\noalign{}
\begin{minipage}[b]{\linewidth}\raggedright
Symbol
\end{minipage} & \begin{minipage}[b]{\linewidth}\raggedright
Meaning
\end{minipage} & \begin{minipage}[b]{\linewidth}\raggedright
Plain Language
\end{minipage} \\
\midrule\noalign{}
\endhead
δ & Trust decay factor & ``Delegated trust decreases by this factor at
each step'' \\
n & Agent count & ``Number of agents in the system'' \\
f & Byzantine agents & ``Maximum number of malicious agents
tolerated'' \\
\bottomrule\noalign{}
\end{longtable}
}
\end{block}

\begin{block}{Trust Decay Explanation}
\protect\phantomsection\label{trust-decay-explanation}
When we write δ = 0.9, this means:

\begin{itemize}
\tightlist
\item
  Direct trust: 100\% of assigned value
\item
  One delegation: 90\% of source trust
\item
  Two delegations: 81\% of source trust
\item
  Three delegations: 73\% of source trust
\end{itemize}

A lower δ (e.g., 0.85) means faster decay, providing more security but
limiting delegation utility.
\end{block}

\begin{block}{Byzantine Tolerance Explanation}
\protect\phantomsection\label{byzantine-tolerance-explanation}
When we say n ≥ 3f + 1:

\begin{itemize}
\tightlist
\item
  To tolerate 1 malicious agent, need at least 4 agents
\item
  To tolerate 2 malicious agents, need at least 7 agents
\item
  To tolerate 3 malicious agents, need at least 10 agents
\end{itemize}
\end{block}

\begin{block}{Full Notation Reference}
\protect\phantomsection\label{full-notation-reference}
For complete formal definitions of all CIF notation, see:

\begin{itemize}
\tightlist
\item
  \textbf{Part 1: Supplementary Section S03: Notation Reference}
\end{itemize}

The formal specification includes \textasciitilde100 symbols covering:

\begin{itemize}
\tightlist
\item
  Agent cognitive state
\item
  Trust calculus operations
\item
  Defense mechanism parameters
\item
  Consensus and coordination
\item
  Information-theoretic bounds
\end{itemize}
\end{block}
\end{frame}

\end{document}
