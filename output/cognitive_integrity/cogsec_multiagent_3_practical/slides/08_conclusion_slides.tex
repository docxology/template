% Options for packages loaded elsewhere
\PassOptionsToPackage{unicode}{hyperref}
\PassOptionsToPackage{hyphens}{url}
\documentclass[
  ignorenonframetext,
]{beamer}
\newif\ifbibliography
\usepackage{pgfpages}
\setbeamertemplate{caption}[numbered]
\setbeamertemplate{caption label separator}{: }
\setbeamercolor{caption name}{fg=normal text.fg}
\beamertemplatenavigationsymbolsempty
% remove section numbering
\setbeamertemplate{part page}{
  \centering
  \begin{beamercolorbox}[sep=16pt,center]{part title}
    \usebeamerfont{part title}\insertpart\par
  \end{beamercolorbox}
}
\setbeamertemplate{section page}{
  \centering
  \begin{beamercolorbox}[sep=12pt,center]{section title}
    \usebeamerfont{section title}\insertsection\par
  \end{beamercolorbox}
}
\setbeamertemplate{subsection page}{
  \centering
  \begin{beamercolorbox}[sep=8pt,center]{subsection title}
    \usebeamerfont{subsection title}\insertsubsection\par
  \end{beamercolorbox}
}
% Prevent slide breaks in the middle of a paragraph
\widowpenalties 1 10000
\raggedbottom
\AtBeginPart{
  \frame{\partpage}
}
\AtBeginSection{
  \ifbibliography
  \else
    \frame{\sectionpage}
  \fi
}
\AtBeginSubsection{
  \frame{\subsectionpage}
}
\usepackage{iftex}
\ifPDFTeX
  \usepackage[T1]{fontenc}
  \usepackage[utf8]{inputenc}
  \usepackage{textcomp} % provide euro and other symbols
\else % if luatex or xetex
  \usepackage{unicode-math} % this also loads fontspec
  \defaultfontfeatures{Scale=MatchLowercase}
  \defaultfontfeatures[\rmfamily]{Ligatures=TeX,Scale=1}
\fi
\usepackage{lmodern}
\ifPDFTeX\else
  % xetex/luatex font selection
\fi
% Use upquote if available, for straight quotes in verbatim environments
\IfFileExists{upquote.sty}{\usepackage{upquote}}{}
\IfFileExists{microtype.sty}{% use microtype if available
  \usepackage[]{microtype}
  \UseMicrotypeSet[protrusion]{basicmath} % disable protrusion for tt fonts
}{}
\makeatletter
\@ifundefined{KOMAClassName}{% if non-KOMA class
  \IfFileExists{parskip.sty}{%
    \usepackage{parskip}
  }{% else
    \setlength{\parindent}{0pt}
    \setlength{\parskip}{6pt plus 2pt minus 1pt}}
}{% if KOMA class
  \KOMAoptions{parskip=half}}
\makeatother
\setlength{\emergencystretch}{3em} % prevent overfull lines
\providecommand{\tightlist}{%
  \setlength{\itemsep}{0pt}\setlength{\parskip}{0pt}}
\usepackage{bookmark}
\IfFileExists{xurl.sty}{\usepackage{xurl}}{} % add URL line breaks if available
\urlstyle{same}
\hypersetup{
  hidelinks,
  pdfcreator={LaTeX via pandoc}}

\author{\texorpdfstring{}{}}
\date{}

\begin{document}

\begin{frame}
\newpage
\end{frame}

\section{Conclusion}\label{sec:conclusion}

\begin{frame}{Summary of Practical Guidance}
\protect\phantomsection\label{summary-of-practical-guidance}
This paper translated the Cognitive Integrity Framework (CIF) from
formal theory and empirical validation into actionable guidance for
practitioners. Our key contributions include:

\textbf{Operator Posture Framework}: The four pillars---trust boundary
awareness, belief provenance consciousness, delegation hygiene, and
coordination integrity---provide a conceptual foundation for cognitive
security readiness assessment.

\textbf{Human-Actionable Checklists}: Step-by-step guidance for
pre-deployment, operational monitoring, and incident response enables
practitioners to implement cognitive security systematically.

\textbf{Agent-Readable Guidelines}: Machine-parseable security rules
enable AI agents to participate in their own cognitive security,
implementing continuous self-monitoring and threat response.

\textbf{Deployment Considerations}: Risk-profile-based configuration
guidance and architecture-specific recommendations enable appropriate
security posture calibration.

\textbf{Risk Assessment Methodology}: Systematic threat modeling for
cognitive attack surfaces helps organizations prioritize security
investments.

\textbf{Common Pitfalls Catalog}: Documented anti-patterns with concrete
mitigations help practitioners avoid known failure modes.
\end{frame}

\begin{frame}{Path Forward}
\protect\phantomsection\label{path-forward}
Cognitive security for multiagent operators remains an emerging
discipline. As these systems become ubiquitous in enterprise and
consumer contexts, the guidance in this paper represents a starting
point rather than an endpoint.

Organizations adopting multiagent AI should:

\begin{enumerate}
\tightlist
\item
  \textbf{Assess current posture} using the four-pillar framework
\item
  \textbf{Implement appropriate defenses} based on risk profile
\item
  \textbf{Monitor continuously} using the operational checklists
\item
  \textbf{Prepare for incidents} with documented response procedures
\item
  \textbf{Iterate and improve} as the threat landscape evolves
\end{enumerate}
\end{frame}

\begin{frame}{Paper Series Integration}
\protect\phantomsection\label{paper-series-integration}
This practical guidance builds on and integrates with:

\begin{itemize}
\tightlist
\item
  \textbf{Part 1 (Formal Foundations)}: Provides the theoretical basis
  for all recommendations
\item
  \textbf{Part 2 (Computational Validation)}: Demonstrates that these
  mechanisms work in practice
\end{itemize}

Together, the three papers provide a complete framework: formal
foundations establishing what cognitive security means, empirical
validation proving that mechanisms work, and practical guidance enabling
deployment.
\end{frame}

\begin{frame}{Final Recommendations}
\protect\phantomsection\label{final-recommendations}
For organizations deploying multiagent AI today:

\begin{enumerate}
\tightlist
\item
  \textbf{Start with awareness}: Recognize that cognitive attack
  surfaces exist
\item
  \textbf{Map trust assumptions}: Know where trust is assumed
  vs.~verified
\item
  \textbf{Implement bounded delegation}: Trust should decay with depth
\item
  \textbf{Deploy layered defense}: No single mechanism provides adequate
  protection
\item
  \textbf{Monitor continuously}: Cognitive integrity requires ongoing
  vigilance
\item
  \textbf{Prepare for attacks}: Incidents will occur; readiness
  determines impact
\end{enumerate}

The cognitive security posture you adopt today will determine your
resilience to the attacks of tomorrow.
\end{frame}

\end{document}
