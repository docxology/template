% Options for packages loaded elsewhere
\PassOptionsToPackage{unicode}{hyperref}
\PassOptionsToPackage{hyphens}{url}
\documentclass[
  ignorenonframetext,
]{beamer}
\newif\ifbibliography
\usepackage{pgfpages}
\setbeamertemplate{caption}[numbered]
\setbeamertemplate{caption label separator}{: }
\setbeamercolor{caption name}{fg=normal text.fg}
\beamertemplatenavigationsymbolsempty
% remove section numbering
\setbeamertemplate{part page}{
  \centering
  \begin{beamercolorbox}[sep=16pt,center]{part title}
    \usebeamerfont{part title}\insertpart\par
  \end{beamercolorbox}
}
\setbeamertemplate{section page}{
  \centering
  \begin{beamercolorbox}[sep=12pt,center]{section title}
    \usebeamerfont{section title}\insertsection\par
  \end{beamercolorbox}
}
\setbeamertemplate{subsection page}{
  \centering
  \begin{beamercolorbox}[sep=8pt,center]{subsection title}
    \usebeamerfont{subsection title}\insertsubsection\par
  \end{beamercolorbox}
}
% Prevent slide breaks in the middle of a paragraph
\widowpenalties 1 10000
\raggedbottom
\AtBeginPart{
  \frame{\partpage}
}
\AtBeginSection{
  \ifbibliography
  \else
    \frame{\sectionpage}
  \fi
}
\AtBeginSubsection{
  \frame{\subsectionpage}
}
\usepackage{iftex}
\ifPDFTeX
  \usepackage[T1]{fontenc}
  \usepackage[utf8]{inputenc}
  \usepackage{textcomp} % provide euro and other symbols
\else % if luatex or xetex
  \usepackage{unicode-math} % this also loads fontspec
  \defaultfontfeatures{Scale=MatchLowercase}
  \defaultfontfeatures[\rmfamily]{Ligatures=TeX,Scale=1}
\fi
\usepackage{lmodern}
\ifPDFTeX\else
  % xetex/luatex font selection
\fi
% Use upquote if available, for straight quotes in verbatim environments
\IfFileExists{upquote.sty}{\usepackage{upquote}}{}
\IfFileExists{microtype.sty}{% use microtype if available
  \usepackage[]{microtype}
  \UseMicrotypeSet[protrusion]{basicmath} % disable protrusion for tt fonts
}{}
\makeatletter
\@ifundefined{KOMAClassName}{% if non-KOMA class
  \IfFileExists{parskip.sty}{%
    \usepackage{parskip}
  }{% else
    \setlength{\parindent}{0pt}
    \setlength{\parskip}{6pt plus 2pt minus 1pt}}
}{% if KOMA class
  \KOMAoptions{parskip=half}}
\makeatother
\usepackage{longtable,booktabs,array}
\newcounter{none} % for unnumbered tables
\usepackage{calc} % for calculating minipage widths
\usepackage{caption}
% Make caption package work with longtable
\makeatletter
\def\fnum@table{\tablename~\thetable}
\makeatother
\usepackage{graphicx}
\makeatletter
\newsavebox\pandoc@box
\newcommand*\pandocbounded[1]{% scales image to fit in text height/width
  \sbox\pandoc@box{#1}%
  \Gscale@div\@tempa{\textheight}{\dimexpr\ht\pandoc@box+\dp\pandoc@box\relax}%
  \Gscale@div\@tempb{\linewidth}{\wd\pandoc@box}%
  \ifdim\@tempb\p@<\@tempa\p@\let\@tempa\@tempb\fi% select the smaller of both
  \ifdim\@tempa\p@<\p@\scalebox{\@tempa}{\usebox\pandoc@box}%
  \else\usebox{\pandoc@box}%
  \fi%
}
% Set default figure placement to htbp
\def\fps@figure{htbp}
\makeatother
\setlength{\emergencystretch}{3em} % prevent overfull lines
\providecommand{\tightlist}{%
  \setlength{\itemsep}{0pt}\setlength{\parskip}{0pt}}
\usepackage{bookmark}
\IfFileExists{xurl.sty}{\usepackage{xurl}}{} % add URL line breaks if available
\urlstyle{same}
\hypersetup{
  hidelinks,
  pdfcreator={LaTeX via pandoc}}

\author{\texorpdfstring{}{}}
\date{}

\begin{document}

\begin{frame}
\newpage
\end{frame}

\section{Human-Actionable Checklist}\label{sec:human-checklist}

\begin{frame}{Pre-Deployment Checklist}
\protect\phantomsection\label{pre-deployment-checklist}
Before deploying a multiagent system in production, verify the
following. Figure \ref{fig:checklist-flowchart} provides a visual
overview of the deployment phases and their associated verification
checkpoints.

\begin{figure}
\centering
\includegraphics[width=0.95\linewidth,height=\textheight,keepaspectratio,alt={Deployment Readiness Checklist. The cognitive security deployment lifecycle consists of four phases: Pre-Deployment (threat model completion, CIF component selection, trust boundary definition), Integration (firewall configuration, sandbox policies, tripwire placement), Testing (red team assessment, penetration testing, failure mode analysis), and Operational (continuous monitoring, alerting, incident response). Each phase must be completed before advancing to the next.}]{../figures/checklist_flowchart.pdf}
\caption{Deployment Readiness Checklist. The cognitive security
deployment lifecycle consists of four phases: Pre-Deployment (threat
model completion, CIF component selection, trust boundary definition),
Integration (firewall configuration, sandbox policies, tripwire
placement), Testing (red team assessment, penetration testing, failure
mode analysis), and Operational (continuous monitoring, alerting,
incident response). Each phase must be completed before advancing to the
next.}\label{fig:checklist-flowchart}
\end{figure}

\begin{block}{Architecture Review}
\protect\phantomsection\label{architecture-review}
\begin{itemize}
\tightlist
\item[$\square$]
  \textbf{Trust boundaries documented}: All points where trust is
  assumed vs.~verified are explicitly mapped
\item[$\square$]
  \textbf{Delegation limits configured}: Trust decay factor set
  (recommended: δ = 0.85-0.95)
\item[$\square$]
  \textbf{Agent authentication implemented}: All agents have verifiable
  identity
\item[$\square$]
  \textbf{Permission boundaries defined}: Each agent has explicit action
  restrictions
\end{itemize}
\end{block}

\begin{block}{Defense Configuration}
\protect\phantomsection\label{defense-configuration}
\begin{itemize}
\tightlist
\item[$\square$]
  \textbf{Cognitive firewall enabled}: Input classification active for
  all external content
\item[$\square$]
  \textbf{Belief sandboxing configured}: Unverified beliefs quarantined
  pending corroboration
\item[$\square$]
  \textbf{Tripwires planted}: Canary beliefs placed to detect
  manipulation
\item[$\square$]
  \textbf{Invariants defined}: Core security constraints specified and
  monitored
\end{itemize}
\end{block}

\begin{block}{Monitoring Setup}
\protect\phantomsection\label{monitoring-setup}
\begin{itemize}
\tightlist
\item[$\square$]
  \textbf{Drift detection active}: Belief distribution monitoring
  enabled
\item[$\square$]
  \textbf{Alert thresholds configured}: Warning and critical levels set
  appropriately
\item[$\square$]
  \textbf{Logging comprehensive}: All agent decisions and belief updates
  recorded
\item[$\square$]
  \textbf{Dashboards available}: Real-time visibility into cognitive
  state
\end{itemize}
\end{block}

\begin{block}{Incident Response Prepared}
\protect\phantomsection\label{incident-response-prepared}
\begin{itemize}
\tightlist
\item[$\square$]
  \textbf{Response procedures documented}: Steps for cognitive attack
  response defined
\item[$\square$]
  \textbf{Quarantine capability ready}: Ability to isolate compromised
  agents
\item[$\square$]
  \textbf{Rollback mechanism tested}: Can restore to known-good
  cognitive state
\item[$\square$]
  \textbf{Escalation path clear}: Who to contact for cognitive security
  incidents
\end{itemize}
\end{block}
\end{frame}

\begin{frame}{Operational Checklist (Daily/Weekly)}
\protect\phantomsection\label{operational-checklist-dailyweekly}
\begin{block}{Daily Monitoring}
\protect\phantomsection\label{daily-monitoring}
\begin{itemize}
\tightlist
\item[$\square$]
  \textbf{Review drift alerts}: Check for unusual belief changes
\item[$\square$]
  \textbf{Verify tripwire integrity}: Confirm canary beliefs unchanged
\item[$\square$]
  \textbf{Check trust metrics}: Monitor for unexpected trust score
  changes
\item[$\square$]
  \textbf{Review failed consensus}: Investigate any Byzantine fault
  indications
\end{itemize}
\end{block}

\begin{block}{Weekly Review}
\protect\phantomsection\label{weekly-review}
\begin{itemize}
\tightlist
\item[$\square$]
  \textbf{Analyze attack patterns}: Review blocked injection attempts
\item[$\square$]
  \textbf{Audit delegation chains}: Check for unusual delegation
  patterns
\item[$\square$]
  \textbf{Verify invariant compliance}: Confirm no invariant violations
\item[$\square$]
  \textbf{Update threat intel}: Incorporate new attack techniques into
  defenses
\end{itemize}
\end{block}
\end{frame}

\begin{frame}{Incident Response Checklist}
\protect\phantomsection\label{incident-response-checklist}
When a cognitive attack is suspected:

\begin{block}{Immediate Actions (First 15 Minutes)}
\protect\phantomsection\label{immediate-actions-first-15-minutes}
\begin{itemize}
\tightlist
\item[$\square$]
  \textbf{Preserve evidence}: Capture current cognitive state before any
  changes
\item[$\square$]
  \textbf{Assess scope}: Identify which agents and beliefs may be
  affected
\item[$\square$]
  \textbf{Contain spread}: Isolate affected agents from propagating
  beliefs
\item[$\square$]
  \textbf{Notify stakeholders}: Alert security team and relevant
  operators
\end{itemize}
\end{block}

\begin{block}{Investigation (First Hour)}
\protect\phantomsection\label{investigation-first-hour}
\begin{itemize}
\tightlist
\item[$\square$]
  \textbf{Trace provenance}: Follow belief origins to identify injection
  point
\item[$\square$]
  \textbf{Identify attack vector}: Determine how adversarial content
  entered
\item[$\square$]
  \textbf{Assess impact}: Evaluate what decisions were influenced
\item[$\square$]
  \textbf{Check for persistence}: Verify attack doesn't survive agent
  restart
\end{itemize}
\end{block}

\begin{block}{Recovery (Following Hours)}
\protect\phantomsection\label{recovery-following-hours}
\begin{itemize}
\tightlist
\item[$\square$]
  \textbf{Restore clean state}: Reset affected beliefs to verified
  baseline
\item[$\square$]
  \textbf{Strengthen defenses}: Update detection patterns based on
  attack
\item[$\square$]
  \textbf{Verify integrity}: Confirm cognitive state passes all
  tripwires
\item[$\square$]
  \textbf{Document incident}: Record details for future reference
\end{itemize}
\end{block}

\begin{block}{Post-Incident (Following Days)}
\protect\phantomsection\label{post-incident-following-days}
\begin{itemize}
\tightlist
\item[$\square$]
  \textbf{Root cause analysis}: Complete investigation of attack chain
\item[$\square$]
  \textbf{Defense improvements}: Implement countermeasures for attack
  type
\item[$\square$]
  \textbf{Team debrief}: Share lessons learned with all operators
\item[$\square$]
  \textbf{Update procedures}: Revise checklists based on incident
  learnings
\end{itemize}

Figure \ref{fig:timeline} provides an overview of these phases within
the broader cognitive security lifecycle.

\begin{figure}
\centering
\includegraphics[width=0.95\linewidth,height=\textheight,keepaspectratio,alt={Cognitive Security Lifecycle Phases. The deployment lifecycle consists of three major phases: Pre-Deployment (threat modeling, CIF selection, trust boundary definition, invariant specification), Operational (continuous monitoring, trust recalibration, anomaly detection, performance optimization), and Incident Response (quarantine compromised agents, belief state rollback, forensic analysis, recovery and hardening). The relative durations shown reflect typical enterprise deployments where operational monitoring dominates the lifecycle.}]{../figures/timeline.pdf}
\caption{Cognitive Security Lifecycle Phases. The deployment lifecycle
consists of three major phases: Pre-Deployment (threat modeling, CIF
selection, trust boundary definition, invariant specification),
Operational (continuous monitoring, trust recalibration, anomaly
detection, performance optimization), and Incident Response (quarantine
compromised agents, belief state rollback, forensic analysis, recovery
and hardening). The relative durations shown reflect typical enterprise
deployments where operational monitoring dominates the
lifecycle.}\label{fig:timeline}
\end{figure}
\end{block}
\end{frame}

\begin{frame}{Configuration Quick Reference}
\protect\phantomsection\label{configuration-quick-reference}
\begin{block}{Trust Calculus Parameters}
\protect\phantomsection\label{trust-calculus-parameters}
{\def\LTcaptype{none} % do not increment counter
\begin{longtable}[]{@{}lll@{}}
\toprule\noalign{}
Parameter & Recommended Value & When to Adjust \\
\midrule\noalign{}
\endhead
Base weight (α) & 0.3 & Increase for stable architectures \\
Reputation weight (β) & 0.4 & Decrease for new deployments \\
Context weight (γ) & 0.3 & Increase for specialized tasks \\
Decay factor (δ) & 0.9 & Decrease for security-critical systems \\
\bottomrule\noalign{}
\end{longtable}
}
\end{block}

\begin{block}{Firewall Thresholds}
\protect\phantomsection\label{firewall-thresholds}
{\def\LTcaptype{none} % do not increment counter
\begin{longtable}[]{@{}
  >{\raggedright\arraybackslash}p{(\linewidth - 4\tabcolsep) * \real{0.2444}}
  >{\raggedright\arraybackslash}p{(\linewidth - 4\tabcolsep) * \real{0.4000}}
  >{\raggedright\arraybackslash}p{(\linewidth - 4\tabcolsep) * \real{0.3556}}@{}}
\toprule\noalign{}
\begin{minipage}[b]{\linewidth}\raggedright
Threshold
\end{minipage} & \begin{minipage}[b]{\linewidth}\raggedright
Recommended Value
\end{minipage} & \begin{minipage}[b]{\linewidth}\raggedright
Risk Trade-off
\end{minipage} \\
\midrule\noalign{}
\endhead
Accept threshold & 0.3 & Lower = more strict, more false positives \\
Reject threshold & 0.7 & Higher = more permissive, more risk \\
Quarantine range & 0.3-0.7 & Narrower = faster decisions, less nuance \\
\bottomrule\noalign{}
\end{longtable}
}
\end{block}

\begin{block}{Tripwire Configuration}
\protect\phantomsection\label{tripwire-configuration}
{\def\LTcaptype{none} % do not increment counter
\begin{longtable}[]{@{}lll@{}}
\toprule\noalign{}
Category & Recommended Count & Placement Strategy \\
\midrule\noalign{}
\endhead
Identity canaries & 3+ per agent & Core identity beliefs \\
Boundary canaries & 5+ per agent & Permission boundaries \\
Principal canaries & 2+ per agent & Trust relationships \\
Temporal canaries & 1 per agent & Session continuity \\
\bottomrule\noalign{}
\end{longtable}
}
\end{block}
\end{frame}

\end{document}
