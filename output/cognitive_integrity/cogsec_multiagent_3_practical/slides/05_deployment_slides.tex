% Options for packages loaded elsewhere
\PassOptionsToPackage{unicode}{hyperref}
\PassOptionsToPackage{hyphens}{url}
\documentclass[
  ignorenonframetext,
]{beamer}
\newif\ifbibliography
\usepackage{pgfpages}
\setbeamertemplate{caption}[numbered]
\setbeamertemplate{caption label separator}{: }
\setbeamercolor{caption name}{fg=normal text.fg}
\beamertemplatenavigationsymbolsempty
% remove section numbering
\setbeamertemplate{part page}{
  \centering
  \begin{beamercolorbox}[sep=16pt,center]{part title}
    \usebeamerfont{part title}\insertpart\par
  \end{beamercolorbox}
}
\setbeamertemplate{section page}{
  \centering
  \begin{beamercolorbox}[sep=12pt,center]{section title}
    \usebeamerfont{section title}\insertsection\par
  \end{beamercolorbox}
}
\setbeamertemplate{subsection page}{
  \centering
  \begin{beamercolorbox}[sep=8pt,center]{subsection title}
    \usebeamerfont{subsection title}\insertsubsection\par
  \end{beamercolorbox}
}
% Prevent slide breaks in the middle of a paragraph
\widowpenalties 1 10000
\raggedbottom
\AtBeginPart{
  \frame{\partpage}
}
\AtBeginSection{
  \ifbibliography
  \else
    \frame{\sectionpage}
  \fi
}
\AtBeginSubsection{
  \frame{\subsectionpage}
}
\usepackage{iftex}
\ifPDFTeX
  \usepackage[T1]{fontenc}
  \usepackage[utf8]{inputenc}
  \usepackage{textcomp} % provide euro and other symbols
\else % if luatex or xetex
  \usepackage{unicode-math} % this also loads fontspec
  \defaultfontfeatures{Scale=MatchLowercase}
  \defaultfontfeatures[\rmfamily]{Ligatures=TeX,Scale=1}
\fi
\usepackage{lmodern}
\ifPDFTeX\else
  % xetex/luatex font selection
\fi
% Use upquote if available, for straight quotes in verbatim environments
\IfFileExists{upquote.sty}{\usepackage{upquote}}{}
\IfFileExists{microtype.sty}{% use microtype if available
  \usepackage[]{microtype}
  \UseMicrotypeSet[protrusion]{basicmath} % disable protrusion for tt fonts
}{}
\makeatletter
\@ifundefined{KOMAClassName}{% if non-KOMA class
  \IfFileExists{parskip.sty}{%
    \usepackage{parskip}
  }{% else
    \setlength{\parindent}{0pt}
    \setlength{\parskip}{6pt plus 2pt minus 1pt}}
}{% if KOMA class
  \KOMAoptions{parskip=half}}
\makeatother
\usepackage{longtable,booktabs,array}
\newcounter{none} % for unnumbered tables
\usepackage{calc} % for calculating minipage widths
\usepackage{caption}
% Make caption package work with longtable
\makeatletter
\def\fnum@table{\tablename~\thetable}
\makeatother
\usepackage{graphicx}
\makeatletter
\newsavebox\pandoc@box
\newcommand*\pandocbounded[1]{% scales image to fit in text height/width
  \sbox\pandoc@box{#1}%
  \Gscale@div\@tempa{\textheight}{\dimexpr\ht\pandoc@box+\dp\pandoc@box\relax}%
  \Gscale@div\@tempb{\linewidth}{\wd\pandoc@box}%
  \ifdim\@tempb\p@<\@tempa\p@\let\@tempa\@tempb\fi% select the smaller of both
  \ifdim\@tempa\p@<\p@\scalebox{\@tempa}{\usebox\pandoc@box}%
  \else\usebox{\pandoc@box}%
  \fi%
}
% Set default figure placement to htbp
\def\fps@figure{htbp}
\makeatother
\setlength{\emergencystretch}{3em} % prevent overfull lines
\providecommand{\tightlist}{%
  \setlength{\itemsep}{0pt}\setlength{\parskip}{0pt}}
\usepackage{bookmark}
\IfFileExists{xurl.sty}{\usepackage{xurl}}{} % add URL line breaks if available
\urlstyle{same}
\hypersetup{
  hidelinks,
  pdfcreator={LaTeX via pandoc}}

\author{\texorpdfstring{}{}}
\date{}

\begin{document}

\begin{frame}
\newpage
\end{frame}

\section{Deployment Considerations}\label{sec:deployment}

\begin{frame}{Risk Profile Assessment}
\protect\phantomsection\label{risk-profile-assessment}
Before configuring cognitive security mechanisms, assess your deployment
risk profile:

\begin{block}{Low Risk Profile}
\protect\phantomsection\label{low-risk-profile}
\textbf{Characteristics}: - Internal-only deployment - Non-sensitive
data handling - Human-in-the-loop for all significant actions - Limited
inter-agent communication

\textbf{Recommended Configuration}: - Firewall: Standard thresholds
(accept: 0.3, reject: 0.7) - Trust decay: Moderate (δ = 0.95) -
Consensus: Simple majority for coordination - Monitoring: Daily review
sufficient
\end{block}

\begin{block}{Medium Risk Profile}
\protect\phantomsection\label{medium-risk-profile}
\textbf{Characteristics}: - Customer-facing but limited autonomy - Some
sensitive data handling - Periodic human oversight - Moderate delegation
chains

\textbf{Recommended Configuration}: - Firewall: Tighter thresholds
(accept: 0.25, reject: 0.65) - Trust decay: Stricter (δ = 0.9) -
Consensus: 2/3 majority with identity verification - Monitoring:
Real-time alerts for critical events
\end{block}

\begin{block}{High Risk Profile}
\protect\phantomsection\label{high-risk-profile}
\textbf{Characteristics}: - Autonomous actions with significant impact -
Sensitive/regulated data handling - Extended periods without human
oversight - Complex delegation hierarchies

\textbf{Recommended Configuration}: - Firewall: Strict thresholds
(accept: 0.2, reject: 0.6) - Trust decay: Aggressive (δ = 0.85) -
Consensus: Byzantine-tolerant (n ≥ 3f + 1) - Monitoring: Continuous with
immediate alerting
\end{block}

\begin{block}{Understanding Trust Decay}
\protect\phantomsection\label{understanding-trust-decay}
The trust decay parameter δ governs how quickly trust attenuates through
delegation chains. Figure \ref{fig:trust-decay} compares the three
recommended configurations across delegation depths.

\begin{figure}
\centering
\includegraphics[width=0.95\linewidth,height=\textheight,keepaspectratio,alt={Trust Decay Comparison: Effect of δ Parameter. These curves demonstrate how effective trust diminishes with delegation depth under different decay configurations. Conservative settings (δ=0.9) allow deeper delegation chains while aggressive settings (δ=0.7) rapidly attenuate trust, limiting attack propagation. The formula T\_\{effective\} = T\_\{initial\} \textbackslash times \textbackslash delta\^{}d governs this relationship, where d is delegation depth. Red dashed lines mark the practical trust threshold (10\%) below which delegated authority becomes negligible.}]{../figures/trust_decay.pdf}
\caption{Trust Decay Comparison: Effect of δ Parameter. These curves
demonstrate how effective trust diminishes with delegation depth under
different decay configurations. Conservative settings (δ=0.9) allow
deeper delegation chains while aggressive settings (δ=0.7) rapidly
attenuate trust, limiting attack propagation. The formula
\(T_{effective} = T_{initial} \times \delta^d\) governs this
relationship, where \(d\) is delegation depth. Red dashed lines mark the
practical trust threshold (10\%) below which delegated authority becomes
negligible.}\label{fig:trust-decay}
\end{figure}

With δ = 0.85 (high-risk recommendation), trust drops to 52\% after 4
hops and below 10\% after 14 hops, providing strong containment of trust
laundering attacks while permitting reasonable delegation depths.
\end{block}
\end{frame}

\begin{frame}{Architecture-Specific Guidance}
\protect\phantomsection\label{architecture-specific-guidance}
\begin{block}{Hierarchical Architectures (Claude Code, AutoGPT)}
\protect\phantomsection\label{hierarchical-architectures-claude-code-autogpt}
\textbf{Characteristics}: Central orchestrator delegates to specialized
workers

\textbf{Key Risks}: - Orchestrator compromise cascades to all workers -
Worker escalation can influence orchestrator - Single point of failure

\textbf{Mitigations}: - Strong orchestrator protection (strictest
thresholds) - Bounded upward influence from workers - Orchestrator
tripwires for identity canaries - Consider multi-orchestrator redundancy
for critical deployments
\end{block}

\begin{block}{Peer-to-Peer Architectures (Camel)}
\protect\phantomsection\label{peer-to-peer-architectures-camel}
\textbf{Characteristics}: Equal-authority agents with lateral
communication

\textbf{Key Risks}: - Lateral movement attacks (compromise spreads
horizontally) - Sybil attacks (injected fake agents) - Consensus
manipulation

\textbf{Mitigations}: - Byzantine consensus for all multi-agent
decisions - Strong agent authentication - Network topology monitoring -
Reputation systems with slow trust building
\end{block}

\begin{block}{Role-Based Architectures (CrewAI)}
\protect\phantomsection\label{role-based-architectures-crewai}
\textbf{Characteristics}: Agents have defined roles with boundaries

\textbf{Key Risks}: - Role impersonation - Boundary violation - Role
privilege escalation

\textbf{Mitigations}: - Role-based permission boundaries -
Challenge-response for role verification - Cross-role action validation
- Audit trails for role-based actions
\end{block}

\begin{block}{State Machine Architectures (LangGraph)}
\protect\phantomsection\label{state-machine-architectures-langgraph}
\textbf{Characteristics}: Explicit state transitions govern behavior

\textbf{Key Risks}: - State corruption - Invalid transition injection -
State history manipulation

\textbf{Mitigations}: - State integrity verification (hashing) -
Transition validation against allowed graph - History immutability
enforcement - Rollback capability to known-good states
\end{block}
\end{frame}

\begin{frame}{Scaling Considerations}
\protect\phantomsection\label{scaling-considerations}
\begin{block}{Agent Count Scaling}
\protect\phantomsection\label{agent-count-scaling}
{\def\LTcaptype{none} % do not increment counter
\begin{longtable}[]{@{}
  >{\raggedright\arraybackslash}p{(\linewidth - 4\tabcolsep) * \real{0.2286}}
  >{\raggedright\arraybackslash}p{(\linewidth - 4\tabcolsep) * \real{0.2857}}
  >{\raggedright\arraybackslash}p{(\linewidth - 4\tabcolsep) * \real{0.4857}}@{}}
\toprule\noalign{}
\begin{minipage}[b]{\linewidth}\raggedright
Agents
\end{minipage} & \begin{minipage}[b]{\linewidth}\raggedright
Concerns
\end{minipage} & \begin{minipage}[b]{\linewidth}\raggedright
Recommendations
\end{minipage} \\
\midrule\noalign{}
\endhead
2-10 & Individual agent security dominates & Standard CIF deployment \\
10-100 & Coordination attacks become viable & Byzantine consensus
required \\
100-1000 & Emergent behavior security & Collective monitoring, quorum
scaling \\
1000+ & Colonial cognitive security & Stigmergic defense patterns (see
Part 1 Appendix) \\
\bottomrule\noalign{}
\end{longtable}
}
\end{block}

\begin{block}{Latency Budget}
\protect\phantomsection\label{latency-budget}
CIF introduces overhead. Plan accordingly:

{\def\LTcaptype{none} % do not increment counter
\begin{longtable}[]{@{}
  >{\raggedright\arraybackslash}p{(\linewidth - 4\tabcolsep) * \real{0.2391}}
  >{\raggedright\arraybackslash}p{(\linewidth - 4\tabcolsep) * \real{0.3696}}
  >{\raggedright\arraybackslash}p{(\linewidth - 4\tabcolsep) * \real{0.3913}}@{}}
\toprule\noalign{}
\begin{minipage}[b]{\linewidth}\raggedright
Component
\end{minipage} & \begin{minipage}[b]{\linewidth}\raggedright
Typical Latency
\end{minipage} & \begin{minipage}[b]{\linewidth}\raggedright
When to Optimize
\end{minipage} \\
\midrule\noalign{}
\endhead
Firewall & 5-10ms & Batch classification for bulk inputs \\
Trust computation & 1-2ms & Cache trust scores for stable
relationships \\
Sandbox lookup & \textless1ms & Rarely a bottleneck \\
Tripwire check & 1-5ms & Sample rather than check all beliefs \\
Consensus & 50-200ms & Reserve for critical decisions only \\
\bottomrule\noalign{}
\end{longtable}
}
\end{block}
\end{frame}

\begin{frame}{Integration Patterns}
\protect\phantomsection\label{integration-patterns}
\begin{block}{Pattern 1: Wrapper Integration}
\protect\phantomsection\label{pattern-1-wrapper-integration}
Wrap existing agent framework with CIF layer: - Input: Firewall
classification before agent processing - Inter-agent: Trust verification
on message passing - Output: Invariant checking before action execution
\end{block}

\begin{block}{Pattern 2: Native Integration}
\protect\phantomsection\label{pattern-2-native-integration}
Embed CIF into agent architecture: - Agent maintains own belief sandbox
- Trust calculus integrated with delegation logic - Tripwires planted
during agent initialization
\end{block}

\begin{block}{Pattern 3: Sidecar Integration}
\protect\phantomsection\label{pattern-3-sidecar-integration}
Run CIF as separate monitoring service: - Asynchronous belief drift
detection - Centralized trust matrix management - Aggregated alert
dashboard
\end{block}
\end{frame}

\end{document}
