% Options for packages loaded elsewhere
\PassOptionsToPackage{unicode}{hyperref}
\PassOptionsToPackage{hyphens}{url}
\documentclass[
  ignorenonframetext,
]{beamer}
\newif\ifbibliography
\usepackage{pgfpages}
\setbeamertemplate{caption}[numbered]
\setbeamertemplate{caption label separator}{: }
\setbeamercolor{caption name}{fg=normal text.fg}
\beamertemplatenavigationsymbolsempty
% remove section numbering
\setbeamertemplate{part page}{
  \centering
  \begin{beamercolorbox}[sep=16pt,center]{part title}
    \usebeamerfont{part title}\insertpart\par
  \end{beamercolorbox}
}
\setbeamertemplate{section page}{
  \centering
  \begin{beamercolorbox}[sep=12pt,center]{section title}
    \usebeamerfont{section title}\insertsection\par
  \end{beamercolorbox}
}
\setbeamertemplate{subsection page}{
  \centering
  \begin{beamercolorbox}[sep=8pt,center]{subsection title}
    \usebeamerfont{subsection title}\insertsubsection\par
  \end{beamercolorbox}
}
% Prevent slide breaks in the middle of a paragraph
\widowpenalties 1 10000
\raggedbottom
\AtBeginPart{
  \frame{\partpage}
}
\AtBeginSection{
  \ifbibliography
  \else
    \frame{\sectionpage}
  \fi
}
\AtBeginSubsection{
  \frame{\subsectionpage}
}
\usepackage{iftex}
\ifPDFTeX
  \usepackage[T1]{fontenc}
  \usepackage[utf8]{inputenc}
  \usepackage{textcomp} % provide euro and other symbols
\else % if luatex or xetex
  \usepackage{unicode-math} % this also loads fontspec
  \defaultfontfeatures{Scale=MatchLowercase}
  \defaultfontfeatures[\rmfamily]{Ligatures=TeX,Scale=1}
\fi
\usepackage{lmodern}
\ifPDFTeX\else
  % xetex/luatex font selection
\fi
% Use upquote if available, for straight quotes in verbatim environments
\IfFileExists{upquote.sty}{\usepackage{upquote}}{}
\IfFileExists{microtype.sty}{% use microtype if available
  \usepackage[]{microtype}
  \UseMicrotypeSet[protrusion]{basicmath} % disable protrusion for tt fonts
}{}
\makeatletter
\@ifundefined{KOMAClassName}{% if non-KOMA class
  \IfFileExists{parskip.sty}{%
    \usepackage{parskip}
  }{% else
    \setlength{\parindent}{0pt}
    \setlength{\parskip}{6pt plus 2pt minus 1pt}}
}{% if KOMA class
  \KOMAoptions{parskip=half}}
\makeatother
\setlength{\emergencystretch}{3em} % prevent overfull lines
\providecommand{\tightlist}{%
  \setlength{\itemsep}{0pt}\setlength{\parskip}{0pt}}
\usepackage{bookmark}
\IfFileExists{xurl.sty}{\usepackage{xurl}}{} % add URL line breaks if available
\urlstyle{same}
\hypersetup{
  hidelinks,
  pdfcreator={LaTeX via pandoc}}

\author{\texorpdfstring{}{}}
\date{}

\begin{document}

\begin{frame}
\vspace*{2cm}

\begin{center}
\begin{minipage}{0.7\textwidth}
\centering
\Large\itshape
``In theory, there is no difference between\\[0.3em]
theory and practice. In practice, there is.''
\vspace{1em}

\normalsize\upshape
--- Yogi Berra (attributed)
\end{minipage}
\end{center}

\vspace{2cm}
\end{frame}

\begin{frame}{Abstract}
\protect\phantomsection\label{abstract}
This paper translates the \textbf{Cognitive Integrity Framework (CIF)}
into actionable guidance for practitioners. Building on formal
foundations (Part 1) and empirical validation (Part 2), we provide
practical recommendations for securing multiagent AI deployments.

\begin{block}{Contributions}
\protect\phantomsection\label{contributions}
\begin{itemize}
\tightlist
\item
  \textbf{Operator Posture Assessment}: Framework for evaluating
  organizational cognitive security readiness
\item
  \textbf{Deployment Checklists}: Step-by-step guidance for
  implementation and monitoring
\item
  \textbf{Agent Guidelines}: Machine-readable rules for AI system
  self-monitoring
\item
  \textbf{Risk Assessment}: Threat modeling methodology for cognitive
  attack surfaces
\item
  \textbf{Common Pitfalls}: Documented anti-patterns with specific
  mitigations
\end{itemize}
\end{block}

\begin{block}{Audience}
\protect\phantomsection\label{audience}
This guidance serves security practitioners, developers, operators, and
compliance teams evaluating multiagent AI deployments. Technical
prerequisites are minimal; readers seeking formal foundations should
consult Part 1.
\end{block}

\begin{block}{Approach}
\protect\phantomsection\label{approach}
We prioritize clarity over comprehensiveness. Each section provides
actionable recommendations with explicit pointers to Parts 1 and 2 for
theoretical grounding and empirical evidence. Notation is used sparingly
and always defined inline.
\end{block}

\begin{block}{Paper Series}
\protect\phantomsection\label{paper-series}
\textbf{DOI}: 10.5281/zenodo.18364130

This is Part 3 of the \emph{Cognitive Security for Multiagent Operators}
series: - \textbf{Part 1} (DOI: 10.5281/zenodo.18364119): Formal
foundations and theoretical analysis - \textbf{Part 2} (DOI:
10.5281/zenodo.18364128): Computational validation and implementation -
\textbf{Part 3} (this paper): Practical deployment guidance
\end{block}
\end{frame}

\end{document}
