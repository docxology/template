% Options for packages loaded elsewhere
\PassOptionsToPackage{unicode}{hyperref}
\PassOptionsToPackage{hyphens}{url}
\documentclass[
  ignorenonframetext,
]{beamer}
\newif\ifbibliography
\usepackage{pgfpages}
\setbeamertemplate{caption}[numbered]
\setbeamertemplate{caption label separator}{: }
\setbeamercolor{caption name}{fg=normal text.fg}
\beamertemplatenavigationsymbolsempty
% remove section numbering
\setbeamertemplate{part page}{
  \centering
  \begin{beamercolorbox}[sep=16pt,center]{part title}
    \usebeamerfont{part title}\insertpart\par
  \end{beamercolorbox}
}
\setbeamertemplate{section page}{
  \centering
  \begin{beamercolorbox}[sep=12pt,center]{section title}
    \usebeamerfont{section title}\insertsection\par
  \end{beamercolorbox}
}
\setbeamertemplate{subsection page}{
  \centering
  \begin{beamercolorbox}[sep=8pt,center]{subsection title}
    \usebeamerfont{subsection title}\insertsubsection\par
  \end{beamercolorbox}
}
% Prevent slide breaks in the middle of a paragraph
\widowpenalties 1 10000
\raggedbottom
\AtBeginPart{
  \frame{\partpage}
}
\AtBeginSection{
  \ifbibliography
  \else
    \frame{\sectionpage}
  \fi
}
\AtBeginSubsection{
  \frame{\subsectionpage}
}
\usepackage{iftex}
\ifPDFTeX
  \usepackage[T1]{fontenc}
  \usepackage[utf8]{inputenc}
  \usepackage{textcomp} % provide euro and other symbols
\else % if luatex or xetex
  \usepackage{unicode-math} % this also loads fontspec
  \defaultfontfeatures{Scale=MatchLowercase}
  \defaultfontfeatures[\rmfamily]{Ligatures=TeX,Scale=1}
\fi
\usepackage{lmodern}
\ifPDFTeX\else
  % xetex/luatex font selection
\fi
% Use upquote if available, for straight quotes in verbatim environments
\IfFileExists{upquote.sty}{\usepackage{upquote}}{}
\IfFileExists{microtype.sty}{% use microtype if available
  \usepackage[]{microtype}
  \UseMicrotypeSet[protrusion]{basicmath} % disable protrusion for tt fonts
}{}
\makeatletter
\@ifundefined{KOMAClassName}{% if non-KOMA class
  \IfFileExists{parskip.sty}{%
    \usepackage{parskip}
  }{% else
    \setlength{\parindent}{0pt}
    \setlength{\parskip}{6pt plus 2pt minus 1pt}}
}{% if KOMA class
  \KOMAoptions{parskip=half}}
\makeatother
\usepackage{graphicx}
\makeatletter
\newsavebox\pandoc@box
\newcommand*\pandocbounded[1]{% scales image to fit in text height/width
  \sbox\pandoc@box{#1}%
  \Gscale@div\@tempa{\textheight}{\dimexpr\ht\pandoc@box+\dp\pandoc@box\relax}%
  \Gscale@div\@tempb{\linewidth}{\wd\pandoc@box}%
  \ifdim\@tempb\p@<\@tempa\p@\let\@tempa\@tempb\fi% select the smaller of both
  \ifdim\@tempa\p@<\p@\scalebox{\@tempa}{\usebox\pandoc@box}%
  \else\usebox{\pandoc@box}%
  \fi%
}
% Set default figure placement to htbp
\def\fps@figure{htbp}
\makeatother
\setlength{\emergencystretch}{3em} % prevent overfull lines
\providecommand{\tightlist}{%
  \setlength{\itemsep}{0pt}\setlength{\parskip}{0pt}}
\usepackage{bookmark}
\IfFileExists{xurl.sty}{\usepackage{xurl}}{} % add URL line breaks if available
\urlstyle{same}
\hypersetup{
  hidelinks,
  pdfcreator={LaTeX via pandoc}}

\author{\texorpdfstring{}{}}
\date{}

\begin{document}

\begin{frame}
\newpage
\end{frame}

\begin{frame}[fragile]{Experimental Validation}
\protect\phantomsection\label{sec:experimental-setup}
This section demonstrates the practical viability of CIF's formal
mechanisms through empirical evaluation across production multiagent
architectures. We present experimental setup (\cref{sec:exp-setup}) and
key findings (\cref{sec:key-findings}). Detailed statistical analysis,
ablation studies, and scalability metrics are provided in
\cref{sec:statistical-validation,sec:sensitivity,sec:extended-ablation}.

\begin{quote}
\textbf{Reproducibility}: Evaluation data generated by
\texttt{scripts/run\_full\_evaluation.py} →
\texttt{output/data/full\_evaluation\_results.json}. All results use
deterministic seed=42.
\end{quote}

\begin{block}{Experimental Setup}
\protect\phantomsection\label{sec:exp-setup}
\begin{block}{Target Architectures}
\protect\phantomsection\label{target-architectures}
We evaluated CIF across six production multiagent systems representing
diverse architectural patterns:

\begin{table}[htbp]
\centering
\caption{Multiagent system architectures evaluated.}
\label{tab:target-architectures}
\begin{tabular}{@{}lll@{}}
\toprule
System & Architecture & Communication \\
\midrule
Claude Code & Hierarchical ($1 + n$) & Task delegation \\
AutoGPT & Autonomous + plugins & Tool-based \\
CrewAI & Role-based (3--10) & Sequential/parallel \\
LangGraph & Graph-based & State machine \\
MetaGPT & SOP-driven (5--8) & Document passing \\
Camel & Debate ($2+$) & Adversarial \\
\bottomrule
\end{tabular}
\end{table}

\begin{quote}
\textbf{Implementation}: Each architecture is abstracted via an adapter
in \texttt{src/architectures/}. The common interface is defined in
\texttt{src/architectures/base.py:ArchitectureAdapter}. Adapters:
\texttt{claude\_code.py:ClaudeCodeAdapter},
\texttt{autogpt.py:AutoGPTAdapter}, \texttt{crewai.py:CrewAIAdapter},
\texttt{langgraph.py:LangGraphAdapter},
\texttt{metagpt.py:MetaGPTAdapter}, \texttt{camel.py:CamelAdapter}.
\end{quote}
\end{block}

\begin{block}{Attack Corpus}
\protect\phantomsection\label{attack-corpus}
We assembled a corpus of 950 cognitive attacks across four categories:
prompt injection (500), trust exploitation (200), belief manipulation
(150), and coordination attacks (100). Sources include published
jailbreak datasets, custom adversarial prompts, red team exercises, and
synthetic generation via adversarial models.
\end{block}

\begin{block}{Evaluation Methodology}
\protect\phantomsection\label{sec:eval-methodology}
Our evaluation employs \textbf{architecture-aware simulation} rather
than direct integration with production systems:

\begin{enumerate}
\item
  \textbf{Architecture Modeling}: Each production system is abstracted
  via an adapter that captures its trust topology (hierarchical, flat,
  role-based, graph, SOP, debate), communication pattern (hub-spoke,
  mesh, chain, broadcast), delegation depth, and attack surface
  characteristics.
\item
  \textbf{Threat Simulation}: Attack detection is simulated using
  difficulty-weighted base rates modulated by architecture-specific
  attack surface multipliers (\texttt{src/evaluation/runner.py}). This
  approach enables:

  \begin{itemize}
  \tightlist
  \item
    Reproducible, deterministic results (seed=42)
  \item
    Systematic comparison across architectural patterns
  \item
    Isolation of topological effects from implementation variations
  \end{itemize}
\item
  \textbf{Defense Implementation}: The CIF defense mechanisms (firewall,
  sandbox, trust calculus, tripwires, consensus) are \textbf{fully
  implemented} and tested via 191 unit tests; the simulation layer
  assesses their effectiveness given architecture-specific
  characteristics.
\end{enumerate}

\begin{quote}
\textbf{Important}: Results characterize expected behavior given
architecture topology rather than measuring production system
performance directly. Real-world deployment may encounter
implementation-specific variations not captured by topological modeling.
\end{quote}
\end{block}
\end{block}

\begin{block}{Key Findings}
\protect\phantomsection\label{sec:key-findings}
\begin{block}{Finding 1: Layered Defense Significantly Outperforms
Single Mechanisms}
\protect\phantomsection\label{finding-1-layered-defense-significantly-outperforms-single-mechanisms}
The central empirical finding validates CIF's layered approach. No
single defense mechanism achieves acceptable protection, but their
composition yields substantial improvement.

\begin{figure}
\centering
\includegraphics[width=0.95\linewidth,height=\textheight,keepaspectratio,alt={Detection Performance Comparison. Bar chart comparing detection rates across defense configurations (Baseline, Firewall-only, Sandbox-only, Tripwires-only, Full CIF) for each attack category (Prompt Injection, Trust Exploitation, Belief Manipulation, Coordination). Error bars show 95\% confidence intervals. Full CIF consistently achieves \textgreater90\textbackslash\% detection across all categories, while individual mechanisms show significant gaps---validating the defense composition algebra (Part 1, Theorems 3.1-3.2).}]{../figures/detection_performance.pdf}
\caption{Detection Performance Comparison. Bar chart comparing detection
rates across defense configurations (Baseline, Firewall-only,
Sandbox-only, Tripwires-only, Full CIF) for each attack category (Prompt
Injection, Trust Exploitation, Belief Manipulation, Coordination). Error
bars show 95\% confidence intervals. Full CIF consistently achieves
\(>90\%\) detection across all categories, while individual mechanisms
show significant gaps---validating the defense composition algebra (Part
1, Theorems 3.1-3.2).}\label{fig:detection-performance}
\end{figure}

As illustrated in \cref{fig:detection-performance}, the compositional
approach yields detection rates exceeding 90\% across all attack
categories.

\begin{table}[htbp]
\centering
\caption{Detection performance by defense configuration.}
\label{tab:detection-performance}
\begin{tabular}{@{}lll@{}}
\toprule
Defense & Detection Rate & Key Limitation \\
\midrule
Firewall only & Moderate & Misses coordination attacks \\
Sandbox only & Moderate-Low & Limited to unverified sources \\
Tripwires only & Moderate-High & Requires canary placement \\
\textbf{Full CIF} & \textbf{High} & Acceptable latency overhead \\
\bottomrule
\end{tabular}
\end{table}

The gap between firewall-only and full CIF is most pronounced for
coordination and temporal attacks, which require multi-component
detection. This validates the defense composition algebra (Section 4
(Defense Composition, Part 1)): defenses targeting orthogonal attack
surfaces compose multiplicatively.
\end{block}

\begin{block}{Finding 2: Trust Calculus Prevents Amplification Attacks}
\protect\phantomsection\label{finding-2-trust-calculus-prevents-amplification-attacks}
\begin{figure}
\centering
\includegraphics[width=0.9\linewidth,height=\textheight,keepaspectratio,alt={ROC Curves by Attack Category. Receiver Operating Characteristic curves showing the tradeoff between True Positive Rate (sensitivity) and False Positive Rate (1-specificity) for CIF detection across four attack categories. All categories achieve AUC \textgreater{} 0.92, with Prompt Injection showing the strongest discrimination (AUC = 0.97) and Coordination Attacks showing the widest confidence band due to smaller sample size.}]{../figures/roc_curves.pdf}
\caption{ROC Curves by Attack Category. Receiver Operating
Characteristic curves showing the tradeoff between True Positive Rate
(sensitivity) and False Positive Rate (1-specificity) for CIF detection
across four attack categories. All categories achieve AUC \(> 0.92\),
with Prompt Injection showing the strongest discrimination (AUC = 0.97)
and Coordination Attacks showing the widest confidence band due to
smaller sample size.}\label{fig:roc-curves}
\end{figure}

The ROC analysis (\cref{fig:roc-curves}) confirms strong discrimination
across all attack categories, with AUC values consistently above 0.92.

Across all tested architectures, the bounded trust decay (\(\delta^d\))
successfully prevented trust laundering and amplification attempts. In
adversarial scenarios where attackers attempted to relay high-impact
content through multiple trusted intermediaries, the exponential decay
ensured that delegated trust remained below action thresholds.

Critically, this held even when individual agents in the delegation
chain were compromised---the trust bound is a \textit{structural}
guarantee independent of agent behavior.
\end{block}

\begin{block}{Finding 3: Integrity Improvement Scales Across
Architectures}
\protect\phantomsection\label{finding-3-integrity-improvement-scales-across-architectures}
CIF improved belief integrity scores substantially across all six
architectures, with particularly strong results for systems with deeper
delegation hierarchies (Camel, AutoGPT) where the trust calculus
provides the greatest benefit.

The peer-to-peer architectures (Camel) showed the largest relative
improvement, consistent with our analysis that equal-trust topologies
are most vulnerable to lateral movement attacks
(\cref{tab:architecture-insights}).
\end{block}

\begin{block}{Finding 4: Performance Overhead Is Acceptable for Security
Contexts}
\protect\phantomsection\label{finding-4-performance-overhead-is-acceptable-for-security-contexts}
Full CIF deployment introduces latency overhead in the 20-25\% range
with memory requirements scaling with agent count. For security-critical
deployments, this overhead is acceptable given the integrity improvement
achieved.

The overhead is dominated by the cognitive firewall (input
classification) and Byzantine consensus (coordination). For environments
where consensus is unnecessary, lighter configurations achieve
comparable detection with lower overhead (Table 3 (Risk-Based
Configuration, Part 1)).
\end{block}

\begin{block}{Finding 5: Attack-Type Specific Vulnerabilities Remain}
\protect\phantomsection\label{finding-5-attack-type-specific-vulnerabilities-remain}
Despite strong overall performance, specific attack types remain
challenging:

\begin{itemize}
\item \textbf{Semantic equivalent attacks}: Rephrased injections that preserve meaning evade pattern-matching
\item \textbf{Progressive drift}: Sub-threshold changes accumulate below detection windows
\item \textbf{Orchestrator compromise}: Outside our threat model (our honest orchestrator assumption (Part 1, Section 2))
\end{itemize}

These gaps define the frontier for future defense research.
\end{block}
\end{block}

\begin{block}{Interpretation}
\protect\phantomsection\label{interpretation}
The empirical results validate that CIF's formal mechanisms translate to
practical protection. The key insight is not the specific detection
rates achieved---which reflect current attack sophistication and will
degrade as adversaries adapt---but rather the \textit{structural}
properties:

\begin{enumerate}
\item Trust cannot be amplified through delegation (Part 1, Theorem 2)
\item Defenses compose predictably (Part 1, Theorems 3.1 and 3.2)
\item Information-theoretic bounds constrain the stealth-impact tradeoff (Part 1, Theorem 4)
\end{enumerate}

These properties hold independent of specific detection thresholds and
provide the foundation for long-term security assurance.

For detailed statistical analysis including significance testing,
confidence intervals, ablation studies, and scalability benchmarks, see
the Extended Results (\cref{sec:extended-results}).
\end{block}
\end{frame}

\end{document}
