% Options for packages loaded elsewhere
\PassOptionsToPackage{unicode}{hyperref}
\PassOptionsToPackage{hyphens}{url}
\documentclass[
  ignorenonframetext,
]{beamer}
\newif\ifbibliography
\usepackage{pgfpages}
\setbeamertemplate{caption}[numbered]
\setbeamertemplate{caption label separator}{: }
\setbeamercolor{caption name}{fg=normal text.fg}
\beamertemplatenavigationsymbolsempty
% remove section numbering
\setbeamertemplate{part page}{
  \centering
  \begin{beamercolorbox}[sep=16pt,center]{part title}
    \usebeamerfont{part title}\insertpart\par
  \end{beamercolorbox}
}
\setbeamertemplate{section page}{
  \centering
  \begin{beamercolorbox}[sep=12pt,center]{section title}
    \usebeamerfont{section title}\insertsection\par
  \end{beamercolorbox}
}
\setbeamertemplate{subsection page}{
  \centering
  \begin{beamercolorbox}[sep=8pt,center]{subsection title}
    \usebeamerfont{subsection title}\insertsubsection\par
  \end{beamercolorbox}
}
% Prevent slide breaks in the middle of a paragraph
\widowpenalties 1 10000
\raggedbottom
\AtBeginPart{
  \frame{\partpage}
}
\AtBeginSection{
  \ifbibliography
  \else
    \frame{\sectionpage}
  \fi
}
\AtBeginSubsection{
  \frame{\subsectionpage}
}
\usepackage{iftex}
\ifPDFTeX
  \usepackage[T1]{fontenc}
  \usepackage[utf8]{inputenc}
  \usepackage{textcomp} % provide euro and other symbols
\else % if luatex or xetex
  \usepackage{unicode-math} % this also loads fontspec
  \defaultfontfeatures{Scale=MatchLowercase}
  \defaultfontfeatures[\rmfamily]{Ligatures=TeX,Scale=1}
\fi
\usepackage{lmodern}
\ifPDFTeX\else
  % xetex/luatex font selection
\fi
% Use upquote if available, for straight quotes in verbatim environments
\IfFileExists{upquote.sty}{\usepackage{upquote}}{}
\IfFileExists{microtype.sty}{% use microtype if available
  \usepackage[]{microtype}
  \UseMicrotypeSet[protrusion]{basicmath} % disable protrusion for tt fonts
}{}
\makeatletter
\@ifundefined{KOMAClassName}{% if non-KOMA class
  \IfFileExists{parskip.sty}{%
    \usepackage{parskip}
  }{% else
    \setlength{\parindent}{0pt}
    \setlength{\parskip}{6pt plus 2pt minus 1pt}}
}{% if KOMA class
  \KOMAoptions{parskip=half}}
\makeatother
\setlength{\emergencystretch}{3em} % prevent overfull lines
\providecommand{\tightlist}{%
  \setlength{\itemsep}{0pt}\setlength{\parskip}{0pt}}
\usepackage{bookmark}
\IfFileExists{xurl.sty}{\usepackage{xurl}}{} % add URL line breaks if available
\urlstyle{same}
\hypersetup{
  hidelinks,
  pdfcreator={LaTeX via pandoc}}

\author{\texorpdfstring{}{}}
\date{}

\begin{document}

\begin{frame}
\vspace*{2cm}

\begin{center}
\begin{minipage}{0.7\textwidth}
\centering
\Large\itshape
``The difference between theory and practice\\[0.3em]
is larger in practice than in theory.''
\vspace{1em}

\normalsize\upshape
--- Jan van de Snepscheut, Computer Scientist
\end{minipage}
\end{center}

\vspace{2cm}
\end{frame}

\begin{frame}{Abstract}
\protect\phantomsection\label{abstract}
As multiagent AI systems transition from research prototypes to
production infrastructure, their security properties remain largely
unvalidated. While formal security frameworks promise principled
protection, a persistent gap exists between theoretical guarantees and
empirical evidence: \emph{do these defenses actually work against real
attacks?} This paper bridges that gap through comprehensive
computational validation of the \textbf{Cognitive Integrity Framework
(CIF)} introduced in Part 1.

We implement the complete CIF defense suite---cognitive firewalls,
belief sandboxes, trust calculus with bounded delegation, identity
tripwires, and Byzantine-tolerant consensus---and evaluate performance
using \textbf{architecture-aware simulation} across topological models
of six production multiagent systems, with a novel corpus of 950
cognitive attacks.

\begin{block}{Contributions}
\protect\phantomsection\label{contributions}
\begin{itemize}
\tightlist
\item
  \textbf{Attack Corpus}: 950 cognitive attacks across four categories
  (prompt injection, trust exploitation, belief manipulation,
  coordination attacks), with full reproducibility via deterministic
  generation
\item
  \textbf{Architecture Modeling}: Topological abstractions of Claude
  Code, AutoGPT, CrewAI, LangGraph, MetaGPT, and Camel---capturing trust
  matrices, communication patterns, and attack surface characteristics
  of hierarchical, autonomous, role-based, graph-based, SOP-driven, and
  debate architectures
\item
  \textbf{Simulated Detection Performance}: 94\% overall detection rate
  with layered defenses (range: 87--98\% by attack type); 20--25\%
  latency overhead acceptable for security-critical contexts
\item
  \textbf{Statistical Rigor}: Significance testing (\(p < 0.0001\) for
  primary hypotheses), large effect sizes (Cohen's \(d > 1.0\)),
  confidence intervals, ablation studies, and scalability benchmarks to
  100 agents
\end{itemize}
\end{block}

\begin{block}{Key Findings}
\protect\phantomsection\label{key-findings}
\begin{enumerate}
\tightlist
\item
  \textbf{Composition is Essential}: No individual defense achieves
  acceptable protection alone; layered composition yields multiplicative
  detection improvement consistent with Part 1's theoretical predictions
  (Theorem 3.2)
\item
  \textbf{Trust Decay Prevents Amplification}: The bounded delegation
  mechanism (\(\delta^d\) decay) successfully prevented trust laundering
  across all tested architectures---a structural guarantee independent
  of attacker sophistication
\item
  \textbf{Architecture Determines Vulnerability Profile}: Peer-to-peer
  systems show the largest relative improvement from CIF deployment
  (+422\% integrity preservation under multi-vector attack), confirming
  the lateral movement analysis from Part 1
\end{enumerate}
\end{block}

\begin{block}{Implications for Practitioners}
\protect\phantomsection\label{implications-for-practitioners}
These results establish that CIF provides \emph{practical, deployable
protection} for production multiagent systems. Organizations deploying
AI agents should: (1) configure all CIF components for security-critical
workloads, (2) calibrate parameters to their specific architecture using
the sensitivity analysis in Section 5, and (3) expect 20--25\% latency
overhead as the cost of validated security.

All notation follows definitions from Part 1 (Supplementary Section
S03). Complete source code is available at:
\textbf{\url{https://github.com/docxology/cognitive_integrity}}
\end{block}

\begin{block}{Paper Series}
\protect\phantomsection\label{paper-series}
\textbf{DOI}: 10.5281/zenodo.18364128

This is Part 2 of the \emph{Cognitive Security for Multiagent Operators}
series:

\begin{itemize}
\tightlist
\item
  \textbf{Part 1} (DOI: 10.5281/zenodo.18364119): Formal foundations and
  theoretical analysis
\item
  \textbf{Part 2} (this paper): Computational validation and
  implementation
\item
  \textbf{Part 3} (DOI: 10.5281/zenodo.18364130): Practical deployment
  guidance
\end{itemize}
\end{block}
\end{frame}

\end{document}
