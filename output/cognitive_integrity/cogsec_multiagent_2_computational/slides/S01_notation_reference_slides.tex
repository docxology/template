% Options for packages loaded elsewhere
\PassOptionsToPackage{unicode}{hyperref}
\PassOptionsToPackage{hyphens}{url}
\documentclass[
  ignorenonframetext,
]{beamer}
\newif\ifbibliography
\usepackage{pgfpages}
\setbeamertemplate{caption}[numbered]
\setbeamertemplate{caption label separator}{: }
\setbeamercolor{caption name}{fg=normal text.fg}
\beamertemplatenavigationsymbolsempty
% remove section numbering
\setbeamertemplate{part page}{
  \centering
  \begin{beamercolorbox}[sep=16pt,center]{part title}
    \usebeamerfont{part title}\insertpart\par
  \end{beamercolorbox}
}
\setbeamertemplate{section page}{
  \centering
  \begin{beamercolorbox}[sep=12pt,center]{section title}
    \usebeamerfont{section title}\insertsection\par
  \end{beamercolorbox}
}
\setbeamertemplate{subsection page}{
  \centering
  \begin{beamercolorbox}[sep=8pt,center]{subsection title}
    \usebeamerfont{subsection title}\insertsubsection\par
  \end{beamercolorbox}
}
% Prevent slide breaks in the middle of a paragraph
\widowpenalties 1 10000
\raggedbottom
\AtBeginPart{
  \frame{\partpage}
}
\AtBeginSection{
  \ifbibliography
  \else
    \frame{\sectionpage}
  \fi
}
\AtBeginSubsection{
  \frame{\subsectionpage}
}
\usepackage{iftex}
\ifPDFTeX
  \usepackage[T1]{fontenc}
  \usepackage[utf8]{inputenc}
  \usepackage{textcomp} % provide euro and other symbols
\else % if luatex or xetex
  \usepackage{unicode-math} % this also loads fontspec
  \defaultfontfeatures{Scale=MatchLowercase}
  \defaultfontfeatures[\rmfamily]{Ligatures=TeX,Scale=1}
\fi
\usepackage{lmodern}
\ifPDFTeX\else
  % xetex/luatex font selection
\fi
% Use upquote if available, for straight quotes in verbatim environments
\IfFileExists{upquote.sty}{\usepackage{upquote}}{}
\IfFileExists{microtype.sty}{% use microtype if available
  \usepackage[]{microtype}
  \UseMicrotypeSet[protrusion]{basicmath} % disable protrusion for tt fonts
}{}
\makeatletter
\@ifundefined{KOMAClassName}{% if non-KOMA class
  \IfFileExists{parskip.sty}{%
    \usepackage{parskip}
  }{% else
    \setlength{\parindent}{0pt}
    \setlength{\parskip}{6pt plus 2pt minus 1pt}}
}{% if KOMA class
  \KOMAoptions{parskip=half}}
\makeatother
\usepackage{longtable,booktabs,array}
\newcounter{none} % for unnumbered tables
\usepackage{calc} % for calculating minipage widths
\usepackage{caption}
% Make caption package work with longtable
\makeatletter
\def\fnum@table{\tablename~\thetable}
\makeatother
\setlength{\emergencystretch}{3em} % prevent overfull lines
\providecommand{\tightlist}{%
  \setlength{\itemsep}{0pt}\setlength{\parskip}{0pt}}
\usepackage{bookmark}
\IfFileExists{xurl.sty}{\usepackage{xurl}}{} % add URL line breaks if available
\urlstyle{same}
\hypersetup{
  hidelinks,
  pdfcreator={LaTeX via pandoc}}

\author{\texorpdfstring{}{}}
\date{}

\begin{document}

\begin{frame}
\newpage
\end{frame}

\begin{frame}{Notation Reference}
\protect\phantomsection\label{sec:notation-reference}
This paper uses notation from the Cognitive Integrity Framework (CIF)
formal specification defined in Part 1 of this series.

\begin{block}{Quick Reference}
\protect\phantomsection\label{quick-reference}
\begin{block}{Core Entities}
\protect\phantomsection\label{core-entities}
{\def\LTcaptype{none} % do not increment counter
\begin{longtable}[]{@{}lll@{}}
\toprule\noalign{}
Symbol & Meaning & Part 1 Reference \\
\midrule\noalign{}
\endhead
\(\mathcal{A}\) & Agent set & Definition 1 \\
\(a_i\) & Individual agent & Definition 1 \\
\(\mathcal{B}_i\) & Belief function for agent \(i\) & Definition 2 \\
\(\mathcal{G}_i\) & Goal set for agent \(i\) & Definition 2 \\
\(\mathcal{I}_i\) & Intention set & Table 1 \\
\(\sigma_i^t\) & Cognitive state at time \(t\) & Definition 2 \\
\bottomrule\noalign{}
\end{longtable}
}
\end{block}

\begin{block}{Trust Calculus}
\protect\phantomsection\label{trust-calculus}
{\def\LTcaptype{none} % do not increment counter
\begin{longtable}[]{@{}lll@{}}
\toprule\noalign{}
Symbol & Meaning & Part 1 Reference \\
\midrule\noalign{}
\endhead
\(\mathcal{T}_{i \to j}\) & Trust from agent \(i\) to \(j\) & Definition
3 \\
\(\delta\) & Trust decay factor & Definition 4 \\
\(\otimes\) & Trust delegation operator & Definition 4 \\
\(\oplus\) & Trust aggregation operator & Definition 4 \\
\(\alpha, \beta, \gamma\) & Trust weight parameters & Equation 5 \\
\bottomrule\noalign{}
\end{longtable}
}
\end{block}

\begin{block}{Defense Mechanisms}
\protect\phantomsection\label{defense-mechanisms}
{\def\LTcaptype{none} % do not increment counter
\begin{longtable}[]{@{}lll@{}}
\toprule\noalign{}
Symbol & Meaning & Part 1 Reference \\
\midrule\noalign{}
\endhead
\(D_i\) & Defense mechanism \(i\) & Definition 5 \\
\(r_i\) & Detection rate of defense \(i\) & Definition 6 \\
\(\tau_{\text{accept}}\) & Firewall accept threshold & Table 2 \\
\(\tau_{\text{reject}}\) & Firewall reject threshold & Table 2 \\
\(\epsilon_{\text{drift}}\) & Drift detection threshold & Equation 8 \\
\bottomrule\noalign{}
\end{longtable}
}
\end{block}

\begin{block}{Consensus and Coordination}
\protect\phantomsection\label{consensus-and-coordination}
{\def\LTcaptype{none} % do not increment counter
\begin{longtable}[]{@{}lll@{}}
\toprule\noalign{}
Symbol & Meaning & Part 1 Reference \\
\midrule\noalign{}
\endhead
\(q\) & Quorum threshold & Definition 7 \\
\(f\) & Maximum Byzantine agents & Theorem 1 \\
\(n\) & Total agent count & Throughout \\
\bottomrule\noalign{}
\end{longtable}
}
\end{block}
\end{block}

\begin{block}{Commonly Confused Symbols}
\protect\phantomsection\label{commonly-confused-symbols}
{\def\LTcaptype{none} % do not increment counter
\begin{longtable}[]{@{}
  >{\raggedright\arraybackslash}p{(\linewidth - 2\tabcolsep) * \real{0.5000}}
  >{\raggedright\arraybackslash}p{(\linewidth - 2\tabcolsep) * \real{0.5000}}@{}}
\toprule\noalign{}
\begin{minipage}[b]{\linewidth}\raggedright
Symbol Pair
\end{minipage} & \begin{minipage}[b]{\linewidth}\raggedright
Distinction
\end{minipage} \\
\midrule\noalign{}
\endhead
\(\mathcal{T}\) vs \(t\) & \(\mathcal{T}\) = trust function; \(t\) =
time index \\
\(\delta\) vs \(d\) & \(\delta\) = decay factor (parameter); \(d\) =
delegation depth (variable) \\
\(\mathcal{B}\) vs \(B\) & \(\mathcal{B}\) = belief function; \(B\) =
specific belief set \\
\(r\) vs \(R\) & \(r\) = detection rate; \(R\) = detection response \\
\(\tau\) vs \(T\) & \(\tau\) = threshold; \(T\) = trust value \\
\bottomrule\noalign{}
\end{longtable}
}
\end{block}

\begin{block}{Typographical Conventions}
\protect\phantomsection\label{typographical-conventions}
{\def\LTcaptype{none} % do not increment counter
\begin{longtable}[]{@{}lll@{}}
\toprule\noalign{}
Convention & Meaning & Example \\
\midrule\noalign{}
\endhead
Calligraphic & Sets and functions & \(\mathcal{A}\), \(\mathcal{T}\) \\
Roman subscript & Descriptive labels & \(\tau_{\text{accept}}\) \\
Italic subscript & Variable indices & \(a_i\), \(\sigma_j^t\) \\
Bold & Vectors and matrices & \(\mathbf{v}\), \(\mathbf{M}\) \\
Sans-serif & Algorithm names & \textsf{CIF}, \textsf{FIREWALL} \\
\bottomrule\noalign{}
\end{longtable}
}
\end{block}

\begin{block}{Canonical Reference}
\protect\phantomsection\label{canonical-reference}
For complete notation definitions, see:

\begin{itemize}
\tightlist
\item
  Part 1: \textbf{Supplementary Section S03: Notation Reference}
\end{itemize}
\end{block}
\end{frame}

\end{document}
