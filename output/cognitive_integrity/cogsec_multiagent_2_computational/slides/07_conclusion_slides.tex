% Options for packages loaded elsewhere
\PassOptionsToPackage{unicode}{hyperref}
\PassOptionsToPackage{hyphens}{url}
\documentclass[
  ignorenonframetext,
]{beamer}
\newif\ifbibliography
\usepackage{pgfpages}
\setbeamertemplate{caption}[numbered]
\setbeamertemplate{caption label separator}{: }
\setbeamercolor{caption name}{fg=normal text.fg}
\beamertemplatenavigationsymbolsempty
% remove section numbering
\setbeamertemplate{part page}{
  \centering
  \begin{beamercolorbox}[sep=16pt,center]{part title}
    \usebeamerfont{part title}\insertpart\par
  \end{beamercolorbox}
}
\setbeamertemplate{section page}{
  \centering
  \begin{beamercolorbox}[sep=12pt,center]{section title}
    \usebeamerfont{section title}\insertsection\par
  \end{beamercolorbox}
}
\setbeamertemplate{subsection page}{
  \centering
  \begin{beamercolorbox}[sep=8pt,center]{subsection title}
    \usebeamerfont{subsection title}\insertsubsection\par
  \end{beamercolorbox}
}
% Prevent slide breaks in the middle of a paragraph
\widowpenalties 1 10000
\raggedbottom
\AtBeginPart{
  \frame{\partpage}
}
\AtBeginSection{
  \ifbibliography
  \else
    \frame{\sectionpage}
  \fi
}
\AtBeginSubsection{
  \frame{\subsectionpage}
}
\usepackage{iftex}
\ifPDFTeX
  \usepackage[T1]{fontenc}
  \usepackage[utf8]{inputenc}
  \usepackage{textcomp} % provide euro and other symbols
\else % if luatex or xetex
  \usepackage{unicode-math} % this also loads fontspec
  \defaultfontfeatures{Scale=MatchLowercase}
  \defaultfontfeatures[\rmfamily]{Ligatures=TeX,Scale=1}
\fi
\usepackage{lmodern}
\ifPDFTeX\else
  % xetex/luatex font selection
\fi
% Use upquote if available, for straight quotes in verbatim environments
\IfFileExists{upquote.sty}{\usepackage{upquote}}{}
\IfFileExists{microtype.sty}{% use microtype if available
  \usepackage[]{microtype}
  \UseMicrotypeSet[protrusion]{basicmath} % disable protrusion for tt fonts
}{}
\makeatletter
\@ifundefined{KOMAClassName}{% if non-KOMA class
  \IfFileExists{parskip.sty}{%
    \usepackage{parskip}
  }{% else
    \setlength{\parindent}{0pt}
    \setlength{\parskip}{6pt plus 2pt minus 1pt}}
}{% if KOMA class
  \KOMAoptions{parskip=half}}
\makeatother
\setlength{\emergencystretch}{3em} % prevent overfull lines
\providecommand{\tightlist}{%
  \setlength{\itemsep}{0pt}\setlength{\parskip}{0pt}}
\usepackage{bookmark}
\IfFileExists{xurl.sty}{\usepackage{xurl}}{} % add URL line breaks if available
\urlstyle{same}
\hypersetup{
  hidelinks,
  pdfcreator={LaTeX via pandoc}}

\author{\texorpdfstring{}{}}
\date{}

\begin{document}

\begin{frame}
\newpage
\end{frame}

\section{Conclusion: Contributions and Practical
Implications}\label{sec:conclusion}

\begin{frame}{Summary of Contributions}
\protect\phantomsection\label{summary-of-contributions}
This paper provided comprehensive computational validation of the
Cognitive Integrity Framework (CIF) introduced in Part 1 of this series
through architecture-aware simulation. Our primary contributions:

\textbf{Implementation}: We implemented the complete CIF defense
suite---cognitive firewalls, belief sandboxes, trust calculus with
bounded delegation, tripwire detection, behavioral invariants, and
Byzantine-tolerant consensus---demonstrating that the formal mechanisms
translate into deployable code with acceptable performance
characteristics.

\textbf{Attack Corpus}: We assembled 950 cognitive attacks across four
categories (prompt injection, trust exploitation, belief manipulation,
coordination attacks), enabling reproducible security evaluation of
multiagent systems. The corpus is available to verified researchers
under controlled access.

\textbf{Architecture Modeling}: We modeled six production multiagent
architectures (Claude Code, AutoGPT, CrewAI, LangGraph, MetaGPT, Camel)
via topological adapters that capture trust matrices, communication
patterns, and attack surface characteristics, demonstrating that formal
guarantees hold across diverse architectural patterns.

\textbf{Statistical Rigor}: We provided significance testing
(\(p < 0.0001\) for primary hypotheses), effect sizes (Cohen's
\(d > 1.0\) for all major comparisons), confidence intervals, and
ablation studies establishing the robustness of our findings beyond
sampling variation.
\end{frame}

\begin{frame}{Key Findings}
\protect\phantomsection\label{key-findings}
\begin{enumerate}
\item
  \textbf{Layered defense is essential}: No single mechanism achieves
  acceptable protection; composition yields multiplicative improvement
  consistent with theoretical predictions (Part 1, Theorem 3.2).
\item
  \textbf{Trust calculus prevents amplification}: The \(\delta^d\) decay
  bound successfully prevented trust laundering across all tested
  architectures---a structural guarantee independent of attacker
  sophistication.
\item
  \textbf{Architecture matters}: Peer-to-peer architectures show
  greatest improvement from CIF deployment (+422\% integrity
  preservation under multi-vector attack), consistent with their
  vulnerability to lateral movement attacks.
\item
  \textbf{Performance overhead is acceptable}: 20--25\% latency overhead
  for full CIF deployment is appropriate for security-critical contexts;
  minimal configurations achieve 90\% detection with only 12\% overhead.
\end{enumerate}
\end{frame}

\begin{frame}{Open Problems}
\protect\phantomsection\label{open-problems}
Despite comprehensive validation, several challenges remain for future
research:

\begin{block}{Adaptive Adversaries}
\protect\phantomsection\label{adaptive-adversaries}
Our evaluation used a fixed attack corpus. Real-world adversaries adapt
to deployed defenses. \emph{Research question}: How quickly do detection
rates degrade as adversaries observe and adapt to CIF's filtering
patterns?
\end{block}

\begin{block}{Semantic Understanding}
\protect\phantomsection\label{semantic-understanding}
Pattern-based detection fails against semantically-equivalent attacks.
\emph{Research question}: Can language model-based semantic analysis
improve detection without prohibitive latency?
\end{block}

\begin{block}{Emergent Behavior Security}
\protect\phantomsection\label{emergent-behavior-security}
As multiagent systems scale, collective behaviors emerge. \emph{Research
question}: How can we distinguish beneficial emergence from
attack-induced coordination?
\end{block}

\begin{block}{Federated Trust}
\protect\phantomsection\label{federated-trust}
Current CIF assumes a single trust domain. \emph{Research question}: How
can trust relationships be established and verified across
organizational boundaries?
\end{block}

\begin{block}{Formal Verification at Scale}
\protect\phantomsection\label{formal-verification-at-scale}
While Part 1 provides theoretical foundations, practical formal
verification remains limited. \emph{Research question}: Can model
checking scale to production-sized multiagent configurations?
\end{block}
\end{frame}

\begin{frame}{Implications for Practitioners}
\protect\phantomsection\label{implications-for-practitioners}
The simulation results indicate that CIF provides practical protection:

\begin{itemize}
\tightlist
\item
  \textbf{Deploy layered defenses}: Configure all CIF components for
  security-critical deployments; the 23\% latency overhead is justified
  by 94\% detection rates
\item
  \textbf{Calibrate to architecture}: Apply architecture-specific
  recommendations from \cref{tab:architecture-insights}---peer-to-peer
  systems need stronger consensus; hierarchical systems need stronger
  orchestrator protection
\item
  \textbf{Monitor continuously}: Detection rates degrade over time as
  adversaries adapt; ongoing vigilance and pattern updates are required
\item
  \textbf{Start with minimal configurations}: For resource-constrained
  deployments, Firewall + Tripwires + Drift Detection achieves 90\%
  detection with only 12\% overhead
\end{itemize}

For detailed deployment guidance, including human-actionable checklists
and agent-readable guidelines, see Part 3 of this series.
\end{frame}

\begin{frame}{Call to Action}
\protect\phantomsection\label{call-to-action}
We invite the research community to extend the attack corpus, validate
on new architectures, contribute defense mechanisms, and report
vulnerabilities through our responsible disclosure process.
\end{frame}

\begin{frame}{Paper Series}
\protect\phantomsection\label{paper-series}
This is Part 2 of the \emph{Cognitive Security for Multiagent Operators}
series:

\begin{itemize}
\tightlist
\item
  \textbf{Part 1: Formal Foundations} - Trust calculus, defense
  composition algebra, information-theoretic bounds
\item
  \textbf{Part 2 (This Paper): Computational Validation} -
  Implementation, attack corpus, simulation-based results
\item
  \textbf{Part 3: Practical Guidance} - Deployment checklists, operator
  posture, risk assessment
\end{itemize}

Together, these papers provide a complete framework for understanding,
implementing, and operating cognitive security in multiagent AI systems.
\end{frame}

\begin{frame}{Acknowledgments}
\protect\phantomsection\label{acknowledgments}
{[}Acknowledgments to be added prior to publication{]}
\end{frame}

\end{document}
