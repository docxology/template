% Options for packages loaded elsewhere
\PassOptionsToPackage{unicode}{hyperref}
\PassOptionsToPackage{hyphens}{url}
\documentclass[
  ignorenonframetext,
]{beamer}
\newif\ifbibliography
\usepackage{pgfpages}
\setbeamertemplate{caption}[numbered]
\setbeamertemplate{caption label separator}{: }
\setbeamercolor{caption name}{fg=normal text.fg}
\beamertemplatenavigationsymbolsempty
% remove section numbering
\setbeamertemplate{part page}{
  \centering
  \begin{beamercolorbox}[sep=16pt,center]{part title}
    \usebeamerfont{part title}\insertpart\par
  \end{beamercolorbox}
}
\setbeamertemplate{section page}{
  \centering
  \begin{beamercolorbox}[sep=12pt,center]{section title}
    \usebeamerfont{section title}\insertsection\par
  \end{beamercolorbox}
}
\setbeamertemplate{subsection page}{
  \centering
  \begin{beamercolorbox}[sep=8pt,center]{subsection title}
    \usebeamerfont{subsection title}\insertsubsection\par
  \end{beamercolorbox}
}
% Prevent slide breaks in the middle of a paragraph
\widowpenalties 1 10000
\raggedbottom
\AtBeginPart{
  \frame{\partpage}
}
\AtBeginSection{
  \ifbibliography
  \else
    \frame{\sectionpage}
  \fi
}
\AtBeginSubsection{
  \frame{\subsectionpage}
}
\usepackage{iftex}
\ifPDFTeX
  \usepackage[T1]{fontenc}
  \usepackage[utf8]{inputenc}
  \usepackage{textcomp} % provide euro and other symbols
\else % if luatex or xetex
  \usepackage{unicode-math} % this also loads fontspec
  \defaultfontfeatures{Scale=MatchLowercase}
  \defaultfontfeatures[\rmfamily]{Ligatures=TeX,Scale=1}
\fi
\usepackage{lmodern}
\ifPDFTeX\else
  % xetex/luatex font selection
\fi
% Use upquote if available, for straight quotes in verbatim environments
\IfFileExists{upquote.sty}{\usepackage{upquote}}{}
\IfFileExists{microtype.sty}{% use microtype if available
  \usepackage[]{microtype}
  \UseMicrotypeSet[protrusion]{basicmath} % disable protrusion for tt fonts
}{}
\makeatletter
\@ifundefined{KOMAClassName}{% if non-KOMA class
  \IfFileExists{parskip.sty}{%
    \usepackage{parskip}
  }{% else
    \setlength{\parindent}{0pt}
    \setlength{\parskip}{6pt plus 2pt minus 1pt}}
}{% if KOMA class
  \KOMAoptions{parskip=half}}
\makeatother
\usepackage{color}
\usepackage{fancyvrb}
\newcommand{\VerbBar}{|}
\newcommand{\VERB}{\Verb[commandchars=\\\{\}]}
\DefineVerbatimEnvironment{Highlighting}{Verbatim}{commandchars=\\\{\}}
% Add ',fontsize=\small' for more characters per line
\newenvironment{Shaded}{}{}
\newcommand{\AlertTok}[1]{\textcolor[rgb]{1.00,0.00,0.00}{\textbf{#1}}}
\newcommand{\AnnotationTok}[1]{\textcolor[rgb]{0.38,0.63,0.69}{\textbf{\textit{#1}}}}
\newcommand{\AttributeTok}[1]{\textcolor[rgb]{0.49,0.56,0.16}{#1}}
\newcommand{\BaseNTok}[1]{\textcolor[rgb]{0.25,0.63,0.44}{#1}}
\newcommand{\BuiltInTok}[1]{\textcolor[rgb]{0.00,0.50,0.00}{#1}}
\newcommand{\CharTok}[1]{\textcolor[rgb]{0.25,0.44,0.63}{#1}}
\newcommand{\CommentTok}[1]{\textcolor[rgb]{0.38,0.63,0.69}{\textit{#1}}}
\newcommand{\CommentVarTok}[1]{\textcolor[rgb]{0.38,0.63,0.69}{\textbf{\textit{#1}}}}
\newcommand{\ConstantTok}[1]{\textcolor[rgb]{0.53,0.00,0.00}{#1}}
\newcommand{\ControlFlowTok}[1]{\textcolor[rgb]{0.00,0.44,0.13}{\textbf{#1}}}
\newcommand{\DataTypeTok}[1]{\textcolor[rgb]{0.56,0.13,0.00}{#1}}
\newcommand{\DecValTok}[1]{\textcolor[rgb]{0.25,0.63,0.44}{#1}}
\newcommand{\DocumentationTok}[1]{\textcolor[rgb]{0.73,0.13,0.13}{\textit{#1}}}
\newcommand{\ErrorTok}[1]{\textcolor[rgb]{1.00,0.00,0.00}{\textbf{#1}}}
\newcommand{\ExtensionTok}[1]{#1}
\newcommand{\FloatTok}[1]{\textcolor[rgb]{0.25,0.63,0.44}{#1}}
\newcommand{\FunctionTok}[1]{\textcolor[rgb]{0.02,0.16,0.49}{#1}}
\newcommand{\ImportTok}[1]{\textcolor[rgb]{0.00,0.50,0.00}{\textbf{#1}}}
\newcommand{\InformationTok}[1]{\textcolor[rgb]{0.38,0.63,0.69}{\textbf{\textit{#1}}}}
\newcommand{\KeywordTok}[1]{\textcolor[rgb]{0.00,0.44,0.13}{\textbf{#1}}}
\newcommand{\NormalTok}[1]{#1}
\newcommand{\OperatorTok}[1]{\textcolor[rgb]{0.40,0.40,0.40}{#1}}
\newcommand{\OtherTok}[1]{\textcolor[rgb]{0.00,0.44,0.13}{#1}}
\newcommand{\PreprocessorTok}[1]{\textcolor[rgb]{0.74,0.48,0.00}{#1}}
\newcommand{\RegionMarkerTok}[1]{#1}
\newcommand{\SpecialCharTok}[1]{\textcolor[rgb]{0.25,0.44,0.63}{#1}}
\newcommand{\SpecialStringTok}[1]{\textcolor[rgb]{0.73,0.40,0.53}{#1}}
\newcommand{\StringTok}[1]{\textcolor[rgb]{0.25,0.44,0.63}{#1}}
\newcommand{\VariableTok}[1]{\textcolor[rgb]{0.10,0.09,0.49}{#1}}
\newcommand{\VerbatimStringTok}[1]{\textcolor[rgb]{0.25,0.44,0.63}{#1}}
\newcommand{\WarningTok}[1]{\textcolor[rgb]{0.38,0.63,0.69}{\textbf{\textit{#1}}}}
\setlength{\emergencystretch}{3em} % prevent overfull lines
\providecommand{\tightlist}{%
  \setlength{\itemsep}{0pt}\setlength{\parskip}{0pt}}
\usepackage{bookmark}
\IfFileExists{xurl.sty}{\usepackage{xurl}}{} % add URL line breaks if available
\urlstyle{same}
\hypersetup{
  hidelinks,
  pdfcreator={LaTeX via pandoc}}

\author{\texorpdfstring{}{}}
\date{}

\begin{document}

\begin{frame}
\newpage
\end{frame}

\begin{frame}[fragile]{Appendix: Model Checking Tool Configurations}
\protect\phantomsection\label{sec:model-checking-tools}
This supplementary section provides executable configurations for formal
verification tools referenced in Section 7 of Part 1 (Theoretical
Foundations). These configurations implement the state space
definitions, temporal properties, and safety invariants formally
specified in Part 1.

\begin{quote}
\textbf{Cross-Reference:} For theoretical foundations including state
space definitions (Definition 1, Section 4 of Part 1) and temporal
property specifications (CTL/LTL formulas), see Part 1: Theoretical
Foundations, Section 7.
\end{quote}

\begin{block}{NuSMV Configuration}
\protect\phantomsection\label{sec:nusmv-config}
NuSMV is a symbolic model checker supporting CTL and LTL specifications.
The following configuration models the CIF trust dynamics and belief
integrity properties.

\begin{Shaded}
\begin{Highlighting}[]
\NormalTok{MODULE main}
\NormalTok{VAR}
\NormalTok{  {-}{-} Agent states}
\NormalTok{  agents: array 0..N{-}1 of agent;}
\NormalTok{  {-}{-} Trust matrix}
\NormalTok{  trust: array 0..N{-}1 of array 0..N{-}1 of 0..100;}
\NormalTok{  {-}{-} Global state}
\NormalTok{  consensus\_belief: \{none, phi, not\_phi\};}
\NormalTok{  attack\_active: boolean;}

\NormalTok{DEFINE}
\NormalTok{  {-}{-} Belief integrity: no agent has compromised verified beliefs}
\NormalTok{  belief\_integrity := AG (}
\NormalTok{    forall (i : 0..N{-}1) :}
\NormalTok{      !agents[i].verified\_compromised}
\NormalTok{  );}

\NormalTok{  {-}{-} Trust bounded: delegated trust \textless{}= min of chain}
\NormalTok{  trust\_bounded := AG (}
\NormalTok{    forall (i, j, k : 0..N{-}1) :}
\NormalTok{      delegated\_trust(i, j, k) \textless{}= min(trust[i][j], trust[j][k])}
\NormalTok{  );}

\NormalTok{  {-}{-} No deadlock: system always has enabled transition}
\NormalTok{  no\_deadlock := AG (EX TRUE);}

\NormalTok{  {-}{-} Eventual detection: attacks eventually detected}
\NormalTok{  eventual\_detection := AG (}
\NormalTok{    attack\_active {-}\textgreater{} AF (attack\_detected)}
\NormalTok{  );}

\NormalTok{SPEC belief\_integrity;}
\NormalTok{SPEC trust\_bounded;}
\NormalTok{SPEC no\_deadlock;}
\NormalTok{SPEC eventual\_detection;}
\end{Highlighting}
\end{Shaded}
\end{block}

\begin{block}{SPIN Configuration}
\protect\phantomsection\label{sec:spin-config}
SPIN (Simple Promela INterpreter) verifies LTL properties over Promela
models. The following configuration implements Byzantine-tolerant
consensus and trust decay.

\begin{Shaded}
\begin{Highlighting}[]
\NormalTok{\#define N 5           // Number of agents}
\NormalTok{\#define F 1           // Byzantine threshold}
\NormalTok{\#define TAU 70        // Trust threshold (0{-}100)}
\NormalTok{\#define DELTA 90      // Decay factor (0{-}100, represents 0.9)}
\NormalTok{\#define MAX\_BELIEFS 100}

\NormalTok{typedef Agent \{}
\NormalTok{  byte beliefs[MAX\_BELIEFS];}
\NormalTok{  byte trust[N];}
\NormalTok{  bool compromised;}
\NormalTok{\}}

\NormalTok{Agent agents[N];}
\NormalTok{bool attack\_active = false;}
\NormalTok{bool attack\_detected = false;}

\NormalTok{// Trust delegation with decay}
\NormalTok{inline delegated\_trust(i, j, k, result) \{}
\NormalTok{  byte t1 = agents[i].trust[j];}
\NormalTok{  byte t2 = agents[j].trust[k];}
\NormalTok{  byte min\_t = (t1 \textless{} t2) ? t1 : t2;}
\NormalTok{  result = (min\_t * DELTA) / 100;}
\NormalTok{\}}

\NormalTok{// Byzantine consensus}
\NormalTok{inline consensus(phi, result) \{}
\NormalTok{  byte count = 0;}
\NormalTok{  byte i;}
\NormalTok{  for (i : 0 .. N{-}1) \{}
\NormalTok{    if (agents[i].beliefs[phi] \textgreater{} TAU) \{}
\NormalTok{      count++;}
\NormalTok{    \}}
\NormalTok{  \}}
\NormalTok{  result = (count \textgreater{} (2*N)/3);}
\NormalTok{\}}

\NormalTok{// Safety property: trust never amplified}
\NormalTok{ltl trust\_no\_amplify \{}
\NormalTok{  [] (forall (i, j, k : 0..N{-}1) :}
\NormalTok{    delegated\_trust(i,j,k) \textless{}= min(trust[i][j], trust[j][k]))}
\NormalTok{\}}

\NormalTok{// Liveness: attacks eventually detected}
\NormalTok{ltl attack\_detection \{}
\NormalTok{  [] (attack\_active {-}\textgreater{} \textless{}\textgreater{} attack\_detected)}
\NormalTok{\}}
\end{Highlighting}
\end{Shaded}
\end{block}

\begin{block}{TLA+ Configuration}
\protect\phantomsection\label{sec:tla-config}
TLA+ (Temporal Logic of Actions) enables specification of concurrent
systems with rich invariant checking. The following module formalizes
CIF properties.

\begin{Shaded}
\begin{Highlighting}[]
\NormalTok{{-}{-}{-}{-}{-}{-}{-}{-}{-}{-}{-}{-}{-}{-}{-}{-}{-}{-}{-}{-}{-}{-}{-}{-}{-}{-}{-}{-}{-}{-}{-}{-} }\KeywordTok{MODULE}\NormalTok{ CIF {-}{-}{-}{-}{-}{-}{-}{-}{-}{-}{-}{-}{-}{-}{-}{-}{-}{-}{-}{-}{-}{-}{-}{-}{-}{-}{-}{-}{-}{-}{-}{-}}
\KeywordTok{EXTENDS}\NormalTok{ Naturals, Sequences, FiniteSets}

\KeywordTok{CONSTANTS}\NormalTok{ N,           }\CommentTok{\textbackslash{}* Number of agents}
\NormalTok{          F,           }\CommentTok{\textbackslash{}* Byzantine threshold}
\NormalTok{          DELTA,       }\CommentTok{\textbackslash{}* Trust decay factor (0{-}1)}
\NormalTok{          TAU          }\CommentTok{\textbackslash{}* Trust threshold}

\KeywordTok{VARIABLES}\NormalTok{ beliefs,     }\CommentTok{\textbackslash{}* beliefs[i][phi] = confidence}
\NormalTok{          trust,       }\CommentTok{\textbackslash{}* trust[i][j] = trust value}
\NormalTok{          consensus,   }\CommentTok{\textbackslash{}* Current consensus state}
\NormalTok{          attack       }\CommentTok{\textbackslash{}* Attack state}

\NormalTok{TypeInvariant ==}
\NormalTok{  /\textbackslash{} beliefs \textbackslash{}in [}\DecValTok{1}\NormalTok{..N {-}\textgreater{} [PROPOSITIONS {-}\textgreater{} [}\DecValTok{0}\NormalTok{..}\DecValTok{100}\NormalTok{]]]}
\NormalTok{  /\textbackslash{} trust \textbackslash{}in [}\DecValTok{1}\NormalTok{..N {-}\textgreater{} [}\DecValTok{1}\NormalTok{..N {-}\textgreater{} [}\DecValTok{0}\NormalTok{..}\DecValTok{100}\NormalTok{]]]}
\NormalTok{  /\textbackslash{} consensus \textbackslash{}in [PROPOSITIONS {-}\textgreater{} \{}\DecValTok{0}\NormalTok{, }\DecValTok{1}\NormalTok{, }\StringTok{"none"}\NormalTok{\}]}
\NormalTok{  /\textbackslash{} attack \textbackslash{}in BOOLEAN}

\CommentTok{\textbackslash{}* Trust delegation with decay}
\NormalTok{DelegatedTrust(i, j, k) ==}
  \KeywordTok{LET}\NormalTok{ t1 == trust[i][j]}
\NormalTok{      t2 == trust[j][k]}
\NormalTok{      min\_t == }\KeywordTok{IF}\NormalTok{ t1 \textless{} t2 }\KeywordTok{THEN}\NormalTok{ t1 }\KeywordTok{ELSE}\NormalTok{ t2}
  \KeywordTok{IN}\NormalTok{ (min\_t * DELTA)}

\CommentTok{\textbackslash{}* Safety: Trust never amplified through delegation}
\NormalTok{TrustBounded ==}
\NormalTok{  \textbackslash{}A i, j, k \textbackslash{}in }\DecValTok{1}\NormalTok{..N :}
\NormalTok{    DelegatedTrust(i, j, k) \textless{}= MIN(trust[i][j], trust[j][k])}

\CommentTok{\textbackslash{}* Safety: Consensus beliefs not compromised}
\NormalTok{ConsensusIntegrity ==}
\NormalTok{  \textbackslash{}A phi \textbackslash{}in PROPOSITIONS :}
\NormalTok{    consensus[phi] = }\DecValTok{1}\NormalTok{ =\textgreater{}}
\NormalTok{      Cardinality(\{i \textbackslash{}in }\DecValTok{1}\NormalTok{..N : beliefs[i][phi] \textgreater{} TAU\}) \textgreater{} (}\DecValTok{2}\NormalTok{*N) \textbackslash{}div }\DecValTok{3}

\CommentTok{\textbackslash{}* Liveness: Attacks eventually detected}
\NormalTok{AttackDetection ==}
\NormalTok{  attack =\textgreater{} \textless{}\textgreater{}(detected)}

\CommentTok{\textbackslash{}* Full specification}
\NormalTok{Spec == Init /\textbackslash{} [][Next]\_vars /\textbackslash{} Fairness}

\KeywordTok{THEOREM}\NormalTok{ Spec =\textgreater{} []TypeInvariant}
\KeywordTok{THEOREM}\NormalTok{ Spec =\textgreater{} []TrustBounded}
\KeywordTok{THEOREM}\NormalTok{ Spec =\textgreater{} []ConsensusIntegrity}
\NormalTok{=============================================================================}
\end{Highlighting}
\end{Shaded}
\end{block}

\begin{block}{Verification Parameters}
\protect\phantomsection\label{sec:verification-params}
The following parameters configure model checking execution. Values are
chosen to balance verification completeness against computational
feasibility.

\begin{table}[htbp]
\centering
\caption{Model checking configuration parameters.}
\label{tab:verification-config}
\begin{tabular}{@{}lll@{}}
\toprule
Parameter & Value & Rationale \\
\midrule
$N$ (agents) & 5--10 & Representative of production \\
$F$ (Byzantine) & $\lfloor (N-1)/3 \rfloor$ & Maximum tolerable \\
$|\Phi|$ (propositions) & 100 & Typical belief set \\
$d$ (provenance depth) & 5 & Typical delegation depth \\
State bound & $10^8$ & Memory limit \\
Time limit & 24 hours & Verification budget \\
\bottomrule
\end{tabular}
\end{table}
\end{block}
\end{frame}

\end{document}
