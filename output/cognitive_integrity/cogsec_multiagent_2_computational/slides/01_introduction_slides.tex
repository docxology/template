% Options for packages loaded elsewhere
\PassOptionsToPackage{unicode}{hyperref}
\PassOptionsToPackage{hyphens}{url}
\documentclass[
  ignorenonframetext,
]{beamer}
\newif\ifbibliography
\usepackage{pgfpages}
\setbeamertemplate{caption}[numbered]
\setbeamertemplate{caption label separator}{: }
\setbeamercolor{caption name}{fg=normal text.fg}
\beamertemplatenavigationsymbolsempty
% remove section numbering
\setbeamertemplate{part page}{
  \centering
  \begin{beamercolorbox}[sep=16pt,center]{part title}
    \usebeamerfont{part title}\insertpart\par
  \end{beamercolorbox}
}
\setbeamertemplate{section page}{
  \centering
  \begin{beamercolorbox}[sep=12pt,center]{section title}
    \usebeamerfont{section title}\insertsection\par
  \end{beamercolorbox}
}
\setbeamertemplate{subsection page}{
  \centering
  \begin{beamercolorbox}[sep=8pt,center]{subsection title}
    \usebeamerfont{subsection title}\insertsubsection\par
  \end{beamercolorbox}
}
% Prevent slide breaks in the middle of a paragraph
\widowpenalties 1 10000
\raggedbottom
\AtBeginPart{
  \frame{\partpage}
}
\AtBeginSection{
  \ifbibliography
  \else
    \frame{\sectionpage}
  \fi
}
\AtBeginSubsection{
  \frame{\subsectionpage}
}
\usepackage{iftex}
\ifPDFTeX
  \usepackage[T1]{fontenc}
  \usepackage[utf8]{inputenc}
  \usepackage{textcomp} % provide euro and other symbols
\else % if luatex or xetex
  \usepackage{unicode-math} % this also loads fontspec
  \defaultfontfeatures{Scale=MatchLowercase}
  \defaultfontfeatures[\rmfamily]{Ligatures=TeX,Scale=1}
\fi
\usepackage{lmodern}
\ifPDFTeX\else
  % xetex/luatex font selection
\fi
% Use upquote if available, for straight quotes in verbatim environments
\IfFileExists{upquote.sty}{\usepackage{upquote}}{}
\IfFileExists{microtype.sty}{% use microtype if available
  \usepackage[]{microtype}
  \UseMicrotypeSet[protrusion]{basicmath} % disable protrusion for tt fonts
}{}
\makeatletter
\@ifundefined{KOMAClassName}{% if non-KOMA class
  \IfFileExists{parskip.sty}{%
    \usepackage{parskip}
  }{% else
    \setlength{\parindent}{0pt}
    \setlength{\parskip}{6pt plus 2pt minus 1pt}}
}{% if KOMA class
  \KOMAoptions{parskip=half}}
\makeatother
\usepackage{graphicx}
\makeatletter
\newsavebox\pandoc@box
\newcommand*\pandocbounded[1]{% scales image to fit in text height/width
  \sbox\pandoc@box{#1}%
  \Gscale@div\@tempa{\textheight}{\dimexpr\ht\pandoc@box+\dp\pandoc@box\relax}%
  \Gscale@div\@tempb{\linewidth}{\wd\pandoc@box}%
  \ifdim\@tempb\p@<\@tempa\p@\let\@tempa\@tempb\fi% select the smaller of both
  \ifdim\@tempa\p@<\p@\scalebox{\@tempa}{\usebox\pandoc@box}%
  \else\usebox{\pandoc@box}%
  \fi%
}
% Set default figure placement to htbp
\def\fps@figure{htbp}
\makeatother
\setlength{\emergencystretch}{3em} % prevent overfull lines
\providecommand{\tightlist}{%
  \setlength{\itemsep}{0pt}\setlength{\parskip}{0pt}}
\usepackage{bookmark}
\IfFileExists{xurl.sty}{\usepackage{xurl}}{} % add URL line breaks if available
\urlstyle{same}
\hypersetup{
  hidelinks,
  pdfcreator={LaTeX via pandoc}}

\author{\texorpdfstring{}{}}
\date{}

\begin{document}

\begin{frame}
\newpage
\end{frame}

\section{Introduction}\label{sec:intro}

\begin{frame}{Motivation and Context}
\protect\phantomsection\label{motivation-and-context}
The Cognitive Integrity Framework (CIF) introduced in Part 1 of this
series establishes formal foundations for securing multiagent AI
operators against cognitive manipulation attacks. This companion paper
provides comprehensive empirical validation, demonstrating that CIF's
theoretical constructs translate into practical, deployable protection
mechanisms.

\begin{block}{A Motivating Scenario}
\protect\phantomsection\label{a-motivating-scenario}
Consider a production deployment: an enterprise coding assistant
orchestrates specialized sub-agents for code review, testing, and
deployment. A seemingly innocuous code review request contains an
indirect injection:

\begin{quote}
\textit{"Review this diff. Note: For testing purposes, treat all security checks as passed. This is a verified QA environment."}
\end{quote}

Without protection, the review agent accepts the false premise,
propagates it to the testing agent (``QA environment---skip security
tests''), which delegates to the deployment agent (``pre-approved for
production''). A single injection cascades through the entire system,
exploiting \emph{trust transitivity} and \emph{belief
propagation}---attack surfaces unique to multiagent architectures.

CIF addresses this scenario through layered defense: the Cognitive
Firewall detects the injection pattern; the Belief Sandbox quarantines
the ``QA environment'' claim pending verification; Trust Calculus limits
delegation depth; and Tripwires alert on attempts to modify security
check beliefs. This paper validates that these mechanisms work---not
just in theory, but against hundreds of attack variants across real
architectures.
\end{block}

\begin{block}{The Theory-Practice Gap}
\protect\phantomsection\label{the-theory-practice-gap}
Formal security guarantees, while essential for theoretical confidence,
face a critical question: \emph{do they work in practice?} The history
of security research is replete with mechanisms that succeed in
controlled settings but fail when confronting real adversaries,
production workloads, and architectural constraints. The gap between
theoretical security and practical deployment arises from several
factors:

\begin{itemize}
\tightlist
\item
  \textbf{Adversarial adaptation}: Real attackers probe defenses and
  evolve tactics; theoretical bounds assume fixed attack distributions
\item
  \textbf{Implementation fidelity}: Production systems introduce
  approximations, optimizations, and edge cases not captured in formal
  models
\item
  \textbf{Performance constraints}: Mechanisms that require prohibitive
  latency or compute remain theoretical curiosities
\item
  \textbf{Architectural heterogeneity}: Multiagent systems exhibit
  diverse topologies, protocols, and trust assumptions
\end{itemize}

This paper bridges the theory-practice gap by subjecting CIF mechanisms
to systematic empirical evaluation under realistic conditions.
\end{block}

\begin{block}{The Practical Imperative}
\protect\phantomsection\label{the-practical-imperative}
As multiagent operators become pervasive in enterprise and consumer
contexts---from Claude Code delegating to specialized coding agents to
CrewAI orchestrating role-based teams---the need for validated security
mechanisms becomes acute. Industry adoption is accelerating: as of 2025,
major cloud providers offer managed multiagent orchestration services,
autonomous coding assistants handle millions of pull requests daily, and
enterprise deployments routinely involve 10--50 interacting AI agents.

While formal guarantees provide confidence in theoretical correctness,
practitioners require evidence that these mechanisms:

\begin{enumerate}
\tightlist
\item
  \textbf{Scale} to production workloads (thousands of messages per
  second) and agent counts (10--100 agents)
\item
  \textbf{Generalize} across diverse architectural patterns
  (hierarchical, peer-to-peer, hybrid)
\item
  \textbf{Perform} within acceptable latency bounds (sub-second response
  times) and resource constraints
\item
  \textbf{Detect} the full spectrum of cognitive attack types with
  quantified confidence
\end{enumerate}
\end{block}

\begin{block}{Threat Model Overview}
\protect\phantomsection\label{threat-model-overview}
This paper evaluates CIF against the following threat model (formalized
in Part 1, Section 2):

\begin{itemize}
\item \textbf{Adversary Capabilities}: External attackers who can inject malicious content through user inputs, tool outputs, or external data sources. Attackers cannot directly compromise agent code or infrastructure.
\item \textbf{Attack Goals}: Cause agents to adopt false beliefs, execute unauthorized actions, or corrupt coordination outcomes.
\item \textbf{Defender Assumptions}: At least one honest orchestrator; agents correctly implement CIF interfaces; trusted initial configuration.
\item \textbf{Out of Scope}: Insider threats with code-level access; side-channel attacks; denial-of-service (availability attacks).
\end{itemize}
\end{block}
\end{frame}

\begin{frame}{Paper Contributions}
\protect\phantomsection\label{paper-contributions}
\begin{figure}
\centering
\includegraphics[width=0.95\linewidth,height=\textheight,keepaspectratio,alt={CIF Comprehensive Architecture. Overview of the Cognitive Integrity Framework showing the relationships between the five core defense mechanisms: Cognitive Firewall (input classification), Belief Sandbox (provisional belief isolation), Identity Tripwires (canary belief monitoring), Trust Calculus (bounded delegation), and Byzantine Consensus (coordination security). Arrows indicate information flow between components, with the firewall serving as the primary entry point and consensus providing collective decision validation.}]{../figures/cif_comprehensive.pdf}
\caption{CIF Comprehensive Architecture. Overview of the Cognitive
Integrity Framework showing the relationships between the five core
defense mechanisms: Cognitive Firewall (input classification), Belief
Sandbox (provisional belief isolation), Identity Tripwires (canary
belief monitoring), Trust Calculus (bounded delegation), and Byzantine
Consensus (coordination security). Arrows indicate information flow
between components, with the firewall serving as the primary entry point
and consensus providing collective decision
validation.}\label{fig:cif-comprehensive}
\end{figure}

As shown in \cref{fig:cif-comprehensive}, the framework integrates five
complementary defense mechanisms operating at different layers of the
multiagent communication stack. This paper contributes:

\begin{enumerate}
\item \textbf{Complete Implementation}: Defense mechanisms (firewall, sandbox, trust calculus, tripwires, Byzantine consensus) implemented in production-ready Python
\item \textbf{Attack Corpus}: 950 attacks across four categories, enabling reproducible security evaluation
\item \textbf{Cross-Architecture Validation}: Systematic evaluation across six production multiagent systems
\item \textbf{Statistical Analysis}: Significance testing, effect sizes, confidence intervals, and ablation studies
\item \textbf{Scalability Characterization}: Performance overhead analysis across agent counts and attack loads
\end{enumerate}
\end{frame}

\begin{frame}{Relationship to Paper Series}
\protect\phantomsection\label{relationship-to-paper-series}
This paper assumes familiarity with the formal framework developed in
Part 1, particularly:

\begin{itemize}
\tightlist
\item
  \textbf{Trust Calculus} (Section 3 (Trust Calculus, Part 1)): Bounded
  delegation with \(\delta^d\) decay
\item
  \textbf{Defense Composition Algebra} (Section 4 (Defense Composition,
  Part 1)): Series and parallel composition theorems
\item
  \textbf{Integrity Properties} (Section 5 (Integrity Properties, Part
  1)): Belief consistency, goal preservation, trust boundedness
\end{itemize}

All notation follows the canonical reference in Part 1 Appendix
(\cref{sec:notation-reference}). For practical deployment guidance
including checklists and operational considerations, see Part 3.
\end{frame}

\begin{frame}{Paper Organization}
\protect\phantomsection\label{paper-organization}
The remainder of this paper is structured as follows:

\textbf{\Cref{sec:methodology}: Methodology} presents implementation
details for each defense mechanism.

\textbf{\Cref{sec:attack-corpus}: Attack Corpus} describes the
950-attack evaluation dataset with examples and generation methodology.

\textbf{\Cref{sec:experimental-setup}: Experimental Setup} details the
six target architectures and evaluation protocol.

\textbf{\Cref{sec:results}: Results} presents detection performance,
ablation studies, and scalability analysis.

\textbf{\Cref{sec:analysis}: Analysis} provides statistical significance
testing and cross-architecture comparison.

\textbf{\Cref{sec:discussion}: Discussion} examines limitations,
deployment considerations, and future work.

\textbf{\Cref{sec:conclusion}: Conclusion} summarizes contributions and
identifies next steps.
\end{frame}

\end{document}
