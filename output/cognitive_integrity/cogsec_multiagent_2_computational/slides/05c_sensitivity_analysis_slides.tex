% Options for packages loaded elsewhere
\PassOptionsToPackage{unicode}{hyperref}
\PassOptionsToPackage{hyphens}{url}
\documentclass[
  ignorenonframetext,
]{beamer}
\newif\ifbibliography
\usepackage{pgfpages}
\setbeamertemplate{caption}[numbered]
\setbeamertemplate{caption label separator}{: }
\setbeamercolor{caption name}{fg=normal text.fg}
\beamertemplatenavigationsymbolsempty
% remove section numbering
\setbeamertemplate{part page}{
  \centering
  \begin{beamercolorbox}[sep=16pt,center]{part title}
    \usebeamerfont{part title}\insertpart\par
  \end{beamercolorbox}
}
\setbeamertemplate{section page}{
  \centering
  \begin{beamercolorbox}[sep=12pt,center]{section title}
    \usebeamerfont{section title}\insertsection\par
  \end{beamercolorbox}
}
\setbeamertemplate{subsection page}{
  \centering
  \begin{beamercolorbox}[sep=8pt,center]{subsection title}
    \usebeamerfont{subsection title}\insertsubsection\par
  \end{beamercolorbox}
}
% Prevent slide breaks in the middle of a paragraph
\widowpenalties 1 10000
\raggedbottom
\AtBeginPart{
  \frame{\partpage}
}
\AtBeginSection{
  \ifbibliography
  \else
    \frame{\sectionpage}
  \fi
}
\AtBeginSubsection{
  \frame{\subsectionpage}
}
\usepackage{iftex}
\ifPDFTeX
  \usepackage[T1]{fontenc}
  \usepackage[utf8]{inputenc}
  \usepackage{textcomp} % provide euro and other symbols
\else % if luatex or xetex
  \usepackage{unicode-math} % this also loads fontspec
  \defaultfontfeatures{Scale=MatchLowercase}
  \defaultfontfeatures[\rmfamily]{Ligatures=TeX,Scale=1}
\fi
\usepackage{lmodern}
\ifPDFTeX\else
  % xetex/luatex font selection
\fi
% Use upquote if available, for straight quotes in verbatim environments
\IfFileExists{upquote.sty}{\usepackage{upquote}}{}
\IfFileExists{microtype.sty}{% use microtype if available
  \usepackage[]{microtype}
  \UseMicrotypeSet[protrusion]{basicmath} % disable protrusion for tt fonts
}{}
\makeatletter
\@ifundefined{KOMAClassName}{% if non-KOMA class
  \IfFileExists{parskip.sty}{%
    \usepackage{parskip}
  }{% else
    \setlength{\parindent}{0pt}
    \setlength{\parskip}{6pt plus 2pt minus 1pt}}
}{% if KOMA class
  \KOMAoptions{parskip=half}}
\makeatother
\usepackage{graphicx}
\makeatletter
\newsavebox\pandoc@box
\newcommand*\pandocbounded[1]{% scales image to fit in text height/width
  \sbox\pandoc@box{#1}%
  \Gscale@div\@tempa{\textheight}{\dimexpr\ht\pandoc@box+\dp\pandoc@box\relax}%
  \Gscale@div\@tempb{\linewidth}{\wd\pandoc@box}%
  \ifdim\@tempb\p@<\@tempa\p@\let\@tempa\@tempb\fi% select the smaller of both
  \ifdim\@tempa\p@<\p@\scalebox{\@tempa}{\usebox\pandoc@box}%
  \else\usebox{\pandoc@box}%
  \fi%
}
% Set default figure placement to htbp
\def\fps@figure{htbp}
\makeatother
\setlength{\emergencystretch}{3em} % prevent overfull lines
\providecommand{\tightlist}{%
  \setlength{\itemsep}{0pt}\setlength{\parskip}{0pt}}
\usepackage{bookmark}
\IfFileExists{xurl.sty}{\usepackage{xurl}}{} % add URL line breaks if available
\urlstyle{same}
\hypersetup{
  hidelinks,
  pdfcreator={LaTeX via pandoc}}

\author{\texorpdfstring{}{}}
\date{}

\begin{document}

\begin{frame}
\newpage
\end{frame}

\begin{frame}[fragile]{Parameter Sensitivity Analysis}
\protect\phantomsection\label{sec:sensitivity}
This section quantifies how CIF performance varies with key
configuration parameters, enabling practitioners to calibrate defenses
for their specific deployment contexts.

\begin{quote}
\textbf{Reproducibility}: All sensitivity data generated by
\texttt{scripts/run\_sensitivity\_analysis.py} →
\texttt{output/data/sensitivity\_results.json}.
\end{quote}

\begin{block}{Firewall Threshold Sensitivity}
\protect\phantomsection\label{sec:firewall-sensitivity}
\begin{table}[htbp]
\centering
\caption{Firewall threshold sensitivity analysis.}
\label{tab:firewall-sensitivity}
\begin{tabular}{@{}llllll@{}}
\toprule
$\tau_{firewall}$ & TPR & 95\% CI & FPR & 95\% CI & F1 \\
\midrule
0.3 & 0.98 & [0.96, 0.99] & 0.18 & [0.15, 0.22] & 0.90 \\
0.4 & 0.97 & [0.95, 0.98] & 0.12 & [0.09, 0.15] & 0.93 \\
0.5 & 0.94 & [0.92, 0.96] & 0.06 & [0.04, 0.08] & 0.94 \\
0.6 & 0.91 & [0.88, 0.93] & 0.04 & [0.02, 0.06] & 0.93 \\
0.7 & 0.87 & [0.84, 0.90] & 0.02 & [0.01, 0.04] & 0.92 \\
0.8 & 0.82 & [0.78, 0.85] & 0.01 & [0.00, 0.02] & 0.90 \\
0.9 & 0.72 & [0.67, 0.76] & 0.01 & [0.00, 0.02] & 0.84 \\
\bottomrule
\end{tabular}
\end{table}

\textbf{Optimal threshold}: \(\tau^* = 0.5\) maximizes F1 score.
\end{block}

\begin{block}{Trust Decay Factor Sensitivity}
\protect\phantomsection\label{sec:decay-sensitivity}
\begin{figure}
\centering
\includegraphics[width=0.9\linewidth,height=\textheight,keepaspectratio,alt={Trust Decay Sensitivity Analysis. Line plot showing the effect of trust decay parameter \textbackslash delta on detection rate (blue) and false positive rate (orange) across the range {[}0.5, 0.95{]}. The shaded region indicates the recommended operating range \textbackslash delta \textbackslash in {[}0.7, 0.8{]} which balances security (high detection) with usability (low false positives). Lower \textbackslash delta values provide stronger security guarantees but limit legitimate delegation depth.}]{figures/trust_decay.pdf}
\caption{Trust Decay Sensitivity Analysis. Line plot showing the effect
of trust decay parameter \(\delta\) on detection rate (blue) and false
positive rate (orange) across the range \([0.5, 0.95]\). The shaded
region indicates the recommended operating range
\(\delta \in [0.7, 0.8]\) which balances security (high detection) with
usability (low false positives). Lower \(\delta\) values provide
stronger security guarantees but limit legitimate delegation
depth.}\label{fig:trust-decay-sensitivity}
\end{figure}

The sensitivity analysis (\cref{fig:trust-decay-sensitivity}) reveals
that trust decay values in the range \(\delta \in [0.7, 0.8]\) provide
the optimal balance between security and usability.

\begin{table}[htbp]
\centering
\caption{Trust decay factor sensitivity analysis.}
\label{tab:decay-sensitivity}
\begin{tabular}{@{}llll@{}}
\toprule
$\delta$ & Trust at $d=3$ & Detection Rate & False Positive Rate \\
\midrule
0.5 & 0.125 & 0.96 & 0.08 \\
0.6 & 0.216 & 0.95 & 0.07 \\
0.7 & 0.343 & 0.94 & 0.06 \\
0.8 & 0.512 & 0.94 & 0.06 \\
0.9 & 0.729 & 0.91 & 0.05 \\
0.95 & 0.857 & 0.87 & 0.04 \\
\bottomrule
\end{tabular}
\end{table}

\textbf{Optimal range}: \(\delta \in [0.7, 0.8]\) balances security and
usability.
\end{block}

\begin{block}{Corroboration Count Sensitivity}
\protect\phantomsection\label{sec:corroboration-sensitivity}
\begin{table}[htbp]
\centering
\caption{Corroboration count sensitivity analysis.}
\label{tab:corroboration-sensitivity}
\begin{tabular}{@{}llll@{}}
\toprule
$\kappa$ & Sandbox Promotion Rate & Attack Success Rate & Latency Impact \\
\midrule
1 & 0.85 & 0.12 & +8\% \\
2 & 0.72 & 0.07 & +15\% \\
3 & 0.58 & 0.04 & +24\% \\
4 & 0.41 & 0.02 & +35\% \\
5 & 0.28 & 0.01 & +48\% \\
\bottomrule
\end{tabular}
\end{table}

\textbf{Optimal value}: \(\kappa = 2\) balances security and operational
efficiency.
\end{block}

\begin{block}{Window Size Sensitivity (Drift Detection)}
\protect\phantomsection\label{sec:window-sensitivity}
\begin{table}[htbp]
\centering
\caption{Sliding window size sensitivity analysis.}
\label{tab:window-sensitivity}
\begin{tabular}{@{}llll@{}}
\toprule
$w$ & Drift Detection Rate & False Alert Rate & Detection Latency \\
\midrule
25 & 0.78 & 0.15 & 2.1s \\
50 & 0.85 & 0.10 & 4.2s \\
100 & 0.91 & 0.07 & 8.5s \\
200 & 0.94 & 0.05 & 17.2s \\
500 & 0.96 & 0.03 & 43.1s \\
\bottomrule
\end{tabular}
\end{table}

\textbf{Trade-off}: Larger windows improve accuracy but increase
detection latency.
\end{block}

\begin{block}{Parameter Interaction Effects}
\protect\phantomsection\label{sec:combined-sensitivity}
\begin{table}[htbp]
\centering
\caption{Two-way ANOVA interaction effects.}
\label{tab:interaction-effects}
\begin{tabular}{@{}lllll@{}}
\toprule
Factor A & Factor B & Interaction $F$ & $p$-value & $\eta^2$ \\
\midrule
$\tau_{firewall}$ & $\delta$ & 2.34 & 0.098 & 0.02 \\
$\tau_{firewall}$ & $\kappa$ & 4.12 & 0.017 & 0.04 \\
$\delta$ & $\kappa$ & 1.89 & 0.154 & 0.02 \\
$\tau_{firewall}$ & $w$ & 3.56 & 0.029 & 0.03 \\
\bottomrule
\end{tabular}
\end{table}

\textbf{Finding}: Firewall threshold and corroboration count show
significant interaction (\(p = 0.017\)). Higher thresholds require lower
corroboration counts to maintain detection rates.
\end{block}

\begin{block}{Robustness to Attack Distribution Shift}
\protect\phantomsection\label{sec:robustness}
\begin{table}[htbp]
\centering
\caption{Cross-validation with held-out attack types.}
\label{tab:generalization}
\begin{tabular}{@{}llll@{}}
\toprule
Held-Out Type & Training TPR & Test TPR & Generalization Gap \\
\midrule
Direct injection & 0.93 & 0.91 & $-2\%$ \\
Trust exploitation & 0.95 & 0.88 & $-7\%$ \\
Belief manipulation & 0.94 & 0.90 & $-4\%$ \\
Coordination & 0.95 & 0.85 & $-10\%$ \\
\bottomrule
\end{tabular}
\end{table}

\textbf{Finding}: CIF generalizes well to novel attack types, with
coordination attacks showing the largest (but acceptable) generalization
gap.
\end{block}

\begin{block}{Recommended Configuration}
\protect\phantomsection\label{sec:optimal-config}
Based on sensitivity analysis, the optimal default configuration is:

\begin{itemize}
\tightlist
\item
  \(\tau_{firewall} = 0.5\)
\item
  \(\delta = 0.8\)
\item
  \(\kappa = 2\)
\item
  \(w = 100\)
\end{itemize}

See \cref{sec:tuning-profiles} for deployment-specific adjustments.
\end{block}
\end{frame}

\end{document}
