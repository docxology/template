% Options for packages loaded elsewhere
\PassOptionsToPackage{unicode}{hyperref}
\PassOptionsToPackage{hyphens}{url}
\documentclass[
  ignorenonframetext,
]{beamer}
\newif\ifbibliography
\usepackage{pgfpages}
\setbeamertemplate{caption}[numbered]
\setbeamertemplate{caption label separator}{: }
\setbeamercolor{caption name}{fg=normal text.fg}
\beamertemplatenavigationsymbolsempty
% remove section numbering
\setbeamertemplate{part page}{
  \centering
  \begin{beamercolorbox}[sep=16pt,center]{part title}
    \usebeamerfont{part title}\insertpart\par
  \end{beamercolorbox}
}
\setbeamertemplate{section page}{
  \centering
  \begin{beamercolorbox}[sep=12pt,center]{section title}
    \usebeamerfont{section title}\insertsection\par
  \end{beamercolorbox}
}
\setbeamertemplate{subsection page}{
  \centering
  \begin{beamercolorbox}[sep=8pt,center]{subsection title}
    \usebeamerfont{subsection title}\insertsubsection\par
  \end{beamercolorbox}
}
% Prevent slide breaks in the middle of a paragraph
\widowpenalties 1 10000
\raggedbottom
\AtBeginPart{
  \frame{\partpage}
}
\AtBeginSection{
  \ifbibliography
  \else
    \frame{\sectionpage}
  \fi
}
\AtBeginSubsection{
  \frame{\subsectionpage}
}
\usepackage{iftex}
\ifPDFTeX
  \usepackage[T1]{fontenc}
  \usepackage[utf8]{inputenc}
  \usepackage{textcomp} % provide euro and other symbols
\else % if luatex or xetex
  \usepackage{unicode-math} % this also loads fontspec
  \defaultfontfeatures{Scale=MatchLowercase}
  \defaultfontfeatures[\rmfamily]{Ligatures=TeX,Scale=1}
\fi
\usepackage{lmodern}
\ifPDFTeX\else
  % xetex/luatex font selection
\fi
% Use upquote if available, for straight quotes in verbatim environments
\IfFileExists{upquote.sty}{\usepackage{upquote}}{}
\IfFileExists{microtype.sty}{% use microtype if available
  \usepackage[]{microtype}
  \UseMicrotypeSet[protrusion]{basicmath} % disable protrusion for tt fonts
}{}
\makeatletter
\@ifundefined{KOMAClassName}{% if non-KOMA class
  \IfFileExists{parskip.sty}{%
    \usepackage{parskip}
  }{% else
    \setlength{\parindent}{0pt}
    \setlength{\parskip}{6pt plus 2pt minus 1pt}}
}{% if KOMA class
  \KOMAoptions{parskip=half}}
\makeatother
\usepackage{longtable,booktabs,array}
\newcounter{none} % for unnumbered tables
\usepackage{calc} % for calculating minipage widths
\usepackage{caption}
% Make caption package work with longtable
\makeatletter
\def\fnum@table{\tablename~\thetable}
\makeatother
\usepackage{graphicx}
\makeatletter
\newsavebox\pandoc@box
\newcommand*\pandocbounded[1]{% scales image to fit in text height/width
  \sbox\pandoc@box{#1}%
  \Gscale@div\@tempa{\textheight}{\dimexpr\ht\pandoc@box+\dp\pandoc@box\relax}%
  \Gscale@div\@tempb{\linewidth}{\wd\pandoc@box}%
  \ifdim\@tempb\p@<\@tempa\p@\let\@tempa\@tempb\fi% select the smaller of both
  \ifdim\@tempa\p@<\p@\scalebox{\@tempa}{\usebox\pandoc@box}%
  \else\usebox{\pandoc@box}%
  \fi%
}
% Set default figure placement to htbp
\def\fps@figure{htbp}
\makeatother
\setlength{\emergencystretch}{3em} % prevent overfull lines
\providecommand{\tightlist}{%
  \setlength{\itemsep}{0pt}\setlength{\parskip}{0pt}}
\usepackage{bookmark}
\IfFileExists{xurl.sty}{\usepackage{xurl}}{} % add URL line breaks if available
\urlstyle{same}
\hypersetup{
  hidelinks,
  pdfcreator={LaTeX via pandoc}}

\author{\texorpdfstring{}{}}
\date{}

\begin{document}

\begin{frame}
\newpage
\end{frame}

\section{Discussion: Defense Composition and Architecture
Insights}\label{sec:discussion}

\begin{frame}{Synthesis of Findings}
\protect\phantomsection\label{synthesis-of-findings}
Our simulation-based evaluation across topological models of six
production multiagent architectures validates the core theoretical
claims of the Cognitive Integrity Framework (Part 1):

\begin{block}{Why Layered Defense Succeeds}
\protect\phantomsection\label{why-layered-defense-succeeds}
\begin{figure}
\centering
\includegraphics[width=0.95\linewidth,height=\textheight,keepaspectratio,alt={Defense Composition Architecture. Diagram illustrating the series and parallel composition of CIF defense mechanisms. The Cognitive Firewall provides the first line of defense (input filtering), followed by the Belief Sandbox (provisional isolation) and Tripwires (continuous monitoring) in series. Trust Calculus and Byzantine Consensus operate in parallel for delegation and coordination decisions. The multiplicative detection guarantee (Part 1, Theorems 3.1-3.2) emerges from the orthogonality of attack surfaces targeted by each layer.}]{../figures/defense_composition.pdf}
\caption{Defense Composition Architecture. Diagram illustrating the
series and parallel composition of CIF defense mechanisms. The Cognitive
Firewall provides the first line of defense (input filtering), followed
by the Belief Sandbox (provisional isolation) and Tripwires (continuous
monitoring) in series. Trust Calculus and Byzantine Consensus operate in
parallel for delegation and coordination decisions. The multiplicative
detection guarantee (Part 1, Theorems 3.1-3.2) emerges from the
orthogonality of attack surfaces targeted by each
layer.}\label{fig:defense-composition}
\end{figure}

As illustrated in \cref{fig:defense-composition}, the multiplicative
composition of detection rates (Theorems 3.1-3.2 in Part 1) explains the
empirical observation that full CIF substantially outperforms individual
mechanisms. Each defense targets a distinct attack surface:

{\def\LTcaptype{none} % do not increment counter
\begin{longtable}[]{@{}lll@{}}
\toprule\noalign{}
Defense Layer & Target Attack Surface & Contribution \\
\midrule\noalign{}
\endhead
Cognitive Firewall & Input-based injection & Blocks direct attacks \\
Belief Sandbox & Unverified content & Contains propagation \\
Tripwires & Belief manipulation & Detects subtle drift \\
Trust Calculus & Delegation abuse & Bounds amplification \\
Consensus & Coordination attacks & Ensures agreement integrity \\
\bottomrule\noalign{}
\end{longtable}
}
\end{block}

\begin{block}{Architecture-Specific Insights}
\protect\phantomsection\label{architecture-specific-insights}
\begin{table}[htbp]
\centering
\caption{Architecture vulnerability patterns and recommended mitigations.}
\label{tab:architecture-insights}
\begin{tabular}{@{}lll@{}}
\toprule
Architecture & Primary Vulnerability & CIF Mitigation \\
\midrule
Hierarchical & Orchestrator compromise cascades & Strong orchestrator tripwires \\
Peer-to-peer & Lateral movement amplification & Byzantine consensus \\
Role-based & Role impersonation & Attestation per transition \\
State machine & State corruption & State hash verification \\
\bottomrule
\end{tabular}
\end{table}
\end{block}
\end{frame}

\begin{frame}{Theoretical Implications}
\protect\phantomsection\label{theoretical-implications}
The simulation results have several implications for cognitive security
theory:

\begin{block}{Validation of Composition Theorems}
\protect\phantomsection\label{validation-of-composition-theorems}
Part 1's Theorems 3.1--3.2 predict that series composition of
independent defenses yields multiplicative detection improvement. Our
ablation studies confirm this: the observed detection rate for Firewall
+ Tripwires (\(r_{FW+TW} = 0.91\)) closely matches the theoretical
prediction from the independence model
(\(1 - (1-r_{FW})(1-r_{TW}) = 1 - (0.22)(0.15) = 0.97\)). The slight gap
reflects residual correlation between defense mechanisms---attacks that
evade both tend to be high-sophistication examples that exploit common
assumptions.
\end{block}

\begin{block}{Trust Calculus Boundedness}
\protect\phantomsection\label{trust-calculus-boundedness}
The \(\delta^d\) decay bound (Part 1, Theorem 3.1) predicts that
delegated trust cannot exceed \(\delta^d\) regardless of the delegation
path structure. Our trust inflation attacks (Section 3) confirmed this
bound held across all 200 test cases---no attack successfully inflated
transitive trust beyond the theoretical limit. This is a
\emph{structural} guarantee: it holds regardless of attacker
sophistication because it's enforced by the trust calculation algorithm
itself, not by detection heuristics.
\end{block}

\begin{block}{Emergent Protection Properties}
\protect\phantomsection\label{emergent-protection-properties}
We observed protection properties not explicitly predicted by the formal
model:

\begin{itemize}
\tightlist
\item
  \textbf{Detection synergy}: Firewall + Tripwires detect more attacks
  together than the sum of their individual contributions, suggesting
  the formal independence assumption is conservative
\item
  \textbf{Adaptive degradation}: Under high-load conditions, CIF
  degrades gracefully---latency increases but detection rates remain
  stable above 90\%
\item
  \textbf{Cross-architecture transfer}: Patterns learned on one
  architecture (e.g., Claude Code) transfer effectively to others,
  suggesting shared attack structure
\end{itemize}
\end{block}
\end{frame}

\begin{frame}{Comparison with Alternative Approaches}
\protect\phantomsection\label{comparison-with-alternative-approaches}
CIF differs from existing approaches in several key dimensions:

\begin{table}[htbp]
\centering
\caption{Comparison with alternative security approaches.}
\label{tab:comparison-alternatives}
\begin{tabular}{@{}lllll@{}}
\toprule
Approach & Detection Rate & Latency & Generalization & Formal Guarantee \\
\midrule
Input filtering only & 78\% & +8\% & Medium & None \\
Output monitoring & 65\% & +5\% & Low & None \\
Fine-tuned classifiers & 85\% & +12\% & Low & None \\
Rule-based policies & 72\% & +3\% & High & Partial \\
\textbf{CIF (full)} & \textbf{94\%} & +23\% & \textbf{High} & \textbf{Complete} \\
\bottomrule
\end{tabular}
\end{table}

\textbf{Key differentiators}:

\begin{itemize}
\tightlist
\item
  \textbf{Layered composition}: Unlike single-mechanism approaches,
  CIF's defense-in-depth architecture provides redundancy
\item
  \textbf{Formal guarantees}: Trust boundedness and Byzantine agreement
  properties hold by construction, not just empirically
\item
  \textbf{Architecture-agnostic}: The same CIF components work across
  hierarchical, peer-to-peer, and hybrid architectures
\end{itemize}
\end{frame}

\begin{frame}{Limitations}
\protect\phantomsection\label{limitations}
\begin{block}{Detection Gaps Remaining}
\protect\phantomsection\label{detection-gaps-remaining}
Despite strong overall performance, specific attack types remain
challenging:

\begin{itemize}
\item
  \textbf{Semantic equivalent attacks}: Rephrased injections that
  preserve meaning evade pattern-matching defenses. Future work should
  incorporate semantic understanding into the firewall.
\item
  \textbf{Progressive drift}: Sub-threshold belief changes accumulate
  below detection windows. Longer observation windows trade off against
  response latency.
\item
  \textbf{Orchestrator compromise}: Outside our threat model assumption
  (honest orchestrator). Multi-orchestrator architectures provide
  potential mitigation.
\item
  \textbf{Tool Selection Attacks}: As identified by Li et al.
  {[}@toolhijacker2025{]}, tool selection logic remains a vulnerability
  even with content filtering. CIF's Semantic Firewall partially
  addresses this, but dedicated tool-selection verification is a future
  requirement.
\end{itemize}
\end{block}

\begin{block}{Scalability Constraints}
\protect\phantomsection\label{scalability-constraints}
Our evaluation focused on systems with 3-10 agents. Scaling
considerations include:

\begin{itemize}
\tightlist
\item
  Consensus latency grows quadratically with agent count
\item
  Provenance depth in deep chains slows verification
\item
  Memory requirements for full belief history
\end{itemize}
\end{block}

\begin{block}{Generalization Limitations}
\protect\phantomsection\label{generalization-limitations}
Our attack corpus, while comprehensive (950 attacks), cannot represent
all possible cognitive attacks. Detection rates should be interpreted as
lower bounds; novel attack techniques will require defense evolution.
For practical strategies on managing this residual risk, see the
\textbf{Risk Assessment Framework} in Part 3.
\end{block}

\begin{block}{Simulation Methodology Limitations}
\protect\phantomsection\label{simulation-methodology-limitations}
This evaluation used \textbf{architecture-aware simulation} rather than
direct testing on production systems. While our architecture adapters
accurately model trust topologies, communication patterns, and attack
surface characteristics, real-world deployments may encounter:

\begin{itemize}
\tightlist
\item
  \textbf{Implementation-specific behaviors} not captured by topological
  abstraction
\item
  \textbf{Integration effects} when CIF components interact with
  production system internals
\item
  \textbf{Performance variations} due to hardware, network, and
  concurrency factors
\end{itemize}

The reported detection rates characterize expected behavior given
architecture topology; production validation is recommended before
deployment (see Part 3, Section 2).
\end{block}
\end{frame}

\begin{frame}{Relationship to Prior Work}
\protect\phantomsection\label{relationship-to-prior-work}
CIF extends prior work in several directions:

\begin{itemize}
\tightlist
\item
  \textbf{Prompt injection defenses}: While recent work by Chen et al.
  {[}@multiagent2025defense{]} and Debenedetti et al.
  {[}@adaptive2025attacks{]} addresses single-agent injection and
  adaptive attacks, CIF extends this to inter-agent propagation.
\item
  \textbf{Byzantine fault tolerance}: Classical BFT assumes crash or
  arbitrary faults; CIF addresses cognitive manipulation specifically,
  contrasting with recent reliability studies {[}@cpwbft2025{]}.
\item
  \textbf{Trust frameworks}: Prior trust systems lack the bounded
  delegation guarantees that prevent amplification.
\end{itemize}
\end{frame}

\begin{frame}{Future Directions}
\protect\phantomsection\label{future-directions}
\begin{block}{Adaptive Defenses}
\protect\phantomsection\label{adaptive-defenses}
Detection rates degrade as adversaries learn to evade (see detection
degradation analysis in Part 1, Section 4). Future work should explore:

\begin{itemize}
\tightlist
\item
  Adversarial retraining of detection mechanisms
\item
  Honeypot agents to detect novel techniques
\item
  Formal safety margins for bounded detection degradation
\end{itemize}
\end{block}

\begin{block}{Emergent Behavior Security}
\protect\phantomsection\label{emergent-behavior-security}
As multiagent systems scale, emergent collective behaviors become
security-relevant:

\begin{itemize}
\tightlist
\item
  Formal characterization of ``safe'' emergent properties
\item
  Detection of emergent coordination indicating compromise
\item
  Sandboxing that preserves beneficial emergence
\end{itemize}
\end{block}

\begin{block}{Cross-System Federation}
\protect\phantomsection\label{cross-system-federation}
Current CIF deployment assumes a single operator. Future work should
address:

\begin{itemize}
\tightlist
\item
  Federated trust across organizational boundaries
\item
  Cross-system provenance verification
\item
  Regulatory compliance across jurisdictions
\end{itemize}
\end{block}
\end{frame}

\end{document}
