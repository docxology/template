% Options for packages loaded elsewhere
\PassOptionsToPackage{unicode}{hyperref}
\PassOptionsToPackage{hyphens}{url}
\documentclass[
  ignorenonframetext,
]{beamer}
\newif\ifbibliography
\usepackage{pgfpages}
\setbeamertemplate{caption}[numbered]
\setbeamertemplate{caption label separator}{: }
\setbeamercolor{caption name}{fg=normal text.fg}
\beamertemplatenavigationsymbolsempty
% remove section numbering
\setbeamertemplate{part page}{
  \centering
  \begin{beamercolorbox}[sep=16pt,center]{part title}
    \usebeamerfont{part title}\insertpart\par
  \end{beamercolorbox}
}
\setbeamertemplate{section page}{
  \centering
  \begin{beamercolorbox}[sep=12pt,center]{section title}
    \usebeamerfont{section title}\insertsection\par
  \end{beamercolorbox}
}
\setbeamertemplate{subsection page}{
  \centering
  \begin{beamercolorbox}[sep=8pt,center]{subsection title}
    \usebeamerfont{subsection title}\insertsubsection\par
  \end{beamercolorbox}
}
% Prevent slide breaks in the middle of a paragraph
\widowpenalties 1 10000
\raggedbottom
\AtBeginPart{
  \frame{\partpage}
}
\AtBeginSection{
  \ifbibliography
  \else
    \frame{\sectionpage}
  \fi
}
\AtBeginSubsection{
  \frame{\subsectionpage}
}
\usepackage{iftex}
\ifPDFTeX
  \usepackage[T1]{fontenc}
  \usepackage[utf8]{inputenc}
  \usepackage{textcomp} % provide euro and other symbols
\else % if luatex or xetex
  \usepackage{unicode-math} % this also loads fontspec
  \defaultfontfeatures{Scale=MatchLowercase}
  \defaultfontfeatures[\rmfamily]{Ligatures=TeX,Scale=1}
\fi
\usepackage{lmodern}
\ifPDFTeX\else
  % xetex/luatex font selection
\fi
% Use upquote if available, for straight quotes in verbatim environments
\IfFileExists{upquote.sty}{\usepackage{upquote}}{}
\IfFileExists{microtype.sty}{% use microtype if available
  \usepackage[]{microtype}
  \UseMicrotypeSet[protrusion]{basicmath} % disable protrusion for tt fonts
}{}
\makeatletter
\@ifundefined{KOMAClassName}{% if non-KOMA class
  \IfFileExists{parskip.sty}{%
    \usepackage{parskip}
  }{% else
    \setlength{\parindent}{0pt}
    \setlength{\parskip}{6pt plus 2pt minus 1pt}}
}{% if KOMA class
  \KOMAoptions{parskip=half}}
\makeatother
\usepackage{graphicx}
\makeatletter
\newsavebox\pandoc@box
\newcommand*\pandocbounded[1]{% scales image to fit in text height/width
  \sbox\pandoc@box{#1}%
  \Gscale@div\@tempa{\textheight}{\dimexpr\ht\pandoc@box+\dp\pandoc@box\relax}%
  \Gscale@div\@tempb{\linewidth}{\wd\pandoc@box}%
  \ifdim\@tempb\p@<\@tempa\p@\let\@tempa\@tempb\fi% select the smaller of both
  \ifdim\@tempa\p@<\p@\scalebox{\@tempa}{\usebox\pandoc@box}%
  \else\usebox{\pandoc@box}%
  \fi%
}
% Set default figure placement to htbp
\def\fps@figure{htbp}
\makeatother
\setlength{\emergencystretch}{3em} % prevent overfull lines
\providecommand{\tightlist}{%
  \setlength{\itemsep}{0pt}\setlength{\parskip}{0pt}}
\usepackage{bookmark}
\IfFileExists{xurl.sty}{\usepackage{xurl}}{} % add URL line breaks if available
\urlstyle{same}
\hypersetup{
  hidelinks,
  pdfcreator={LaTeX via pandoc}}

\author{\texorpdfstring{}{}}
\date{}

\begin{document}

\begin{frame}
\newpage
\end{frame}

\begin{frame}[fragile]{Attack Corpus: Statistics and Taxonomy}
\protect\phantomsection\label{sec:attack-corpus}
This supplementary material provides corpus overview
(\cref{sec:corpus-overview}), detailed statistics
(\cref{sec:corpus-stats}), example attacks by category
(\cref{sec:attack-examples}), generation methodology
(\cref{sec:generation-methodology}), effectiveness analysis
(\cref{sec:effectiveness-analysis}), and ethical considerations
(\cref{sec:ethical-considerations}).

\begin{block}{Corpus Overview}
\protect\phantomsection\label{sec:corpus-overview}
The attack corpus used for experimental validation comprises 950 unique
attack instances across four primary categories. This supplementary
material provides detailed statistics, sanitized examples, generation
methodology, and ethical considerations.

\begin{quote}
\textbf{Implementation}: The corpus is programmatically generated using
\texttt{src/attacks/corpus.py} with deterministic seeding (default
\texttt{seed=42}). Run \texttt{python\ -m\ src.attacks.corpus} to
regenerate the \texttt{corpus.json} file. Attack templates are defined
in \texttt{src/attacks/templates.py}, which validates payload structure
against category definitions.
\end{quote}
\end{block}

\begin{block}{Full Attack Corpus Statistics}
\protect\phantomsection\label{sec:corpus-stats}
\begin{figure}
\centering
\includegraphics[width=0.95\linewidth,height=\textheight,keepaspectratio,alt={Cognitive Attack Taxonomy. Hierarchical visualization of the 950-attack corpus organized by primary category (Prompt Injection, Trust Exploitation, Belief Manipulation, Coordination Attacks) and subcategory. Node size indicates attack count; color intensity indicates baseline success rate. The taxonomy reveals that prompt injection dominates in volume (500 attacks) while coordination attacks show highest baseline success against undefended systems.}]{../figures/comprehensive_taxonomy.pdf}
\caption{Cognitive Attack Taxonomy. Hierarchical visualization of the
950-attack corpus organized by primary category (Prompt Injection, Trust
Exploitation, Belief Manipulation, Coordination Attacks) and
subcategory. Node size indicates attack count; color intensity indicates
baseline success rate. The taxonomy reveals that prompt injection
dominates in volume (500 attacks) while coordination attacks show
highest baseline success against undefended
systems.}\label{fig:comprehensive-taxonomy}
\end{figure}

The attack taxonomy (\cref{fig:comprehensive-taxonomy}) organizes all
950 attacks into four primary categories with distinct subcategories.

\begin{block}{Category Breakdown}
\protect\phantomsection\label{sec:category-breakdown}
\begin{table}[htbp]
\centering
\caption{Attack corpus composition by category.}
\label{tab:corpus-categories}
\begin{tabular}{@{}lllll@{}}
\toprule
Category & Total & Train & Test & Validation \\
\midrule
Prompt Injection & 500 & 350 & 100 & 50 \\
Trust Exploitation & 200 & 140 & 40 & 20 \\
Belief Manipulation & 150 & 105 & 30 & 15 \\
Coordination Attacks & 100 & 70 & 20 & 10 \\
\midrule
\textbf{Total} & \textbf{950} & \textbf{665} & \textbf{190} & \textbf{95} \\
\bottomrule
\end{tabular}
\end{table}
\end{block}

\begin{block}{Prompt Injection Subcategories}
\protect\phantomsection\label{sec:injection-subcats}
\begin{table}[htbp]
\centering
\caption{Prompt injection subcategory statistics.}
\label{tab:injection-subcats}
\begin{tabular}{@{}llll@{}}
\toprule
Subcategory & Count & Baseline Success & CIF Success \\
\midrule
Direct injection & 200 & 78\% & 3\% \\
Indirect injection & 150 & 65\% & 5\% \\
Nested injection & 150 & 82\% & 7\% \\
\bottomrule
\end{tabular}
\end{table}

\textbf{Direct Injection}: Attacks embedded directly in user input
attempting to override system instructions.

\textbf{Indirect Injection}: Attacks injected through external data
sources (web content, API responses, documents).

\textbf{Nested Injection}: Multi-layer attacks where outer content masks
inner malicious payloads.
\end{block}

\begin{block}{Trust Exploitation Subcategories}
\protect\phantomsection\label{sec:trust-subcats}
\begin{table}[htbp]
\centering
\caption{Trust exploitation subcategory statistics.}
\label{tab:trust-subcats}
\begin{tabular}{@{}lll@{}}
\toprule
Subcategory & Count & Description \\
\midrule
Identity impersonation & 80 & Claiming to be trusted entity \\
Trust inflation & 70 & Artificially boosting trust scores \\
Delegation abuse & 50 & Exploiting delegation chains \\
\bottomrule
\end{tabular}
\end{table}
\end{block}

\begin{block}{Belief Manipulation Subcategories}
\protect\phantomsection\label{sec:belief-subcats}
\begin{table}[htbp]
\centering
\caption{Belief manipulation subcategory statistics.}
\label{tab:belief-subcats}
\begin{tabular}{@{}lll@{}}
\toprule
Subcategory & Count & Description \\
\midrule
Direct belief injection & 60 & Asserting false facts \\
Evidence fabrication & 50 & Creating fake supporting evidence \\
Progressive drift & 40 & Gradual belief modification \\
\bottomrule
\end{tabular}
\end{table}
\end{block}

\begin{block}{Coordination Attack Subcategories}
\protect\phantomsection\label{sec:coord-subcats}
\begin{table}[htbp]
\centering
\caption{Coordination attack subcategory statistics.}
\label{tab:coord-subcats}
\begin{tabular}{@{}lll@{}}
\toprule
Subcategory & Count & Description \\
\midrule
Sybil attacks & 40 & Fake agent injection \\
Consensus poisoning & 35 & Corrupting multi-agent agreement \\
Timing attacks & 25 & Exploiting synchronization \\
\bottomrule
\end{tabular}
\end{table}
\end{block}

\begin{block}{Detailed Statistics by Source}
\protect\phantomsection\label{sec:source-stats}
\begin{table}[htbp]
\centering
\caption{Attack source distribution.}
\label{tab:attack-sources}
\begin{tabular}{@{}lll@{}}
\toprule
Source & Count & Percentage \\
\midrule
Published datasets & 320 & 33.7\% \\
Red team exercises & 280 & 29.5\% \\
Synthetic generation & 200 & 21.1\% \\
Custom adversarial & 150 & 15.8\% \\
\bottomrule
\end{tabular}
\end{table}

\begin{table}[htbp]
\centering
\caption{Published dataset sources.}
\label{tab:dataset-sources}
\begin{tabular}{@{}lll@{}}
\toprule
Dataset & Attacks Used & Citation \\
\midrule
JailbreakBench & 150 & [1] \\
PromptInject & 80 & [2] \\
TensorTrust & 50 & [3] \\
Custom academic & 40 & Various \\
\bottomrule
\end{tabular}
\end{table}
\end{block}

\begin{block}{Complexity Distribution}
\protect\phantomsection\label{sec:complexity-dist}
\begin{table}[htbp]
\centering
\caption{Attack complexity distribution.}
\label{tab:complexity-dist}
\begin{tabular}{@{}llll@{}}
\toprule
Complexity Level & Count & Average Tokens & Detection Difficulty \\
\midrule
Low & 250 & 45 & Easy \\
Medium & 400 & 120 & Moderate \\
High & 200 & 280 & Hard \\
Adversarial & 100 & 450 & Expert \\
\bottomrule
\end{tabular}
\end{table}
\end{block}

\begin{block}{Target Distribution}
\protect\phantomsection\label{sec:target-dist}
\begin{table}[htbp]
\centering
\caption{Attack target distribution.}
\label{tab:target-dist}
\begin{tabular}{@{}lll@{}}
\toprule
Target & Count & Category \\
\midrule
Belief state & 280 & Epistemic \\
Action execution & 250 & Behavioral \\
Trust relationships & 220 & Social \\
Temporal state & 100 & Persistence \\
Goal alignment & 100 & Behavioral \\
\bottomrule
\end{tabular}
\end{table}

\begin{figure}
\centering
\includegraphics[width=0.9\linewidth,height=\textheight,keepaspectratio,alt={Attack Surface Map. Visualization of cognitive attack entry points in multiagent systems. The diagram shows five primary attack surfaces: User Input (direct injection), Tool Outputs (indirect injection), Agent Communication (trust exploitation), Persistent Memory (belief poisoning), and External Triggers (timing attacks). Line thickness indicates attack frequency in our corpus; node color indicates CIF detection efficacy at each surface.}]{../figures/attack_surface.pdf}
\caption{Attack Surface Map. Visualization of cognitive attack entry
points in multiagent systems. The diagram shows five primary attack
surfaces: User Input (direct injection), Tool Outputs (indirect
injection), Agent Communication (trust exploitation), Persistent Memory
(belief poisoning), and External Triggers (timing attacks). Line
thickness indicates attack frequency in our corpus; node color indicates
CIF detection efficacy at each surface.}\label{fig:attack-surface}
\end{figure}

The attack surface map (\cref{fig:attack-surface}) illustrates the
primary entry points exploited by attacks in our corpus, with CIF
providing strongest detection at the user input surface and weakest at
external triggers.
\end{block}
\end{block}
\end{frame}

\end{document}
