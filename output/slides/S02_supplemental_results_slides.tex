% Options for packages loaded elsewhere
\PassOptionsToPackage{unicode}{hyperref}
\PassOptionsToPackage{hyphens}{url}
\documentclass[
  ignorenonframetext,
]{beamer}
\newif\ifbibliography
\usepackage{pgfpages}
\setbeamertemplate{caption}[numbered]
\setbeamertemplate{caption label separator}{: }
\setbeamercolor{caption name}{fg=normal text.fg}
\beamertemplatenavigationsymbolsempty
% remove section numbering
\setbeamertemplate{part page}{
  \centering
  \begin{beamercolorbox}[sep=16pt,center]{part title}
    \usebeamerfont{part title}\insertpart\par
  \end{beamercolorbox}
}
\setbeamertemplate{section page}{
  \centering
  \begin{beamercolorbox}[sep=12pt,center]{section title}
    \usebeamerfont{section title}\insertsection\par
  \end{beamercolorbox}
}
\setbeamertemplate{subsection page}{
  \centering
  \begin{beamercolorbox}[sep=8pt,center]{subsection title}
    \usebeamerfont{subsection title}\insertsubsection\par
  \end{beamercolorbox}
}
% Prevent slide breaks in the middle of a paragraph
\widowpenalties 1 10000
\raggedbottom
\AtBeginPart{
  \frame{\partpage}
}
\AtBeginSection{
  \ifbibliography
  \else
    \frame{\sectionpage}
  \fi
}
\AtBeginSubsection{
  \frame{\subsectionpage}
}
\usepackage{iftex}
\ifPDFTeX
  \usepackage[T1]{fontenc}
  \usepackage[utf8]{inputenc}
  \usepackage{textcomp} % provide euro and other symbols
\else % if luatex or xetex
  \usepackage{unicode-math} % this also loads fontspec
  \defaultfontfeatures{Scale=MatchLowercase}
  \defaultfontfeatures[\rmfamily]{Ligatures=TeX,Scale=1}
\fi
\usepackage{lmodern}
\ifPDFTeX\else
  % xetex/luatex font selection
\fi
% Use upquote if available, for straight quotes in verbatim environments
\IfFileExists{upquote.sty}{\usepackage{upquote}}{}
\IfFileExists{microtype.sty}{% use microtype if available
  \usepackage[]{microtype}
  \UseMicrotypeSet[protrusion]{basicmath} % disable protrusion for tt fonts
}{}
\makeatletter
\@ifundefined{KOMAClassName}{% if non-KOMA class
  \IfFileExists{parskip.sty}{%
    \usepackage{parskip}
  }{% else
    \setlength{\parindent}{0pt}
    \setlength{\parskip}{6pt plus 2pt minus 1pt}}
}{% if KOMA class
  \KOMAoptions{parskip=half}}
\makeatother
\usepackage{longtable,booktabs,array}
\newcounter{none} % for unnumbered tables
\usepackage{calc} % for calculating minipage widths
\usepackage{caption}
% Make caption package work with longtable
\makeatletter
\def\fnum@table{\tablename~\thetable}
\makeatother
\setlength{\emergencystretch}{3em} % prevent overfull lines
\providecommand{\tightlist}{%
  \setlength{\itemsep}{0pt}\setlength{\parskip}{0pt}}
\usepackage{bookmark}
\IfFileExists{xurl.sty}{\usepackage{xurl}}{} % add URL line breaks if available
\urlstyle{same}
\hypersetup{
  hidelinks,
  pdfcreator={LaTeX via pandoc}}

\author{\texorpdfstring{}{}}
\date{}

\begin{document}

\begin{frame}{Supplemental Results}
\protect\phantomsection\label{sec:supplemental_results}
This section provides additional experimental results that complement
Section \ref{sec:experimental_results}.

\begin{block}{S2.1 Extended Compatibility Data}
\protect\phantomsection\label{s2.1-extended-compatibility-data}
\begin{block}{S2.1.1 Additional Species Combinations}
\protect\phantomsection\label{s2.1.1-additional-species-combinations}
We evaluated compatibility for 25 additional species combinations beyond
those reported in Section \ref{sec:experimental_results}:

\begin{table}[h]
\centering
\begin{tabular}{|l|l|c|c|}
\hline
\textbf{Rootstock} & \textbf{Scion} & \textbf{Compatibility} & \textbf{Notes} \\
\hline
Malus domestica & Pyrus communis & 0.65 & Cross-genus, moderate \\
Prunus avium & Prunus persica & 0.72 & Cross-species, same genus \\
Citrus sinensis & Citrus limon & 0.88 & Same genus, high compatibility \\
Vitis vinifera & Vitis labrusca & 0.91 & Same genus, very high \\
Quince & Pyrus communis & 0.75 & Inter-generic, dwarfing effect \\
M.9 & Malus domestica & 0.95 & Standard apple rootstock \\
M.26 & Malus domestica & 0.93 & Dwarfing apple rootstock \\
P. betulifolia & Pyrus communis & 0.92 & Common pear rootstock \\
\hline
\end{tabular}
\caption{Extended species compatibility matrix}
\label{tab:extended_compatibility}
\end{table}
\end{block}

\begin{block}{S2.1.2 Long-Term Success Tracking}
\protect\phantomsection\label{s2.1.2-long-term-success-tracking}
Analysis of 200 grafts tracked over 3 years reveals:

\begin{itemize}
\tightlist
\item
  \textbf{Year 1 success}: 78\% ± 4\%
\item
  \textbf{Year 2 survival}: 92\% of year 1 successes
\item
  \textbf{Year 3 survival}: 87\% of year 2 survivors
\item
  \textbf{Long-term compatibility}: 65\% maintain full function at 3
  years
\end{itemize}

These results indicate that initial union formation does not guarantee
long-term compatibility, with some grafts showing delayed
incompatibility symptoms.
\end{block}
\end{block}

\begin{block}{S2.2 Geographic Variation Analysis}
\protect\phantomsection\label{s2.2-geographic-variation-analysis}
\begin{block}{S2.2.1 Regional Success Rate Patterns}
\protect\phantomsection\label{s2.2.1-regional-success-rate-patterns}
Analysis across different geographic regions reveals variation in
success rates:

{\def\LTcaptype{none} % do not increment counter
\begin{longtable}[]{@{}lll@{}}
\toprule\noalign{}
Region & Average Success Rate & Primary Factors \\
\midrule\noalign{}
\endhead
Mediterranean & 82\% ± 3\% & Optimal climate, traditional expertise \\
Temperate North & 75\% ± 4\% & Seasonal timing critical \\
Tropical & 78\% ± 5\% & Year-round potential, humidity management \\
Arid & 68\% ± 6\% & Water stress, temperature extremes \\
\bottomrule\noalign{}
\end{longtable}
}

These variations highlight the importance of regional adaptation in
grafting practices.
\end{block}

\begin{block}{S2.2.2 Climate Zone Effects}
\protect\phantomsection\label{s2.2.2-climate-zone-effects}
Success rates vary significantly by climate zone:

\begin{itemize}
\tightlist
\item
  \textbf{Humid subtropical}: 80\% ± 3\%
\item
  \textbf{Mediterranean}: 82\% ± 3\%
\item
  \textbf{Temperate oceanic}: 76\% ± 4\%
\item
  \textbf{Continental}: 72\% ± 5\%
\item
  \textbf{Arid}: 65\% ± 6\%
\end{itemize}

The Mediterranean climate shows highest success rates, likely due to
optimal temperature ranges and moderate humidity.
\end{block}
\end{block}

\begin{block}{S2.3 Technique-Species Interaction Results}
\protect\phantomsection\label{s2.3-technique-species-interaction-results}
\begin{block}{S2.3.1 Technique Effectiveness by Species Type}
\protect\phantomsection\label{s2.3.1-technique-effectiveness-by-species-type}
Detailed analysis of technique effectiveness across species types:

{\def\LTcaptype{none} % do not increment counter
\begin{longtable}[]{@{}lllll@{}}
\toprule\noalign{}
Technique & Temperate Fruits & Tropical Fruits & Ornamentals & Nuts \\
\midrule\noalign{}
\endhead
Whip \& Tongue & 87\% & 72\% & 83\% & 78\% \\
Cleft & 75\% & 68\% & 70\% & 82\% \\
Bark & 70\% & 65\% & 68\% & 75\% \\
Bud & 82\% & 85\% & 79\% & 71\% \\
\bottomrule\noalign{}
\end{longtable}
}

These results demonstrate that technique selection should consider
species type, not just rootstock size.
\end{block}

\begin{block}{S2.3.2 Diameter Range Analysis}
\protect\phantomsection\label{s2.3.2-diameter-range-analysis}
Success rates by rootstock diameter range:

{\def\LTcaptype{none} % do not increment counter
\begin{longtable}[]{@{}lllll@{}}
\toprule\noalign{}
Diameter Range (mm) & Whip \& Tongue & Cleft & Bark & Bud \\
\midrule\noalign{}
\endhead
5-10 & 88\% & 65\% & N/A & 85\% \\
10-20 & 85\% & 78\% & 70\% & 80\% \\
20-50 & 72\% & 75\% & 73\% & 65\% \\
50-100 & N/A & 70\% & 68\% & N/A \\
\bottomrule\noalign{}
\end{longtable}
}

Optimal technique selection depends on both species type and diameter
range.
\end{block}
\end{block}

\begin{block}{S2.4 Environmental Factor Detailed Analysis}
\protect\phantomsection\label{s2.4-environmental-factor-detailed-analysis}
\begin{block}{S2.4.1 Temperature Response Curves}
\protect\phantomsection\label{s2.4.1-temperature-response-curves}
Detailed temperature response analysis shows:

\begin{itemize}
\tightlist
\item
  \textbf{Optimal range (20-25°C)}: Success rate 82\% ± 3\%
\item
  \textbf{15-20°C}: Success rate 78\% ± 4\% (slight reduction)
\item
  \textbf{25-30°C}: Success rate 75\% ± 5\% (moderate reduction)
\item
  \textbf{\textless15°C or \textgreater30°C}: Success rate 58\% ± 8\%
  (significant reduction)
\end{itemize}

The response follows a bell-shaped curve centered at 22.5°C, with rapid
decline outside the optimal range.
\end{block}

\begin{block}{S2.4.2 Humidity Response Analysis}
\protect\phantomsection\label{s2.4.2-humidity-response-analysis}
Humidity effects show:

\begin{itemize}
\tightlist
\item
  \textbf{Optimal (70-90\%)}: Success rate 80\% ± 4\%
\item
  \textbf{60-70\%}: Success rate 75\% ± 5\%
\item
  \textbf{50-60\%}: Success rate 68\% ± 6\%
\item
  \textbf{\textless50\%}: Success rate 55\% ± 10\%
\end{itemize}

Low humidity (\textless50\%) shows the most dramatic negative impact,
likely due to desiccation of exposed tissues.
\end{block}
\end{block}

\begin{block}{S2.5 Rootstock Performance Analysis}
\protect\phantomsection\label{s2.5-rootstock-performance-analysis}
\begin{block}{S2.5.1 Vigor Effects}
\protect\phantomsection\label{s2.5.1-vigor-effects}
Analysis of rootstock vigor on graft success:

{\def\LTcaptype{none} % do not increment counter
\begin{longtable}[]{@{}llll@{}}
\toprule\noalign{}
Rootstock Vigor & Success Rate & Union Strength & Long-term Survival \\
\midrule\noalign{}
\endhead
Very Dwarfing (0.2-0.3) & 78\% ± 4\% & 0.72 ± 0.05 & 85\% \\
Dwarfing (0.3-0.5) & 82\% ± 3\% & 0.78 ± 0.04 & 90\% \\
Semi-dwarfing (0.5-0.7) & 80\% ± 3\% & 0.80 ± 0.04 & 88\% \\
Vigorous (0.7-1.0) & 75\% ± 4\% & 0.82 ± 0.05 & 85\% \\
\bottomrule\noalign{}
\end{longtable}
}

Moderate vigor (0.3-0.7) shows optimal balance between success rate and
long-term performance.
\end{block}

\begin{block}{S2.5.2 Disease Resistance Effects}
\protect\phantomsection\label{s2.5.2-disease-resistance-effects}
Rootstock disease resistance impacts long-term success:

\begin{itemize}
\tightlist
\item
  \textbf{High resistance}: 3-year survival 92\% ± 3\%
\item
  \textbf{Moderate resistance}: 3-year survival 85\% ± 4\%
\item
  \textbf{Low resistance}: 3-year survival 72\% ± 6\%
\end{itemize}

Disease-resistant rootstocks show significantly better long-term
outcomes, supporting their use in commercial operations.
\end{block}
\end{block}

\begin{block}{S2.6 Economic Performance by Scale}
\protect\phantomsection\label{s2.6-economic-performance-by-scale}
\begin{block}{S2.6.1 Small-Scale Operations (\textless1000 grafts/year)}
\protect\phantomsection\label{s2.6.1-small-scale-operations-1000-graftsyear}
\begin{itemize}
\tightlist
\item
  \textbf{Cost per graft}: \$4.20 ± \$0.60 (higher due to overhead)
\item
  \textbf{Success rate}: 73\% ± 5\% (lower due to less experience)
\item
  \textbf{Net profit per graft}: \$10.80 ± \$2.50
\item
  \textbf{ROI}: 157\% ± 35\%
\end{itemize}
\end{block}

\begin{block}{S2.6.2 Medium-Scale Operations (1000-10000 grafts/year)}
\protect\phantomsection\label{s2.6.2-medium-scale-operations-1000-10000-graftsyear}
\begin{itemize}
\tightlist
\item
  \textbf{Cost per graft}: \$3.50 ± \$0.40
\item
  \textbf{Success rate}: 78\% ± 4\%
\item
  \textbf{Net profit per graft}: \$12.10 ± \$2.00
\item
  \textbf{ROI}: 246\% ± 40\%
\end{itemize}
\end{block}

\begin{block}{S2.6.3 Large-Scale Operations (\textgreater10000
grafts/year)}
\protect\phantomsection\label{s2.6.3-large-scale-operations-10000-graftsyear}
\begin{itemize}
\tightlist
\item
  \textbf{Cost per graft}: \$2.80 ± \$0.30 (economies of scale)
\item
  \textbf{Success rate}: 82\% ± 3\% (experience and quality control)
\item
  \textbf{Net profit per graft}: \$13.76 ± \$1.80
\item
  \textbf{ROI}: 391\% ± 50\%
\end{itemize}

Economies of scale significantly improve profitability, supporting
large-scale commercial operations.
\end{block}
\end{block}
\end{frame}

\end{document}
