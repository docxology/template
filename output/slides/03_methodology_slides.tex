% Options for packages loaded elsewhere
\PassOptionsToPackage{unicode}{hyperref}
\PassOptionsToPackage{hyphens}{url}
\documentclass[
  ignorenonframetext,
]{beamer}
\newif\ifbibliography
\usepackage{pgfpages}
\setbeamertemplate{caption}[numbered]
\setbeamertemplate{caption label separator}{: }
\setbeamercolor{caption name}{fg=normal text.fg}
\beamertemplatenavigationsymbolsempty
% remove section numbering
\setbeamertemplate{part page}{
  \centering
  \begin{beamercolorbox}[sep=16pt,center]{part title}
    \usebeamerfont{part title}\insertpart\par
  \end{beamercolorbox}
}
\setbeamertemplate{section page}{
  \centering
  \begin{beamercolorbox}[sep=12pt,center]{section title}
    \usebeamerfont{section title}\insertsection\par
  \end{beamercolorbox}
}
\setbeamertemplate{subsection page}{
  \centering
  \begin{beamercolorbox}[sep=8pt,center]{subsection title}
    \usebeamerfont{subsection title}\insertsubsection\par
  \end{beamercolorbox}
}
% Prevent slide breaks in the middle of a paragraph
\widowpenalties 1 10000
\raggedbottom
\AtBeginPart{
  \frame{\partpage}
}
\AtBeginSection{
  \ifbibliography
  \else
    \frame{\sectionpage}
  \fi
}
\AtBeginSubsection{
  \frame{\subsectionpage}
}
\usepackage{iftex}
\ifPDFTeX
  \usepackage[T1]{fontenc}
  \usepackage[utf8]{inputenc}
  \usepackage{textcomp} % provide euro and other symbols
\else % if luatex or xetex
  \usepackage{unicode-math} % this also loads fontspec
  \defaultfontfeatures{Scale=MatchLowercase}
  \defaultfontfeatures[\rmfamily]{Ligatures=TeX,Scale=1}
\fi
\usepackage{lmodern}
\ifPDFTeX\else
  % xetex/luatex font selection
\fi
% Use upquote if available, for straight quotes in verbatim environments
\IfFileExists{upquote.sty}{\usepackage{upquote}}{}
\IfFileExists{microtype.sty}{% use microtype if available
  \usepackage[]{microtype}
  \UseMicrotypeSet[protrusion]{basicmath} % disable protrusion for tt fonts
}{}
\makeatletter
\@ifundefined{KOMAClassName}{% if non-KOMA class
  \IfFileExists{parskip.sty}{%
    \usepackage{parskip}
  }{% else
    \setlength{\parindent}{0pt}
    \setlength{\parskip}{6pt plus 2pt minus 1pt}}
}{% if KOMA class
  \KOMAoptions{parskip=half}}
\makeatother
\setlength{\emergencystretch}{3em} % prevent overfull lines
\providecommand{\tightlist}{%
  \setlength{\itemsep}{0pt}\setlength{\parskip}{0pt}}
\usepackage{bookmark}
\IfFileExists{xurl.sty}{\usepackage{xurl}}{} % add URL line breaks if available
\urlstyle{same}
\hypersetup{
  hidelinks,
  pdfcreator={LaTeX via pandoc}}

\author{\texorpdfstring{}{}}
\date{}

\begin{document}

\section{Methodology}\label{sec:methodology}

\begin{frame}{Biological Mechanisms}
\protect\phantomsection\label{biological-mechanisms}
\begin{block}{Cambium Alignment and Contact}
\protect\phantomsection\label{cambium-alignment-and-contact}
The success of graft union formation fundamentally depends on precise
alignment of the cambium layers, the thin meristematic tissue
responsible for secondary growth in plants
\cite{melnyk2018, goldschmidt2014}. The cambium, located between the
xylem and phloem, contains actively dividing cells that generate new
vascular tissue. For successful grafting, the cambium layers of
rootstock and scion must be brought into direct contact, enabling
cell-to-cell communication and tissue integration.

The cambium contact area can be quantified as:

\begin{equation}\label{eq:cambium_contact}
C(t) = C_0 + \int_0^t r_c(\tau) \cdot A(\tau) \, d\tau
\end{equation}

where \(C(t)\) is the cambium contact area at time \(t\), \(C_0\) is the
initial contact area (determined by technique quality), \(r_c(\tau)\) is
the cambium growth rate, and \(A(\tau)\) is the available contact area.
\end{block}

\begin{block}{Callus Formation}
\protect\phantomsection\label{callus-formation}
Following cambium contact, callus tissue forms at the graft interface.
Callus consists of undifferentiated parenchyma cells that proliferate to
bridge the gap between rootstock and scion \cite{melnyk2018}. The callus
formation process follows an exponential growth pattern:

\begin{equation}\label{eq:callus_formation}
F(t) = F_{\max} \left(1 - e^{-\lambda_c t}\right)
\end{equation}

where \(F(t)\) is the callus formation fraction (0-1), \(F_{\max}\) is
the maximum possible formation (typically 0.9-1.0), and \(\lambda_c\) is
the formation rate constant, which depends on species compatibility,
temperature, and humidity.
\end{block}

\begin{block}{Vascular Connection}
\protect\phantomsection\label{vascular-connection}
The final stage of graft union involves differentiation of callus cells
into functional vascular tissue (xylem and phloem), establishing
nutrient and water transport between rootstock and scion
\cite{melnyk2018}. Vascular connection strength can be modeled as:

\begin{equation}\label{eq:vascular_connection}
V(t) = V_{\max} \cdot \min\left(1, \frac{F(t) - F_{threshold}}{F_{max} - F_{threshold}}\right)
\end{equation}

where \(V(t)\) is the vascular connection strength, \(F_{threshold}\) is
the minimum callus formation required for vascular differentiation
(typically 0.5), and \(V_{\max}\) is the maximum connection strength.
\end{block}
\end{frame}

\begin{frame}{Grafting Techniques}
\protect\phantomsection\label{grafting-techniques}
\begin{block}{Whip and Tongue Grafting}
\protect\phantomsection\label{whip-and-tongue-grafting}
Whip and tongue grafting (also called splice grafting) is among the most
precise methods, suitable for rootstock and scion of similar diameter
(5-25 mm) \cite{garner2013, hartmann2014}. The technique involves:

\begin{enumerate}
\tightlist
\item
  Making matching 30-45° angle cuts on both rootstock and scion
\item
  Creating interlocking tongues (notches) on both pieces
\item
  Aligning cambium layers precisely
\item
  Securing with grafting tape or wax
\item
  Protecting from desiccation
\end{enumerate}

Success rates typically range from 75-90\%, depending on species
compatibility and execution quality.
\end{block}

\begin{block}{Cleft Grafting}
\protect\phantomsection\label{cleft-grafting}
Cleft grafting is suitable for larger diameter rootstock (10-50 mm) and
is particularly useful for top-working established trees
\cite{webster2002}. The procedure involves:

\begin{enumerate}
\tightlist
\item
  Making a vertical split in the rootstock
\item
  Preparing wedge-shaped scion with 2-3 buds
\item
  Inserting scion into cleft, ensuring cambium alignment
\item
  Sealing with grafting wax
\item
  Protecting from weather
\end{enumerate}

Success rates are typically 70-80\%, with higher success for larger
diameter matches.
\end{block}

\begin{block}{Bark Grafting}
\protect\phantomsection\label{bark-grafting}
Bark grafting is employed for large diameter rootstock (20-100 mm) and
is useful for mature tree renovation \cite{garner2013}. The method
involves:

\begin{enumerate}
\tightlist
\item
  Making vertical cut through bark on rootstock
\item
  Loosening bark flaps
\item
  Preparing scion with beveled cut
\item
  Inserting scion under bark, aligning cambium
\item
  Securing and sealing
\end{enumerate}

Success rates range from 65-75\%, with optimal timing in early spring
when bark is slipping.
\end{block}

\begin{block}{Bud Grafting (T-budding)}
\protect\phantomsection\label{bud-grafting-t-budding}
Bud grafting (T-budding) is highly efficient for mass propagation, using
a single bud rather than a complete scion \cite{hartmann2014}. The
technique involves:

\begin{enumerate}
\tightlist
\item
  Making T-shaped cut in rootstock bark
\item
  Removing bud from scion with shield
\item
  Inserting bud under bark flaps
\item
  Wrapping securely with budding tape
\item
  Removing tape after bud takes (typically 2-4 weeks)
\end{enumerate}

Success rates are typically 75-85\%, making this method highly efficient
for commercial propagation.
\end{block}
\end{frame}

\begin{frame}{Compatibility Theory}
\protect\phantomsection\label{compatibility-theory}
\begin{block}{Phylogenetic Distance Model}
\protect\phantomsection\label{phylogenetic-distance-model}
Phylogenetic distance is the strongest predictor of graft compatibility
\cite{stebbins1950, goldschmidt2014}. Closely related species share
similar vascular anatomy, biochemical pathways, and growth patterns,
enabling successful union formation. Compatibility decreases
exponentially with phylogenetic distance:

\begin{equation}\label{eq:phylogenetic_compatibility}
P_{phyl}(d) = e^{-k \cdot d / d_{max}}
\end{equation}

where \(P_{phyl}(d)\) is the phylogenetic compatibility (0-1), \(d\) is
the phylogenetic distance, \(d_{max}\) is the maximum distance for
compatibility, and \(k\) is a decay constant (typically
\(k \approx 2.0\)).
\end{block}

\begin{block}{Cambium Match Model}
\protect\phantomsection\label{cambium-match-model}
Similar cambium thickness indicates better alignment potential and
reduced stress at the union interface:

\begin{equation}\label{eq:cambium_match}
P_{camb}(r_s, r_r) = 1 - \min\left(1, \frac{|r_s - r_r|}{\tau \cdot r_r}\right)
\end{equation}

where \(P_{camb}\) is the cambium match score, \(r_s\) and \(r_r\) are
scion and rootstock cambium thicknesses, and \(\tau\) is the tolerance
threshold (typically 0.2).
\end{block}

\begin{block}{Growth Rate Compatibility}
\protect\phantomsection\label{growth-rate-compatibility}
Similar growth rates reduce stress at the graft union, preventing
overgrowth or undergrowth issues:

\begin{equation}\label{eq:growth_compatibility}
P_{growth}(g_s, g_r) = 1 - \min\left(1, \frac{|g_s - g_r|}{\tau_g \cdot g_r}\right)
\end{equation}

where \(P_{growth}\) is the growth rate compatibility, \(g_s\) and
\(g_r\) are scion and rootstock growth rates, and \(\tau_g\) is the
growth rate tolerance (typically 0.3).
\end{block}

\begin{block}{Combined Compatibility Score}
\protect\phantomsection\label{combined-compatibility-score}
The overall compatibility prediction combines multiple factors:

\begin{equation}\label{eq:combined_compatibility}
P_{total} = w_1 P_{phyl} + w_2 P_{camb} + w_3 P_{growth}
\end{equation}

where \(w_1 = 0.5\), \(w_2 = 0.3\), and \(w_3 = 0.2\) are weights
determined through empirical analysis.
\end{block}
\end{frame}

\begin{frame}{Success Factors}
\protect\phantomsection\label{success-factors}
\begin{block}{Environmental Conditions}
\protect\phantomsection\label{environmental-conditions}
Optimal environmental conditions are critical for graft success:

\begin{itemize}
\tightlist
\item
  \textbf{Temperature}: 20-25°C optimal, 15-30°C acceptable range
\item
  \textbf{Humidity}: 70-90\% relative humidity optimal
\item
  \textbf{Light}: Moderate indirect light, avoid direct sun exposure
\item
  \textbf{Season}: Late winter to early spring for temperate species
\end{itemize}

The environmental suitability score can be calculated as:

\begin{equation}\label{eq:environmental_score}
E(T, H) = E_T(T) \cdot E_H(H)
\end{equation}

where \(E_T(T)\) and \(E_H(H)\) are temperature and humidity suitability
functions, respectively.
\end{block}

\begin{block}{Technique Quality}
\protect\phantomsection\label{technique-quality}
The quality of technique execution significantly impacts success rates.
Key factors include:

\begin{itemize}
\tightlist
\item
  Precision of cuts and alignment
\item
  Speed of operation (minimizing desiccation)
\item
  Proper sealing and protection
\item
  Post-grafting care
\end{itemize}

Technique quality can be quantified on a 0-1 scale, with values above
0.8 associated with success rates 15-20\% higher than values below 0.6.
\end{block}
\end{frame}

\begin{frame}{Computational Framework}
\protect\phantomsection\label{computational-framework}
\begin{block}{Biological Process Simulation}
\protect\phantomsection\label{biological-process-simulation}
Our simulation framework models the temporal dynamics of graft healing
using a system of differential equations:

\begin{equation}\label{eq:healing_dynamics}
\frac{dC}{dt} = r_c \cdot (1 - C) \cdot E(T, H) \cdot P_{total}
\end{equation}

\begin{equation}\label{eq:callus_dynamics}
\frac{dF}{dt} = r_f \cdot C \cdot (1 - F) \cdot E(T, H) \cdot P_{total}
\end{equation}

\begin{equation}\label{eq:vascular_dynamics}
\frac{dV}{dt} = r_v \cdot F \cdot (1 - V) \cdot E(T, H) \cdot P_{total}
\end{equation}

where \(r_c\), \(r_f\), and \(r_v\) are growth rate constants for
cambium contact, callus formation, and vascular connection,
respectively.
\end{block}

\begin{block}{Success Probability Prediction}
\protect\phantomsection\label{success-probability-prediction}
The overall graft success probability combines compatibility, technique
quality, environmental conditions, and seasonal timing:

\begin{equation}\label{eq:success_probability}
P_{success} = 0.4 P_{total} + 0.3 Q_{tech} + 0.2 E(T, H) + 0.1 S_{timing}
\end{equation}

where \(Q_{tech}\) is technique quality (0-1) and \(S_{timing}\) is
seasonal timing score (0-1).
\end{block}
\end{frame}

\begin{frame}[fragile]{Implementation Details}
\protect\phantomsection\label{implementation-details}
The computational toolkit implements these models through modular Python
packages:

\begin{itemize}
\tightlist
\item
  \textbf{\texttt{graft\_basics.py}}: Core grafting calculations and
  compatibility checks
\item
  \textbf{\texttt{biological\_simulation.py}}: Simulation framework for
  healing processes
\item
  \textbf{\texttt{compatibility\_prediction.py}}: Compatibility
  prediction algorithms
\item
  \textbf{\texttt{species\_database.py}}: Database of species
  compatibility information
\item
  \textbf{\texttt{technique\_library.py}}: Encyclopedia of grafting
  techniques
\item
  \textbf{\texttt{graft\_statistics.py}}: Statistical analysis of
  grafting outcomes
\item
  \textbf{\texttt{graft\_analysis.py}}: Factor analysis and outcome
  evaluation
\end{itemize}

All implementations follow the thin orchestrator pattern, with business
logic in \texttt{src/} modules and orchestration in \texttt{scripts/}
files, ensuring maintainability and testability.
\end{frame}

\begin{frame}{Validation Framework}
\protect\phantomsection\label{validation-framework}
To validate our models and predictions, we use:

\begin{enumerate}
\tightlist
\item
  \textbf{Literature Review}: Comparison with published success rates
  and compatibility data
\item
  \textbf{Synthetic Data Generation}: Realistic trial data based on
  known biological parameters
\item
  \textbf{Statistical Validation}: Hypothesis testing and correlation
  analysis
\item
  \textbf{Cross-Validation}: Model performance on held-out data
\end{enumerate}

The validation framework ensures that predictions align with established
horticultural knowledge and biological principles.
\end{frame}

\end{document}
