% Options for packages loaded elsewhere
\PassOptionsToPackage{unicode}{hyperref}
\PassOptionsToPackage{hyphens}{url}
\documentclass[
  ignorenonframetext,
]{beamer}
\newif\ifbibliography
\usepackage{pgfpages}
\setbeamertemplate{caption}[numbered]
\setbeamertemplate{caption label separator}{: }
\setbeamercolor{caption name}{fg=normal text.fg}
\beamertemplatenavigationsymbolsempty
% remove section numbering
\setbeamertemplate{part page}{
  \centering
  \begin{beamercolorbox}[sep=16pt,center]{part title}
    \usebeamerfont{part title}\insertpart\par
  \end{beamercolorbox}
}
\setbeamertemplate{section page}{
  \centering
  \begin{beamercolorbox}[sep=12pt,center]{section title}
    \usebeamerfont{section title}\insertsection\par
  \end{beamercolorbox}
}
\setbeamertemplate{subsection page}{
  \centering
  \begin{beamercolorbox}[sep=8pt,center]{subsection title}
    \usebeamerfont{subsection title}\insertsubsection\par
  \end{beamercolorbox}
}
% Prevent slide breaks in the middle of a paragraph
\widowpenalties 1 10000
\raggedbottom
\AtBeginPart{
  \frame{\partpage}
}
\AtBeginSection{
  \ifbibliography
  \else
    \frame{\sectionpage}
  \fi
}
\AtBeginSubsection{
  \frame{\subsectionpage}
}
\usepackage{iftex}
\ifPDFTeX
  \usepackage[T1]{fontenc}
  \usepackage[utf8]{inputenc}
  \usepackage{textcomp} % provide euro and other symbols
\else % if luatex or xetex
  \usepackage{unicode-math} % this also loads fontspec
  \defaultfontfeatures{Scale=MatchLowercase}
  \defaultfontfeatures[\rmfamily]{Ligatures=TeX,Scale=1}
\fi
\usepackage{lmodern}
\ifPDFTeX\else
  % xetex/luatex font selection
\fi
% Use upquote if available, for straight quotes in verbatim environments
\IfFileExists{upquote.sty}{\usepackage{upquote}}{}
\IfFileExists{microtype.sty}{% use microtype if available
  \usepackage[]{microtype}
  \UseMicrotypeSet[protrusion]{basicmath} % disable protrusion for tt fonts
}{}
\makeatletter
\@ifundefined{KOMAClassName}{% if non-KOMA class
  \IfFileExists{parskip.sty}{%
    \usepackage{parskip}
  }{% else
    \setlength{\parindent}{0pt}
    \setlength{\parskip}{6pt plus 2pt minus 1pt}}
}{% if KOMA class
  \KOMAoptions{parskip=half}}
\makeatother
\usepackage{color}
\usepackage{fancyvrb}
\newcommand{\VerbBar}{|}
\newcommand{\VERB}{\Verb[commandchars=\\\{\}]}
\DefineVerbatimEnvironment{Highlighting}{Verbatim}{commandchars=\\\{\}}
% Add ',fontsize=\small' for more characters per line
\newenvironment{Shaded}{}{}
\newcommand{\AlertTok}[1]{\textcolor[rgb]{1.00,0.00,0.00}{\textbf{#1}}}
\newcommand{\AnnotationTok}[1]{\textcolor[rgb]{0.38,0.63,0.69}{\textbf{\textit{#1}}}}
\newcommand{\AttributeTok}[1]{\textcolor[rgb]{0.49,0.56,0.16}{#1}}
\newcommand{\BaseNTok}[1]{\textcolor[rgb]{0.25,0.63,0.44}{#1}}
\newcommand{\BuiltInTok}[1]{\textcolor[rgb]{0.00,0.50,0.00}{#1}}
\newcommand{\CharTok}[1]{\textcolor[rgb]{0.25,0.44,0.63}{#1}}
\newcommand{\CommentTok}[1]{\textcolor[rgb]{0.38,0.63,0.69}{\textit{#1}}}
\newcommand{\CommentVarTok}[1]{\textcolor[rgb]{0.38,0.63,0.69}{\textbf{\textit{#1}}}}
\newcommand{\ConstantTok}[1]{\textcolor[rgb]{0.53,0.00,0.00}{#1}}
\newcommand{\ControlFlowTok}[1]{\textcolor[rgb]{0.00,0.44,0.13}{\textbf{#1}}}
\newcommand{\DataTypeTok}[1]{\textcolor[rgb]{0.56,0.13,0.00}{#1}}
\newcommand{\DecValTok}[1]{\textcolor[rgb]{0.25,0.63,0.44}{#1}}
\newcommand{\DocumentationTok}[1]{\textcolor[rgb]{0.73,0.13,0.13}{\textit{#1}}}
\newcommand{\ErrorTok}[1]{\textcolor[rgb]{1.00,0.00,0.00}{\textbf{#1}}}
\newcommand{\ExtensionTok}[1]{#1}
\newcommand{\FloatTok}[1]{\textcolor[rgb]{0.25,0.63,0.44}{#1}}
\newcommand{\FunctionTok}[1]{\textcolor[rgb]{0.02,0.16,0.49}{#1}}
\newcommand{\ImportTok}[1]{\textcolor[rgb]{0.00,0.50,0.00}{\textbf{#1}}}
\newcommand{\InformationTok}[1]{\textcolor[rgb]{0.38,0.63,0.69}{\textbf{\textit{#1}}}}
\newcommand{\KeywordTok}[1]{\textcolor[rgb]{0.00,0.44,0.13}{\textbf{#1}}}
\newcommand{\NormalTok}[1]{#1}
\newcommand{\OperatorTok}[1]{\textcolor[rgb]{0.40,0.40,0.40}{#1}}
\newcommand{\OtherTok}[1]{\textcolor[rgb]{0.00,0.44,0.13}{#1}}
\newcommand{\PreprocessorTok}[1]{\textcolor[rgb]{0.74,0.48,0.00}{#1}}
\newcommand{\RegionMarkerTok}[1]{#1}
\newcommand{\SpecialCharTok}[1]{\textcolor[rgb]{0.25,0.44,0.63}{#1}}
\newcommand{\SpecialStringTok}[1]{\textcolor[rgb]{0.73,0.40,0.53}{#1}}
\newcommand{\StringTok}[1]{\textcolor[rgb]{0.25,0.44,0.63}{#1}}
\newcommand{\VariableTok}[1]{\textcolor[rgb]{0.10,0.09,0.49}{#1}}
\newcommand{\VerbatimStringTok}[1]{\textcolor[rgb]{0.25,0.44,0.63}{#1}}
\newcommand{\WarningTok}[1]{\textcolor[rgb]{0.38,0.63,0.69}{\textbf{\textit{#1}}}}
\setlength{\emergencystretch}{3em} % prevent overfull lines
\providecommand{\tightlist}{%
  \setlength{\itemsep}{0pt}\setlength{\parskip}{0pt}}
\usepackage{bookmark}
\IfFileExists{xurl.sty}{\usepackage{xurl}}{} % add URL line breaks if available
\urlstyle{same}
\hypersetup{
  hidelinks,
  pdfcreator={LaTeX via pandoc}}

\author{\texorpdfstring{}{}}
\date{}

\begin{document}

\section{Methodology}\label{sec:methodology}

\begin{frame}[fragile]{Database Structure and Conversion}
\protect\phantomsection\label{database-structure-and-conversion}
\begin{block}{Source Data}
\protect\phantomsection\label{source-data}
The research draws on a MySQL database dump containing 11 tables
documenting Andrius Kulikauskas's Ways of Figuring Things Out framework.
The primary data table \texttt{ways} contains 210 documented ways with
the following key fields (see Table \ref{tab:ways_schema}):

\begin{itemize}
\tightlist
\item
  \texttt{way}: The name/identifier of the way
\item
  \texttt{dialoguewith}: The dialogue partner or conversant
\item
  \texttt{dialoguetype}: The type of dialogue (Absolute, Relative,
  Embrace God)
\item
  \texttt{dialoguetypetype}: Sub-type classification
\item
  \texttt{mene}: Room assignment in the House of Knowledge (24 rooms)
\item
  \texttt{Dievas}: Relationship to God/the divine
\item
  \texttt{examples}: Examples and descriptions
\item
  \texttt{comments}: Additional comments and notes
\end{itemize}

\begin{table}[h]
\centering
\begin{tabular}{|l|l|}
\hline
\textbf{Field} & \textbf{Description} \\
\hline
\texttt{way} & The name/identifier of the way \\
\texttt{dialoguewith} & The dialogue partner or conversant \\
\texttt{dialoguetype} & The type of dialogue (Absolute, Relative, Embrace God) \\
\texttt{dialoguetypetype} & Sub-type classification \\
\texttt{mene} & Room assignment in the House of Knowledge (24 rooms) \\
\texttt{Dievas} & Relationship to God/the divine \\
\texttt{examples} & Examples and descriptions \\
\texttt{comments} & Additional comments and notes \\
\hline
\end{tabular}
\caption{Key fields in the \texttt{ways} table schema}
\label{tab:ways_schema}
\end{table}
\end{block}

\begin{block}{SQLite Conversion}
\protect\phantomsection\label{sqlite-conversion}
For analysis and portability, the MySQL dump was converted to SQLite
format. The conversion process:

\begin{enumerate}
\tightlist
\item
  \textbf{Schema Conversion}: MySQL-specific syntax (AUTO\_INCREMENT,
  ENGINE, COLLATE) converted to SQLite-compatible syntax
\item
  \textbf{Table Renaming}: Tables renamed for clarity
  (\texttt{20100422ways} → \texttt{ways}, \texttt{menes} →
  \texttt{rooms}, etc.)
\item
  \textbf{Index Handling}: Index names adjusted to avoid conflicts with
  table names (SQLite restriction)
\item
  \textbf{Data Preservation}: All data preserved during conversion with
  proper encoding handling
\end{enumerate}

The resulting SQLite database (\texttt{db/ways.db}) provides a portable,
queryable format for analysis. The complete database schema is
documented in Section \ref{sec:appendix} (Appendix A).
\end{block}

\begin{block}{Implementation Modules}
\protect\phantomsection\label{implementation-modules}
The analysis is implemented using several specialized modules in
\texttt{project/src/}:

\begin{itemize}
\tightlist
\item
  \textbf{\texttt{database.py}}: SQLAlchemy ORM models for Ways, Rooms,
  Questions, and database access
\item
  \textbf{\texttt{sql\_queries.py}}: Pre-built SQL queries for common
  analysis operations
\item
  \textbf{\texttt{ways\_analysis.py}}: High-level ways characterization
  and analysis functions
\item
  \textbf{\texttt{network\_analysis.py}}: Graph-based network analysis
  of way relationships
\item
  \textbf{\texttt{house\_of\_knowledge.py}}: Analysis of the 24-room
  House of Knowledge framework
\item
  \textbf{\texttt{statistics.py}}: Statistical analysis functions
  including \texttt{analyze\_way\_distributions()},
  \texttt{compute\_way\_correlations()},
  \texttt{compute\_way\_diversity\_metrics()}
\item
  \textbf{\texttt{metrics.py}}: Performance metrics including
  \texttt{compute\_way\_coverage\_metrics()},
  \texttt{compute\_way\_interconnectedness()}
\item
  \textbf{\texttt{models.py}}: Data classes and enums for type-safe data
  handling
\end{itemize}
\end{block}
\end{frame}

\begin{frame}[fragile]{House of Knowledge Framework}
\protect\phantomsection\label{house-of-knowledge-framework}
\begin{block}{24-Room Structure}
\protect\phantomsection\label{room-structure}
The Ways framework organizes knowledge into 24 rooms within the ``House
of Knowledge.'' Each room represents a different aspect of how we come
to know and understand:

\begin{equation}\label{eq:house_structure}
\text{House of Knowledge} = \{\text{Room}_1, \text{Room}_2, \ldots, \text{Room}_{24}\}
\end{equation}

The rooms are organized according to three fundamental structures:

\begin{enumerate}
\tightlist
\item
  \textbf{Believing (1-2-3-4)}: Four levels of belief structure
\item
  \textbf{Caring (1-2-3-4)}: Four levels of care structure\\
\item
  \textbf{Relative Learning}: The cycle of taking a stand, following
  through, and reflecting
\end{enumerate}
\end{block}

\begin{block}{Room Categories}
\protect\phantomsection\label{room-categories}
Each way is assigned to one or more rooms via the \texttt{mene} field,
creating a mapping:

\begin{equation}\label{eq:way_room_mapping}
\text{Way}_i \mapsto \{\text{Room}_j : \text{Way}_i \text{ belongs to Room}_j\}
\end{equation}

This mapping enables analysis of how ways cluster within rooms and how
rooms relate to one another.
\end{block}
\end{frame}

\begin{frame}{Dialogue Type Classification}
\protect\phantomsection\label{dialogue-type-classification}
\begin{block}{Three Main Types}
\protect\phantomsection\label{three-main-types}
Ways are classified according to three primary dialogue types:

\begin{enumerate}
\tightlist
\item
  \textbf{Absolute}: Ways that reference absolute truth or structure
\item
  \textbf{Relative}: Ways that engage with relative perspectives
\item
  \textbf{Embrace God}: Ways that explicitly engage with the divine or
  transcendent
\end{enumerate}

The distribution of ways across dialogue types provides insight into the
balance of different epistemological approaches in the framework.
\end{block}

\begin{block}{Dialogue Type Analysis}
\protect\phantomsection\label{dialogue-type-analysis}
For each way \(w_i\), we extract:

\begin{equation}\label{eq:dialogue_type}
\text{Type}(w_i) \in \{\text{Absolute}, \text{Relative}, \text{Embrace God}\}
\end{equation}

This classification enables statistical analysis of type distributions
and relationships.
\end{block}
\end{frame}

\begin{frame}[fragile]{Network Analysis Methodology}
\protect\phantomsection\label{network-analysis-methodology}
\begin{block}{Graph Construction}
\protect\phantomsection\label{graph-construction}
We construct a weighted network graph \(G = (V, E, w)\) where:

\begin{itemize}
\tightlist
\item
  \textbf{Vertices \(V\)}: Each way \(w_i\) is a node \(v_i \in V\),
  with \(|V| = 210\)
\item
  \textbf{Edges \(E\)}: Connections between ways based on:

  \begin{itemize}
  \tightlist
  \item
    Shared dialogue partners (\texttt{dialoguewith}): \(e_{ij} \in E\)
    if \(\text{dialoguewith}(w_i) = \text{dialoguewith}(w_j)\)
  \item
    Shared room assignments (\texttt{mene}): \(e_{ij} \in E\) if
    \(\text{mene}(w_i) = \text{mene}(w_j)\)
  \item
    Similar dialogue types: \(e_{ij} \in E\) if
    \(\text{dialoguetype}(w_i) = \text{dialoguetype}(w_j)\)
  \item
    Question relationships (\texttt{klausimobudai} table):
    \(e_{ij} \in E\) if
    \(\exists q: (w_i, q) \in Q \land (w_j, q) \in Q\)
  \end{itemize}
\item
  \textbf{Edge weights \(w\)}: \(w(e_{ij}) \in \{0.6, 0.8, 1.0\}\) based
  on relationship type (type, partner, room respectively)
\end{itemize}

The resulting network contains \(|E| = 1,290\) edges connecting the 210
ways.
\end{block}

\begin{block}{Centrality Metrics}
\protect\phantomsection\label{centrality-metrics}
We compute several centrality metrics to identify important ways:

\textbf{Degree Centrality:} \begin{equation}\label{eq:degree_centrality}
C_D(v) = \frac{\deg(v)}{|V| - 1}
\end{equation}

\textbf{Betweenness Centrality:}
\begin{equation}\label{eq:betweenness_centrality}
C_B(v) = \sum_{s \neq v \neq t} \frac{\sigma_{st}(v)}{\sigma_{st}}
\end{equation}

where \(\sigma_{st}\) is the number of shortest paths from \(s\) to
\(t\), and \(\sigma_{st}(v)\) is the number of those paths passing
through \(v\).

\textbf{Clustering Coefficient:} \begin{equation}\label{eq:clustering}
C_C(v) = \frac{2e_v}{k_v(k_v - 1)}
\end{equation}

where \(e_v\) is the number of edges between neighbors of \(v\), and
\(k_v\) is the degree of \(v\).
\end{block}
\end{frame}

\begin{frame}[fragile]{Statistical Analysis Methods}
\protect\phantomsection\label{statistical-analysis-methods}
\begin{block}{Distribution Analysis}
\protect\phantomsection\label{distribution-analysis}
We analyze the distribution of ways across:

\begin{enumerate}
\tightlist
\item
  \textbf{Dialogue Types}: Count and percentage by type, with 38
  distinct types observed
\item
  \textbf{Rooms}: Distribution across 24 rooms, with B2 containing the
  most ways (23)
\item
  \textbf{Dialogue Partners}: Frequency of conversants, with 196 unique
  partners
\item
  \textbf{God Relationships}: Distribution of \texttt{Dievas} values
\end{enumerate}
\end{block}

\begin{block}{Information-Theoretic Metrics}
\protect\phantomsection\label{information-theoretic-metrics}
We compute Shannon entropy to quantify the diversity of distributions:

\begin{equation}\label{eq:entropy}
H(X) = -\sum_{i=1}^{k} p_i \log_2(p_i)
\end{equation}

where \(p_i\) is the proportion in category \(i\) and \(k\) is the
number of categories.

\textbf{Mutual Information} between dialogue types and rooms:

\begin{equation}\label{eq:mutual_info}
I(X;Y) = \sum_{x,y} p(x,y) \log_2 \frac{p(x,y)}{p(x)p(y)}
\end{equation}

This quantifies the strength of association between dialogue types and
room assignments.
\end{block}

\begin{block}{Cross-Tabulation}
\protect\phantomsection\label{cross-tabulation}
Cross-tabulation analysis examines relationships between:

\begin{itemize}
\tightlist
\item
  Dialogue type × Room assignment (visualized in Figure
  \ref{fig:type_room_heatmap})
\item
  Dialogue type × Dialogue partner
\item
  Room × God relationship
\end{itemize}

This reveals patterns in how different dimensions of the framework
relate, with the cross-tabulation matrix showing concentrations of ways
at specific type-room intersections.
\end{block}
\end{frame}

\begin{frame}[fragile]{Text Analysis}
\protect\phantomsection\label{text-analysis}
\begin{block}{Way Descriptions}
\protect\phantomsection\label{way-descriptions}
For ways with text descriptions in \texttt{ways.md}, we perform:

\begin{enumerate}
\tightlist
\item
  \textbf{Keyword Extraction}: Identify key terms and concepts
\item
  \textbf{Theme Analysis}: Extract recurring themes
\item
  \textbf{Example Analysis}: Analyze examples to understand way
  applications
\item
  \textbf{Relationship Extraction}: Identify references to other ways or
  concepts
\end{enumerate}
\end{block}

\begin{block}{Philosophical Structure Analysis}
\protect\phantomsection\label{philosophical-structure-analysis}
Text analysis also examines:

\begin{itemize}
\tightlist
\item
  How ways relate to the believing/caring/learning structures
\item
  References to the House of Knowledge framework
\item
  Connections to broader philosophical concepts
\end{itemize}
\end{block}
\end{frame}

\begin{frame}[fragile]{Data Processing Pipeline}
\protect\phantomsection\label{data-processing-pipeline}
\begin{block}{Extraction}
\protect\phantomsection\label{extraction}
\begin{enumerate}
\tightlist
\item
  \textbf{Database Query}: Extract ways data from SQLite database
\item
  \textbf{Text Parsing}: Parse \texttt{ways.md} for additional context
\item
  \textbf{Relationship Extraction}: Build network from relationship
  tables
\end{enumerate}
\end{block}

\begin{block}{Transformation}
\protect\phantomsection\label{transformation}
\begin{enumerate}
\tightlist
\item
  \textbf{Normalization}: Standardize way names and categories
\item
  \textbf{Encoding}: Handle Lithuanian/English text encoding
\item
  \textbf{Cleaning}: Remove duplicates and handle missing data
\end{enumerate}
\end{block}

\begin{block}{Analysis}
\protect\phantomsection\label{analysis}
\begin{enumerate}
\tightlist
\item
  \textbf{Statistical Computation}: Calculate distributions and metrics
\item
  \textbf{Network Construction}: Build graph structures
\item
  \textbf{Visualization Generation}: Create plots and network diagrams
\end{enumerate}
\end{block}
\end{frame}

\begin{frame}{Validation Framework}
\protect\phantomsection\label{validation-framework}
\begin{block}{Data Quality Checks}
\protect\phantomsection\label{data-quality-checks}
\begin{enumerate}
\tightlist
\item
  \textbf{Completeness}: Verify all ways have required fields
\item
  \textbf{Consistency}: Check for conflicting assignments
\item
  \textbf{Referential Integrity}: Validate room and relationship
  references
\end{enumerate}
\end{block}

\begin{block}{Analysis Validation}
\protect\phantomsection\label{analysis-validation}
\begin{enumerate}
\tightlist
\item
  \textbf{Reproducibility}: Ensure analyses are reproducible
\item
  \textbf{Sensitivity}: Test sensitivity to data variations
\item
  \textbf{Robustness}: Verify results are robust to missing data
\end{enumerate}
\end{block}
\end{frame}

\begin{frame}[fragile]{SQL Query Examples}
\protect\phantomsection\label{sql-query-examples}
Key analyses are performed using SQL queries against the SQLite
database. Example queries include:

\textbf{Dialogue Type Distribution:}

\begin{Shaded}
\begin{Highlighting}[]
\KeywordTok{SELECT}\NormalTok{ dialoguetype, }\FunctionTok{COUNT}\NormalTok{(}\OperatorTok{*}\NormalTok{) }\KeywordTok{as} \FunctionTok{count}
\KeywordTok{FROM}\NormalTok{ ways}
\KeywordTok{GROUP} \KeywordTok{BY}\NormalTok{ dialoguetype}
\KeywordTok{ORDER} \KeywordTok{BY} \FunctionTok{count} \KeywordTok{DESC}\NormalTok{;}
\end{Highlighting}
\end{Shaded}

\textbf{Room-Way Cross-Tabulation:}

\begin{Shaded}
\begin{Highlighting}[]
\KeywordTok{SELECT}\NormalTok{ dialoguetype, mene, }\FunctionTok{COUNT}\NormalTok{(}\OperatorTok{*}\NormalTok{) }\KeywordTok{as} \FunctionTok{count}
\KeywordTok{FROM}\NormalTok{ ways}
\KeywordTok{WHERE}\NormalTok{ mene }\OperatorTok{!=} \StringTok{\textquotesingle{}\textquotesingle{}} \KeywordTok{AND}\NormalTok{ dialoguetype }\OperatorTok{!=} \StringTok{\textquotesingle{}\textquotesingle{}}
\KeywordTok{GROUP} \KeywordTok{BY}\NormalTok{ dialoguetype, mene}
\KeywordTok{ORDER} \KeywordTok{BY} \FunctionTok{count} \KeywordTok{DESC}\NormalTok{;}
\end{Highlighting}
\end{Shaded}

\textbf{Network Edge Construction (Room-based):}

\begin{Shaded}
\begin{Highlighting}[]
\KeywordTok{SELECT}\NormalTok{ w1.}\KeywordTok{ID} \KeywordTok{as}\NormalTok{ way1\_id, w2.}\KeywordTok{ID} \KeywordTok{as}\NormalTok{ way2\_id}
\KeywordTok{FROM}\NormalTok{ ways w1}
\KeywordTok{JOIN}\NormalTok{ ways w2 }\KeywordTok{ON}\NormalTok{ w1.mene }\OperatorTok{=}\NormalTok{ w2.mene}
\KeywordTok{WHERE}\NormalTok{ w1.}\KeywordTok{ID} \OperatorTok{\textless{}}\NormalTok{ w2.}\KeywordTok{ID} \KeywordTok{AND}\NormalTok{ w1.mene }\OperatorTok{!=} \StringTok{\textquotesingle{}\textquotesingle{}}\NormalTok{;}
\end{Highlighting}
\end{Shaded}

\textbf{Central Ways Identification:}

\begin{Shaded}
\begin{Highlighting}[]
\KeywordTok{SELECT}\NormalTok{ way, }\FunctionTok{COUNT}\NormalTok{(}\OperatorTok{*}\NormalTok{) }\KeywordTok{as}\NormalTok{ connection\_count}
\KeywordTok{FROM}\NormalTok{ (}
    \KeywordTok{SELECT}\NormalTok{ w1.way, w2.}\KeywordTok{ID}
    \KeywordTok{FROM}\NormalTok{ ways w1}
    \KeywordTok{JOIN}\NormalTok{ ways w2 }\KeywordTok{ON}\NormalTok{ w1.mene }\OperatorTok{=}\NormalTok{ w2.mene}
    \KeywordTok{WHERE}\NormalTok{ w1.}\KeywordTok{ID} \OperatorTok{!=}\NormalTok{ w2.}\KeywordTok{ID} \KeywordTok{AND}\NormalTok{ w1.mene }\OperatorTok{!=} \StringTok{\textquotesingle{}\textquotesingle{}}
    \KeywordTok{UNION}
    \KeywordTok{SELECT}\NormalTok{ w1.way, w2.}\KeywordTok{ID}
    \KeywordTok{FROM}\NormalTok{ ways w1}
    \KeywordTok{JOIN}\NormalTok{ ways w2 }\KeywordTok{ON}\NormalTok{ w1.dialoguewith }\OperatorTok{=}\NormalTok{ w2.dialoguewith}
    \KeywordTok{WHERE}\NormalTok{ w1.}\KeywordTok{ID} \OperatorTok{!=}\NormalTok{ w2.}\KeywordTok{ID} \KeywordTok{AND}\NormalTok{ w1.dialoguewith }\OperatorTok{!=} \StringTok{\textquotesingle{}\textquotesingle{}}
\NormalTok{)}
\KeywordTok{GROUP} \KeywordTok{BY}\NormalTok{ way}
\KeywordTok{ORDER} \KeywordTok{BY}\NormalTok{ connection\_count }\KeywordTok{DESC}
\KeywordTok{LIMIT} \DecValTok{10}\NormalTok{;}
\end{Highlighting}
\end{Shaded}
\end{frame}

\begin{frame}[fragile]{Implementation}
\protect\phantomsection\label{implementation}
The analysis is implemented using several specialized Python modules:

\begin{block}{Core Analysis Modules}
\protect\phantomsection\label{core-analysis-modules}
\begin{itemize}
\tightlist
\item
  \textbf{\texttt{database.py}}: SQLAlchemy ORM with
  \texttt{WaysDatabase} class for database access
\item
  \textbf{\texttt{sql\_queries.py}}: \texttt{WaysSQLQueries} class with
  pre-built analysis queries
\item
  \textbf{\texttt{ways\_analysis.py}}: \texttt{WaysAnalyzer} class for
  comprehensive ways characterization
\item
  \textbf{\texttt{network\_analysis.py}}: \texttt{WaysNetworkAnalyzer}
  class for graph-based relationship analysis
\item
  \textbf{\texttt{house\_of\_knowledge.py}}: Framework analysis for the
  24-room House of Knowledge
\item
  \textbf{\texttt{statistics.py}}: Statistical functions including
  \texttt{analyze\_way\_distributions()},
  \texttt{compute\_way\_correlations()}
\item
  \textbf{\texttt{metrics.py}}: Performance metrics including
  \texttt{compute\_way\_coverage\_metrics()},
  \texttt{compute\_way\_interconnectedness()}
\end{itemize}
\end{block}

\begin{block}{Infrastructure}
\protect\phantomsection\label{infrastructure}
\begin{itemize}
\tightlist
\item
  \textbf{Python 3.10+}: Primary analysis language
\item
  \textbf{SQLite}: Database backend via SQLAlchemy ORM
\item
  \textbf{NetworkX}: Network analysis and graph algorithms
\item
  \textbf{Matplotlib/Seaborn}: Statistical visualization and plotting
\item
  \textbf{NumPy/Pandas}: Numerical computing and data manipulation
\end{itemize}

All code follows the thin orchestrator pattern, with business logic in
\texttt{project/src/} modules and orchestration in
\texttt{project/scripts/}.
\end{block}
\end{frame}

\begin{frame}{Ethical Considerations}
\protect\phantomsection\label{ethical-considerations}
This research documents and analyzes publicly available philosophical
work by Andrius Kulikauskas. All data is in the public domain as stated
in the source documentation. The analysis respects the original
philosophical framework while providing systematic documentation and
quantitative insights.
\end{frame}

\end{document}
