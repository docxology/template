% Options for packages loaded elsewhere
\PassOptionsToPackage{unicode}{hyperref}
\PassOptionsToPackage{hyphens}{url}
\documentclass[
  ignorenonframetext,
]{beamer}
\newif\ifbibliography
\usepackage{pgfpages}
\setbeamertemplate{caption}[numbered]
\setbeamertemplate{caption label separator}{: }
\setbeamercolor{caption name}{fg=normal text.fg}
\beamertemplatenavigationsymbolsempty
% remove section numbering
\setbeamertemplate{part page}{
  \centering
  \begin{beamercolorbox}[sep=16pt,center]{part title}
    \usebeamerfont{part title}\insertpart\par
  \end{beamercolorbox}
}
\setbeamertemplate{section page}{
  \centering
  \begin{beamercolorbox}[sep=12pt,center]{section title}
    \usebeamerfont{section title}\insertsection\par
  \end{beamercolorbox}
}
\setbeamertemplate{subsection page}{
  \centering
  \begin{beamercolorbox}[sep=8pt,center]{subsection title}
    \usebeamerfont{subsection title}\insertsubsection\par
  \end{beamercolorbox}
}
% Prevent slide breaks in the middle of a paragraph
\widowpenalties 1 10000
\raggedbottom
\AtBeginPart{
  \frame{\partpage}
}
\AtBeginSection{
  \ifbibliography
  \else
    \frame{\sectionpage}
  \fi
}
\AtBeginSubsection{
  \frame{\subsectionpage}
}
\usepackage{iftex}
\ifPDFTeX
  \usepackage[T1]{fontenc}
  \usepackage[utf8]{inputenc}
  \usepackage{textcomp} % provide euro and other symbols
\else % if luatex or xetex
  \usepackage{unicode-math} % this also loads fontspec
  \defaultfontfeatures{Scale=MatchLowercase}
  \defaultfontfeatures[\rmfamily]{Ligatures=TeX,Scale=1}
\fi
\usepackage{lmodern}
\ifPDFTeX\else
  % xetex/luatex font selection
\fi
% Use upquote if available, for straight quotes in verbatim environments
\IfFileExists{upquote.sty}{\usepackage{upquote}}{}
\IfFileExists{microtype.sty}{% use microtype if available
  \usepackage[]{microtype}
  \UseMicrotypeSet[protrusion]{basicmath} % disable protrusion for tt fonts
}{}
\makeatletter
\@ifundefined{KOMAClassName}{% if non-KOMA class
  \IfFileExists{parskip.sty}{%
    \usepackage{parskip}
  }{% else
    \setlength{\parindent}{0pt}
    \setlength{\parskip}{6pt plus 2pt minus 1pt}}
}{% if KOMA class
  \KOMAoptions{parskip=half}}
\makeatother
\usepackage{color}
\usepackage{fancyvrb}
\newcommand{\VerbBar}{|}
\newcommand{\VERB}{\Verb[commandchars=\\\{\}]}
\DefineVerbatimEnvironment{Highlighting}{Verbatim}{commandchars=\\\{\}}
% Add ',fontsize=\small' for more characters per line
\newenvironment{Shaded}{}{}
\newcommand{\AlertTok}[1]{\textcolor[rgb]{1.00,0.00,0.00}{\textbf{#1}}}
\newcommand{\AnnotationTok}[1]{\textcolor[rgb]{0.38,0.63,0.69}{\textbf{\textit{#1}}}}
\newcommand{\AttributeTok}[1]{\textcolor[rgb]{0.49,0.56,0.16}{#1}}
\newcommand{\BaseNTok}[1]{\textcolor[rgb]{0.25,0.63,0.44}{#1}}
\newcommand{\BuiltInTok}[1]{\textcolor[rgb]{0.00,0.50,0.00}{#1}}
\newcommand{\CharTok}[1]{\textcolor[rgb]{0.25,0.44,0.63}{#1}}
\newcommand{\CommentTok}[1]{\textcolor[rgb]{0.38,0.63,0.69}{\textit{#1}}}
\newcommand{\CommentVarTok}[1]{\textcolor[rgb]{0.38,0.63,0.69}{\textbf{\textit{#1}}}}
\newcommand{\ConstantTok}[1]{\textcolor[rgb]{0.53,0.00,0.00}{#1}}
\newcommand{\ControlFlowTok}[1]{\textcolor[rgb]{0.00,0.44,0.13}{\textbf{#1}}}
\newcommand{\DataTypeTok}[1]{\textcolor[rgb]{0.56,0.13,0.00}{#1}}
\newcommand{\DecValTok}[1]{\textcolor[rgb]{0.25,0.63,0.44}{#1}}
\newcommand{\DocumentationTok}[1]{\textcolor[rgb]{0.73,0.13,0.13}{\textit{#1}}}
\newcommand{\ErrorTok}[1]{\textcolor[rgb]{1.00,0.00,0.00}{\textbf{#1}}}
\newcommand{\ExtensionTok}[1]{#1}
\newcommand{\FloatTok}[1]{\textcolor[rgb]{0.25,0.63,0.44}{#1}}
\newcommand{\FunctionTok}[1]{\textcolor[rgb]{0.02,0.16,0.49}{#1}}
\newcommand{\ImportTok}[1]{\textcolor[rgb]{0.00,0.50,0.00}{\textbf{#1}}}
\newcommand{\InformationTok}[1]{\textcolor[rgb]{0.38,0.63,0.69}{\textbf{\textit{#1}}}}
\newcommand{\KeywordTok}[1]{\textcolor[rgb]{0.00,0.44,0.13}{\textbf{#1}}}
\newcommand{\NormalTok}[1]{#1}
\newcommand{\OperatorTok}[1]{\textcolor[rgb]{0.40,0.40,0.40}{#1}}
\newcommand{\OtherTok}[1]{\textcolor[rgb]{0.00,0.44,0.13}{#1}}
\newcommand{\PreprocessorTok}[1]{\textcolor[rgb]{0.74,0.48,0.00}{#1}}
\newcommand{\RegionMarkerTok}[1]{#1}
\newcommand{\SpecialCharTok}[1]{\textcolor[rgb]{0.25,0.44,0.63}{#1}}
\newcommand{\SpecialStringTok}[1]{\textcolor[rgb]{0.73,0.40,0.53}{#1}}
\newcommand{\StringTok}[1]{\textcolor[rgb]{0.25,0.44,0.63}{#1}}
\newcommand{\VariableTok}[1]{\textcolor[rgb]{0.10,0.09,0.49}{#1}}
\newcommand{\VerbatimStringTok}[1]{\textcolor[rgb]{0.25,0.44,0.63}{#1}}
\newcommand{\WarningTok}[1]{\textcolor[rgb]{0.38,0.63,0.69}{\textbf{\textit{#1}}}}
\setlength{\emergencystretch}{3em} % prevent overfull lines
\providecommand{\tightlist}{%
  \setlength{\itemsep}{0pt}\setlength{\parskip}{0pt}}
\usepackage{bookmark}
\IfFileExists{xurl.sty}{\usepackage{xurl}}{} % add URL line breaks if available
\urlstyle{same}
\hypersetup{
  hidelinks,
  pdfcreator={LaTeX via pandoc}}

\author{\texorpdfstring{}{}}
\date{}

\begin{document}

\section{Introduction}\label{sec:introduction}

\begin{frame}{Historical Overview}
\protect\phantomsection\label{historical-overview}
Tree grafting stands as one of humanity's most enduring agricultural
innovations, with archaeological evidence suggesting its practice dates
to at least 2000 BCE in ancient Mesopotamia and China \cite{garner2013}.
The technique has been independently developed across multiple
civilizations, from the sophisticated fruit tree cultivation of ancient
Rome documented by Cato and Pliny \cite{white1970}, to the elaborate
grafting practices of imperial Chinese gardens \cite{needham1984}, to
the traditional knowledge systems of indigenous peoples worldwide. This
4,000+ year history demonstrates grafting's fundamental importance to
human agriculture and food security.
\end{frame}

\begin{frame}{Modern Context and Agricultural Importance}
\protect\phantomsection\label{modern-context-and-agricultural-importance}
In contemporary agriculture, grafting remains essential for commercial
fruit and nut production, enabling the combination of desirable scion
characteristics (fruit quality, yield, disease resistance) with
rootstock advantages (vigor control, soil adaptation, pest resistance)
\cite{webster2002, hartmann2014}. The global fruit industry, valued at
over \$100 billion annually, relies heavily on grafted trees for
consistent production, quality control, and disease management. Beyond
commercial agriculture, grafting serves critical roles in ornamental
horticulture, forest restoration, urban tree management, and
conservation of rare or endangered species \cite{stebbins1950}.
\end{frame}

\begin{frame}{Economic Scale and Impact}
\protect\phantomsection\label{economic-scale-and-impact}
The economic impact of grafting extends far beyond direct agricultural
production. Grafted trees enable: - \textbf{Increased productivity}:
20-40\% yield improvements through optimized rootstock-scion
combinations - \textbf{Disease resistance}: Protection against
soil-borne pathogens through resistant rootstocks - \textbf{Climate
adaptation}: Extension of cultivation ranges through rootstock selection
- \textbf{Quality consistency}: Uniform fruit characteristics across
orchards - \textbf{Cost efficiency}: Reduced pesticide use and improved
resource utilization

These benefits translate to significant economic value, with grafting
operations representing a multi-billion dollar industry supporting
millions of livelihoods worldwide.
\end{frame}

\begin{frame}[fragile]{Project Structure and Objectives}
\protect\phantomsection\label{project-structure-and-objectives}
This research project provides both a comprehensive transdisciplinary
review of tree grafting and a computational toolkit for practical
application. The project follows a standardized structure:

\begin{itemize}
\tightlist
\item
  \textbf{\texttt{src/}} - Source code implementing grafting analysis
  algorithms, compatibility prediction, biological simulation, and
  statistical analysis
\item
  \textbf{\texttt{tests/}} - Comprehensive test suite ensuring 100\%
  code coverage
\item
  \textbf{\texttt{scripts/}} - Analysis scripts for generating figures,
  running simulations, and creating reports
\item
  \textbf{\texttt{manuscript/}} - Markdown source files for the
  comprehensive review manuscript
\item
  \textbf{\texttt{output/}} - Generated outputs (PDFs, figures, data,
  reports)
\end{itemize}
\end{frame}

\begin{frame}{Key Features of the Toolkit}
\protect\phantomsection\label{key-features-of-the-toolkit}
\begin{block}{Compatibility Prediction}
\protect\phantomsection\label{compatibility-prediction}
The toolkit provides algorithms for predicting graft compatibility based
on phylogenetic distance, cambium characteristics, growth rates, and
environmental factors, enabling informed rootstock-scion pair selection.
\end{block}

\begin{block}{Biological Process Simulation}
\protect\phantomsection\label{biological-process-simulation}
Simulation models capture the temporal dynamics of graft healing,
including cambium integration, callus formation, and vascular
connection, providing insights into union development.
\end{block}

\begin{block}{Statistical Analysis}
\protect\phantomsection\label{statistical-analysis}
Comprehensive statistical tools analyze success rates, factor
importance, technique comparisons, and survival curves, supporting
evidence-based decision making.
\end{block}

\begin{block}{Decision Support Systems}
\protect\phantomsection\label{decision-support-systems}
Interactive tools assist with rootstock selection, technique
recommendation, seasonal planning, and economic analysis, making expert
knowledge accessible to practitioners.
\end{block}
\end{frame}

\begin{frame}{Manuscript Organization}
\protect\phantomsection\label{manuscript-organization}
The manuscript is organized into several key sections:

\begin{enumerate}
\tightlist
\item
  \textbf{Abstract} (Section \ref{sec:abstract}): Comprehensive overview
  of grafting and toolkit contributions
\item
  \textbf{Introduction} (Section \ref{sec:introduction}): Historical
  context, modern importance, and project structure
\item
  \textbf{Methodology} (Section \ref{sec:methodology}): Biological
  mechanisms, grafting techniques, compatibility theory, and
  computational framework
\item
  \textbf{Experimental Results} (Section
  \ref{sec:experimental_results}): Compatibility database results,
  technique effectiveness, environmental analysis, and model validation
\item
  \textbf{Discussion} (Section \ref{sec:discussion}): Biological
  insights, technical implications, agricultural applications, and
  economic considerations
\item
  \textbf{Conclusion} (Section \ref{sec:conclusion}): Synthesis of
  findings, practical recommendations, and future research directions
\item
  \textbf{References} (Section \ref{sec:references}): Comprehensive
  bibliography of grafting literature
\end{enumerate}
\end{frame}

\begin{frame}{Example Figure}
\protect\phantomsection\label{example-figure}
The following figure demonstrates graft union anatomy:

\begin{figure}[h]
\centering
\includegraphics[width=0.8\textwidth]{../output/figures/graft_anatomy.png}
\caption{Anatomical diagram showing graft union with cambium alignment between rootstock and scion}
\label{fig:graft_anatomy}
\end{figure}

As shown in Figure \ref{fig:graft_anatomy}, successful grafting requires
precise alignment of the cambium layers, the thin meristematic tissue
responsible for secondary growth. This alignment enables vascular
connection and callus formation, ultimately establishing a functional
union between rootstock and scion.
\end{frame}

\begin{frame}[fragile]{Data Availability and Reproducibility}
\protect\phantomsection\label{data-availability-and-reproducibility}
All generated data, figures, and analysis results are saved for
reproducibility:

\begin{itemize}
\tightlist
\item
  \textbf{Figures}: PNG and PDF formats in \texttt{output/figures/}
\item
  \textbf{Data}: NPZ and CSV formats in \texttt{output/data/}
\item
  \textbf{Simulations}: JSON and NPZ formats in
  \texttt{output/simulations/}
\item
  \textbf{Reports}: Markdown and HTML formats in
  \texttt{output/reports/}
\item
  \textbf{PDFs}: Individual and combined documents in
  \texttt{output/pdf/}
\end{itemize}
\end{frame}

\begin{frame}[fragile]{Usage}
\protect\phantomsection\label{usage}
To generate the complete manuscript and run analyses:

\begin{Shaded}
\begin{Highlighting}[]
\CommentTok{\# Run complete pipeline (tests + analysis + PDF generation)}
\ExtensionTok{python3}\NormalTok{ scripts/run\_all.py}

\CommentTok{\# Or use the shell script}
\ExtensionTok{./run.sh} \AttributeTok{{-}{-}pipeline}
\end{Highlighting}
\end{Shaded}

The system automatically: 1. Runs all tests with 100\% coverage
requirement 2. Executes grafting analysis scripts to generate figures
and data 3. Validates markdown references and images 4. Generates
individual and combined PDFs 5. Creates comprehensive reports
\end{frame}

\begin{frame}[fragile]{Cross-Referencing System}
\protect\phantomsection\label{cross-referencing-system}
The manuscript demonstrates comprehensive cross-referencing:

\begin{itemize}
\tightlist
\item
  \textbf{Section References}: Use
  \texttt{\textbackslash{}ref\{sec:section\_name\}} for sections
\item
  \textbf{Equation References}: Use
  \texttt{\textbackslash{}eqref\{eq:equation\_name\}} for equations
\item
  \textbf{Figure References}: Use
  \texttt{\textbackslash{}ref\{fig:figure\_name\}} for figures
\item
  \textbf{Table References}: Use
  \texttt{\textbackslash{}ref\{tab:table\_name\}} for tables
\item
  \textbf{Citation References}: Use
  \texttt{\textbackslash{}cite\{author\_year\}} for literature citations
\end{itemize}

This system ensures proper navigation and maintains consistency
throughout the document.
\end{frame}

\end{document}
