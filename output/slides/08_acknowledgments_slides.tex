% Options for packages loaded elsewhere
\PassOptionsToPackage{unicode}{hyperref}
\PassOptionsToPackage{hyphens}{url}
\documentclass[
  ignorenonframetext,
]{beamer}
\newif\ifbibliography
\usepackage{pgfpages}
\setbeamertemplate{caption}[numbered]
\setbeamertemplate{caption label separator}{: }
\setbeamercolor{caption name}{fg=normal text.fg}
\beamertemplatenavigationsymbolsempty
% remove section numbering
\setbeamertemplate{part page}{
  \centering
  \begin{beamercolorbox}[sep=16pt,center]{part title}
    \usebeamerfont{part title}\insertpart\par
  \end{beamercolorbox}
}
\setbeamertemplate{section page}{
  \centering
  \begin{beamercolorbox}[sep=12pt,center]{section title}
    \usebeamerfont{section title}\insertsection\par
  \end{beamercolorbox}
}
\setbeamertemplate{subsection page}{
  \centering
  \begin{beamercolorbox}[sep=8pt,center]{subsection title}
    \usebeamerfont{subsection title}\insertsubsection\par
  \end{beamercolorbox}
}
% Prevent slide breaks in the middle of a paragraph
\widowpenalties 1 10000
\raggedbottom
\AtBeginPart{
  \frame{\partpage}
}
\AtBeginSection{
  \ifbibliography
  \else
    \frame{\sectionpage}
  \fi
}
\AtBeginSubsection{
  \frame{\subsectionpage}
}
\usepackage{iftex}
\ifPDFTeX
  \usepackage[T1]{fontenc}
  \usepackage[utf8]{inputenc}
  \usepackage{textcomp} % provide euro and other symbols
\else % if luatex or xetex
  \usepackage{unicode-math} % this also loads fontspec
  \defaultfontfeatures{Scale=MatchLowercase}
  \defaultfontfeatures[\rmfamily]{Ligatures=TeX,Scale=1}
\fi
\usepackage{lmodern}
\ifPDFTeX\else
  % xetex/luatex font selection
\fi
% Use upquote if available, for straight quotes in verbatim environments
\IfFileExists{upquote.sty}{\usepackage{upquote}}{}
\IfFileExists{microtype.sty}{% use microtype if available
  \usepackage[]{microtype}
  \UseMicrotypeSet[protrusion]{basicmath} % disable protrusion for tt fonts
}{}
\makeatletter
\@ifundefined{KOMAClassName}{% if non-KOMA class
  \IfFileExists{parskip.sty}{%
    \usepackage{parskip}
  }{% else
    \setlength{\parindent}{0pt}
    \setlength{\parskip}{6pt plus 2pt minus 1pt}}
}{% if KOMA class
  \KOMAoptions{parskip=half}}
\makeatother
\setlength{\emergencystretch}{3em} % prevent overfull lines
\providecommand{\tightlist}{%
  \setlength{\itemsep}{0pt}\setlength{\parskip}{0pt}}
\usepackage{bookmark}
\IfFileExists{xurl.sty}{\usepackage{xurl}}{} % add URL line breaks if available
\urlstyle{same}
\hypersetup{
  hidelinks,
  pdfcreator={LaTeX via pandoc}}

\author{\texorpdfstring{}{}}
\date{}

\begin{document}

\begin{frame}[fragile]{Acknowledgments}
\protect\phantomsection\label{sec:acknowledgments}
We gratefully acknowledge the contributions that made this research
possible.

\begin{block}{Primary Source}
\protect\phantomsection\label{primary-source}
This research is based entirely on the philosophical work of
\textbf{Andrius Kulikauskas}, who developed the ``Ways of Figuring
Things Out'' framework and documented 284 ways of knowledge acquisition.
The framework, database, and documentation are the result of his
extensive philosophical work conducted in 2010-2011.
\end{block}

\begin{block}{Data Availability}
\protect\phantomsection\label{data-availability}
All data used in this research is in the \textbf{Public Domain} as
stated in the source documentation. The MySQL database dump and text
documentation (\texttt{ways.md}) are publicly available and were used
with appropriate attribution.
\end{block}

\begin{block}{Framework Development}
\protect\phantomsection\label{framework-development}
The House of Knowledge framework, the 24-room structure, and the
dialogue type classifications are all part of Andrius Kulikauskas's
original philosophical work. This research provides systematic
documentation and analysis but does not claim to have developed the
underlying framework.
\end{block}

\begin{block}{Technical Infrastructure}
\protect\phantomsection\label{technical-infrastructure}
This research builds upon:

\begin{itemize}
\tightlist
\item
  \textbf{Python scientific computing stack} (NumPy, SciPy, Pandas,
  NetworkX, Matplotlib)
\item
  \textbf{SQLite} database system for data storage and querying
\item
  \textbf{LaTeX and Pandoc} for document preparation
\item
  \textbf{Open-source tools} for data analysis and visualization
\end{itemize}
\end{block}

\begin{block}{Research Context}
\protect\phantomsection\label{research-context}
This work contributes to the systematic documentation and analysis of
philosophical frameworks, demonstrating how quantitative methods can
complement qualitative understanding. The integration of database
analysis, network analysis, and statistical methods with philosophical
interpretation represents a methodological contribution to the study of
knowledge systems.
\end{block}

\begin{block}{Future Contributions}
\protect\phantomsection\label{future-contributions}
Future researchers building on this work should acknowledge: - Andrius
Kulikauskas as the originator of the Ways framework - The public domain
status of the source data - The systematic analysis and documentation
provided by this research
\end{block}
\end{frame}

\begin{frame}
\emph{All errors and omissions in the analysis and interpretation remain
the sole responsibility of the authors. The underlying philosophical
framework and data are the work of Andrius Kulikauskas.}
\end{frame}

\end{document}
