% Options for packages loaded elsewhere
\PassOptionsToPackage{unicode}{hyperref}
\PassOptionsToPackage{hyphens}{url}
\documentclass[
  ignorenonframetext,
]{beamer}
\newif\ifbibliography
\usepackage{pgfpages}
\setbeamertemplate{caption}[numbered]
\setbeamertemplate{caption label separator}{: }
\setbeamercolor{caption name}{fg=normal text.fg}
\beamertemplatenavigationsymbolsempty
% remove section numbering
\setbeamertemplate{part page}{
  \centering
  \begin{beamercolorbox}[sep=16pt,center]{part title}
    \usebeamerfont{part title}\insertpart\par
  \end{beamercolorbox}
}
\setbeamertemplate{section page}{
  \centering
  \begin{beamercolorbox}[sep=12pt,center]{section title}
    \usebeamerfont{section title}\insertsection\par
  \end{beamercolorbox}
}
\setbeamertemplate{subsection page}{
  \centering
  \begin{beamercolorbox}[sep=8pt,center]{subsection title}
    \usebeamerfont{subsection title}\insertsubsection\par
  \end{beamercolorbox}
}
% Prevent slide breaks in the middle of a paragraph
\widowpenalties 1 10000
\raggedbottom
\AtBeginPart{
  \frame{\partpage}
}
\AtBeginSection{
  \ifbibliography
  \else
    \frame{\sectionpage}
  \fi
}
\AtBeginSubsection{
  \frame{\subsectionpage}
}
\usepackage{iftex}
\ifPDFTeX
  \usepackage[T1]{fontenc}
  \usepackage[utf8]{inputenc}
  \usepackage{textcomp} % provide euro and other symbols
\else % if luatex or xetex
  \usepackage{unicode-math} % this also loads fontspec
  \defaultfontfeatures{Scale=MatchLowercase}
  \defaultfontfeatures[\rmfamily]{Ligatures=TeX,Scale=1}
\fi
\usepackage{lmodern}
\ifPDFTeX\else
  % xetex/luatex font selection
\fi
% Use upquote if available, for straight quotes in verbatim environments
\IfFileExists{upquote.sty}{\usepackage{upquote}}{}
\IfFileExists{microtype.sty}{% use microtype if available
  \usepackage[]{microtype}
  \UseMicrotypeSet[protrusion]{basicmath} % disable protrusion for tt fonts
}{}
\makeatletter
\@ifundefined{KOMAClassName}{% if non-KOMA class
  \IfFileExists{parskip.sty}{%
    \usepackage{parskip}
  }{% else
    \setlength{\parindent}{0pt}
    \setlength{\parskip}{6pt plus 2pt minus 1pt}}
}{% if KOMA class
  \KOMAoptions{parskip=half}}
\makeatother
\setlength{\emergencystretch}{3em} % prevent overfull lines
\providecommand{\tightlist}{%
  \setlength{\itemsep}{0pt}\setlength{\parskip}{0pt}}
\usepackage{bookmark}
\IfFileExists{xurl.sty}{\usepackage{xurl}}{} % add URL line breaks if available
\urlstyle{same}
\hypersetup{
  hidelinks,
  pdfcreator={LaTeX via pandoc}}

\author{\texorpdfstring{}{}}
\date{}

\begin{document}

\begin{frame}{Acknowledgments}
\protect\phantomsection\label{acknowledgments}
This work stands on the foundations laid by G. Spencer-Brown, whose
\emph{Laws of Form} (1969) opened a new path in mathematical logic. We
acknowledge the profound influence of his insight that distinction
precedes all else.

We are grateful to Louis H. Kauffman for his extensive work connecting
the calculus of indications to knot theory, self-reference, and category
theory, and for keeping the Laws of Form tradition alive in contemporary
mathematics.

William Bricken's development of boundary mathematics for computation
demonstrated the practical viability of iconic notation and inspired the
computational framework presented here.

The philosophical grounding draws extensively from the North American
pragmatist tradition---Charles Sanders Peirce, William James, John
Dewey---whose emphasis on consequences and operations aligns with the
calculus's operational character. We also acknowledge the
neo-materialist contributions of Karen Barad, Donna Haraway, and Jane
Bennett, whose work on agential cuts and relational ontology illuminates
the metaphysical significance of distinction.

The infrastructure for this research project was developed using the
Research Project Template, providing reproducible build processes,
automated testing, and integrated literature management.

Computational verification was performed using Python with NumPy and
Matplotlib for visualization. All source code is available in the
accompanying repository under the Apache 2.0 license.
\end{frame}

\begin{frame}
\emph{``Draw a distinction.''}\\
--- G. Spencer-Brown, \emph{Laws of Form} (1969)
\end{frame}

\end{document}
