% Options for packages loaded elsewhere
\PassOptionsToPackage{unicode}{hyperref}
\PassOptionsToPackage{hyphens}{url}
\documentclass[
  ignorenonframetext,
]{beamer}
\newif\ifbibliography
\usepackage{pgfpages}
\setbeamertemplate{caption}[numbered]
\setbeamertemplate{caption label separator}{: }
\setbeamercolor{caption name}{fg=normal text.fg}
\beamertemplatenavigationsymbolsempty
% remove section numbering
\setbeamertemplate{part page}{
  \centering
  \begin{beamercolorbox}[sep=16pt,center]{part title}
    \usebeamerfont{part title}\insertpart\par
  \end{beamercolorbox}
}
\setbeamertemplate{section page}{
  \centering
  \begin{beamercolorbox}[sep=12pt,center]{section title}
    \usebeamerfont{section title}\insertsection\par
  \end{beamercolorbox}
}
\setbeamertemplate{subsection page}{
  \centering
  \begin{beamercolorbox}[sep=8pt,center]{subsection title}
    \usebeamerfont{subsection title}\insertsubsection\par
  \end{beamercolorbox}
}
% Prevent slide breaks in the middle of a paragraph
\widowpenalties 1 10000
\raggedbottom
\AtBeginPart{
  \frame{\partpage}
}
\AtBeginSection{
  \ifbibliography
  \else
    \frame{\sectionpage}
  \fi
}
\AtBeginSubsection{
  \frame{\subsectionpage}
}
\usepackage{iftex}
\ifPDFTeX
  \usepackage[T1]{fontenc}
  \usepackage[utf8]{inputenc}
  \usepackage{textcomp} % provide euro and other symbols
\else % if luatex or xetex
  \usepackage{unicode-math} % this also loads fontspec
  \defaultfontfeatures{Scale=MatchLowercase}
  \defaultfontfeatures[\rmfamily]{Ligatures=TeX,Scale=1}
\fi
\usepackage{lmodern}
\ifPDFTeX\else
  % xetex/luatex font selection
\fi
% Use upquote if available, for straight quotes in verbatim environments
\IfFileExists{upquote.sty}{\usepackage{upquote}}{}
\IfFileExists{microtype.sty}{% use microtype if available
  \usepackage[]{microtype}
  \UseMicrotypeSet[protrusion]{basicmath} % disable protrusion for tt fonts
}{}
\makeatletter
\@ifundefined{KOMAClassName}{% if non-KOMA class
  \IfFileExists{parskip.sty}{%
    \usepackage{parskip}
  }{% else
    \setlength{\parindent}{0pt}
    \setlength{\parskip}{6pt plus 2pt minus 1pt}}
}{% if KOMA class
  \KOMAoptions{parskip=half}}
\makeatother
\usepackage{color}
\usepackage{fancyvrb}
\newcommand{\VerbBar}{|}
\newcommand{\VERB}{\Verb[commandchars=\\\{\}]}
\DefineVerbatimEnvironment{Highlighting}{Verbatim}{commandchars=\\\{\}}
% Add ',fontsize=\small' for more characters per line
\newenvironment{Shaded}{}{}
\newcommand{\AlertTok}[1]{\textcolor[rgb]{1.00,0.00,0.00}{\textbf{#1}}}
\newcommand{\AnnotationTok}[1]{\textcolor[rgb]{0.38,0.63,0.69}{\textbf{\textit{#1}}}}
\newcommand{\AttributeTok}[1]{\textcolor[rgb]{0.49,0.56,0.16}{#1}}
\newcommand{\BaseNTok}[1]{\textcolor[rgb]{0.25,0.63,0.44}{#1}}
\newcommand{\BuiltInTok}[1]{\textcolor[rgb]{0.00,0.50,0.00}{#1}}
\newcommand{\CharTok}[1]{\textcolor[rgb]{0.25,0.44,0.63}{#1}}
\newcommand{\CommentTok}[1]{\textcolor[rgb]{0.38,0.63,0.69}{\textit{#1}}}
\newcommand{\CommentVarTok}[1]{\textcolor[rgb]{0.38,0.63,0.69}{\textbf{\textit{#1}}}}
\newcommand{\ConstantTok}[1]{\textcolor[rgb]{0.53,0.00,0.00}{#1}}
\newcommand{\ControlFlowTok}[1]{\textcolor[rgb]{0.00,0.44,0.13}{\textbf{#1}}}
\newcommand{\DataTypeTok}[1]{\textcolor[rgb]{0.56,0.13,0.00}{#1}}
\newcommand{\DecValTok}[1]{\textcolor[rgb]{0.25,0.63,0.44}{#1}}
\newcommand{\DocumentationTok}[1]{\textcolor[rgb]{0.73,0.13,0.13}{\textit{#1}}}
\newcommand{\ErrorTok}[1]{\textcolor[rgb]{1.00,0.00,0.00}{\textbf{#1}}}
\newcommand{\ExtensionTok}[1]{#1}
\newcommand{\FloatTok}[1]{\textcolor[rgb]{0.25,0.63,0.44}{#1}}
\newcommand{\FunctionTok}[1]{\textcolor[rgb]{0.02,0.16,0.49}{#1}}
\newcommand{\ImportTok}[1]{\textcolor[rgb]{0.00,0.50,0.00}{\textbf{#1}}}
\newcommand{\InformationTok}[1]{\textcolor[rgb]{0.38,0.63,0.69}{\textbf{\textit{#1}}}}
\newcommand{\KeywordTok}[1]{\textcolor[rgb]{0.00,0.44,0.13}{\textbf{#1}}}
\newcommand{\NormalTok}[1]{#1}
\newcommand{\OperatorTok}[1]{\textcolor[rgb]{0.40,0.40,0.40}{#1}}
\newcommand{\OtherTok}[1]{\textcolor[rgb]{0.00,0.44,0.13}{#1}}
\newcommand{\PreprocessorTok}[1]{\textcolor[rgb]{0.74,0.48,0.00}{#1}}
\newcommand{\RegionMarkerTok}[1]{#1}
\newcommand{\SpecialCharTok}[1]{\textcolor[rgb]{0.25,0.44,0.63}{#1}}
\newcommand{\SpecialStringTok}[1]{\textcolor[rgb]{0.73,0.40,0.53}{#1}}
\newcommand{\StringTok}[1]{\textcolor[rgb]{0.25,0.44,0.63}{#1}}
\newcommand{\VariableTok}[1]{\textcolor[rgb]{0.10,0.09,0.49}{#1}}
\newcommand{\VerbatimStringTok}[1]{\textcolor[rgb]{0.25,0.44,0.63}{#1}}
\newcommand{\WarningTok}[1]{\textcolor[rgb]{0.38,0.63,0.69}{\textbf{\textit{#1}}}}
\usepackage{longtable,booktabs,array}
\newcounter{none} % for unnumbered tables
\usepackage{calc} % for calculating minipage widths
\usepackage{caption}
% Make caption package work with longtable
\makeatletter
\def\fnum@table{\tablename~\thetable}
\makeatother
\setlength{\emergencystretch}{3em} % prevent overfull lines
\providecommand{\tightlist}{%
  \setlength{\itemsep}{0pt}\setlength{\parskip}{0pt}}
\usepackage{bookmark}
\IfFileExists{xurl.sty}{\usepackage{xurl}}{} % add URL line breaks if available
\urlstyle{same}
\hypersetup{
  hidelinks,
  pdfcreator={LaTeX via pandoc}}

\author{\texorpdfstring{}{}}
\date{}

\begin{document}

\section{Appendix}\label{appendix}

\begin{frame}{A. Complete Axiom Derivations}
\protect\phantomsection\label{a.-complete-axiom-derivations}
\begin{block}{A.1 Calling Axiom (J1) Proof}
\protect\phantomsection\label{a.1-calling-axiom-j1-proof}
\textbf{Statement}: \(\langle\langle a \rangle\rangle = a\)

\textbf{Spatial Interpretation}: Consider a space with form \(a\).
Enclosing \(a\) creates \(\langle a \rangle\)---we are now ``outside''
\(a\) (inside the boundary around \(a\)). Enclosing again creates
\(\langle\langle a \rangle\rangle\)---we are now ``outside'' of being
``outside'' \(a\), which returns us to \(a\).

\textbf{Algebraic Proof}: Let \(\llbracket \cdot \rrbracket\) denote
truth value evaluation. -
\(\llbracket\langle\langle a \rangle\rangle\rrbracket = \neg\llbracket\langle a \rangle\rrbracket\)
(by enclosure semantics) - \(= \neg\neg\llbracket a \rrbracket\) (by
enclosure semantics again) - \(= \llbracket a \rrbracket\) (by double
negation)

Since truth values are preserved and the calculus is sound,
\(\langle\langle a \rangle\rangle = a\).
\end{block}

\begin{block}{A.2 Crossing Axiom (J2) Proof}
\protect\phantomsection\label{a.2-crossing-axiom-j2-proof}
\textbf{Statement}:
\(\langle\ \rangle\langle\ \rangle = \langle\ \rangle\)

\textbf{Spatial Interpretation}: Two marks side by side both indicate
``the marked state.'' Indicating the same state twice does not change
what is indicated.

\textbf{Algebraic Proof}: -
\(\llbracket\langle\ \rangle\langle\ \rangle\rrbracket = \llbracket\langle\ \rangle\rrbracket \land \llbracket\langle\ \rangle\rrbracket\)
(by juxtaposition semantics) - \(= \text{TRUE} \land \text{TRUE}\) (mark
is TRUE) - \(= \text{TRUE}\) -
\(= \llbracket\langle\ \rangle\rrbracket\)
\end{block}
\end{frame}

\begin{frame}{B. Consequence Derivations}
\protect\phantomsection\label{b.-consequence-derivations}
\begin{block}{B.1 C1: Position}
\protect\phantomsection\label{b.1-c1-position}
\textbf{Statement}: \(\langle\langle a \rangle b \rangle a = a\)

\textbf{Derivation}: Consider the Boolean interpretation: - LHS =
\(\neg(\neg a \land b) \land a\) - \(= (a \lor \neg b) \land a\) (De
Morgan) - \(= a \land (a \lor \neg b)\) (commutative) - \(= a\)
(absorption)
\end{block}

\begin{block}{B.2 C3: Generation (Law of Excluded Middle)}
\protect\phantomsection\label{b.2-c3-generation-law-of-excluded-middle}
\textbf{Statement}:
\(\langle\langle a \rangle a \rangle = \langle\ \rangle\)

\textbf{Derivation}: - LHS = \(\langle\langle a \rangle a \rangle\) -
\(= \neg(\neg a \land a)\) (Boolean interpretation) -
\(= \neg(\text{FALSE})\) (contradiction) - \(= \text{TRUE}\) -
\(= \langle\ \rangle\)

This confirms that \(a \lor \neg a = \text{TRUE}\).
\end{block}

\begin{block}{B.3 C6: Iteration (Idempotence)}
\protect\phantomsection\label{b.3-c6-iteration-idempotence}
\textbf{Statement}: \(aa = a\)

\textbf{Derivation}: -
\(\llbracket aa \rrbracket = \llbracket a \rrbracket \land \llbracket a \rrbracket\)
(juxtaposition) - \(= \llbracket a \rrbracket\) (idempotence of AND)
\end{block}
\end{frame}

\begin{frame}{C. Boolean Algebra Correspondence}
\protect\phantomsection\label{c.-boolean-algebra-correspondence}
\begin{block}{C.1 Complete Translation Table}
\protect\phantomsection\label{c.1-complete-translation-table}
{\def\LTcaptype{none} % do not increment counter
\begin{longtable}[]{@{}
  >{\raggedright\arraybackslash}p{(\linewidth - 4\tabcolsep) * \real{0.2571}}
  >{\raggedright\arraybackslash}p{(\linewidth - 4\tabcolsep) * \real{0.4286}}
  >{\raggedright\arraybackslash}p{(\linewidth - 4\tabcolsep) * \real{0.3143}}@{}}
\toprule\noalign{}
\begin{minipage}[b]{\linewidth}\raggedright
Boolean
\end{minipage} & \begin{minipage}[b]{\linewidth}\raggedright
Boundary Form
\end{minipage} & \begin{minipage}[b]{\linewidth}\raggedright
Reduction
\end{minipage} \\
\midrule\noalign{}
\endhead
TRUE & \(\langle\ \rangle\) & canonical \\
FALSE & void & canonical \\
\(\neg a\) & \(\langle a \rangle\) & --- \\
\(a \land b\) & \(ab\) & --- \\
\(a \lor b\) & \(\langle\langle a \rangle\langle b \rangle\rangle\) &
--- \\
\(a \to b\) & \(\langle a\langle b \rangle\rangle\) & --- \\
\(a \leftrightarrow b\) &
\(\langle\langle ab \rangle\langle\langle a \rangle\langle b \rangle\rangle\rangle\)
& --- \\
\(a \oplus b\) (XOR) &
\(\langle\langle\langle a \rangle b \rangle\langle a\langle b \rangle\rangle\rangle\)
& --- \\
\(a\) NAND \(b\) & \(\langle ab \rangle\) & --- \\
\(a\) NOR \(b\) &
\(\langle\langle\langle a \rangle\langle b \rangle\rangle\rangle\) &
--- \\
\bottomrule\noalign{}
\end{longtable}
}
\end{block}

\begin{block}{C.2 NAND Completeness}
\protect\phantomsection\label{c.2-nand-completeness}
All Boolean operations expressible via NAND (\(\langle ab \rangle\)):

\begin{itemize}
\tightlist
\item
  NOT \(a\) = \(a\) NAND \(a\) =
  \(\langle aa \rangle = \langle a \rangle\)
\item
  \(a\) AND \(b\) = NOT(\(a\) NAND \(b\)) =
  \(\langle\langle ab \rangle\rangle = ab\)
\item
  \(a\) OR \(b\) = (NOT \(a\)) NAND (NOT \(b\)) =
  \(\langle\langle a \rangle\langle b \rangle\rangle\)
\end{itemize}
\end{block}
\end{frame}

\begin{frame}[fragile]{D. Reduction Algorithm Details}
\protect\phantomsection\label{d.-reduction-algorithm-details}
\begin{block}{D.1 Pattern Matching}
\protect\phantomsection\label{d.1-pattern-matching}
\textbf{Calling Pattern}:

\begin{verbatim}
Match: Form with is_marked=True, contents=[Form with is_marked=True, contents=[a]]
Result: a
\end{verbatim}

\textbf{Crossing Pattern}:

\begin{verbatim}
Match: Form with multiple simple marks in contents
Result: Single mark with non-mark contents preserved
\end{verbatim}
\end{block}

\begin{block}{D.2 Trace Format}
\protect\phantomsection\label{d.2-trace-format}
Each reduction step records:

\begin{verbatim}
ReductionStep:
  - before: Form (pre-reduction)
  - after: Form (post-reduction)
  - rule: ReductionRule (CALLING | CROSSING | VOID_ELIMINATION)
  - location: str (where rule applied)
\end{verbatim}
\end{block}

\begin{block}{D.3 Termination Proof}
\protect\phantomsection\label{d.3-termination-proof}
\textbf{Theorem}: The reduction algorithm terminates for all well-formed
inputs.

\textbf{Proof}: Define measure
\(\mu(f) = (\text{depth}(f), \text{size}(f))\) with lexicographic
ordering.

\begin{enumerate}
\tightlist
\item
  \textbf{Calling}: Reduces depth by 2 (removes two enclosures)
\item
  \textbf{Crossing}: Reduces size (removes marks)
\item
  \textbf{Void Elimination}: Reduces size (removes void)
\item
  \textbf{Recursive}: Applies to subforms with strictly smaller measure
\end{enumerate}

Each rule application strictly decreases \(\mu(f)\). Since
\(\mu(f) \geq (0, 0)\) and the ordering is well-founded, the algorithm
terminates. \(\square\)
\end{block}
\end{frame}

\begin{frame}{E. Test Coverage Details}
\protect\phantomsection\label{e.-test-coverage-details}
\begin{block}{E.1 Test Categories}
\protect\phantomsection\label{e.1-test-categories}
{\def\LTcaptype{none} % do not increment counter
\begin{longtable}[]{@{}lll@{}}
\toprule\noalign{}
Category & Tests & Coverage Target \\
\midrule\noalign{}
\endhead
Unit (forms.py) & 36 & 95\%+ \\
Unit (reduction.py) & 27 & 95\%+ \\
Unit (algebra.py) & 22 & 90\%+ \\
Integration & 15 & 90\%+ \\
Theorem verification & 12 & 100\% \\
Edge cases & 18 & Comprehensive \\
\bottomrule\noalign{}
\end{longtable}
}
\end{block}

\begin{block}{E.2 Property-Based Testing}
\protect\phantomsection\label{e.2-property-based-testing}
Random form generation tests: - Depth: 1-6 (uniform) - Width: 1-4
(uniform) - Samples: 500 per test run - Seed: 42 (reproducible)

Verified properties: - All forms reduce to canonical - Canonical forms
are stable (re-reduction yields same) - Equivalent forms have equal
canonical forms
\end{block}
\end{frame}

\begin{frame}[fragile]{F. Notation Reference}
\protect\phantomsection\label{f.-notation-reference}
{\def\LTcaptype{none} % do not increment counter
\begin{longtable}[]{@{}lll@{}}
\toprule\noalign{}
Symbol & Meaning & LaTeX \\
\midrule\noalign{}
\endhead
\(\langle\ \rangle\) & Mark (TRUE) &
\texttt{\textbackslash{}langle\textbackslash{}\ \textbackslash{}rangle} \\
\(\emptyset\) & Void (FALSE) & \texttt{\textbackslash{}emptyset} \\
\(\langle a \rangle\) & Enclosure (NOT) &
\texttt{\textbackslash{}langle\ a\ \textbackslash{}rangle} \\
\(ab\) & Juxtaposition (AND) & \texttt{ab} \\
\(\llbracket f \rrbracket\) & Truth value &
\texttt{\textbackslash{}llbracket\ f\ \textbackslash{}rrbracket} \\
\(j\) & Imaginary value & \texttt{j} \\
\bottomrule\noalign{}
\end{longtable}
}
\end{frame}

\begin{frame}[fragile]{G. Implementation Reference}
\protect\phantomsection\label{g.-implementation-reference}
\begin{block}{G.1 Module Structure}
\protect\phantomsection\label{g.1-module-structure}
\begin{verbatim}
project/src/
├── forms.py        # Form class and construction
├── reduction.py    # Reduction engine
├── algebra.py      # Boolean correspondence
├── evaluation.py   # Truth value extraction
├── theorems.py     # Theorem definitions
├── verification.py # Formal verification
├── visualization.py # Diagram generation
└── __init__.py     # Package exports
\end{verbatim}
\end{block}

\begin{block}{G.2 Key APIs}
\protect\phantomsection\label{g.2-key-apis}
\begin{Shaded}
\begin{Highlighting}[]
\CommentTok{\# Form construction}
\NormalTok{make\_void() }\OperatorTok{{-}\textgreater{}}\NormalTok{ Form}
\NormalTok{make\_mark() }\OperatorTok{{-}\textgreater{}}\NormalTok{ Form}
\NormalTok{enclose(form: Form) }\OperatorTok{{-}\textgreater{}}\NormalTok{ Form}
\NormalTok{juxtapose(}\OperatorTok{*}\NormalTok{forms: Form) }\OperatorTok{{-}\textgreater{}}\NormalTok{ Form}

\CommentTok{\# Reduction}
\NormalTok{reduce\_form(form: Form) }\OperatorTok{{-}\textgreater{}}\NormalTok{ Form}
\NormalTok{reduce\_with\_trace(form: Form) }\OperatorTok{{-}\textgreater{}}\NormalTok{ Tuple[Form, ReductionTrace]}

\CommentTok{\# Evaluation}
\NormalTok{evaluate(form: Form) }\OperatorTok{{-}\textgreater{}}\NormalTok{ EvaluationResult}
\NormalTok{truth\_value(form: Form) }\OperatorTok{{-}\textgreater{}} \BuiltInTok{bool}

\CommentTok{\# Verification}
\NormalTok{verify\_axioms() }\OperatorTok{{-}\textgreater{}}\NormalTok{ VerificationReport}
\NormalTok{full\_verification() }\OperatorTok{{-}\textgreater{}}\NormalTok{ VerificationReport}
\end{Highlighting}
\end{Shaded}
\end{block}
\end{frame}

\end{document}
