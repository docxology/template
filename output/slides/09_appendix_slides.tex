% Options for packages loaded elsewhere
\PassOptionsToPackage{unicode}{hyperref}
\PassOptionsToPackage{hyphens}{url}
\documentclass[
  ignorenonframetext,
]{beamer}
\newif\ifbibliography
\usepackage{pgfpages}
\setbeamertemplate{caption}[numbered]
\setbeamertemplate{caption label separator}{: }
\setbeamercolor{caption name}{fg=normal text.fg}
\beamertemplatenavigationsymbolsempty
% remove section numbering
\setbeamertemplate{part page}{
  \centering
  \begin{beamercolorbox}[sep=16pt,center]{part title}
    \usebeamerfont{part title}\insertpart\par
  \end{beamercolorbox}
}
\setbeamertemplate{section page}{
  \centering
  \begin{beamercolorbox}[sep=12pt,center]{section title}
    \usebeamerfont{section title}\insertsection\par
  \end{beamercolorbox}
}
\setbeamertemplate{subsection page}{
  \centering
  \begin{beamercolorbox}[sep=8pt,center]{subsection title}
    \usebeamerfont{subsection title}\insertsubsection\par
  \end{beamercolorbox}
}
% Prevent slide breaks in the middle of a paragraph
\widowpenalties 1 10000
\raggedbottom
\AtBeginPart{
  \frame{\partpage}
}
\AtBeginSection{
  \ifbibliography
  \else
    \frame{\sectionpage}
  \fi
}
\AtBeginSubsection{
  \frame{\subsectionpage}
}
\usepackage{iftex}
\ifPDFTeX
  \usepackage[T1]{fontenc}
  \usepackage[utf8]{inputenc}
  \usepackage{textcomp} % provide euro and other symbols
\else % if luatex or xetex
  \usepackage{unicode-math} % this also loads fontspec
  \defaultfontfeatures{Scale=MatchLowercase}
  \defaultfontfeatures[\rmfamily]{Ligatures=TeX,Scale=1}
\fi
\usepackage{lmodern}
\ifPDFTeX\else
  % xetex/luatex font selection
\fi
% Use upquote if available, for straight quotes in verbatim environments
\IfFileExists{upquote.sty}{\usepackage{upquote}}{}
\IfFileExists{microtype.sty}{% use microtype if available
  \usepackage[]{microtype}
  \UseMicrotypeSet[protrusion]{basicmath} % disable protrusion for tt fonts
}{}
\makeatletter
\@ifundefined{KOMAClassName}{% if non-KOMA class
  \IfFileExists{parskip.sty}{%
    \usepackage{parskip}
  }{% else
    \setlength{\parindent}{0pt}
    \setlength{\parskip}{6pt plus 2pt minus 1pt}}
}{% if KOMA class
  \KOMAoptions{parskip=half}}
\makeatother
\usepackage{color}
\usepackage{fancyvrb}
\newcommand{\VerbBar}{|}
\newcommand{\VERB}{\Verb[commandchars=\\\{\}]}
\DefineVerbatimEnvironment{Highlighting}{Verbatim}{commandchars=\\\{\}}
% Add ',fontsize=\small' for more characters per line
\newenvironment{Shaded}{}{}
\newcommand{\AlertTok}[1]{\textcolor[rgb]{1.00,0.00,0.00}{\textbf{#1}}}
\newcommand{\AnnotationTok}[1]{\textcolor[rgb]{0.38,0.63,0.69}{\textbf{\textit{#1}}}}
\newcommand{\AttributeTok}[1]{\textcolor[rgb]{0.49,0.56,0.16}{#1}}
\newcommand{\BaseNTok}[1]{\textcolor[rgb]{0.25,0.63,0.44}{#1}}
\newcommand{\BuiltInTok}[1]{\textcolor[rgb]{0.00,0.50,0.00}{#1}}
\newcommand{\CharTok}[1]{\textcolor[rgb]{0.25,0.44,0.63}{#1}}
\newcommand{\CommentTok}[1]{\textcolor[rgb]{0.38,0.63,0.69}{\textit{#1}}}
\newcommand{\CommentVarTok}[1]{\textcolor[rgb]{0.38,0.63,0.69}{\textbf{\textit{#1}}}}
\newcommand{\ConstantTok}[1]{\textcolor[rgb]{0.53,0.00,0.00}{#1}}
\newcommand{\ControlFlowTok}[1]{\textcolor[rgb]{0.00,0.44,0.13}{\textbf{#1}}}
\newcommand{\DataTypeTok}[1]{\textcolor[rgb]{0.56,0.13,0.00}{#1}}
\newcommand{\DecValTok}[1]{\textcolor[rgb]{0.25,0.63,0.44}{#1}}
\newcommand{\DocumentationTok}[1]{\textcolor[rgb]{0.73,0.13,0.13}{\textit{#1}}}
\newcommand{\ErrorTok}[1]{\textcolor[rgb]{1.00,0.00,0.00}{\textbf{#1}}}
\newcommand{\ExtensionTok}[1]{#1}
\newcommand{\FloatTok}[1]{\textcolor[rgb]{0.25,0.63,0.44}{#1}}
\newcommand{\FunctionTok}[1]{\textcolor[rgb]{0.02,0.16,0.49}{#1}}
\newcommand{\ImportTok}[1]{\textcolor[rgb]{0.00,0.50,0.00}{\textbf{#1}}}
\newcommand{\InformationTok}[1]{\textcolor[rgb]{0.38,0.63,0.69}{\textbf{\textit{#1}}}}
\newcommand{\KeywordTok}[1]{\textcolor[rgb]{0.00,0.44,0.13}{\textbf{#1}}}
\newcommand{\NormalTok}[1]{#1}
\newcommand{\OperatorTok}[1]{\textcolor[rgb]{0.40,0.40,0.40}{#1}}
\newcommand{\OtherTok}[1]{\textcolor[rgb]{0.00,0.44,0.13}{#1}}
\newcommand{\PreprocessorTok}[1]{\textcolor[rgb]{0.74,0.48,0.00}{#1}}
\newcommand{\RegionMarkerTok}[1]{#1}
\newcommand{\SpecialCharTok}[1]{\textcolor[rgb]{0.25,0.44,0.63}{#1}}
\newcommand{\SpecialStringTok}[1]{\textcolor[rgb]{0.73,0.40,0.53}{#1}}
\newcommand{\StringTok}[1]{\textcolor[rgb]{0.25,0.44,0.63}{#1}}
\newcommand{\VariableTok}[1]{\textcolor[rgb]{0.10,0.09,0.49}{#1}}
\newcommand{\VerbatimStringTok}[1]{\textcolor[rgb]{0.25,0.44,0.63}{#1}}
\newcommand{\WarningTok}[1]{\textcolor[rgb]{0.38,0.63,0.69}{\textbf{\textit{#1}}}}
\setlength{\emergencystretch}{3em} % prevent overfull lines
\providecommand{\tightlist}{%
  \setlength{\itemsep}{0pt}\setlength{\parskip}{0pt}}
\usepackage{bookmark}
\IfFileExists{xurl.sty}{\usepackage{xurl}}{} % add URL line breaks if available
\urlstyle{same}
\hypersetup{
  hidelinks,
  pdfcreator={LaTeX via pandoc}}

\author{\texorpdfstring{}{}}
\date{}

\begin{document}

\begin{frame}[fragile]{Appendix}
\protect\phantomsection\label{sec:appendix}
This appendix provides additional technical details supporting the main
results.

\begin{block}{A. Database Schema Details}
\protect\phantomsection\label{a.-database-schema-details}
\begin{block}{A.1 Complete Table Schemas}
\protect\phantomsection\label{a.1-complete-table-schemas}
\begin{block}{Ways Table Schema}
\protect\phantomsection\label{ways-table-schema}
\begin{Shaded}
\begin{Highlighting}[]
\KeywordTok{CREATE} \KeywordTok{TABLE}\NormalTok{ ways (}
\NormalTok{    way TEXT }\KeywordTok{NOT} \KeywordTok{NULL}\NormalTok{,}
\NormalTok{    dialoguewith TEXT }\KeywordTok{NOT} \KeywordTok{NULL}\NormalTok{,}
\NormalTok{    dialoguetype TEXT }\KeywordTok{NOT} \KeywordTok{NULL}\NormalTok{,}
\NormalTok{    dialoguetypetype TEXT }\KeywordTok{NOT} \KeywordTok{NULL}\NormalTok{,}
    \KeywordTok{ID} \DataTypeTok{INTEGER} \KeywordTok{PRIMARY} \KeywordTok{KEY}\NormalTok{ AUTOINCREMENT,}
\NormalTok{    wayurl TEXT }\KeywordTok{NOT} \KeywordTok{NULL}\NormalTok{,}
\NormalTok{    examples TEXT }\KeywordTok{NOT} \KeywordTok{NULL}\NormalTok{,}
\NormalTok{    dialoguetypetypetype TEXT }\KeywordTok{NOT} \KeywordTok{NULL}\NormalTok{,}
\NormalTok{    mene TEXT }\KeywordTok{NOT} \KeywordTok{NULL}\NormalTok{,}
\NormalTok{    Dievas TEXT }\KeywordTok{NOT} \KeywordTok{NULL}\NormalTok{,}
\NormalTok{    comments TEXT }\KeywordTok{NOT} \KeywordTok{NULL}\NormalTok{,}
\NormalTok{    laikinas TEXT }\KeywordTok{NOT} \KeywordTok{NULL}
\NormalTok{);}
\end{Highlighting}
\end{Shaded}
\end{block}

\begin{block}{Rooms Table Schema}
\protect\phantomsection\label{rooms-table-schema}
\begin{Shaded}
\begin{Highlighting}[]
\KeywordTok{CREATE} \KeywordTok{TABLE}\NormalTok{ rooms (}
\NormalTok{    santrumpa TEXT }\KeywordTok{NOT} \KeywordTok{NULL} \KeywordTok{PRIMARY} \KeywordTok{KEY}\NormalTok{,}
\NormalTok{    savoka TEXT }\KeywordTok{NOT} \KeywordTok{NULL}\NormalTok{,}
\NormalTok{    issiaiskinimas TEXT }\KeywordTok{NOT} \KeywordTok{NULL}\NormalTok{,}
    \CommentTok{{-}{-} Additional fields for ordering and relationships}
\NormalTok{);}
\end{Highlighting}
\end{Shaded}
\end{block}
\end{block}

\begin{block}{A.2 Index Definitions}
\protect\phantomsection\label{a.2-index-definitions}
Key indexes for performance:

\begin{itemize}
\tightlist
\item
  Index on \texttt{way} for way lookups
\item
  Index on \texttt{mene} for room-based queries
\item
  Index on \texttt{dialoguetype} for type filtering
\item
  Index on \texttt{dialoguewith} for partner analysis
\end{itemize}
\end{block}
\end{block}

\begin{block}{B. Network Analysis Algorithms}
\protect\phantomsection\label{b.-network-analysis-algorithms}
\begin{block}{B.1 Actual Network Metrics}
\protect\phantomsection\label{b.1-actual-network-metrics}
The network analysis was performed using NetworkX on the complete ways
database:

\textbf{Network Structure:} - \textbf{Nodes}: 210 ways - \textbf{Edges}:
1,290 connections - \textbf{Average degree}: 12.29 connections per way -
\textbf{Network density}: 0.058 (5.8\% of possible edges present) -
\textbf{Clustering coefficient}: 0.886 (high local clustering) -
\textbf{Connected components}: Multiple components detected -
\textbf{Largest component}: Contains majority of ways

\textbf{Centrality Metrics:} - \textbf{Degree centrality}: Range from
0.0 to 0.162 (way 84, 156, 211 with highest degree: 34) -
\textbf{Betweenness centrality}: Identifies bridge ways connecting
different communities - \textbf{Closeness centrality}: Measures average
distance to all other ways - \textbf{Eigenvector centrality}: Identifies
ways connected to highly central ways

\textbf{Community Detection:} - Modularity-based community detection
reveals natural clusters - Communities correspond to room assignments
and dialogue types - Largest communities align with most populated rooms
(B2, C4, R)
\end{block}

\begin{block}{B.2 Graph Construction Implementation}
\protect\phantomsection\label{b.2-graph-construction-implementation}
The network is constructed using \texttt{WaysNetworkAnalyzer} with three
edge types:

\begin{Shaded}
\begin{Highlighting}[]
\ImportTok{from}\NormalTok{ collections }\ImportTok{import}\NormalTok{ defaultdict}
\ImportTok{import}\NormalTok{ networkx }\ImportTok{as}\NormalTok{ nx}
\ImportTok{from}\NormalTok{ src.models }\ImportTok{import}\NormalTok{ Way}

\KeywordTok{def}\NormalTok{ \_build\_ways\_network(ways: List[Way]) }\OperatorTok{{-}\textgreater{}}\NormalTok{ nx.Graph:}
    \CommentTok{"""Build network graph from ways data.}
\CommentTok{    }
\CommentTok{    Edges are created based on:}
\CommentTok{    1. Same room (weight=1.0, edge\_type=\textquotesingle{}same\_room\textquotesingle{})}
\CommentTok{    2. Same dialogue partner (weight=0.8, edge\_type=\textquotesingle{}same\_partner\textquotesingle{})}
\CommentTok{    3. Same dialogue type (weight=0.6, edge\_type=\textquotesingle{}same\_type\textquotesingle{})}
\CommentTok{    """}
\NormalTok{    G }\OperatorTok{=}\NormalTok{ nx.Graph()}
    
    \CommentTok{\# Add nodes with attributes}
    \ControlFlowTok{for}\NormalTok{ way }\KeywordTok{in}\NormalTok{ ways:}
\NormalTok{        G.add\_node(way.}\BuiltInTok{id}\NormalTok{, }
\NormalTok{                   way\_text}\OperatorTok{=}\NormalTok{way.way,}
\NormalTok{                   room}\OperatorTok{=}\NormalTok{way.mene,}
\NormalTok{                   dialogue\_type}\OperatorTok{=}\NormalTok{way.dialoguetype,}
\NormalTok{                   dialogue\_partner}\OperatorTok{=}\NormalTok{way.dialoguewith)}
    
    \CommentTok{\# Group ways by room}
\NormalTok{    room\_ways }\OperatorTok{=}\NormalTok{ defaultdict(}\BuiltInTok{list}\NormalTok{)}
    \ControlFlowTok{for}\NormalTok{ way }\KeywordTok{in}\NormalTok{ ways:}
\NormalTok{        room\_ways[way.mene].append(way.}\BuiltInTok{id}\NormalTok{)}
    
    \CommentTok{\# Add room edges (highest weight)}
    \ControlFlowTok{for}\NormalTok{ room, way\_ids }\KeywordTok{in}\NormalTok{ room\_ways.items():}
        \ControlFlowTok{if} \BuiltInTok{len}\NormalTok{(way\_ids) }\OperatorTok{\textgreater{}} \DecValTok{1}\NormalTok{:}
            \ControlFlowTok{for}\NormalTok{ i, way1\_id }\KeywordTok{in} \BuiltInTok{enumerate}\NormalTok{(way\_ids):}
                \ControlFlowTok{for}\NormalTok{ way2\_id }\KeywordTok{in}\NormalTok{ way\_ids[i}\OperatorTok{+}\DecValTok{1}\NormalTok{:]:}
\NormalTok{                    G.add\_edge(way1\_id, way2\_id, }
\NormalTok{                              edge\_type}\OperatorTok{=}\StringTok{\textquotesingle{}same\_room\textquotesingle{}}\NormalTok{,}
\NormalTok{                              room}\OperatorTok{=}\NormalTok{room,}
\NormalTok{                              weight}\OperatorTok{=}\FloatTok{1.0}\NormalTok{)}
    
    \CommentTok{\# Add partner edges (medium weight)}
\NormalTok{    partner\_ways }\OperatorTok{=}\NormalTok{ defaultdict(}\BuiltInTok{list}\NormalTok{)}
    \ControlFlowTok{for}\NormalTok{ way }\KeywordTok{in}\NormalTok{ ways:}
\NormalTok{        partner\_ways[way.dialoguewith].append(way.}\BuiltInTok{id}\NormalTok{)}
    
    \ControlFlowTok{for}\NormalTok{ partner, way\_ids }\KeywordTok{in}\NormalTok{ partner\_ways.items():}
        \ControlFlowTok{if} \BuiltInTok{len}\NormalTok{(way\_ids) }\OperatorTok{\textgreater{}} \DecValTok{1}\NormalTok{:}
            \ControlFlowTok{for}\NormalTok{ i, way1\_id }\KeywordTok{in} \BuiltInTok{enumerate}\NormalTok{(way\_ids):}
                \ControlFlowTok{for}\NormalTok{ way2\_id }\KeywordTok{in}\NormalTok{ way\_ids[i}\OperatorTok{+}\DecValTok{1}\NormalTok{:]:}
                    \ControlFlowTok{if} \KeywordTok{not}\NormalTok{ G.has\_edge(way1\_id, way2\_id):}
\NormalTok{                        G.add\_edge(way1\_id, way2\_id,}
\NormalTok{                                  edge\_type}\OperatorTok{=}\StringTok{\textquotesingle{}same\_partner\textquotesingle{}}\NormalTok{,}
\NormalTok{                                  partner}\OperatorTok{=}\NormalTok{partner,}
\NormalTok{                                  weight}\OperatorTok{=}\FloatTok{0.8}\NormalTok{)}
    
    \CommentTok{\# Add type edges (lowest weight)}
\NormalTok{    type\_ways }\OperatorTok{=}\NormalTok{ defaultdict(}\BuiltInTok{list}\NormalTok{)}
    \ControlFlowTok{for}\NormalTok{ way }\KeywordTok{in}\NormalTok{ ways:}
\NormalTok{        type\_ways[way.dialoguetype].append(way.}\BuiltInTok{id}\NormalTok{)}
    
    \ControlFlowTok{for}\NormalTok{ dtype, way\_ids }\KeywordTok{in}\NormalTok{ type\_ways.items():}
        \ControlFlowTok{if} \BuiltInTok{len}\NormalTok{(way\_ids) }\OperatorTok{\textgreater{}} \DecValTok{1}\NormalTok{:}
            \ControlFlowTok{for}\NormalTok{ i, way1\_id }\KeywordTok{in} \BuiltInTok{enumerate}\NormalTok{(way\_ids):}
                \ControlFlowTok{for}\NormalTok{ way2\_id }\KeywordTok{in}\NormalTok{ way\_ids[i}\OperatorTok{+}\DecValTok{1}\NormalTok{:]:}
                    \ControlFlowTok{if} \KeywordTok{not}\NormalTok{ G.has\_edge(way1\_id, way2\_id):}
\NormalTok{                        G.add\_edge(way1\_id, way2\_id,}
\NormalTok{                                  edge\_type}\OperatorTok{=}\StringTok{\textquotesingle{}same\_type\textquotesingle{}}\NormalTok{,}
\NormalTok{                                  dialogue\_type}\OperatorTok{=}\NormalTok{dtype,}
\NormalTok{                                  weight}\OperatorTok{=}\FloatTok{0.6}\NormalTok{)}
    
    \ControlFlowTok{return}\NormalTok{ G}
\end{Highlighting}
\end{Shaded}
\end{block}

\begin{block}{B.3 Centrality Computation}
\protect\phantomsection\label{b.3-centrality-computation}
Centrality metrics are computed using NetworkX functions within
\texttt{WaysNetworkAnalyzer.compute\_centrality\_metrics()}:

\begin{itemize}
\tightlist
\item
  \texttt{nx.degree\_centrality(G)}: Normalized degree (0-1 range)
\item
  \texttt{nx.betweenness\_centrality(G)}: Bridge identification
\item
  \texttt{nx.closeness\_centrality(G)}: Average path length
\item
  \texttt{nx.eigenvector\_centrality(G,\ max\_iter=1000)}: Influence
  propagation
\item
  \texttt{nx.average\_clustering(G)}: Local clustering coefficient
\end{itemize}

The implementation handles edge cases (disconnected graphs, single
nodes) and returns a \texttt{NetworkMetrics} dataclass with all computed
values.
\end{block}
\end{block}

\begin{block}{C. Statistical Analysis Formulas}
\protect\phantomsection\label{c.-statistical-analysis-formulas}
\begin{block}{C.1 Distribution Metrics}
\protect\phantomsection\label{c.1-distribution-metrics}
For a categorical variable with \(k\) categories:

\begin{equation}\label{eq:entropy_appendix}
H(X) = -\sum_{i=1}^{k} p_i \log_2(p_i)
\end{equation}

where \(p_i\) is the proportion in category \(i\).
\end{block}

\begin{block}{C.2 Association Measures}
\protect\phantomsection\label{c.2-association-measures}
For cross-tabulation analysis:

\begin{equation}\label{eq:cramers_v}
V = \sqrt{\frac{\chi^2}{n \min(r-1, c-1)}}
\end{equation}

where \(\chi^2\) is the chi-square statistic, \(n\) is sample size, and
\(r\), \(c\) are row and column counts.
\end{block}
\end{block}

\begin{block}{D. Visualization Specifications}
\protect\phantomsection\label{d.-visualization-specifications}
\begin{block}{D.1 Network Visualization Parameters}
\protect\phantomsection\label{d.1-network-visualization-parameters}
\begin{itemize}
\tightlist
\item
  \textbf{Layout Algorithm}: Force-directed (Fruchterman-Reingold)
\item
  \textbf{Node Size}: Proportional to centrality score
\item
  \textbf{Node Color}: By dialogue type
\item
  \textbf{Edge Width}: By relationship strength
\item
  \textbf{Edge Color}: By relationship type
\end{itemize}
\end{block}

\begin{block}{D.2 Color Schemes}
\protect\phantomsection\label{d.2-color-schemes}
\begin{itemize}
\tightlist
\item
  \textbf{Dialogue Types}:

  \begin{itemize}
  \tightlist
  \item
    Absolute: Blue shades
  \item
    Relative: Green shades
  \item
    Embrace God: Purple shades
  \end{itemize}
\item
  \textbf{Rooms}: Sequential color scheme for 24 rooms
\end{itemize}
\end{block}
\end{block}

\begin{block}{E. Data Processing Pipeline}
\protect\phantomsection\label{e.-data-processing-pipeline}
\begin{block}{E.1 Database Initialization}
\protect\phantomsection\label{e.1-database-initialization}
\begin{Shaded}
\begin{Highlighting}[]
\ImportTok{from}\NormalTok{ src.database }\ImportTok{import}\NormalTok{ WaysDatabase, initialize\_database}

\KeywordTok{def}\NormalTok{ setup\_ways\_database(mysql\_dump\_path: }\BuiltInTok{str} \OperatorTok{=} \VariableTok{None}\NormalTok{, }
\NormalTok{                        sqlite\_path: }\BuiltInTok{str} \OperatorTok{=} \StringTok{"project/db/ways.db"}\NormalTok{) }\OperatorTok{{-}\textgreater{}}\NormalTok{ WaysDatabase:}
    \CommentTok{"""Initialize SQLite database from MySQL dump or existing database.}
\CommentTok{    }
\CommentTok{    Args:}
\CommentTok{        mysql\_dump\_path: Path to MySQL dump file (optional)}
\CommentTok{        sqlite\_path: Path to SQLite database file}
\CommentTok{        }
\CommentTok{    Returns:}
\CommentTok{        Initialized WaysDatabase instance}
\CommentTok{    """}
    \ControlFlowTok{if}\NormalTok{ mysql\_dump\_path:}
        \CommentTok{\# Convert MySQL dump to SQLite}
\NormalTok{        initialize\_database(mysql\_dump\_path, sqlite\_path)}
    
    \CommentTok{\# Return database connection}
\NormalTok{    db }\OperatorTok{=}\NormalTok{ WaysDatabase(sqlite\_path)}
    
    \CommentTok{\# Validate database integrity}
\NormalTok{    stats }\OperatorTok{=}\NormalTok{ db.get\_way\_statistics()}
    \ControlFlowTok{assert}\NormalTok{ stats[}\StringTok{\textquotesingle{}total\_ways\textquotesingle{}}\NormalTok{] }\OperatorTok{\textgreater{}} \DecValTok{0}\NormalTok{, }\StringTok{"Database must contain ways"}
    
    \ControlFlowTok{return}\NormalTok{ db}
\end{Highlighting}
\end{Shaded}
\end{block}

\begin{block}{E.2 Data Access and Querying}
\protect\phantomsection\label{e.2-data-access-and-querying}
\begin{Shaded}
\begin{Highlighting}[]
\ImportTok{from}\NormalTok{ src.database }\ImportTok{import}\NormalTok{ WaysDatabase}
\ImportTok{from}\NormalTok{ src.sql\_queries }\ImportTok{import}\NormalTok{ WaysSQLQueries}
\ImportTok{from}\NormalTok{ src.models }\ImportTok{import}\NormalTok{ Way}

\KeywordTok{def}\NormalTok{ query\_ways\_data(db\_path: }\BuiltInTok{str} \OperatorTok{=} \StringTok{"project/db/ways.db"}\NormalTok{) }\OperatorTok{{-}\textgreater{}}\NormalTok{ Dict[}\BuiltInTok{str}\NormalTok{, Any]:}
    \CommentTok{"""Query ways data using SQL queries module.}
\CommentTok{    }
\CommentTok{    Returns:}
\CommentTok{        Dictionary with ways, rooms, and statistics}
\CommentTok{    """}
\NormalTok{    db }\OperatorTok{=}\NormalTok{ WaysDatabase(db\_path)}
\NormalTok{    queries }\OperatorTok{=}\NormalTok{ WaysSQLQueries(db\_path)}
    
    \CommentTok{\# Get all ways}
\NormalTok{    \_, ways\_data }\OperatorTok{=}\NormalTok{ queries.get\_all\_ways\_sql()}
\NormalTok{    ways }\OperatorTok{=}\NormalTok{ [Way.from\_sqlalchemy(row) }\ControlFlowTok{for}\NormalTok{ row }\KeywordTok{in}\NormalTok{ ways\_data]}
    
    \CommentTok{\# Get room distribution}
\NormalTok{    \_, room\_counts }\OperatorTok{=}\NormalTok{ queries.count\_ways\_by\_room\_sql()}
\NormalTok{    room\_dist }\OperatorTok{=}\NormalTok{ \{room: count }\ControlFlowTok{for}\NormalTok{ room, count }\KeywordTok{in}\NormalTok{ room\_counts\}}
    
    \CommentTok{\# Get type distribution}
\NormalTok{    \_, type\_counts }\OperatorTok{=}\NormalTok{ queries.count\_ways\_by\_type\_sql()}
\NormalTok{    type\_dist }\OperatorTok{=}\NormalTok{ \{dtype: count }\ControlFlowTok{for}\NormalTok{ dtype, count }\KeywordTok{in}\NormalTok{ type\_counts\}}
    
    \CommentTok{\# Get cross{-}tabulation}
\NormalTok{    \_, crosstab }\OperatorTok{=}\NormalTok{ queries.cross\_tabulate\_type\_room\_sql()}
    
    \ControlFlowTok{return}\NormalTok{ \{}
        \StringTok{\textquotesingle{}ways\textquotesingle{}}\NormalTok{: ways,}
        \StringTok{\textquotesingle{}room\_distribution\textquotesingle{}}\NormalTok{: room\_dist,}
        \StringTok{\textquotesingle{}type\_distribution\textquotesingle{}}\NormalTok{: type\_dist,}
        \StringTok{\textquotesingle{}crosstab\textquotesingle{}}\NormalTok{: crosstab,}
        \StringTok{\textquotesingle{}total\_ways\textquotesingle{}}\NormalTok{: }\BuiltInTok{len}\NormalTok{(ways)}
\NormalTok{    \}}
\end{Highlighting}
\end{Shaded}
\end{block}

\begin{block}{E.3 Analysis Script}
\protect\phantomsection\label{e.3-analysis-script}
The comprehensive analysis script integrates multiple analysis modules:

\begin{Shaded}
\begin{Highlighting}[]
\ImportTok{from}\NormalTok{ src.ways\_analysis }\ImportTok{import}\NormalTok{ WaysAnalyzer}
\ImportTok{from}\NormalTok{ src.network\_analysis }\ImportTok{import}\NormalTok{ WaysNetworkAnalyzer}
\ImportTok{from}\NormalTok{ src.database }\ImportTok{import}\NormalTok{ WaysDatabase}
\ImportTok{from}\NormalTok{ src.sql\_queries }\ImportTok{import}\NormalTok{ WaysSQLQueries}

\KeywordTok{def}\NormalTok{ analyze\_ways\_comprehensive(db\_path: }\BuiltInTok{str} \OperatorTok{=} \VariableTok{None}\NormalTok{) }\OperatorTok{{-}\textgreater{}}\NormalTok{ Dict[}\BuiltInTok{str}\NormalTok{, Any]:}
    \CommentTok{"""Comprehensive analysis of ways database.}
\CommentTok{    }
\CommentTok{    Args:}
\CommentTok{        db\_path: Optional path to SQLite database}
\CommentTok{        }
\CommentTok{    Returns:}
\CommentTok{        Dictionary containing all analysis results}
\CommentTok{    """}
    \CommentTok{\# Initialize analyzers}
\NormalTok{    analyzer }\OperatorTok{=}\NormalTok{ WaysAnalyzer(db\_path)}
\NormalTok{    network\_analyzer }\OperatorTok{=}\NormalTok{ WaysNetworkAnalyzer(db\_path)}
\NormalTok{    db }\OperatorTok{=}\NormalTok{ WaysDatabase(db\_path)}
\NormalTok{    queries }\OperatorTok{=}\NormalTok{ WaysSQLQueries(db\_path)}
    
    \CommentTok{\# Distribution analysis}
\NormalTok{    characterization }\OperatorTok{=}\NormalTok{ analyzer.characterize\_ways()}
\NormalTok{    type\_dist }\OperatorTok{=}\NormalTok{ characterization.dialogue\_types}
\NormalTok{    room\_dist }\OperatorTok{=}\NormalTok{ characterization.room\_distribution}
    
    \CommentTok{\# Network analysis}
\NormalTok{    network }\OperatorTok{=}\NormalTok{ network\_analyzer.build\_ways\_network()}
\NormalTok{    metrics }\OperatorTok{=}\NormalTok{ network\_analyzer.compute\_centrality\_metrics()}
\NormalTok{    central\_ways }\OperatorTok{=}\NormalTok{ network\_analyzer.find\_central\_ways()}
    
    \CommentTok{\# Statistical analysis}
\NormalTok{    \_, crosstab\_results }\OperatorTok{=}\NormalTok{ queries.cross\_tabulate\_type\_room\_sql()}
\NormalTok{    crosstab }\OperatorTok{=}\NormalTok{ \{\}}
    \ControlFlowTok{for}\NormalTok{ dtype, room, count }\KeywordTok{in}\NormalTok{ crosstab\_results:}
        \ControlFlowTok{if}\NormalTok{ dtype }\KeywordTok{not} \KeywordTok{in}\NormalTok{ crosstab:}
\NormalTok{            crosstab[dtype] }\OperatorTok{=}\NormalTok{ \{\}}
\NormalTok{        crosstab[dtype][room] }\OperatorTok{=}\NormalTok{ count}
    
    \ControlFlowTok{return}\NormalTok{ \{}
        \StringTok{\textquotesingle{}characterization\textquotesingle{}}\NormalTok{: \{}
            \StringTok{\textquotesingle{}total\_ways\textquotesingle{}}\NormalTok{: characterization.total\_ways,}
            \StringTok{\textquotesingle{}room\_diversity\textquotesingle{}}\NormalTok{: characterization.room\_diversity,}
            \StringTok{\textquotesingle{}type\_diversity\textquotesingle{}}\NormalTok{: characterization.type\_diversity,}
            \StringTok{\textquotesingle{}most\_common\_room\textquotesingle{}}\NormalTok{: characterization.most\_common\_room,}
            \StringTok{\textquotesingle{}most\_common\_type\textquotesingle{}}\NormalTok{: characterization.most\_common\_type}
\NormalTok{        \},}
        \StringTok{\textquotesingle{}network\_metrics\textquotesingle{}}\NormalTok{: \{}
            \StringTok{\textquotesingle{}node\_count\textquotesingle{}}\NormalTok{: metrics.node\_count,}
            \StringTok{\textquotesingle{}edge\_count\textquotesingle{}}\NormalTok{: metrics.edge\_count,}
            \StringTok{\textquotesingle{}density\textquotesingle{}}\NormalTok{: metrics.density,}
            \StringTok{\textquotesingle{}average\_degree\textquotesingle{}}\NormalTok{: metrics.average\_degree,}
            \StringTok{\textquotesingle{}clustering\_coefficient\textquotesingle{}}\NormalTok{: metrics.clustering\_coefficient}
\NormalTok{        \},}
        \StringTok{\textquotesingle{}central\_ways\textquotesingle{}}\NormalTok{: \{}
            \StringTok{\textquotesingle{}by\_degree\textquotesingle{}}\NormalTok{: central\_ways.by\_degree[:}\DecValTok{10}\NormalTok{],}
            \StringTok{\textquotesingle{}by\_betweenness\textquotesingle{}}\NormalTok{: central\_ways.by\_betweenness[:}\DecValTok{10}\NormalTok{]}
\NormalTok{        \},}
        \StringTok{\textquotesingle{}crosstab\textquotesingle{}}\NormalTok{: crosstab}
\NormalTok{    \}}
\end{Highlighting}
\end{Shaded}
\end{block}
\end{block}

\begin{block}{F. Validation Procedures}
\protect\phantomsection\label{f.-validation-procedures}
\begin{block}{F.1 Data Quality Checks}
\protect\phantomsection\label{f.1-data-quality-checks}
\begin{enumerate}
\tightlist
\item
  \textbf{Completeness Check}: Verify all required fields present
\item
  \textbf{Consistency Check}: Check for conflicting assignments
\item
  \textbf{Referential Integrity}: Validate foreign key relationships
\item
  \textbf{Encoding Check}: Verify UTF-8 encoding
\end{enumerate}
\end{block}

\begin{block}{F.2 Analysis Validation}
\protect\phantomsection\label{f.2-analysis-validation}
\begin{enumerate}
\tightlist
\item
  \textbf{Reproducibility}: Fixed random seeds, deterministic algorithms
\item
  \textbf{Sensitivity}: Test with missing data, parameter variations
\item
  \textbf{Robustness}: Verify results stable under different conditions
\item
  \textbf{Cross-Validation}: Validate findings across data subsets
\end{enumerate}
\end{block}
\end{block}

\begin{block}{G. Computational Environment}
\protect\phantomsection\label{g.-computational-environment}
\begin{block}{G.1 Software Versions}
\protect\phantomsection\label{g.1-software-versions}
\begin{itemize}
\tightlist
\item
  Python 3.10+
\item
  SQLite 3.x
\item
  NetworkX 2.x+
\item
  Pandas 1.x+
\item
  Matplotlib 3.x+
\item
  NumPy 1.x+
\end{itemize}
\end{block}

\begin{block}{G.2 Hardware Requirements}
\protect\phantomsection\label{g.2-hardware-requirements}
\begin{itemize}
\tightlist
\item
  Minimum: 4GB RAM, single core
\item
  Recommended: 8GB+ RAM, multi-core
\item
  Storage: \textasciitilde100MB for database and outputs
\end{itemize}
\end{block}
\end{block}

\begin{block}{H. Additional Tables and Figures}
\protect\phantomsection\label{h.-additional-tables-and-figures}
\begin{block}{H.1 Extended Distribution Tables}
\protect\phantomsection\label{h.1-extended-distribution-tables}
\begin{block}{Complete Room Distribution}
\protect\phantomsection\label{complete-room-distribution}
The complete distribution of all 24 rooms in the House of Knowledge:

\begin{table}[h]
\centering
\small
\begin{tabular}{|l|c|c||l|c|c|}
\hline
\textbf{Room} & \textbf{Count} & \textbf{\%} & \textbf{Room} & \textbf{Count} & \textbf{\%} \\
\hline
B2 & 23 & 11.0\% & C2 & 7 & 3.3\% \\
C4 & 17 & 8.1\% & B4 & 7 & 3.3\% \\
R & 16 & 7.6\% & 1 & 7 & 3.3\% \\
32 & 13 & 6.2\% & F & 6 & 2.9\% \\
C3 & 13 & 6.2\% & 20 & 5 & 2.4\% \\
BB & 12 & 5.7\% & A & 4 & 1.9\% \\
CB & 10 & 4.8\% & 30 & 4 & 1.9\% \\
21 & 9 & 4.3\% & BC & 3 & 1.4\% \\
B3 & 9 & 4.3\% & B & 1 & 0.5\% \\
CC & 9 & 4.3\% & C & 1 & 0.5\% \\
O & 9 & 4.3\% & & & \\
T & 9 & 4.3\% & & & \\
10 & 8 & 3.8\% & & & \\
31 & 8 & 3.8\% & & & \\
\hline
\multicolumn{3}{|c||}{\textbf{Total}} & \multicolumn{3}{c|}{\textbf{210 (100\%)}} \\
\hline
\end{tabular}
\caption{Complete distribution of ways across all 24 rooms}
\label{tab:complete_room_distribution}
\end{table}
\end{block}

\begin{block}{Complete Dialogue Type Distribution}
\protect\phantomsection\label{complete-dialogue-type-distribution}
The complete distribution of all 38 dialogue types (presented in two
parts):

\begin{table}[h]
\centering
\small
\begin{tabular}{|l|c|c||l|c|c|}
\hline
\textbf{Type} & \textbf{Count} & \textbf{\%} & \textbf{Type} & \textbf{Count} & \textbf{\%} \\
\hline
goodness & 15 & 7.1\% & my mind & 7 & 3.3\% \\
other & 15 & 7.1\% & opposing view & 7 & 3.3\% \\
regularity & 11 & 5.2\% & unknown & 7 & 3.3\% \\
I & 9 & 4.3\% & conviction & 5 & 2.4\% \\
answer & 9 & 4.3\% & interlocutor & 5 & 2.4\% \\
knowledge & 8 & 3.8\% & my fate & 5 & 2.4\% \\
life & 8 & 3.8\% & my knowledge & 5 & 2.4\% \\
mind & 8 & 3.8\% & my purpose & 5 & 2.4\% \\
God & 6 & 2.9\% & wholeness & 5 & 2.4\% \\
divineness & 6 & 2.9\% & behavior & 4 & 1.9\% \\
purpose & 6 & 2.9\% & capability & 4 & 1.9\% \\
solution & 6 & 2.9\% & God's perspective & 4 & 1.9\% \\
God's perspective & 4 & 1.9\% & inspiration & 4 & 1.9\% \\
possibilities & 4 & 1.9\% & possibility & 4 & 1.9\% \\
self-check & 4 & 1.9\% & structure & 4 & 1.9\% \\
\hline
\end{tabular}
\caption{Dialogue type distribution (Part 1: Top 19 types)}
\label{tab:type_distribution_part1}
\end{table}

\begin{table}[h]
\centering
\small
\begin{tabular}{|l|c|c||l|c|c|}
\hline
\textbf{Type} & \textbf{Count} & \textbf{\%} & \textbf{Type} & \textbf{Count} & \textbf{\%} \\
\hline
conditionality & 3 & 1.4\% & invalidity & 2 & 1.0\% \\
example & 3 & 1.4\% & misfortune & 2 & 1.0\% \\
given & 3 & 1.4\% & phenomenon & 2 & 1.0\% \\
impossibility & 3 & 1.4\% & depths & 1 & 0.5\% \\
& & & infinity & 1 & 0.5\% \\
\hline
\multicolumn{3}{|c||}{\textbf{Total}} & \multicolumn{3}{c|}{\textbf{210 (100\%)}} \\
\hline
\end{tabular}
\caption{Dialogue type distribution (Part 2: Remaining 19 types)}
\label{tab:type_distribution_part2}
\end{table}
\end{block}

\begin{block}{Top Cross-Tabulation Combinations}
\protect\phantomsection\label{top-cross-tabulation-combinations}
The most frequent dialogue type × room combinations:

\begin{table}[h]
\centering
\small
\begin{tabular}{|l|l|c|}
\hline
\textbf{Dialogue Type} & \textbf{Room} & \textbf{Count} \\
\hline
goodness & B2 & 3 \\
goodness & R & 3 \\
goodness & T & 2 \\
I & O & 9 \\
answer & 1 & 4 \\
answer & 32 & 2 \\
knowledge & CC & 8 \\
divineness & C4 & 6 \\
God & B2 & 5 \\
God's perspective & R & 4 \\
capability & B3 & 4 \\
inspiration & B4 & 4 \\
conviction & B3 & 5 \\
\hline
\end{tabular}
\caption{Top dialogue type × room combinations (count ≥ 2)}
\label{tab:top_crosstab}
\end{table}
\end{block}
\end{block}

\begin{block}{H.2 Extended Network Visualizations}
\protect\phantomsection\label{h.2-extended-network-visualizations}
\begin{block}{Ways Network Visualization}
\protect\phantomsection\label{ways-network-visualization}
Figure \ref{fig:ways_network} shows the complete network graph of 210
ways with 1,290 edges. The visualization uses a force-directed layout
(Fruchterman-Reingold algorithm) with: - \textbf{Node colors}: Coded by
dialogue type (38 distinct types) - \textbf{Node sizes}: Proportional to
degree centrality - \textbf{Edge types}: Three relationship types (same
room: weight 1.0, same partner: weight 0.8, same type: weight 0.6) -
\textbf{Layout}: Optimized for visual clarity with community clustering
visible

The network exhibits high clustering (coefficient: 0.886) indicating
strong room-based communities. The largest connected component contains
the majority of ways, with smaller isolated components representing
specialized dialogue patterns.
\end{block}

\begin{block}{Room Hierarchy Visualization}
\protect\phantomsection\label{room-hierarchy-visualization}
Figure \ref{fig:room_hierarchy} presents a hierarchical bar chart
showing the distribution of ways across all 24 rooms. The visualization
organizes rooms by their position in the House of Knowledge framework,
revealing: - \textbf{Most populated rooms}: B2 (23 ways), C4 (17 ways),
R (16 ways) - \textbf{Framework structure}: Clear patterns in believing
(B-series) and caring (C-series) hierarchies - \textbf{Relative learning
rooms}: R (Reflecting), O (Obeying), T (Taking a Stand) show balanced
distributions
\end{block}

\begin{block}{Framework Treemap}
\protect\phantomsection\label{framework-treemap}
Figure \ref{fig:framework_treemap} provides a treemap visualization of
the framework structure, where: - \textbf{Area}: Proportional to number
of ways in each room - \textbf{Color}: Indicates framework category
(believing, caring, relative learning) - \textbf{Hierarchy}: Shows
nested relationships within the House of Knowledge

This visualization highlights the structural organization of the
framework and the relative emphasis on different aspects of knowledge
acquisition.
\end{block}
\end{block}

\begin{block}{H.3 Extended Statistical Plots}
\protect\phantomsection\label{h.3-extended-statistical-plots}
\begin{block}{Dialogue Type Distribution}
\protect\phantomsection\label{dialogue-type-distribution}
Figure \ref{fig:type_distribution} displays a bar chart of all 38
dialogue types ranked by frequency. The visualization shows: -
\textbf{Top types}: ``goodness'' and ``other'' (15 each, 7.1\%),
``regularity'' (11, 5.2\%) - \textbf{Distribution pattern}: Long tail
with many types having 1-4 occurrences - \textbf{Balance}: Relatively
even distribution across types, indicating diverse epistemological
approaches
\end{block}

\begin{block}{Type × Room Heatmap}
\protect\phantomsection\label{type-room-heatmap}
Figure \ref{fig:type_room_heatmap} presents a heatmap of the
cross-tabulation between dialogue types (rows) and rooms (columns). The
visualization reveals: - \textbf{Hotspots}: Strong associations between
specific types and rooms (e.g., ``I'' × ``O'', ``knowledge'' × ``CC'') -
\textbf{Sparse regions}: Many type-room combinations have zero or low
counts - \textbf{Patterns}: Clustering of similar dialogue types in
related rooms

This heatmap provides insight into how dialogue types are distributed
across the House of Knowledge structure.
\end{block}

\begin{block}{Dialogue Partner Word Cloud}
\protect\phantomsection\label{dialogue-partner-word-cloud}
Figure \ref{fig:partner_wordcloud} shows a word cloud visualization of
dialogue partners, where: - \textbf{Font size}: Proportional to
frequency of partnership - \textbf{196 unique partners}: Most partners
appear only once or twice - \textbf{Top partners}: ``God's will'',
``God's wishes'', ``answer'', ``circumstances'' (2 occurrences each)

The word cloud highlights the diversity of dialogue partners and the
personalized nature of many ways.
\end{block}

\begin{block}{Example Length Distribution}
\protect\phantomsection\label{example-length-distribution}
Figure \ref{fig:example_length_violin} displays a violin plot showing
the distribution of example text lengths by dialogue type. The
visualization shows: - \textbf{Distribution shape}: Varies by dialogue
type, with some types having longer examples - \textbf{Average length}:
80.2 characters across all ways - \textbf{Coverage}: All 210 ways have
examples (100\% coverage)

This plot reveals patterns in how different dialogue types are
exemplified and documented.
\end{block}
\end{block}
\end{block}

\begin{block}{I. Code Availability}
\protect\phantomsection\label{i.-code-availability}
All code for this research is available in the project repository:

\begin{itemize}
\tightlist
\item
  \textbf{Database Module}: \texttt{project/src/database.py}
\item
  \textbf{Models}: \texttt{project/src/models.py}
\item
  \textbf{Analysis Scripts}: \texttt{project/scripts/}
\item
  \textbf{Tests}: \texttt{project/tests/}
\end{itemize}

The code follows the thin orchestrator pattern with business logic in
\texttt{src/} modules and orchestration in \texttt{scripts/}.
\end{block}

\begin{block}{J. Data Availability}
\protect\phantomsection\label{j.-data-availability}
The source data (MySQL dump and \texttt{ways.md}) are in the public
domain as stated in the original documentation. The converted SQLite
database and analysis results are available upon request or through the
project repository.
\end{block}

\begin{block}{K. Reproducibility}
\protect\phantomsection\label{k.-reproducibility}
To reproduce the analyses:

\begin{enumerate}
\tightlist
\item
  Initialize database: \texttt{python\ scripts/db\_setup.py}
\item
  Run analysis: \texttt{python\ scripts/analysis\_pipeline.py}
\item
  Generate visualizations: \texttt{python\ scripts/generate\_figures.py}
\item
  Build manuscript: \texttt{python\ scripts/03\_render\_pdf.py}
\end{enumerate}

All random operations use fixed seeds for reproducibility.
\end{block}
\end{frame}

\end{document}
