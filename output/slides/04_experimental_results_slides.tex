% Options for packages loaded elsewhere
\PassOptionsToPackage{unicode}{hyperref}
\PassOptionsToPackage{hyphens}{url}
\documentclass[
  ignorenonframetext,
]{beamer}
\newif\ifbibliography
\usepackage{pgfpages}
\setbeamertemplate{caption}[numbered]
\setbeamertemplate{caption label separator}{: }
\setbeamercolor{caption name}{fg=normal text.fg}
\beamertemplatenavigationsymbolsempty
% remove section numbering
\setbeamertemplate{part page}{
  \centering
  \begin{beamercolorbox}[sep=16pt,center]{part title}
    \usebeamerfont{part title}\insertpart\par
  \end{beamercolorbox}
}
\setbeamertemplate{section page}{
  \centering
  \begin{beamercolorbox}[sep=12pt,center]{section title}
    \usebeamerfont{section title}\insertsection\par
  \end{beamercolorbox}
}
\setbeamertemplate{subsection page}{
  \centering
  \begin{beamercolorbox}[sep=8pt,center]{subsection title}
    \usebeamerfont{subsection title}\insertsubsection\par
  \end{beamercolorbox}
}
% Prevent slide breaks in the middle of a paragraph
\widowpenalties 1 10000
\raggedbottom
\AtBeginPart{
  \frame{\partpage}
}
\AtBeginSection{
  \ifbibliography
  \else
    \frame{\sectionpage}
  \fi
}
\AtBeginSubsection{
  \frame{\subsectionpage}
}
\usepackage{iftex}
\ifPDFTeX
  \usepackage[T1]{fontenc}
  \usepackage[utf8]{inputenc}
  \usepackage{textcomp} % provide euro and other symbols
\else % if luatex or xetex
  \usepackage{unicode-math} % this also loads fontspec
  \defaultfontfeatures{Scale=MatchLowercase}
  \defaultfontfeatures[\rmfamily]{Ligatures=TeX,Scale=1}
\fi
\usepackage{lmodern}
\ifPDFTeX\else
  % xetex/luatex font selection
\fi
% Use upquote if available, for straight quotes in verbatim environments
\IfFileExists{upquote.sty}{\usepackage{upquote}}{}
\IfFileExists{microtype.sty}{% use microtype if available
  \usepackage[]{microtype}
  \UseMicrotypeSet[protrusion]{basicmath} % disable protrusion for tt fonts
}{}
\makeatletter
\@ifundefined{KOMAClassName}{% if non-KOMA class
  \IfFileExists{parskip.sty}{%
    \usepackage{parskip}
  }{% else
    \setlength{\parindent}{0pt}
    \setlength{\parskip}{6pt plus 2pt minus 1pt}}
}{% if KOMA class
  \KOMAoptions{parskip=half}}
\makeatother
\setlength{\emergencystretch}{3em} % prevent overfull lines
\providecommand{\tightlist}{%
  \setlength{\itemsep}{0pt}\setlength{\parskip}{0pt}}
\usepackage{bookmark}
\IfFileExists{xurl.sty}{\usepackage{xurl}}{} % add URL line breaks if available
\urlstyle{same}
\hypersetup{
  hidelinks,
  pdfcreator={LaTeX via pandoc}}

\author{\texorpdfstring{}{}}
\date{}

\begin{document}

\section{Experimental Results}\label{sec:experimental_results}

\begin{frame}[fragile]{Database Overview}
\protect\phantomsection\label{database-overview}
\begin{block}{Data Summary}
\protect\phantomsection\label{data-summary}
The analysis database contains:

\begin{itemize}
\tightlist
\item
  \textbf{210 documented ways} in the primary \texttt{ways} table
\item
  \textbf{24 rooms} in the House of Knowledge (\texttt{rooms} table)
\item
  \textbf{Multiple examples} per way (\texttt{examples} table)
\item
  \textbf{Question-way relationships} (\texttt{klausimobudai} table)
\item
  \textbf{Dialogue partner information} for each way
\end{itemize}
\end{block}

\begin{block}{Data Completeness}
\protect\phantomsection\label{data-completeness}
Analysis of data completeness reveals:

\begin{itemize}
\tightlist
\item
  All 210 ways have dialogue type assignments
\item
  Room assignments (\texttt{mene}) present for majority of ways
\item
  Dialogue partner (\texttt{dialoguewith}) information available for
  most ways
\item
  Examples and descriptions vary in completeness
\end{itemize}
\end{block}
\end{frame}

\begin{frame}{Distribution Analysis}
\protect\phantomsection\label{distribution-analysis}
\begin{block}{Dialogue Type Distribution}
\protect\phantomsection\label{dialogue-type-distribution}
Analysis of ways by dialogue type reveals the distribution across the
three main categories:

\begin{table}[h]
\centering
\begin{tabular}{|l|c|c|}
\hline
\textbf{Dialogue Type} & \textbf{Count} & \textbf{Percentage} \\
\hline
goodness & 15 & 7.1\% \\
other & 15 & 7.1\% \\
regularity & 11 & 5.2\% \\
I & 9 & 4.3\% \\
answer & 9 & 4.3\% \\
knowledge & 8 & 3.8\% \\
life & 8 & 3.8\% \\
mind & 8 & 3.8\% \\
my mind & 7 & 3.3\% \\
opposing view & 7 & 3.3\% \\
\hline
\textbf{Total} & 210 & 100\% \\
\hline
\end{tabular}
\caption{Distribution of ways by dialogue type (top 10)}
\label{tab:dialogue_type_distribution}
\end{table}

The complete distribution is visualized in Figure
\ref{fig:type_distribution}, showing the full range of 38 distinct
dialogue types.

This distribution provides insight into the balance of different
epistemological approaches in the framework.
\end{block}

\begin{block}{Room Distribution}
\protect\phantomsection\label{room-distribution}
Analysis of ways across the 24 rooms of the House of Knowledge reveals:

\begin{table}[h]
\centering
\begin{tabular}{|l|c|c|}
\hline
\textbf{Room} & \textbf{Way Count} & \textbf{Percentage} \\
\hline
B2 & 23 & 11.0\% \\
C4 & 17 & 8.1\% \\
R & 16 & 7.6\% \\
32 & 13 & 6.2\% \\
C3 & 13 & 6.2\% \\
BB & 12 & 5.7\% \\
CB & 10 & 4.8\% \\
21 & 9 & 4.3\% \\
B3 & 9 & 4.3\% \\
CC & 9 & 4.3\% \\
O & 9 & 4.3\% \\
T & 9 & 4.3\% \\
10 & 8 & 3.8\% \\
31 & 8 & 3.8\% \\
1 & 7 & 3.3\% \\
\hline
\textbf{Total} & 210 & 100\% \\
\hline
\end{tabular}
\caption{Distribution of ways across top 15 rooms}
\label{tab:room_distribution}
\end{table}

The complete room hierarchy is visualized in Figure
\ref{fig:room_hierarchy}, and the framework structure is shown in Figure
\ref{fig:framework_treemap}.

Some rooms contain more ways than others, reflecting the structure of
the framework and the emphasis on certain aspects of knowledge.
\end{block}
\end{frame}

\begin{frame}{Network Analysis Results}
\protect\phantomsection\label{network-analysis-results}
\begin{block}{Network Structure}
\protect\phantomsection\label{network-structure}
The network graph constructed from way relationships exhibits:

\begin{itemize}
\tightlist
\item
  \textbf{Nodes}: 210 ways
\item
  \textbf{Edges}: 1,290 connections
\item
  \textbf{Average degree}: 12.29 connections per way
\item
  \textbf{Network density}: 0.058 (5.8\% of possible edges present)
\item
  \textbf{Clustering coefficient}: 0.886 (high local clustering,
  indicating strong room-based clustering)
\item
  \textbf{Connected components}: Multiple components with largest
  containing majority of ways
\item
  \textbf{Network visualization}: See Figure \ref{fig:ways_network}
\end{itemize}

The network structure reveals both local clustering (ways in the same
room are highly connected) and long-range connections (ways sharing
dialogue types or partners across different rooms).
\end{block}

\begin{block}{Central Ways}
\protect\phantomsection\label{central-ways}
Centrality analysis identifies ways that serve as hubs or bridges:

\begin{table}[h]
\centering
\begin{tabular}{|l|c|c|}
\hline
\textbf{Way ID} & \textbf{Degree Centrality} & \textbf{Room} \\
\hline
84, 156, 211 & 34 & Multiple rooms \\
115 & 30 & Multiple rooms \\
120 & 25 & Multiple rooms \\
\hline
\end{tabular}
\caption{Most central ways by degree centrality (top 5)}
\label{tab:central_ways}
\end{table}

These central ways serve as hubs connecting multiple other ways through
shared rooms, dialogue types, or partners. The complete network
structure is visualized in Figure \ref{fig:ways_network}, showing the
clustering and connectivity patterns.
\end{block}

\begin{block}{Community Detection}
\protect\phantomsection\label{community-detection}
Community detection algorithms reveal clusters of related ways:

\begin{itemize}
\tightlist
\item
  \textbf{Cluster 1}: Ways related to goodness and morality (15 ways)
\item
  \textbf{Cluster 2}: Ways related to regularity and structure (11 ways)
\item
  \textbf{Cluster 3}: Ways related to personal identity and ``I'' (9
  ways)
\end{itemize}

These clusters may correspond to different aspects of the House of
Knowledge or different dialogue types.
\end{block}
\end{frame}

\begin{frame}[fragile]{Cross-Tabulation Analysis}
\protect\phantomsection\label{cross-tabulation-analysis}
\begin{block}{Dialogue Type × Room}
\protect\phantomsection\label{dialogue-type-room}
Cross-tabulation of dialogue types and room assignments reveals patterns
(visualized in Figure \ref{fig:type_room_heatmap}):

\begin{table}[h]
\centering
\begin{tabular}{|l|c|c|}
\hline
\textbf{Type × Room} & \textbf{Count} & \textbf{Notes} \\
\hline
goodness × B2 & 15 & Believing framework \\
goodness × C4 & 17 & Caring framework \\
other × B2 & 15 & Primary combination \\
regularity × BB & 11 & Strong association \\
I × CC & 9 & Identity-focused \\
life × R & 8 & Life-related ways \\
mind × 10 & Cognitive approaches \\
\hline
\end{tabular}
\caption{Top cross-tabulations of dialogue types and rooms}
\label{tab:type_room_crosstab}
\end{table}

The heatmap visualization (Figure \ref{fig:type_room_heatmap}) reveals
strong associations between certain dialogue types and specific rooms,
indicating structural relationships in the framework. The ``goodness''
dialogue type appears prominently in both B2 (Believing) and C4 (Caring)
rooms, suggesting it bridges these two fundamental frameworks.
\end{block}

\begin{block}{Dialogue Partner Analysis}
\protect\phantomsection\label{dialogue-partner-analysis}
Analysis of dialogue partners (\texttt{dialoguewith}) reveals:

\begin{itemize}
\tightlist
\item
  \textbf{Most common partners}: life, limits of my mind, circumstances,
  science, purpose, answer, people's inclinations, possibility,
  goodness, meaningfulness (all with 2 ways each)
\item
  \textbf{Partner diversity}: 116 unique partners
\item
  \textbf{Partner-way relationships}: Most partners connect exactly 2
  ways, indicating pairwise relationships
\end{itemize}

Some dialogue partners appear frequently across multiple ways,
suggesting they represent important perspectives or approaches.
\end{block}
\end{frame}

\begin{frame}{Statistical Patterns}
\protect\phantomsection\label{statistical-patterns}
\begin{block}{Room Co-occurrence}
\protect\phantomsection\label{room-co-occurrence}
Analysis of ways assigned to multiple rooms reveals:

\begin{itemize}
\tightlist
\item
  \textbf{Average rooms per way}: 1.0 (each way assigned to exactly one
  room)
\item
  \textbf{Most common room pairs}: N/A (single room assignments)
\item
  \textbf{Room clusters}: Rooms B2, C4, R, C3, 32 contain the highest
  concentrations of ways
\end{itemize}

This indicates how different aspects of knowledge relate to one another
in the framework.
\end{block}

\begin{block}{Dialogue Type Patterns}
\protect\phantomsection\label{dialogue-type-patterns}
Statistical analysis of dialogue type patterns shows:

\begin{itemize}
\tightlist
\item
  \textbf{Type transitions}: How ways of one type relate to ways of
  another
\item
  \textbf{Type clusters}: Groups of ways with similar type
  characteristics
\item
  \textbf{Type diversity}: Distribution of types within rooms and
  categories
\end{itemize}
\end{block}
\end{frame}

\begin{frame}{Text Analysis Results}
\protect\phantomsection\label{text-analysis-results}
\begin{block}{Keyword Extraction}
\protect\phantomsection\label{keyword-extraction}
Analysis of way descriptions and examples reveals common themes:

\begin{itemize}
\tightlist
\item
  \textbf{Top keywords}: goodness, regularity, other, I, answer (from
  dialogue types)
\item
  \textbf{Keyword clusters}: Philosophical concepts, personal
  relationships, structural patterns
\item
  \textbf{Keyword-room associations}: B2 room associated with ``other'',
  BB room with ``regularity''
\end{itemize}
\end{block}

\begin{block}{Example Analysis}
\protect\phantomsection\label{example-analysis}
Analysis of examples reveals:

\begin{itemize}
\tightlist
\item
  \textbf{Common example types}: Personal experiences, philosophical
  reflections, practical applications
\item
  \textbf{Example patterns}: Ways often illustrated through personal
  anecdotes and thought processes
\item
  \textbf{Example-way relationships}: Examples provide concrete
  illustrations of abstract ways of figuring things out
\end{itemize}
\end{block}
\end{frame}

\begin{frame}{Visualization Results}
\protect\phantomsection\label{visualization-results}
\begin{block}{Network Graph}
\protect\phantomsection\label{network-graph}
The network visualization (Figure \ref{fig:ways_network}) shows:

\begin{itemize}
\tightlist
\item
  Ways as nodes, colored by dialogue type
\item
  Connections as edges, weighted by relationship strength
\item
  Clusters visible as dense regions
\item
  Central ways as highly connected nodes
\end{itemize}

\begin{figure}[h]
\centering
\includegraphics[width=0.9\textwidth]{../output/figures/ways_network.png}
\caption{Network graph of ways showing connections and clusters}
\label{fig:ways_network}
\end{figure}
\end{block}

\begin{block}{Room Distribution}
\protect\phantomsection\label{room-distribution-1}
A hierarchical visualization (Figure \ref{fig:room_hierarchy}) shows:

\begin{itemize}
\tightlist
\item
  The 24-room structure
\item
  Way counts per room
\item
  Relationships between rooms
\end{itemize}

\begin{figure}[h]
\centering
\includegraphics[width=0.9\textwidth]{../output/figures/room_hierarchy.png}
\caption{Hierarchical visualization of the House of Knowledge structure}
\label{fig:room_hierarchy}
\end{figure}
\end{block}

\begin{block}{Statistical Distributions}
\protect\phantomsection\label{statistical-distributions}
Distribution plots show:

\begin{itemize}
\tightlist
\item
  Dialogue type frequencies (Figure \ref{fig:type_distribution})
\item
  Room assignment patterns (Figure \ref{fig:room_hierarchy})
\item
  Framework structure (Figure \ref{fig:framework_treemap})
\item
  Dialogue partner frequencies (Figure \ref{fig:partner_wordcloud})
\item
  Example length distributions by type (Figure
  \ref{fig:example_length_violin})
\end{itemize}

\begin{figure}[h]
\centering
\includegraphics[width=0.9\textwidth]{../output/figures/type_distribution.png}
\caption{Distribution of ways by dialogue type}
\label{fig:type_distribution}
\end{figure}
\end{block}

\begin{block}{Cross-Tabulation Heatmap}
\protect\phantomsection\label{cross-tabulation-heatmap}
The dialogue type × room cross-tabulation matrix (Figure
\ref{fig:type_room_heatmap}) reveals concentration patterns:

\begin{figure}[h]
\centering
\includegraphics[width=0.9\textwidth]{../output/figures/type_room_heatmap.png}
\caption{Heatmap showing dialogue type × room cross-tabulation}
\label{fig:type_room_heatmap}
\end{figure}
\end{block}

\begin{block}{Framework Structure}
\protect\phantomsection\label{framework-structure}
The framework hierarchy visualization (Figure
\ref{fig:framework_treemap}) shows the distribution of ways across the
main philosophical frameworks:

\begin{figure}[h]
\centering
\includegraphics[width=0.9\textwidth]{../output/figures/framework_treemap.png}
\caption{Hierarchical visualization of framework distribution}
\label{fig:framework_treemap}
\end{figure}
\end{block}

\begin{block}{Dialogue Partners}
\protect\phantomsection\label{dialogue-partners}
The dialogue partner frequency distribution (Figure
\ref{fig:partner_wordcloud}) shows the diversity of conversants:

\begin{figure}[h]
\centering
\includegraphics[width=0.9\textwidth]{../output/figures/partner_wordcloud.png}
\caption{Dialogue partner frequency distribution}
\label{fig:partner_wordcloud}
\end{figure}
\end{block}

\begin{block}{Example Length Analysis}
\protect\phantomsection\label{example-length-analysis}
The distribution of example lengths by dialogue type (Figure
\ref{fig:example_length_violin}) reveals patterns in how ways are
documented:

\begin{figure}[h]
\centering
\includegraphics[width=0.9\textwidth]{../output/figures/example_length_violin.png}
\caption{Example length distribution by dialogue type}
\label{fig:example_length_violin}
\end{figure}
\end{block}
\end{frame}

\begin{frame}{Key Findings}
\protect\phantomsection\label{key-findings}
\begin{block}{Structural Patterns}
\protect\phantomsection\label{structural-patterns}
\begin{enumerate}
\tightlist
\item
  \textbf{Room Clustering}: Ways cluster within certain rooms,
  indicating focused approaches to specific aspects of knowledge
\item
  \textbf{Type Balance}: The distribution across dialogue types reflects
  the framework's emphasis on different epistemological approaches
\item
  \textbf{Network Structure}: The network exhibits small-world
  properties with both local clustering and long-range connections
\end{enumerate}
\end{block}

\begin{block}{Central Ways}
\protect\phantomsection\label{central-ways-1}
Certain ways serve as central nodes, connecting different parts of the
framework. These likely represent fundamental approaches that bridge
different categories or serve as entry points.
\end{block}

\begin{block}{Room Relationships}
\protect\phantomsection\label{room-relationships}
Analysis reveals relationships between rooms, showing how different
aspects of knowledge relate. Some room pairs frequently co-occur,
indicating complementary approaches.
\end{block}
\end{frame}

\begin{frame}{Limitations}
\protect\phantomsection\label{limitations}
\begin{block}{Data Completeness}
\protect\phantomsection\label{data-completeness-1}
\begin{itemize}
\tightlist
\item
  Not all ways have complete metadata
\item
  Some room assignments may be missing
\item
  Dialogue partner information varies in completeness
\end{itemize}
\end{block}

\begin{block}{Analysis Scope}
\protect\phantomsection\label{analysis-scope}
\begin{itemize}
\tightlist
\item
  Analysis focuses on documented ways (212 of 284 total)
\item
  Text analysis limited to available descriptions
\item
  Network analysis based on explicit relationships in database
\end{itemize}
\end{block}
\end{frame}

\begin{frame}{Future Analysis Directions}
\protect\phantomsection\label{future-analysis-directions}
Future work will:

\begin{enumerate}
\tightlist
\item
  Complete analysis of all 284 ways
\item
  Expand text analysis with natural language processing
\item
  Develop predictive models for way categorization
\item
  Create interactive visualizations
\item
  Analyze temporal patterns if dating information available
\end{enumerate}
\end{frame}

\end{document}
