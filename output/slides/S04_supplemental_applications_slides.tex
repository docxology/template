% Options for packages loaded elsewhere
\PassOptionsToPackage{unicode}{hyperref}
\PassOptionsToPackage{hyphens}{url}
\documentclass[
  ignorenonframetext,
]{beamer}
\newif\ifbibliography
\usepackage{pgfpages}
\setbeamertemplate{caption}[numbered]
\setbeamertemplate{caption label separator}{: }
\setbeamercolor{caption name}{fg=normal text.fg}
\beamertemplatenavigationsymbolsempty
% remove section numbering
\setbeamertemplate{part page}{
  \centering
  \begin{beamercolorbox}[sep=16pt,center]{part title}
    \usebeamerfont{part title}\insertpart\par
  \end{beamercolorbox}
}
\setbeamertemplate{section page}{
  \centering
  \begin{beamercolorbox}[sep=12pt,center]{section title}
    \usebeamerfont{section title}\insertsection\par
  \end{beamercolorbox}
}
\setbeamertemplate{subsection page}{
  \centering
  \begin{beamercolorbox}[sep=8pt,center]{subsection title}
    \usebeamerfont{subsection title}\insertsubsection\par
  \end{beamercolorbox}
}
% Prevent slide breaks in the middle of a paragraph
\widowpenalties 1 10000
\raggedbottom
\AtBeginPart{
  \frame{\partpage}
}
\AtBeginSection{
  \ifbibliography
  \else
    \frame{\sectionpage}
  \fi
}
\AtBeginSubsection{
  \frame{\subsectionpage}
}
\usepackage{iftex}
\ifPDFTeX
  \usepackage[T1]{fontenc}
  \usepackage[utf8]{inputenc}
  \usepackage{textcomp} % provide euro and other symbols
\else % if luatex or xetex
  \usepackage{unicode-math} % this also loads fontspec
  \defaultfontfeatures{Scale=MatchLowercase}
  \defaultfontfeatures[\rmfamily]{Ligatures=TeX,Scale=1}
\fi
\usepackage{lmodern}
\ifPDFTeX\else
  % xetex/luatex font selection
\fi
% Use upquote if available, for straight quotes in verbatim environments
\IfFileExists{upquote.sty}{\usepackage{upquote}}{}
\IfFileExists{microtype.sty}{% use microtype if available
  \usepackage[]{microtype}
  \UseMicrotypeSet[protrusion]{basicmath} % disable protrusion for tt fonts
}{}
\makeatletter
\@ifundefined{KOMAClassName}{% if non-KOMA class
  \IfFileExists{parskip.sty}{%
    \usepackage{parskip}
  }{% else
    \setlength{\parindent}{0pt}
    \setlength{\parskip}{6pt plus 2pt minus 1pt}}
}{% if KOMA class
  \KOMAoptions{parskip=half}}
\makeatother
\usepackage{color}
\usepackage{fancyvrb}
\newcommand{\VerbBar}{|}
\newcommand{\VERB}{\Verb[commandchars=\\\{\}]}
\DefineVerbatimEnvironment{Highlighting}{Verbatim}{commandchars=\\\{\}}
% Add ',fontsize=\small' for more characters per line
\newenvironment{Shaded}{}{}
\newcommand{\AlertTok}[1]{\textcolor[rgb]{1.00,0.00,0.00}{\textbf{#1}}}
\newcommand{\AnnotationTok}[1]{\textcolor[rgb]{0.38,0.63,0.69}{\textbf{\textit{#1}}}}
\newcommand{\AttributeTok}[1]{\textcolor[rgb]{0.49,0.56,0.16}{#1}}
\newcommand{\BaseNTok}[1]{\textcolor[rgb]{0.25,0.63,0.44}{#1}}
\newcommand{\BuiltInTok}[1]{\textcolor[rgb]{0.00,0.50,0.00}{#1}}
\newcommand{\CharTok}[1]{\textcolor[rgb]{0.25,0.44,0.63}{#1}}
\newcommand{\CommentTok}[1]{\textcolor[rgb]{0.38,0.63,0.69}{\textit{#1}}}
\newcommand{\CommentVarTok}[1]{\textcolor[rgb]{0.38,0.63,0.69}{\textbf{\textit{#1}}}}
\newcommand{\ConstantTok}[1]{\textcolor[rgb]{0.53,0.00,0.00}{#1}}
\newcommand{\ControlFlowTok}[1]{\textcolor[rgb]{0.00,0.44,0.13}{\textbf{#1}}}
\newcommand{\DataTypeTok}[1]{\textcolor[rgb]{0.56,0.13,0.00}{#1}}
\newcommand{\DecValTok}[1]{\textcolor[rgb]{0.25,0.63,0.44}{#1}}
\newcommand{\DocumentationTok}[1]{\textcolor[rgb]{0.73,0.13,0.13}{\textit{#1}}}
\newcommand{\ErrorTok}[1]{\textcolor[rgb]{1.00,0.00,0.00}{\textbf{#1}}}
\newcommand{\ExtensionTok}[1]{#1}
\newcommand{\FloatTok}[1]{\textcolor[rgb]{0.25,0.63,0.44}{#1}}
\newcommand{\FunctionTok}[1]{\textcolor[rgb]{0.02,0.16,0.49}{#1}}
\newcommand{\ImportTok}[1]{\textcolor[rgb]{0.00,0.50,0.00}{\textbf{#1}}}
\newcommand{\InformationTok}[1]{\textcolor[rgb]{0.38,0.63,0.69}{\textbf{\textit{#1}}}}
\newcommand{\KeywordTok}[1]{\textcolor[rgb]{0.00,0.44,0.13}{\textbf{#1}}}
\newcommand{\NormalTok}[1]{#1}
\newcommand{\OperatorTok}[1]{\textcolor[rgb]{0.40,0.40,0.40}{#1}}
\newcommand{\OtherTok}[1]{\textcolor[rgb]{0.00,0.44,0.13}{#1}}
\newcommand{\PreprocessorTok}[1]{\textcolor[rgb]{0.74,0.48,0.00}{#1}}
\newcommand{\RegionMarkerTok}[1]{#1}
\newcommand{\SpecialCharTok}[1]{\textcolor[rgb]{0.25,0.44,0.63}{#1}}
\newcommand{\SpecialStringTok}[1]{\textcolor[rgb]{0.73,0.40,0.53}{#1}}
\newcommand{\StringTok}[1]{\textcolor[rgb]{0.25,0.44,0.63}{#1}}
\newcommand{\VariableTok}[1]{\textcolor[rgb]{0.10,0.09,0.49}{#1}}
\newcommand{\VerbatimStringTok}[1]{\textcolor[rgb]{0.25,0.44,0.63}{#1}}
\newcommand{\WarningTok}[1]{\textcolor[rgb]{0.38,0.63,0.69}{\textbf{\textit{#1}}}}
\usepackage{longtable,booktabs,array}
\newcounter{none} % for unnumbered tables
\usepackage{calc} % for calculating minipage widths
\usepackage{caption}
% Make caption package work with longtable
\makeatletter
\def\fnum@table{\tablename~\thetable}
\makeatother
\setlength{\emergencystretch}{3em} % prevent overfull lines
\providecommand{\tightlist}{%
  \setlength{\itemsep}{0pt}\setlength{\parskip}{0pt}}
\usepackage{bookmark}
\IfFileExists{xurl.sty}{\usepackage{xurl}}{} % add URL line breaks if available
\urlstyle{same}
\hypersetup{
  hidelinks,
  pdfcreator={LaTeX via pandoc}}

\author{\texorpdfstring{}{}}
\date{}

\begin{document}

\section{Supplemental Applications}\label{supplemental-applications}

\subsection{S4.1 Digital Circuit
Design}\label{s4.1-digital-circuit-design}

\begin{frame}{NAND-Based Synthesis}
\protect\phantomsection\label{nand-based-synthesis}
The NAND gate is functionally complete---all Boolean functions are
expressible using only NAND. In boundary logic:

\[a \text{ NAND } b = \langle ab \rangle\]

\begin{block}{All Gates from NAND}
\protect\phantomsection\label{all-gates-from-nand}
{\def\LTcaptype{none} % do not increment counter
\begin{longtable}[]{@{}
  >{\raggedright\arraybackslash}p{(\linewidth - 6\tabcolsep) * \real{0.1667}}
  >{\raggedright\arraybackslash}p{(\linewidth - 6\tabcolsep) * \real{0.2500}}
  >{\raggedright\arraybackslash}p{(\linewidth - 6\tabcolsep) * \real{0.3056}}
  >{\raggedright\arraybackslash}p{(\linewidth - 6\tabcolsep) * \real{0.2778}}@{}}
\toprule\noalign{}
\begin{minipage}[b]{\linewidth}\raggedright
Gate
\end{minipage} & \begin{minipage}[b]{\linewidth}\raggedright
Boolean
\end{minipage} & \begin{minipage}[b]{\linewidth}\raggedright
NAND Form
\end{minipage} & \begin{minipage}[b]{\linewidth}\raggedright
Boundary
\end{minipage} \\
\midrule\noalign{}
\endhead
NOT & \(\neg a\) & \(a\) NAND \(a\) &
\(\langle aa \rangle = \langle a \rangle\) \\
AND & \(a \land b\) & NOT(\(a\) NAND \(b\)) &
\(\langle\langle ab \rangle\rangle = ab\) \\
OR & \(a \lor b\) & (NOT \(a\)) NAND (NOT \(b\)) &
\(\langle\langle a \rangle\langle b \rangle\rangle\) \\
XOR & \(a \oplus b\) & Complex &
\(\langle\langle\langle a \rangle b \rangle\langle a\langle b \rangle\rangle\rangle\) \\
\bottomrule\noalign{}
\end{longtable}
}
\end{block}
\end{frame}

\begin{frame}{Circuit Optimization}
\protect\phantomsection\label{circuit-optimization}
Boundary reduction rules translate to circuit transformations:

{\def\LTcaptype{none} % do not increment counter
\begin{longtable}[]{@{}
  >{\raggedright\arraybackslash}p{(\linewidth - 2\tabcolsep) * \real{0.4211}}
  >{\raggedright\arraybackslash}p{(\linewidth - 2\tabcolsep) * \real{0.5789}}@{}}
\toprule\noalign{}
\begin{minipage}[b]{\linewidth}\raggedright
Reduction Rule
\end{minipage} & \begin{minipage}[b]{\linewidth}\raggedright
Circuit Transformation
\end{minipage} \\
\midrule\noalign{}
\endhead
Calling (\(\langle\langle a \rangle\rangle = a\)) & Remove
double-inverter \\
Crossing (\(\langle\ \rangle\langle\ \rangle = \langle\ \rangle\)) &
Merge parallel power lines \\
Void elimination & Remove disconnected components \\
\bottomrule\noalign{}
\end{longtable}
}
\end{frame}

\begin{frame}{Layout Example}
\protect\phantomsection\label{layout-example}
A full adder in boundary notation:

\textbf{Sum}: \(S = a \oplus b \oplus c_{in}\) \textbf{Carry}:
\(c_{out} = (a \land b) \lor (c_{in} \land (a \oplus b))\)

The boundary forms directly map to circuit layout with nested regions
representing signal containment.
\end{frame}

\subsection{S4.2 Cognitive Science
Applications}\label{s4.2-cognitive-science-applications}

\begin{frame}{Perception as Distinction}
\protect\phantomsection\label{perception-as-distinction}
The calculus models fundamental perceptual operations:

{\def\LTcaptype{none} % do not increment counter
\begin{longtable}[]{@{}ll@{}}
\toprule\noalign{}
Perceptual Process & Boundary Operation \\
\midrule\noalign{}
\endhead
Figure-ground separation & Making a mark \\
Object recognition & Canonical form identification \\
Categorization & Reduction to equivalence class \\
Attention & Enclosure (isolating from context) \\
\bottomrule\noalign{}
\end{longtable}
}
\end{frame}

\begin{frame}{Binary Classification}
\protect\phantomsection\label{binary-classification}
Any binary classifier implements boundary logic: - Decision boundary =
mark - Class 1 = inside - Class 0 = outside

Neural network classifiers learn to draw effective marks in feature
space.
\end{frame}

\begin{frame}{Self-Reference and Consciousness}
\protect\phantomsection\label{self-reference-and-consciousness}
The imaginary value \(j = \langle j \rangle\) models self-referential
consciousness: - Consciousness observing itself - The observer is inside
what it observes - Oscillation between subject and object positions

This aligns with theories of consciousness as recursive self-modeling.
\end{frame}

\subsection{S4.3 Programming Language
Applications}\label{s4.3-programming-language-applications}

\begin{frame}{Type Systems}
\protect\phantomsection\label{type-systems}
Boundary logic maps to type theory:

{\def\LTcaptype{none} % do not increment counter
\begin{longtable}[]{@{}ll@{}}
\toprule\noalign{}
Boundary & Type Theory \\
\midrule\noalign{}
\endhead
Void & Empty type (⊥) \\
Mark & Unit type (⊤) \\
Enclosure & Negation type \\
Juxtaposition & Product type \\
De Morgan form & Sum type \\
\bottomrule\noalign{}
\end{longtable}
}
\end{frame}

\begin{frame}[fragile]{Pattern Matching}
\protect\phantomsection\label{pattern-matching}
Form patterns translate to match expressions:

\begin{Shaded}
\begin{Highlighting}[]
\ControlFlowTok{match}\NormalTok{ form:}
    \ControlFlowTok{case}\NormalTok{ Form(is\_marked}\OperatorTok{=}\VariableTok{False}\NormalTok{, contents}\OperatorTok{=}\NormalTok{[]):}
        \ControlFlowTok{return} \StringTok{"void"}
    \ControlFlowTok{case}\NormalTok{ Form(is\_marked}\OperatorTok{=}\VariableTok{True}\NormalTok{, contents}\OperatorTok{=}\NormalTok{[]):}
        \ControlFlowTok{return} \StringTok{"mark"}
    \ControlFlowTok{case}\NormalTok{ Form(is\_marked}\OperatorTok{=}\VariableTok{True}\NormalTok{, contents}\OperatorTok{=}\NormalTok{[inner]):}
        \ControlFlowTok{return} \SpecialStringTok{f"enclose(}\SpecialCharTok{\{}\NormalTok{process(inner)}\SpecialCharTok{\}}\SpecialStringTok{)"}
    \ControlFlowTok{case}\NormalTok{ Form(contents}\OperatorTok{=}\NormalTok{children):}
        \ControlFlowTok{return} \SpecialStringTok{f"juxtapose(}\SpecialCharTok{\{}\StringTok{\textquotesingle{}, \textquotesingle{}}\SpecialCharTok{.}\NormalTok{join(process(c) }\ControlFlowTok{for}\NormalTok{ c }\KeywordTok{in}\NormalTok{ children)}\SpecialCharTok{\}}\SpecialStringTok{)"}
\end{Highlighting}
\end{Shaded}
\end{frame}

\begin{frame}[fragile]{Expression Languages}
\protect\phantomsection\label{expression-languages}
A boundary expression language:

\begin{verbatim}
<program> ::= <form>
<form> ::= '.' | '<>' | '<' <form>* '>'
\end{verbatim}

Where \texttt{.} = void, \texttt{\textless{}\textgreater{}} = mark,
\texttt{\textless{}...\textgreater{}} = enclosure.
\end{frame}

\subsection{S4.4 Knowledge
Representation}\label{s4.4-knowledge-representation}

\begin{frame}{Ontology Design}
\protect\phantomsection\label{ontology-design}
Boundary forms represent ontological distinctions:

{\def\LTcaptype{none} % do not increment counter
\begin{longtable}[]{@{}ll@{}}
\toprule\noalign{}
Ontological Concept & Boundary Representation \\
\midrule\noalign{}
\endhead
Class & Marked region \\
Instance & Point within region \\
Subclass & Nested enclosure \\
Disjoint classes & Separate marks \\
Complement & Enclosure \\
\bottomrule\noalign{}
\end{longtable}
}
\end{frame}

\begin{frame}[fragile]{Semantic Web}
\protect\phantomsection\label{semantic-web}
RDF triples map to boundary structures: - Subject: Outermost boundary -
Predicate: Enclosure operation - Object: Inner content

\begin{verbatim}
"Dog" "is-a" "Animal"  →  ⟨Animal⟨Dog⟩⟩
\end{verbatim}
\end{frame}

\begin{frame}{Logic Programming}
\protect\phantomsection\label{logic-programming}
Boundary forms as logic programs: - Mark = fact (true assertion) - Void
= absence (closed world) - Enclosure = negation as failure - Reduction =
resolution
\end{frame}

\subsection{S4.5 Mathematical
Education}\label{s4.5-mathematical-education}

\begin{frame}{Teaching Boolean Logic}
\protect\phantomsection\label{teaching-boolean-logic}
Boundary notation provides intuitive visualization:

{\def\LTcaptype{none} % do not increment counter
\begin{longtable}[]{@{}
  >{\raggedright\arraybackslash}p{(\linewidth - 6\tabcolsep) * \real{0.3529}}
  >{\raggedright\arraybackslash}p{(\linewidth - 6\tabcolsep) * \real{0.2353}}
  >{\raggedright\arraybackslash}p{(\linewidth - 6\tabcolsep) * \real{0.1961}}
  >{\raggedright\arraybackslash}p{(\linewidth - 6\tabcolsep) * \real{0.2157}}@{}}
\toprule\noalign{}
\begin{minipage}[b]{\linewidth}\raggedright
Standard Notation
\end{minipage} & \begin{minipage}[b]{\linewidth}\raggedright
Difficulty
\end{minipage} & \begin{minipage}[b]{\linewidth}\raggedright
Boundary
\end{minipage} & \begin{minipage}[b]{\linewidth}\raggedright
Advantage
\end{minipage} \\
\midrule\noalign{}
\endhead
\(\neg\neg P\) & Double negative confusion &
\(\langle\langle P \rangle\rangle\) & Visible cancellation \\
\(P \land \neg P\) & Abstract contradiction & \(P\langle P \rangle\) &
Spatial conflict \\
\(P \lor \neg P\) & Abstract tautology &
\(\langle\langle P \rangle P\rangle\) & Reduces to mark \\
\bottomrule\noalign{}
\end{longtable}
}
\end{frame}

\begin{frame}{Proof Visualization}
\protect\phantomsection\label{proof-visualization}
Students can manipulate diagrams: 1. Draw forms as nested boxes 2. Apply
reduction rules visually 3. See equivalence by reaching same canonical
form
\end{frame}

\begin{frame}{Curricular Integration}
\protect\phantomsection\label{curricular-integration}
Suggested progression: 1. \textbf{Elementary}: Distinguish shapes
(making marks) 2. \textbf{Middle School}: Boolean operations as spatial
3. \textbf{High School}: Formal reduction and proof 4.
\textbf{University}: Theoretical foundations
\end{frame}

\subsection{S4.6 Quantum Computing
Analogies}\label{s4.6-quantum-computing-analogies}

\begin{frame}{Superposition and Imaginary Values}
\protect\phantomsection\label{superposition-and-imaginary-values}
Quantum superposition parallels imaginary Boolean values:

{\def\LTcaptype{none} % do not increment counter
\begin{longtable}[]{@{}ll@{}}
\toprule\noalign{}
Quantum & Boundary Logic \\
\midrule\noalign{}
\endhead
\(\|0\rangle\) & Void \\
\(\|1\rangle\) & Mark \\
\(\alpha\|0\rangle + \beta\|1\rangle\) & Imaginary \(j\) \\
Measurement & Forcing to canonical form \\
\bottomrule\noalign{}
\end{longtable}
}
\end{frame}

\begin{frame}{Quantum Gates}
\protect\phantomsection\label{quantum-gates}
Some quantum gates have boundary analogs:

{\def\LTcaptype{none} % do not increment counter
\begin{longtable}[]{@{}
  >{\raggedright\arraybackslash}p{(\linewidth - 4\tabcolsep) * \real{0.1935}}
  >{\raggedright\arraybackslash}p{(\linewidth - 4\tabcolsep) * \real{0.2581}}
  >{\raggedright\arraybackslash}p{(\linewidth - 4\tabcolsep) * \real{0.5484}}@{}}
\toprule\noalign{}
\begin{minipage}[b]{\linewidth}\raggedright
Gate
\end{minipage} & \begin{minipage}[b]{\linewidth}\raggedright
Matrix
\end{minipage} & \begin{minipage}[b]{\linewidth}\raggedright
Boundary Analog
\end{minipage} \\
\midrule\noalign{}
\endhead
NOT (X) & \(\begin{pmatrix}0&1\\1&0\end{pmatrix}\) & Enclosure
\(\langle\ \rangle\) \\
Identity (I) & \(\begin{pmatrix}1&0\\0&1\end{pmatrix}\) & Void
operation \\
Z & \(\begin{pmatrix}1&0\\0&-1\end{pmatrix}\) & Phase (no classical
analog) \\
\bottomrule\noalign{}
\end{longtable}
}
\end{frame}

\begin{frame}{Entanglement}
\protect\phantomsection\label{entanglement}
Multi-qubit entanglement might map to form sharing: - Entangled forms
share substructure - Measurement of one affects canonical form of both -
Non-local correlations through reduction
\end{frame}

\subsection{S4.7 Systems Theory}\label{s4.7-systems-theory}

\begin{frame}{Boundaries and Systems}
\protect\phantomsection\label{boundaries-and-systems}
General systems theory uses boundaries extensively:

{\def\LTcaptype{none} % do not increment counter
\begin{longtable}[]{@{}ll@{}}
\toprule\noalign{}
Systems Concept & Boundary Analog \\
\midrule\noalign{}
\endhead
System boundary & Mark \\
Open system & Permeable boundary \\
Closed system & Complete enclosure \\
System hierarchy & Nested enclosures \\
Feedback & Self-referential form \\
\bottomrule\noalign{}
\end{longtable}
}
\end{frame}

\begin{frame}{Autopoiesis}
\protect\phantomsection\label{autopoiesis}
Maturana and Varela's autopoiesis: - Self-producing systems maintain
their boundary - The boundary defines the system - Production occurs
within the boundary

Autopoietic systems = forms that reduce to themselves under
perturbation.
\end{frame}

\begin{frame}[fragile]{Cybernetic Loops}
\protect\phantomsection\label{cybernetic-loops}
Feedback loops in boundary notation:

\begin{verbatim}
f = ⟨input ⟨f⟩⟩
\end{verbatim}

The system's output becomes input through enclosure---recursively
defined.
\end{frame}

\subsection{S4.8 Art and Design}\label{s4.8-art-and-design}

\begin{frame}{Generative Art}
\protect\phantomsection\label{generative-art}
Form generation produces visual patterns: - Random forms → diverse
nested structures - Reduction → simplified compositions - Canonical
forms → fundamental patterns
\end{frame}

\begin{frame}{Visual Language}
\protect\phantomsection\label{visual-language}
Designers can use boundary logic: - Mark = focus element - Enclosure =
framing - Juxtaposition = composition - Reduction = simplification
\end{frame}

\begin{frame}{Interactive Installations}
\protect\phantomsection\label{interactive-installations}
Physical boundary installations: - Visitors enter/exit regions - Sensors
detect boundary crossings - System state = current form - Interactions =
reductions
\end{frame}

\subsection{S4.9 Future Applications}\label{s4.9-future-applications}

\begin{frame}{Anticipated Domains}
\protect\phantomsection\label{anticipated-domains}
\begin{enumerate}
\tightlist
\item
  \textbf{Blockchain}: Smart contracts as reducible forms
\item
  \textbf{IoT}: Sensor networks as boundary systems
\item
  \textbf{Robotics}: Spatial reasoning with boundaries
\item
  \textbf{Medicine}: Diagnostic categorization
\item
  \textbf{Law}: Jurisdictional boundaries
\end{enumerate}
\end{frame}

\begin{frame}{Research Directions}
\protect\phantomsection\label{research-directions}
\begin{enumerate}
\tightlist
\item
  \textbf{Efficient reduction hardware}: ASICs for boundary logic
\item
  \textbf{Distributed forms}: Network-distributed boundary computation
\item
  \textbf{Temporal extensions}: Forms evolving over time
\item
  \textbf{Probabilistic forms}: Uncertainty in boundaries
\end{enumerate}
\end{frame}

\begin{frame}{Open Problems}
\protect\phantomsection\label{open-problems}
\begin{enumerate}
\tightlist
\item
  \textbf{Optimal encoding}: Best form representation for specific
  domains
\item
  \textbf{Learning boundaries}: ML to discover effective distinctions
\item
  \textbf{Scaling}: Boundary logic for large-scale systems
\item
  \textbf{Integration}: Combining with existing formal methods
\end{enumerate}
\end{frame}

\end{document}
