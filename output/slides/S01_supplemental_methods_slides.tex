% Options for packages loaded elsewhere
\PassOptionsToPackage{unicode}{hyperref}
\PassOptionsToPackage{hyphens}{url}
\documentclass[
  ignorenonframetext,
]{beamer}
\newif\ifbibliography
\usepackage{pgfpages}
\setbeamertemplate{caption}[numbered]
\setbeamertemplate{caption label separator}{: }
\setbeamercolor{caption name}{fg=normal text.fg}
\beamertemplatenavigationsymbolsempty
% remove section numbering
\setbeamertemplate{part page}{
  \centering
  \begin{beamercolorbox}[sep=16pt,center]{part title}
    \usebeamerfont{part title}\insertpart\par
  \end{beamercolorbox}
}
\setbeamertemplate{section page}{
  \centering
  \begin{beamercolorbox}[sep=12pt,center]{section title}
    \usebeamerfont{section title}\insertsection\par
  \end{beamercolorbox}
}
\setbeamertemplate{subsection page}{
  \centering
  \begin{beamercolorbox}[sep=8pt,center]{subsection title}
    \usebeamerfont{subsection title}\insertsubsection\par
  \end{beamercolorbox}
}
% Prevent slide breaks in the middle of a paragraph
\widowpenalties 1 10000
\raggedbottom
\AtBeginPart{
  \frame{\partpage}
}
\AtBeginSection{
  \ifbibliography
  \else
    \frame{\sectionpage}
  \fi
}
\AtBeginSubsection{
  \frame{\subsectionpage}
}
\usepackage{iftex}
\ifPDFTeX
  \usepackage[T1]{fontenc}
  \usepackage[utf8]{inputenc}
  \usepackage{textcomp} % provide euro and other symbols
\else % if luatex or xetex
  \usepackage{unicode-math} % this also loads fontspec
  \defaultfontfeatures{Scale=MatchLowercase}
  \defaultfontfeatures[\rmfamily]{Ligatures=TeX,Scale=1}
\fi
\usepackage{lmodern}
\ifPDFTeX\else
  % xetex/luatex font selection
\fi
% Use upquote if available, for straight quotes in verbatim environments
\IfFileExists{upquote.sty}{\usepackage{upquote}}{}
\IfFileExists{microtype.sty}{% use microtype if available
  \usepackage[]{microtype}
  \UseMicrotypeSet[protrusion]{basicmath} % disable protrusion for tt fonts
}{}
\makeatletter
\@ifundefined{KOMAClassName}{% if non-KOMA class
  \IfFileExists{parskip.sty}{%
    \usepackage{parskip}
  }{% else
    \setlength{\parindent}{0pt}
    \setlength{\parskip}{6pt plus 2pt minus 1pt}}
}{% if KOMA class
  \KOMAoptions{parskip=half}}
\makeatother
\setlength{\emergencystretch}{3em} % prevent overfull lines
\providecommand{\tightlist}{%
  \setlength{\itemsep}{0pt}\setlength{\parskip}{0pt}}
\usepackage{bookmark}
\IfFileExists{xurl.sty}{\usepackage{xurl}}{} % add URL line breaks if available
\urlstyle{same}
\hypersetup{
  hidelinks,
  pdfcreator={LaTeX via pandoc}}

\author{\texorpdfstring{}{}}
\date{}

\begin{document}

\begin{frame}{Supplemental Methods}
\protect\phantomsection\label{sec:supplemental_methods}
This section provides detailed methodological information that
supplements Section \ref{sec:methodology}.

\begin{block}{S1.1 Extended Grafting Techniques}
\protect\phantomsection\label{s1.1-extended-grafting-techniques}
\begin{block}{S1.1.1 Approach Grafting}
\protect\phantomsection\label{s1.1.1-approach-grafting}
Approach grafting (also called inarching) involves joining two growing
plants while both remain on their own roots, then severing the scion
from its roots after union formation. This technique is particularly
useful for difficult-to-graft species or when precise alignment is
challenging.

\textbf{Procedure}: 1. Select healthy rootstock and scion plants in
close proximity 2. Make matching cuts on both plants (30-40° angle) 3.
Align cambium layers and secure together 4. Allow union to form over 4-8
weeks 5. Gradually reduce scion root system 6. Sever scion from its
roots after full union establishment

\textbf{Success Rate}: 70-80\% for compatible species, 50-60\% for
difficult combinations
\end{block}

\begin{block}{S1.1.2 Bridge Grafting}
\protect\phantomsection\label{s1.1.2-bridge-grafting}
Bridge grafting is used to repair damaged bark by bridging wounds with
scion pieces. This technique is essential for tree rescue operations and
bark damage repair.

\textbf{Procedure}: 1. Prepare damaged area by cleaning and removing
dead tissue 2. Make cuts above and below the damaged region 3. Prepare
scion pieces (typically 2-4 pieces depending on wound size) 4. Insert
scion pieces to bridge the gap, aligning cambium 5. Secure and seal all
connections 6. Monitor and protect until union forms

\textbf{Success Rate}: 60-70\% depending on wound severity and timing
\end{block}

\begin{block}{S1.1.3 Inarching}
\protect\phantomsection\label{s1.1.3-inarching}
Inarching involves grafting rootstock seedlings to established trees to
add roots, improving root system health and stability.

\textbf{Procedure}: 1. Prepare rootstock seedlings (typically 1-2 years
old) 2. Make matching cuts on tree and rootstock 3. Join and secure with
cambium alignment 4. Allow union to form (6-12 weeks) 5. Rootstock
provides additional root system support

\textbf{Success Rate}: 65-75\% for compatible species
\end{block}
\end{block}

\begin{block}{S1.2 Detailed Technique Protocols}
\protect\phantomsection\label{s1.2-detailed-technique-protocols}
\begin{block}{S1.2.1 Whip and Tongue Grafting - Step by Step}
\protect\phantomsection\label{s1.2.1-whip-and-tongue-grafting---step-by-step}
\textbf{Materials Required}: - Sharp grafting knife - Grafting tape or
wax - Rootstock and scion of matching diameter - Protective covering

\textbf{Detailed Steps}:

\begin{enumerate}
\tightlist
\item
  \textbf{Rootstock Preparation}:

  \begin{itemize}
  \tightlist
  \item
    Select healthy rootstock with diameter 5-25 mm
  \item
    Make 30-45° angle cut, 2-3 cm long
  \item
    Create tongue (notch) 1/3 from top of cut, 1 cm deep
  \end{itemize}
\item
  \textbf{Scion Preparation}:

  \begin{itemize}
  \tightlist
  \item
    Select dormant scion with 2-4 buds
  \item
    Make matching angle cut and tongue
  \item
    Ensure cambium is visible on both sides
  \end{itemize}
\item
  \textbf{Joining}:

  \begin{itemize}
  \tightlist
  \item
    Insert scion tongue into rootstock notch
  \item
    Align cambium layers precisely on both sides
  \item
    Ensure tight fit with no gaps
  \end{itemize}
\item
  \textbf{Securing}:

  \begin{itemize}
  \tightlist
  \item
    Wrap with grafting tape, starting below union
  \item
    Overlap tape by 50\% for complete coverage
  \item
    Seal exposed surfaces with grafting wax
  \end{itemize}
\item
  \textbf{Post-Grafting Care}:

  \begin{itemize}
  \tightlist
  \item
    Protect from direct sunlight
  \item
    Maintain humidity 70-90\%
  \item
    Monitor for 4-6 weeks
  \item
    Remove tape after union forms
  \end{itemize}
\end{enumerate}
\end{block}

\begin{block}{S1.2.2 Cleft Grafting - Detailed Protocol}
\protect\phantomsection\label{s1.2.2-cleft-grafting---detailed-protocol}
\textbf{Optimal Conditions}: - Rootstock diameter: 10-50 mm - Timing:
Late winter to early spring - Temperature: 15-25°C - Humidity: 70-85\%

\textbf{Procedure Details}:

\begin{enumerate}
\tightlist
\item
  \textbf{Rootstock Preparation}:

  \begin{itemize}
  \tightlist
  \item
    Cut rootstock horizontally at desired height
  \item
    Make vertical split 3-5 cm deep using grafting tool
  \item
    Keep split open with wedge if needed
  \end{itemize}
\item
  \textbf{Scion Preparation}:

  \begin{itemize}
  \tightlist
  \item
    Select scion with 2-3 buds
  \item
    Make wedge-shaped cut (30-40° angle on both sides)
  \item
    Ensure cambium exposed on both sides of wedge
  \end{itemize}
\item
  \textbf{Insertion}:

  \begin{itemize}
  \tightlist
  \item
    Insert scion into cleft, aligning cambium on one side
  \item
    For large rootstock, insert 2 scions (one on each side)
  \item
    Remove wedge and allow rootstock to close
  \end{itemize}
\item
  \textbf{Sealing}:

  \begin{itemize}
  \tightlist
  \item
    Apply grafting wax to all exposed surfaces
  \item
    Cover entire union area
  \item
    Protect from weather
  \end{itemize}
\end{enumerate}
\end{block}
\end{block}

\begin{block}{S1.3 Regional Variations and Adaptations}
\protect\phantomsection\label{s1.3-regional-variations-and-adaptations}
\begin{block}{S1.3.1 Mediterranean Techniques}
\protect\phantomsection\label{s1.3.1-mediterranean-techniques}
Mediterranean grafting practices emphasize: - Timing: Late fall to early
spring - Emphasis on olive and citrus grafting - Use of traditional
tools (grafting knives, waxes) - Emphasis on water management
post-grafting
\end{block}

\begin{block}{S1.3.2 Asian Techniques}
\protect\phantomsection\label{s1.3.2-asian-techniques}
Asian grafting traditions include: - Emphasis on precision and alignment
- Use of specialized tools for delicate operations - Integration with
traditional agricultural calendars - Focus on ornamental and fruit tree
combinations
\end{block}

\begin{block}{S1.3.3 Tropical Adaptations}
\protect\phantomsection\label{s1.3.3-tropical-adaptations}
Tropical grafting adaptations: - Year-round grafting potential -
Emphasis on humidity management - Protection from intense sunlight -
Disease prevention measures
\end{block}
\end{block}

\begin{block}{S1.4 Tool Specifications and Requirements}
\protect\phantomsection\label{s1.4-tool-specifications-and-requirements}
\begin{block}{S1.4.1 Grafting Knives}
\protect\phantomsection\label{s1.4.1-grafting-knives}
\textbf{Essential Characteristics}: - Sharp, single-bevel blade - Blade
length: 5-8 cm - Handle: Comfortable grip, non-slip - Material:
High-carbon steel or stainless steel

\textbf{Maintenance}: - Regular sharpening to maintain edge -
Sterilization between uses - Proper storage to prevent rust
\end{block}

\begin{block}{S1.4.2 Grafting Tape and Wax}
\protect\phantomsection\label{s1.4.2-grafting-tape-and-wax}
\textbf{Grafting Tape}: - Material: Polyethylene or rubber-based -
Width: 1-2 cm - Stretchability: 200-300\% elongation - UV resistance for
outdoor use

\textbf{Grafting Wax}: - Composition: Beeswax, resin, and oil - Melting
point: 60-70°C - Application temperature: 80-90°C - Protection duration:
3-6 months
\end{block}
\end{block}

\begin{block}{S1.5 Specialized Grafting Methods}
\protect\phantomsection\label{s1.5-specialized-grafting-methods}
\begin{block}{S1.5.1 Nurse Seed Grafting}
\protect\phantomsection\label{s1.5.1-nurse-seed-grafting}
Used for difficult species or very young rootstock: - Graft scion to
temporary nurse plant - Allow union to form - Transfer to permanent
rootstock - Success rate: 50-65\%
\end{block}

\begin{block}{S1.5.2 Four-Flap Grafting}
\protect\phantomsection\label{s1.5.2-four-flap-grafting}
Advanced technique for large diameter rootstock: - Create four flaps on
rootstock - Prepare scion with matching cuts - Insert and align cambium
- Success rate: 70-80\%
\end{block}

\begin{block}{S1.5.3 Chip Budding}
\protect\phantomsection\label{s1.5.3-chip-budding}
Variation of bud grafting: - Remove chip of bark with bud - Insert into
matching cut on rootstock - Simpler than T-budding - Success rate:
75-85\%
\end{block}
\end{block}

\begin{block}{S1.6 Quality Control Measures}
\protect\phantomsection\label{s1.6-quality-control-measures}
\begin{block}{S1.6.1 Pre-Grafting Assessment}
\protect\phantomsection\label{s1.6.1-pre-grafting-assessment}
Before grafting, assess: - Rootstock health and vigor - Scion quality
and dormancy - Diameter matching (within 10-20\%) - Environmental
conditions - Tool condition and sterility
\end{block}

\begin{block}{S1.6.2 Post-Grafting Monitoring}
\protect\phantomsection\label{s1.6.2-post-grafting-monitoring}
Monitor grafts for: - Union formation (visual inspection) - Callus
development (4-7 days) - Vascular connection (14-28 days) - Scion growth
initiation - Signs of rejection or disease
\end{block}

\begin{block}{S1.6.3 Success Evaluation}
\protect\phantomsection\label{s1.6.3-success-evaluation}
Evaluate success at: - \textbf{30 days}: Initial union formation -
\textbf{60 days}: Vascular connection established - \textbf{90 days}:
Full union strength - \textbf{1 year}: Long-term compatibility

Success criteria: - Visible callus formation - Scion bud break and
growth - No signs of rejection - Strong union (resistance to movement)
\end{block}
\end{block}
\end{frame}

\end{document}
