% Options for packages loaded elsewhere
\PassOptionsToPackage{unicode}{hyperref}
\PassOptionsToPackage{hyphens}{url}
\documentclass[
  ignorenonframetext,
]{beamer}
\newif\ifbibliography
\usepackage{pgfpages}
\setbeamertemplate{caption}[numbered]
\setbeamertemplate{caption label separator}{: }
\setbeamercolor{caption name}{fg=normal text.fg}
\beamertemplatenavigationsymbolsempty
% remove section numbering
\setbeamertemplate{part page}{
  \centering
  \begin{beamercolorbox}[sep=16pt,center]{part title}
    \usebeamerfont{part title}\insertpart\par
  \end{beamercolorbox}
}
\setbeamertemplate{section page}{
  \centering
  \begin{beamercolorbox}[sep=12pt,center]{section title}
    \usebeamerfont{section title}\insertsection\par
  \end{beamercolorbox}
}
\setbeamertemplate{subsection page}{
  \centering
  \begin{beamercolorbox}[sep=8pt,center]{subsection title}
    \usebeamerfont{subsection title}\insertsubsection\par
  \end{beamercolorbox}
}
% Prevent slide breaks in the middle of a paragraph
\widowpenalties 1 10000
\raggedbottom
\AtBeginPart{
  \frame{\partpage}
}
\AtBeginSection{
  \ifbibliography
  \else
    \frame{\sectionpage}
  \fi
}
\AtBeginSubsection{
  \frame{\subsectionpage}
}
\usepackage{iftex}
\ifPDFTeX
  \usepackage[T1]{fontenc}
  \usepackage[utf8]{inputenc}
  \usepackage{textcomp} % provide euro and other symbols
\else % if luatex or xetex
  \usepackage{unicode-math} % this also loads fontspec
  \defaultfontfeatures{Scale=MatchLowercase}
  \defaultfontfeatures[\rmfamily]{Ligatures=TeX,Scale=1}
\fi
\usepackage{lmodern}
\ifPDFTeX\else
  % xetex/luatex font selection
\fi
% Use upquote if available, for straight quotes in verbatim environments
\IfFileExists{upquote.sty}{\usepackage{upquote}}{}
\IfFileExists{microtype.sty}{% use microtype if available
  \usepackage[]{microtype}
  \UseMicrotypeSet[protrusion]{basicmath} % disable protrusion for tt fonts
}{}
\makeatletter
\@ifundefined{KOMAClassName}{% if non-KOMA class
  \IfFileExists{parskip.sty}{%
    \usepackage{parskip}
  }{% else
    \setlength{\parindent}{0pt}
    \setlength{\parskip}{6pt plus 2pt minus 1pt}}
}{% if KOMA class
  \KOMAoptions{parskip=half}}
\makeatother
\setlength{\emergencystretch}{3em} % prevent overfull lines
\providecommand{\tightlist}{%
  \setlength{\itemsep}{0pt}\setlength{\parskip}{0pt}}
\usepackage{bookmark}
\IfFileExists{xurl.sty}{\usepackage{xurl}}{} % add URL line breaks if available
\urlstyle{same}
\hypersetup{
  hidelinks,
  pdfcreator={LaTeX via pandoc}}

\author{\texorpdfstring{}{}}
\date{}

\begin{document}

\begin{frame}{Abstract}
\protect\phantomsection\label{abstract}
This paper presents a comprehensive analysis of gradient descent
optimization algorithms applied to quadratic minimization problems. We
implement and evaluate the classical gradient descent method with fixed
step size, examining convergence behavior across a range of learning
rates from \(\alpha = 0.01\) to \(\alpha = 0.20\). Our experimental
framework includes theoretical convergence bounds, numerical stability
analysis, and performance benchmarking using infrastructure-backed
scientific utilities.

The key contributions of this work are: (1) a rigorously tested
implementation of gradient descent with 96\%+ test coverage and
deterministic reproducibility via fixed random seeds; (2) empirical
validation of theoretical convergence rates on quadratic objective
functions; (3) automated analysis pipelines generating
publication-quality visualizations; and (4) integration patterns
demonstrating how optimization algorithms connect with infrastructure
modules for logging, validation, and performance monitoring.

Results confirm that all tested step sizes converge to the analytical
optimum \(x^* = 1.0\) with objective value \(f(x^*) = -0.5\), with
larger step sizes achieving faster convergence (9 iterations for
\(\alpha = 0.20\) versus 165 iterations for \(\alpha = 0.01\)). The
implementation validates the template's capability to support
computational research projects from algorithm development through
manuscript generation, serving as an exemplar for reproducible numerical
optimization studies.

\textbf{Keywords:} gradient descent, numerical optimization, convergence
analysis, quadratic minimization, reproducible research
\end{frame}

\end{document}
