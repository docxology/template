% Options for packages loaded elsewhere
\PassOptionsToPackage{unicode}{hyperref}
\PassOptionsToPackage{hyphens}{url}
\documentclass[
  ignorenonframetext,
]{beamer}
\newif\ifbibliography
\usepackage{pgfpages}
\setbeamertemplate{caption}[numbered]
\setbeamertemplate{caption label separator}{: }
\setbeamercolor{caption name}{fg=normal text.fg}
\beamertemplatenavigationsymbolsempty
% remove section numbering
\setbeamertemplate{part page}{
  \centering
  \begin{beamercolorbox}[sep=16pt,center]{part title}
    \usebeamerfont{part title}\insertpart\par
  \end{beamercolorbox}
}
\setbeamertemplate{section page}{
  \centering
  \begin{beamercolorbox}[sep=12pt,center]{section title}
    \usebeamerfont{section title}\insertsection\par
  \end{beamercolorbox}
}
\setbeamertemplate{subsection page}{
  \centering
  \begin{beamercolorbox}[sep=8pt,center]{subsection title}
    \usebeamerfont{subsection title}\insertsubsection\par
  \end{beamercolorbox}
}
% Prevent slide breaks in the middle of a paragraph
\widowpenalties 1 10000
\raggedbottom
\AtBeginPart{
  \frame{\partpage}
}
\AtBeginSection{
  \ifbibliography
  \else
    \frame{\sectionpage}
  \fi
}
\AtBeginSubsection{
  \frame{\subsectionpage}
}
\usepackage{iftex}
\ifPDFTeX
  \usepackage[T1]{fontenc}
  \usepackage[utf8]{inputenc}
  \usepackage{textcomp} % provide euro and other symbols
\else % if luatex or xetex
  \usepackage{unicode-math} % this also loads fontspec
  \defaultfontfeatures{Scale=MatchLowercase}
  \defaultfontfeatures[\rmfamily]{Ligatures=TeX,Scale=1}
\fi
\usepackage{lmodern}
\ifPDFTeX\else
  % xetex/luatex font selection
\fi
% Use upquote if available, for straight quotes in verbatim environments
\IfFileExists{upquote.sty}{\usepackage{upquote}}{}
\IfFileExists{microtype.sty}{% use microtype if available
  \usepackage[]{microtype}
  \UseMicrotypeSet[protrusion]{basicmath} % disable protrusion for tt fonts
}{}
\makeatletter
\@ifundefined{KOMAClassName}{% if non-KOMA class
  \IfFileExists{parskip.sty}{%
    \usepackage{parskip}
  }{% else
    \setlength{\parindent}{0pt}
    \setlength{\parskip}{6pt plus 2pt minus 1pt}}
}{% if KOMA class
  \KOMAoptions{parskip=half}}
\makeatother
\setlength{\emergencystretch}{3em} % prevent overfull lines
\providecommand{\tightlist}{%
  \setlength{\itemsep}{0pt}\setlength{\parskip}{0pt}}
\usepackage{bookmark}
\IfFileExists{xurl.sty}{\usepackage{xurl}}{} % add URL line breaks if available
\urlstyle{same}
\hypersetup{
  hidelinks,
  pdfcreator={LaTeX via pandoc}}

\author{\texorpdfstring{}{}}
\date{}

\begin{document}

\begin{frame}{Introduction}
\protect\phantomsection\label{introduction}
This small code project demonstrates a fully-tested numerical
optimization implementation with comprehensive analysis and
visualization capabilities. The project showcases the complete research
pipeline from algorithm implementation through testing to result
visualization.

\begin{block}{Research Context}
\protect\phantomsection\label{research-context}
Numerical optimization forms the foundation of many scientific and
engineering applications. This project implements and analyzes gradient
descent methods for solving optimization problems of the form:

\[\min_{x \in \mathbb{R}^n} f(x)\]

where \(f: \mathbb{R}^n \rightarrow \mathbb{R}\) is a continuously
differentiable objective function.
\end{block}

\begin{block}{Key Components}
\protect\phantomsection\label{key-components}
The implementation includes:

\begin{itemize}
\tightlist
\item
  \textbf{Gradient descent algorithm} with configurable parameters
\item
  \textbf{Quadratic function test problems} with known analytical
  solutions
\item
  \textbf{Comprehensive test suite} covering functionality and edge
  cases
\item
  \textbf{Analysis scripts} that generate convergence plots and
  performance data
\item
  \textbf{Manuscript integration} with automatically generated figures
\end{itemize}
\end{block}

\begin{block}{Algorithm Overview}
\protect\phantomsection\label{algorithm-overview}
The gradient descent algorithm iteratively updates the solution using:

\[x_{k+1} = x_k - \alpha \nabla f(x_k)\]

where: - \(\alpha > 0\) is the step size (learning rate) -
\(\nabla f(x_k)\) is the gradient of the objective function at iteration
\(k\)
\end{block}

\begin{block}{Implementation Goals}
\protect\phantomsection\label{implementation-goals}
This project demonstrates:

\begin{enumerate}
\tightlist
\item
  \textbf{Clean, testable code} with proper separation of concerns
\item
  \textbf{Numerical accuracy} through comprehensive testing
\item
  \textbf{Performance analysis} with convergence visualization
\item
  \textbf{Research reproducibility} through automated analysis scripts
\item
  \textbf{Documentation integration} with figure generation and
  referencing
\end{enumerate}
\end{block}
\end{frame}

\end{document}
