% Options for packages loaded elsewhere
\PassOptionsToPackage{unicode}{hyperref}
\PassOptionsToPackage{hyphens}{url}
\documentclass[
]{article}
\usepackage{xcolor}
\usepackage[margin=1in]{geometry}
\usepackage{amsmath,amssymb}
\setcounter{secnumdepth}{5}
\usepackage{iftex}
\ifPDFTeX
  \usepackage[T1]{fontenc}
  \usepackage[utf8]{inputenc}
  \usepackage{textcomp} % provide euro and other symbols
\else % if luatex or xetex
  \usepackage{unicode-math} % this also loads fontspec
  \defaultfontfeatures{Scale=MatchLowercase}
  \defaultfontfeatures[\rmfamily]{Ligatures=TeX,Scale=1}
\fi
% \usepackage{lmodern}
\ifPDFTeX\else
  % xetex/luatex font selection
\fi
% Use upquote if available, for straight quotes in verbatim environments
\IfFileExists{upquote.sty}{\usepackage{upquote}}{}
\IfFileExists{microtype.sty}{% use microtype if available
  \usepackage[]{microtype}
  \UseMicrotypeSet[protrusion]{basicmath} % disable protrusion for tt fonts
}{}
\makeatletter
\@ifundefined{KOMAClassName}{% if non-KOMA class
  \IfFileExists{parskip.sty}{%
    \usepackage{parskip}
  }{% else
    \setlength{\parindent}{0pt}
    \setlength{\parskip}{6pt plus 2pt minus 1pt}}
}{% if KOMA class
  \KOMAoptions{parskip=half}}
\makeatother
\usepackage{longtable,booktabs,array}
\newcounter{none} % for unnumbered tables
\usepackage{calc} % for calculating minipage widths
% Correct order of tables after \paragraph or \subparagraph
\usepackage{etoolbox}
\makeatletter
\patchcmd\longtable{\par}{\if@noskipsec\mbox{}\fi\par}{}{}
\makeatother
% Allow footnotes in longtable head/foot
\IfFileExists{footnotehyper.sty}{\usepackage{footnotehyper}}{\usepackage{footnote}}
\makesavenoteenv{longtable}
\usepackage{graphicx}
\makeatletter
\newsavebox\pandoc@box
\newcommand*\pandocbounded[1]{% scales image to fit in text height/width
  \sbox\pandoc@box{#1}%
  \Gscale@div\@tempa{\textheight}{\dimexpr\ht\pandoc@box+\dp\pandoc@box\relax}%
  \Gscale@div\@tempb{\linewidth}{\wd\pandoc@box}%
  \ifdim\@tempb\p@<\@tempa\p@\let\@tempa\@tempb\fi% select the smaller of both
  \ifdim\@tempa\p@<\p@\scalebox{\@tempa}{\usebox\pandoc@box}%
  \else\usebox{\pandoc@box}%
  \fi%
}
% Set default figure placement to htbp
\def\fps@figure{htbp}
\makeatother
\setlength{\emergencystretch}{3em} % prevent overfull lines
\providecommand{\tightlist}{%
  \setlength{\itemsep}{0pt}\setlength{\parskip}{0pt}}
\usepackage[]{natbib}
\bibliographystyle{plainnat}
\usepackage{bookmark}
\IfFileExists{xurl.sty}{\usepackage{xurl}}{} % add URL line breaks if available
\urlstyle{same}
\hypersetup{
  hidelinks,
  pdfcreator={LaTeX via pandoc}}

\author{}
\date{}

\title{The Doors of Perception are the Threshold of Prediction\\\normalsize Active Inference and William Blake's Theory of Seeing}
\author{Daniel Ari Friedman\\\footnotesize{Active Inference Institute}\\\footnotesize{\texttt{daniel@activeinference.institute}}\\\footnotesize{\href{https://orcid.org/0000-0001-6232-9096}{ORCID: 0000-0001-6232-9096}} \\ \footnotesize{\href{https://doi.org/10.5281/zenodo.18600041}{DOI: 10.5281/zenodo.18600041}} \\ \footnotesize{February 12, 2026}}

\usepackage{graphicx}
\usepackage{amsmath}
\usepackage{amssymb}
\usepackage{amsfonts}
\usepackage{amsthm}
\usepackage{geometry}
\usepackage{float}
\usepackage{booktabs}
\usepackage{longtable}
\usepackage{array}
\usepackage{xcolor}
\definecolor{darkred}{RGB}{139,0,0}
\usepackage{hyperref}
\hypersetup{
  colorlinks=true,
  linkcolor=darkred,
  citecolor=darkred,
  urlcolor=darkred
}
\usepackage{natbib}
\date{\vspace{0.5cm} \includegraphics[width=6cm]{../figures/ancient_of_days_blake_bw.jpg}}

\begin{document}

\maketitle
\thispagestyle{empty}


\newpage

\section{Thematic Atlas}\label{thematic-atlas}

\emph{``What is now proved was once only imagined.''} --- Blake

Here, eight concordances are (sk)etched between Blake's prophetic vision
and the mathematics of Active Inference.

\subsection{Table of Contents by
Theme}\label{table-of-contents-by-theme}

{\def\LTcaptype{none} % do not increment counter
\begin{longtable}[]{@{}
  >{\raggedright\arraybackslash}p{(\linewidth - 6\tabcolsep) * \real{0.1552}}
  >{\raggedright\arraybackslash}p{(\linewidth - 6\tabcolsep) * \real{0.2759}}
  >{\raggedright\arraybackslash}p{(\linewidth - 6\tabcolsep) * \real{0.3103}}
  >{\raggedright\arraybackslash}p{(\linewidth - 6\tabcolsep) * \real{0.2586}}@{}}
\toprule\noalign{}
\begin{minipage}[b]{\linewidth}\raggedright
Section
\end{minipage} & \begin{minipage}[b]{\linewidth}\raggedright
Blake's Vision
\end{minipage} & \begin{minipage}[b]{\linewidth}\raggedright
Active Inference
\end{minipage} & \begin{minipage}[b]{\linewidth}\raggedright
Key Quotation
\end{minipage} \\
\midrule\noalign{}
\endhead
\bottomrule\noalign{}
\endlastfoot
\href{04a_boundary.md}{Boundary} & Doors of Perception & Markov Blanket
& ``If the doors of perception were cleansed\ldots{}'' \\
\href{04b_vision.md}{Vision} & Fourfold Vision & Hierarchical Processing
& ``Now I a fourfold vision see\ldots{}'' \\
\href{04c_states.md}{States} & Newton's Sleep & Prior Dominance &
``Single vision \& Newton's sleep!'' \\
\href{04d_imagination.md}{Imagination} & Human Existence & Generative
Model & ``Imagination is Human Existence itself'' \\
\href{04e_time.md}{Time} & Eternity in Hour & Temporal Horizons &
``Hold\ldots{} Eternity in an hour'' \\
\href{04f_space.md}{Space} & World in Grain & Spatial Inference & ``To
see a World in a Grain of Sand'' \\
\href{04g_action.md}{Action} & Cleansing & Free Energy Minimization &
``every thing would appear\ldots{} as it is: infinite'' \\
\href{04h_collectives.md}{Collectives} & Building Jerusalem & Shared \&
Factorized Models & ``Till we have built Jerusalem\ldots{}'' \\
\end{longtable}
}

\tableofcontents
\newpage

\newpage

\section{Abstract: The Prophetic
Synthesis}\label{abstract-the-prophetic-synthesis}

Looking at the sun, William Blake saw an innumerable company of the
heavenly host where Newton's heirs saw only a golden coin. ``If the
doors of perception were cleansed,'' Blake wrote, ``every thing would
appear to man as it is: infinite.'' This paper argues that Blake's
prophetic vocabulary, far from being merely poetic, constitutes an
anticipatory phenomenological insight into the cognitive architecture
that Active Inference now formalizes mathematically. Blake's ``doors''
are statistical boundaries separating self from world; his ``Newton's
sleep'' is the pathology of rigid priors crushing sensory evidence; his
``fourfold vision'' maps onto hierarchical precision-weighting across
processing depths; his insistence that ``Imagination is the Human
Existence itself'' anticipates the insight that selfhood is constituted
by the generative model. These are not retrospective metaphors imposed
on a Romantic poet, but convergent descriptions of the same perceptual
territory, arrived at through radically different methods two centuries
apart. We approach this convergence in the spirit of Hesse's Glass Bead
Game: not as proof that one tradition vindicates or completes the other,
but as a synthetic juxtaposition of Art and Science---two moves in the
same ancient, ongoing game of making sense of sense-making.

Through close reading of \emph{The Marriage of Heaven and Hell},
\emph{Milton}, \emph{Jerusalem}, and other works, we trace eight
structural correspondences between Blake's perceptual philosophy and the
Active Inference framework: Boundary, Vision, States, Imagination, Time,
Space, Action, and Collectives---the last encompassing Blake's Four Zoas
as a factorized model of collective mind. Each correspondence begins
with Blake's phenomenological fire---his exact words, his illuminated
images---and follows the mathematical shadow that Active Inference casts
across the same ground: Markov blankets, hierarchical generative models,
precision dynamics, temporal depth, spatial inference, free energy
minimization, and multi-agent coordination. The formalism developed by
the Active Inference community provides mathematical precision, yet we
resist treating it as a finished edifice; the framework is better
understood as one contemporary articulation of principles that Blake,
and traditions before him, grasped through other means. The synthesis
contributes to both lineages: Blake scholarship gains formal grounding
of insights long dismissed as mystical enthusiasm; cognitive science
gains phenomenological depth, historical precedent, and the humbling
recognition that its discoveries may be rediscoveries after all. The
doors of perception have always been thresholds of prediction---Blake's
visions and the equations point towards the same boundary, and the
conversation between them remains open evermore.

Epistemic status: ``delighted with the enjoyments'' of AI which ``look
like torment and insanity''. Take all syntax and semantics with a
``grain of sand''. For my personal limitations and typographical errors
I plead ``Mutual Forgiveness of each Vice''.

\textbf{Keywords:} Active Inference · William Blake · Free Energy
Principle · Predictive Processing · Markov Blanket · Generative Model ·
Philosophy of Mind · Romanticism · Glass Bead Game

\newpage

\section{Introduction: The Threshold}\label{introduction}

\begin{quote}
\emph{``If the doors of perception were cleansed every thing would
appear to man as it is: infinite. For man has closed himself up, till he
sees all things thro' narrow chinks of his cavern.''}

--- Blake, \emph{Marriage of Heaven and Hell}, Plate 14
\citep{blake1790marriage}
\end{quote}

\subsection{The Threshold}\label{threshold}

Between perceiver and perceived lies a boundary. Blake called it a door.
In causal inference, that boundary may be called a blanket. The exoteric
syntax differs; the esoteric semantics does not.

William Blake (1757--1827) composed his prophetic works during the
consolidation of Newtonian mechanism---the reduction of cosmos to
clockwork, of vision to optics, of mind to matter arranged
\citep{raine1968blake}. His response was not retreat into mysticism but
a vigorous \emph{expansion}: a fourfold epistemology that could contain
Newton's single vision while transcending it.

The Active Inference framework, developed by a growing community of
researchers worldwide, offers a formal complement to Blake's insights.
The free energy principle formalizes how self-organizing systems
maintain existence by minimizing prediction error
\citep{friston2010free}. Perception and action unite in a single
imperative---to reduce the gap between expectation and evidence.

This paper explores how Blake's intuitions and Active Inference's
equations resonate. The former prophesied; the latter formalizes. We do
not claim that Blake was a proto-Bayesian statistician, nor that Active
Inference is a ``Blakean'' science, but rather that both systems grapple
with the same fundamental problem: how a bounded agent maintains its
existence and makes sense of an infinite world. We offer a
\emph{synthetic juxtapositional intelligence}---placing the poet's
vision alongside the physicist's variables to reveal the structural
identity of their insights.

The spirit of this enterprise owes something to Hesse's \emph{Glass Bead
Game}: an abstract synthesis of all arts and sciences, where the player
discovers hidden affinities between seemingly unrelated disciplines
\citep{hesse1943glass}. Like Hesse's Castalian scholars, we do not seek
to reduce one tradition to the other, but to illuminate the structural
resonances that emerge when both are held in the same contemplative
field.

This synthesis arrives at a moment of convergence. On one side,
predictive processing and active inference are being applied with
increasing sophistication to aesthetics and literary engagement---most
notably in the 2024 \emph{Philosophical Transactions of the Royal
Society B} theme issue on art and predictive processing
\citep{vandecruys2024order}, and in Kukkonen's work modeling literary
experience through prediction error \citep{kukkonen2020probability}. On
the other, cognitive approaches to Romanticism are deepening: Savarese's
\emph{Romanticism's Other Minds} \citep{savarese2020romanticism} reveals
a ``prehistory of cognitive approaches to literature'' within the
Romantic tradition itself. Our paper sits at the intersection of these
two currents, offering what neither can alone: the formal mathematics
that makes the poetic claim testable, and the phenomenological richness
that makes the formalism legible.

\subsection{The Correspondences}\label{correspondences}

Eight thematic correspondences anchor our synthesis (see
\hyperref[tbl-themes]{Thematic Atlas}):

\begin{longtable}[]{@{}lll@{}}
\caption{Thematic Atlas: Structural correspondences between Blake's
visionary phenomenology and Active
Inference.}\label{tbl-themes}\tabularnewline
\toprule\noalign{}
Theme & Blake's Term & Active Inference Term \\
\midrule\noalign{}
\endfirsthead
\toprule\noalign{}
Theme & Blake's Term & Active Inference Term \\
\midrule\noalign{}
\endhead
\bottomrule\noalign{}
\endlastfoot
\textbf{Boundary} & Doors of Perception & Markov Blanket \\
\textbf{Vision} & Fourfold Vision & Hierarchical Processing \\
\textbf{States} & Newton's Sleep & Prior Dominance \\
\textbf{Imagination} & Human Existence & Generative Model \\
\textbf{Time} & Eternity in an Hour & Temporal Horizons \\
\textbf{Space} & World in a Grain of Sand & Spatial Inference \\
\textbf{Action} & Cleansing & Free Energy Minimization \\
\textbf{Collectives} & Building Jerusalem & Shared Generative Models \\
\end{longtable}

\begin{quote}
\emph{``May God us keep / From Single vision \& Newton's sleep!''}

--- Blake, Letter to Butts, November 1802 \citep{blake1802butts}
\end{quote}

\subsection{Method}\label{method}

We proceed now through three main movements:

\begin{itemize}
\tightlist
\item
  \textbf{§2}: Related scholarship: Blake and cognition, Romanticism and
  neuroscience, situating our contribution
\item
  \textbf{§3}: Theoretical Foundations: free energy, Markov blankets,
  precision
\item
  \textbf{§4}: Synthesis: eight themed correspondences with equations
  and figures
\end{itemize}

Each theme in our synthesis (\hyperref[tbl-themes]{Thematic Atlas})
begins with Blake's fire, then traces its mathematical shadow. The
conclusion (\hyperref[implications]{§5--6}) draws implications for
philosophy of mind, cognitive science, and creativity, while engaging
counter-arguments and acknowledging limitations.

\newpage

\section{Related Work: Scholarship \& Context}\label{related-work}

\emph{Situating the correspondence within existing scholarship.}

\subsection{Blake and Embodied
Cognition}\label{blake-and-embodied-cognition}

The mapping between Romantic poetry and cognitive science has
precursors. Three pillars of Blake scholarship make our formal mapping
possible. Northrop Frye's \emph{Fearful Symmetry}
\citep{frye1947fearful} established the systematic reading of Blake's
symbolism as a coherent intellectual structure rather than private
mythology. S. Foster Damon's \emph{A Blake Dictionary}
\citep{damon1988blake} provides the essential lexicon of Blake's
symbolic system, establishing the correspondences among his mythological
figures that a formal mapping requires. Peter Ackroyd's definitive
biography \citep{ackroyd1995blake} demonstrates how Blake's visionary
epistemology was inseparable from his lived practice as engraver,
printer, and painter---an embodied creativity that resists reduction to
disembodied ideas.

From the cognitive science side, two works converge on the same insight.
Mark Johnson's \emph{The Body in the Mind} \citep{johnson1987body}
argues that abstract thought is grounded in embodied image
schemas---exactly the kind of perceptual-motor structures that Active
Inference formalizes as generative models. Lakoff and Johnson's
\emph{Metaphors We Live By} \citep{lakoff1980metaphors} demonstrated
that conceptual structure is metaphorical and embodied, not abstract and
propositional.

Blake anticipated both traditions by two centuries. His insistence that
``Man has no Body distinct from his Soul'' (MHH Plate 4) is not metaphor
but proto-enactivism: the body is not a container for mind but the very
medium of inference.

\subsection{Hemispheric
Lateralization}\label{hemispheric-lateralization}

Iain McGilchrist's \emph{The Master and His Emissary}
\citep{mcgilchrist2009master} proposes that the left hemisphere
prioritizes narrow, focused, already-known categories while the right
hemisphere attends to the broad, contextual, and novel. This
lateralization maps suggestively onto Blake's mythology: Urizen (left
hemisphere)---the lawgiver who ``closed the tent of the Universe,''
imposing rigid categories---versus Los/Urthona (right hemisphere)---the
creative imagination that builds Golgonooza, perpetually open to new
form. McGilchrist's thesis that Western civilization has progressively
over-valued left-hemispheric cognition parallels Blake's diagnosis of
``Newton's sleep'' as civilizational pathology.

McGilchrist's magisterial follow-up, \emph{The Matter with Things}
\citep{mcgilchrist2021matter}, deepens this analysis with direct
engagement with Blake. McGilchrist treats imagination not as mere
fantasy but as a ``key faculty'' for revealing reality---echoing Blake's
own elevation of imagination above reason. The dynamic tension of
Blake's \emph{Marriage of Heaven and Hell}, where ``contraries''
generate movement toward deeper consciousness, exemplifies what
McGilchrist identifies as the right hemisphere's mode of understanding:
holding opposites in creative tension rather than collapsing them into
categories.

\subsection{Romanticism and the Science of
Mind}\label{romanticism-and-the-science-of-mind}

Alan Richardson's \emph{British Romanticism and the Science of the Mind}
\citep{richardson2001british} documents how Romantic poets engaged
seriously with contemporary brain science, not as opponents but as
creative interlocutors. Richardson shows that the Romantic critique of
mechanism was not anti-scientific but proto-cognitive---anticipating
embodied, situated, and enactive approaches. Our paper extends
Richardson's historical argument by providing the formal bridge: Active
Inference supplies the mathematics that connects Blake's
phenomenological observations to contemporary computational
neuroscience.

\subsection{Neuroaesthetics}\label{neuroaesthetics}

The emerging field of neuroaesthetics investigates how art engages
perceptual and cognitive systems. Ramachandran and Hirstein
\citep{ramachandran1999science} proposed that art exploits principles of
perceptual processing---peak shift, isolation, and grouping. In Active
Inference terms, art offers generative models that resolve free energy
in novel ways, restructuring the viewer's predictions. Blake's
illuminated books---integrating visual, verbal, and material elements
into composite artworks---represent an extreme case: each plate offers
not merely an aesthetic experience but a complete alternative generative
model for perception.

\subsection{Social Neuroscience and Joint
Improvisation}\label{social-neuroscience-and-joint-improvisation}

Recent work in social neuroscience and art therapy emphasizes the role
of joint improvisation in synchronizing neural states. Mikhailova and
Friedman's ``Partner Pen Play in Parallel'' (PPPiP)
\citep{mikhailova2018partner} proposes that simultaneous, non-verbal
co-creation on a shared surface facilitates ``controlled novelty'' and
inter-brain synchrony. This practice operationalizes the Active
Inference account of communication not merely as signal transmission but
as the mutual alignment of generative models. Just as Blake's ``fourfold
vision'' integrates diverse faculties, PPPiP demonstrates how shared
aesthetic action can construct a ``collective self-evidencing'' dynamic,
where the relationship itself becomes the agent minimizing surprise.

\subsection{Psychedelics and the Predictive
Mind}\label{psychedelics-and-the-predictive-mind}

Aldous Huxley's \emph{The Doors of Perception}
\citep{huxley1954doors}---its very title drawn from Blake---proposed
that psychedelic experience reveals perception ordinarily filtered by
the brain's ``reducing valve.'' Carhart-Harris and Friston's REBUS model
\citep{carthartharris2019rebus} formalized this intuition, showing that
psychedelics reduce the precision of high-level priors. Safron and
colleagues' ALBUS framework \citep{safron2025albus} now extends REBUS
into a comprehensive account: psychedelics can both relax beliefs and
strengthen them, producing the full spectrum of altered states from
prior dissolution to intensified meaning-making. This is Blake's
``cleansing'' rendered computational---the doors of perception swing
open when prior dynamics shift, allowing sensory evidence to reshape the
model. The continuity from Blake through Huxley to ALBUS illustrates how
the same phenomenological insight has been rediscovered across centuries
and progressively formalized.

\subsection{Northrop Frye's Systematic
Blake}\label{northrop-fryes-systematic-blake}

Frye's \emph{Fearful Symmetry} \citep{frye1947fearful} remains the
foundational systematic treatment of Blake's mythology. Frye
demonstrated that Blake's prophetic books constitute a coherent
cosmological system, not isolated flights of fancy. Our paper depends on
Frye's insight that Blake's symbolism is systematic---without that
systematicity, the structural correspondences with Active Inference
would dissolve into vague analogy. Where Frye mapped Blake's system as
literary criticism, we map it as cognitive architecture.

\subsection{Comparative Systems: Blake and
Fuller}\label{comparative-systems-blake-and-fuller}

While Frye elucidated the internal coherence of Blake's system, recent
comparative work highlights Blake's role as a system-\emph{builder} akin
to modern comprehensivists. Friedman's study of Blake and Buckminster
Fuller \citep{friedman2023blake} juxtaposes Blake's mythopoetic
architecture with the Synergetics of Fuller and Ed J Applewhite. Both
thinkers confronted the ``single vision'' of their respective
eras---Newtonian mechanics for Blake, specialization and technocracy for
Fuller---by constructing comprehensive, fourfold (or tetrahedral)
epistemologies. This comparison underscores that Blake's ``system'' was
not a static dogma but a dynamic \emph{tool for thought}, designed to
prevent enslavement by another's system.

\subsection{Phenomenological
Traditions}\label{phenomenological-traditions}

The phenomenological tradition provides crucial methodological
precedent. Merleau-Ponty's \emph{Phenomenology of Perception}
\citep{merleau1962phenomenology} argues that perception is fundamentally
embodied---the body is not an object among objects but the condition of
objecthood itself. This directly parallels Active Inference's claim that
the Markov blanket \emph{constitutes} the distinction between agent and
environment. Blake's rejection of Cartesian dualism---``Man has no Body
distinct from his Soul''---anticipates Merleau-Ponty's overcoming of the
mind-body problem through embodied intentionality.

Husserl's concept of intentionality---that consciousness is always
\emph{consciousness of} something---prefigures the Active Inference
insight that inference is always inference \emph{about} hidden states.
The noematic content (what is intended) depends on the noetic act (how
it is intended), just as the posterior depends on how priors and
likelihoods are weighted. Blake's ``As a man is, so he sees'' expresses
this dependency of object on mode of perception.

\subsection{Extended Mind and 4E
Cognition}\label{extended-mind-and-4e-cognition}

Clark and Chalmers' ``Extended Mind'' thesis \citep{clark1998extended}
argues that cognitive processes extend beyond the skull into
environmental structures. This resonates with Blake's insistence that
imagination is not a private mental faculty but participates in a cosmic
creativity: ``Man is All Imagination God is Man \& exists in us \& we in
him.'' The recursive embedding---existing in each other---describes
precisely the nested Markov blankets that enable multi-agent
coordination.

The broader 4E cognition movement (embodied, embedded, enacted,
extended) provides contemporary articulation of Blake's critique of
disembodied reason. Varela, Thompson, and Rosch's \emph{The Embodied
Mind} \citep{varela1991embodied} argues for the inseparability of
cognition from sensorimotor engagement---exactly Blake's claim that
``Energy is Eternal Delight'' and that perception requires active
participation, not passive reception.

\subsection{Affect Theory and Precision
Weighting}\label{affect-theory-and-precision-weighting}

Contemporary affect theory illuminates the role of precision in shaping
inference. Damasio's somatic marker hypothesis
\citep{damasio1994descartes} proposes that emotional states guide
decision-making by tagging options with bodily valence---a form of
affective precision weighting. Blake's Luvah (passion) and his claim
that ``thought alone can make monsters, but the affections cannot''
anticipates this: pure reasoning unmoored from bodily affect produces
biologically non-viable conclusions.

Precision weighting in Active Inference formalizes what matters: high
precision signals ``attend to this.'' Affect theory recognizes that
mattering is not cognitive but visceral---we \emph{feel} significance
before we reason about it. Blake's repeated insistence on passion,
energy, and delight as constitutive of vision (not decorative
enhancements) aligns with this affective grounding of inference.

\subsection{Predictive Processing and Aesthetic
Experience}\label{predictive-processing-and-aesthetic-experience}

The application of predictive processing to aesthetics has matured
rapidly. The 2024 \emph{Philosophical Transactions of the Royal Society
B} theme issue on art, aesthetics, and predictive processing
\citep{vandecruys2024order} represents a watershed: for the first time,
a major scientific journal dedicated an entire volume to exploring how
the brain's predictive architecture shapes aesthetic engagement. Van de
Cruys, Bervoets, and Moors argue that aesthetic experience arises from
the interplay of order and change---precisely the dynamic Blake
dramatized as the ``Marriage of Heaven and Hell,'' where reason (order)
and energy (change) are both ``necessary to Human existence.''

Kukkonen's \emph{Probability Designs} \citep{kukkonen2020probability}
extends predictive processing to literary engagement, modeling how
readers generate predictions, encounter surprise, and update their
models during narrative comprehension. This work provides methodological
precedent for our approach: if predictive processing can illuminate how
readers engage with novels, it can equally illuminate how Blake's
prophetic structures engage the perceptual system.

\subsection{Consciousness as Controlled
Hallucination}\label{consciousness-as-controlled-hallucination}

Anil Seth's \emph{Being You} \citep{seth2021being} advances the thesis
that all perception is a form of ``controlled hallucination''---the
brain's best guess about the causes of sensory signals, constrained but
never determined by incoming evidence. This language---perception as
active construction rather than passive reception---resonates strikingly
with Blake's insistence that we see ``through'' the eye, not ``with''
it. Where Seth's framework emphasizes the constructive, model-dependent
nature of all experience, Blake had already proclaimed: ``A fool sees
not the same tree that a wise man sees'' (\emph{Marriage of Heaven and
Hell}, Plate 7). Both thinkers deny the Enlightenment premise that
perception is simply the imprint of an external world on a passive
receiver.

\subsection{Cognitive Romanticism}\label{cognitive-romanticism}

A new field is coalescing at the intersection of Romantic literary
studies and cognitive science. Savarese's \emph{Romanticism's Other
Minds} \citep{savarese2020romanticism} reassesses early relationships
between Romantic poetry and scientific thought, uncovering a
``prehistory of cognitive approaches to literature'' within the Romantic
tradition itself. The Romantic poets---Wordsworth, Coleridge, Shelley,
and Blake---were not merely literary figures but active theorists of
mind, perception, and social cognition. Our paper extends this tradition
by providing what the Romantics lacked: the formal apparatus to make
their deepest intuitions computationally precise.

\subsection{Cultural Affordances and Shared
Models}\label{cultural-affordances-and-shared-models}

Veissière and colleagues \citep{veissiere2020thinking} apply Active
Inference to cultural cognition, arguing that shared generative
models---``thinking through other minds''---constitute the mechanism of
cultural transmission and niche construction. Their framework treats
culture not as a static repository of information but as a living system
of shared priors, jointly updated through epistemic foraging and
cooperative action. This directly informs our reading of Blake's
Jerusalem: the city is not merely a utopian vision but a formally
specifiable shared generative niche, constructed and maintained through
the ``Mental Fight'' of collective inference.

\subsection{Our Contribution}\label{our-contribution}

Prior scholarship has noted resonances between Romantic thought and
cognitive science at the level of general themes (embodiment,
creativity, the limits of mechanism). Our paper is the first to provide
\emph{specific formal mappings} between Blake's prophetic system and the
mathematical apparatus of Active Inference. We move beyond analogy to
structural correspondence: identifying not merely thematic overlap but
shared topology (the Markov blanket as Blake's door), shared dynamics
(free energy minimization as cleansing), and shared architecture
(hierarchical generative models as fourfold vision). This synthesis
arrives at a moment when both fields---predictive processing aesthetics
and cognitive Romanticism---are independently converging on the same
questions. Our contribution is to continue work on that bridge (or at
least point towards the gap to be respected!).

\newpage

\section{Theoretical Foundations}\label{theory}

\emph{The mathematics of self.} This section reviews the formal
apparatus of the Free Energy Principle and Active Inference. We present
the core formalisms---variational free energy, Markov blankets,
hierarchical generative models, precision weighting, prediction error,
expected free energy, and multi-agent extensions---that the subsequent
synthesis will bring into structural alignment with Blake's prophetic
phenomenology.

\subsection{The Free Energy Principle}\label{fep}

Self-organizing systems persist by minimizing surprise (realism), or at
least can be viewed as if they do (instrumentalism). Friston's Free
Energy Principle (FEP) formalizes this imperative
\citep{friston2010free, friston2006free}, now comprehensively
synthesized in Parr, Pezzulo, and Friston's canonical textbook
\citep{parr2022active}.

\textbf{Variational free energy} provides a tractable upper bound on
surprise (negative log model evidence):

\begin{equation}\label{eq:free_energy}
F = \mathbb{E}_q[\ln q(\theta) - \ln p(o, \theta)]
\end{equation}

where \(o\) denotes observations, \(\theta\) denotes hidden states
(causes), \(q(\theta)\) is a variational density encoding the agent's
beliefs, and \(p(o, \theta)\) is the generative model specifying how
hidden states produce observations.

\textbf{Decomposition} reveals the relationship between free energy,
divergence, and surprise:

\begin{equation}\label{eq:fe_decomposition}
F = D_{KL}[q(\theta) \| p(\theta | o)] - \ln p(o)
\end{equation}

Since KL-divergence is non-negative, free energy upper-bounds surprise:

\begin{equation}\label{eq:surprise_bound}
F \geq -\ln p(o)
\end{equation}

This bound is tight when \(q(\theta) = p(\theta | o)\), i.e., when the
agent's beliefs equal the true posterior. Minimizing \(F\) thus serves
two functions simultaneously: it makes beliefs more accurate (reducing
the divergence term) and implicitly minimizes surprise (the model
evidence term).

\subsubsection{Minimization Pathways}\label{minimization-pathways}

Two complementary pathways reduce free energy (Equation
\ref{eq:free_energy}):

\begin{enumerate}
\def\labelenumi{\arabic{enumi}.}
\tightlist
\item
  \textbf{Perceptual inference} --- Update beliefs \(q(\theta)\) toward
  the true posterior \(p(\theta | o)\). This is changing mind to fit
  world.
\item
  \textbf{Active inference} --- Select actions \(a\) that sample
  observations \(o\) consistent with predictions. This is changing world
  to fit mind.
\end{enumerate}

Both pathways reduce the same objective. The agent that updates its
beliefs \emph{and} acts on the world is performing complete free energy
minimization.

\subsubsection{Expected Free Energy and Policy
Selection}\label{expected-free-energy-and-policy-selection}

Agents must also select among possible courses of action (policies
\(\pi)). The \textbf{expected free energy} \(G(\pi)\) evaluates
policies by their anticipated consequences:

\begin{equation}\label{eq:expected_free_energy}
G(\pi) = -\mathbb{E}_{\tilde{q}}[\ln p(o_\tau | C)] + \mathbb{E}_{\tilde{q}}[D_{KL}[q(\theta_\tau | o_\tau, \pi) \| q(\theta_\tau | \pi)]]
\end{equation}

where \(C\) encodes preferred observations (prior preferences), and
\(\tilde{q}\) denotes the predictive density under the policy. The first
term drives the agent toward outcomes it prefers; the second drives it
to resolve uncertainty about hidden states. Optimal policies minimize
\(G(\pi)\), balancing exploitation (pragmatic value) against exploration
(epistemic value) \citep{dacosta2020active, parr2022active}.

This decomposition is central to the synthesis that follows: it formally
separates the \emph{habitual} from the \emph{curious}, the routine from
the exploratory---categories that recur throughout the humanistic
tradition under different names.

\subsection{The Markov Blanket}\label{blanket}

The Markov blanket defines the statistical boundary of any autonomous
system, partitioning states into internal, external, and blanket
(interface) components \citep{friston2019markov, kirchhoff2018markov}.

\textbf{Conditional independence:}

\begin{equation}\label{eq:conditional_independence}
p(\mu | \eta, B) = p(\mu | B)
\end{equation}

Internal states \(\mu\) are conditionally independent of external states
\(\eta) given blanket states \(B\). The blanket comprises two
complementary channels:

{\def\LTcaptype{none} % do not increment counter
\begin{longtable}[]{@{}
  >{\raggedright\arraybackslash}p{(\linewidth - 6\tabcolsep) * \real{0.2683}}
  >{\raggedright\arraybackslash}p{(\linewidth - 6\tabcolsep) * \real{0.1951}}
  >{\raggedright\arraybackslash}p{(\linewidth - 6\tabcolsep) * \real{0.3902}}
  >{\raggedright\arraybackslash}p{(\linewidth - 6\tabcolsep) * \real{0.1463}}@{}}
\toprule\noalign{}
\begin{minipage}[b]{\linewidth}\raggedright
Component
\end{minipage} & \begin{minipage}[b]{\linewidth}\raggedright
Symbol
\end{minipage} & \begin{minipage}[b]{\linewidth}\raggedright
Flow Direction
\end{minipage} & \begin{minipage}[b]{\linewidth}\raggedright
Role
\end{minipage} \\
\midrule\noalign{}
\endhead
\bottomrule\noalign{}
\endlastfoot
Sensory states & \(s\) & World \(\to\) Self & Carry observations \\
Active states & \(a\) & Self \(\to\) World & Carry interventions \\
\textbf{Blanket} & \(B = \{s, a\}\) & Bidirectional & The statistical
interface \\
\end{longtable}
}

Every self-organizing system---from cell to organism to social
group---possesses a Markov blanket. The blanket is constitutive: without
it, there is no distinction between system and environment, hence no
inference. The topology of this partition---what is inside, what is
outside, what mediates---determines the scope and character of an
agent's engagement with its world.

\subsubsection{Nested Blankets and Multi-Scale
Organization}\label{nested-blankets-and-multi-scale-organization}

Markov blankets nest recursively: cells within organs, organs within
organisms, organisms within social groups. Each scale defines its own
internal/external partition and performs its own inference
\citep{kirchhoff2018markov, ramstead2018answering}. This nesting is not
merely a descriptive convenience but a formal property of hierarchical
self-organization.

\subsection{Hierarchical Generative Models}\label{hierarchy}

Generative models are typically layered, with each level predicting the
activity of the level below
\citep{clark2016surfing, hohwy2013predictive}.

\textbf{Hierarchical factorization:}

\begin{equation}\label{eq:hierarchical_model}
p(o, \theta) = p(o | \theta_1) \prod_{i=1}^{n-1} p(\theta_i | \theta_{i+1}) \cdot p(\theta_n)
\end{equation}

At the lowest level, \(\theta_1\) generates observations through the
likelihood \(p(o | \theta_1)\). Each higher level \(\theta_{i+1}\)
provides the prior context for the level below. The deepest level
\(\theta_n\) encodes the most abstract, slowly varying regularities of
the environment.

This architecture has several key properties:

\begin{itemize}
\tightlist
\item
  \textbf{Abstraction increases with depth.} Low levels encode fast
  sensory features; high levels encode slow contextual structure.
\item
  \textbf{Temporal scale separation.} Higher levels change more slowly,
  providing a stable context for faster dynamics below
  \citep{kiebel2008hierarchy, friston2017deep}.
\item
  \textbf{Bidirectional message passing.} Top-down predictions and
  bottom-up prediction errors flow through the hierarchy, settling
  jointly to minimize free energy.
\end{itemize}

The depth of the hierarchy determines the scope of patterns the model
can represent---from local texture to global meaning.

\subsubsection{Model Evidence and
Complexity}\label{model-evidence-and-complexity}

The marginal likelihood (model evidence) quantifies how well a
generative model accounts for observations:

\begin{equation}\label{eq:model_evidence}
\ln p(o) = \mathbb{E}_{q}[\ln p(o | \theta)] - D_{KL}[q(\theta) \| p(\theta)]
\end{equation}

Good models maximize accuracy while minimizing complexity---a formal
instantiation of Occam's razor. Overly simple models are inaccurate;
overly complex models overfit. The free energy bound (Equation
\ref{eq:surprise_bound}) ensures that minimizing \(F\) implicitly
maximizes model evidence, favoring parsimonious yet accurate
explanations.

\textbf{Model comparison:}

\begin{equation}\label{eq:model_complexity}
F_{\text{simple}} \gg F_{\text{rich}}
\end{equation}

A model of insufficient depth incurs high free energy because it cannot
account for the hierarchical structure of observations. A richer model,
one with appropriate depth and structure, achieves lower free energy by
capturing regularities that the shallow model misses (though a larger
model may have other tradeoffs or penalization terms applied, balancing
the tendency to inflate the number of parameters).

\subsection{Precision}\label{precision}

Precision is the inverse variance of a probability distribution---a
measure of confidence or reliability:

\begin{equation}\label{eq:precision}
\pi = \sigma^{-1}
\end{equation}

In hierarchical inference, precision weights determine how strongly each
level of the hierarchy influences the overall posterior. Two sources of
precision compete at every level:

\begin{itemize}
\tightlist
\item
  \textbf{Prior precision} (\(\pi_{\text{prior}}\)): confidence in
  top-down predictions
\item
  \textbf{Sensory precision} (\(\pi_{\text{sensory}}\)): confidence in
  bottom-up evidence
\end{itemize}

Their balance determines the character of inference:

{\def\LTcaptype{none} % do not increment counter
\begin{longtable}[]{@{}
  >{\raggedright\arraybackslash}p{(\linewidth - 4\tabcolsep) * \real{0.2105}}
  >{\raggedright\arraybackslash}p{(\linewidth - 4\tabcolsep) * \real{0.2895}}
  >{\raggedright\arraybackslash}p{(\linewidth - 4\tabcolsep) * \real{0.5000}}@{}}
\toprule\noalign{}
\begin{minipage}[b]{\linewidth}\raggedright
Regime
\end{minipage} & \begin{minipage}[b]{\linewidth}\raggedright
Condition
\end{minipage} & \begin{minipage}[b]{\linewidth}\raggedright
Perceptual Effect
\end{minipage} \\
\midrule\noalign{}
\endhead
\bottomrule\noalign{}
\endlastfoot
Prior-dominated & \(\pi_{\text{prior}} \gg \pi_{\text{sensory}}\) &
Expectations override evidence; hallucination-like states \\
Sensory-dominated & \(\pi_{\text{sensory}} \gg \pi_{\text{prior}}\) &
Sensory flooding; loss of contextual interpretation \\
Balanced & \(\pi_{\text{prior}} \approx \pi_{\text{sensory}}\) & Optimal
inference; accurate and contextually rich perception \\
\end{longtable}
}

Attention, in this framework, is the optimization of precision---the
process by which the brain infers the reliability of its own prediction
errors and weights them accordingly
\citep{feldman2010attention, parr2019attention}.

\subsubsection{Precision Dynamics and
Pathology}\label{precision-dynamics-and-pathology}

When prior precision becomes extreme:

\textbf{Prior dominance:}

\begin{equation}\label{eq:prior_dominance}
\pi_{\text{prior}} \gg \pi_{\text{sensory}}
\end{equation}

the agent's beliefs become insensitive to new evidence. The generative
model ceases to update, and perception rigidifies. Conversely, when
sensory precision vastly exceeds prior precision, the agent is
overwhelmed by unstructured input, unable to extract meaning.
Pathological states---from delusions to anxiety disorders---can be
understood as failures of precision optimization
\citep{adams2013computational}.

\subsection{Prediction Error and Message Passing}\label{error}

At each level of the hierarchy, the brain computes prediction
error---the discrepancy between what was expected and what was observed:

\begin{equation}\label{eq:prediction_error}
\varepsilon_i = o_i - g_i(\theta_{i+1})
\end{equation}

where \(g_i(\cdot)\) is the generative function mapping higher-level
states to predicted observations at level \(i\). Errors ascend the
hierarchy; predictions descend. The system settles when
\(\varepsilon \rightarrow 0\) across all levels---when predictions match
observations at every scale.

Each error signal (Equation \ref{eq:prediction_error}) propagates
through the hierarchy defined in Equation \ref{eq:hierarchical_model},
weighted by the precision (Equation \ref{eq:precision}) assigned to that
level. High-precision errors demand model revision; low-precision errors
are discounted. This \textbf{precision-weighted prediction error} is the
fundamental currency of hierarchical inference.

The bidirectional cascade of predictions and errors constitutes
perception itself: a continuous, iterative process of generating
hypotheses, testing them against evidence, and revising. Action enters
when the system changes the world to reduce prediction error rather than
changing beliefs.

\subsection{Temporal Depth}\label{temporal-depth}

Generative models can extend across time, encoding dependencies between
successive observations:

\textbf{Temporal hierarchy:}

\begin{equation}\label{eq:temporal_hierarchy}
p(o_{1:T}, \theta) = \prod_{t=1}^{T} p(o_t | \theta_t) \cdot p(\theta_t | \theta_{t-1})
\end{equation}

Higher levels of the hierarchy encode slower dynamics, providing a
context for the faster fluctuations below. The lowest levels track
moment-to-moment sensory input; intermediate levels integrate over
seconds to minutes; the deepest levels encode regularities persisting
across hours, years, or longer
\citep{kiebel2008hierarchy, friston2017deep}.

The \textbf{temporal depth} of a model determines how far into the past
and future its predictions extend. A shallow model is reactive, bound to
immediate stimulus; a deep model integrates broad temporal context into
present inference. Extending temporal depth imposes computational cost
but enables the agent to detect and exploit regularities that span long
durations.

\subsection{Multi-Agent Inference}\label{multi-agent}

Active Inference extends naturally to systems of coupled agents, each
bounded by its own Markov blanket but sharing statistical structure:

\textbf{Multi-agent coordination:}

\begin{equation}\label{eq:multi_agent}
p(o, \theta) = \prod_{i=1}^{N} p(o_i | \theta_i) \cdot p(\theta_i | \theta_{\text{shared}}) \cdot p(\theta_{\text{shared}})
\end{equation}

Multiple agents share a common prior \(\theta_{\text{shared}}\)---the
cultural, institutional, or ecological generative model that aligns
their individual inferences. Communication between agents can be
formalized as generalized synchronization, where coupled systems entrain
their internal dynamics to infer each other's hidden states
\citep{friston2015duet, veissiere2020thinking}.

\subsubsection{Mean-Field Factorization}\label{mean-field-factorization}

When the joint posterior over all hidden states is intractable,
variational inference approximates it by assuming independence between
factors:

\textbf{Mean-field approximation:}

\begin{equation}\label{eq:mean_field}
q(\theta) \approx \prod_{k=1}^{K} q(\theta_k)
\end{equation}

This factorization makes computation tractable but introduces
coordination costs: correlations between components are lost. The
quality of inference depends on how well the factorization structure
matches the true dependencies in the generative model. Structured
variational families that preserve key correlations improve upon the
fully factorized approximation.

This formalized understanding of collective intelligence provides the
necessary bridge to the aesthetic domain. If culture is a shared
generative model, then art is the engineering of that model---a
``cognitive'' intervention that reshapes the priors of the collective.

\subsection{Cognitive Art and the
Fourfold}\label{cognitive-art-and-the-fourfold}

The integration of Active Inference with broad-scale historical and
aesthetic systems suggests a ``cognitive art''---a practice of mind that
is both rigorous and generative. Friedman's recent work on ``Cognitive
Art \& Science'' \citep{friedman2025cognitive} proposes a fourfold
schema for intelligence that maps directly onto the Blakean/Fristonian
synthesis. This framework distinguishes between the ``Low Road''
(\(2 \rightarrow 3\)) of explanatory modeling---fitting data to
priors---and the ``High Road'' (\(4 \rightarrow 3\)) of anticipatory
wisdom---shaping the niche to afford new forms of life. Blake's
rejection of ``Single Vision'' (pure 2nd-ness) in favor of ``Fourfold
Vision'' (integrated 1st, 2nd, 3rd, and 4th-ness) prefigures the move
from mere error minimization to the active construction of a ``wise''
sensorimotor niche.

\subsection{Summary of Formal
Apparatus}\label{summary-of-formal-apparatus}

The following table collects the core equations and their roles in the
synthesis that follows:

{\def\LTcaptype{none} % do not increment counter
\begin{longtable}[]{@{}
  >{\raggedright\arraybackslash}p{(\linewidth - 4\tabcolsep) * \real{0.1795}}
  >{\raggedright\arraybackslash}p{(\linewidth - 4\tabcolsep) * \real{0.2051}}
  >{\raggedright\arraybackslash}p{(\linewidth - 4\tabcolsep) * \real{0.6154}}@{}}
\toprule\noalign{}
\begin{minipage}[b]{\linewidth}\raggedright
Equation
\end{minipage} & \begin{minipage}[b]{\linewidth}\raggedright
Name
\end{minipage} & \begin{minipage}[b]{\linewidth}\raggedright
Role in Synthesis (§4)
\end{minipage} \\
\midrule\noalign{}
\endhead
\bottomrule\noalign{}
\endlastfoot
\ref{eq:free_energy} & Variational Free Energy & Objective function for
perception and action \\
\ref{eq:fe_decomposition} & FEP Decomposition & Relation of divergence
and surprise \\
\ref{eq:surprise_bound} & Surprise Bound & Evidence lower bound (ELBO)
logic \\
\ref{eq:expected_free_energy} & Expected Free Energy & Policy selection
(exploration/exploitation) \\
\ref{eq:conditional_independence} & Conditional Independence & Markov
blanket as statistical boundary \\
\ref{eq:hierarchical_model} & Hierarchical Factorization & Depth of
generative model \\
\ref{eq:model_evidence} & Model Evidence & Accuracy--Complexity
trade-off \\
\ref{eq:model_complexity} & Model Comparison & Necessity of hierarchical
depth \\
\ref{eq:precision} & Precision & Confidence weighting \\
\ref{eq:prior_dominance} & Prior Dominance & Pathological rigidity \\
\ref{eq:prediction_error} & Prediction Error & Bidirectional message
passing \\
\ref{eq:temporal_hierarchy} & Temporal Hierarchy & Depth of temporal
prediction \\
\ref{eq:multi_agent} & Multi-Agent Coordination & Shared priors and
collective inference \\
\ref{eq:mean_field} & Mean-Field Approximation & Factorized variational
inference \\
\end{longtable}
}

Each of these formalisms will be brought into structural alignment with
a specific aspect of Blake's prophetic phenomenology in the sections
that follow.

\newpage

\section{Synthesis: Eight Themes of Vision}\label{synthesis}

Hold your hand in front of your face. The boundary between your skin and
the surrounding air is your Markov blanket---the statistical interface
through which all inference flows. Blake called this the ``door of
perception.'' Two names for the same structure, separated by two
centuries of intellectual history.

In what follows, we trace eight such structural identities between
Blake's prophetic vision and the Active Inference framework, each
demonstrated through specific perceptual scenarios that ground abstract
formalism in lived experience. Each theme begins with Blake's fire---his
phenomenological observation, expressed in the language of prophecy and
illuminated printing---and then traces its mathematical shadow in the
equations, architectures, and dynamics of computational neuroscience.
The correspondences are not approximate analogies but precise structural
mappings: shared topology (the Markov blanket as Blake's door), shared
dynamics (free energy minimization as the cleansing of perception),
shared architecture (hierarchical generative models as fourfold vision),
and shared pathology (prior dominance as Newton's sleep). Figure
\ref{fig:atlas} provides an overview of these eight thematic
correspondences arranged as a visual atlas.

\begin{figure}
\centering
\pandocbounded{\includegraphics[keepaspectratio,alt={Thematic Atlas. Eight structural correspondences between Blake's prophetic vision (left column) and Active Inference formalism (right column), connected by bidirectional arcs indicating the nature of each mapping. Themes span boundary topology (Doors/Markov Blanket), processing hierarchy (Fourfold Vision/Hierarchical Models), cognitive rigidity (Newton's Sleep/Prior Dominance), agent identity (Imagination/Generative Model), temporal depth (Eternity in Hour/Temporal Horizons), spatial inference (World in Grain/Scale Invariance), optimization (Cleansing/Free Energy Minimization), and collective coordination with modular cognition (Jerusalem \& Zoas/Shared \& Factorized Models). Color-coding groups related themes; each correspondence is developed in a dedicated subsection below.}]{../figures/fig0_thematic_atlas.png}}
\caption{\textbf{Thematic Atlas.} Eight structural correspondences
between Blake's prophetic vision (left column) and Active Inference
formalism (right column), connected by bidirectional arcs indicating the
nature of each mapping. Themes span boundary topology (Doors/Markov
Blanket), processing hierarchy (Fourfold Vision/Hierarchical Models),
cognitive rigidity (Newton's Sleep/Prior Dominance), agent identity
(Imagination/Generative Model), temporal depth (Eternity in
Hour/Temporal Horizons), spatial inference (World in Grain/Scale
Invariance), optimization (Cleansing/Free Energy Minimization), and
collective coordination with modular cognition (Jerusalem \& Zoas/Shared
\& Factorized Models). Color-coding groups related themes; each
correspondence is developed in a dedicated subsection
below.}\label{fig:atlas}
\end{figure}

{\def\LTcaptype{none} % do not increment counter
\begin{longtable}[]{@{}
  >{\raggedright\arraybackslash}p{(\linewidth - 6\tabcolsep) * \real{0.2121}}
  >{\raggedright\arraybackslash}p{(\linewidth - 6\tabcolsep) * \real{0.2121}}
  >{\raggedright\arraybackslash}p{(\linewidth - 6\tabcolsep) * \real{0.2727}}
  >{\raggedright\arraybackslash}p{(\linewidth - 6\tabcolsep) * \real{0.3030}}@{}}
\toprule\noalign{}
\begin{minipage}[b]{\linewidth}\raggedright
Theme
\end{minipage} & \begin{minipage}[b]{\linewidth}\raggedright
Blake
\end{minipage} & \begin{minipage}[b]{\linewidth}\raggedright
Friston
\end{minipage} & \begin{minipage}[b]{\linewidth}\raggedright
Identity
\end{minipage} \\
\midrule\noalign{}
\endhead
\bottomrule\noalign{}
\endlastfoot
\href{04a_boundary.md}{Boundary} & Doors of Perception & Markov Blanket
& Interface topology \\
\href{04b_vision.md}{Vision} & Fourfold Vision & Hierarchical Model &
Processing depth \\
\href{04c_states.md}{States} & Newton's Sleep & Prior Dominance &
Cognitive rigidity \\
\href{04d_imagination.md}{Imagination} & Human Existence & Generative
Model & Agent identity \\
\href{04e_time.md}{Time} & Eternity in Hour & Temporal Horizons &
Prediction depth \\
\href{04f_space.md}{Space} & World in Grain & Spatial Hierarchy &
Evidence integration \\
\href{04g_action.md}{Action} & Cleansing & Free Energy Minimization &
Optimization \\
\href{04h_collectives.md}{Collectives} & Building Jerusalem / Four
Mighty Ones & Shared \& Factorized Models & Multi-agent coordination \\
\end{longtable}
}

\newpage

\subsection{Boundary: The Doors of Perception}\label{boundary}

\begin{quote}
\emph{``The cherub with his flaming sword is hereby commanded to leave
his guard at the tree of life, and when he does, the whole creation will
be consumed and appear infinite and holy, whereas it now appears finite
and corrupt.''}

--- \emph{Marriage of Heaven and Hell}, Plate 14
\citep{blake1790marriage}
\end{quote}

\subsubsection{The Markov Blanket}\label{the-markov-blanket}

The cherub's flaming sword is the guardian of a boundary---the partition
between accessible and inaccessible states. Blake's ``door'' is
Friston's \emph{boundary}---the statistical partition separating
internal from external states:

\begin{quote}
``The Markov blanket defines what is inside vs.~outside any autonomous
system---the statistical partition separating internal from external
states.''

--- Kirchhoff et al.~(2018) \citep{kirchhoff2018markov}
\end{quote}

This IS Blake's ``door''---the boundary that mediates all contact
between self and world. The blanket is constitutive, not optional.

{\def\LTcaptype{none} % do not increment counter
\begin{longtable}[]{@{}lll@{}}
\toprule\noalign{}
Blanket Component & Symbol & Blake's Image \\
\midrule\noalign{}
\endhead
\bottomrule\noalign{}
\endlastfoot
External states & \(\eta) & ``the Infinite'' beyond \\
Sensory states & \(s\) & Inflow through doors \\
Active states & \(a\) & Outflow through doors \\
Internal states & \(\mu\) & The perceiver in the ``cavern'' \\
Blanket & \(B\) & ``The doors of perception'' \\
\end{longtable}
}

This conditional independence structure (Equation
\ref{eq:conditional_independence}) means that the blanket mediates all
contact. Internal states access external states \emph{only} through the
interface. The doors are not optional---they are constitutive.

Blake's ``cavern'' is not metaphor but \emph{phenomenology}: the
subjective space of one whose doors are narrowed. The ``chinks'' are the
impoverished sensory channels of a rigid generative model.

Blake articulated this boundary condition repeatedly. The bounded itself
produces suffering:

\begin{quote}
\emph{``The Bounded is loathed by its possessor. The same dull round,
even of a Universe, would soon become a Mill with complicated wheels.''}

--- \emph{There is No Natural Religion}, Series B
\citep{blake1788natural}
\end{quote}

Yet energy is the fundamental currency crossing the boundary:

\begin{quote}
\emph{``Energy is the only life and is from the Body and Reason is the
bound or outward circumference of Energy. Energy is Eternal Delight.''}

--- \emph{Marriage of Heaven and Hell}, Plate 4
\citep{blake1790marriage}
\end{quote}

Blake identifies Reason as the ``bound''---the circumference or blanket
edge that delimits the system. Energy, by contrast, is the vital flow
that crosses this boundary. From this distinction follows his
foundational claim about the nature of perception:

\begin{quote}
\emph{``Man's Perceptions are not bounded by Organs of Perception; he
perceives more than Sense (tho' ever so acute) can discover.''}

--- \emph{There is No Natural Religion}, Series B
\citep{blake1788natural}
\end{quote}

The Markov blanket is necessary but not sufficient. The door mediates;
cleansing transforms how it mediates. Figure \ref{fig:doors} illustrates
this statistical boundary and its Blakean phenomenology.

\begin{quote}
\textbf{Demonstration: The Sunrise}

Blake famously contrasted two ways of seeing the sun:

\emph{``When the sun rises, do you not see a round disk of fire somewhat
like a guinea? O no, no, I see an Innumerable company of the Heavenly
Host crying, `Holy, Holy, Holy.'\,''}

Both observers share the same Markov blanket---the same sensory
channels, the same visual cortex. What differs is how the door mediates:

{\def\LTcaptype{none} % do not increment counter
\begin{longtable}[]{@{}
  >{\raggedright\arraybackslash}p{(\linewidth - 4\tabcolsep) * \real{0.2115}}
  >{\raggedright\arraybackslash}p{(\linewidth - 4\tabcolsep) * \real{0.4038}}
  >{\raggedright\arraybackslash}p{(\linewidth - 4\tabcolsep) * \real{0.3846}}@{}}
\toprule\noalign{}
\begin{minipage}[b]{\linewidth}\raggedright
Component
\end{minipage} & \begin{minipage}[b]{\linewidth}\raggedright
Friend's Experience
\end{minipage} & \begin{minipage}[b]{\linewidth}\raggedright
Blake's Experience
\end{minipage} \\
\midrule\noalign{}
\endhead
\bottomrule\noalign{}
\endlastfoot
\textbf{Sensory states} (\(s\)) & Photons → ``round, golden,
\textasciitilde30° altitude'' & Same photons → rich associative
cascade \\
\textbf{Internal states} (\(\mu\)) & Minimal categorical prediction:
``sun'' & Full generative model: cosmic meaning \\
\textbf{Active states} (\(a\)) & Glance, categorize, move on & Sustained
attention, devotional engagement \\
\end{longtable}
}

\textbf{Same door. Different cleansing.} The friend's perception is not
wrong---but it is shallow. Blake's perception engages deeper layers of
the generative model. Both are valid inferences; one draws on vastly
more model depth.
\end{quote}

\subsubsection{Boundary Constitution: Naming Creates
Separation}\label{boundary-constitution-naming-creates-separation}

Blake understood that boundaries are \emph{constituted}, not given. The
act of naming creates the inside/outside distinction:

\begin{quote}
\emph{``they gave to it a Space \& namd the Space Ulro''}

--- \emph{Vala, or The Four Zoas}, Night the First
\citep[E303]{blake1797fourzoas}
\end{quote}

The Markov blanket is not discovered but \emph{constituted}. ``Naming
the Space'' IS boundary formation. Ulro---Blake's realm of materialist
limitation---comes into being through the act of partition.

\subsubsection{The Abyss as
KL-Divergence}\label{the-abyss-as-kl-divergence}

When Urizen separates from Ahania (his emanation, his feminine
counterpart), Blake describes the resulting gap in terms that directly
anticipate information-theoretic distance:

\begin{quote}
\emph{``Ahania (so name his parted soul)\ldots{} how wide the Abyss
Between Ahania and thee!''}

--- \emph{The Book of Ahania}, Chapter III \citep{blake1795ahania}
\end{quote}

This ``Abyss'' evokes information-theoretic distance---the separation
between states that should be unified. Blake's mythic language of
partition anticipates what Active Inference formalizes as divergence
between distributions. The ``parted soul'' represents a split generative
model; the wider the Abyss, the greater the separation.

\subsubsection{Forged Boundaries}\label{forged-boundaries}

The blanket is not simply given and absolute, but \emph{constructed}:

\begin{quote}
\emph{``He forg'd nets of iron around''}

--- \emph{The Book of Ahania} \citep{blake1795ahania}
\end{quote}

``Forging'' emphasizes the active construction and identification of
boundary conditions. The Markov blanket is manufactured materially and
mentally. This has profound implications: what is forged can be
reforged, and what is identified can be mis- and re-identified.

\subsubsection{Body as Blanket
Interface}\label{body-as-blanket-interface}

Blake's most direct statement of embodied cognition anticipates the
blanket formalism:

\begin{quote}
\emph{``Man has no Body distinct from his Soul for that calld Body is a
portion of Soul discernd by the five Senses, the chief inlets of Soul in
this age''}

--- \emph{Marriage of Heaven and Hell}, Plate 4
\citep{blake1790marriage}
\end{quote}

``Inlets'' = sensory states \(s\). The body IS how mind interfaces with
world---not a separate substance but the blanket itself. No ontological
separation exists; only the functional partition of the blanket.

\subsubsection{Imagination's Outline vs.~Nature's
Dissolution}\label{imaginations-outline-vs.-natures-dissolution}

Blake's late work makes explicit the distinction between raw data and
model structure:

\begin{quote}
\emph{``Nature has no Outline: but Imagination has. Nature has no Tune:
but Imagination has. Nature has no Supernatural \& dissolves:
Imagination is Eternity''}

--- \emph{The Ghost of Abel}, Plate 1 \citep{blake1822abel}
\end{quote}

Nature (observations \(o\)) is unstructured flow. Imagination (the
generative model \(p(o, \theta)\)) provides boundaries, form, temporal
structure (``Tune''). The model is more real than the data because it is
what makes data \emph{intelligible}. Without the model's outline,
perception dissolves into chaos.

\subsubsection{Contemporary Resonance: From Blake Through Huxley to
ALBUS}\label{contemporary-resonance-from-blake-through-huxley-to-albus}

The lineage from Blake's ``Doors'' to contemporary neuroscience runs
through a single remarkable chain. Aldous Huxley borrowed Blake's phrase
for \emph{The Doors of Perception} \citep{huxley1954doors}, proposing
that the brain operates as a ``reducing valve'' filtering the totality
of experience into manageable form. Carhart-Harris and Friston's REBUS
model \citep{carthartharris2019rebus} formalized this intuition by
showing that psychedelics relax the precision of high-level priors.
Safron and colleagues' ALBUS framework \citep{safron2025albus} now
extends this account: psychedelics do not merely relax beliefs (REBUS)
but can also strengthen them (SEBUS), producing a richer taxonomy of
altered states---from the dissolution of rigid priors to the
intensification of meaning-making. The result encompasses Blake's
``cleansing''---not the destruction of the boundary but the
recalibration of its precision weighting, allowing prediction error to
propagate more freely up the hierarchy. What Blake described as seeing
``every thing\ldots{} as it is: infinite'' corresponds to a state of
altered prior dynamics where sensory evidence reshapes inference rather
than being suppressed by entrenched expectations. This is not metaphor:
it is the same computational operation described in different
vocabularies across three centuries.

\begin{figure}
\centering
\pandocbounded{\includegraphics[keepaspectratio,alt={The Doors of Perception as Markov Blanket. The statistical boundary (B = \textbackslash\{s, a\textbackslash\}) partitions external states \textbackslash eta (``the Infinite'') from internal states \textbackslash mu (``the Cavern'') through two complementary channels: sensory states s mediating world-to-self flow (observation, perception) and active states a mediating self-to-world flow (action, decision). This implements Blake's phenomenology from The Marriage of Heaven and Hell, Plate 14: ``If the doors of perception were cleansed every thing would appear to man as it is, Infinite. For man has closed himself up, till he sees all things thro' narrow chinks of his cavern'' {[}@blake1790marriage{]}. The blanket is constitutive---not an optional filter but the necessary interface through which all inference occurs. Cleansing the doors corresponds to optimizing the model's precision weighting (Equation ), not to eliminating the boundary itself.}]{../figures/fig1_doors_of_perception.png}}
\caption{\textbf{The Doors of Perception as Markov Blanket.} The
statistical boundary (\(B = \{s, a\}\)) partitions external states
\(\eta) (``the Infinite'') from internal states \(\mu\) (``the
Cavern'') through two complementary channels: sensory states \(s\)
mediating world-to-self flow (observation, perception) and active states
\(a\) mediating self-to-world flow (action, decision). This implements
Blake's phenomenology from \emph{The Marriage of Heaven and Hell}, Plate
14: ``If the doors of perception were cleansed every thing would appear
to man as it is, Infinite. For man has closed himself up, till he sees
all things thro' narrow chinks of his cavern''
\citep{blake1790marriage}. The blanket is constitutive---not an optional
filter but the necessary interface through which all inference occurs.
Cleansing the doors corresponds to optimizing the model's precision
weighting (Equation \ref{eq:conditional_independence}), not to
eliminating the boundary itself.}\label{fig:doors}
\end{figure}

\newpage

\subsection{Vision: The Fourfold Hierarchy}\label{vision}

\begin{quote}
\emph{``The great City of Golgonooza: fourfold toward the north / And
toward the south fourfold, \& fourfold toward the east \& west / Each
within other toward the four points''}

--- \emph{Jerusalem}, Plate 12 \citep{blake1804jerusalem}
\end{quote}

\subsubsection{Hierarchical Generative
Models}\label{hierarchical-generative-models}

Golgonooza---Blake's city of art, built by the imagination---provides
the architectural metaphor for hierarchical inference: fourfold in every
direction, each level nested within the others. The predictive brain
generates perception actively:

\begin{quote}
``The brain is revealed as an active, generative organ: one that
continually predicts its own current sensory states, using those
predictions to explain away the incoming sensory signal.''

--- \citep{clark2013whatever}
\end{quote}

This bidirectional cascade IS perception---errors ascend, predictions
descend:

\begin{quote}
``Feedback connections from a higher- to a lower-order visual cortical
area carry predictions of lower-level neural activities, whereas the
feedforward connections carry the residual errors between the
predictions and the actual lower-level activities.''

--- \citep{rao1999predictive}
\end{quote}

Four levels of perception correspond to four depths of the generative
hierarchy:

{\def\LTcaptype{none} % do not increment counter
\begin{longtable}[]{@{}
  >{\raggedright\arraybackslash}p{(\linewidth - 6\tabcolsep) * \real{0.2653}}
  >{\raggedright\arraybackslash}p{(\linewidth - 6\tabcolsep) * \real{0.1633}}
  >{\raggedright\arraybackslash}p{(\linewidth - 6\tabcolsep) * \real{0.3265}}
  >{\raggedright\arraybackslash}p{(\linewidth - 6\tabcolsep) * \real{0.2449}}@{}}
\toprule\noalign{}
\begin{minipage}[b]{\linewidth}\raggedright
Blake Level
\end{minipage} & \begin{minipage}[b]{\linewidth}\raggedright
Symbol
\end{minipage} & \begin{minipage}[b]{\linewidth}\raggedright
Cognitive Mode
\end{minipage} & \begin{minipage}[b]{\linewidth}\raggedright
Processing
\end{minipage} \\
\midrule\noalign{}
\endhead
\bottomrule\noalign{}
\endlastfoot
\textbf{Single} (Ulro) & \(\theta_1\) & Quantitative & Sensory
features \\
\textbf{Twofold} (Generation) & \(\theta_2\) & Emotional & Affective
encoding \\
\textbf{Threefold} (Beulah) & \(\theta_3\) & Imaginative & Symbolic
integration \\
\textbf{Fourfold} (Jerusalem) & \(\theta_4\) & Unified & Complete model
engagement \\
\end{longtable}
}

\textbf{Fourfold hierarchical factorization:}

\begin{equation}\label{eq:fourfold_hierarchy}
p(o, \theta_{1:4}) = p(o | \theta_1) \prod_{i=1}^{3} p(\theta_i | \theta_{i+1}) \cdot p(\theta_4)
\end{equation}

Fourfold vision engages all levels of the hierarchy (Equation
\ref{eq:fourfold_hierarchy}; see Figure \ref{fig:fourfold}). Single
vision collapses to \(\theta_1\) alone, reducing the general
hierarchical model (Equation \ref{eq:hierarchical_model}) to a single
layer. The hierarchy is not ornament---it is the architecture of
meaning.

Blake grasped this hierarchical principle:

\begin{quote}
\emph{``The Eye sees more than the Heart knows.''}

--- \emph{Visions of the Daughters of Albion}, title page
\citep{blake1793visions}
\end{quote}

Even the lower level (eye/sensation) accesses more than higher cognition
(heart/understanding) can process. The crooked roads of genius
circumvent linear reasoning:

\begin{quote}
\emph{``Improvement makes strait roads; but the crooked roads without
Improvement are roads of Genius.''}

--- \emph{Marriage of Heaven and Hell}, Proverbs of Hell
\citep{blake1790marriage}
\end{quote}

Hierarchy need not mean rigid order---the genius finds shortcuts through
visionary compression. Worton's analysis of Blake's intertextuality
reveals that these ``crooked roads'' function as radical
reconfigurations of existing models, not mere deviations from linearity
\citep{worton1982blake}.

\subsubsection{Golgonooza: The Architecture of the Generative
Model}\label{golgonooza-the-architecture-of-the-generative-model}

Blake's mythic city Golgonooza---the city of art, built by Los the
imagination---provides a structural diagram of hierarchical inference.
Recall the passage quoted at the opening of this section: ``fourfold
toward the north / And toward the south fourfold, \& fourfold toward the
east \& west / Each within other toward the four points.''

Four directions = four hierarchical levels. ``Each within other'' =
nested structure. Golgonooza IS the generative model's architecture---a
city that is simultaneously spatial and cognitive, built from the
material of imagination itself.

The fourfold structure extends in all dimensions: north/south/east/west
map to the Four Zoas (Urthona/Urizen/Luvah/Tharmas), each representing a
distinct mode of inference. The city is not static but perpetually under
construction---Los labors at the furnaces, continually rebuilding the
model.

\subsubsection{Organs of Perception as
Model-Dependent}\label{organs-of-perception-as-model-dependent}

Blake makes explicit that perception is not passive reception but active
model-dependent construction:

\begin{quote}
\emph{``Creating Space, Creating Time\ldots{} such was the variation of
Time \& Space, which vary according as the Organs of Perception vary''}

--- \emph{Jerusalem}, Plate 98 \citep{blake1804jerusalem}
\end{quote}

Space and time are not objective containers but generative model
outputs. Different models produce different space-times. The ``Organs of
Perception'' are not fixed biological apparatus but the structure of
inference itself---and this structure can vary.

This anticipates the Active Inference insight that even basic phenomenal
properties like spatial extent and temporal duration are inferred, not
given. The model creates the coordinate system within which observations
are interpreted.

Anil Seth's contemporary formulation crystallizes this point: all
perception is a ``controlled hallucination''---the brain's best guess
about the causes of sensory signals, constrained but not determined by
incoming evidence \citep{seth2021being}. Blake's fourfold vision is, in
these terms, a taxonomy of hallucination depths: single vision is a
shallow, rigid hallucination dominated by sensory constraint; fourfold
vision is a deep, flexible hallucination where the generative model's
own creative structure participates fully in what is perceived. The
``fool'' and the ``wise man'' who see different trees are running
different models on the same data---and both perceptions are, in Seth's
precise sense, controlled hallucinations.

Blake was acutely aware that deeper vision is not merely unseen but
actively \emph{pathologized} by the regime of single vision. In
\emph{Milton}, he names this suppression directly:

\begin{quote}
\emph{``Calling the Human Imagination: which is the Divine Vision \&
Fruition\emph{ }In which Man liveth eternally: madness \& blasphemy,
against\emph{ }Its own Qualities, which are Servants of Humanity, not
Gods or Lords.''}

--- \emph{Milton}, Plate 32 \citep{blake1804milton}
\end{quote}

``Madness \& blasphemy'' is the diagnostic frame that prior-dominated
inference applies to perception that exceeds its own model. From within
Newton's Sleep, fourfold vision looks pathological precisely because the
shallow model cannot represent the hierarchical depth that makes it
possible---it can only classify what it cannot compute as error,
delusion, or transgression. Blake's counter-move is to insist that
imagination's ``Qualities'' are ``Servants of Humanity, not Gods or
Lords'': the deeper levels of the generative model serve the agent's
self-evidencing; they are not external authorities but functional
capacities. This anticipates contemporary debates in psychedelic
neuroscience, where expanded perceptual states---once dismissed as mere
hallucination---are increasingly recognized as alternative precision
regimes with their own epistemic validity
\citep{carthartharris2019rebus}.

\begin{figure}
\centering
\pandocbounded{\includegraphics[keepaspectratio,alt={The Fourfold Vision Hierarchy. Blake's four perceptual levels mapped to corresponding depths of the Active Inference hierarchical generative model (Equation ). Single Vision (Ulro, \textbackslash theta\_1, gray): quantitative sensory registration---``Newton's sleep,'' seeing a rose as cells and chemistry. Twofold Vision (Generation, \textbackslash theta\_2, blue): emotional-intellectual engagement---perceiving beauty, desire, and symbolic meaning. Threefold Vision (Beulah, \textbackslash theta\_3, purple): imaginative synthesis---``soft Beulah's night,'' where contraries reconcile in art and myth. Fourfold Vision (Jerusalem, \textbackslash theta\_4, gold): full hierarchical integration---``supreme delight,'' unified engagement of all model depths. Left column: Blake's phenomenological descriptions; right column: Active Inference processing levels. Ascending arrows indicate increasing hierarchical depth and precision integration. Source: Letter to Thomas Butts, 22 November 1802 {[}@blake1802butts{]}.}]{../figures/fig2_fourfold_vision.png}}
\caption{\textbf{The Fourfold Vision Hierarchy.} Blake's four perceptual
levels mapped to corresponding depths of the Active Inference
hierarchical generative model (Equation \ref{eq:fourfold_hierarchy}).
\textbf{Single Vision} (Ulro, \(\theta_1\), gray): quantitative sensory
registration---``Newton's sleep,'' seeing a rose as cells and chemistry.
\textbf{Twofold Vision} (Generation, \(\theta_2\), blue):
emotional-intellectual engagement---perceiving beauty, desire, and
symbolic meaning. \textbf{Threefold Vision} (Beulah, \(\theta_3\),
purple): imaginative synthesis---``soft Beulah's night,'' where
contraries reconcile in art and myth. \textbf{Fourfold Vision}
(Jerusalem, \(\theta_4\), gold): full hierarchical
integration---``supreme delight,'' unified engagement of all model
depths. Left column: Blake's phenomenological descriptions; right
column: Active Inference processing levels. Ascending arrows indicate
increasing hierarchical depth and precision integration. Source: Letter
to Thomas Butts, 22 November 1802
\citep{blake1802butts}.}\label{fig:fourfold}
\end{figure}

\newpage

\subsection{States: Newton's Sleep \& Prior Dominance}\label{states}

\begin{quote}
\emph{``If the Sun \& Moon should doubt / They'd immediately Go out.''}

--- \emph{Auguries of Innocence} \citep{blake1803auguries}
\end{quote}

\begin{quote}
\emph{``The will of the Immortal expanded / Or contracted his
all-flexible senses''}

--- \emph{Book of Urizen}, Plate 3 \citep{blake1794urizen}
\end{quote}

\subsubsection{Prior Dominance}\label{prior-dominance}

These two quotations frame the full spectrum of precision dynamics. The
Sun and Moon that never doubt represent the cosmic necessity of
confident priors---without them, the world ``goes out.'' Yet the
Immortal's ``all-flexible senses'' describe the capacity to modulate
those priors at will. ``Newton's sleep'' is what happens when
flexibility is lost. It is \emph{rigid inference}---the condition where
prior beliefs overwhelm sensory evidence. Active Inference formalizes
this:

\begin{quote}
``Attention can be understood as inferring the level of uncertainty or
precision during hierarchical perception.''

--- Feldman \& Friston (2010) \citep{feldman2010attention}
\end{quote}

Parr and Friston further develop this insight, showing that attention
optimizes the precision of prediction errors at every level of the
cortical hierarchy, selecting which sensory channels carry reliable
information \citep{parr2019attention}.

When prior precision vastly exceeds sensory precision:

\textbf{Prior dominance:}

\begin{equation}\label{eq:prior_dominance}
\pi_{\text{prior}} \gg \pi_{\text{sensory}}
\end{equation}

When prior precision vastly exceeds sensory precision (Equation
\ref{eq:prior_dominance}), observations are discounted. The world
conforms to expectation---free energy (Equation \ref{eq:free_energy})
ceases to drive model revision because the precision weighting (Equation
\ref{eq:precision}) favors priors. This is Blake's
``Urizen''---\emph{your reason} frozen into \emph{horizon} (limit).

{\def\LTcaptype{none} % do not increment counter
\begin{longtable}[]{@{}
  >{\raggedright\arraybackslash}p{(\linewidth - 4\tabcolsep) * \real{0.3333}}
  >{\raggedright\arraybackslash}p{(\linewidth - 4\tabcolsep) * \real{0.2424}}
  >{\raggedright\arraybackslash}p{(\linewidth - 4\tabcolsep) * \real{0.4242}}@{}}
\toprule\noalign{}
\begin{minipage}[b]{\linewidth}\raggedright
Condition
\end{minipage} & \begin{minipage}[b]{\linewidth}\raggedright
Effect
\end{minipage} & \begin{minipage}[b]{\linewidth}\raggedright
Blake's Term
\end{minipage} \\
\midrule\noalign{}
\endhead
\bottomrule\noalign{}
\endlastfoot
\(\pi_{\text{prior}} \gg \pi_{\text{sensory}}\) & Expectations dominate
& ``Newton's sleep'' \\
\(\pi_{\text{sensory}} \gg \pi_{\text{prior}}\) & Sensory overwhelm &
Chaos, dissolution \\
\(\pi_{\text{prior}} \approx \pi_{\text{sensory}}\) & Optimal inference
& ``Cleansed perception'' \\
\end{longtable}
}

The ``guinea sun''---Blake's mockery of seeing the sun as mere golden
disk---is exactly this: prior-locked inference refusing sensory update.
Figure \ref{fig:newtons_sleep} illustrates the contrast between
prior-dominated and balanced inference.

\begin{quote}
\textbf{Demonstration: The Familiar Street}

Walk down a street you travel daily. You may fail to notice a new
shopfront, a repainted door, a changed sign. Your prior model (``this
street looks like X'') overwhelms sensory evidence of change. This is
\(\pi_{\text{prior}} \gg \pi_{\text{sensory}}\) in action.

Blake's Urizen has ``contracted his all-flexible senses'' into ``little
orbs\ldots{} hiding from the wind.'' The narrowed perception is not
sensory failure but \emph{model rigidity}.

\textbf{Contrast: Tourist Vision}

Now visit a new city. \emph{Everything} glows with detail. Unfamiliar
streets demand notice---each doorway, sign, and face registers
distinctly. This is
\(\pi_{\text{prior}} \approx \pi_{\text{sensory}}\)---balanced precision
where the model cannot coast on expectation.

The tourist sees more not because their eyes are better, but because
their priors are weaker. Their doors are cleansed by unfamiliarity.
\end{quote}

\begin{figure}
\centering
\pandocbounded{\includegraphics[keepaspectratio,alt={Newton's Sleep vs.~Cleansed Perception. Two contrasting precision regimes illustrated as balance beams. Left panel (``Newton's Sleep''): prior precision \textbackslash pi\_\{\textbackslash text\{prior\}\} vastly exceeds sensory precision \textbackslash pi\_\{\textbackslash text\{sensory\}\} (large red weight vs.~small teal weight), producing rigid, expectation-dominated inference where ``the will of the Immortal\ldots{} contracted his all-flexible senses'' (Book of Urizen, Plate 3 {[}@blake1794urizen{]}). Observations are discounted; the world conforms to frozen expectation. Right panel (``Cleansed Perception''): balanced precision weighting (\textbackslash pi\_\{\textbackslash text\{prior\}\} \textbackslash approx \textbackslash pi\_\{\textbackslash text\{sensory\}\}, equal green and teal weights) enables optimal inference where ``every thing would appear to man as it is, Infinite'' (Marriage of Heaven and Hell, Plate 14 {[}@blake1790marriage{]}). The balance beam metaphor captures the precision dynamics formalized in Equation . Blake's ``guinea sun'' exemplifies the left state; his ``Innumerable company of the Heavenly Host'' exemplifies the right.}]{../figures/fig4_newtons_sleep.png}}
\caption{\textbf{Newton's Sleep vs.~Cleansed Perception.} Two
contrasting precision regimes illustrated as balance beams. \textbf{Left
panel} (``Newton's Sleep''): prior precision \(\pi_{\text{prior}}\)
vastly exceeds sensory precision \(\pi_{\text{sensory}}\) (large red
weight vs.~small teal weight), producing rigid, expectation-dominated
inference where ``the will of the Immortal\ldots{} contracted his
all-flexible senses'' (\emph{Book of Urizen}, Plate 3
\citep{blake1794urizen}). Observations are discounted; the world
conforms to frozen expectation. \textbf{Right panel} (``Cleansed
Perception''): balanced precision weighting
(\(\pi_{\text{prior}} \approx \pi_{\text{sensory}}\), equal green and
teal weights) enables optimal inference where ``every thing would appear
to man as it is, Infinite'' (\emph{Marriage of Heaven and Hell}, Plate
14 \citep{blake1790marriage}). The balance beam metaphor captures the
precision dynamics formalized in Equation \ref{eq:prior_dominance}.
Blake's ``guinea sun'' exemplifies the left state; his ``Innumerable
company of the Heavenly Host'' exemplifies the
right.}\label{fig:newtons_sleep}
\end{figure}

The question becomes: is Newton's sleep a permanent condition, or a
temporary one? Blake's answer is emphatic---and computationally
significant. In a pivotal passage from \emph{Milton}, he develops the
distinction with systematic precision:

\begin{quote}
\emph{``We are not Individuals but States: Combinations of
Individuals\emph{ }We were Angels of the Divine Presence\ldots.\emph{
}Calling the Human Imagination: which is the Divine Vision \&
Fruition\emph{ }In which Man liveth eternally: madness \& blasphemy,
against\emph{ }Its own Qualities, which are Servants of Humanity, not
Gods or Lords.\emph{ }Distinguish therefore States from Individuals in
those States.\emph{ }States Change: but Individual Identities never
change nor cease:\emph{ }You cannot go to Eternal Death in that which
can never Die.\emph{ }Satan \& Adam are States Created into Twenty-seven
Churches\ldots.\emph{ }States that are not, but ah! Seem to be.''}

--- \emph{Milton}, Plate 32 \citep{blake1804milton}
\end{quote}

This passage is computationally dense. ``Combinations of Individuals''
reframes what a state \emph{is}: not a personal mood but a composite
model configuration---a particular factorization of beliefs, precisions,
and action policies that many agents can share. ``Twenty-seven
Churches'' names a discrete, enumerable state-space: Satan and Adam are
not persons but \emph{attractors} in model-space, recurring
configurations that agents fall into and can emerge from. The crucial
claim---``States that are not, but ah! Seem to be''---is the recognition
that states feel ontologically real from within but are transient
configurations of the generative model, not permanent features of
reality.

Equally striking is Blake's simultaneous attack on reductive spatial
models within the same passage:

\begin{quote}
\emph{``those combind by Satans Tyranny\ldots{} are Shapeless
Rocks\emph{ }Retaining only Satans Mathematic Holiness, Length: Bredth
\& Highth''}

--- \emph{Milton}, Plate 32 \citep{blake1804milton}
\end{quote}

``Mathematic Holiness''---the worship of pure quantitative
extension---is Blake's name for the impoverished generative model that
retains only Euclidean coordinates. When imaginative depth is stripped
away, what remains is geometry without meaning: ``Shapeless Rocks'' that
have collapsed to minimum model complexity. This directly parallels the
Active Inference diagnosis of Newton's Sleep: a state where the model's
rich hierarchical structure has been flattened to shallow, quantitative
prediction.

The sleeping perceiver is in a \emph{state}, not an \emph{identity}.
States change; the door can be cleansed. The miser is Blake's figure for
this pathology: hoarding certainty like gold, refusing sensory update,
clinging to priors that have calcified into identity.

And the transformation:

\begin{quote}
\emph{``If the fool would persist in his folly he would become wise.''}

--- \emph{Marriage of Heaven and Hell}, Proverbs of Hell
\citep{blake1790marriage}
\end{quote}

Persistence breaks through prior rigidity. The fool's persistence is a
form of precision reweighting---eventually, accumulated evidence
overwhelms the prior.

Blake records this exchange in a famous passage (see
\hyperref[boundary]{§4.1 Boundary}): his friend perceives only a golden
disk; Blake perceives the Heavenly Host. The friend's response
represents prior-locked perception; Blake's answer demonstrates cleansed
vision.

\subsubsection{The Spectre's Steel Ratio: Prior Dominance
Formalized}\label{the-spectres-steel-ratio-prior-dominance-formalized}

Blake's most precise formulation of prior-dominated inference appears in
\emph{Jerusalem}:

\begin{quote}
\emph{``The Spectre is the Reasoning Power in Man; \& when separated /
From Imagination, and closing itself as in steel, in a Ratio / Of the
Things of Memory. It thence frames Laws \& Moralities / To destroy
Imagination!''}

--- \emph{Jerusalem}, Plate 74, lines 10-13 \citep{blake1804jerusalem}
\end{quote}

This is the complete phenomenology of rigid inference:

\begin{itemize}
\tightlist
\item
  ``Closing itself as in steel'' = \(\pi_{\text{prior}} \to \infty\)
  (infinite prior precision)
\item
  ``Ratio of Things of Memory'' = frozen historical priors refusing
  sensory update
\item
  ``Laws \& Moralities'' = rigid predictive structures that suppress
  model revision
\item
  ``Destroy Imagination'' = elimination of model flexibility
\end{itemize}

The Spectre is not evil but \emph{separated}---cut off from the
generative model's creative capacity to revise itself. When reasoning
power closes itself off from imagination, it becomes a self-reinforcing
loop of confirmation bias.

\subsubsection{The Philosophy of Five
Senses}\label{the-philosophy-of-five-senses}

Blake traces the intellectual history of perceptual restriction:

\begin{quote}
\emph{``Till a Philosophy of Five Senses was complete / Urizen wept \&
gave it into the hands of Newton \& Locke''}

--- \emph{The Song of Los}, Plate 4 \citep{blake1795songlos}
\end{quote}

The ``Philosophy of Five Senses'' is the sensory bottleneck
doctrine---the restriction of inference to immediate observations
without hierarchical depth. Newton and Locke represent single-level
empiricism: the belief that knowledge comes only from sensory evidence,
without recognizing the generative models that make evidence
intelligible.

Urizen ``weeps'' because even he---the principle of rational
limitation---recognizes this as impoverishment. The complete collapse to
sensory level is Newton's sleep at civilizational scale.

\subsubsection{Descending into Division}\label{descending-into-division}

The Fall is described as a collapse through hierarchical levels:

\begin{quote}
\emph{``I was divided: descending down I sunk along / The goary tide
even to the place of seed \& there / Dividing I was buried''}

--- \emph{Vala, or The Four Zoas}, Night the First
\citep{blake1797fourzoas}
\end{quote}

``Division'' = separation from higher-level update. ``Descending'' =
collapse to lower hierarchical levels. ``Buried'' = inference locked in
prior state, unable to revise.

The ``goary tide'' suggests the violence of this collapse---the tearing
apart of an integrated model into isolated fragments, each trapped in
its own local minimum.

\subsubsection{The Natural Man vs.~Spiritual
Man}\label{the-natural-man-vs.-spiritual-man}

Blake's critique of Wordsworth illuminates the tension between
likelihood and prior:

\begin{quote}
\emph{``I see in Wordsworth the natural man rising up against the
spiritual man continually''}

--- Annotations to Wordsworth's Poems \citep{blake1826wordsworth}
\end{quote}

``Natural man'' = sensory likelihood \(p(o|\theta)\). ``Spiritual man''
= prior beliefs \(p(\theta)\). Wordsworth, in Blake's view, over-weights
sensory precision at the expense of imaginative priors. The ``rising
up'' is the dominance of bottom-up over top-down---the inverse of
Newton's sleep, but equally unbalanced.

\subsubsection{Hemispheric Resonance}\label{hemispheric-resonance}

McGilchrist's hemispheric hypothesis
\citep{mcgilchrist2009master, mcgilchrist2021matter} provides a
neuroanatomical substrate for the pathology Blake diagnoses. The left
hemisphere's mode of attention---narrow, focused, categorical,
already-knowing---maps precisely onto Newton's Sleep: the
prior-dominated regime where familiar categories suppress novel
inference. Urizen \emph{is} the left hemisphere enthroned, the lawgiver
who ``closed the tent of the Universe'' by imposing rigid categorical
boundaries. Conversely, Blake's Los---the creative imagination
perpetually rebuilding the model at the furnaces of
Golgonooza---embodies what McGilchrist identifies as the right
hemisphere's broader, contextual, novelty-seeking attention. The
clinical implication is striking: if Newton's Sleep is a form of
hemispheric imbalance, then ``awakening'' (restoring balanced precision
weighting) may require not more data but a different \emph{mode of
attention}---a shift in which hemisphere leads the inference.

\newpage

\subsection{Imagination: The Generative
Model}\label{imagination-synthesis}

\begin{quote}
\emph{``Man is All Imagination God is Man \& exists in us \& we in
him''}

--- Annotations to Berkeley's \emph{Siris} \citep{blake1820berkeley}
\end{quote}

\subsubsection{The Generative Model as
Self}\label{the-generative-model-as-self}

Blake's claim is radical and recursive: not merely that imagination
constitutes human existence, but that the relationship between agent and
model is one of mutual entailment---``exists in us \& we in him.'' The
generative model does not belong to an agent; the generative model
\emph{is} the agent. Active Inference formalizes this:

\begin{quote}
``Consciousness is nothing more than inference about my future; namely,
the self-evidencing consequences of what I could do.''

--- \citep{friston2018self}
\end{quote}

The self as process, not entity:

\begin{quote}
``The self is the result of an ongoing predictive process within a
generative model that is centered onto the organism.''

--- \citep{limanowski2013minimal}
\end{quote}

And the body as probabilistically ``most likely to be me'':

\begin{quote}
``One's own body is the one which has the highest probability of being
`me' as other objects are probabilistically less likely to evoke the
same sensory inputs.''

--- \citep{apps2014free}
\end{quote}

\textbf{Agent identity:}

\begin{equation}\label{eq:agent_identity}
\text{Self} \equiv p(o, \theta)
\end{equation}

The generative model defines:

\begin{itemize}
\tightlist
\item
  What counts as inside/outside (blanket structure)
\item
  What states are expected (prior beliefs)
\item
  What observations mean (likelihood mapping)
\item
  What actions are available (policy repertoire)
\end{itemize}

Without model, no agent. The self \emph{is} the generative model
(Equation \ref{eq:agent_identity}), bounded by its Markov blanket
(Equation \ref{eq:conditional_independence}). Seth's ``cybernetic
Bayesian brain'' makes this explicit: selfhood arises from the brain's
predictive model of its own body, a ``controlled hallucination''
grounded in interoceptive and proprioceptive inference
\citep{seth2014cybernetic}. Blake saw this two centuries earlier: ``As a
man is, so he sees.'' The internal structure determines external
appearance.

\begin{quote}
\textbf{Demonstration: Who Is Seeing?}

Close your eyes. Imagine a red rose---its color, curve, scent.

Now ask: \emph{who} is imagining? You might answer ``I am.'' But look
closer. The generative model that produces ``rose'' \emph{is} the entity
doing the imagining. There is no homunculus watching the mental screen;
the screen is the seer.

Blake: ``Man is All Imagination God is Man \& exists in us \& we in
him.''

The recursive ``in us / we in him'' captures the autopoietic loop: the
model models itself modeling. No agent exists beneath the model. The
model \emph{is} the agent.
\end{quote}

Blake's embodied vision:

\begin{quote}
\emph{``The Eternal Body of Man is The Imagination, that is, God
himself, The Divine Body.''}

--- Annotation to Laocoon \citep{blake1826laocoon}
\end{quote}

\begin{quote}
\emph{``Mental Things are alone Real; what is call'd Corporeal Nobody
knows of its Dwelling Place; it is in Fallacy \& its Existence an
Imposture.''}

--- \emph{Vision of the Last Judgment} \citep{blake1810judgment}
\end{quote}

And the constitutive claim:

\begin{quote}
\emph{``As a man is, so he sees. As the Eye is formed, such are its
Powers.''}

--- Letter to Dr.~Trusler, 23 August 1799 \citep{blake1799trusler}
\end{quote}

The generative model shapes what can be perceived. The ``formed Eye'' is
the structure of inference itself.

\begin{quote}
\emph{``The world of imagination is the world of eternity.''}

--- \emph{Vision of the Last Judgment} \citep{blake1810judgment}
\end{quote}

\subsubsection{Man Is All Imagination: Autopoiesis and
Self-Modeling}\label{man-is-all-imagination-autopoiesis-and-self-modeling}

The recursive structure of Blake's epigraph above---``exists in us \& we
in him''---is autopoiesis: the agent IS its generative model. ``God in
us'' = our model of world. ``We in him'' = we are part of what the model
represents. The nested embedding describes Markov blankets within Markov
blankets---the recursive self-modeling that constitutes agency.

There is no man ``underneath'' imagination; imagination is what man
\emph{is}. The generative model doesn't belong to an agent---the
generative model IS the agent.

\subsubsection{Mental Things Alone Real}\label{mental-things-alone-real}

Blake extends this to a full phenomenological idealism. His declaration
that ``Mental Things are alone Real'' and that the ``Corporeal'' is ``in
Fallacy \& its Existence an Imposture'' (\emph{Vision of the Last
Judgment} \citep{blake1810judgment}) maps directly onto the epistemic
structure of Active Inference: internal states (beliefs) are all we
access. External states are inferred, never directly known.
``Corporeal'' = external states beyond the Markov blanket. We have no
direct access to the world-in-itself; only to our model's predictions
about it. What we call ``corporeal'' is a posit of inference, not an
immediate given.

This is not solipsism but epistemic humility: acknowledging that all our
knowledge is model-mediated.

\subsubsection{Knowledge by Perception, Not
Deduction}\label{knowledge-by-perception-not-deduction}

Blake distinguishes parallel inference from serial reasoning:

\begin{quote}
\emph{``Knowledge is not by deduction but Immediate by Perception or
Sense at once Christ addresses himself to the Man not to his Reason''}

--- Annotations to Berkeley's \emph{Siris} \citep{blake1820berkeley}
\end{quote}

Perception as parallel inference, not serial deduction. Posterior
beliefs emerge from free energy minimization directly---not through
step-by-step logical chains but through the simultaneous settling of the
entire generative model. ``Christ addresses himself to the Man'' = the
world speaks to the whole agent, not just the reasoning faculty.

This anticipates the Active Inference insight that perception is not a
conclusion of reasoning but an immediate update of the entire belief
distribution.

\subsubsection{Innate Ideas as Structural
Priors}\label{innate-ideas-as-structural-priors}

Against empiricist doctrine, Blake insists on innate structure:

\begin{quote}
\emph{``The Man who says that we have No Innate Ideas must be a Fool \&
Knave\ldots{} Knowledge of Ideal Beauty is Not to be Acquired It is Born
with us''}

--- Annotations to Reynolds' Discourses \citep{blake1808reynolds}
\end{quote}

Structural priors are architectural, not learned. ``Innate Ideas'' = the
priors that make inference possible. They cannot be ``acquired'' because
they define the hypothesis space within which acquisition occurs.
Without prior structure, there is nothing to update---no model to
receive evidence.

\subsubsection{Imagination as Epistemic
Foraging}\label{imagination-as-epistemic-foraging}

If imagination is human existence---if the self \emph{is} the generative
model---then what Blake calls ``creative vision'' amounts to a specific
computational strategy: \emph{epistemic foraging} through counterfactual
model-space. Rather than passively receiving data, the imaginative agent
actively explores hypothetical configurations of its own generative
model, testing alternatives that minimize expected free energy over long
horizons \citep{veissiere2020thinking}. This is niche construction at
the cognitive level: the imagination does not merely adapt to the world
as found but actively reshapes the model-space within which future
inference occurs. Blake's insistence that ``What is now proved was once
only imagin'd'' is, in Active Inference terms, the claim that today's
priors were yesterday's epistemic actions---imaginative explorations
that became entrained belief structures. The artist, the prophet, the
visionary is thus not an escapist but an \emph{epistemic pioneer},
foraging at the frontier of model-space.

\newpage

\subsection{Time: Temporal Horizons}\label{time}

\begin{quote}
\emph{``I see the Past, Present \& Future, existing all at once}\\
\emph{Before me.''}

--- \emph{Jerusalem}, Plate 15 \citep{blake1804jerusalem}
\end{quote}

\subsubsection{Temporal Horizons}\label{temporal-horizons}

``Eternity in an hour'' is not mysticism---it is \emph{deep temporal
modeling}. In Active Inference, agents construct hierarchical generative
models that encode predictions at progressively longer time scales, from
millisecond sensory fluctuations to narratives spanning years. The
deeper the hierarchy, the wider the temporal horizon the agent can
integrate into present awareness. Blake's compression of eternity into
an hour describes exactly this capacity. Active Inference agents
maintain predictions across multiple time scales:

\begin{quote}
``The lowest level of this hierarchy corresponds to fast fluctuations
associated with sensory processing, whereas the highest levels encode
slow contextual changes in the environment.''

--- \citep{kiebel2008hierarchy}
\end{quote}

\begin{quote}
``Slowly changing neuronal states can encode the paths or trajectories
of faster sensory states.''

--- \citep{friston2017deep}
\end{quote}

\textbf{Temporal hierarchy:}

\begin{equation}\label{eq:temporal_hierarchy}
p(o_{1:T}, \theta) = \prod_{t=1}^{T} p(o_t | \theta_t) \cdot p(\theta_t | \theta_{t-1})
\end{equation}

The deeper the temporal model (Equation \ref{eq:temporal_hierarchy}),
the longer the horizon of prediction. An impoverished model predicts
only the immediate. A rich model extends to \emph{aeonic} time. Figure
\ref{fig:temporal} illustrates this hierarchy of temporal scales.

\begin{figure}
\centering
\pandocbounded{\includegraphics[keepaspectratio,alt={Temporal Horizons of Inference. Four stacked trapezoid bands, widening from bottom to top, represent increasing temporal depth in the hierarchical generative model (Equation ). Fast (red, milliseconds): sensory processing---the immediate registration of prediction errors. Mid (steel blue, seconds to minutes): emotional integration---affective states that modulate precision weighting across brief episodes. Slow (purple, hours to years): narrative construction---the autobiographical and cultural models that contextualize experience. Deep/Aeonic (gold, lifetimes to eternity): the deepest temporal priors encoding civilizational and transpersonal patterns. The vertical arrow indicates increasing temporal depth. Blake's compression of temporal scales---``Every Time less than a pulsation of the artery / Is equal in its period \& value to Six Thousand Years'' (Milton, Plate 29 {[}@blake1804milton{]})---describes the hierarchical nesting where each level encodes trajectories of the level below, and ``Eternity is in love with the productions of time'' (Marriage of Heaven and Hell, Proverbs of Hell {[}@blake1790marriage{]}).}]{../figures/fig6_temporal_horizons.png}}
\caption{\textbf{Temporal Horizons of Inference.} Four stacked trapezoid
bands, widening from bottom to top, represent increasing temporal depth
in the hierarchical generative model (Equation
\ref{eq:temporal_hierarchy}). \textbf{Fast} (red, milliseconds): sensory
processing---the immediate registration of prediction errors.
\textbf{Mid} (steel blue, seconds to minutes): emotional
integration---affective states that modulate precision weighting across
brief episodes. \textbf{Slow} (purple, hours to years): narrative
construction---the autobiographical and cultural models that
contextualize experience. \textbf{Deep/Aeonic} (gold, lifetimes to
eternity): the deepest temporal priors encoding civilizational and
transpersonal patterns. The vertical arrow indicates increasing temporal
depth. Blake's compression of temporal scales---``Every Time less than a
pulsation of the artery / Is equal in its period \& value to Six
Thousand Years'' (\emph{Milton}, Plate 29
\citep{blake1804milton})---describes the hierarchical nesting where each
level encodes trajectories of the level below, and ``Eternity is in love
with the productions of time'' (\emph{Marriage of Heaven and Hell},
Proverbs of Hell \citep{blake1790marriage}).}\label{fig:temporal}
\end{figure}

Blake's ``Eternity'' is not timelessness but \emph{trans-temporal
integration}---the capacity to hold all moments within the present
perception. The hour contains eternity because the model reaches across
all scales.

In \emph{Milton}, Blake provides temporal metrics:

\begin{quote}
\emph{``Every Time less than a pulsation of the artery\emph{ }Is equal
in its period \& value to Six Thousand Years,\emph{ }For in this Period
the Poet's Work is Done, and all the Great\emph{ }Events of Time start
forth \& are conceiv'd in such a Period,\emph{ }Within a Moment, a
Pulsation of the Artery.''}

--- \emph{Milton}, Plate 29 \citep{blake1804milton}
\end{quote}

A pulsation equals six thousand years---the prophetic compression of
temporal scale. And the relationship is not opposition but love:

\begin{quote}
\emph{``Eternity is in love with the productions of time.''}

--- \emph{Marriage of Heaven and Hell}, Proverbs of Hell
\citep{blake1790marriage}
\end{quote}

The eternal reaches toward the temporal particular. Deep temporal models
don't transcend time but \emph{integrate} it---holding the trajectory of
moments within present awareness.

\subsubsection{The Temporal Architecture of
Los}\label{the-temporal-architecture-of-los}

But Blake does not stop at compression. In the lines immediately
preceding the ``pulsation'' climax, he constructs the most extraordinary
precursor to hierarchical temporal modeling in all of literature:

\begin{quote}
\emph{``But others of the Sons of Los build Moments \& Minutes \&
Hours\emph{ }And Days \& Months \& Years \& Ages \& Periods; wondrous
buildings\emph{ }And every Moment has a Couch of gold for soft
repose,\emph{ }(A Moment equals a pulsation of the artery),\emph{ }And
between every two Moments stands a Daughter of Beulah\emph{ }To feed the
Sleepers on their Couches with maternal care.\emph{ }And every Minute
has an azure Tent with silken Veils.\emph{ }And every Hour has a bright
golden Gate carved with skill.\emph{ }And every Day \& Night, has Walls
of brass \& Gates of adamant,\emph{ }Shining like precious stones \&
ornamented with appropriate signs:\emph{ }And every Month, a silver
paved Terrace builded high:\emph{ }And every Year, invulnerable Barriers
with high Towers.\emph{ }And every Age is Moated deep with Bridges of
silver \& gold.\emph{ }And every Seven Ages is Incircled with a Flaming
Fire.''}

--- \emph{Milton}, Plate 28 \citep{blake1804milton}
\end{quote}

This is a hierarchical generative model of temporal experience,
described two centuries before the formal mathematics. Each timescale
has distinct \emph{architectural properties}---distinct structure,
material, and boundary conditions:

{\def\LTcaptype{none} % do not increment counter
\begin{longtable}[]{@{}
  >{\raggedright\arraybackslash}p{(\linewidth - 4\tabcolsep) * \real{0.2462}}
  >{\raggedright\arraybackslash}p{(\linewidth - 4\tabcolsep) * \real{0.3385}}
  >{\raggedright\arraybackslash}p{(\linewidth - 4\tabcolsep) * \real{0.4154}}@{}}
\toprule\noalign{}
\begin{minipage}[b]{\linewidth}\raggedright
Temporal Scale
\end{minipage} & \begin{minipage}[b]{\linewidth}\raggedright
Blake's Architecture
\end{minipage} & \begin{minipage}[b]{\linewidth}\raggedright
Active Inference Analogue
\end{minipage} \\
\midrule\noalign{}
\endhead
\bottomrule\noalign{}
\endlastfoot
Moment (pulse) & Couch of gold & Fast sensory states: soft, immediate,
precious \\
Minute & Azure Tent, silken Veils & Short-term precision: flexible,
semi-transparent \\
Hour & Golden Gate, carved & Attentional boundaries: structured,
deliberate \\
Day/Night & Brass Walls, adamant Gates & Circadian priors: rigid,
durable \\
Month & Silver Terrace, builded high & Seasonal/contextual priors:
elevated perspective \\
Year & Invulnerable Barriers, high Towers & Biographical priors:
strongly defended \\
Age & Moated deep, Bridges of silver \& gold & Cultural priors: deep
separation, costly access \\
Seven Ages & Flaming Fire & Civilizational priors: ultimate boundary \\
\end{longtable}
}

The ascending material solidity---from gold couches to flaming
fire---mirrors the increasing precision and decreasing update rate of
deeper temporal levels. A ``Couch'' yields to the body; a ``Flaming
Fire'' does not. The Sons of Los \emph{build} these structures actively:
temporal experience is not passively received but constructed through
inference. And the ``Daughter of Beulah'' standing between each two
Moments, feeding ``the Sleepers on their Couches with maternal care,''
is the precision-weighting mechanism that mediates between adjacent
temporal levels---ensuring smooth transitions and preventing
catastrophic discontinuity.

\subsubsection{Drawing Out to Seven Thousand
Years}\label{drawing-out-to-seven-thousand-years}

Blake describes the active construction of temporal depth through the
figure of Eno, an ancient prophetess:

\begin{quote}
\emph{``\[Eno\] drew it out to Seven thousand years with much care \&
affliction''}

--- \emph{Vala, or The Four Zoas}, Night the First
\citep[E300]{blake1797fourzoas}
\end{quote}

The model's temporal horizon can be extended---``drawn out''---but not
without cost. Blake's ``care \& affliction'' captures the computational
expense of deep temporal models: maintaining predictions across long
horizons demands sustained precision allocation and imposes metabolic
cost on the system. The model does not naturally reach across millennia
without deliberate cultivation of the hierarchical structure that makes
such depth possible.

Seven thousand years echoes the Biblical span of human history---Eno's
work is to hold the entire trajectory of human time within present
awareness.

\subsubsection{Eternity Obliterated}\label{eternity-obliterated}

The collapse of temporal depth is catastrophic:

\begin{quote}
\emph{``Till like a dream Eternity was obliterated \& erasd''}

--- \emph{The Song of Los}, Plate 3 \citep{blake1795songlos}
\end{quote}

Deep temporal models (Eternity) compressed to shallow prediction
(dream-like immediate present). When the temporal horizon collapses,
what remains is reactive, stimulus-bound perception---the ``dream''
state of shallow inference.

This describes the Fall as temporal impoverishment: from aeonic
awareness to moment-to-moment reaction.

\subsubsection{Drunk with the Wine of
Ages}\label{drunk-with-the-wine-of-ages}

But temporal integration has limits:

\begin{quote}
\emph{``drunk with the wine of ages''}

--- \emph{Vala, or The Four Zoas}, Night the Ninth
\citep{blake1797fourzoas}
\end{quote}

Over-accumulation of historical evidence (``wine of ages'') impairs
present inference. The temporal model can become so saturated with past
that it loses responsiveness to present. This is the opposite failure
mode: not temporal collapse but temporal rigidity---being so weighted by
history that current observation cannot update the model.

In Active Inference terms, this corresponds to an over-fitted temporal
model whose deep priors have accumulated excessive precision. When
\(\pi_{\text{prior}}\) at the slowest temporal scales grows unboundedly,
the agent becomes captive to historical regularities and cannot
accommodate novel evidence. The system is ``drunk''---its inference is
dominated by the accumulated weight of temporal priors rather than
responsive to present sensory input. Blake thus identifies both failure
modes of temporal modeling: too shallow (``Eternity obliterated'') and
too deep (``drunk with the wine of ages'').

\begin{quote}
\textbf{Demonstration: The Déjà Vu Moment}

You walk into an unfamiliar room and feel, with absolute conviction,
that you have been here before. The moment is uncanny precisely because
two temporal scales are colliding: your shallow present-moment model
(novel room, first visit) contradicts a deeper temporal pattern-match
(this configuration of light, shape, and atmosphere resonates with a
prior encoded at longer timescales). In Active Inference terms, déjà vu
arises when a slow-timescale prior generates a strong top-down
prediction that the current observation is \emph{familiar}, while the
fast-timescale model correctly registers \emph{novelty}. The eerie
sensation is the prediction error between hierarchical temporal
levels---Blake's ``Eternity in an hour'' experienced as perceptual
vertigo. Temporal depth is not abstract: it is the felt layering of past
within present.
\end{quote}

\newpage

\subsection{Space: Spatial Hierarchy}\label{space}

\begin{quote}
\emph{``To see a World in a Grain of Sand}\\
\emph{And a Heaven in a Wild Flower\ldots``}

--- \emph{Auguries of Innocence} \citep{blake1803auguries}
\end{quote}

\subsubsection{Spatial Scale Invariance}\label{spatial-scale-invariance}

The grain contains the world through \emph{hierarchical evidence
accumulation}. At each level of the generative model, increasingly
abstract regularities are extracted from observations: texture gives way
to form, form to spatial relations, spatial relations to universal
principles. A sufficiently deep hierarchy finds the cosmos in the
particular because the abstract structure encoded at its highest levels
applies universally. The predictive brain finds universal structure in
particular observations:

\begin{quote}
``By formulating Helmholtz's original ideas on perception in terms of
modern-day statistical theories, one arrives at a model of perceptual
inference and learning that can explain a remarkable range of
neurobiological facts.''

--- \citep{friston2005theory}
\end{quote}

\textbf{Model complexity:}

\begin{equation}\label{eq:model_complexity}
F_{\text{simple}} \gg F_{\text{rich}}
\end{equation}

A shallow model incurs vastly more free energy (Equation
\ref{eq:model_complexity}) than a rich one. The shallow model sees only
surface: the grain is sand. A deep model extracts universal structure:
the grain is cosmos in miniature.

{\def\LTcaptype{none} % do not increment counter
\begin{longtable}[]{@{}lll@{}}
\toprule\noalign{}
Model Depth & Perception & Blake's Image \\
\midrule\noalign{}
\endhead
\bottomrule\noalign{}
\endlastfoot
Shallow & Isolated particulars & ``narrow chinks'' \\
Intermediate & Contextual patterns & ``twofold vision'' \\
Deep & Universal in particular & ``World in a Grain'' \\
\end{longtable}
}

The ``Wild Flower'' opens to ``Heaven'' because a sufficiently deep
generative model finds infinite structure in finite observation. This is
not enhancement---it is \emph{accuracy}.

Blake's spatial cosmology extends the particularity principle:

\begin{quote}
\emph{``And every Space smaller than a Globule of Man's blood
opens\emph{ }Into Eternity of which this vegetable Earth is but a
shadow.''}

--- \emph{Milton}, Plate 29 \citep{blake1804milton}
\end{quote}

The vortex phenomenon---perception's expansive/contractive dynamics:

\begin{quote}
\emph{``The nature of infinity is this: That every thing has its\emph{
}Own Vortex, and when once a traveller thro' Eternity\emph{ }Has passd
that Vortex, he perceives it roll backward behind\emph{ }His path, into
a globe itself infolding like a sun.''}

--- \emph{Milton}, Plate 15 \citep{blake1804milton}
\end{quote}

And the expansion principle:

\begin{quote}
\emph{``If the Spectator could Enter into these Images in his
Imagination, approaching them on the Fiery Chariot of his Contemplative
Thought\ldots{} then would he arise from his Grave, then would he meet
the Lord in the Air \& then he would be happy.''}

--- \emph{Vision of the Last Judgment} \citep{blake1810judgment}
\end{quote}

Spatial perception is active---the spectator \emph{enters} images,
doesn't passively receive them. The generative model projects into the
world as much as it receives from it.

\subsubsection{The Vortex: Each Entity's Own Markov
Blanket}\label{the-vortex-each-entitys-own-markov-blanket}

Blake's fullest treatment of perspectival space appears in
\emph{Milton}:

\begin{quote}
\emph{``The nature of infinity is this: That every thing has its / Own
Vortex; and when once a traveller thro Eternity / Has passd that Vortex,
he percieves it roll backward behind / His path, into a globe itself
infolding like a sun''}

--- \emph{Milton}, Plate 15, lines 21-35 \citep{blake1804milton}
\end{quote}

Each entity has its own Markov blanket and generative model---its ``Own
Vortex.'' ``Passing the Vortex'' = adopting a new perspective, entering
another entity's coordinate frame. ``Globe itself infolding'' = the old
model becoming an object within the new model's representation.

The vortex is the boundary between reference frames. When you pass
through another entity's vortex, you see from their perspective---but
looking back, your old perspective has become a distant object, a
``globe'' or ``sun'' behind you.

In Active Inference, this corresponds to \emph{model switching}---the
capacity to adopt another agent's generative model as one's own frame of
reference. Friston and Frith's ``Duet for One'' \citep{friston2015duet}
formalizes precisely this process: two agents can achieve generalized
synchronization by inferring each other's hidden states, effectively
``passing through'' each other's Markov blankets. Blake's vortex adds a
phenomenological dimension that the formalism captures only
structurally: the felt experience of entering another perspective and
finding that one's prior viewpoint has become a distant object, reduced
to a ``globe'' in the rearview of awareness.

This is the phenomenology of empathy formalized: to understand another
is to pass through their vortex, to see from within their generative
model.

\subsubsection{Mathematic Holiness: The Impoverished Spatial
Model}\label{mathematic-holiness-the-impoverished-spatial-model}

Blake names the pathological reduction of spatial perception with
devastating precision. In \emph{Milton}, he describes those whose models
have been stripped of imaginative depth:

\begin{quote}
\emph{``those combind by Satans Tyranny\ldots{} are Shapeless
Rocks\emph{ }Retaining only Satans Mathematic Holiness, Length: Bredth
\& Highth''}

--- \emph{Milton}, Plate 32 \citep{blake1804milton}
\end{quote}

``Mathematic Holiness'' is the worship of pure geometric
extension---Euclidean coordinates elevated to metaphysical status. When
imaginative depth is removed from spatial perception, what remains is
the thin, quantitative skeleton: length, breadth, height---the minimum
parameters of a bounding box. The ``Shapeless Rocks'' are agents whose
generative models have collapsed to this minimum complexity: they can
localize objects in three-dimensional space but cannot perceive the
hierarchical structure that makes space \emph{meaningful}---the nested
affordances, the perspectival depth, the relational richness that a deep
model extracts from the same optical array.

In Active Inference terms, this is the spatial analogue of Newton's
Sleep (see \hyperref[states]{§4.3 States}). A model that retains
``only'' length, breadth, and height is a model whose spatial hierarchy
has been flattened to a single level: the likelihood function \(p(o|s)\)
computes geometric coordinates, but the prior structure \(p(s)\) that
would encode ecological meaning, bodily affordance, and perspectival
significance has been suppressed. The result is the ``vegetable'' space
of Newtonian mechanics---measurable, uniform, and dead. Blake's ``World
in a Grain of Sand'' is the antithesis: spatial perception at full
hierarchical depth, where the generative model finds universal structure
in local observation because its prior hierarchy is rich enough to make
the extraction possible.

\subsubsection{Looking Through, Not
With}\label{looking-through-not-with}

Blake distinguishes the sensor from the inference process:

\begin{quote}
\emph{``I question not my Corporeal or Vegetative Eye any more than I
would Question a Window concerning a Sight I look thro it \& not with
it''}

--- \emph{Vision of the Last Judgment} \citep[E566]{blake1810judgment}
\end{quote}

Blake distinguishes the sensor from the inference engine with surgical
precision. The eye is the likelihood function \(p(o|s)\)---the mapping
from hidden states to observations---not the inference process itself.
To look ``through'' the eye is to use the full generative model: the
inverse mapping from observations to beliefs about hidden causes. To
look ``with'' the eye is to confuse the sensor with inference, mistaking
the data channel for the interpretation.

The eye provides observations; it does not provide understanding.
Understanding requires the generative model that interprets what the eye
delivers. To look ``with'' the eye is to mistake the window for the
view---an error that reduces perception to passive reception and
forecloses the active, model-driven inference that constitutes genuine
seeing.

This is a statement about the distinction between the likelihood
function and the full Bayesian inversion. The eye computes \(p(o|s)\);
seeing requires the posterior \(p(s|o) \propto p(o|s) \cdot p(s)\).
Confusing the two is the fundamental error of empiricism---Blake's
``single vision.''

\begin{quote}
\textbf{Demonstration: The Microscope Effect}

Place a leaf under a microscope. At first you see nothing but
undifferentiated green blur---single vision. Adjust the focus: suddenly,
cells appear---a world \emph{within} the leaf, invisible to the unaided
eye. Increase magnification: within each cell, organelles, a further
world. Blake's ``world in a grain of sand'' is not metaphor but
phenomenological report: the grain reveals a world precisely when the
generative model gains hierarchical depth, when new layers of the
likelihood function \(p(o|s)\) map observations to increasingly
fine-grained hidden causes. The microscope changes the observation
channel, but it is the model---the skilled biologist's expectations
about what cell types, structures, and processes \emph{should} be
visible---that transforms undifferentiated green into meaningful
architecture. The instrument is the window; the model is the sight.
\end{quote}

\newpage

\subsection{Action: Free Energy Minimization}\label{action}

\begin{quote}
\emph{``I must Create a System. or be enslav'd by another Mans. I will
not Reason \& Compare: my business is to Create.''}

--- \emph{Jerusalem}, Plate 10 \citep{blake1804jerusalem}
\end{quote}

\subsubsection{Free Energy Minimization}\label{free-energy-minimization}

Los's imperative to \emph{create} rather than merely reason captures the
essence of active inference: the agent does not passively receive the
world but actively constructs its model. The doors are cleansed when the
generative model accurately mirrors reality. As the foundational paper
states:

\begin{quote}
``The free-energy principle provides a unified account of action,
perception and learning based on minimizing a measure of surprise.''

--- \citep{friston2010free}
\end{quote}

\begin{quote}
``Cortical responses can be seen as the brain's attempt to minimize the
free energy induced by a stimulus and thereby encode the most likely
cause of that stimulus.''

--- \citep{friston2006free}
\end{quote}

\textbf{Cleansing formalized:}

\begin{equation}\label{eq:cleansing}
\text{Cleansed perception} \iff \arg\min_q F(q, o)
\end{equation}

Free energy reaches minimum when:

\begin{itemize}
\tightlist
\item
  Model predictions match observations
\item
  Prior beliefs calibrate to evidence
\item
  Precision-weighting is optimal
\end{itemize}

Two paths to cleansing:

\begin{enumerate}
\def\labelenumi{\arabic{enumi}.}
\tightlist
\item
  \textbf{Perception} --- Update beliefs to fit world
  (\(q(\theta) \rightarrow p(\theta | o)\))
\item
  \textbf{Action} --- Change world to fit model
  (\(o \rightarrow \hat{o}\))
\end{enumerate}

Both are ``cleansing.'' Both reduce free energy (Equation
\ref{eq:cleansing}), driving the variational bound (Equation
\ref{eq:free_energy}) toward its minimum (Figure \ref{fig:cycle}). Blake
unified what later theory separated.

The generative model anticipates sensory confirmation---imagination
precedes proof, the prophet \emph{predicts} what observation will
confirm. What Blake imagined, Friston's equations now prove.

Blake's energy philosophy aligns with active inference's emphasis on
action:

\begin{quote}
\emph{``Energy is Eternal Delight.''}

--- \emph{Marriage of Heaven and Hell}, Plate 4
\citep{blake1790marriage}
\end{quote}

The labor principle extends this creative imperative to situated
practice:

\begin{quote}
\emph{``Great things are done when Men \& Mountains meet;\emph{ }This is
not done by Jostling in the Street.''}

--- Notebook (c.~1807-1809) \citep[E516]{blake1988complete}
\end{quote}

\begin{quote}
\emph{``Labour well the Minute Particulars, attend to the
Little-ones.''}

--- \emph{Jerusalem}, Plate 55, line 51 \citep{blake1804jerusalem}
\end{quote}

Action on the minute particular---the precise, situated
intervention---is how free energy is minimized in practice. Not grand
abstraction but attentive labor.

\subsubsection{Flexible Senses: Precision Dynamics Before the
Fall}\label{flexible-senses-precision-dynamics-before-the-fall}

Blake's most direct statement of voluntary precision modulation appears
in \emph{The Book of Urizen}:

\begin{quote}
\emph{``The will of the Immortal expanded / Or contracted his all
flexible senses''}

--- \emph{The Book of Urizen}, Chapter II, Plate 3
\citep{blake1794urizen}
\end{quote}

Before the Fall, precision (\(\pi)) could be adjusted at
will---``expanded or contracted.'' This is Blake's clearest statement of
precision dynamics: the unfallen condition is one where attention can be
freely allocated.

The ``all flexible senses'' are not fixed channels but adjustable
receptors. Flexibility is the default; rigidity is fallen.

\subsubsection{Contracting to the Honey
Bee}\label{contracting-to-the-honey-bee}

Blake elaborates the scale of precision adjustment:

\begin{quote}
\emph{``Contracting or expanding their all flexible senses / At will to
murmur in the flowers small as the honey bee''}

--- \emph{Vala, or The Four Zoas}, Night the First
\citep{blake1797fourzoas}
\end{quote}

Senses can zoom to ``honey bee'' scale---precision determines
resolution. The unfallen beings can adjust their sensory precision to
perceive at any scale, from cosmic to microscopic. This is optimal
precision assignment: the ability to weight sensory channels
appropriately for the task at hand.

\subsubsection{Love as Affective
Precision}\label{love-as-affective-precision}

Precision weighting operates through affect:

\begin{quote}
\emph{``He who Loves feels love descend into him \& if he has wisdom may
percieve it is from the Poetic Genius which is the Lord''}

--- Annotations to Swedenborg's \emph{Divine Love and Divine Wisdom}
\citep{blake1788swedenborg}
\end{quote}

``Love'' = affective precision. ``Descend'' = top-down modulation.
``Wisdom'' = meta-cognitive precision awareness---the ability to
perceive the source of one's attention.

Affect is not separate from inference but constitutive of it. Love
determines what matters, what receives high precision weighting.

\subsubsection{Thought Alone Makes
Monsters}\label{thought-alone-makes-monsters}

But inference without affective grounding drifts pathologically:

\begin{quote}
\emph{``Thought alone can make monsters, but the affections cannot''}

--- Annotations to Swedenborg \citep{blake1788swedenborg}
\end{quote}

Inference without affective grounding produces biologically non-viable
beliefs. ``Monsters'' = pathological states that no embodied system
could inhabit. Affect anchors inference to survival, to what matters for
the organism's persistence.

Pure reasoning, unmoored from bodily concern, can generate coherent but
monstrous conclusions. The affections keep inference honest. Damasio's
somatic marker hypothesis formalizes Blake's insight with striking
precision: bodily affect tags candidate beliefs and actions with valence
\emph{before} deliberative reasoning can evaluate them
\citep{damasio1994descartes}. The ``affections'' are somatic
markers---precision signals weighted by visceral relevance. Without
these embodied priors, the reasoning system falls into what Damasio
terms the ``high-reason'' catastrophe: logically valid but existentially
disastrous decisions, monsters of pure thought.

\begin{quote}
\textbf{Demonstration: The Musician's Practice}

A pianist learning a new piece begins haltingly: each note requires
conscious prediction, deliberate action, effortful error correction. The
prediction errors are large and frequent---every misplayed note
generates surprise. Through practice, the generative model refines: the
fingers begin to anticipate the score, action policy and sensory
prediction converge, surprise decreases. But something else happens: the
music begins to \emph{feel} right. Affect---the bodily sense of
rightness---becomes the precision signal that guides further refinement.
The musician doesn't just minimize error; she shapes her model until it
produces the \emph{felt quality} of beautiful performance. This is
Blake's ``Mental Fight'' rendered as craft: active inference through
embodied skill, where action transforms both world (the sound produced)
and model (the musician's internal representation). The pianist
``creates a system'' note by note, bar by bar---and in doing so, is
herself transformed.
\end{quote}

\subsubsection{Cogs Tyrannic vs.~Free
Wheels}\label{cogs-tyrannic-vs.-free-wheels}

Blake contrasts deterministic with flexible inference:

\begin{quote}
\emph{``Wheel without wheel, with cogs tyrannic / Moving by compulsion
each other: not as those in Eden, which / Wheel within Wheel in freedom
revolve in harmony \& peace''}

--- \emph{Jerusalem}, Plate 15 \citep{blake1804jerusalem}
\end{quote}

``Cogs tyrannic'' = fixed precision, mechanical prediction. No free
parameters, no flexibility---each wheel forces the next. ``Wheel within
Wheel in freedom'' = adjustable precision weighting, where components
coordinate but are not rigidly locked.

The Edenic state is not absence of structure but \emph{flexible}
structure---wheels that revolve together in harmony without tyrannical
compulsion. This is the difference between a generative model that can
revise itself and one frozen in Newton's sleep.

\subsubsection{The Path of Least Action}\label{the-path-of-least-action}

Friston's path integral formulation of the free energy principle
\citep{friston2023path} reveals a deeper connection to Blake's economy
of perception. The path integral expresses system trajectories as
weighted sums over all possible paths, with the most probable trajectory
being the one that minimizes action---the ``path of least action.''
Blake's imperative to ``cleanse the doors'' is, in this formalism, a
call to find the self-evidencing trajectory: the path through
model-space that minimizes free energy while maintaining the system's
structural integrity. The ``cogs tyrannic'' are paths constrained to a
single rigid trajectory; the ``wheels in freedom'' are paths that
explore the full space of possibilities while converging on the
variational minimum. Art, for Blake, is precisely this search: the
creative act finds the path of least free energy through the space of
possible forms, arriving at the work that resolves the most prediction
error with the most elegant structure.

\begin{figure}
\centering
\pandocbounded{\includegraphics[keepaspectratio,alt={The Perception-Action Cycle. Active Inference's dual optimization pathways annotated with corresponding Blake quotations. The central generative model p(o, \textbackslash theta) (orange hub) issues top-down predictions and receives bottom-up prediction errors \textbackslash varepsilon = o - g(\textbackslash theta) through six interconnected stages: Prediction (``What is now proved was once imagined''), Sensory Input (``The senses discover'd the infinite''), Prediction Error (``Narrow chinks of his cavern''), Model Update (``Cleansed perception''), Action Selection (``Mental Fight''), and World Change (``Building Jerusalem''). Perceptual inference updates beliefs q(\textbackslash theta) toward the true posterior (changing mind to fit world); active inference samples observations matching predictions (changing world to fit mind). Both pathways reduce variational free energy F (Equation ). The cycle's continuous operation corresponds to Blake's vision of perpetual creative labor: ``I must Create a System. or be enslav'd by another Mans'' (Jerusalem, Plate 10 {[}@blake1804jerusalem{]}).}]{../figures/fig3_perception_action_cycle.png}}
\caption{\textbf{The Perception-Action Cycle.} Active Inference's dual
optimization pathways annotated with corresponding Blake quotations. The
central generative model \(p(o, \theta)\) (orange hub) issues top-down
predictions and receives bottom-up prediction errors
\(\varepsilon = o - g(\theta)\) through six interconnected stages:
\textbf{Prediction} (``What is now proved was once imagined''),
\textbf{Sensory Input} (``The senses discover'd the infinite''),
\textbf{Prediction Error} (``Narrow chinks of his cavern''),
\textbf{Model Update} (``Cleansed perception''), \textbf{Action
Selection} (``Mental Fight''), and \textbf{World Change} (``Building
Jerusalem''). Perceptual inference updates beliefs \(q(\theta)\) toward
the true posterior (changing mind to fit world); active inference
samples observations matching predictions (changing world to fit mind).
Both pathways reduce variational free energy \(F\) (Equation
\ref{eq:cleansing}). The cycle's continuous operation corresponds to
Blake's vision of perpetual creative labor: ``I must Create a System. or
be enslav'd by another Mans'' (\emph{Jerusalem}, Plate 10
\citep{blake1804jerusalem}).}\label{fig:cycle}
\end{figure}

\newpage

\subsection{Collectives: Shared Generative Models}\label{collectives}

\begin{quote}
\emph{``I will not cease from Mental Fight,}\\
\emph{Nor shall my Sword sleep in my hand:}\\
\emph{Till we have built Jerusalem,}\\
\emph{In England's green \& pleasant Land.''}

--- \emph{Milton}, Preface \citep{blake1804milton}
\end{quote}

\subsubsection{Shared Generative Models}\label{shared-generative-models}

Blake's vision extends beyond individual perception to \emph{collective
awakening}. Jerusalem is not merely personal enlightenment but a shared
visionary capacity---a coordinated mode of seeing that transcends the
individual.

Active Inference extends naturally to multi-agent systems:

\begin{quote}
``Generalized synchronization \[emerges\] as a mathematical image of
communication that enables two Bayesian brains to entrain each other
and, effectively, share the same dynamical narrative.''

--- \citep{friston2015duet}
\end{quote}

\begin{quote}
``Human agents learn the shared habits, norms, and expectations of their
culture through immersive participation in patterned cultural practices
that selectively pattern attention and behaviour.''

--- \citep{veissiere2020thinking}
\end{quote}

This is TTOM---Thinking Through Other Minds---the mechanism of
collective awakening.

\textbf{Multi-agent coordination:}

\begin{equation}\label{eq:multi_agent}
p(o, \theta) = \prod_{i=1}^{N} p(o_i | \theta_i) \cdot p(\theta_i | \theta_{\text{shared}}) \cdot p(\theta_{\text{shared}})
\end{equation}

Multiple agents share a common prior \(\theta_{\text{shared}}\)
(Equation \ref{eq:multi_agent})---the cultural generative model that
enables coordinated perception and action. Figure \ref{fig:jerusalem}
illustrates this multi-agent architecture.

\begin{figure}
\centering
\pandocbounded{\includegraphics[keepaspectratio,alt={Building Jerusalem: Collective Generative Models. Three individual agents, each bounded by its own Markov blanket, contribute to and draw from a shared generative model (``Jerusalem,'' \textbackslash theta\_\{\textbackslash text\{shared\}\}). The ``Mental Fight'' zone represents the active process of model-building and coordination through which individual inference aligns with collective priors (Equation ).}]{../figures/fig7_collective_jerusalem.png}}
\caption{Building Jerusalem: Collective Generative Models. Three
individual agents, each bounded by its own Markov blanket, contribute to
and draw from a shared generative model (``Jerusalem,''
\(\theta_{\text{shared}}\)). The ``Mental Fight'' zone represents the
active process of model-building and coordination through which
individual inference aligns with collective priors (Equation
\ref{eq:multi_agent}).}\label{fig:jerusalem}
\end{figure}

Each individual agent remains bounded by its own Markov blanket
(Equation \ref{eq:conditional_independence}), but the shared prior
aligns their inference.

Blake's most radical claim about the collective nature of identity
appears in \emph{Milton}:

\begin{quote}
\emph{``We are not Individuals but States: Combinations of
Individuals''}

--- \emph{Milton}, Plate 32 \citep{blake1804milton}
\end{quote}

This is not merely the claim that agents share priors---it is the deeper
assertion that \emph{individual identity itself} is a collective
phenomenon. A ``Combination of Individuals'' is a factorization of
agency: what appears to be a single self is in fact a composition of
shared model components, cultural priors, and socially entrained
precision weightings. The shared prior \(\theta_{\text{shared}}\) in
Equation \ref{eq:multi_agent} is not external to individuals but
\emph{constitutive} of them---the individual \(\theta_i\) cannot be
separated from the collective without remainder. Blake's ``States'' are
thus collective attractors in the multi-agent generative model:
configurations that groups of agents fall into together, and that
``Change'' (dissolve, reconfigure) when the collective model is revised.
``Satan \& Adam are States Created into Twenty-seven Churches''---not
individuals who happened to organize churches, but \emph{model
configurations that manifest as institutional structures}.

{\def\LTcaptype{none} % do not increment counter
\begin{longtable}[]{@{}lll@{}}
\toprule\noalign{}
Component & Symbol & Blake's Image \\
\midrule\noalign{}
\endhead
\bottomrule\noalign{}
\endlastfoot
Individual models & \(\theta_i\) & Each perceiver \\
Shared prior & \(\theta_{\text{shared}}\) & Jerusalem \\
Collective action & \(a_{\text{collective}}\) & ``Mental Fight'' \\
Coordinated perception & \(o_{\text{aligned}}\) & Shared vision \\
\end{longtable}
}

\subsubsection{The Mental Fight}\label{the-mental-fight}

Blake's ``Mental Fight'' is model-building at civilizational scale:

\begin{itemize}
\tightlist
\item
  \textbf{Education} shapes the generative models of the young
\item
  \textbf{Art} restructures perception through alternative priors
\item
  \textbf{Contemplative practice} adjusts precision weighting
\item
  \textbf{Cultural production} creates shared predictive structures
\end{itemize}

\begin{quote}
\emph{``The Nature of my Work is Visionary or Imaginative; it is an
Endeavour to Restore what the Ancients called the Golden Age.''}

--- \emph{Vision of the Last Judgment} \citep{blake1810judgment}
\end{quote}

The Golden Age is not historical but \emph{perceptual}---a state where
collective generative models enable richer inference.

\subsubsection{Coordinated Inference}\label{coordinated-inference}

Active Inference extends naturally to multi-agent systems
\citep{friston2019markov}. Ramstead, Badcock, and Friston formalize this
extension through nested Markov blankets: blankets within blankets,
individuals within communities within cultures, each scale operating as
an autonomous inference system while coupling to the scales above and
below \citep{ramstead2018answering}. When agents share priors, they:

\begin{enumerate}
\def\labelenumi{\arabic{enumi}.}
\tightlist
\item
  \textbf{Align predictions} --- Expectations converge across the
  collective
\item
  \textbf{Coordinate action} --- Behavior becomes mutually
  intelligible\\
\item
  \textbf{Distribute computation} --- Complex inference divides across
  agents
\item
  \textbf{Accumulate evidence} --- Collective learning exceeds
  individual capacity
\end{enumerate}

Blake's Jerusalem is precisely this: a shared generative model enabling
coordinated cleansing of perception. The doors open not one by one, but
together.

\begin{quote}
\emph{``Mutual Forgiveness of each Vice---Such are the Gates of
Paradise.''}

--- \emph{For the Sexes: The Gates of Paradise}
\citep{blake1988complete}
\end{quote}

Paradise requires \emph{mutual} forgiveness---collective precision
adjustment where agents release rigid priors toward one another. The
gates open through shared model revision.

Blake's collective vision finds its fullest expression in
\emph{Jerusalem}:

\begin{quote}
\emph{``This is Jerusalem in every Man\emph{ }A Tent \& Tabernacle of
Mutual Forgiveness.''}

--- \emph{Jerusalem}, Plate 54 \citep{blake1804jerusalem}
\end{quote}

\begin{quote}
\emph{``O lovely Emanation of Albion Awake and overspread all Nations as
in Ancient Time\emph{ }For lo! the Night of Death is past and the
Eternal Day\emph{ }Appears upon our Hills.''}

--- \emph{Jerusalem}, Plate 97 \citep{blake1804jerusalem}
\end{quote}

The awakening is collective---Albion (England/humanity) awakens as a
unified agent.

\subsubsection{From Single to Collective
Vision}\label{from-single-to-collective-vision}

The fourfold hierarchy applies not only to individuals but to societies:

{\def\LTcaptype{none} % do not increment counter
\begin{longtable}[]{@{}lll@{}}
\toprule\noalign{}
Level & Individual & Collective \\
\midrule\noalign{}
\endhead
\bottomrule\noalign{}
\endlastfoot
\textbf{Single} & Mechanical perception & Industrial society \\
\textbf{Twofold} & Emotional engagement & Artistic communities \\
\textbf{Threefold} & Imaginative vision & Cultural movements \\
\textbf{Fourfold} & Integrated awareness & Jerusalem \\
\end{longtable}
}

Blake's critique of ``dark Satanic Mills'' is computational: industrial
modernity imposes shallow, prior-dominated generative models on the
collective. Building Jerusalem means reconstructing shared priors to
enable deeper inference.

\begin{quote}
\emph{``England! awake! awake! awake!}\\
\emph{Jerusalem thy Sister calls!{\kern0pt}``}

--- \emph{Jerusalem}, Plate 77 \citep{blake1804jerusalem}
\end{quote}

The awakening is collective. The sister calls to the nation. The doors
of perception---once cleansed---reveal not isolated infinity but
\emph{shared} infinity. The prophet's vision becomes the people's sight.

\subsubsection{Fall into Division and Resurrection to
Unity}\label{fall-into-division-and-resurrection-to-unity}

Blake frames the cosmic narrative as multi-agent decomposition and
re-coordination:

\begin{quote}
\emph{``Sing His fall into Division \& his Resurrection to Unity''}

--- \emph{Vala, or The Four Zoas}, Night the First
\citep{blake1797fourzoas}
\end{quote}

``Division'' = factorization into separate agents, each with its own
Markov blanket and generative model. ``Unity'' = re-coordination into a
shared generative model (Jerusalem). The Fall is not moral failure but
\emph{factorization}---the breaking apart of an integrated system into
competing subsystems.

Resurrection is the inverse: the re-establishment of shared priors that
enable coordinated inference across agents.

\subsubsection{The Eternal Man Is Risen}\label{the-eternal-man-is-risen}

The achievement of collective coordination:

\begin{quote}
\emph{``Rise from the dews of death for the Eternal Man is Risen''}

--- \emph{Vala, or The Four Zoas}, Night the Ninth
\citep{blake1797fourzoas}
\end{quote}

``Eternal Man'' (Albion) = the multi-agent system as unified entity.
When the Four Zoas are re-integrated, Albion rises---not as the sum of
parts but as the emergent coordination that parts enable.

\subsubsection{Human Harvest}\label{human-harvest}

Collective free energy minimization under stress:

\begin{quote}
\emph{``In pain the human harvest wavd in horrible groans of woe''}

--- \emph{Vala, or The Four Zoas} \citep{blake1797fourzoas}
\end{quote}

``Harvest'' = coordinated action across agents. ``Pain'' = high free
energy state. The collective strives toward lower free energy, but the
process is not painless---coordination requires the adjustment of
individual models to shared constraints.

\subsubsection{The Four Zoas: Factorized Collective Mind}\label{zoas}

Blake's most systematic account of multi-agent cognition appears in his
unfinished epic \emph{Vala, or The Four Zoas}. The Four Zoas---Urizen,
Urthona (fallen as Los), Luvah (fallen as Orc), and Tharmas---represent
not merely allegorical faculties but a \emph{factorized generative
model} where independent components must coordinate to achieve unified
inference.

\begin{quote}
\emph{``Four Mighty Ones are in every Man; a Perfect Unity / Cannot
Exist but from the Universal Brotherhood of Eden''}

--- \emph{Vala, or The Four Zoas}, Night the First
\citep{blake1797fourzoas}
\end{quote}

In Active Inference terms, factorization enables tractable computation:

\textbf{Factorized model:}

\begin{equation}\label{eq:factorized_model}
p(o, \theta) = p(o | \theta_U, \theta_L, \theta_{Lv}, \theta_T) \cdot p(\theta_U) \cdot p(\theta_L) \cdot p(\theta_{Lv}) \cdot p(\theta_T)
\end{equation}

where subscripts denote the four Zoas' contributions to the joint model.
This extends the general hierarchical factorization (Equation
\ref{eq:hierarchical_model}) into a multi-component architecture. Figure
\ref{fig:zoas} illustrates the compass-rose arrangement of the four
Zoas.

\begin{figure}
\centering
\pandocbounded{\includegraphics[keepaspectratio,alt={The Four Zoas: A Factorized Model of Mind. The four Zoas occupy cardinal positions---Urizen (South, reason/likelihood), Urthona/Los (North, imagination/prior), Luvah/Orc (East, passion/precision), and Tharmas (West, body/interoception)---around a central hub of unified inference. Coordination arcs connect adjacent Zoas; failure modes (prior dominance, model collapse, affective flooding, dissociation) appear when any single component tyrannizes. ``Perfect Unity'' requires the ``Universal Brotherhood'' of all four modes (Equation ).}]{../figures/fig5_four_zoas.png}}
\caption{The Four Zoas: A Factorized Model of Mind. The four Zoas occupy
cardinal positions---Urizen (South, reason/likelihood), Urthona/Los
(North, imagination/prior), Luvah/Orc (East, passion/precision), and
Tharmas (West, body/interoception)---around a central hub of unified
inference. Coordination arcs connect adjacent Zoas; failure modes (prior
dominance, model collapse, affective flooding, dissociation) appear when
any single component tyrannizes. ``Perfect Unity'' requires the
``Universal Brotherhood'' of all four modes (Equation
\ref{eq:factorized_model}).}\label{fig:zoas}
\end{figure}

\begin{figure}
\centering
\pandocbounded{\includegraphics[keepaspectratio,alt={William Blake, Milton a Poem, Plate 32 (c.~1804--1811). The four Zoas in their cosmic arrangement---the mythological source for the factorized model above. Blake depicts the fourfold division and interdependence of the faculties through characteristic visual symbolism. Relief etching with hand coloring. Courtesy of the William Blake Archive {[}@blake1804milton{]}.}]{../figures/the_four_zoas_egg_color.jpg}}
\caption{William Blake, \emph{Milton a Poem}, Plate 32 (c.~1804--1811).
The four Zoas in their cosmic arrangement---the mythological source for
the factorized model above. Blake depicts the fourfold division and
interdependence of the faculties through characteristic visual
symbolism. Relief etching with hand coloring. Courtesy of the William
Blake Archive \citep{blake1804milton}.}\label{fig:zoas_plate}
\end{figure}

\paragraph{The Four Components}\label{the-four-components}

{\def\LTcaptype{none} % do not increment counter
\begin{longtable}[]{@{}
  >{\raggedright\arraybackslash}p{(\linewidth - 8\tabcolsep) * \real{0.0962}}
  >{\raggedright\arraybackslash}p{(\linewidth - 8\tabcolsep) * \real{0.2115}}
  >{\raggedright\arraybackslash}p{(\linewidth - 8\tabcolsep) * \real{0.1538}}
  >{\raggedright\arraybackslash}p{(\linewidth - 8\tabcolsep) * \real{0.2692}}
  >{\raggedright\arraybackslash}p{(\linewidth - 8\tabcolsep) * \real{0.2692}}@{}}
\toprule\noalign{}
\begin{minipage}[b]{\linewidth}\raggedright
Zoa
\end{minipage} & \begin{minipage}[b]{\linewidth}\raggedright
Direction
\end{minipage} & \begin{minipage}[b]{\linewidth}\raggedright
Domain
\end{minipage} & \begin{minipage}[b]{\linewidth}\raggedright
AIF Function
\end{minipage} & \begin{minipage}[b]{\linewidth}\raggedright
Failure Mode
\end{minipage} \\
\midrule\noalign{}
\endhead
\bottomrule\noalign{}
\endlastfoot
\textbf{Urizen} & South & Reason, Law & Likelihood \(p(o\|\theta)\) &
Prior dominance (Newton's sleep) \\
\textbf{Urthona/Los} & North & Imagination, Prophecy & Prior structure
\(p(\theta)\) & Model collapse (despair) \\
\textbf{Luvah/Orc} & East & Passion, Emotion & Precision \(\pi) &
Affective flooding (chaos) \\
\textbf{Tharmas} & West & Body, Instinct & Interoception & Dissociation
(abstraction) \\
\end{longtable}
}

\paragraph{Urizen: The Likelihood
Function}\label{urizen-the-likelihood-function}

\begin{quote}
\emph{``And his Soul sicken'd! he curs'd / Both sons \& daughters; for
he saw / That no flesh nor spirit could keep / His iron laws one
moment.''}

--- \emph{The Book of Urizen}, Plate 23 \citep{blake1794urizen}
\end{quote}

Urizen represents the rational processing of evidence---the likelihood
function that evaluates how well observations fit hypotheses. His ``iron
laws'' are the regularities that structure prediction. But when Urizen
dominates, the system becomes rigid: prior-locked, unable to revise.

Urizen's failure is \emph{over-precision of priors}:
\(\pi_{\text{prior}} \to \infty\). The laws become tyrannical because
they cannot update.

\paragraph{Urthona/Los: Prior
Structure}\label{urthonalos-prior-structure}

\begin{quote}
\emph{``Los built the Walls of Golgonooza against the stirring battle''}

--- \emph{Jerusalem}, Plate 12 \citep{blake1804jerusalem}
\end{quote}

Urthona (unfallen) / Los (fallen) represents the creative
imagination---the prior structure that makes inference possible. Los
``builds'' Golgonooza, the city of art, which IS the generative model's
architecture.

Without Urthona/Los, there is no hypothesis space. The prior structure
defines what can be believed, what states are even conceivable. Los's
labor at the furnaces is the continuous work of model construction.

Los's failure is \emph{model collapse}: when imagination fails, the
prior structure dissolves, leaving no framework for inference. This is
despair---the inability to conceive alternatives.

\paragraph{Luvah/Orc: Precision
Weighting}\label{luvahorc-precision-weighting}

\begin{quote}
\emph{``Luvah is France, the Victim of the Spectres of Albion''}

--- \emph{Jerusalem}, Plate 66 \citep{blake1804jerusalem}
\end{quote}

Luvah (unfallen) / Orc (fallen) represents passion, emotion,
desire---the precision weighting that determines what matters, what
receives attention. Luvah controls the ``chariots of the morning''---the
affective salience that drives engagement.

Precision weighting is not merely attention but \emph{care}: what the
system treats as important, what prediction errors warrant response.

Luvah's failure is \emph{affective flooding}: when emotion dominates,
precision weights become extreme, the system oscillates chaotically,
unable to maintain stable inference. Orc's revolutionary fire burns
without discrimination.

\paragraph{Tharmas: Interoceptive
Inference}\label{tharmas-interoceptive-inference}

Blake's symbolic system assigns each Zoa to a distinct domain: Tharmas
governs the vegetal (bodily/sensory) world, Luvah the world of
sensations and emotion, Urizen the world of reason, and Urthona the
world of imagination. These correspondences pervade \emph{The Four Zoas}
though Blake expresses them through narrative action rather than
explicit formula.

Tharmas represents embodiment---the ``vegetal'' instinctual life,
interoceptive inference about the body's state. Blake dramatizes
Tharmas's devastation when separated from the other Zoas:

\begin{quote}
\emph{``Tharmas groand among his Clouds / Weeping, then silent
thundering he burst the bounds of Destiny / And shook the heavens with
wrath''}

--- \emph{Vala, or The Four Zoas}, Night the Third
\citep{blake1797fourzoas}
\end{quote}

Tharmas is often described as the most damaged of the Zoas, reduced to
helpless weeping in the sea of time and space---the body in distress
when severed from imagination, reason, and affect.

Seth's interoceptive inference framework formalizes this Blakean
insight: the self is constituted not merely by exteroceptive prediction
but by the body's ongoing inference about its own visceral states
\citep{seth2013interoceptive}. Tharmas IS interoceptive inference---the
felt sense of aliveness that grounds all other cognitive modes in
biological reality. Without Tharmas, the model floats free of
embodiment---pure abstraction without survival relevance.

Tharmas's failure is \emph{dissociation}: when embodiment is severed,
inference loses its anchor in biological viability. The system can
reason but cannot feel, can think but cannot care.

\paragraph{Coordination and Pathology}\label{coordination-and-pathology}

The Four Zoas must coordinate for healthy inference:

\begin{itemize}
\tightlist
\item
  \textbf{Urizen + Los}: Reason and imagination must balance---priors
  that can update, regularities that can revise
\item
  \textbf{Luvah + Tharmas}: Emotion and embodiment must align---what
  matters must connect to survival
\item
  \textbf{All Four}: The complete agent requires all four modes
  operating in ``Universal Brotherhood''
\end{itemize}

Pathology arises from \emph{imbalance}:

{\def\LTcaptype{none} % do not increment counter
\begin{longtable}[]{@{}lll@{}}
\toprule\noalign{}
Dominant Zoa & Condition & Clinical Parallel \\
\midrule\noalign{}
\endhead
\bottomrule\noalign{}
\endlastfoot
Urizen alone & Rigid rationalism & Obsessive-compulsive patterns \\
Luvah alone & Affective chaos & Borderline dysregulation \\
Los alone & Dissociated fantasy & Schizotypal withdrawal \\
Tharmas alone & Instinctual flooding & Panic, somatic fixation \\
\end{longtable}
}

\paragraph{The Unfallen State}\label{the-unfallen-state}

\begin{quote}
\emph{``they gave to it a Space \& namd the Space Ulro''}

--- \emph{Vala, or The Four Zoas}, Night the First
\citep[E303]{blake1797fourzoas}
\end{quote}

Before the Fall, the Zoas did not have separate spaces---they
coordinated seamlessly within a unified model. The Fall is precisely the
factorization into competing subsystems, each claiming territory.

Redemption is re-coordination: not the dominance of one Zoa but the
restoration of ``Universal Brotherhood''---a shared generative model
where each component contributes its proper inference without
tyrannizing the others.

\paragraph{Implications for Active
Inference}\label{implications-for-active-inference}

Blake's Four Zoas anticipate the insight that complex inference requires
factorization, but factorization introduces coordination problems. The
multi-agent mind must:

\begin{enumerate}
\def\labelenumi{\arabic{enumi}.}
\tightlist
\item
  \textbf{Maintain distinct components} --- Each Zoa has its proper
  function
\item
  \textbf{Coordinate across components} --- Shared priors enable unified
  behavior
\item
  \textbf{Prevent dominance} --- No single factor should monopolize
  inference
\item
  \textbf{Ground in embodiment} --- Tharmas anchors the system in
  biological reality
\end{enumerate}

The Zoas are not personality types but \emph{inference
modes}---different aspects of the generative model that must harmonize
for cleansed perception.

\paragraph{The Mean-Field
Approximation}\label{the-mean-field-approximation}

Blake's factorization prefigures the \textbf{mean-field approximation}
in variational inference. When exact inference is intractable, we
approximate the true posterior by assuming independence between factors:

\textbf{Mean-field factorization:}

\begin{equation}\label{eq:mean_field}
q(\theta) \approx q(\theta_U) \cdot q(\theta_L) \cdot q(\theta_{Lv}) \cdot q(\theta_T)
\end{equation}

This factorization enables tractable computation but introduces
\textbf{coordination costs}---precisely Blake's diagnosis that the Zoas'
separation produces suffering. The ``torments'' arise because mean-field
assumes independence where correlation should exist. Full variational
inference would preserve correlations; the factorized approximation
trades accuracy for tractability.

Blake's vision of ``Eternal Brotherhood'' corresponds to structured
variational families that preserve key correlations while remaining
tractable. The goal is not to eliminate factorization but to coordinate
the factors---each Zoa maintaining its function while communicating with
the others.

\begin{quote}
\emph{``And they conversed together in Visionary forms dramatic which
bright / Redounded from their Tongues in thunderous majesty, in Visions
/ In new Expanses''}

--- \emph{Jerusalem}, Plate 98 \citep{blake1804jerusalem}
\end{quote}

When the Zoas converse---when the model's components communicate---new
expanses open. This is the fourfold vision realized: not single-track
inference but multi-modal coordination, each perspective enriching the
others.

\begin{quote}
\textbf{Demonstration: The Stadium Wave}

Sixty thousand spectators rise and sit in sequence, producing a
traveling wave that circles the stadium in seconds. No one coordinates
the wave; no conductor signals the timing. Each individual infers from
their neighbors' actions when to rise---a local prediction, locally
tested, locally corrected. Yet the global pattern emerges: a coherent
wave far larger than any individual's perceptual horizon. This is
Blake's ``Jerusalem'' in miniature: a collective structure that
transcends individual agency while depending entirely upon it. The
shared generative model is not held in any single mind but distributed
across the blanket boundaries of thousands of coupled agents, each
minimizing their own surprise by predicting their neighbors and acting
accordingly. When the wave coheres, it feels like something
\emph{beyond} the individuals---an emergent collective agency that Blake
would recognize as the Eternal Man arising.
\end{quote}

\subsubsection{Jerusalem as Cultural
Niche}\label{jerusalem-as-cultural-niche}

Veissière and colleagues' framework of ``Thinking Through Other Minds''
\citep{veissiere2020thinking} illuminates a final dimension of Blake's
Jerusalem: the city is not merely a shared generative model but a
\emph{cultural niche}---an environment of shared priors, affordances,
and epistemic resources constructed and maintained through collective
inference. Culture, in this view, is not a static repository of
information passed down through generations but a living system of
shared expectations, jointly calibrated through what Active Inference
calls epistemic foraging and what Blake calls ``Mental Fight.'' The
laborers of Golgonooza are not merely building a model; they are
constructing the \emph{conditions} under which future inference can
occur---the affordance landscape that will shape subsequent generations'
priors. Jerusalem, once built, becomes the niche within which new minds
are enculturated, new Zoas coordinated, new visions made possible. The
social construction of reality is, in the deepest sense, the
collaborative construction of a shared generative model---and Blake's
prophetic vision of this process anticipates by two centuries the formal
framework that now makes it computationally precise.

\newpage

\section{Implications: The Wider Fields}\label{implications}

\emph{The doors open onto wider fields.}

\subsection{Philosophy of Mind}\label{philosophy-of-mind}

\subsubsection{The Romantic Computational
Mind}\label{the-romantic-computational-mind}

The foregoing synthesis raises fundamental questions for philosophy of
mind, cognitive science, and how we understand the relationship between
artistic and scientific knowledge. If Blake's phenomenological
observations and Friston's mathematical framework describe the same
cognitive architecture, then the Romantic tradition is not
anti-cognitive but proto-computational---articulating in the language of
vision what neuroscience would later formalize.

Blake's ``As a man is, so he sees'' is not merely prescient metaphor; it
is a precise statement of the Active Inference thesis that perception is
model-dependent inference. The convergence suggests that
phenomenological observation and formal modeling approach the same
cognitive reality from different directions, each illuminating what the
other cannot express. Against McGinn's ``mysterianism''---the claim that
consciousness constitutes an irresolvable problem for human cognition
\citep{mcginn2004consciousness}---the Blake--Active Inference synthesis
suggests a third way: neither reductive explanation nor mysterian
agnosticism, but phenomenological-formal complementarity, where
visionary description and mathematical formalism illuminate different
facets of the same cognitive architecture.

\subsubsection{Consciousness as Hierarchical
Depth}\label{consciousness-as-hierarchical-depth}

Blake's fourfold hierarchy (Figure \ref{fig:fourfold}) implies degrees
of awareness. Single vision is diminished consciousness. Fourfold vision
is full integration.

If cleansed perception = optimized free energy minimization (Equation
\ref{eq:cleansing}), then consciousness correlates with well-calibrated
generative models. The dimness of Newton's sleep is computational: poor
precision weighting (Equation \ref{eq:prior_dominance}) produces
impoverished inference, collapsing the prediction error signal (Equation
\ref{eq:prediction_error}). Conversely, expanded consciousness
corresponds to deeper hierarchical models that compress more temporal
structure (Equation \ref{eq:temporal_hierarchy}) and richer spatial
detail (Equation \ref{eq:model_complexity}) into unified awareness.

\subsection{Cognitive Science}\label{cognitive-science}

\subsubsection{Predictions}\label{predictions}

Three empirical implications follow from the Blake--Active Inference
correspondence:

\begin{enumerate}
\def\labelenumi{\arabic{enumi}.}
\item
  \textbf{Expert perception} --- If fourfold vision reflects
  hierarchical depth, then artists and naturalists should exhibit richer
  hierarchical representations than novices. Studies of perceptual
  expertise in visual art \citep{chamberlain2013perceptual} already
  document enhanced configural processing in trained observers,
  consistent with deeper generative models.
\item
  \textbf{Precision modulation} --- Contemplative practices that adjust
  precision weighting should alter perceptual content in predictable
  ways. Meditation traditions emphasizing open monitoring (reduced prior
  precision) versus focused attention (increased sensory precision)
  provide natural experimental conditions for testing Blake's claim that
  perception varies with the ``Organs of Perception.''
\item
  \textbf{Psychedelic states} --- Substances that alter precision
  constraints should produce Blake-like perception of infinite detail.
  The ALBUS framework \citep{safron2025albus}, extending the earlier
  REBUS model \citep{carthartharris2019rebus}, formalizes this
  prediction: psychedelics can both relax and strengthen beliefs,
  reshaping the balance between prior expectations and sensory
  evidence---precisely the ``cleansing'' Blake described.
\end{enumerate}

\subsubsection{Neural Correlates}\label{neural-correlates}

The contrast between ``guinea sun'' and ``Heavenly Host'' implies
differentiated processing. Neuroimaging could investigate whether
aesthetic transport exhibits relaxed prior precision and enhanced
sensory processing.

\subsection{Creativity}\label{creativity}

\subsubsection{The Artist as
Model-Builder}\label{the-artist-as-model-builder}

If imagination = generative model, then creativity = model construction.

Blake's illuminated books---integrating poetry, image, and
print---instantiate this claim. Each work offers a generative model to
the viewer, restructuring their perception.

\subsubsection{Aesthetic Free Energy}\label{aesthetic-free-energy}

Great art offers models that resolve more free energy (Equation
\ref{eq:free_energy}) than ordinary perception---making more sense of
more experience.

\begin{quote}
\emph{``To the eyes of the man of imagination, nature is imagination
itself.''}

--- Blake \citep{blake1810judgment}
\end{quote}

The developed perceiver experiences nature as already structured. The
generative model meets itself in the world.

\subsection{Transpersonal Experience}\label{transpersonal-experience}

\subsubsection{Mystical Perception}\label{mystical-perception}

Blake's visions---``the Innumerable company of the Heavenly
Host''---admit computational interpretation: extreme precision
relaxation allowing radical belief update.

Mysticism on this account is not separate reality but \emph{profound
model revision}. The doors swing so wide that habitual priors dissolve.

\subsubsection{Building Jerusalem}\label{jerusalem}

Blake's vision of Jerusalem as collective awakening maps directly onto
the multi-agent framework: Jerusalem = shared generative model (Equation
\ref{eq:multi_agent}) = collective prior enabling coordinated
perception. The Mental Fight---the tireless labor of
model-building---operates at civilizational scale, reshaping shared
priors through education, art, contemplative practice, and cultural
production.

\subsection{Counter-Arguments}\label{counter-arguments}

Four objections deserve explicit engagement:

\textbf{The Overfitting Objection.} Any sufficiently general
mathematical framework can be mapped onto any sufficiently general
philosophy; the Blake--Active Inference correspondence may reflect the
breadth of both systems rather than genuine structural alignment. We
concede that generality increases the risk of spurious correspondence.
However, the specificity of our mappings---not merely ``Blake values
perception'' but ``Blake's fourfold hierarchy structurally mirrors the
factorization of hierarchical generative models''---resists this charge.
The correspondences are not one-to-one between vague themes but between
precise structural features: boundary topology, precision dynamics,
temporal depth, multi-agent factorization.

\textbf{The Anachronism Objection.} Blake intended no Active Inference
meanings; reading them into his work is historical projection. This
objection applies to all retrospective intellectual history. We do not
claim that Blake \emph{intended} to describe Markov blankets. We claim
that the phenomenological structures he observed and articulated in his
prophetic poetry exhibit formal properties that Active Inference
independently identifies. Convergent description does not require shared
intention---Darwin and Wallace converged on natural selection without
coordination.

\textbf{The Formalization Objection.} Poetry resists equations;
translating Blake's visionary language into mathematical notation
inevitably loses what makes it meaningful. We agree that the translation
is lossy. The formal apparatus captures structural relations---topology,
dynamics, factorization---but not the affective, aesthetic, and
spiritual dimensions of Blake's work. The equations are not replacements
for the poetry but supplements: making explicit what the poetry implies
about the architecture of perception. The two languages illuminate
different aspects of the same cognitive reality.

\textbf{The Selectivity Objection.} The paper cherry-picks favorable
passages while ignoring Blake's many statements that resist
computational interpretation---his antinomianism, his mythological
personifications, his explicit hostility to ``Newton's Particles of
light.'' We acknowledge selection. Our eight themes represent the
strongest structural correspondences, not the totality of Blake's
thought. Blake's anti-Newtonian polemic, far from undermining our
thesis, \emph{supports} it: his critique targets precisely the ``single
vision'' (shallow, prior-locked inference) that Active Inference
identifies as pathological. The correspondence holds not despite Blake's
hostility to mechanism but \emph{because} of it.

\textbf{The Presentism Objection.} The most subtle risk is that we are
committing ``presentism''---reading modern scientific concepts back into
a historical figure whose intellectual context was radically different.
Blake's sources were Swedenborg, Boehme, and the Book of Ezekiel, not
Helmholtz or Bayes. We take this objection seriously. Our claim is not
causal (that Blake influenced neuroscience) but \emph{structural} (that
his phenomenological observations and the formal framework converge on
the same cognitive architecture). The defense against presentism is
specificity: vague analogies between ``Romanticism'' and ``creativity''
would indeed be presentist, but precise mappings between Blake's
fourfold hierarchy and the factorization of hierarchical generative
models resist this charge precisely because they are falsifiable. If the
structural correspondences broke down under scrutiny---if Blake's
categories did not map onto computationally distinct operations---the
project would fail. That they hold is evidence of convergent insight,
not anachronistic projection.

\subsection{Limitations}\label{limitations}

\textbf{Scope.} This paper treats Blake's perceptual philosophy through
the lens of a single formal framework. Other formalisms---enactivism,
dynamical systems theory, integrated information theory---might
illuminate different aspects of Blake's vision. The Active Inference
lens is not exhaustive.

\textbf{Empirical standing.} The predictions generated in §5.2 remain
untested. While the ALBUS framework (extending the earlier REBUS model)
provides some indirect support for the psychedelic prediction, and
expertise studies align with the hierarchical depth prediction, no
experiment has been designed to test the Blake--Active Inference
correspondence directly. Future work should operationalize specific
predictions---for example, measuring hierarchical model depth in
experienced contemplatives versus novices using computational
phenotyping.

\textbf{Translation fidelity.} The Erdman edition provides our textual
authority, but Blake's composite art---where image, text, and color form
a unified expression---resists reduction to quotation. Our analysis
necessarily privileges the verbal component of works that Blake designed
as visual-verbal wholes. The illuminated books demand a richer formalism
than equations alone can provide.

\textbf{Historical context.} Blake's religious commitments---his
unorthodox Christianity, his engagement with Swedenborg and Boehme, his
prophetic self-understanding---provide the matrix within which his
perceptual philosophy developed. Abstracting his insights into secular
computational language risks stripping away the very context that gave
them meaning. We proceed with awareness that translation always
transforms.

\subsection{Contemporary Applications}\label{contemporary-applications}

The Blake-Active Inference synthesis is not merely historical---it
illuminates contemporary challenges at the intersection of artificial
intelligence, mental health, and embodied robotics. In each domain,
Blake's phenomenological vocabulary provides intuitive access to formal
mechanisms that would otherwise remain opaque.

\subsubsection{AI Consciousness and Machine
Imagination}\label{ai-consciousness-and-machine-imagination}

Blake's claim that ``Imagination is Human Existence Itself'' speaks
directly to contemporary debates about machine consciousness. If the
self \emph{is} the generative model, what of artificial generative
models? Blake's framework suggests consciousness requires not merely
prediction but \emph{creative} model-building---the capacity to ``Create
a System'' rather than merely optimize within one. Furthermore, his Four
Zoas (Equation \ref{eq:factorized_model}) suggest consciousness requires
embodiment (Tharmas) and affective grounding (Luvah)---dimensions absent
from current AI systems.

\subsubsection{Mental Health
Interventions}\label{mental-health-interventions}

The synthesis illuminates mechanisms underlying several therapeutic
modalities:

\begin{itemize}
\tightlist
\item
  \textbf{Psychedelic therapy}: The ALBUS framework formalizes Blake's
  ``door cleansing'' as precision modulation---encompassing both the
  relaxation of rigid beliefs and the strengthening of therapeutic
  insights---explaining therapeutic effects through prior restructuring.
\item
  \textbf{Cognitive-behavioral therapy}: CBT operates through prior
  revision---identifying and restructuring the ``mind-forg'd manacles''
  of automatic thoughts.
\item
  \textbf{Contemplative practice}: Meditation traditions cultivate what
  Blake called ``flexible senses''---the capacity to modulate precision
  dynamically rather than remaining locked in prior-dominated inference.
\item
  \textbf{Narrative therapy}: Blake's distinction between ``States'' and
  ``Identities'' anticipates the therapeutic move from fixed trait
  identification to fluid state recognition.
\end{itemize}

\subsubsection{Embodied AI and Multi-Agent
Coordination}\label{embodied-ai-and-multi-agent-coordination}

Active Inference robotics embodies Blake's perception-action
unity---artificial agents that perceive \emph{through} acting, not
merely before acting. Blake's ``Energy is Eternal Delight'' provides a
phenomenological gloss on the utility-free motivation of free energy
minimization: agents seek not pleasure but self-evidence.

Multi-agent robotic coordination---swarm robotics, collaborative
manipulation---instantiates ``Building Jerusalem'' at the material
level: shared generative models enabling collective action without
centralized control. Blake's vision of ``Universal Brotherhood'' among
the Zoas maps onto the coordination problem in multi-agent systems: how
can diverse inference modes harmonize without homogenization?

\subsubsection{Emerging Field
Convergence}\label{emerging-field-convergence}

This synthesis sits at the intersection of two rapidly converging
fields: predictive processing approaches to aesthetics
\citep{vandecruys2024order} and cognitive approaches to Romanticism
\citep{savarese2020romanticism}. The convergence is not
coincidental---both fields independently arrived at the same fundamental
question: how does the brain's predictive architecture shape creative
experience? The Phil Trans B 2024 theme issue demonstrates that the
neuroscience community now takes aesthetic prediction error seriously as
a research program. Simultaneously, literary scholars increasingly
recognize that the Romantic poets were sophisticated theorists of mind,
not naïve pre-scientific mystics. Our contribution bridges these fields
by providing what neither has yet achieved: a formal, equation-level
mapping between a specific historical poet's cognitive phenomenology and
the mathematical apparatus of contemporary computational neuroscience.

\newpage

\section{Conclusion: The Threshold}\label{conclusion}

\subsection{The Synthesis}\label{the-synthesis}

Eight correspondences (see \hyperref[tbl-summary]{Table 1}):

\begin{longtable}[]{@{}
  >{\raggedright\arraybackslash}p{(\linewidth - 4\tabcolsep) * \real{0.2692}}
  >{\raggedright\arraybackslash}p{(\linewidth - 4\tabcolsep) * \real{0.3462}}
  >{\raggedright\arraybackslash}p{(\linewidth - 4\tabcolsep) * \real{0.3846}}@{}}
\caption{Eight core correspondences
summarized.}\label{tbl-summary}\tabularnewline
\toprule\noalign{}
\begin{minipage}[b]{\linewidth}\raggedright
Blake
\end{minipage} & \begin{minipage}[b]{\linewidth}\raggedright
Friston
\end{minipage} & \begin{minipage}[b]{\linewidth}\raggedright
Identity
\end{minipage} \\
\midrule\noalign{}
\endfirsthead
\toprule\noalign{}
\begin{minipage}[b]{\linewidth}\raggedright
Blake
\end{minipage} & \begin{minipage}[b]{\linewidth}\raggedright
Friston
\end{minipage} & \begin{minipage}[b]{\linewidth}\raggedright
Identity
\end{minipage} \\
\midrule\noalign{}
\endhead
\bottomrule\noalign{}
\endlastfoot
Doors of Perception & Markov Blanket & Interface topology \\
Fourfold Vision & Hierarchical Model & Processing depth \\
Newton's Sleep & Prior Dominance & Cognitive rigidity \\
Imagination & Generative Model & Agent identity \\
Eternity in an Hour & Temporal Horizons & Prediction depth \\
World in a Grain of Sand & Spatial Hierarchy & Evidence integration \\
Cleansing & Free Energy Minimization & Optimization \\
Building Jerusalem / Four Zoas & Shared \& Factorized Models &
Multi-agent coordination \\
\end{longtable}

Blake discovered through phenomenological observation what Active
Inference formalizes mathematically---two centuries before the equations
existed.

\subsection{The Threshold}\label{the-threshold}

Our title captures the synthesis:

\textbf{The Doors of Perception are the Threshold of Prediction.}

For both Blake and Friston, perception occurs at a boundary---door or
blanket (Equation \ref{eq:conditional_independence}; Figure
\ref{fig:doors})---separating self from world. The boundary is not
passive but \emph{predictive}: anticipating, shaping, generating
experience.

\begin{quote}
\emph{``If the doors of perception were cleansed every thing would
appear to man as it is: infinite.''}

--- Blake \citep{blake1790marriage}
\end{quote}

In Active Inference: if the generative model were optimally calibrated,
prediction error (Equation \ref{eq:prediction_error}) would minimize
across all hierarchical levels (Equation \ref{eq:hierarchical_model};
Figure \ref{fig:fourfold}). The ``Infinite'' is not mystical beyond but
\emph{vast information content} that rigid priors and shallow
hierarchies fail to access.

\subsection{Newton Still Sleeps}\label{newton-still-sleeps}

The critique remains devastatingly relevant. The danger is not reason
but \emph{absolutized} reason---single vision elevated to only vision.

Active Inference provides tools for understanding this danger
computationally:

\begin{itemize}
\tightlist
\item
  Excessive prior precision
\item
  Rigid model architecture\\
\item
  Shallow hierarchical processing
\end{itemize}

All produce impoverished perception. The remedy is not irrationalism but
\emph{expanded rationality}: richer models, flexible precision, deeper
hierarchies.

Contemporary manifestations of Newton's sleep abound. Algorithmic
attention capture---social media feeds optimized for
engagement---represents industrial-scale prior dominance: platforms that
systematically increase the precision of a narrow set of priors
(outrage, novelty, social comparison) while suppressing the broader,
slower processing that fourfold vision requires. The result is a
civilization of ``narrow chinks''---millions of caverns, algorithmically
reinforced. Similarly, the challenges of AI alignment can be understood
as the problem of building generative models that lack Tharmas and
Luvah: systems that reason (Urizen) and create (Los) but cannot feel
(Luvah) or be embodied (Tharmas). Blake's mythology suggests that such
factorized models, missing their affective and interoceptive components,
will inevitably produce ``monsters''---logically coherent but
existentially catastrophic outcomes.

\subsection{The Reciprocal Gift}\label{the-reciprocal-gift}

The synthesis is not one-directional. Blake gives Active Inference what
formalism alone cannot provide: a phenomenological vocabulary for the
felt experience of inference, a taxonomy of failure modes grounded in
lived perception (``Newton's sleep,'' ``Ulro,'' ``single vision''), and
the insistence that mathematical description is not exhaustive
description. Active Inference gives Blake what prophetic vision alone
cannot achieve: mathematical precision, empirical testability, and a
bridge to contemporary neuroscience that demonstrates these are not
archaic metaphors but accurate structural descriptions of cognitive
architecture. Neither tradition is complete without the other. The
equations need the visions; the visions need the equations.

\subsection{Building Jerusalem}\label{building-jerusalem}

Blake envisioned collective awakening---``Jerusalem'' as shared
visionary capacity.

In Active Inference: cultural generative models (Equation
\ref{eq:multi_agent}) enabling richer collective inference. Education,
art, contemplative practice, cultural production---all shape the models
through which communities perceive.

\begin{quote}
\emph{``I will not cease from Mental Fight,}\\
\emph{Nor shall my Sword sleep in my hand:}\\
\emph{Till we have built Jerusalem,}\\
\emph{In England's green \& pleasant Land.''}

--- \emph{Milton}, Preface \citep{blake1804milton}
\end{quote}

The Mental Fight is model-building at civilizational scale. Shared
priors enable coordinated perception. The awakening is collective.

\subsection{Future Directions}\label{future-directions}

Three research programs emerge from this synthesis:

\begin{enumerate}
\def\labelenumi{\arabic{enumi}.}
\item
  \textbf{Computational modeling of fourfold vision.} Using tools such
  as \texttt{pymdp} (the standard Python implementation of Active
  Inference \citep{heins2022pymdp}), one could construct hierarchical
  generative models of varying depth and test whether the
  phenomenological differences Blake describes between single, twofold,
  threefold, and fourfold vision correspond to quantifiable differences
  in model evidence, prediction error profile, and temporal horizon.
\item
  \textbf{Cross-cultural precision modulation.} If Blake's ``cleansing''
  maps to precision rebalancing, then contemplative practices across
  traditions---Zen kōan study, Sufi \emph{dhikr}, Buddhist
  \emph{vipassanā}, psychedelic-assisted therapy---may achieve analogous
  effects through culturally specific means. Comparative studies using
  the Active Inference framework could identify shared computational
  mechanisms beneath surface diversity.
\item
  \textbf{Neuroaesthetic experiments.} Blake's illuminated plates could
  be used as stimuli in fMRI and EEG studies designed to test whether
  viewing visionary art modulates precision weighting in ways consistent
  with the ALBUS framework---specifically, whether exposure to Blake's
  composite imagery alters the balance between high-level priors and
  sensory precision, producing measurable shifts toward ``cleansed
  perception.''
\end{enumerate}

\subsection{Final Reflection}\label{final-reflection}

Revolutionary London around the turn of the 19th century. Computational
neuroscience around the turn of the 21st. Two centuries, an age apart,
one shared Golden Thread.

The human situation admits description from radically different
perspectives---poetic and mathematical, Romantic and computational,
prophetic and scientific.

Phenomenological observation and formal modeling are not antagonists but
\emph{partners}. Blake's visions, and scientific equations, are
different doors opening onto the same threshold---the boundary at which
prediction meets reality. In the spirit of the Glass Bead Game, we have
played these two great systems with one another not to declare a winner,
but to reveal the hidden harmony of their structures.

\begin{quote}
\emph{``Without contraries is no progression.''}

--- Blake \citep{blake1790marriage}
\end{quote}



\bibliographystyle{unsrt}
\bibliography{references}
\end{document}
