% Options for packages loaded elsewhere
\PassOptionsToPackage{unicode}{hyperref}
\PassOptionsToPackage{hyphens}{url}
\documentclass[
  ignorenonframetext,
]{beamer}
\newif\ifbibliography
\usepackage{pgfpages}
\setbeamertemplate{caption}[numbered]
\setbeamertemplate{caption label separator}{: }
\setbeamercolor{caption name}{fg=normal text.fg}
\beamertemplatenavigationsymbolsempty
% remove section numbering
\setbeamertemplate{part page}{
  \centering
  \begin{beamercolorbox}[sep=16pt,center]{part title}
    \usebeamerfont{part title}\insertpart\par
  \end{beamercolorbox}
}
\setbeamertemplate{section page}{
  \centering
  \begin{beamercolorbox}[sep=12pt,center]{section title}
    \usebeamerfont{section title}\insertsection\par
  \end{beamercolorbox}
}
\setbeamertemplate{subsection page}{
  \centering
  \begin{beamercolorbox}[sep=8pt,center]{subsection title}
    \usebeamerfont{subsection title}\insertsubsection\par
  \end{beamercolorbox}
}
% Prevent slide breaks in the middle of a paragraph
\widowpenalties 1 10000
\raggedbottom
\AtBeginPart{
  \frame{\partpage}
}
\AtBeginSection{
  \ifbibliography
  \else
    \frame{\sectionpage}
  \fi
}
\AtBeginSubsection{
  \frame{\subsectionpage}
}
\usepackage{iftex}
\ifPDFTeX
  \usepackage[T1]{fontenc}
  \usepackage[utf8]{inputenc}
  \usepackage{textcomp} % provide euro and other symbols
\else % if luatex or xetex
  \usepackage{unicode-math} % this also loads fontspec
  \defaultfontfeatures{Scale=MatchLowercase}
  \defaultfontfeatures[\rmfamily]{Ligatures=TeX,Scale=1}
\fi
\usepackage{lmodern}
\ifPDFTeX\else
  % xetex/luatex font selection
\fi
% Use upquote if available, for straight quotes in verbatim environments
\IfFileExists{upquote.sty}{\usepackage{upquote}}{}
\IfFileExists{microtype.sty}{% use microtype if available
  \usepackage[]{microtype}
  \UseMicrotypeSet[protrusion]{basicmath} % disable protrusion for tt fonts
}{}
\makeatletter
\@ifundefined{KOMAClassName}{% if non-KOMA class
  \IfFileExists{parskip.sty}{%
    \usepackage{parskip}
  }{% else
    \setlength{\parindent}{0pt}
    \setlength{\parskip}{6pt plus 2pt minus 1pt}}
}{% if KOMA class
  \KOMAoptions{parskip=half}}
\makeatother
\usepackage{longtable,booktabs,array}
\newcounter{none} % for unnumbered tables
\usepackage{calc} % for calculating minipage widths
\usepackage{caption}
% Make caption package work with longtable
\makeatletter
\def\fnum@table{\tablename~\thetable}
\makeatother
\usepackage{graphicx}
\makeatletter
\newsavebox\pandoc@box
\newcommand*\pandocbounded[1]{% scales image to fit in text height/width
  \sbox\pandoc@box{#1}%
  \Gscale@div\@tempa{\textheight}{\dimexpr\ht\pandoc@box+\dp\pandoc@box\relax}%
  \Gscale@div\@tempb{\linewidth}{\wd\pandoc@box}%
  \ifdim\@tempb\p@<\@tempa\p@\let\@tempa\@tempb\fi% select the smaller of both
  \ifdim\@tempa\p@<\p@\scalebox{\@tempa}{\usebox\pandoc@box}%
  \else\usebox{\pandoc@box}%
  \fi%
}
% Set default figure placement to htbp
\def\fps@figure{htbp}
\makeatother
\setlength{\emergencystretch}{3em} % prevent overfull lines
\providecommand{\tightlist}{%
  \setlength{\itemsep}{0pt}\setlength{\parskip}{0pt}}
\usepackage{bookmark}
\IfFileExists{xurl.sty}{\usepackage{xurl}}{} % add URL line breaks if available
\urlstyle{same}
\hypersetup{
  hidelinks,
  pdfcreator={LaTeX via pandoc}}

\author{\texorpdfstring{}{}}
\date{}

\begin{document}

\begin{frame}{Results}
\protect\phantomsection\label{results}
This section presents the experimental results from the gradient descent
optimization study, including convergence analysis and performance
comparisons.

\begin{block}{Convergence Analysis}
\protect\phantomsection\label{convergence-analysis}
Figure 1 shows the convergence behavior of gradient descent for
different step sizes, starting from the initial point \(x_0 = 0\).

\begin{figure}
\centering
\pandocbounded{\includegraphics[keepaspectratio,alt={Gradient Descent Convergence}]{figures/convergence_plot.png}}
\caption{Gradient Descent Convergence}\label{fig:convergence}
\end{figure}

The plot demonstrates several key observations:

\begin{enumerate}
\tightlist
\item
  \textbf{Step size impact}: Larger step sizes generally lead to faster
  initial progress but may exhibit oscillatory behavior
\item
  \textbf{Convergence rate}: All tested step sizes eventually converge
  to the analytical optimum at \(x = 1\)
\item
  \textbf{Stability}: Conservative step sizes (\(\alpha = 0.01\)) show
  smooth, monotonic convergence
\end{enumerate}
\end{block}

\begin{block}{Quantitative Results}
\protect\phantomsection\label{quantitative-results}
The optimization results for different step sizes are summarized in the
following table:

{\def\LTcaptype{none} % do not increment counter
\begin{longtable}[]{@{}
  >{\raggedright\arraybackslash}p{(\linewidth - 8\tabcolsep) * \real{0.2113}}
  >{\raggedright\arraybackslash}p{(\linewidth - 8\tabcolsep) * \real{0.2254}}
  >{\raggedright\arraybackslash}p{(\linewidth - 8\tabcolsep) * \real{0.2394}}
  >{\raggedright\arraybackslash}p{(\linewidth - 8\tabcolsep) * \real{0.1690}}
  >{\raggedright\arraybackslash}p{(\linewidth - 8\tabcolsep) * \real{0.1549}}@{}}
\toprule\noalign{}
\begin{minipage}[b]{\linewidth}\raggedright
Step Size (α)
\end{minipage} & \begin{minipage}[b]{\linewidth}\raggedright
Final Solution
\end{minipage} & \begin{minipage}[b]{\linewidth}\raggedright
Objective Value
\end{minipage} & \begin{minipage}[b]{\linewidth}\raggedright
Iterations
\end{minipage} & \begin{minipage}[b]{\linewidth}\raggedright
Converged
\end{minipage} \\
\midrule\noalign{}
\endhead
0.01 & 0.9999 & -0.5000 & 165 & Yes \\
0.05 & 1.0000 & -0.5000 & 34 & Yes \\
0.10 & 1.0000 & -0.5000 & 17 & Yes \\
0.20 & 1.0000 & -0.5000 & 9 & Yes \\
\bottomrule\noalign{}
\end{longtable}
}

\textbf{Table 1:} Optimization results showing solution accuracy and
convergence speed for different step sizes.
\end{block}

\begin{block}{Performance Analysis}
\protect\phantomsection\label{performance-analysis}
\begin{block}{Convergence Speed}
\protect\phantomsection\label{convergence-speed}
The results show a clear trade-off between step size and convergence
speed: - Small step sizes require more iterations but provide stable
convergence - Large step sizes converge faster but may be less stable in
more complex problems
\end{block}

\begin{block}{Solution Accuracy}
\protect\phantomsection\label{solution-accuracy}
All tested step sizes achieved the analytical optimum within numerical
precision: - Target solution: \(x = 1.0000\) - Target objective:
\(f(x) = -0.5000\)

This demonstrates the algorithm's ability to solve simple quadratic
optimization problems reliably.
\end{block}
\end{block}

\begin{block}{Algorithm Characteristics}
\protect\phantomsection\label{algorithm-characteristics}
\begin{block}{Strengths}
\protect\phantomsection\label{strengths}
\begin{itemize}
\tightlist
\item
  \textbf{Simplicity}: Easy to implement and understand
\item
  \textbf{Generality}: Applicable to any differentiable objective
  function
\item
  \textbf{Reliability}: Converges for convex functions under appropriate
  conditions
\end{itemize}
\end{block}

\begin{block}{Limitations}
\protect\phantomsection\label{limitations}
\begin{itemize}
\tightlist
\item
  \textbf{Step size sensitivity}: Performance depends critically on step
  size selection
\item
  \textbf{Local convergence}: May converge to local minima in non-convex
  problems
\item
  \textbf{Fixed step size}: No adaptation to problem characteristics
\end{itemize}
\end{block}
\end{block}

\begin{block}{Computational Performance}
\protect\phantomsection\label{computational-performance}
The algorithm demonstrates efficient performance for small-scale
problems: - Fast convergence (typically \textless{} 20 iterations for
this problem) - Minimal computational overhead per iteration -
Memory-efficient implementation suitable for high-dimensional problems
\end{block}

\begin{block}{Validation}
\protect\phantomsection\label{validation}
The implementation was validated through: - \textbf{Unit tests} covering
all core functionality - \textbf{Integration tests} verifying algorithm
convergence - \textbf{Numerical accuracy} checks against analytical
solutions - \textbf{Edge case handling} for boundary conditions

All tests pass with 100\% coverage, ensuring implementation correctness
and reliability.
\end{block}

\begin{block}{Discussion}
\protect\phantomsection\label{discussion}
The experimental results validate the gradient descent implementation
and provide insights into algorithm behavior under different parameter
settings. The automated analysis pipeline successfully generated both
visual and numerical outputs for manuscript integration.

Future work could extend this analysis to: - Non-convex optimization
problems - Adaptive step size strategies - Comparison with other
optimization algorithms - Large-scale problem applications
\end{block}
\end{frame}

\end{document}
