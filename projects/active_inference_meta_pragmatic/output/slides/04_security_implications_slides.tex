% Options for packages loaded elsewhere
\PassOptionsToPackage{unicode}{hyperref}
\PassOptionsToPackage{hyphens}{url}
\documentclass[
  ignorenonframetext,
]{beamer}
\newif\ifbibliography
\usepackage{pgfpages}
\setbeamertemplate{caption}[numbered]
\setbeamertemplate{caption label separator}{: }
\setbeamercolor{caption name}{fg=normal text.fg}
\beamertemplatenavigationsymbolsempty
% remove section numbering
\setbeamertemplate{part page}{
  \centering
  \begin{beamercolorbox}[sep=16pt,center]{part title}
    \usebeamerfont{part title}\insertpart\par
  \end{beamercolorbox}
}
\setbeamertemplate{section page}{
  \centering
  \begin{beamercolorbox}[sep=12pt,center]{section title}
    \usebeamerfont{section title}\insertsection\par
  \end{beamercolorbox}
}
\setbeamertemplate{subsection page}{
  \centering
  \begin{beamercolorbox}[sep=8pt,center]{subsection title}
    \usebeamerfont{subsection title}\insertsubsection\par
  \end{beamercolorbox}
}
% Prevent slide breaks in the middle of a paragraph
\widowpenalties 1 10000
\raggedbottom
\AtBeginPart{
  \frame{\partpage}
}
\AtBeginSection{
  \ifbibliography
  \else
    \frame{\sectionpage}
  \fi
}
\AtBeginSubsection{
  \frame{\subsectionpage}
}
\usepackage{iftex}
\ifPDFTeX
  \usepackage[T1]{fontenc}
  \usepackage[utf8]{inputenc}
  \usepackage{textcomp} % provide euro and other symbols
\else % if luatex or xetex
  \usepackage{unicode-math} % this also loads fontspec
  \defaultfontfeatures{Scale=MatchLowercase}
  \defaultfontfeatures[\rmfamily]{Ligatures=TeX,Scale=1}
\fi
\usepackage{lmodern}
\ifPDFTeX\else
  % xetex/luatex font selection
\fi
% Use upquote if available, for straight quotes in verbatim environments
\IfFileExists{upquote.sty}{\usepackage{upquote}}{}
\IfFileExists{microtype.sty}{% use microtype if available
  \usepackage[]{microtype}
  \UseMicrotypeSet[protrusion]{basicmath} % disable protrusion for tt fonts
}{}
\makeatletter
\@ifundefined{KOMAClassName}{% if non-KOMA class
  \IfFileExists{parskip.sty}{%
    \usepackage{parskip}
  }{% else
    \setlength{\parindent}{0pt}
    \setlength{\parskip}{6pt plus 2pt minus 1pt}}
}{% if KOMA class
  \KOMAoptions{parskip=half}}
\makeatother
\usepackage{longtable,booktabs,array}
\newcounter{none} % for unnumbered tables
\usepackage{calc} % for calculating minipage widths
\usepackage{caption}
% Make caption package work with longtable
\makeatletter
\def\fnum@table{\tablename~\thetable}
\makeatother
\setlength{\emergencystretch}{3em} % prevent overfull lines
\providecommand{\tightlist}{%
  \setlength{\itemsep}{0pt}\setlength{\parskip}{0pt}}
\usepackage{bookmark}
\IfFileExists{xurl.sty}{\usepackage{xurl}}{} % add URL line breaks if available
\urlstyle{same}
\hypersetup{
  hidelinks,
  pdfcreator={LaTeX via pandoc}}

\author{\texorpdfstring{}{}}
\date{}

\begin{document}

\begin{frame}{Security Implications}
\protect\phantomsection\label{sec:security}
The meta-level framework has significant implications for cognitive
security, AI safety, and the robustness of belief systems. Understanding
meta-cognitive processing reveals vulnerabilities that traditional
security models miss, while also suggesting principled defense
strategies.

\begin{block}{Cognitive Security Framework}
\protect\phantomsection\label{sec:cognitive_security_framework}
Active Inference's quadrant structure provides a systematic way to
analyze cognitive vulnerabilities. Each quadrant represents a potential
attack surface with distinct vulnerability profiles and defense
requirements.

\begin{block}{Attack Surface by Quadrant}
\protect\phantomsection\label{attack-surface-by-quadrant}
{\def\LTcaptype{none} % do not increment counter
\begin{longtable}[]{@{}
  >{\raggedright\arraybackslash}p{(\linewidth - 6\tabcolsep) * \real{0.2439}}
  >{\raggedright\arraybackslash}p{(\linewidth - 6\tabcolsep) * \real{0.1951}}
  >{\raggedright\arraybackslash}p{(\linewidth - 6\tabcolsep) * \real{0.3659}}
  >{\raggedright\arraybackslash}p{(\linewidth - 6\tabcolsep) * \real{0.1951}}@{}}
\toprule\noalign{}
\begin{minipage}[b]{\linewidth}\raggedright
Quadrant
\end{minipage} & \begin{minipage}[b]{\linewidth}\raggedright
Target
\end{minipage} & \begin{minipage}[b]{\linewidth}\raggedright
Vulnerability
\end{minipage} & \begin{minipage}[b]{\linewidth}\raggedright
Impact
\end{minipage} \\
\midrule\noalign{}
\endhead
Q1 & Sensory data & Observation manipulation & Belief distortion \\
Q2 & Meta-data & Quality score falsification & Confidence
miscalibration \\
Q3 & Self-monitoring & Confidence mechanism hijacking & Strategy
corruption \\
Q4 & Framework parameters & Epistemic/pragmatic subversion &
Architectural compromise \\
\bottomrule\noalign{}
\end{longtable}
}

Higher quadrants represent more fundamental vulnerabilities: while
Quadrant 1 attacks can distort specific beliefs, Quadrant 4 attacks can
compromise the entire cognitive architecture.
\end{block}
\end{block}

\begin{block}{Meta-Cognitive Vulnerabilities}
\protect\phantomsection\label{sec:vulnerabilities}
\begin{block}{Quadrant 3 Attacks: Confidence Manipulation}
\protect\phantomsection\label{quadrant-3-attacks-confidence-manipulation}
Manipulation of confidence assessment mechanisms can undermine
meta-cognitive control:

\textbf{False Confidence Calibration:} Adversaries provide feedback that
systematically miscalibrates confidence assessments, causing agents to
over-trust or under-trust their inferences.

\textbf{Induced Over/Under-Confidence:} By manipulating confidence
assessment inputs, attackers can cause agents to: - Become overly
conservative when exploration is needed - Become overconfident when
caution is warranted - Switch strategies inappropriately

\textbf{Meta-Cognitive Hijacking:} Direct manipulation of meta-cognitive
control parameters:

\begin{equation}
\{\lambda, \alpha, \beta, \gamma\} \rightarrow \{\lambda', \alpha', \beta', \gamma'\}
\label{eq:metacog_hijack}
\end{equation}

Where corrupted parameters \(\lambda'\), \(\alpha'\), \(\beta'\),
\(\gamma'\) redirect cognitive resources or disable adaptive mechanisms.
\end{block}

\begin{block}{Quadrant 4 Attacks: Framework Subversion}
\protect\phantomsection\label{quadrant-4-attacks-framework-subversion}
Framework-level manipulation targets the fundamental cognitive
architecture:

\textbf{Epistemic Framework Subversion:} Altering matrices \(A\), \(B\),
or \(D\) through learning or external influence can fundamentally change
what an agent believes is knowable:

\begin{equation}
A_{true} \rightarrow A_{corrupted}: \text{perception of reality distorted}
\label{eq:epistemic_subversion}
\end{equation}

\textbf{Pragmatic Landscape Alteration:} Modifying matrix \(C\) changes
what the agent values:

\begin{equation}
C_{original} \rightarrow C_{corrupted}: \text{goal structure compromised}
\label{eq:pragmatic_alteration}
\end{equation}

This potentially redirects all goal-directed behavior without the
agent's awareness.

\textbf{Higher-Order Reasoning Corruption:} Manipulating framework
optimization processes (Equation \eqref{eq:framework_optimization}) can
cause agents to evolve toward vulnerable or exploitable cognitive
architectures.
\end{block}

\begin{block}{Attack Vector Analysis}
\protect\phantomsection\label{attack-vector-analysis}
\textbf{Gradual vs.~Sudden Attacks:} - Gradual: Slow parameter drift
below detection threshold - Sudden: Rapid framework changes triggering
immediate adaptation

\textbf{External vs.~Internal:} - External: Environmental manipulation
of observations - Internal: Direct parameter injection through learning
mechanisms

\textbf{Targeted vs.~Systemic:} - Targeted: Specific quadrant or
parameter manipulation - Systemic: Cascading attacks affecting multiple
levels
\end{block}
\end{block}

\begin{block}{Defense Strategies}
\protect\phantomsection\label{sec:defense_strategies}
The framework suggests defense approaches operating at multiple levels,
with higher-level defenses providing more fundamental protection.

\begin{block}{Meta-Cognitive Monitoring (Quadrant 3 Defense)}
\protect\phantomsection\label{meta-cognitive-monitoring-quadrant-3-defense}
Continuous validation of confidence assessments:

\begin{equation}
validation(c) = |accuracy_{predicted}(c) - accuracy_{actual}|
\label{eq:confidence_validation}
\end{equation}

\textbf{Defense Mechanisms:} - Cross-validation of confidence with
actual performance - Detection of miscalibration patterns - Anomaly
detection for confidence trajectories - Automatic recalibration when
drift detected
\end{block}

\begin{block}{Framework Integrity Checks (Quadrant 4 Defense)}
\protect\phantomsection\label{framework-integrity-checks-quadrant-4-defense}
Verification of epistemic and pragmatic consistency:

\begin{equation}
integrity(\Theta) = \|\Theta_t - \Theta_{baseline}\| < \epsilon
\label{eq:framework_integrity}
\end{equation}

\textbf{Defense Mechanisms:} - Monitoring framework parameters for
unexpected changes - Detecting drift in matrices \(A\), \(B\), \(C\),
\(D\) - Regularization terms \(\mathcal{R}(\Theta)\) penalizing
inconsistent specifications - Framework coherence validation
\end{block}

\begin{block}{Recursive Validation (Multi-Level Defense)}
\protect\phantomsection\label{recursive-validation-multi-level-defense}
Higher-order checking of meta-level processes:

\textbf{Three-Layer Validation:} 1. \textbf{Level 1:} Validate primary
inference processes 2. \textbf{Level 2:} Validate meta-cognitive
monitoring itself 3. \textbf{Level 3:} Validate framework integrity
checking

This recursive structure ensures that each security layer is itself
protected by higher layers.
\end{block}

\begin{block}{Defense Portfolio}
\protect\phantomsection\label{defense-portfolio}
{\def\LTcaptype{none} % do not increment counter
\begin{longtable}[]{@{}lll@{}}
\toprule\noalign{}
Defense Layer & Mechanism & Protects Against \\
\midrule\noalign{}
\endhead
Observation validation & Signal integrity & Q1 attacks \\
Meta-data verification & Source authentication & Q2 attacks \\
Confidence monitoring & Calibration checking & Q3 attacks \\
Framework integrity & Parameter bounds & Q4 attacks \\
Recursive validation & Self-checking & Multi-level attacks \\
\bottomrule\noalign{}
\end{longtable}
}
\end{block}
\end{block}

\begin{block}{AI Safety and Value Alignment}
\protect\phantomsection\label{sec:ai_safety}
The framework provides principled approaches to AI safety challenges:

\begin{block}{Value Specification through Matrix C}
\protect\phantomsection\label{value-specification-through-matrix-c}
Active Inference enables precise value specification:

\begin{equation}
C_{safe} = \text{specification of safe preferences}
\label{eq:safe_preferences}
\end{equation}

\textbf{Advantages over reward functions:} - Multi-dimensional
preference landscapes - Trade-off specification between competing values
- Ethical considerations directly encoded - Value hierarchies with
priority structures
\end{block}

\begin{block}{Epistemic Boundary Protection}
\protect\phantomsection\label{epistemic-boundary-protection}
Clear limits on what AI systems can know and assume:

\textbf{Bounded Epistemic Frameworks:} - Matrix \(A\) specifications
limit observation reliability assumptions - Matrix \(D\) priors
constrain initial state assumptions - Matrix \(B\) causal models bound
action effect assumptions
\end{block}

\begin{block}{Framework Integrity for AI Systems}
\protect\phantomsection\label{framework-integrity-for-ai-systems}
Protection against value drift and epistemic corruption:

\textbf{Meta-Monitoring Requirements:} - Self-watchful AI systems
monitoring their own frameworks - Anomaly detection for framework
parameter changes - Rollback capabilities for detected corruption -
Human-in-the-loop for framework modifications
\end{block}

\begin{block}{Alignment through Framework Specification}
\protect\phantomsection\label{alignment-through-framework-specification}
The meta-pragmatic aspect enables principled alignment: 1. \textbf{Value
Learning:} Systems develop value structures through matrix \(C\)
optimization 2. \textbf{Epistemic Constraints:} Matrix \(A\), \(B\),
\(D\) specifications limit inference scope 3. \textbf{Meta-Cognitive
Oversight:} Quadrant 3 monitoring ensures alignment maintenance 4.
\textbf{Framework Stability:} Quadrant 4 regularization prevents
unauthorized evolution
\end{block}
\end{block}

\begin{block}{Societal Implications}
\protect\phantomsection\label{sec:societal_security}
\begin{block}{Information Warfare}
\protect\phantomsection\label{information-warfare}
The framework reveals meta-level manipulation of public belief systems:

\textbf{Epistemic Attacks on Societies:} - Systematic manipulation of
information quality (meta-data) - Undermining confidence in legitimate
information sources - Framework-level attacks on shared epistemological
foundations

\textbf{Defense Implications:} - Education in meta-cognitive awareness -
Institutional meta-data verification - Collective framework integrity
monitoring
\end{block}

\begin{block}{Educational System Resilience}
\protect\phantomsection\label{educational-system-resilience}
Development of curricula building meta-cognitive resilience:

\textbf{Training Quadrant 3 Skills:} - Self-monitoring and confidence
assessment - Strategy adaptation under uncertainty - Meta-cognitive
awareness

\textbf{Training Quadrant 4 Skills:} - Framework evaluation and critique
- Epistemic framework comparison - Value system analysis
\end{block}

\begin{block}{Collective Cognitive Security}
\protect\phantomsection\label{collective-cognitive-security}
Protection of group-level cognitive processes:

\textbf{Shared Framework Protection:} - Collective monitoring of
epistemic drift - Group-level confidence calibration - Democratic
framework governance

\textbf{Institutional Safeguards:} - Verification of information sources
- Meta-data authenticity standards - Framework change transparency
\end{block}
\end{block}

\begin{block}{Ethical Considerations}
\protect\phantomsection\label{sec:security_ethics}
\begin{block}{Manipulation Risks}
\protect\phantomsection\label{manipulation-risks}
Meta-level cognition raises concerns about: - Potential for
sophisticated cognitive manipulation - Exploitation of framework
vulnerabilities - Asymmetric knowledge advantages
\end{block}

\begin{block}{Responsibility in Framework Design}
\protect\phantomsection\label{responsibility-in-framework-design}
Designers of cognitive systems bear responsibility for: - Secure
framework specifications - Robust defense mechanisms - Transparent
vulnerability disclosure
\end{block}

\begin{block}{Self-Determination}
\protect\phantomsection\label{self-determination}
Protection of individual and collective: - Epistemic autonomy: freedom
to form beliefs - Pragmatic autonomy: freedom to set values -
Meta-cognitive autonomy: freedom to adapt frameworks
\end{block}
\end{block}
\end{frame}

\end{document}
