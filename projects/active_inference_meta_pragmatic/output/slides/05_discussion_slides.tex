% Options for packages loaded elsewhere
\PassOptionsToPackage{unicode}{hyperref}
\PassOptionsToPackage{hyphens}{url}
\documentclass[
  ignorenonframetext,
]{beamer}
\newif\ifbibliography
\usepackage{pgfpages}
\setbeamertemplate{caption}[numbered]
\setbeamertemplate{caption label separator}{: }
\setbeamercolor{caption name}{fg=normal text.fg}
\beamertemplatenavigationsymbolsempty
% remove section numbering
\setbeamertemplate{part page}{
  \centering
  \begin{beamercolorbox}[sep=16pt,center]{part title}
    \usebeamerfont{part title}\insertpart\par
  \end{beamercolorbox}
}
\setbeamertemplate{section page}{
  \centering
  \begin{beamercolorbox}[sep=12pt,center]{section title}
    \usebeamerfont{section title}\insertsection\par
  \end{beamercolorbox}
}
\setbeamertemplate{subsection page}{
  \centering
  \begin{beamercolorbox}[sep=8pt,center]{subsection title}
    \usebeamerfont{subsection title}\insertsubsection\par
  \end{beamercolorbox}
}
% Prevent slide breaks in the middle of a paragraph
\widowpenalties 1 10000
\raggedbottom
\AtBeginPart{
  \frame{\partpage}
}
\AtBeginSection{
  \ifbibliography
  \else
    \frame{\sectionpage}
  \fi
}
\AtBeginSubsection{
  \frame{\subsectionpage}
}
\usepackage{iftex}
\ifPDFTeX
  \usepackage[T1]{fontenc}
  \usepackage[utf8]{inputenc}
  \usepackage{textcomp} % provide euro and other symbols
\else % if luatex or xetex
  \usepackage{unicode-math} % this also loads fontspec
  \defaultfontfeatures{Scale=MatchLowercase}
  \defaultfontfeatures[\rmfamily]{Ligatures=TeX,Scale=1}
\fi
\usepackage{lmodern}
\ifPDFTeX\else
  % xetex/luatex font selection
\fi
% Use upquote if available, for straight quotes in verbatim environments
\IfFileExists{upquote.sty}{\usepackage{upquote}}{}
\IfFileExists{microtype.sty}{% use microtype if available
  \usepackage[]{microtype}
  \UseMicrotypeSet[protrusion]{basicmath} % disable protrusion for tt fonts
}{}
\makeatletter
\@ifundefined{KOMAClassName}{% if non-KOMA class
  \IfFileExists{parskip.sty}{%
    \usepackage{parskip}
  }{% else
    \setlength{\parindent}{0pt}
    \setlength{\parskip}{6pt plus 2pt minus 1pt}}
}{% if KOMA class
  \KOMAoptions{parskip=half}}
\makeatother
\setlength{\emergencystretch}{3em} % prevent overfull lines
\providecommand{\tightlist}{%
  \setlength{\itemsep}{0pt}\setlength{\parskip}{0pt}}
\usepackage{bookmark}
\IfFileExists{xurl.sty}{\usepackage{xurl}}{} % add URL line breaks if available
\urlstyle{same}
\hypersetup{
  hidelinks,
  pdfcreator={LaTeX via pandoc}}

\author{\texorpdfstring{}{}}
\date{}

\begin{document}

\begin{frame}{Discussion}
\protect\phantomsection\label{sec:discussion}
The (2 \times 2) matrix structure reveals Active Inference as a
fundamentally meta-level methodology with profound implications for
cognitive science, artificial intelligence, and our understanding of
intelligence itself. This section explores the theoretical implications
of viewing Active Inference through the lens of meta-pragmatic and
meta-epistemic operation, examining how specification power enables new
forms of cognitive analysis, design, and understanding that transcend
traditional approaches to cognition. By allowing researchers to specify
epistemic and pragmatic frameworks rather than working within fixed
structures, Active Inference creates a meta-methodology that makes
framework design itself a research question.

\begin{block}{Meta-Pragmatic Implications}
\protect\phantomsection\label{sec:meta_pragmatic_implications}
Active Inference's meta-pragmatic nature transcends traditional
approaches to goal-directed behavior by allowing modelers to specify
pragmatic frameworks (through matrix (C)) rather than simple reward
functions. This specification power supports researchers in exploring
how different value systems shape cognition and behavior, making value
system design itself a research question rather than an external
constraint. The transformation from fixed rewards to specifiable
preference landscapes opens new possibilities for understanding
value-driven cognition.

\begin{block}{Beyond Reward Functions}
\protect\phantomsection\label{beyond-reward-functions}
Traditional reinforcement learning specifies rewards as scalar values:

\[R(s,a) \in \mathbb{R}\label{eq:traditional_reward}\]

Active Inference, however, enables specification of preference
landscapes through matrix (C) (see Equation \eqref{eq:matrix_c}):

\[C(o) \in \mathbb{R}^{|\mathcal{O}|}\label{eq:active_inference_preferences}\]

This supports modeling of value systems far richer than scalar rewards,
enabling formal analysis of complex value structures:

\begin{itemize}
\tightlist
\item
  \textbf{Complex Value Structures:} Multi-dimensional preferences with
  trade-offs, where agents balance competing goals (e.g., efficiency
  vs.~safety, individual vs.~collective benefit)
\item
  \textbf{Ethical Considerations:} Incorporation of moral and social
  values directly into the preference landscape, enabling agents to
  reason about ethical implications of actions
\item
  \textbf{Contextual Goals:} Situation-dependent value hierarchies,
  where what matters changes based on context (e.g., survival values in
  danger, exploration values in safety)
\item
  \textbf{Meta-Preferences:} Preferences about preference structures
  themselves, enabling agents to value having certain types of values
  (e.g., valuing being the kind of agent that values fairness)
\end{itemize}

The matrix (C) specification thus becomes a design space for exploring
different value systems and their cognitive consequences, rather than a
fixed reward function.
\end{block}

\begin{block}{Pragmatic Framework Design}
\protect\phantomsection\label{pragmatic-framework-design}
The meta-pragmatic power supports researchers in exploring: - How
different societies develop different value systems - How individual
development shapes personal pragmatic frameworks - How cultural
evolution influences collective goal structures - How artificial agents
might develop their own pragmatic frameworks
\end{block}
\end{block}

\begin{block}{Meta-Epistemic Implications}
\protect\phantomsection\label{sec:meta_epistemic_implications}
Active Inference supports specification of epistemic frameworks through
matrices (A) (Equation \eqref{eq:matrix_a}), (B) (Equation
\eqref{eq:matrix_b}), and (D) (Equation \eqref{eq:matrix_d}), allowing
modelers to define not just what agents believe, but how they come to
know the world---determining what knowledge is possible, how evidence
accumulates, and what causal structures are assumed. This meta-epistemic
power makes epistemological framework design a research question,
supporting systematic exploration of how different ways of knowing lead
to different cognitive outcomes. The transformation from fixed
observation models to specifiable epistemic structures opens new
possibilities for understanding knowledge-driven cognition.

\begin{block}{Epistemological Pluralism}
\protect\phantomsection\label{epistemological-pluralism}
Different epistemic frameworks can be specified through generative model
parameters:

\textbf{Empirical Framework:}

\[A_{\text{empirical}} = \begin{pmatrix} 0.95 & 0.05 \\ 0.05 & 0.95 \end{pmatrix}\label{eq:empirical_framework}\]

High confidence in sensory observations (diagonal entries near 1.0), low
uncertainty (off-diagonal entries near 0). This framework assumes
observations are highly reliable indicators of underlying states,
appropriate for well-calibrated sensors in controlled environments.
Agents with this framework trust their observations and make rapid state
inferences.

\textbf{Skeptical Framework:}

\[A_{\text{skeptical}} = \begin{pmatrix} 0.6 & 0.4 \\ 0.4 & 0.6 \end{pmatrix}\label{eq:skeptical_framework}\]

Lower confidence (diagonal entries 0.6), higher epistemic caution
(off-diagonal entries 0.4). This framework maintains greater
uncertainty, requiring more evidence before committing to state beliefs.
Appropriate for noisy environments or when observation reliability is
questionable. Agents with this framework are more cautious, gathering
more information before acting.

\textbf{Dogmatic Framework:}

\[A_{\text{dogmatic}} = \begin{pmatrix} 1.0 & 0.0 \\ 0.0 & 1.0 \end{pmatrix}\label{eq:dogmatic_framework}\]

Absolute certainty (perfect diagonal), no epistemic doubt (zero
off-diagonal). This framework represents perfect observation-state
mapping with no uncertainty. While rarely realistic in practice, it
illustrates the extreme case where observations are assumed to perfectly
reveal states, leading to immediate, unshakeable beliefs. Such
frameworks are vulnerable to systematic errors or deception.
\end{block}

\begin{block}{Knowledge Architecture Design}
\protect\phantomsection\label{knowledge-architecture-design}
Active Inference enables design of knowledge acquisition systems by
specifying how beliefs form, update, and interact:

\begin{itemize}
\item
  \textbf{Learning Mechanisms:} How beliefs update over time through
  matrix (B) (Equation \eqref{eq:matrix_b}) specifications, defining how
  actions and observations change state beliefs. Different learning
  mechanisms (rapid adaptation vs.~conservative updating) can be modeled
  by varying transition dynamics.
\item
  \textbf{Uncertainty Handling:} Approaches to ambiguous information
  through matrix (A) (Equation \eqref{eq:matrix_a}) specifications,
  defining how observation uncertainty propagates to belief uncertainty.
  Different uncertainty handling strategies (risk-averse
  vs.~risk-seeking) emerge from different observation model
  specifications.
\item
  \textbf{Evidence Integration:} How multiple sources combine through
  generative model structure, enabling modeling of multi-modal
  perception, conflicting evidence resolution, and source reliability
  weighting. The framework naturally handles situations where different
  observation modalities provide complementary or contradictory
  information.
\item
  \textbf{Hypothesis Testing:} Frameworks for belief validation through
  prior specification (matrix (D), Equation \eqref{eq:matrix_d}) and
  observation models (matrix (A), Equation \eqref{eq:matrix_a}),
  enabling modeling of how agents test hypotheses, accumulate evidence,
  and update confidence in beliefs. Different hypothesis testing
  strategies emerge from different framework specifications.
\end{itemize}
\end{block}
\end{block}

\begin{block}{The Modeler's Dual Role}
\protect\phantomsection\label{sec:modeler_dual_role}
The framework reveals the recursive relationship between modeler and
modeled system.

\begin{block}{As Architect}
\protect\phantomsection\label{as-architect}
The modeler specifies the boundaries of cognition: - \textbf{Epistemic
Boundaries:} What can be known (matrix (A), Equation
\eqref{eq:matrix_a}) - \textbf{Pragmatic Landscape:} What matters
(matrix (C), Equation \eqref{eq:matrix_c}) - \textbf{Causal Structure:}
What can be controlled (matrix (B), Equation \eqref{eq:matrix_b}) -
\textbf{Initial Assumptions:} What is taken for granted (matrix (D),
Equation \eqref{eq:matrix_d})
\end{block}

\begin{block}{As Subject}
\protect\phantomsection\label{as-subject}
The modeler applies Active Inference to their own cognition: - Uses
meta-epistemic principles to design research methodologies - Employs
meta-pragmatic frameworks for scientific decision-making - Engages in
recursive self-modeling of cognitive processes
\end{block}

\begin{block}{Recursive Self-Understanding}
\protect\phantomsection\label{recursive-self-understanding}
This creates a recursive loop of understanding: 1. Modeler uses Active
Inference to model cognitive systems 2. Insights from modeling improve
understanding of modeler's own cognition 3. Improved self-understanding
leads to better models 4. Cycle continues with increasing sophistication
\end{block}
\end{block}

\begin{block}{Cognitive Security Implications}
\protect\phantomsection\label{sec:cognitive_security_implications}
The meta-level framework has significant implications for cognitive
security and the robustness of belief systems.

\begin{block}{Meta-Cognitive Vulnerabilities}
\protect\phantomsection\label{meta-cognitive-vulnerabilities}
Understanding meta-cognitive processing reveals potential
vulnerabilities that traditional security models miss:

\textbf{Quadrant 3 Attacks:} Manipulation of confidence assessment
mechanisms - \textbf{False confidence calibration:} Adversaries can
provide feedback that systematically miscalibrates confidence
assessments, causing agents to over-trust or under-trust their
inferences - \textbf{Induced over/under-confidence:} By manipulating the
confidence assessment function inputs, attackers can cause agents to
switch strategies inappropriately (e.g., becoming overly conservative
when they should be exploratory) - \textbf{Meta-cognitive hijacking:}
Direct manipulation of meta-cognitive control parameters ((\lambda),
(\alpha), (\beta), (\gamma)) can redirect cognitive resources or disable
adaptive mechanisms

\textbf{Quadrant 4 Attacks:} Framework-level manipulation -
\textbf{Epistemic framework subversion:} Altering matrices (A), (B), or
(D) through learning or external influence can fundamentally change what
an agent believes is knowable - \textbf{Pragmatic landscape alteration:}
Modifying matrix (C) changes what the agent values, potentially
redirecting all goal-directed behavior - \textbf{Higher-order reasoning
corruption:} Manipulating framework optimization processes can cause
agents to evolve toward vulnerable or exploitable cognitive
architectures
\end{block}

\begin{block}{Defense Strategies}
\protect\phantomsection\label{defense-strategies}
The framework suggests defense approaches that operate at multiple
levels:

\textbf{Meta-Cognitive Monitoring (Quadrant 3):} Continuous validation
of confidence assessments through cross-validation with actual
performance, detecting miscalibration and triggering recalibration
mechanisms. This includes monitoring confidence trajectories, comparing
expected vs.~actual accuracy, and detecting anomalous confidence
patterns that may indicate manipulation.

\textbf{Framework Integrity Checks (Quadrant 4):} Verification of
epistemic and pragmatic consistency by monitoring framework parameters
for unexpected changes, detecting drift in matrices (A), (B), (C), (D)
that may indicate subversion, and maintaining framework coherence
through regularization terms (\mathcal{R}(\Theta)) that penalize
inconsistent specifications.

\textbf{Recursive Validation (Multi-Level):} Higher-order checking of
meta-level processes by applying the same validation mechanisms to
meta-cognitive systems themselves, creating recursive security layers.
This includes validating that confidence assessment mechanisms are
themselves functioning correctly, and that framework optimization
processes are not being corrupted.
\end{block}

\begin{block}{Societal Implications}
\protect\phantomsection\label{societal-implications}
These insights extend to societal cognitive security:

\begin{itemize}
\tightlist
\item
  \textbf{Information Warfare:} Meta-level manipulation of public belief
  systems
\item
  \textbf{AI Safety:} Ensuring artificial agents maintain meta-cognitive
  frameworks
\item
  \textbf{Educational Systems:} Developing curricula that build
  meta-cognitive resilience
\end{itemize}
\end{block}
\end{block}

\begin{block}{Free Energy Principle Integration}
\protect\phantomsection\label{sec:fep_integration}
The structure integrates seamlessly with the Free Energy Principle,
providing a concrete realization of FEP's abstract principles across
multiple scales of organization. This integration reveals how free
energy minimization operates at different levels simultaneously, from
physical boundary maintenance to cognitive belief updating to
meta-cognitive framework adaptation.

\begin{block}{What Is a Thing?}
\protect\phantomsection\label{what-is-a-thing}
The FEP defines a ``thing'' as a system that maintains its structure
over time through free energy minimization. Our framework shows how this
operates across multiple levels, revealing a nested hierarchy of
``things'':

\textbf{Physical Level:} Boundary maintenance through Markov
blankets---systems maintain physical structure by minimizing
thermodynamic free energy, creating boundaries that separate internal
from external states.

\textbf{Cognitive Level:} Belief updating through EFE
minimization---cognitive agents maintain accurate world models by
minimizing expected free energy, updating beliefs through Bayesian
inference while selecting actions that reduce uncertainty.

\textbf{Meta-Cognitive Level:} Framework adaptation through higher-order
reasoning---meta-cognitive systems maintain adaptive cognitive
architectures by optimizing framework parameters, evolving their own
processing structures based on performance analysis.

\textbf{Meta-Theoretical Level:} Scientific understanding through
recursive modeling---researchers maintain coherent theoretical
frameworks by applying Active Inference to understand Active Inference
itself, creating recursive self-improvement in scientific understanding.

This multi-level perspective reveals that ``things'' exist at multiple
scales simultaneously, each maintaining structure through free energy
minimization at their respective levels, creating a unified view of
organization from physics to cognition to science.
\end{block}

\begin{block}{Unification Across Domains}
\protect\phantomsection\label{unification-across-domains}
The structure provides a unified approach to diverse phenomena,
revealing common principles across scales:

\textbf{Biological Systems:} Organisms maintaining homeostasis through
metabolic processes that minimize thermodynamic free energy, creating
stable internal states despite environmental fluctuations.

\textbf{Artificial Agents:} AI systems with meta-learning capabilities
that minimize expected free energy through perception and action, while
also optimizing their own learning frameworks (Quadrant 4 operation).

\textbf{Social Systems:} Groups maintaining collective identity through
shared beliefs and values, where the group acts as a system minimizing
free energy at the social level through communication and coordination.

\textbf{Scientific Communities:} Knowledge accumulation through paradigm
shifts, where the scientific community maintains coherent theoretical
frameworks (minimizing free energy) while adapting those frameworks
based on empirical evidence and theoretical insights.
\end{block}
\end{block}

\begin{block}{Methodological Contributions}
\protect\phantomsection\label{sec:methodological_contributions}
The structure advances Active Inference methodology in several ways,
providing new tools for cognitive science research:

\begin{block}{Systematic Analysis Structure}
\protect\phantomsection\label{systematic-analysis-structure}
Provides a systematic way to analyze meta-level phenomena that were
previously difficult to formalize: - \textbf{Clear distinctions between
processing levels:} The (2 \times 2) structure creates unambiguous
boundaries between data/meta-data and cognitive/meta-cognitive
processing, enabling precise categorization of cognitive operations -
\textbf{Hierarchical organization of cognitive processes:} The quadrant
structure reveals how cognitive processes at different levels interact,
with higher quadrants building upon and regulating lower quadrants -
\textbf{Integration of multiple abstraction levels:} The structure
enables analysis of how processes at different scales (from basic
inference to framework evolution) operate simultaneously and influence
each other
\end{block}

\begin{block}{Research Design Tool}
\protect\phantomsection\label{research-design-tool}
Enables researchers to: - Design experiments targeting specific
quadrants - Compare interventions across processing levels - Develop
targeted cognitive enhancement strategies
\end{block}

\begin{block}{Theoretical Integration}
\protect\phantomsection\label{theoretical-integration}
Bridges multiple theoretical traditions, creating connections that were
previously difficult to formalize: - \textbf{Active Inference with
Meta-Cognition Research:} The framework formalizes meta-cognitive
processes (Quadrants 3 and 4) within the Active Inference mathematical
structure, enabling quantitative analysis of self-monitoring, confidence
assessment, and framework adaptation mechanisms.

\begin{itemize}
\item
  \textbf{Free Energy Principle with Cognitive Architectures:} The
  framework shows how FEP principles operate across multiple levels of
  cognitive organization, from basic inference (Quadrant 1) to framework
  evolution (Quadrant 4), providing a unified view of cognitive
  architecture design.
\item
  \textbf{Pragmatic Reasoning with Epistemic Logic:} The meta-pragmatic
  and meta-epistemic dimensions enable formal analysis of how value
  systems (pragmatic) and knowledge frameworks (epistemic) interact,
  creating a unified framework for understanding how ``what matters''
  and ``what can be known'' shape cognition together.
\end{itemize}
\end{block}
\end{block}

\begin{block}{Limitations and Future Directions}
\protect\phantomsection\label{sec:limitations_future}
\begin{block}{Current Limitations}
\protect\phantomsection\label{current-limitations}
\textbf{Empirical Validation:} The structure is primarily theoretical;
empirical validation is needed to confirm that the quadrant distinctions
correspond to measurable differences in cognitive processing. While the
mathematical formulations are theoretically sound, experimental
paradigms targeting each quadrant need development to validate the
framework's predictive power.

\textbf{Computational Complexity:} Higher quadrants involve complex
optimization problems. Quadrant 4's framework-level optimization
(\min\_\{\Theta\} \mathcal{F}(\pi; \Theta) + \mathcal{R}(\Theta))
requires searching high-dimensional parameter spaces, which can be
computationally expensive for large-scale systems. Efficient
approximation algorithms are needed for practical applications.

\textbf{Measurement Challenges:} Meta-level processes are difficult to
measure directly. While confidence assessment (Quadrant 3) and framework
parameters (Quadrant 4) can be inferred from behavior, direct
measurement of meta-cognitive processes remains challenging. Novel
measurement techniques combining behavioral, neural, and computational
approaches are needed.

\textbf{Scale Issues:} The structure's scaling to complex real-world
systems with thousands of states and actions requires further
development. While the theoretical framework applies at any scale,
computational methods for large-scale implementation need refinement,
particularly for Quadrants 3 and 4 where meta-cognitive processing adds
computational overhead.
\end{block}

\begin{block}{Future Research Directions}
\protect\phantomsection\label{future-research-directions}
\textbf{Empirical Studies:} Develop experimental paradigms for each
quadrant that can isolate and measure quadrant-specific processing. For
Quadrant 1, this might involve tasks requiring basic inference and
action selection. For Quadrant 2, tasks with varying observation quality
and meta-data availability. For Quadrant 3, tasks requiring confidence
assessment and strategy adaptation. For Quadrant 4, longitudinal studies
tracking framework parameter evolution over time.

\textbf{Computational Methods:} Develop efficient algorithms for
meta-level optimization, including approximate methods for Quadrant 4's
framework parameter search, hierarchical optimization techniques that
leverage the quadrant structure, and parallel computation strategies for
large-scale systems. Gradient-based and evolutionary approaches to
framework optimization need further development.

\textbf{Measurement Techniques:} Create novel approaches to
meta-cognitive process measurement, combining behavioral indicators
(response times, strategy switching), neural markers (brain activity
patterns associated with confidence and meta-cognitive control), and
computational modeling (inferring meta-cognitive parameters from
behavior). Multi-modal measurement approaches that triangulate across
methods will be particularly valuable.

\textbf{Applications:} Deploy the structure in AI systems for
meta-learning and self-improvement, in cognitive enhancement
interventions targeting specific quadrants, in educational systems for
meta-cognitive training, and in clinical applications for understanding
and treating cognitive disorders. Each application domain will provide
validation and refinement opportunities.
\end{block}

\begin{block}{Extension Possibilities}
\protect\phantomsection\label{extension-possibilities}
\textbf{Multi-Agent Systems:} Framework extension to social cognition
\textbf{Developmental Psychology:} Application to cognitive development
\textbf{Clinical Applications:} Therapeutic interventions targeting
specific quadrants \textbf{Educational Technology:} Meta-cognitive
training systems
\end{block}
\end{block}

\begin{block}{Broader Philosophical Implications}
\protect\phantomsection\label{sec:philosophical_implications}
The framework touches on fundamental questions about cognition and
reality.

\begin{block}{Nature of Intelligence}
\protect\phantomsection\label{nature-of-intelligence}
Active Inference suggests intelligence emerges from: - \textbf{Epistemic
Competence:} Ability to construct accurate world models -
\textbf{Pragmatic Wisdom:} Capacity for effective goal-directed behavior
- \textbf{Meta-Level Reflection:} Self-awareness and adaptive control -
\textbf{Framework Flexibility:} Ability to modify fundamental cognitive
structures
\end{block}

\begin{block}{Reality and Representation}
\protect\phantomsection\label{reality-and-representation}
The meta-epistemic aspect raises questions about: - \textbf{Multiple
Realities:} Different epistemic frameworks construct different worlds -
\textbf{Framework Relativity:} Cognitive adequacy depends on framework
appropriateness - \textbf{Reality Construction:} Cognition as active
construction, not passive reception
\end{block}

\begin{block}{Consciousness and Self-Awareness}
\protect\phantomsection\label{consciousness-and-self-awareness}
The recursive nature of meta-cognition suggests: -
\textbf{Self-Modeling:} Consciousness as modeling one's own cognitive
processes - \textbf{Hierarchical Self-Awareness:} Three levels of
self-reflection - \textbf{Emergent Properties:} Consciousness emerging
from meta-level cognitive organization
\end{block}
\end{block}
\end{frame}

\end{document}
