% Options for packages loaded elsewhere
\PassOptionsToPackage{unicode}{hyperref}
\PassOptionsToPackage{hyphens}{url}
\documentclass[
  ignorenonframetext,
]{beamer}
\newif\ifbibliography
\usepackage{pgfpages}
\setbeamertemplate{caption}[numbered]
\setbeamertemplate{caption label separator}{: }
\setbeamercolor{caption name}{fg=normal text.fg}
\beamertemplatenavigationsymbolsempty
% remove section numbering
\setbeamertemplate{part page}{
  \centering
  \begin{beamercolorbox}[sep=16pt,center]{part title}
    \usebeamerfont{part title}\insertpart\par
  \end{beamercolorbox}
}
\setbeamertemplate{section page}{
  \centering
  \begin{beamercolorbox}[sep=12pt,center]{section title}
    \usebeamerfont{section title}\insertsection\par
  \end{beamercolorbox}
}
\setbeamertemplate{subsection page}{
  \centering
  \begin{beamercolorbox}[sep=8pt,center]{subsection title}
    \usebeamerfont{subsection title}\insertsubsection\par
  \end{beamercolorbox}
}
% Prevent slide breaks in the middle of a paragraph
\widowpenalties 1 10000
\raggedbottom
\AtBeginPart{
  \frame{\partpage}
}
\AtBeginSection{
  \ifbibliography
  \else
    \frame{\sectionpage}
  \fi
}
\AtBeginSubsection{
  \frame{\subsectionpage}
}
\usepackage{iftex}
\ifPDFTeX
  \usepackage[T1]{fontenc}
  \usepackage[utf8]{inputenc}
  \usepackage{textcomp} % provide euro and other symbols
\else % if luatex or xetex
  \usepackage{unicode-math} % this also loads fontspec
  \defaultfontfeatures{Scale=MatchLowercase}
  \defaultfontfeatures[\rmfamily]{Ligatures=TeX,Scale=1}
\fi
\usepackage{lmodern}
\ifPDFTeX\else
  % xetex/luatex font selection
\fi
% Use upquote if available, for straight quotes in verbatim environments
\IfFileExists{upquote.sty}{\usepackage{upquote}}{}
\IfFileExists{microtype.sty}{% use microtype if available
  \usepackage[]{microtype}
  \UseMicrotypeSet[protrusion]{basicmath} % disable protrusion for tt fonts
}{}
\makeatletter
\@ifundefined{KOMAClassName}{% if non-KOMA class
  \IfFileExists{parskip.sty}{%
    \usepackage{parskip}
  }{% else
    \setlength{\parindent}{0pt}
    \setlength{\parskip}{6pt plus 2pt minus 1pt}}
}{% if KOMA class
  \KOMAoptions{parskip=half}}
\makeatother
\setlength{\emergencystretch}{3em} % prevent overfull lines
\providecommand{\tightlist}{%
  \setlength{\itemsep}{0pt}\setlength{\parskip}{0pt}}
\usepackage{bookmark}
\IfFileExists{xurl.sty}{\usepackage{xurl}}{} % add URL line breaks if available
\urlstyle{same}
\hypersetup{
  hidelinks,
  pdfcreator={LaTeX via pandoc}}

\author{\texorpdfstring{}{}}
\date{}

\begin{document}

\begin{frame}{Abstract}
\protect\phantomsection\label{sec:abstract}
Active Inference is a theoretical framework that unifies perception,
action, and learning within a single mathematical formalism.
Traditionally understood as providing a principled account of how
biological agents minimize surprise in their interactions with the
world, Active Inference operates at a fundamentally meta-level. Active
Inference is both \emph{meta-pragmatic} and \emph{meta-epistemic},
enabling modelers to specify particular pragmatic and epistemic
frameworks for the entities they study.

Our analysis introduces a (2 \times 2) matrix framework that structures
Active Inference's theoretical contributions across four quadrants
defined by the axes of Data/Meta-Data and Cognitive/Meta-Cognitive
processing. This framework reveals how Active Inference transcends
traditional reinforcement learning approaches by enabling modelers to
specify pragmatic landscapes within which agents operate.

We show that the Expected Free Energy (EFE) formulation, while appearing
to combine epistemic and pragmatic terms, actually operates at a
meta-level where the modeler specifies the boundaries of both domains.
Through this lens, Active Inference becomes a methodology for cognitive
science that enables researchers to explore how different epistemic and
pragmatic frameworks shape cognition, decision-making, and behavior.

The implications extend to cognitive security, where understanding
meta-level cognitive processing becomes crucial for defending against
manipulation of belief formation and value structures. The framework
provides a systematic approach for analyzing these meta-level phenomena
and their societal implications.

\textbf{Keywords:} active inference, free energy principle,
meta-cognition, meta-pragmatic, meta-epistemic, cognitive science,
cognitive security

\textbf{MSC2020:} 68T01, 91E10, 92B05
\end{frame}

\end{document}
