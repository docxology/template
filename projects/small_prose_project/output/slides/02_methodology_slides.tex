% Options for packages loaded elsewhere
\PassOptionsToPackage{unicode}{hyperref}
\PassOptionsToPackage{hyphens}{url}
\documentclass[
  ignorenonframetext,
]{beamer}
\newif\ifbibliography
\usepackage{pgfpages}
\setbeamertemplate{caption}[numbered]
\setbeamertemplate{caption label separator}{: }
\setbeamercolor{caption name}{fg=normal text.fg}
\beamertemplatenavigationsymbolsempty
% remove section numbering
\setbeamertemplate{part page}{
  \centering
  \begin{beamercolorbox}[sep=16pt,center]{part title}
    \usebeamerfont{part title}\insertpart\par
  \end{beamercolorbox}
}
\setbeamertemplate{section page}{
  \centering
  \begin{beamercolorbox}[sep=12pt,center]{section title}
    \usebeamerfont{section title}\insertsection\par
  \end{beamercolorbox}
}
\setbeamertemplate{subsection page}{
  \centering
  \begin{beamercolorbox}[sep=8pt,center]{subsection title}
    \usebeamerfont{subsection title}\insertsubsection\par
  \end{beamercolorbox}
}
% Prevent slide breaks in the middle of a paragraph
\widowpenalties 1 10000
\raggedbottom
\AtBeginPart{
  \frame{\partpage}
}
\AtBeginSection{
  \ifbibliography
  \else
    \frame{\sectionpage}
  \fi
}
\AtBeginSubsection{
  \frame{\subsectionpage}
}
\usepackage{iftex}
\ifPDFTeX
  \usepackage[T1]{fontenc}
  \usepackage[utf8]{inputenc}
  \usepackage{textcomp} % provide euro and other symbols
\else % if luatex or xetex
  \usepackage{unicode-math} % this also loads fontspec
  \defaultfontfeatures{Scale=MatchLowercase}
  \defaultfontfeatures[\rmfamily]{Ligatures=TeX,Scale=1}
\fi
\usepackage{lmodern}
\ifPDFTeX\else
  % xetex/luatex font selection
\fi
% Use upquote if available, for straight quotes in verbatim environments
\IfFileExists{upquote.sty}{\usepackage{upquote}}{}
\IfFileExists{microtype.sty}{% use microtype if available
  \usepackage[]{microtype}
  \UseMicrotypeSet[protrusion]{basicmath} % disable protrusion for tt fonts
}{}
\makeatletter
\@ifundefined{KOMAClassName}{% if non-KOMA class
  \IfFileExists{parskip.sty}{%
    \usepackage{parskip}
  }{% else
    \setlength{\parindent}{0pt}
    \setlength{\parskip}{6pt plus 2pt minus 1pt}}
}{% if KOMA class
  \KOMAoptions{parskip=half}}
\makeatother
\setlength{\emergencystretch}{3em} % prevent overfull lines
\providecommand{\tightlist}{%
  \setlength{\itemsep}{0pt}\setlength{\parskip}{0pt}}
\usepackage{bookmark}
\IfFileExists{xurl.sty}{\usepackage{xurl}}{} % add URL line breaks if available
\urlstyle{same}
\hypersetup{
  hidelinks,
  pdfcreator={LaTeX via pandoc}}

\author{\texorpdfstring{}{}}
\date{}

\begin{document}

\begin{frame}{Methodology}
\protect\phantomsection\label{methodology}
This section presents the methodological approach used in this research
project, demonstrating various mathematical concepts and notation.

\begin{block}{Mathematical Framework}
\protect\phantomsection\label{mathematical-framework}
We employ standard mathematical notation throughout our analysis.
Consider the following optimization problem:

\[\min_{x \in \mathbb{R}^n} f(x)\]

subject to:

\[g_i(x) \leq 0, \quad i = 1, \dots, m\]
\[h_j(x) = 0, \quad j = 1, \dots, p\]

where \(f: \mathbb{R}^n \rightarrow \mathbb{R}\) is the objective
function.
\end{block}

\begin{block}{Algorithm Development}
\protect\phantomsection\label{algorithm-development}
The proposed algorithm follows these steps:

\begin{enumerate}
\tightlist
\item
  \textbf{Initialization}

  \begin{itemize}
  \tightlist
  \item
    Set initial point \(x_0 \in \mathbb{R}^n\)
  \item
    Choose parameters \(\alpha, \beta, \gamma > 0\)
  \end{itemize}
\item
  \textbf{Iteration Process}

  \begin{itemize}
  \tightlist
  \item
    Compute gradient: \(\nabla f(x_k)\)
  \item
    Update direction: \(d_k = -\nabla f(x_k)\)
  \item
    Line search for step size \(\alpha_k\)
  \item
    Update: \(x_{k+1} = x_k + \alpha_k d_k\)
  \end{itemize}
\item
  \textbf{Convergence Check}

  \begin{itemize}
  \tightlist
  \item
    Test stopping criteria: \(\|\nabla f(x_k)\| < \epsilon\)
  \item
    If converged, return \(x_k\)
  \item
    Otherwise, increment \(k\) and repeat
  \end{itemize}
\end{enumerate}
\end{block}

\begin{block}{Convergence Analysis}
\protect\phantomsection\label{convergence-analysis}
The algorithm's convergence properties are analyzed using the following
theorem:

\textbf{Theorem 1.} If \(f\) is convex and continuously differentiable,
and the step sizes satisfy the Wolfe conditions, then the algorithm
converges to a stationary point.

\textbf{Proof sketch:} - By convexity:
\(f(y) \geq f(x) + \nabla f(x)^T (y - x)\) - Line search ensures
sufficient decrease - Gradient descent property guarantees convergence
to critical points
\end{block}

\begin{block}{Implementation Considerations}
\protect\phantomsection\label{implementation-considerations}
Key implementation aspects include:

\begin{itemize}
\tightlist
\item
  \textbf{Numerical stability}: Using appropriate floating-point
  precision
\item
  \textbf{Termination criteria}: Multiple stopping conditions
\item
  \textbf{Performance optimization}: Efficient gradient computation
\item
  \textbf{Error handling}: Robust exception management
\end{itemize}
\end{block}

\begin{block}{Validation Strategy}
\protect\phantomsection\label{validation-strategy}
The methodology is validated through:

\begin{itemize}
\tightlist
\item
  \textbf{Mathematical correctness}: Verification of derivations
\item
  \textbf{Numerical accuracy}: Comparison with known solutions
\item
  \textbf{Computational efficiency}: Performance benchmarking
\item
  \textbf{Robustness testing}: Edge case analysis
\end{itemize}

This approach ensures both theoretical soundness and practical
applicability of the proposed methods.
\end{block}
\end{frame}

\end{document}
