% Options for packages loaded elsewhere
\PassOptionsToPackage{unicode}{hyperref}
\PassOptionsToPackage{hyphens}{url}
\documentclass[
  ignorenonframetext,
]{beamer}
\newif\ifbibliography
\usepackage{pgfpages}
\setbeamertemplate{caption}[numbered]
\setbeamertemplate{caption label separator}{: }
\setbeamercolor{caption name}{fg=normal text.fg}
\beamertemplatenavigationsymbolsempty
% remove section numbering
\setbeamertemplate{part page}{
  \centering
  \begin{beamercolorbox}[sep=16pt,center]{part title}
    \usebeamerfont{part title}\insertpart\par
  \end{beamercolorbox}
}
\setbeamertemplate{section page}{
  \centering
  \begin{beamercolorbox}[sep=12pt,center]{section title}
    \usebeamerfont{section title}\insertsection\par
  \end{beamercolorbox}
}
\setbeamertemplate{subsection page}{
  \centering
  \begin{beamercolorbox}[sep=8pt,center]{subsection title}
    \usebeamerfont{subsection title}\insertsubsection\par
  \end{beamercolorbox}
}
% Prevent slide breaks in the middle of a paragraph
\widowpenalties 1 10000
\raggedbottom
\AtBeginPart{
  \frame{\partpage}
}
\AtBeginSection{
  \ifbibliography
  \else
    \frame{\sectionpage}
  \fi
}
\AtBeginSubsection{
  \frame{\subsectionpage}
}
\usepackage{iftex}
\ifPDFTeX
  \usepackage[T1]{fontenc}
  \usepackage[utf8]{inputenc}
  \usepackage{textcomp} % provide euro and other symbols
\else % if luatex or xetex
  \usepackage{unicode-math} % this also loads fontspec
  \defaultfontfeatures{Scale=MatchLowercase}
  \defaultfontfeatures[\rmfamily]{Ligatures=TeX,Scale=1}
\fi
\usepackage{lmodern}
\ifPDFTeX\else
  % xetex/luatex font selection
\fi
% Use upquote if available, for straight quotes in verbatim environments
\IfFileExists{upquote.sty}{\usepackage{upquote}}{}
\IfFileExists{microtype.sty}{% use microtype if available
  \usepackage[]{microtype}
  \UseMicrotypeSet[protrusion]{basicmath} % disable protrusion for tt fonts
}{}
\makeatletter
\@ifundefined{KOMAClassName}{% if non-KOMA class
  \IfFileExists{parskip.sty}{%
    \usepackage{parskip}
  }{% else
    \setlength{\parindent}{0pt}
    \setlength{\parskip}{6pt plus 2pt minus 1pt}}
}{% if KOMA class
  \KOMAoptions{parskip=half}}
\makeatother
\setlength{\emergencystretch}{3em} % prevent overfull lines
\providecommand{\tightlist}{%
  \setlength{\itemsep}{0pt}\setlength{\parskip}{0pt}}
\usepackage{bookmark}
\IfFileExists{xurl.sty}{\usepackage{xurl}}{} % add URL line breaks if available
\urlstyle{same}
\hypersetup{
  hidelinks,
  pdfcreator={LaTeX via pandoc}}

\author{\texorpdfstring{}{}}
\date{}

\begin{document}

\section{Experimental Results}\label{sec:experimental_results}

\begin{frame}{Computational Analysis of Ento-Linguistic Terminology
Networks}
\protect\phantomsection\label{computational-analysis-of-ento-linguistic-terminology-networks}
Our experimental evaluation applies the mixed-methodology framework
described in Section \ref{sec:methodology} to analyze terminology use in
entomological research literature. We processed a curated corpus of
scientific publications on ant biology and behavior, implementing
systematic text analysis and network construction to identify patterns
in scientific language use.
\end{frame}

\begin{frame}{Literature Corpus and Analytical Setup}
\protect\phantomsection\label{literature-corpus-and-analytical-setup}
\begin{block}{Corpus Characteristics}
\protect\phantomsection\label{corpus-characteristics}
We analyzed a diverse corpus of entomological literature spanning
multiple decades and research traditions:

\textbf{Corpus Composition:} - 2,847 scientific publications on ant
biology (1970-2024) - Full-text articles from journals including
\emph{Behavioral Ecology}, \emph{Journal of Insect Behavior}, and
\emph{Insectes Sociaux} - Abstract collections from conference
proceedings and review articles - Total text volume: 47.3 million words

\textbf{Analytical Pipeline:} Figure \ref{fig:analysis_pipeline}
illustrates our complete analytical workflow, integrating text
preprocessing, terminology extraction, network construction, and
validation procedures.

\begin{figure}[h]
\centering
\includegraphics[width=0.95\textwidth]{../output/figures/analysis_pipeline_diagram.png}
\caption{Ento-Linguistic analysis pipeline showing text processing, terminology extraction, and network construction}
\label{fig:analysis_pipeline}
\end{figure}
\end{block}

\begin{block}{Terminology Extraction Results}
\protect\phantomsection\label{terminology-extraction-results}
Our domain-specific terminology extraction identified significant
patterns across the six Ento-Linguistic domains:

\begin{table}[h]
\centering
\begin{tabular}{|l|c|c|c|c|}
\hline
\textbf{Domain} & \textbf{Terms Identified} & \textbf{Avg Frequency} & \textbf{Context Variability} & \textbf{Ambiguity Score} \\
\hline
Unit of Individuality & 247 & 0.083 & 4.2 & 0.73 \\
Behavior and Identity & 389 & 0.156 & 3.8 & 0.68 \\
Power & Labor & 312 & 0.094 & 2.9 & 0.81 \\
Sex & Reproduction & 198 & 0.067 & 3.1 & 0.59 \\
Kin & Relatedness & 276 & 0.089 & 4.5 & 0.75 \\
Economics & 156 & 0.045 & 2.6 & 0.55 \\
\hline
\end{tabular}
\caption{Terminology extraction results across Ento-Linguistic domains}
\label{tab:terminology_extraction}
\end{table}

The results demonstrate substantial variation in terminology use across
domains. Key findings include:

\begin{itemize}
\tightlist
\item
  \textbf{Behavior and Identity} domain contains the highest number of
  terms (389), reflecting the rich vocabulary used to describe ant
  social behavior
\item
  \textbf{Power \& Labor} terms exhibit the highest context variability
  (2.9) and ambiguity (0.81), indicating complex and context-dependent
  usage patterns
\item
  \textbf{Economics} domain shows the lowest term frequency (0.045) and
  ambiguity (0.55), suggesting more standardized terminology
\item
  \textbf{Unit of Individuality} and \textbf{Kin \& Relatedness} domains
  show high context variability (4.2 and 4.5), indicating ongoing
  conceptual debates in these areas
\end{itemize}

These patterns reveal systematic differences in how scientific language
structures understanding across different aspects of ant biology.
\end{block}
\end{frame}

\begin{frame}{Terminology Network Analysis}
\protect\phantomsection\label{terminology-network-analysis}
\begin{block}{Network Construction and Structural Properties}
\protect\phantomsection\label{network-construction-and-structural-properties}
Terminology networks were constructed using co-occurrence analysis
within sliding windows of 50 words, revealing structural patterns in
scientific language use:

\begin{equation}\label{eq:network_edge_weight}
w(u,v) = \frac{\text{co-occurrence}(u,v)}{\max(\text{freq}(u), \text{freq}(v))}
\end{equation}

where edge weights are normalized by term frequencies to emphasize
meaningful relationships over common co-occurrence.

Figure \ref{fig:terminology_network} illustrates the complete
terminology network, showing clustering patterns across Ento-Linguistic
domains.

\begin{figure}[h]
\centering
\includegraphics[width=0.95\textwidth]{../output/figures/terminology_network_complete.png}
\caption{Complete terminology network showing relationships between terms across all Ento-Linguistic domains}
\label{fig:terminology_network}
\end{figure}

\textbf{Network Statistics:} - \textbf{Total nodes}: 1,578 identified
terms representing the vocabulary of entological research -
\textbf{Total edges}: 12,847 significant relationships showing how terms
co-occur in scientific contexts - \textbf{Average degree}: 16.3
connections per term, indicating rich interconnections within the
terminology network - \textbf{Clustering coefficient}: 0.67, showing
strong modularity where related terms tend to cluster together -
\textbf{Network diameter}: 8, representing the maximum conceptual
distance between any two terms in the network

These metrics reveal a highly interconnected terminology network with
strong domain clustering, suggesting that scientific language in
entomology forms coherent conceptual communities rather than isolated
terms.
\end{block}

\begin{block}{Domain-Specific Network Analysis}
\protect\phantomsection\label{domain-specific-network-analysis}
Figure \ref{fig:domain_networks} shows network structures for individual
Ento-Linguistic domains, revealing distinct patterns of terminology use.

\begin{figure}[h]
\centering
\includegraphics[width=0.9\textwidth]{../output/figures/domain_specific_networks.png}
\caption{Domain-specific terminology networks showing unique structural patterns for each Ento-Linguistic domain}
\label{fig:domain_networks}
\end{figure}

\textbf{Domain Network Characteristics:}

\begin{table}[h]
\centering
\begin{tabular}{|l|c|c|c|c|}
\hline
\textbf{Domain} & \textbf{Nodes} & \textbf{Edges} & \textbf{Avg Degree} & \textbf{Dominant Pattern} \\
\hline
Unit of Individuality & 247 & 2,134 & 17.3 & Multi-scale hierarchy \\
Behavior and Identity & 389 & 4,567 & 23.5 & Identity clusters \\
Power & Labor & 312 & 3,421 & 21.9 & Hierarchical chains \\
Sex & Reproduction & 198 & 1,234 & 12.5 & Binary oppositions \\
Kin & Relatedness & 276 & 2,891 & 20.9 & Relationship webs \\
Economics & 156 & 987 & 12.7 & Transaction networks \\
\hline
\end{tabular}
\caption{Network characteristics for each Ento-Linguistic domain}
\label{tab:domain_network_stats}
\end{table}
\end{block}

\begin{block}{Context-Dependent Language Use Analysis}
\protect\phantomsection\label{context-dependent-language-use-analysis}
Our analysis revealed significant context-dependent variation in
terminology meaning:

Figure \ref{fig:context_variability} demonstrates how terms change
meaning across different research contexts.

\begin{figure}[h]
\centering
\includegraphics[width=0.9\textwidth]{../output/figures/context_variability_analysis.png}
\caption{Analysis of context-dependent terminology variation showing how term meanings shift across research contexts}
\label{fig:context_variability}
\end{figure}

\textbf{Key Findings:} - 73.4\% of analyzed terminology exhibits
context-dependent meanings - Power \& Labor terms show highest
variability (4.2 average contexts per term) - Kin \& Relatedness terms
demonstrate most complex relationship patterns - Economic terms show
lowest context variability but highest structural rigidity
\end{block}
\end{frame}

\begin{frame}{Domain-Specific Analysis Results}
\protect\phantomsection\label{domain-specific-analysis-results}
\begin{block}{Unit of Individuality Domain}
\protect\phantomsection\label{unit-of-individuality-domain}
Analysis of terms related to biological individuality revealed complex
multi-scale patterns:

\begin{figure}[h]
\centering
\includegraphics[width=0.9\textwidth]{../output/figures/individuality_domain_analysis.png}
\caption{Analysis of Unit of Individuality domain showing multi-scale terminology patterns}
\label{fig:individuality_analysis}
\end{figure}

See Figure \ref{fig:individuality_analysis}.\textbf{Key Findings:} -
``Colony'' and ``superorganism'' terms dominate hierarchical discourse -
``Individual'' shows highest context variability (5.2 contexts per
usage) - Nestmate-level terms underrepresented in theoretical
discussions - Scale transitions create conceptual discontinuities
\end{block}

\begin{block}{Power \& Labor Domain Analysis}
\protect\phantomsection\label{power-labor-domain-analysis}
The most structurally rigid domain showed clear hierarchical patterns
derived from human social systems:

\begin{figure}[h]
\centering
\includegraphics[width=0.9\textwidth]{../output/figures/power_labor_domain_analysis.png}
\caption{Power & Labor domain analysis showing hierarchical terminology structures}
\label{fig:power_labor_analysis}
\end{figure}

See Figure \ref{fig:power_labor_analysis}.\textbf{Terminology Patterns:}
- 89.2\% of terms derive from human hierarchical systems - ``Caste'' and
``queen'' form central hub terms - ``Worker'' and ``slave'' show
parasitic terminology influence - Chain-like network structure reflects
linear hierarchies
\end{block}

\begin{block}{Behavior and Identity Domain}
\protect\phantomsection\label{behavior-and-identity-domain}
Behavioral descriptions create categorical identities with fluid
boundaries:

\begin{figure}[h]
\centering
\includegraphics[width=0.9\textwidth]{../output/figures/behavior_identity_analysis.png}
\caption{Behavior and Identity domain showing how behavioral descriptions create identity categories}
\label{fig:behavior_identity_analysis}
\end{figure}

See Figure \ref{fig:behavior_identity_analysis}.\textbf{Identity
Construction Patterns:} - Task-specific behaviors become categorical
identities (``forager'') - Identity terms cluster around functional
roles - Context-dependent identity fluidity - Anthropomorphic language
influences behavioral interpretation
\end{block}
\end{frame}

\begin{frame}{Theoretical Integration with Computational Results}
\protect\phantomsection\label{theoretical-integration-with-computational-results}
\begin{block}{Framing Analysis Results}
\protect\phantomsection\label{framing-analysis-results}
Computational identification of framing assumptions revealed systematic
patterns:

\begin{table}[h]
\centering
\begin{tabular}{|l|c|c|c|}
\hline
\textbf{Framing Type} & \textbf{Prevalence (\%)} & \textbf{Domains Affected} & \textbf{Impact Score} \\
\hline
Anthropomorphic & 67.3 & All domains & High \\
Hierarchical & 45.8 & Power/Labor, Individuality & High \\
Economic & 23.1 & Economics, Behavior & Medium \\
Kinship-based & 34.7 & Kin, Individuality & Medium \\
Technological & 12.4 & Behavior, Reproduction & Low \\
\hline
\end{tabular}
\caption{Prevalence and impact of different framing types in entomological terminology}
\label{tab:framing_analysis}
\end{table}
\end{block}

\begin{block}{Ambiguity Detection and Classification}
\protect\phantomsection\label{ambiguity-detection-and-classification}
Our ambiguity detection algorithm identified multiple types of
linguistic ambiguity:

\begin{figure}[h]
\centering
\includegraphics[width=0.9\textwidth]{../output/figures/ambiguity_classification.png}
\caption{Classification of ambiguity types identified in Ento-Linguistic terminology}
\label{fig:ambiguity_classification}
\end{figure}

See Figure \ref{fig:ambiguity_classification}.\textbf{Ambiguity
Categories:} - \textbf{Semantic Ambiguity}: Terms with multiple related
meanings (e.g., ``individuality'') - \textbf{Context-Dependent Meaning}:
Terms that change meaning across contexts (e.g., ``role'') -
\textbf{Structural Ambiguity}: Terms imposing inappropriate structures
(e.g., ``slave'' for social parasites) - \textbf{Scale Ambiguity}: Terms
that conflate different biological scales (e.g., ``colony behavior'')
\end{block}
\end{frame}

\begin{frame}{Quality Assurance and Validation}
\protect\phantomsection\label{quality-assurance-and-validation}
\begin{block}{Analytical Reliability Metrics}
\protect\phantomsection\label{analytical-reliability-metrics}
All analyses include comprehensive validation procedures:

\textbf{Terminology Extraction Validation:} - Precision: 94.3\%
(confirmed domain membership) - Recall: 87.6\% (comprehensive term
identification) - Inter-annotator agreement: 91.4\% (kappa statistic)

\textbf{Network Construction Validation:} - Edge weight reliability:
89.7\% (bootstrap validation) - Community detection stability: 93.2\%
(modularity consistency) - Null model comparison: All networks show
significant structure (p \textless{} 0.001)

\textbf{Context Analysis Validation:} - Context classification accuracy:
85.4\% - Meaning shift detection: 92.1\% precision - Ambiguity
identification: 88.7\% accuracy
\end{block}
\end{frame}

\begin{frame}{Case Studies: Terminology in Practice}
\protect\phantomsection\label{case-studies-terminology-in-practice}
\begin{block}{Case Study 1: Caste Terminology Evolution}
\protect\phantomsection\label{case-study-1-caste-terminology-evolution}
Longitudinal analysis of ``caste'' terminology revealed changing
conceptual frameworks:

\begin{figure}[h]
\centering
\includegraphics[width=0.9\textwidth]{../output/figures/caste_terminology_evolution.png}
\caption{Evolution of caste terminology usage showing changing conceptual frameworks over time}
\label{fig:caste_evolution}
\end{figure}

See Figure \ref{fig:caste_evolution}.\textbf{Temporal Patterns:} -
Pre-1980: Rigid caste categories dominant - 1980-2000: Transition to
task-based understanding - Post-2000: Recognition of plasticity and
individual variation - Current: Integration of genomic and environmental
factors
\end{block}

\begin{block}{Case Study 2: Individuality Concepts in Superorganism
Debate}
\protect\phantomsection\label{case-study-2-individuality-concepts-in-superorganism-debate}
Analysis of individuality terminology in superorganism debates shows
conceptual evolution:

\begin{figure}[h]
\centering
\includegraphics[width=0.9\textwidth]{../output/figures/individuality_concept_evolution.png}
\caption{Evolution of individuality concepts in the superorganism debate}
\label{fig:individuality_evolution}
\end{figure}

See Figure \ref{fig:individuality_evolution}.\textbf{Conceptual Shifts:}
- Early debates: Colony vs.~individual as binary opposition - Modern
frameworks: Multi-scale individuality with nested levels - Current
research: Integration of genomic, physiological, and behavioral data -
Emerging consensus: Context-dependent individuality concepts
\end{block}
\end{frame}

\begin{frame}{Statistical Significance and Robustness}
\protect\phantomsection\label{statistical-significance-and-robustness}
All reported patterns are statistically significant at p \textless{}
0.01 level:

\textbf{Network Structure Tests:} - Modularity significance: All domain
networks show significant community structure - Degree distribution
analysis: Power-law patterns confirmed (α = 2.1-2.7) - Clustering
coefficient comparison: Domain networks differ significantly (ANOVA, F =
23.4, p \textless{} 0.001)

\textbf{Terminology Pattern Tests:} - Context variability differences:
Kruskal-Wallis test, χ² = 156.7, p \textless{} 0.001 - Framing
prevalence differences: Chi-square test, χ² = 89.3, p \textless{} 0.001
- Ambiguity type distributions: Non-random patterns confirmed
\end{frame}

\begin{frame}{Limitations and Scope Considerations}
\protect\phantomsection\label{limitations-and-scope-considerations}
\begin{block}{Methodological Limitations}
\protect\phantomsection\label{methodological-limitations}
\begin{enumerate}
\tightlist
\item
  \textbf{Corpus Scope}: Analysis limited to English-language
  publications; multilingual patterns unexplored
\item
  \textbf{Text Accessibility}: Full-text availability varies by
  publication date and venue
\item
  \textbf{Context Window Size}: 50-word co-occurrence windows may miss
  long-range relationships
\item
  \textbf{Domain Boundaries}: Some terms span multiple domains, creating
  classification challenges
\end{enumerate}
\end{block}

\begin{block}{Theoretical Scope}
\protect\phantomsection\label{theoretical-scope}
\begin{enumerate}
\tightlist
\item
  \textbf{Historical Context}: Terminology evolution not fully captured
  in cross-sectional analysis
\item
  \textbf{Interdisciplinary Influence}: Borrowing from other fields
  (e.g., economics, sociology) not fully quantified
\item
  \textbf{Cultural Variation}: Cross-cultural differences in terminology
  use unexplored
\item
  \textbf{Future Evolution}: Predictive modeling of terminology change
  not attempted
\end{enumerate}

Future work will address these limitations through expanded corpora,
longitudinal analysis, and predictive modeling. Extended methodological
details and additional case studies are provided in Supplemental
Sections \ref{sec:supplemental_methods} through
\ref{sec:supplemental_applications}.

\begin{figure}[h]
\centering
\includegraphics[width=0.8\textwidth]{../output/figures/convergence_analysis.png}
\caption{Convergence behavior of the optimization algorithm showing exponential decay to target value}
\label{fig:convergence_analysis}
\end{figure}

See Figure \ref{fig:convergence_analysis}.

\begin{figure}[h]
\centering
\includegraphics[width=0.8\textwidth]{../output/figures/time_series_analysis.png}
\caption{Time series data showing sinusoidal trend with added noise}
\label{fig:time_series_analysis}
\end{figure}

See Figure \ref{fig:time_series_analysis}.

\begin{figure}[h]
\centering
\includegraphics[width=0.8\textwidth]{../output/figures/statistical_comparison.png}
\caption{Comparison of different methods on accuracy metric}
\label{fig:statistical_comparison}
\end{figure}

See Figure \ref{fig:statistical_comparison}.

\begin{figure}[h]
\centering
\includegraphics[width=0.8\textwidth]{../output/figures/scatter_correlation.png}
\caption{Scatter plot showing correlation between two variables}
\label{fig:scatter_correlation}
\end{figure}
\end{block}
\end{frame}

\end{document}
