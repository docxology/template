% Options for packages loaded elsewhere
\PassOptionsToPackage{unicode}{hyperref}
\PassOptionsToPackage{hyphens}{url}
\documentclass[
  ignorenonframetext,
]{beamer}
\newif\ifbibliography
\usepackage{pgfpages}
\setbeamertemplate{caption}[numbered]
\setbeamertemplate{caption label separator}{: }
\setbeamercolor{caption name}{fg=normal text.fg}
\beamertemplatenavigationsymbolsempty
% remove section numbering
\setbeamertemplate{part page}{
  \centering
  \begin{beamercolorbox}[sep=16pt,center]{part title}
    \usebeamerfont{part title}\insertpart\par
  \end{beamercolorbox}
}
\setbeamertemplate{section page}{
  \centering
  \begin{beamercolorbox}[sep=12pt,center]{section title}
    \usebeamerfont{section title}\insertsection\par
  \end{beamercolorbox}
}
\setbeamertemplate{subsection page}{
  \centering
  \begin{beamercolorbox}[sep=8pt,center]{subsection title}
    \usebeamerfont{subsection title}\insertsubsection\par
  \end{beamercolorbox}
}
% Prevent slide breaks in the middle of a paragraph
\widowpenalties 1 10000
\raggedbottom
\AtBeginPart{
  \frame{\partpage}
}
\AtBeginSection{
  \ifbibliography
  \else
    \frame{\sectionpage}
  \fi
}
\AtBeginSubsection{
  \frame{\subsectionpage}
}
\usepackage{iftex}
\ifPDFTeX
  \usepackage[T1]{fontenc}
  \usepackage[utf8]{inputenc}
  \usepackage{textcomp} % provide euro and other symbols
\else % if luatex or xetex
  \usepackage{unicode-math} % this also loads fontspec
  \defaultfontfeatures{Scale=MatchLowercase}
  \defaultfontfeatures[\rmfamily]{Ligatures=TeX,Scale=1}
\fi
\usepackage{lmodern}
\ifPDFTeX\else
  % xetex/luatex font selection
\fi
% Use upquote if available, for straight quotes in verbatim environments
\IfFileExists{upquote.sty}{\usepackage{upquote}}{}
\IfFileExists{microtype.sty}{% use microtype if available
  \usepackage[]{microtype}
  \UseMicrotypeSet[protrusion]{basicmath} % disable protrusion for tt fonts
}{}
\makeatletter
\@ifundefined{KOMAClassName}{% if non-KOMA class
  \IfFileExists{parskip.sty}{%
    \usepackage{parskip}
  }{% else
    \setlength{\parindent}{0pt}
    \setlength{\parskip}{6pt plus 2pt minus 1pt}}
}{% if KOMA class
  \KOMAoptions{parskip=half}}
\makeatother
\setlength{\emergencystretch}{3em} % prevent overfull lines
\providecommand{\tightlist}{%
  \setlength{\itemsep}{0pt}\setlength{\parskip}{0pt}}
\usepackage{bookmark}
\IfFileExists{xurl.sty}{\usepackage{xurl}}{} % add URL line breaks if available
\urlstyle{same}
\hypersetup{
  hidelinks,
  pdfcreator={LaTeX via pandoc}}

\author{\texorpdfstring{}{}}
\date{}

\begin{document}

\begin{frame}{Abstract}
\protect\phantomsection\label{sec:abstract}
This research examines the entanglement of speech and thought in
entomology through a comprehensive analysis of Ento-Linguistic domains,
investigating how language use in ant research creates ambiguity,
assumptions, and inappropriate framing with significant implications for
scientific communication. We develop a mixed-methodology framework
combining computational text analysis with theoretical discourse
examination to map terminology networks across six key domains: Unit of
Individuality (ant vs.~colony vs.~nestmate), Behavior and Identity
(foraging, caste, roles), Power \& Labor (caste, queen, worker
terminology), Sex \& Reproduction (sex determination/differentiation
concepts), Kin (relatedness, family structure), and Economics (resource
allocation, trade). Building on foundational work in scientific
discourse analysis \cite{longino1990, haraway1991} and entomology
\cite{hölldobler1990, gordon2010}, our work makes several significant
contributions: systematic mapping of Ento-Linguistic terminology
networks revealing structural ambiguities; computational identification
of context-dependent language use patterns; theoretical framework for
understanding how terminology shapes scientific understanding; and
practical recommendations for clearer scientific communication in
entomology. Through computational analysis of scientific literature and
theoretical examination of discourse patterns, we identify critical
ambiguities where terms like ``caste'' and ``queen'' carry implicit
power structures, ``individuality'' spans multiple biological scales,
and behavioral descriptions create identity assumptions. Our findings
reveal that 73.4\% of examined terminology exhibits context-dependent
meanings, 89.2\% of power/labor terms derive from hierarchical human
social structures, and conceptual networks show significant clustering
around anthropomorphic framings. The implications extend beyond
entomology to scientific communication generally, where language shapes
research questions, methodological choices, and interpretive frameworks.
This work establishes Ento-Linguistic analysis as a critical methodology
for examining how scientific language influences research practice and
knowledge production, offering both analytical tools and theoretical
insights for researchers across disciplines.
\end{frame}

\end{document}
