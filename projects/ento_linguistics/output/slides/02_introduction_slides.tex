% Options for packages loaded elsewhere
\PassOptionsToPackage{unicode}{hyperref}
\PassOptionsToPackage{hyphens}{url}
\documentclass[
  ignorenonframetext,
]{beamer}
\newif\ifbibliography
\usepackage{pgfpages}
\setbeamertemplate{caption}[numbered]
\setbeamertemplate{caption label separator}{: }
\setbeamercolor{caption name}{fg=normal text.fg}
\beamertemplatenavigationsymbolsempty
% remove section numbering
\setbeamertemplate{part page}{
  \centering
  \begin{beamercolorbox}[sep=16pt,center]{part title}
    \usebeamerfont{part title}\insertpart\par
  \end{beamercolorbox}
}
\setbeamertemplate{section page}{
  \centering
  \begin{beamercolorbox}[sep=12pt,center]{section title}
    \usebeamerfont{section title}\insertsection\par
  \end{beamercolorbox}
}
\setbeamertemplate{subsection page}{
  \centering
  \begin{beamercolorbox}[sep=8pt,center]{subsection title}
    \usebeamerfont{subsection title}\insertsubsection\par
  \end{beamercolorbox}
}
% Prevent slide breaks in the middle of a paragraph
\widowpenalties 1 10000
\raggedbottom
\AtBeginPart{
  \frame{\partpage}
}
\AtBeginSection{
  \ifbibliography
  \else
    \frame{\sectionpage}
  \fi
}
\AtBeginSubsection{
  \frame{\subsectionpage}
}
\usepackage{iftex}
\ifPDFTeX
  \usepackage[T1]{fontenc}
  \usepackage[utf8]{inputenc}
  \usepackage{textcomp} % provide euro and other symbols
\else % if luatex or xetex
  \usepackage{unicode-math} % this also loads fontspec
  \defaultfontfeatures{Scale=MatchLowercase}
  \defaultfontfeatures[\rmfamily]{Ligatures=TeX,Scale=1}
\fi
\usepackage{lmodern}
\ifPDFTeX\else
  % xetex/luatex font selection
\fi
% Use upquote if available, for straight quotes in verbatim environments
\IfFileExists{upquote.sty}{\usepackage{upquote}}{}
\IfFileExists{microtype.sty}{% use microtype if available
  \usepackage[]{microtype}
  \UseMicrotypeSet[protrusion]{basicmath} % disable protrusion for tt fonts
}{}
\makeatletter
\@ifundefined{KOMAClassName}{% if non-KOMA class
  \IfFileExists{parskip.sty}{%
    \usepackage{parskip}
  }{% else
    \setlength{\parindent}{0pt}
    \setlength{\parskip}{6pt plus 2pt minus 1pt}}
}{% if KOMA class
  \KOMAoptions{parskip=half}}
\makeatother
\setlength{\emergencystretch}{3em} % prevent overfull lines
\providecommand{\tightlist}{%
  \setlength{\itemsep}{0pt}\setlength{\parskip}{0pt}}
\usepackage{bookmark}
\IfFileExists{xurl.sty}{\usepackage{xurl}}{} % add URL line breaks if available
\urlstyle{same}
\hypersetup{
  hidelinks,
  pdfcreator={LaTeX via pandoc}}

\author{\texorpdfstring{}{}}
\date{}

\begin{document}

\section{Introduction}\label{sec:introduction}

\begin{frame}{Speech and Thought Entanglement in Scientific
Communication}
\protect\phantomsection\label{speech-and-thought-entanglement-in-scientific-communication}
Speech and thought are inextricably entangled, particularly in
scientific discourse where language not only describes phenomena but
actively shapes how we perceive, categorize, and investigate them. This
entanglement becomes especially critical in entomology, where
researchers employ anthropomorphic terminology that carries implicit
assumptions about individuality, agency, and social structure. Our work
examines this entanglement through systematic analysis of
Ento-Linguistic domains---specific areas where language use in ant
research creates ambiguity, assumptions, or inappropriate framing.
\end{frame}

\begin{frame}{Motivation: Clear Communication as Ethical Imperative}
\protect\phantomsection\label{motivation-clear-communication-as-ethical-imperative}
Given the value-aligned nature of scientific communication, where
researchers communicate with present and future colleagues on their
``best behavior,'' there is compelling motivation to examine and improve
how language shapes scientific understanding. This motivation stems from
recognition that language is not merely descriptive but
constitutive---it actively structures research questions, methodological
approaches, and interpretive frameworks.

The consequential imperative is that this represents the optimal moment
to examine and improve scientific language use. Rather than perpetuating
potentially problematic terminology, researchers have an ethical
responsibility to critically examine how language influences scientific
practice and knowledge production.
\end{frame}

\begin{frame}{Addressing the Preliminary Objection}
\protect\phantomsection\label{addressing-the-preliminary-objection}
A common objection to improving scientific language is that changing
terminology creates disconnection from existing literature, making it
difficult to locate relevant research. For instance, if entomologists
abandon terms like ``caste'' or ``slave,'' how would researchers find
papers about task performance in ants?

However, this objection inadvertently strengthens our motivation. If we
continue using potentially problematic terminology merely for
convenience, we perpetuate and compound existing issues rather than
addressing them. The appropriate response is not to maintain the status
quo, but to actively work toward clearer communication while developing
the necessary tools for literature synthesis.

The solution lies not in avoidance, but in embracing the challenge: we
should restructure information from past literature (including original
data and documents where possible) and establish new meta-standards for
scientific communication. This represents an exciting opportunity to set
standards for how we care about scientific literature, research
communities, and the systems we study.
\end{frame}

\begin{frame}{Ento-Linguistic Domains: A Framework for Analysis}
\protect\phantomsection\label{ento-linguistic-domains-a-framework-for-analysis}
Our analysis centers on six key Ento-Linguistic domains where language
use can be particularly ambiguous, assumptive, or inappropriate:

\begin{block}{1. Unit of Individuality}
\protect\phantomsection\label{unit-of-individuality}
What constitutes an ``ant''---the nestmate, the colony, or something
else? This domain encompasses debates about biological individuality,
from individual nestmates to super-organismal colony concepts, examining
how terminology influences research at different scales of analysis.
\end{block}

\begin{block}{2. Behavior and Identity}
\protect\phantomsection\label{behavior-and-identity}
How do behavioral descriptions create identity assumptions? When an ant
is observed carrying a seed, is it meaningfully described as
``foraging,'' and does this make it ``a forager''? This domain examines
how behavioral language creates categorical identities that may not
reflect biological reality.
\end{block}

\begin{block}{3. Power \& Labor}
\protect\phantomsection\label{power-labor}
What social structures do terms like ``caste,'' ``queen,'' ``worker,''
and ``slave'' impose on ant societies? This domain investigates how
terminology derived from human hierarchical systems shapes scientific
understanding of ant social organization.
\end{block}

\begin{block}{4. Sex \& Reproduction}
\protect\phantomsection\label{sex-reproduction}
How do sex/gender concepts from human societies influence entomological
research? Terms like ``sex determination'' and ``sex differentiation''
carry implicit assumptions about binary gender systems that may not map
cleanly to ant reproductive biology.
\end{block}

\begin{block}{5. Kin and Relatedness}
\protect\phantomsection\label{kin-and-relatedness}
What constitutes ``kin'' in ant societies, and how are different forms
of relatedness (genetic, epigenetic, chemical, spatial) conceptualized?
This domain examines how human kinship terminology influences
understanding of ant social relationships.
\end{block}

\begin{block}{6. Economics}
\protect\phantomsection\label{economics}
How do economic concepts structure understanding of resource allocation
and trade in ant societies? This domain investigates how human economic
terminology shapes analysis of ant foraging, resource distribution, and
colony-level resource management.
\end{block}
\end{frame}

\begin{frame}{Research Approach}
\protect\phantomsection\label{research-approach}
This work employs a mixed-methodology framework combining computational
text analysis with theoretical discourse examination. We systematically
map terminology networks, identify context-dependent language use, and
develop recommendations for clearer scientific communication. The
computational component processes large corpora of entomological
literature to identify statistical patterns in language use, while the
theoretical component examines how these patterns reflect deeper
conceptual structures. Together, these approaches provide both empirical
evidence and interpretive depth for understanding how scientific
language constitutes research objects and relationships.
\end{frame}

\begin{frame}{Manuscript Organization}
\protect\phantomsection\label{manuscript-organization}
The manuscript develops this analysis through several interconnected
sections:

\begin{enumerate}
\tightlist
\item
  \textbf{Abstract} (Section \ref{sec:abstract}): Overview of
  Ento-Linguistic research and key contributions
\item
  \textbf{Introduction} (Section \ref{sec:introduction}): Speech/thought
  entanglement and research motivation
\item
  \textbf{Methodology} (Section \ref{sec:methodology}):
  Mixed-methodological framework for Ento-Linguistic analysis
\item
  \textbf{Experimental Results} (Section
  \ref{sec:experimental_results}): Computational analysis of terminology
  networks
\item
  \textbf{Discussion} (Section \ref{sec:discussion}): Theoretical
  implications for scientific communication
\item
  \textbf{Conclusion} (Section \ref{sec:conclusion}): Future directions
  and meta-standards for clear communication
\item
  \textbf{Supplemental Materials}: Extended analyses, case studies, and
  methodological details
\item
  \textbf{References} (Section \ref{sec:references}): Bibliography and
  cited works
\end{enumerate}
\end{frame}

\begin{frame}{Example Analysis: Terminology Network Visualization}
\protect\phantomsection\label{example-analysis-terminology-network-visualization}
The following figure demonstrates our computational approach to mapping
Ento-Linguistic terminology networks:

\begin{figure}[h]
\centering
\includegraphics[width=0.9\textwidth]{../output/figures/terminology_network_example.png}
\caption{Example terminology network showing relationships between terms across Ento-Linguistic domains. Nodes represent terms, edges represent co-occurrence relationships, and colors indicate domain classifications.}
\label{fig:terminology_network_example}
\end{figure}

As shown in Figure \ref{fig:terminology_network_example}, computational
analysis reveals structural patterns in scientific terminology that
influence research discourse. This visualization demonstrates how terms
cluster around conceptual domains and create networks of meaning that
shape scientific understanding.
\end{frame}

\begin{frame}{Data and Analysis Framework}
\protect\phantomsection\label{data-and-analysis-framework}
Our analysis framework integrates multiple data sources and
methodological approaches:

\begin{itemize}
\tightlist
\item
  \textbf{Literature Corpus}: Scientific publications on ant biology and
  behavior
\item
  \textbf{Terminology Database}: Curated collection of Ento-Linguistic
  terms with usage contexts
\item
  \textbf{Computational Analysis}: Text mining, network analysis, and
  pattern detection
\item
  \textbf{Theoretical Examination}: Discourse analysis and conceptual
  mapping
\item
  \textbf{Visualization}: Interactive networks and domain-specific
  analyses
\end{itemize}

All data and analysis code are fully reproducible and available for
validation and extension.
\end{frame}

\begin{frame}{Implications for Scientific Practice}
\protect\phantomsection\label{implications-for-scientific-practice}
This work has broader implications for how scientists communicate across
disciplines. By examining language use in entomology---a field with rich
descriptive traditions and complex social systems---we develop
principles that apply to scientific communication generally. The goal is
not merely to critique existing practice, but to establish foundations
for clearer, more precise scientific discourse that better serves
research communities and the phenomena they study.
\end{frame}

\begin{frame}{Cross-Referencing Scientific Concepts}
\protect\phantomsection\label{cross-referencing-scientific-concepts}
The manuscript employs comprehensive cross-referencing to connect
concepts across domains:

\begin{itemize}
\tightlist
\item
  \textbf{Domain References}: Cross-references between Ento-Linguistic
  domains (e.g., how power terminology influences individuality
  concepts)
\item
  \textbf{Terminology Networks}: References to computational analyses of
  term relationships
\item
  \textbf{Theoretical Frameworks}: Connections between computational
  findings and theoretical implications
\item
  \textbf{Methodological Integration}: Links between analytical
  approaches and interpretive frameworks
\end{itemize}

All references are automatically numbered and updated, ensuring the
manuscript maintains coherence as analyses develop and interconnect.
\end{frame}

\end{document}
