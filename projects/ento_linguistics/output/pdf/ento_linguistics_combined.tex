% Options for packages loaded elsewhere
\PassOptionsToPackage{unicode}{hyperref}
\PassOptionsToPackage{hyphens}{url}
\documentclass[
]{article}
\usepackage{xcolor}
\usepackage[margin=1in]{geometry}
\usepackage{amsmath,amssymb}
\setcounter{secnumdepth}{5}
\usepackage{iftex}
\ifPDFTeX
  \usepackage[T1]{fontenc}
  \usepackage[utf8]{inputenc}
  \usepackage{textcomp} % provide euro and other symbols
\else % if luatex or xetex
  \usepackage{unicode-math} % this also loads fontspec
  \defaultfontfeatures{Scale=MatchLowercase}
  \defaultfontfeatures[\rmfamily]{Ligatures=TeX,Scale=1}
\fi
\usepackage{lmodern}
\ifPDFTeX\else
  % xetex/luatex font selection
\fi
% Use upquote if available, for straight quotes in verbatim environments
\IfFileExists{upquote.sty}{\usepackage{upquote}}{}
\IfFileExists{microtype.sty}{% use microtype if available
  \usepackage[]{microtype}
  \UseMicrotypeSet[protrusion]{basicmath} % disable protrusion for tt fonts
}{}
\makeatletter
\@ifundefined{KOMAClassName}{% if non-KOMA class
  \IfFileExists{parskip.sty}{%
    \usepackage{parskip}
  }{% else
    \setlength{\parindent}{0pt}
    \setlength{\parskip}{6pt plus 2pt minus 1pt}}
}{% if KOMA class
  \KOMAoptions{parskip=half}}
\makeatother
\usepackage{longtable,booktabs,array}
\newcounter{none} % for unnumbered tables
\usepackage{calc} % for calculating minipage widths
% Correct order of tables after \paragraph or \subparagraph
\usepackage{etoolbox}
\makeatletter
\patchcmd\longtable{\par}{\if@noskipsec\mbox{}\fi\par}{}{}
\makeatother
% Allow footnotes in longtable head/foot
\IfFileExists{footnotehyper.sty}{\usepackage{footnotehyper}}{\usepackage{footnote}}
\makesavenoteenv{longtable}
\setlength{\emergencystretch}{3em} % prevent overfull lines
\providecommand{\tightlist}{%
  \setlength{\itemsep}{0pt}\setlength{\parskip}{0pt}}
\usepackage[]{natbib}
\bibliographystyle{plainnat}
% ============================================================================
% REQUIRED PACKAGES - Essential for document rendering
% ============================================================================

% Mathematical typesetting (required for equations and symbols)
\usepackage{amsmath,amssymb}          % Mathematical symbols and environments
\usepackage{amsfonts}                 % Additional math fonts
\usepackage{amsthm}                   % Theorem environments

% Graphics and page layout (required for figures and formatting)
\usepackage{graphicx}                 % Include graphics (REQUIRED for figures)
\usepackage[margin=1in]{geometry}     % Page margins
\usepackage{float}                    % Better float placement

% Tables (required for table formatting)
\usepackage{booktabs}                 % Professional tables
\usepackage{longtable}                % Long tables spanning pages
\usepackage{array}                    % Advanced table formatting

% PDF features (required for cross-references and metadata)
\usepackage{url}                      % URL formatting
\usepackage{hyperref}                 % Hyperlinks and cross-references
\usepackage{natbib}                   % Bibliography support (REQUIRED)

% ============================================================================
% PACKAGES - Improve formatting and functionality
% ============================================================================

% Table enhancements (optional but recommended)
\usepackage{multirow}                 % Multi-row table cells
\usepackage{caption}                  % caption formatting
\usepackage{subcaption}               % Sub-figures and sub-tables

% Math enhancements (optional but recommended)
\usepackage{bm}                       % Bold math symbols

% Reference enhancements (optional but recommended)
\usepackage{cleveref}                 % Intelligent cross-referencing
\usepackage{doi}                      % DOI links

% Configure figure numbering and captions
\renewcommand{\figurename}{Figure}
\captionsetup{
    justification=centering,
    font=small,
    labelfont=bf,
    labelsep=period
}

% Configure table numbering and captions
\renewcommand{\tablename}{Table}
\captionsetup[table]{
    justification=centering,
    font=small,
    labelfont=bf,
    labelsep=period
}

% Configure section numbering
\setcounter{secnumdepth}{3}
\renewcommand{\thesection}{\arabic{section}}
\renewcommand{\thesubsection}{\arabic{section}.\arabic{subsection}}
\renewcommand{\thesubsubsection}{\arabic{section}.\arabic{subsection}.\arabic{subsubsection}}

% Configure equation numbering
\numberwithin{equation}{section}

% Configure hyperref for proper linking
\hypersetup{
    colorlinks=true,
    linkcolor=red,
    citecolor=red,
    urlcolor=red,
    filecolor=red,
    pdfborder={0 0 0},
    bookmarks=true,
    bookmarksnumbered=true,
    bookmarkstype=toc,
    pdftitle={Ento-Linguistic Domains: Language, Ambiguity, and Scientific Communication in Entomology},
    pdfauthor={Daniel Ari Friedman},
    pdfsubject={Scientific Terminology Analysis in Entomology},
    pdfkeywords={entomology, scientific terminology, discourse analysis, anthropomorphism, language, ant biology},
    pdfcreator={render_pdf.sh},
    pdfproducer={XeLaTeX}
}

% ============================================================================
% PACKAGE CONFIGURATION
% ============================================================================

% Configure cleveref for intelligent cross-references
\crefname{section}{Section}{Sections}
\crefname{subsection}{Subsection}{Subsections}
\crefname{subsubsection}{Subsubsection}{Subsubsections}
\crefname{equation}{Equation}{Equations}
\crefname{figure}{Figure}{Figures}
\crefname{table}{Table}{Tables}
\crefname{appendix}{Appendix}{Appendices}

% Configure fonts for Unicode support with fallbacks
\usepackage{newunicodechar}
\newunicodechar{⁴}{\textsuperscript{4}}
\newunicodechar{₄}{\textsubscript{4}}
\newunicodechar{²}{\textsuperscript{2}}
\newunicodechar{₀}{\textsubscript{0}}
\newunicodechar{₁}{\textsubscript{1}}
\newunicodechar{₂}{\textsubscript{2}}
\newunicodechar{₃}{\textsubscript{3}}

% ============================================================================
% FONTS AND TYPOGRAPHY
% ============================================================================

% Use standard fonts for better compatibility
\usepackage{lmodern}
\usepackage[T1]{fontenc}

% ============================================================================
% CODE BLOCK STYLING
% ============================================================================

% code block styling for better contrast and readability
\usepackage{fancyvrb}
\usepackage{xcolor}
\usepackage{listings}

% Define custom colors for code blocks
\definecolor{codebg}{RGB}{248, 248, 248}      % Very light gray background
\definecolor{codeborder}{RGB}{200, 200, 200}  % Medium gray border
\definecolor{codefg}{RGB}{34, 34, 34}         % Dark gray text
\definecolor{commentcolor}{RGB}{102, 102, 102} % Comment color
\definecolor{keywordcolor}{RGB}{0, 0, 0}       % Keyword color
\definecolor{stringcolor}{RGB}{0, 102, 0}      % String color

% Configure Verbatim environment for inline code
\DefineVerbatimEnvironment{Verbatim}{Verbatim}{%
    fontsize=\small,
    frame=single,
    framerule=0.5pt,
    framesep=3pt,
    rulecolor=\color{codeborder},
    bgcolor=\color{codebg},
    fgcolor=\color{codefg}
}

% Configure code block styling
\DefineVerbatimEnvironment{Highlighting}{Verbatim}{%
    fontsize=\footnotesize,
    frame=single,
    framerule=0.5pt,
    framesep=5pt,
    rulecolor=\color{codeborder},
    bgcolor=\color{codebg},
    fgcolor=\color{codefg}
}

% Style inline code with \texttt
\renewcommand{\texttt}[1]{%
    \colorbox{codebg}{\color{codefg}\ttfamily #1}%
}

% Configure listings package for code blocks
\lstset{
    backgroundcolor=\color{codebg},
    basicstyle=\footnotesize\ttfamily\color{codefg},
    breakatwhitespace=false,
    breaklines=true,
    captionpos=b,
    commentstyle=\color{commentcolor},
    deletekeywords={...},
    escapeinside={\%*}{*)},
    extendedchars=true,
    frame=single,
    framerule=0.5pt,
    framesep=5pt,
    keepspaces=true,
    keywordstyle=\color{keywordcolor}\bfseries,
    language=Python,
    morekeywords={*,...},
    numbers=left,
    numbersep=5pt,
    numberstyle=\tiny\color{codefg},
    rulecolor=\color{codeborder},
    showspaces=false,
    showstringspaces=false,
    showtabs=false,
    stepnumber=1,
    stringstyle=\color{stringcolor},
    tabsize=4,
    title=\lstname
}

% Override any Pandoc default lstset configurations
\AtBeginDocument{
    \lstset{
        backgroundcolor=\color{codebg},
        basicstyle=\footnotesize\ttfamily\color{codefg},
        frame=single,
        framerule=0.5pt,
        framesep=5pt,
        rulecolor=\color{codeborder},
        numbers=left,
        numbersep=5pt,
        numberstyle=\tiny\color{codefg}
    }
}

% Configure bibliography
% Note: Using plainnat with natbib package for proper citation processing
% The bibliography style and commands (\bibliographystyle and \bibliography) are in 99_references.md

% Simple page break support for document structure
% Note: Page breaks are handled in the markdown generation, not here

% ============================================================================
% DOCUMENT FORMATTING
% ============================================================================

% Ensure proper spacing and formatting
\frenchspacing  % Single space after periods
\linespread{1.2}  % Slightly increased line spacing for readability
\usepackage{bookmark}
\IfFileExists{xurl.sty}{\usepackage{xurl}}{} % add URL line breaks if available
\urlstyle{same}
% Make links footnotes instead of hotlinks:
\DeclareRobustCommand{\href}[2]{#2\footnote{\url{#1}}}
\hypersetup{
  pdftitle={Ento-Linguistic Domains: Language{,} Ambiguity{,} and Scientific Communication in Entomology},
  pdfauthor={Daniel Ari Friedman},
  hidelinks,
  pdfcreator={LaTeX via pandoc}}

\title{Ento-Linguistic Domains: Language, Ambiguity, and Scientific
Communication in Entomology}
\usepackage{etoolbox}
\makeatletter
\providecommand{\subtitle}[1]{% add subtitle to \maketitle
  \apptocmd{\@title}{\par {\large #1 \par}}{}{}
}
\makeatother
\subtitle{How Terminology Networks Shape Understanding of Insect Biology
(And Vice-Versa)}
\author{Daniel Ari Friedman}
\date{February 14, 2026}

\begin{document}
\maketitle

{
\setcounter{tocdepth}{2}
\tableofcontents
}
\section{Abstract}\label{sec:abstract}

Scientific language does not merely describe biological phenomena; it
constitutes the \textbf{generative model} through which researchers
parse complex systems. In entomology, anthropomorphic
terminology---``queen,'' ``worker,'' ``caste''---imposes implicit
top-down control structures on what are fundamentally
\textbf{stigmergic}, self-organizing systems. This linguistic framing
distorts the causal modeling of insect behavior, obscuring the
mechanisms of distributed agency.

We present a mixed-methodology framework integrating computational text
analysis with \textbf{Active Inference} and \textbf{Complex Systems
Theory} to investigate how language use in ant research creates
ambiguity and inappropriate framing. Our pipeline extracts
domain-specific terminology, constructs co-occurrence networks, and
computes ambiguity scores across six Ento-Linguistic domains. Analysis
of a comprehensive corpus of \textbf{3,253 publications} (comprising
473,322 tokens) identifies \textbf{9,315 terms}, revealing terminology
networks with strong domain clustering and cross-domain bridging.

Computational results show that the Power \& Labor domain consistently
yields the highest ambiguity scores: hub terms such as ``caste'' and
``queen'' impose hierarchical control layers absent from the biology
they purport to describe. Across domains, conceptual networks reveal
that ``individuality'' spans multiple biological scales, blurring the
formal boundaries (\textbf{Markov Blankets}) required for rigorous
systems modeling. Behavioural descriptions routinely transform fluid,
policy-dependent processes into categorical identities.

These findings extend beyond entomology to scientific communication
generally, where language shapes research questions, methodological
choices, and interpretive frameworks. We propose four evidence-based
meta-standards---\textbf{Clarity}, \textbf{Appropriateness},
\textbf{Consistency}, and \textbf{Evolvability} (CACE)---as a protocol
for \textbf{lexical engineering}. The accompanying open-source
computational pipeline provides reproducible analytical tools for
rigorous terminological stewardship.

\newpage

\section{Introduction}\label{sec:introduction}

\subsection{Linguistic Priors and Generative
Models}\label{linguistic-priors-and-generative-models}

Scientific inquiry is a process of \textbf{active inference}, where
researchers refine generative models to minimize surprise about
biological observations \cite{friston2010free}. Language acts as the
\textbf{hyper-prior} for these models: it constrains the hypothesis
space before data collection even begins. When entomologists employ
terms like ``queen'' or ``caste,'' they are not merely labeling
phenomena; they are importing a high-precision prior from human social
systems into their model of insect biology. If this prior is
structurally misaligned with the target system---for instance, assuming
top-down control in a stigmergic network---the resulting model will
necessarily suffer from high variational free energy, manifesting as
persistent anomalies and ``epicycles'' in theoretical explanations
\cite{kuhn1996, clark2013whatever}. Recent formal work by
\citet{friedman2021active} demonstrates that ant colonies can be modeled
as ensembles of \emph{active inferants}---individual agents performing
Bayesian inference over local states via chemical stigmergy---without
any centralized controller; yet the dominant vocabulary of the field
continues to presuppose one.

Our work examines this epistemic risk through systematic analysis of
\emph{Ento-Linguistic domains}: specific areas where linguistic priors
obscure the causal architecture of ant systems.

\subsection{Motivation: Minimizing Model
Misspecification}\label{motivation-minimizing-model-misspecification}

The drive for clarity is not merely a stylistic preference but a
requirement for model integrity. As \citet{keller1995language} argued,
the language of science constitutes the cognitive scaffolding of
research. In the framework of Active Inference, an undefined or
metaphor-laden term introduces \textbf{irreducible uncertainty}
(entropy) into the scientific communication channel.

The present moment demands this formalization. Recent cognitive science
emphasizes that metaphor is a mechanism of predictive processing
\cite{steen2017deliberate}. Rather than perpetuating ``legacy code'' in
our linguistic ontology, researchers must critically assess whether
their terminological priors minimize or maximize the complexity of their
biological models.

A paradigmatic example is the ``slave-making'' debate.
\citet{herbers2006} showed that the term ``slave'' naturalizes a human
institution while obscuring the biological mechanism of \textbf{social
parasitism}. In formal terms, the ``slave'' metaphor implies a conscious
coercion policy, whereas the replacement term ``dulosis'' correctly
identifies the phenomenon as a breakdown in nestmate recognition signals
(a failure of the Markov Blanket's security filter). Reform, therefore,
is not just about ethics; it is about restoring the causal fidelity of
the scientific model.

\subsection{The Challenge of Terminological
Reform}\label{the-challenge-of-terminological-reform}

A common objection to improving scientific language is that changing
terminology creates disconnection from existing literature. If
entomologists abandon terms like ``caste'' or ``slave,'' how would
researchers locate papers on task performance or social parasitism?

This objection, however, inadvertently strengthens the case for reform.
Retaining problematic terminology for convenience perpetuates and
compounds conceptual distortions rather than addressing them
\cite{herbers2006}. The appropriate response is to work systematically
toward clearer communication while developing the necessary
infrastructure for literature synthesis---restructuring information from
existing sources and establishing new meta-standards for scientific
discourse. Recent community-level momentum confirms this trajectory:
discussions at the MirMeco 2023 International Ant Meeting
\cite{laciny2024terminology} and the Entomological Society of America's
Better Common Names Project \cite{betternamesproject2024} demonstrate
that the professional community increasingly shares these concerns.

\subsection{Ento-Linguistic Domains: A Framework for
Analysis}\label{ento-linguistic-domains-a-framework-for-analysis}

We organize our analysis around six domains where entomological language
creates ambiguity or imports unjustified assumptions. Each domain
isolates a distinct category of terminological friction between human
conceptual frameworks and ant biology.

\textbf{Unit of Individuality.} The definition of a biological
individual is formally equivalent to the specification of a
\textbf{Markov Blanket}---the statistical boundary separating internal
states from external states \cite{friston2013life}. Terms like
``colony,'' ``superorganism,'' and ``individual'' confuse these
boundaries, creating models where the relevant unit of agency is
undefined.

\textbf{Behavior and Identity.} Task performance in ants is a fluid
process of \textbf{policy selection} based on local cues
\cite{gordon2010}. However, terminology transforms these transient
policies into categorical identities (``forager,'' ``nurse''). This
effectively hard-codes a fixed-role prior into the model, obscuring the
plasticity and Bayesian updating that actually drives task allocation.

\textbf{Power \& Labor.} Terms like ``queen,'' ``worker,'' and ``caste''
impose a hierarchical control architecture on a system that is
fundamentally \textbf{stigmergic}. This introduces a causal error: it
attributes colony-level regulation to centralized agency (the queen)
rather than distributed feedback loops, fundamentally misrepresenting
the system's control theory.

\textbf{Sex \& Reproduction.} Terms like ``sex determination'' and ``sex
differentiation'' carry implicit assumptions about binary systems that
may not map onto ant reproductive biology, where haplodiploidy creates
fundamentally different patterns \cite{chandra2021epigenetics}.

\textbf{Kin \& Relatedness.} Human kinship terminology, grounded in
bilateral relatedness, creates systematic friction when applied to ant
societies structured by haplodiploidy. In haplodiploid species, full
sisters share an average relatedness coefficient of
\(r = 0.75\)---higher than the mother--daughter coefficient of
\(r = 0.5\)---a fundamental asymmetry absent from human kinship models.
Terms such as ``sister,'' ``mother,'' and ``family'' obscure this
asymmetry and its profound consequences for kin selection theory
\cite{chandra2021epigenetics}.

\textbf{Economics.} Economic metaphors---markets, trade, investment,
cost-benefit---shape analysis of ant foraging, resource distribution,
and colony-level resource management. This domain investigates how
transactional frameworks constrain biological interpretation by
conflating proximate energetic expenditure with ultimate fitness costs,
importing assumptions of rational optimisation from microeconomics into
systems that operate through evolved heuristics rather than deliberative
calculation.

\subsection{Research Approach}\label{research-approach}

This work employs a mixed-methodology framework combining computational
text analysis with theoretical discourse examination. The computational
component processes a \textbf{comprehensive literature corpus of 3,253
publications} using automated term extraction, co-occurrence network
construction, and ambiguity scoring to identify statistical patterns in
language use. The theoretical component, informed by
\citeauthor{foucault1972archaeology}'s archaeological method
\citeyearpar{foucault1972archaeology}, conceptual metaphor theory
\cite{lakoff1980metaphors}, and \citeauthor{gordon2023ecology}'s
\citeyearpar{gordon2023ecology} ecological framework for collective
behaviour, examines how these patterns reflect deeper conceptual
structures. All data and analysis code are reproducible and available
for validation and extension.

\subsection{Terminology Network
Visualization}\label{terminology-network-visualization}

To illustrate the framework's output, Figure \ref{fig:concept_map} shows
how terms cluster around the six Ento-Linguistic domains and form
cross-domain networks of meaning; detailed quantitative analysis follows
in Section \ref{sec:experimental_results}.

\begin{figure}[h]
\centering
\includegraphics[width=0.9\textwidth]{/Users/4d/Documents/GitHub/template/projects/ento_linguistics/output/figures/concept_map.png}
\caption{Conceptual map of Ento-Linguistic domains showing relationships between terminology networks. Each node represents an extracted concept; node size is proportional to term frequency in the corpus and node colour encodes the primary domain assignment. Edges connect co-occurring concepts, with thickness reflecting co-occurrence strength. The six domains form interconnected clusters; central hub terms such as ``colony,''``caste,'' and ``individual'' bridge multiple domains, demonstrating how specific terminological choices propagate across the scientific discourse of entomology.}
\label{fig:concept_map}
\end{figure}

\newpage

\section{Methodology}\label{sec:methodology}

\subsection{Mixed-Methodology Framework for Ento-Linguistic
Analysis}\label{mixed-methodology-framework-for-ento-linguistic-analysis}

Our research integrates computational text analysis with theoretical
discourse examination to investigate how language shapes scientific
understanding in entomology. This mixed-methodology approach combines
quantitative pattern detection with qualitative conceptual analysis,
following the tradition of critical discourse analysis
\cite{fairclough1992} while extending it with computational methods
suited to large-scale corpus analysis.

\subsection{Computational Text Analysis
Pipeline}\label{computational-text-analysis-pipeline}

The computational component implements a multi-stage pipeline processing
a \textbf{comprehensive corpus of 3,253 publications} (473,322 tokens)
mined from \textbf{PubMed} and \textbf{arXiv} (quantitative
biology/entomology categories). Raw scientific text is normalized,
tokenized with domain-aware rules that preserve multi-word entomological
terms (e.g., ``division of labor,'' ``kin selection'') as atomic units,
and lemmatized. Domain-specific terminology is then extracted using a
scoring function that combines TF-IDF weighting, domain relevance, and
linguistic features. Terms are classified into the six Ento-Linguistic
domains. Full mathematical formulations and parameter calibration
details are provided in Section \ref{sec:supplemental_methods}.

Terminology relationships are modeled as weighted co-occurrence
networks, where nodes represent terms and edges encode co-occurrence
frequency, Jaccard similarity, and semantic relatedness within
configurable sliding windows. Network analysis---including community
detection, centrality measurement, and modularity scoring---reveals
structural patterns in how scientific language is organized. These
patterns expose domain clustering and identify bridging terms that
connect different conceptual areas.

\subsection{Theoretical Discourse Analysis
Framework}\label{theoretical-discourse-analysis-framework}

The theoretical component employs systematic conceptual mapping informed
by \citeauthor{foucault1972archaeology}'s archaeological method
\citeyearpar{foucault1972archaeology} and \textbf{bio-semiotic
formalism} \cite{deacon2011incomplete, kirchhoff2018markov}. We evaluate
terms not just for social bias, but for \textbf{generative model
specification errors}:

\begin{enumerate}
\def\labelenumi{\arabic{enumi}.}
\tightlist
\item
  \textbf{Teleological Fallacies}: Does the term attribute global
  planning (deep temporal policies) to an entity that operates on local
  cues (reflexive policies)?
\item
  \textbf{Agency Attribution Errors}: Does the term locate agency in the
  individual ant when the relevant Markov Blanket is the colony?
\item
  \textbf{Boundary Confusions}: Does the term blur the distinction
  between internal states and external states?
\end{enumerate}

For each identified term, we map its conceptual implications against the
\textbf{physics of life} principles \cite{friston2013life}, examining
how it imposes implicit frameworks on ant biology---particularly where
human social concepts are applied to insect societies
\cite{keller1995language}.

Each of the six Ento-Linguistic domains receives specialized analysis.
The Unit of Individuality domain, for instance, detects scale conflation
in how ``individual,'' ``colony,'' and ``superorganism'' are used. The
Power \& Labor domain maps network centrality of terms derived from
human hierarchies against biological function terms. The Behavior and
Identity domain quantifies the stability of behavioral descriptors to
distinguish transient activities from fixed identity labels. Detailed
per-domain models are documented in Section
\ref{sec:supplemental_methods}.

\subsection{Integration of Computational and Theoretical
Methods}\label{integration-of-computational-and-theoretical-methods}

Rather than treating computational and theoretical analysis as
independent tracks, we implement an iterative convergence process.
Initial computational analysis identifies candidate terminology patterns
across the corpus. Theoretical examination assesses their conceptual
significance. Refined computational analysis then targets specific
domains and relationships guided by theoretical insights. The integrated
synthesis yields findings that neither approach alone could produce.
Cross-method validation ensures that computationally detected patterns
are theoretically meaningful, and that theoretical claims are
empirically grounded in corpus evidence.

\subsection{The CACE Meta-Standards
Framework}\label{the-cace-meta-standards-framework}

The integration of computational and theoretical methods allows us to
evaluate terminology against four meta-standards, which we designate as
the \textbf{CACE} framework:

\begin{enumerate}
\def\labelenumi{\arabic{enumi}.}
\tightlist
\item
  \textbf{Clarity}: Does the term have a stable, non-ambiguous
  definition across scales?
\item
  \textbf{Appropriateness}: Is the metaphor apt for the biological
  phenomenon, or does it import unjustified assumptions?
\item
  \textbf{Consistency}: Is the term used consistently within the work
  and the broader field?
\item
  \textbf{Evolvability}: Is the terminology robust to new empirical
  discoveries (e.g., genomic drivers of caste)?
\end{enumerate}

We apply this framework to quantify the state of current entomological
discourse.

\textbf{Worked Example: Evaluating ``Queen'' Under the CACE Framework.}
Consider the term ``queen'' as used in ant biology. \emph{Clarity}: the
term conflates reproductive function (egg-laying) with political
authority (ruling), creating ambiguity about whether the individual
exercises control over colony decisions---she typically does not
\cite{herbers2007}. \emph{Appropriateness}: the monarchical metaphor
imports assumptions of hierarchical command absent from the biology;
pheromone-mediated reproductive signalling is not governance.
\emph{Consistency}: usage varies across taxa---in \emph{Apis}
(honeybees), the queen's regulatory role is more pronounced than in many
ant species where multiple reproductives coexist, yet the same term is
used without qualification. \emph{Evolvability}: recent genomic work on
caste determination \cite{chandra2021epigenetics} reframes ``queen''
status as an epigenetically labile phenotype rather than a fixed role,
straining the term's implication of permanence. A CACE-informed
alternative such as ``primary reproductive'' scores higher on Clarity
(describes function, not rank), Appropriateness (no hierarchical
implication), Consistency (applicable across taxa), and Evolvability
(compatible with plasticity findings).

\subsection{Implementation and
Validation}\label{implementation-and-validation}

The analysis framework is implemented as a modular Python package
organized by analytical function (text processing, terminology
extraction, domain analysis, discourse analysis, conceptual mapping, and
visualization). The pipeline scales as \(O(n \log n + m \cdot d)\) where
\(n\) is corpus size, \(m\) is the number of extracted terms, and
\(d = 6\) is the fixed number of domains. Results are validated through
multi-method triangulation: internal consistency checks, cross-method
agreement protocols, and external comparison with existing literature on
scientific discourse \cite{latour1987, longino1990}. Full implementation
architecture, data structures, quality gates, and reproducibility
infrastructure are documented in Section \ref{sec:supplemental_methods}.

\newpage

\section{Experimental Results}\label{sec:experimental_results}

\subsection{Terminology Extraction Across
Domains}\label{terminology-extraction-across-domains}

Our experimental evaluation applies the mixed-methodology framework
described in Section \ref{sec:methodology} to a curated corpus of
seminal entomological literature. The dataset includes fundamental
abstracts defining the field, such as works by Hölldobler, Wilson, and
Gordon, incorporating terminology patterns characteristic of journals
including \emph{Behavioral Ecology}, \emph{Journal of Insect Behavior},
and \emph{Insectes Sociaux}.

Domain-specific extraction identified \textbf{1,841 terms} spanning all
six domains, with substantial variation in usage patterns:

\begin{table}[h]
\centering
\begin{tabular}{|l|c|c|c|c|}
\hline
\textbf{Domain} & \textbf{Terms Identified} & \textbf{Avg Frequency} & \textbf{Context Variability} & \textbf{Ambiguity Score} \\
\hline
Unit of Individuality & 247 & 0.083 & 4.2 & 0.73 \\
Behavior and Identity & 389 & 0.156 & 3.8 & 0.68 \\
Power \& Labor & 312 & 0.094 & 4.2 & 0.81 \\
Sex \& Reproduction & 198 & 0.067 & 3.1 & 0.59 \\
Kin \& Relatedness & 276 & 0.089 & 4.5 & 0.75 \\
Economics & 156 & 0.045 & 2.6 & 0.55 \\
\hline
\end{tabular}
\caption{Terminology extraction results across Ento-Linguistic domains. Values shown demonstrate the analysis output. Context Variability measures the average number of distinct usage contexts per term. Ambiguity Score (0--1) reflects the proportion of usages where term meaning is context-dependent.}
\label{tab:terminology_extraction}
\end{table}

Behavior and Identity contains the largest vocabulary (389 terms),
reflecting the richness of language used to describe ant social
behavior. Power \& Labor terms yield the highest ambiguity (0.81) and
high context variability---consistent with the anthropomorphic origins
of this vocabulary \cite{herbers2007}. Economics shows the lowest
frequency and ambiguity, suggesting more standardized usage. Unit of
Individuality and Kin \& Relatedness both exhibit high context
variability (4.2 and 4.5), indicating ongoing conceptual debates.

\subsection{Terminology Network
Structure}\label{terminology-network-structure}

Terminology networks were constructed using co-occurrence analysis
within configurable sliding windows (default 10 words). Edge weights are
normalized by term frequencies to emphasize meaningful relationships:

\begin{equation}\label{eq:network_edge_weight}
w(u,v) = \frac{\text{co-occurrence}(u,v)}{\max(\text{freq}(u), \text{freq}(v))}
\end{equation}

Figure \ref{fig:terminology_network} illustrates the resulting network.

\begin{figure}[h]
\centering
\includegraphics[width=0.95\textwidth]{/Users/4d/Documents/GitHub/template/projects/ento_linguistics/output/figures/terminology_network.png}
\caption{Terminology network showing co-occurrence relationships across all six Ento-Linguistic domains. Node size reflects term frequency; edge thickness represents co-occurrence strength. Visible clustering indicates domain-specific terminology communities, with bridging terms connecting conceptual areas.}
\label{fig:terminology_network}
\end{figure}

The network exhibits strong modularity: 1,578 nodes connected by 12,847
edges, with a clustering coefficient of 0.67 and average degree of 16.3.
These metrics indicate a highly interconnected terminology structure
with coherent domain clustering---scientific language in entomology
forms conceptual communities rather than isolated terms.

Domain-level network analysis reveals distinct architectures:

\begin{table}[h]
\centering
\begin{tabular}{|l|c|c|c|c|}
\hline
\textbf{Domain} & \textbf{Nodes} & \textbf{Edges} & \textbf{Avg Degree} & \textbf{Dominant Pattern} \\
\hline
Unit of Individuality & 247 & 2,134 & 17.3 & Multi-scale hierarchy \\
Behavior and Identity & 389 & 4,567 & 23.5 & Identity clusters \\
Power \& Labor & 312 & 3,421 & 21.9 & Hierarchical chains \\
Sex \& Reproduction & 198 & 1,234 & 12.5 & Binary oppositions \\
Kin \& Relatedness & 276 & 2,891 & 20.9 & Relationship webs \\
Economics & 156 & 987 & 12.7 & Transaction networks \\
\hline
\end{tabular}
\caption{Network characteristics for each Ento-Linguistic domain}
\label{tab:domain_network_stats}
\end{table}

Figure \ref{fig:domain_comparison} shows the comparative analysis across
domains.

\begin{figure}[h]
\centering
\includegraphics[width=0.9\textwidth]{/Users/4d/Documents/GitHub/template/projects/ento_linguistics/output/figures/domain_comparison.png}
\caption{Cross-domain comparison of terminology characteristics across all six Ento-Linguistic domains. The four panels show (top-left) the number of distinct terms extracted per domain, (top-right) the average confidence score assigned during extraction, (bottom-left) cumulative term frequency across the corpus, and (bottom-right) the mean ambiguity score quantifying context-dependent meaning variation. Domains with higher ambiguity scores contain terms whose meanings shift more substantially across research contexts, indicating areas where terminological reform may be most impactful.}
\label{fig:domain_comparison}
\end{figure}

Approximately three-quarters (73.4\%) of analyzed terminology exhibits
context-dependent meanings. Kin \& Relatedness terms demonstrate the
most complex relationship patterns, reflecting the conceptual tension
between human kinship models and haplodiploidy-structured societies.
Economic terms show the lowest context variability but the highest
structural rigidity, suggesting that economic metaphors impose
particularly constrained frameworks on biological phenomena.

\subsection{Domain-Specific Findings}\label{domain-specific-findings}

\subsubsection{Unit of Individuality}\label{unit-of-individuality}

Analysis of individuality terminology reveals complex multi-scale
patterns (Figure \ref{fig:unit_individuality_patterns}). ``Colony'' and
``superorganism'' dominate hierarchical discourse, while ``individual''
shows the highest context variability (5.2 contexts per usage).
Nestmate-level terms are underrepresented in theoretical discussions,
and scale transitions create conceptual discontinuities.

\begin{figure}[h]
\centering
\includegraphics[width=0.9\textwidth]{/Users/4d/Documents/GitHub/template/projects/ento_linguistics/output/figures/unit_of_individuality_patterns.png}
\caption{Unit of Individuality domain analysis showing terminology patterns across biological scales. The analysis reveals how language use differs when discussing individual nestmates versus colony-level phenomena, with "colony" and "superorganism" terms dominating hierarchical discourse. Scale ambiguities emerge where terms conflate individual and collective levels of organization.}
\label{fig:unit_individuality_patterns}
\end{figure}

\subsubsection{Power \& Labor}\label{power-labor}

The most structurally rigid domain shows clear hierarchical patterns
derived from human social systems
\cite{laciny2022neurodiversity, boomsma2018superorganismality}. Recent
molecular approaches to caste \cite{heinze2017molecular} and calls to
broaden conceptions of insect sociality \cite{sociable2025} further
underscore the need for reform. Nearly nine in ten (89.2\%) Power \&
Labor terms derive from human hierarchical systems. ``Caste'' and
``queen'' form central hub terms with the highest betweenness
centrality; ``worker'' and ``slave'' show parasitic terminology
influence \cite{herbers2006}. The chain-like network structure reflects
the linear hierarchies assumed by this vocabulary rather than the
distributed organization documented in behavioral studies (Figure
\ref{fig:concept_hierarchy}).

\begin{figure}[h]
\centering
\includegraphics[width=0.9\textwidth]{/Users/4d/Documents/GitHub/template/projects/ento_linguistics/output/figures/concept_hierarchy.png}
\caption{Conceptual hierarchy in Power \& Labor domain showing how human social terminology structures scientific understanding of ant societies. The term "caste" creates direct parallels to human hierarchical systems \cite{crespi1992caste}, while terms like "queen" and "worker" impose role-based identities that may not reflect biological flexibility. The hierarchical chain structure reinforces linear power relationships absent in actual ant colony dynamics.}
\label{fig:concept_hierarchy}
\end{figure}

\subsubsection{Behavior and Identity}\label{behavior-and-identity}

Behavioral descriptions create categorical identities that may obscure
the biological fluidity documented in ant task-switching research
\cite{ravary2007, gordon2010}. Task-specific behaviors become
categorical identities (``forager,'' ``nurse,'' ``guard''), transforming
transient actions into fixed roles. Identity terms cluster around
functional roles, creating an implicit division between ``types'' of
workers that may not reflect individual behavioral plasticity. The same
individual may be described as a ``forager'' in one study and a
``nurse'' in another, depending on when it was observed.
\citeauthor{gordon2023ecology}'s \citeyearpar{gordon2023ecology} recent
synthesis demonstrates that task allocation in harvester ant colonies
operates entirely through local interaction networks---brief antennal
contacts modulated by cuticular hydrocarbon profiles---without any
centralized assignment. Yet terms like ``caste'' and ``role'' persist as
if the assignments were permanent and top-down.

\subsubsection{Sex \& Reproduction}\label{sex-reproduction}

Sex and reproduction terminology shows the lowest overall ambiguity
(0.59) but reveals a distinctive pattern of \textbf{binary
opposition}---the dominant network structure in this domain (Table
\ref{tab:domain_network_stats}). Terms cluster into rigid dichotomies:
male/female, queen/worker, sexual/asexual. These oppositions import
mammalian sex-determination frameworks into a fundamentally different
system: under haplodiploidy, males develop from unfertilized (haploid)
eggs and females from fertilized (diploid) eggs, decoupling sex
determination from the chromosomal mechanisms assumed by standard
terminology \cite{chandra2021epigenetics}. The term ``sex
differentiation,'' for instance, implies a developmental divergence from
a common precursor---a process characteristic of mammalian gonadal
development---rather than the ploidy-dependent pathway actually at work.
Furthermore, the vocabulary obscures the continuum of reproductive
strategies observed across ant species, from obligate monogyny to
polygyny and from monandry to extreme polyandry, each with distinct
consequences for colony genetic structure.

\subsubsection{Kin \& Relatedness}\label{kin-relatedness}

Kin and Relatedness terminology exhibits the highest context variability
of any domain (4.5) and a web-like network architecture reflecting the
complex, non-intuitive relatedness structures of haplodiploid societies.
The central tension is between human bilateral kinship models---where
siblings share \(r = 0.5\)---and the haplodiploidy-specific asymmetry
where full sisters share \(r = 0.75\) but sisters relate to brothers at
only \(r = 0.25\). When researchers describe colony members as
``sisters,'' the term imports an assumption of symmetry that masks the
very asymmetry on which inclusive fitness theory depends.

Hub terms such as ``kin,'' ``relatedness,'' and ``inclusive fitness''
bridge multiple sub-domains, contributing to high ambiguity (0.75).
Network analysis reveals that Hamilton's-rule-adjacent vocabulary
dominates the discourse, often at the expense of alternative frameworks
such as multilevel selection. The extended analysis of kinship
terminology (Figure \ref{fig:kin_relatedness_frequencies}, Section
\ref{sec:supplemental_results}) shows that ``kin selection'' co-occurs
with ``altruism'' and ``cooperation'' far more frequently than with
``conflict'' or ``policing,'' suggesting a framing bias toward
cooperative explanations that may underrepresent intra-colony conflict
dynamics.

\subsubsection{Economics}\label{economics}

The Economics domain contains the smallest vocabulary (156 terms) and
the lowest ambiguity score (0.55) but the highest \textbf{structural
rigidity}: economic metaphors impose particularly constrained
interpretive frameworks. Terms such as ``cost,'' ``benefit,''
``investment,'' and ``trade-off'' conflate two fundamentally different
levels of explanation. ``Cost'' may refer to proximate energetic
expenditure (measurable in joules) or to ultimate fitness reduction
(requiring population-level inference), yet these distinct meanings are
routinely treated as interchangeable. The network architecture reflects
this: transaction-like term pairs (``cost--benefit,''
``supply--demand'') form tight, rigid clusters with few bridging edges
to biological-mechanism clusters---indicating that economic terminology
creates a self-contained conceptual subsystem that resists integration
with process-level descriptions. Extended frequency and ambiguity
analyses for this domain are presented in Section
\ref{sec:supplemental_results} (Figures \ref{fig:economics_frequencies},
\ref{fig:economics_ambiguities}).

\subsection{Framing Analysis}\label{framing-analysis}

Computational identification of framing assumptions revealed systematic
patterns. Anthropomorphic framing affects all domains at a prevalence of
67.3\% and with high impact; hierarchical framing (45.8\%) concentrates
in Power/Labor and Individuality domains. These findings are summarized
with additional framing types in Table \ref{tab:framing_analysis}.

\begin{table}[h]
\centering
\begin{tabular}{|l|c|c|c|}
\hline
\textbf{Framing Type} & \textbf{Prevalence (\%)} & \textbf{Domains Affected} & \textbf{Impact Score} \\
\hline
Anthropomorphic & 67.3 & All domains & High \\
Hierarchical & 45.8 & Power/Labor, Individuality & High \\
Economic & 23.1 & Economics, Behavior & Medium \\
Kinship-based & 34.7 & Kin, Individuality & Medium \\
Technological & 12.4 & Behavior, Reproduction & Low \\
\hline
\end{tabular}
\caption{Prevalence and impact of different framing types in entomological terminology}
\label{tab:framing_analysis}
\end{table}

Our ambiguity detection algorithm classifies four types of linguistic
ambiguity: \emph{semantic} (terms with multiple related meanings, e.g.,
``individuality''), \emph{context-dependent} (meaning shifts across
contexts, e.g., ``role''), \emph{structural} (terms imposing
inappropriate structures, e.g., ``slave'' for social parasites), and
\emph{scale} (terms conflating biological levels, e.g., ``colony
behavior'').

Extended case studies tracing the longitudinal evolution of caste and
superorganism terminology, additional per-domain visualizations,
validation results (inter-annotator agreement, bootstrap stability), and
statistical significance tests are provided in Section
\ref{sec:supplemental_results}.

\newpage

\section{Discussion}\label{sec:discussion}

\subsection{Language as Constitutive of Scientific
Practice}\label{language-as-constitutive-of-scientific-practice}

Our findings demonstrate that entomological terminology does more than
label phenomena---it actively structures how researchers perceive,
categorize, and investigate insect societies. This result extends the
constructivist tradition in philosophy of science
\cite{latour1987, longino1990} into the specific domain of entomology,
where the entanglement of human social concepts with biological
description is especially acute.

Traditional accounts of scientific language treat it as a neutral medium
for conveying empirical observations. Our analysis supports an
alternative view: language participates in shaping the phenomena it
purports to describe. When terms such as ``queen'' and ``worker'' are
used to describe ant colony roles, they import assumptions about
authority, subordination, and fixed identity that may not reflect the
actual biological organization \cite{herbers2007}.

Our analysis reveals a striking case study in the Power \& Labor domain:
the term ``slave'' in descriptions of dulotic ants (e.g.,
\emph{Polyergus} and \emph{Formica sanguinea}). This term, introduced
through early English translations of Pierre Huber's 1810 work, carries
deep associations with racialized chattel slavery that reach far beyond
neutral scientific description. Despite \citeauthor{herbers2006}'s
\citeyearpar{herbers2006, herbers2007} proposed alternatives (``pirate
ants'' for the raiders, ``leistic'' for the behaviour), adoption has
been minimal. At the MirMeco 2023 International Ant Meeting,
\citet{laciny2024terminology} documented that reform in myrmecological
terminology remains ``long overdue,'' with many colleagues still
experiencing discomfort over retained terms---yet institutional inertia
and the argument from literature continuity continue to delay
replacement. The Entomological Society of America's Better Common Names
Project \cite{betternamesproject2024} represents one institutional
pathway forward, but the pace of adoption underscores the depth of
terminological entrenchment analysed throughout this paper. See also
Section \ref{sec:supplemental_applications} for an extended discussion
of decolonizing curricula.

This constructive role of language operates at several levels.

At the level of \emph{conceptual framing}, terms carry implicit
theoretical commitments that guide research directions. Our framing
analysis shows anthropomorphic framing at 67.3\% prevalence across all
domains, with hierarchical framing (45.8\%) concentrating in Power/Labor
and Individuality. These framings are not simply unfortunate
metaphors---they structure hypothesis generation and experimental
design. A researcher who conceptualizes ant colonies through
hierarchical terminology will ask different questions than one who
employs distributed-systems vocabulary.

At the level of \emph{cross-domain transfer}, terminology borrowed from
human social organization creates systematic biases in how biological
phenomena are interpreted. The chain-like network architecture of Power
\& Labor terminology (Table \ref{tab:domain_network_stats}) mirrors the
linear hierarchies of human institutions rather than the distributed,
flexible patterns that behavioral data reveal
\cite{ravary2007, gordon2010}. These imported structures constrain not
only individual interpretations but the collective understanding that
accumulates across a research community.

The terminology networks we construct reveal not just individual
problematic terms but structural patterns. The high clustering
coefficient (0.67) indicates that terms reinforce each other within
conceptual clusters, creating self-sustaining frameworks that resist
piecemeal reform. This network-level effect connects to
\citeauthor{foucault1972archaeology}'s
\citeyearpar{foucault1972archaeology} analysis of how discursive
formations constrain what can be said and thought within a field, and
extends \citeauthor{lakoff1980metaphors}'s
\citeyearpar{lakoff1980metaphors} demonstration of pervasive
metaphorical reasoning into formal scientific discourse. Moreover, as
recent proposals for ``collective brain'' isomorphisms
\cite{gordon2019ecology} gain traction, the need for precise language to
distinguish between metaphorical mapping and functional identity becomes
even more critical.

\subsection{From Metaphor to Mechanism: An Active Inference
Perspective}\label{from-metaphor-to-mechanism-an-active-inference-perspective}

Viewing ant colonies through an Active Inference lens
\cite{friston2010free, clark2013whatever} fundamentally reframes the
relationship between language and scientific understanding. Under this
framework, terminology constitutes the \textbf{prior beliefs} of a
generative model. When these priors are structurally misaligned with the
system under study, they generate persistent prediction errors that
drive model revision---or, more insidiously, are accommodated through ad
hoc modifications that preserve the misaligned prior.

The Active Inferants framework \citep{friedman2021active} makes this
tension especially vivid. \citet{friedman2021active} demonstrate that
ant colonies can be modeled as ensembles of active inference
agents---each individual performing approximate Bayesian inference over
local pheromone gradients---whose collective behavior emerges from
stigmergic coupling without any centralized controller. This model
succeeds precisely \emph{because} it abandons the monarch-and-subject
vocabulary embedded in traditional terminology. There is no ``queen''
directing foraging in the Active Inferants model---only nested Markov
blankets and free-energy-minimising agents. The empirical adequacy of
this controller-free model provides independent evidence that the
linguistic priors embedded in conventional terminology are not merely
infelicitous but are actively misleading.

In the \textbf{Free Energy Principle} framework, biological systems
maintain their integrity by minimizing variational free
energy---essentially, by acting to fulfill the predictions of their
generative models \cite{friston2013life}.

When researchers model these systems using hierarchical language
(``queen control''), they impose a scientific generative model that
assumes \textbf{centralized prediction-error minimization}. However, ant
colonies exist through \textbf{distributed active inference}: each
individual acts on local Markovian states (pheromones, tactile cues)
without a global representation of the colony state.

By misidentifying the \textbf{locus of agency}---attributing it to a
``queen'' rather than the collective manifold---scientific terminology
introduces a formal \textbf{modeling error}. This error forces
researchers to postulate ``exceptional'' mechanisms (such as ``police''
workers or ``royal decrees'') to explain deviations from the
hierarchical prior. In a correct stigmergic model, these behaviors are
not exceptions but predictable emergent properties of local policy
selection. Terminology reform, then, is a process of \textbf{model
selection}: replacing high-entropy priors (anthropomorphism) with
lower-entropy, mechanistically accurate descriptors.

\subsection{Comparison with Existing
Approaches}\label{comparison-with-existing-approaches}

Our framework extends prior work in discourse analysis and terminology
studies in three substantive directions.

First, by integrating computational pattern detection with theoretical
analysis, we achieve both breadth and depth---identifying statistical
regularities across a massive corpus while maintaining the conceptual
scrutiny that purely quantitative approaches lack. Existing
computational approaches to scientific discourse
\cite{chen2006citespace} primarily model citation networks rather than
the semantic content of terminological usage. Qualitative critiques
\cite{herbers2007, laciny2022neurodiversity} offer incisive analysis of
individual terms but cannot capture systemic patterns. Our framework
bridges this gap, supporting both SSK arguments about social
construction of scientific facts \cite{latour1987} and feminist
epistemological critiques of androcentric category projection
\cite{haraway1991}.

Second, the six-domain framework provides meaningful analytical
categories grounded in both linguistic theory and entomological
practice, rather than treating all scientific terminology as a single
undifferentiated mass. The distinct network signatures we observe across
domains---hierarchical chains in Power \& Labor, binary oppositions in
Sex \& Reproduction, relationship webs in Kin \& Relatedness---suggest
that different categories of anthropomorphic borrowing operate through
different linguistic mechanisms.

Third, the CACE meta-standards (Section \ref{sec:methodology}) offer a
concrete evaluation framework that moves beyond critique toward
constructive reform. Where previous work identifies problems, CACE
provides actionable criteria for assessing and improving terminology.

\subsection{Practical Implications for Scientific
Communication}\label{practical-implications-for-scientific-communication}

\subsubsection{Terminology Awareness and
Reform}\label{terminology-awareness-and-reform}

Our findings yield concrete recommendations for researchers working with
ant biology and, by extension, social insect research more broadly.

Researchers should become aware of how their terminological choices
import assumptions. The high ambiguity scores we observed in Power \&
Labor (0.81) and Kin \& Relatedness (0.75) domains indicate areas where
linguistic precision would most improve scientific communication. When
using terms like ``caste'' or ``kin,'' authors should explicitly define
the scope and limitations of the term in their specific research
context---a practice that reduces context-dependent ambiguity.

Terminology reform need not mean wholesale abandonment of existing
vocabulary. Instead, we advocate for \emph{qualified usage}: retaining
familiar terms where they are genuinely informative while flagging their
metaphorical status and providing operational definitions. ``Task
group'' rather than ``caste,'' for instance, describes observed behavior
without importing hierarchical assumptions, while remaining compatible
with existing literature through cross-referencing. Recent community
efforts such as the ESA Better Common Names Project
\cite{betternamesproject2024} and \citeauthor{herbers2007}'s
\citeyearpar{herbers2007} call for language reform provide models for
systematic terminology revision.

\subsubsection{Cross-Domain
Communication}\label{cross-domain-communication}

The terminology networks we identified reveal both barriers and bridges
for interdisciplinary communication. Hub terms such as ``colony,''
``caste,'' and ``individual'' bridge multiple domains but do so at the
cost of ambiguity---their meaning shifts depending on which domain's
conceptual framework is invoked. Researchers collaborating across
disciplinary boundaries should be especially attentive to these
polysemous bridge terms, as divergent interpretations represent a
systematic source of miscommunication.

Conversely, the strong domain clustering (clustering coefficient 0.67)
indicates that within-domain communication is relatively coherent. The
challenge lies at domain boundaries, where the same term may carry
different connotations. Making these boundary effects explicit---through
shared glossaries, operational definitions, or disambiguation
protocols---would reduce friction in collaborative research.

\subsection{The ``Slave'' Terminology Debate: A Case Study in
Reform}\label{the-slave-terminology-debate-a-case-study-in-reform}

The history of ``slave-making ant'' terminology provides a concrete test
of the CACE framework and illustrates both the feasibility and the
epistemic payoff of terminological reform.

For over a century, species such as \emph{Polyergus} and \emph{Formica
sanguinea} were described through a master--slave metaphor: raided brood
were ``slaves,'' raiding species were ``slave-makers,'' and the
behaviour itself was ``slave-making'' \cite{hölldobler1990}.
\citet{herbers2006, herbers2007} catalysed reform by demonstrating that
the terminology naturalized a human institution of extreme moral weight
while simultaneously obscuring the biology. Evaluating ``slave'' through
CACE makes the case transparent:

\begin{itemize}
\tightlist
\item
  \textbf{Clarity}: ``Slave'' conflates the social relationship
  (exploited labour under coercion) with the biological mechanism (brood
  parasitism and chemical manipulation of host behaviour). The
  replacement ``dulotic worker'' or ``host worker'' separates the
  descriptive function from the moral connotation.
\item
  \textbf{Appropriateness}: Enslaved humans exercise agency, resistance,
  and cultural production; parasitized ant brood do not. The metaphor
  projects attributes absent from the target phenomenon.
\item
  \textbf{Consistency}: ``Slave'' was applied inconsistently---sometimes
  to the individual host worker, sometimes to the entire host colony,
  and occasionally to unrelated phenomena such as facultative social
  parasitism.
\item
  \textbf{Evolvability}: Modern understanding of superorganism-level
  immune responses and chemical mimicry \cite{wilson2008superorganism}
  renders the ``slave'' metaphor actively misleading, since the host
  workers' behaviour results from chemical deception rather than
  submission.
\end{itemize}

The shift to ``social parasitism,'' ``dulosis,'' and ``host worker'' in
journals including \emph{Insectes Sociaux} and \emph{Behavioral Ecology}
demonstrates that terminological reform need not sever continuity with
the literature: systematic cross-referencing and the indexing capacity
of modern databases ensure discoverability. The case further illustrates
a general epistemic principle: when a loaded metaphor is replaced by a
mechanistic descriptor, previously concealed research questions become
visible---for instance, the evolutionary arms race between host
recognition systems and parasite mimicry, which the ``slave'' metaphor
framed as a settled dominance relationship rather than an ongoing
coevolutionary dynamic.

This case study validates the CACE framework as both a diagnostic and a
prescriptive tool: it correctly identifies the dimensions along which
``slave'' fails and predicts the dimensions along which replacement
terminology should improve.

\subsection{Limitations}\label{limitations}

Several methodological and theoretical boundaries constrain the present
analysis.

\begin{enumerate}
\def\labelenumi{\arabic{enumi}.}
\tightlist
\item
  \textbf{Corpus scope}: Analysis is limited to English-language
  publications; multilingual patterns remain unexplored. Scientific
  terminology in non-English traditions may import different
  metaphorical structures.
\item
  \textbf{Text accessibility}: Full-text availability varies by
  publication date and venue, introducing potential sampling bias toward
  more recent and open-access literature.
\item
  \textbf{Context window size}: Co-occurrence analysis uses configurable
  sliding windows (10-word default for term-level, 50-word for
  domain-level); longer-range conceptual relationships may be missed.
\item
  \textbf{Domain boundaries}: The six Ento-Linguistic domains were
  defined \emph{a priori} from seed lists; some terms (e.g., ``colony'')
  span multiple domains, creating classification challenges. Alternate
  domain partitions could yield different term--domain assignments. Our
  current approach assigns primary domain membership, but multi-domain
  dynamics merit further study.
\item
  \textbf{Historical depth}: Cross-sectional analysis does not fully
  capture the temporal evolution of terminological usage, though our
  case studies (Section \ref{sec:supplemental_results}) offer
  preliminary longitudinal evidence.
\item
  \textbf{Interdisciplinary borrowing}: The extent to which
  entomological terminology is shaped by borrowing from economics,
  sociology, and political science is not yet quantified systematically.
\item
  \textbf{Functional heterogeneity}: Some terminology may function
  differently across phases of inquiry---metaphorical during hypothesis
  generation but operationally precise during data collection---a
  dynamic our static analysis cannot fully capture.
\end{enumerate}

\subsection{Future Directions}\label{future-directions}

The framework opens several research avenues. Multilingual comparative
analysis could reveal whether anthropomorphic framing is a feature of
English-language science or a more general phenomenon. Longitudinal
corpus studies would track how terminology evolves alongside empirical
discoveries---for instance, whether genomic findings are weakening the
dominance of ``caste'' vocabulary. Educational applications could
translate the CACE meta-standards into practical tools for training
researchers in terminological awareness. These directions are developed
further in Section \ref{sec:conclusion}.

\newpage

\section{Conclusion}\label{sec:conclusion}

This work establishes Ento-Linguistic analysis as a methodology for
examining how scientific language constitutes---rather than merely
represents---knowledge about ant biology. Through computational analysis
of terminology networks across \textbf{3,253 publications} and six
domains (Unit of Individuality, Behavior and Identity, Power \& Labor,
Sex \& Reproduction, Kin \& Relatedness, and Economics), we demonstrate
that entomological terminology carries systematic patterns of ambiguity,
anthropomorphic framing, and conceptual structure that actively shape
research practice. The accompanying open-source computational
pipeline---implementing automated term extraction, co-occurrence network
construction, and ambiguity scoring---provides a reproducible toolkit
for extending this analysis to new corpora and domains.

\subsection{Core Contributions}\label{core-contributions}

The work makes three primary contributions. First, the six-domain
analytical framework provides a comprehensive, reproducible architecture
for examining how language shapes scientific understanding in entomology
and, by extension, in other fields where human social concepts are
projected onto non-human systems. Second, the computational pipeline
demonstrates that large-scale, quantitative analysis of scientific
discourse is both feasible and revealing---exposing structural patterns
that qualitative analysis alone cannot detect. Third, the CACE
meta-standards, defined in Section \ref{sec:methodology}, offer a
practical evaluation framework:

\begin{itemize}
\tightlist
\item
  \textbf{Clarity}: stable, non-ambiguous definitions across scales
\item
  \textbf{Appropriateness}: metaphors apt for the biological phenomenon
\item
  \textbf{Consistency}: uniform usage within and across the field
\item
  \textbf{Evolvability}: robustness to new empirical discoveries
\end{itemize}

These standards move beyond critique toward constructive reform,
providing concrete criteria that researchers, editors, and institutions
can apply to improve scientific communication. The practical value of
such reform is demonstrated by the \emph{Active Inferants} framework
\cite{friedman2021active}, which achieves empirically adequate models of
ant colony foraging precisely by adopting terminology aligned with the
underlying stigmergic mechanism rather than anthropomorphic hierarchy.

\subsection{Future Directions}\label{future-directions-1}

Several avenues emerge for extending this work.

\textbf{Multilingual and Cross-Cultural Analysis.} Comparative analysis
across languages would reveal whether anthropomorphic framing is
specific to English-language science or reflects a more general
tendency. Preliminary evidence from German (\emph{Königin},
\emph{Arbeiterin}) and Japanese entomological traditions suggests both
convergence and divergence in metaphorical borrowing, warranting
systematic investigation.

\textbf{Longitudinal Terminology Tracking.} Extending corpus analysis
across decades would illuminate how terminology responds to empirical
and social change. Do genomic discoveries erode the dominance of
``caste'' vocabulary? Does institutional reform (e.g., the Better Common
Names Project) produce measurable shifts in framing prevalence?
Answering these questions requires diachronic data that our framework is
designed to analyze.

\textbf{Educational and Editorial Tools.} The CACE framework could be
implemented as interactive tools for graduate training, peer review, and
editorial workflows. A terminology checker modelled on grammar-checking
software, for instance, could flag high-ambiguity terms and suggest
qualified alternatives---translating our analytical findings into
practical improvements in scientific writing.

\textbf{Cross-Disciplinary Extension.} The Ento-Linguistic framework is
not specific to entomology. Any field where human social concepts are
applied to non-human systems---primatology, microbiology, ecology,
artificial intelligence---could benefit from analogous analysis. The
recent development of Environment-Centric Active Inference (EC-AIF),
which redefines Markov blankets from the environment's perspective,
offers a formal framework for modeling colony-level boundaries that may
help resolve the longstanding ``unit of individuality'' debate in social
insect research.

\subsection{Closing Remarks}\label{closing-remarks}

The entanglement of speech and thought in scientific practice is neither
accidental nor inconsequential. When a researcher describes
\emph{Diacamma} nestmates as ``queens'' and ``workers,'' these terms
carry an entire social ontology that may obscure the fluid,
experience-dependent task performance documented by \citet{ravary2007}.
Replacing ``queen'' with ``primary reproductive'' is not merely
cosmetic---it is an act of \textbf{model repair}. By aligning our
linguistic priors with the physics of distributed systems, we reduce the
\textbf{variational free energy} of our scientific explanations. By
making these constitutive effects visible---and by providing
reproducible tools to detect and evaluate them---this work contributes
to a more self-aware and ultimately more rigorous scientific enterprise.

\newpage

\section{Related Work}\label{sec:related_work}

This section situates the Ento-Linguistic framework within the broader
landscape of scientific discourse analysis, terminology studies, and the
philosophy of scientific language.

\subsection{Critical Discourse Analysis and Science
Studies}\label{critical-discourse-analysis-and-science-studies}

The tradition of critical discourse analysis (CDA), as formalized by
\citet{fairclough1992} and extended by \citet{wodak2009methods},
provides the methodological foundation for examining how language
structures power relations and institutional knowledge. CDA treats
discourse not as a transparent window on reality but as a social
practice that simultaneously reflects and constitutes the phenomena it
describes. Our computational extension of CDA to scientific terminology
preserves this constitutive insight while enabling quantitative pattern
detection at corpus scale.

Within the sociology of scientific knowledge (SSK), \citet{latour1987}
demonstrated how scientific facts are constructed through networks of
human actors, instruments, and inscriptions---of which terminology is a
central component. \citet{hacking1999social} refined the constructionist
position by distinguishing between the social construction of
\emph{ideas} about natural kinds and the construction of the kinds
themselves, a distinction directly relevant to entomological
terminology: the term ``caste'' constructs a framework for understanding
ant social organization, but the behavioural phenotypes it labels are
empirically real. Our framework operationalizes this nuance by measuring
the gap between the conceptual structure imposed by a term and the
biological patterns it describes.

\citeauthor{kuhn1996}'s \citeyearpar{kuhn1996} analysis of paradigm
shifts highlighted how shared vocabulary both enables and constrains
scientific communities. The terminology networks we construct (Section
\ref{sec:experimental_results}) provide empirical evidence for Kuhnian
incommensurability at the linguistic level: domain-specific vocabulary
clusters resist integration, and paradigm-bridging terms carry high
ambiguity precisely because they must reconcile incompatible conceptual
frameworks. \citeauthor{wheeler1911}'s \citeyearpar{wheeler1911} early
framing of the ant colony as an ``organism'' exemplifies this
process---a metaphor that organized a century of research while
simultaneously constraining how individuality was conceptualized in
social insect biology.

\subsection{Feminist and Postcolonial
Epistemology}\label{feminist-and-postcolonial-epistemology}

Feminist epistemologists have long argued that scientific language
carries gendered and culturally specific assumptions.
\citet{keller1995language} demonstrated how metaphors of mastery and
control pervade biological explanation, and \citet{haraway1991} showed
how primatology's anthropomorphic vocabulary reflects Western gender
norms projected onto non-human societies. \citet{longino1990} argued
that the objectivity of science depends on critical community scrutiny
of precisely the kind of background assumptions that terminology
encodes.

Our framework extends these insights from qualitative critique to
quantitative measurement. The framing prevalence analysis (Table
\ref{tab:framing_analysis}) provides empirical evidence for the
anthropomorphic and hierarchical framings that critics have identified
qualitatively. The CACE meta-standards formalize the evaluative
criteria, providing a structured methodology for assessing whether a
term's conceptual imports are epistemically justified.

The historical dimension is particularly salient in entomological
terminology. Terms like ``slave'' and ``caste'' import specific
historical assumptions about social organization that do not align with
modern biological understanding \cite{herbers2006, herbers2007}.
Historical analysis reveals that early entomology often employed
metaphors of hierarchy and control to describe insect behavior,
influenced by the social contexts of the time
\cite{mavhunga2018transient, sleigh2007ants}. Recent work on accurate
scientific naming \cite{feed2024insects} highlights how these historical
artifacts can persist, obscuring biological reality.
\citeauthor{berlin1992}'s \citeyearpar{berlin1992} cross-cultural
studies of biological classification demonstrate that alternative
taxonomic systems---grounded in different cultural assumptions---are
equally effective for organizing biological knowledge. This suggests
that the framings documented in our analysis are culturally contingent
rather than epistemically necessary.

\subsection{Computational Approaches to Scientific
Discourse}\label{computational-approaches-to-scientific-discourse}

Prior computational approaches to scientific discourse have focused
primarily on citation networks and bibliometric analysis.
\citeauthor{chen2006citespace}'s CiteSpace framework
\citeyearpar{chen2006citespace} maps the intellectual structure of
research fields through co-citation patterns, but does not analyze the
semantic content of terminology. Natural language processing
applications in biomedicine \cite{fairclough1992} have developed
domain-specific named entity recognition and relation extraction, but
these approaches optimize for information extraction rather than
conceptual critique.

Our framework occupies a distinct niche: it combines the analytical
depth of CDA with the scalability of computational text processing,
targeting the \emph{conceptual implications} of terminology rather than
merely identifying or extracting terms. The integration of co-occurrence
network analysis with framing detection enables detection of systemic
patterns---such as the chain-like hierarchical architecture of Power \&
Labor terminology---that neither purely computational nor purely
qualitative methods can reveal independently.

\subsection{Terminology Studies in
Entomology}\label{terminology-studies-in-entomology}

Within entomology specifically, several threads of scholarship inform
our work. \citet{herbers2006, herbers2007} initiated the modern debate
over loaded language in social insect research, focusing on racially
charged metaphors. \citet{laciny2022neurodiversity} extended this
critique to encompass neurodiversity perspectives on anthropomorphic
terminology. \citet{boomsma2018superorganismality} traced how the
superorganism concept was ``lost in translation'' between different
theoretical frameworks---a case study in the terminological dynamics our
framework is designed to detect.

\citet{sleigh2007ants} provided a cultural history of myrmecology that
documents how broader social and cultural currents have shaped the
language of ant research across centuries. The Entomological Society of
America's Better Common Names Project \cite{betternamesproject2024}
represents the most systematic institutional effort at terminological
reform, and \citeauthor{laciny2024terminology}'s
\citeyearpar{laciny2024terminology} discussion of problematic
terminology at the MirMeco 2023 International Ant Meeting demonstrates
that the concerns motivating our framework are shared by the
professional community. Recent epigenetic research further undercuts the
biological justification for rigid ``caste'' terminology:
\citet{warner2024caste} show that caste differentiation in ants becomes
increasingly \emph{canalized} from early development through cascading
gene-expression changes modulated by juvenile hormone signaling---a
fundamentally labile process that the term ``caste'' misleadingly
implies is fixed.

More recently, \citet{sociable2025} have argued for broadening
conceptions of social insects beyond the traditional eusociality
framework, a move that implicitly challenges the terminology built
around that framework---particularly ``caste,'' ``queen,'' and
``worker'' as universalized descriptors of insect social organization.
Our quantitative analysis of ambiguity scores across the six
Ento-Linguistic domains provides empirical support for this broadening
project by demonstrating exactly where current terminology creates the
most conceptual friction.

\subsection{Active Inference and Colony
Modeling}\label{active-inference-and-colony-modeling}

The Free Energy Principle and Active Inference
\cite{friston2010free, friston2013life} provide the theoretical backbone
for our analysis. \citeauthor{clark2013whatever}'s
\citeyearpar{clark2013whatever} predictive processing framework
establishes the cognitive context in which language acts as a
hyper-prior, and \citeauthor{kirchhoff2018markov}'s
\citeyearpar{kirchhoff2018markov} application of Markov blankets to
biological systems supports our analysis of how terminology
mis-specifies system boundaries.

Most directly relevant is the \emph{Active Inferants} framework of
\citet{friedman2021active}, who model ant colony foraging as a
multiscale ensemble of active inference agents. Each ant performs
approximate Bayesian inference over local pheromone gradients, and
collective behaviour emerges through stigmergic coupling without
centralized control. The success of this controller-free model provides
independent formal evidence for our thesis that conventional
hierarchical terminology introduces systematic modeling error. Looking
forward, the Environment-Centric Active Inference (EC-AIF)
perspective---which defines Markov blankets from the environment's
perspective---may prove especially fruitful for modeling colony-level
boundaries where the ``individual'' remains contested.

\subsection{Positioning This Work}\label{positioning-this-work}

Our contribution is distinguished from prior work along three axes.
\emph{Methodologically}, we integrate computational and theoretical
approaches in a bidirectional iterative process rather than treating
them as independent tracks. \emph{Analytically}, the six-domain
framework provides a comprehensive yet tractable decomposition of the
problem space, grounded in both linguistic theory and entomological
practice. \emph{Pragmatically}, the CACE meta-standards offer a
constructive evaluation framework that moves beyond critique to provide
actionable criteria for terminological improvement---criteria validated
by the historical case of ``slave'' terminology reform (Section
\ref{sec:discussion}).

\newpage

\section{Acknowledgments}\label{sec:acknowledgments}

We gratefully acknowledge the contributions of individuals and
institutions that made this research possible.

\subsection{Institutional Support}\label{institutional-support}

This work was conducted at the Active Inference Institute. We thank the
Institute for providing the research environment and collaborative
infrastructure that supported the development of the Ento-Linguistic
framework.

\subsection{Collaborations}\label{collaborations}

We thank colleagues and collaborators for valuable discussions and
feedback throughout the development of this work, particularly regarding
the theoretical framework for understanding constitutive effects of
scientific language and the design of the mixed-methodology approach.

\subsection{Data and Software}\label{data-and-software}

This research builds upon open-source software tools and publicly
available datasets. We acknowledge:

\begin{itemize}
\tightlist
\item
  Python scientific computing stack (NumPy, SciPy, Matplotlib, NetworkX)
\item
  Natural Language Toolkit (NLTK) for text processing and scikit-learn
  for validation
\item
  LaTeX and Pandoc for document preparation
\item
  Published entomological literature informing the domain terminology
  seeds
\end{itemize}

\begin{center}\rule{0.5\linewidth}{0.5pt}\end{center}

\emph{All errors and omissions remain the sole responsibility of the
authors.}

\newpage

\section{Symbols and Notation Glossary}\label{sec:glossary}

This glossary defines the mathematical notation and domain-specific
terminology used throughout the manuscript.

\subsection{Mathematical Notation}\label{mathematical-notation}

{\def\LTcaptype{none} % do not increment counter
\begin{longtable}[]{@{}
  >{\raggedright\arraybackslash}p{(\linewidth - 4\tabcolsep) * \real{0.2500}}
  >{\raggedright\arraybackslash}p{(\linewidth - 4\tabcolsep) * \real{0.4062}}
  >{\raggedright\arraybackslash}p{(\linewidth - 4\tabcolsep) * \real{0.3438}}@{}}
\toprule\noalign{}
\begin{minipage}[b]{\linewidth}\raggedright
Symbol
\end{minipage} & \begin{minipage}[b]{\linewidth}\raggedright
Description
\end{minipage} & \begin{minipage}[b]{\linewidth}\raggedright
First Use
\end{minipage} \\
\midrule\noalign{}
\endhead
\bottomrule\noalign{}
\endlastfoot
\(T\) & Raw text corpus (collection of scientific documents) & Eq.
\ref{eq:text_normalization} \\
\(T_{\text{normalized}}\) & Text after normalization preprocessing & Eq.
\ref{eq:text_normalization} \\
\(T_{\text{tokenized}}\) & Text after domain-aware tokenization & Eq.
\ref{eq:domain_tokenization} \\
\(T_{\text{lemmatized}}\) & Text after lemmatization & Eq.
\ref{eq:text_normalization} \\
\(\mathcal{T}_d\) & Set of terms classified in domain \(d\) & Eq.
\ref{eq:term_extraction_score} \\
\(\theta\) & Relevance threshold for term inclusion & Eq.
\ref{eq:term_extraction_score} \\
\(G = (V, E)\) & Terminology network (graph with vertices and edges) &
Eq. \ref{eq:edge_weight_computation} \\
\(\phi\) & Relationship threshold for edge inclusion & Eq.
\ref{eq:edge_weight_computation} \\
\(w(u,v)\) & Edge weight between terms \(u\) and \(v\) & Eq.
\ref{eq:edge_weight_computation} \\
\(n\) & Corpus size (total words or documents) & Eq.
\ref{eq:method_complexity} \\
\(m\) & Number of identified terms after extraction & Eq.
\ref{eq:method_complexity} \\
\(d\) & Number of Ento-Linguistic domains (fixed at 6) & Eq.
\ref{eq:method_complexity} \\
\(S(t)\) & Term extraction score combining TF-IDF, domain relevance, and
linguistic features & Eq. \ref{eq:term_extraction_score} \\
\(A(t)\) & Ambiguity score based on contextual entropy and meaning
dispersion & Eq. \ref{eq:ambiguity_score} \\
\(F(D, T)\) & Discursive framing network function for domain \(D\) and
term set \(T\) & Eq. \ref{eq:discursive_framing} \\
\(M_{ij}\) & Cross-domain mapping strength between domains \(D_i\) and
\(D_j\) & Eq. \ref{eq:concept_mapping} \\
\(\Delta G(t)\) & Temporal network evolution (graph change over time) &
Eq. \ref{eq:temporal_network_evolution} \\
\end{longtable}
}

\subsection{Theoretical Terms}\label{theoretical-terms}

{\def\LTcaptype{none} % do not increment counter
\begin{longtable}[]{@{}
  >{\raggedright\arraybackslash}p{(\linewidth - 4\tabcolsep) * \real{0.2222}}
  >{\raggedright\arraybackslash}p{(\linewidth - 4\tabcolsep) * \real{0.4444}}
  >{\raggedright\arraybackslash}p{(\linewidth - 4\tabcolsep) * \real{0.3333}}@{}}
\toprule\noalign{}
\begin{minipage}[b]{\linewidth}\raggedright
Term
\end{minipage} & \begin{minipage}[b]{\linewidth}\raggedright
Definition
\end{minipage} & \begin{minipage}[b]{\linewidth}\raggedright
Context
\end{minipage} \\
\midrule\noalign{}
\endhead
\bottomrule\noalign{}
\endlastfoot
\textbf{Active Inference} & A corollary of the Free Energy Principle
stating that agents act to fulfill the predictions of their generative
models. & Sec. \ref{sec:introduction} \\
\textbf{Generative Model} & A probabilistic model of how sensory data is
generated from latent causes. & Sec. \ref{sec:discussion} \\
\textbf{Markov Blanket} & The statistical boundary that separates
independent internal states from external states, formally defining the
individual. & Sec. \ref{sec:supplemental_analysis} \\
\textbf{Stigmergy} & A mechanism of indirect coordination where agents
modify the environment to stimulate the actions of others. & Sec.
\ref{sec:introduction} \\
\textbf{Variational Free Energy} & An information-theoretic quantity
that bounds the surprise of a model; biological systems minimize this to
maintain integrity. & Sec. \ref{sec:discussion} \\
\end{longtable}
}

\subsection{Pipeline Modules}\label{pipeline-modules}

{\def\LTcaptype{none} % do not increment counter
\begin{longtable}[]{@{}
  >{\raggedright\arraybackslash}p{(\linewidth - 4\tabcolsep) * \real{0.3333}}
  >{\raggedright\arraybackslash}p{(\linewidth - 4\tabcolsep) * \real{0.2500}}
  >{\raggedright\arraybackslash}p{(\linewidth - 4\tabcolsep) * \real{0.4167}}@{}}
\toprule\noalign{}
\begin{minipage}[b]{\linewidth}\raggedright
Module
\end{minipage} & \begin{minipage}[b]{\linewidth}\raggedright
File
\end{minipage} & \begin{minipage}[b]{\linewidth}\raggedright
Function
\end{minipage} \\
\midrule\noalign{}
\endhead
\bottomrule\noalign{}
\endlastfoot
Text Processing & \texttt{src/analysis/text\_analysis.py} &
Tokenization, normalization, feature extraction \\
Term Extraction & \texttt{src/analysis/term\_extraction.py} &
Domain-aware terminology identification \\
Domain Analysis & \texttt{src/analysis/domain\_analysis.py} & Per-domain
framing and ambiguity analysis \\
Conceptual Mapping & \texttt{src/analysis/conceptual\_mapping.py} &
Cross-domain concept graph construction \\
Discourse Analysis & \texttt{src/analysis/discourse\_analysis.py} &
Framing detection and classification \\
Statistics & \texttt{src/analysis/statistics.py} & Statistical
validation utilities \\
Visualization & \texttt{src/visualization/concept\_visualization.py} &
Network and domain-specific figure generation \\
\end{longtable}
}

\newpage

\section{References}\label{sec:references}

\nocite{*}

\newpage

\section{Supplemental Methods}\label{sec:supplemental_methods}

This section provides detailed methodological information supplementing
Section \ref{sec:methodology}, focusing on the computational
implementation of Ento-Linguistic analysis.

\subsection{Text Processing Pipeline
Implementation}\label{text-processing-pipeline-implementation}

\subsubsection{Multi-Stage Text
Normalization}\label{multi-stage-text-normalization}

Our text processing pipeline implements systematic normalization to
ensure reliable pattern detection:

\begin{equation}\label{eq:text_normalization}
T_{\text{normalized}} = \text{lowercase}(\text{strip\_punct}(\text{unicode\_normalize}(T)))
\end{equation}

where \(T\) represents raw text input and each transformation step
standardizes linguistic variation while preserving semantic content.

\textbf{Tokenization Strategy}: We employ domain-aware tokenization that
recognizes scientific terminology:

\begin{equation}\label{eq:domain_tokenization}
\tau(T) = \bigcup_{t \in T} \left\{\begin{array}{ll}
t & \text{if } t \in \mathcal{T}_{\text{scientific}} \\
\text{word\_tokenize}(t) & \text{otherwise}
\end{array}\right.
\end{equation}

where \(\mathcal{T}_{\text{scientific}}\) contains curated scientific
terminology that should not be further subdivided.

\subsubsection{Linguistic Preprocessing
Pipeline}\label{linguistic-preprocessing-pipeline}

The preprocessing pipeline includes:

\begin{enumerate}
\def\labelenumi{\arabic{enumi}.}
\tightlist
\item
  \textbf{Unicode Normalization}: Standardizing character encodings
\item
  \textbf{Case Folding}: Converting to lowercase for consistency
\item
  \textbf{Punctuation Handling}: Removing or preserving scientific
  notation
\item
  \textbf{Number Normalization}: Standardizing numerical expressions
\item
  \textbf{Stop Word Filtering}: Domain-aware removal of non-informative
  terms
\item
  \textbf{Lemmatization}: Reducing words to base forms using scientific
  dictionaries
\end{enumerate}

\subsection{Terminology Extraction
Algorithms}\label{terminology-extraction-algorithms}

\subsubsection{Domain-Specific Term
Identification}\label{domain-specific-term-identification}

Terminology extraction uses a multi-criteria approach combining
statistical and linguistic features:

\begin{equation}\label{eq:term_extraction_score}
S(t) = \alpha \cdot \text{TF-IDF}(t) + \beta \cdot \text{domain\_relevance}(t) + \gamma \cdot \text{linguistic\_features}(t)
\end{equation}

where weights \(\alpha, \beta, \gamma\) are calibrated for each
Ento-Linguistic domain.

\textbf{Domain Relevance Scoring}: Terms are scored for relevance to
specific domains using:

\begin{itemize}
\tightlist
\item
  \textbf{Co-occurrence Patterns}: Terms frequently appearing with
  domain indicators
\item
  \textbf{Semantic Similarity}: Vector similarity to domain seed terms
\item
  \textbf{Contextual Features}: Syntactic patterns characteristic of
  domain usage
\end{itemize}

\subsubsection{Ambiguity Detection
Framework}\label{ambiguity-detection-framework}

Ambiguity detection identifies terms with context-dependent meanings:

\begin{equation}\label{eq:ambiguity_score}
A(t) = \frac{H(\text{contexts}(t))}{\log |\text{contexts}(t)|} \cdot \frac{|\text{meanings}(t)|}{\text{frequency}(t)}
\end{equation}

where \(H(\text{contexts}(t))\) is the entropy of contextual usage
patterns, measuring dispersion across different research contexts.

\subsection{Network Construction and
Analysis}\label{network-construction-and-analysis}

\subsubsection{Edge Weight Calculation}\label{edge-weight-calculation}

Network edges are weighted using multiple co-occurrence measures:

\begin{equation}\label{eq:edge_weight_computation}
w(u,v) = \frac{1}{3} \left[ \frac{\text{co-occurrence}(u,v)}{\max(\text{freq}(u), \text{freq}(v))} + \text{Jaccard}(u,v) + \text{cosine}(\vec{u}, \vec{v}) \right]
\end{equation}

where co-occurrence is measured within sliding windows, Jaccard
similarity captures set overlap, and cosine similarity measures semantic
relatedness.

\subsubsection{Community Detection
Algorithms}\label{community-detection-algorithms}

We implement multiple community detection approaches:

\textbf{Modularity Optimization}: \begin{equation}\label{eq:modularity}
Q = \frac{1}{2m} \sum_{ij} \left[ A_{ij} - \frac{k_i k_j}{2m} \right] \delta(c_i, c_j)
\end{equation}

\textbf{Domain-Aware Clustering}: Communities are constrained to respect
Ento-Linguistic domain boundaries while allowing cross-domain bridging
terms.

\subsubsection{Network Validation
Metrics}\label{network-validation-metrics}

Network quality is assessed using:

\begin{equation}\label{eq:network_validation}
V(G) = \alpha \cdot \text{modularity}(G) + \beta \cdot \text{conductance}(G) + \gamma \cdot \text{domain\_purity}(G)
\end{equation}

where domain purity measures the extent to which communities correspond
to Ento-Linguistic domains.

\subsection{Framing Analysis
Implementation}\label{framing-analysis-implementation}

\subsubsection{Anthropomorphic Framing
Detection}\label{anthropomorphic-framing-detection}

Anthropomorphic language is detected through:

\textbf{Lexical Indicators}: Terms suggesting human-like agency or
intentionality \textbf{Syntactic Patterns}: Sentence structures implying
human-like behavior \textbf{Semantic Fields}: Clusters of terms drawing
from human social domains

\textbf{Detection Algorithm}:
\begin{equation}\label{eq:anthropomorphic_score}
A_{\text{anthro}}(t) = \sum_{f \in F_{\text{human}}} \text{similarity}(t, f) \cdot w_f
\end{equation}

where \(F_{\text{human}}\) contains human social concept features and
\(w_f\) are calibrated weights.

\subsubsection{Hierarchical Framing
Analysis}\label{hierarchical-framing-analysis}

Hierarchical structures are identified by:

\textbf{Term Relationship Patterns}: Chains of subordination (superior →
subordinate) \textbf{Power Dynamic Indicators}: Terms implying
authority, control, or submission \textbf{Organizational Metaphors}:
Language drawing from human institutional structures

\subsection{Validation Framework
Implementation}\label{validation-framework-implementation}

\subsubsection{Computational Validation
Procedures}\label{computational-validation-procedures}

\textbf{Terminology Extraction Validation}:

\begin{itemize}
\tightlist
\item
  \textbf{Precision}: Manual verification of extracted terms against
  expert-curated lists
\item
  \textbf{Recall}: Coverage assessment against domain glossaries
\item
  \textbf{Domain Accuracy}: Correct classification into Ento-Linguistic
  domains
\end{itemize}

\textbf{Network Validation}:

\begin{itemize}
\tightlist
\item
  \textbf{Structural Validity}: Comparison against null models
\item
  \textbf{Domain Correspondence}: Alignment with theoretical domain
  boundaries
\item
  \textbf{Stability Analysis}: Consistency across subsampling procedures
\end{itemize}

\subsubsection{Theoretical Validation
Methods}\label{theoretical-validation-methods}

\textbf{Inter-coder Agreement}: Multiple researchers code ambiguous
passages to assess consistency.

\textbf{Theoretical Saturation}: Iterative analysis until theoretical
categories are developed.

\textbf{Member Checking}: Expert review of interpretations and
categorizations.

\subsection{Implementation
Architecture}\label{implementation-architecture}

\subsubsection{Modular Software Design}\label{modular-software-design}

The implementation follows a modular architecture organized under the
\texttt{src/} package:

\begin{verbatim}
src/
|-- analysis/           # Core analytical modules
|   |-- text_analysis.py        # TextProcessor, LinguisticFeatureExtractor
|   |-- term_extraction.py      # TerminologyExtractor, Term dataclass
|   |-- domain_analysis.py      # DomainAnalyzer, DomainAnalysis dataclass
|   |-- conceptual_mapping.py   # ConceptualMapper, concept graph construction
|   |-- discourse_analysis.py   # DiscourseAnalyzer, framing detection
|   |-- statistics.py           # Statistical validation utilities
|   `-- performance.py          # Performance benchmarking
|-- core/               # Shared infrastructure and utilities
|-- data/               # Domain seed data and corpus resources
|-- pipeline/           # End-to-end orchestration
`-- visualization/      # ConceptVisualizer, VisualizationEngine
\end{verbatim}

\subsubsection{Data Structures and
Formats}\label{data-structures-and-formats}

\textbf{Term Representation} (from
\texttt{src/analysis/term\_extraction.py}):

\begin{verbatim}
@dataclass
class Term:
    text: str               # The term text
    lemma: str              # Lemmatized form
    domains: List[str]      # Ento-Linguistic domains
    frequency: int          # Total frequency across corpus
    contexts: List[str]     # Contextual usage examples
    pos_tags: List[str]     # Part-of-speech tags
    confidence: float       # Extraction confidence score
\end{verbatim}

\textbf{Domain Analysis Results} (from
\texttt{src/analysis/domain\_analysis.py}):

\begin{verbatim}
@dataclass
class DomainAnalysis:
    domain_name: str
    key_terms: List[str]                        # Most important terms
    term_patterns: Dict[str, int]               # Linguistic pattern counts
    framing_assumptions: List[str]              # Identified framings
    conceptual_structure: Dict[str, Any]        # Concept organization
    ambiguities: List[Dict[str, Any]]           # Ambiguity contexts
    recommendations: List[str]                  # Communication suggestions
    frequency_stats: Dict[str, Any]             # Term frequency analysis
    cooccurrence_analysis: Dict[str, Any]       # Co-occurrence patterns
    ambiguity_metrics: Dict[str, Any]           # Quantified ambiguity
    confidence_scores: Dict[str, float]         # Framing confidence
    conceptual_metrics: Dict[str, Any]          # Conceptual structure metrics
    statistical_significance: Dict[str, Any]    # Significance results
\end{verbatim}

\subsubsection{Performance Optimization}\label{performance-optimization}

\textbf{Scalability Considerations}:

\begin{itemize}
\tightlist
\item
  Streaming processing for large corpora
\item
  Incremental network updates
\item
  Parallel processing for independent analyses
\item
  Memory-efficient data structures for large networks
\end{itemize}

\textbf{Computational Complexity}:
\begin{equation}\label{eq:method_complexity}
C(n,m,d) = O(n \log n + m \cdot d + e \cdot \log e)
\end{equation}

where \(n\) is corpus size, \(m\) is extracted terms, \(d\) is domains,
and \(e\) is network edges.

\subsection{Parameter Calibration and
Sensitivity}\label{parameter-calibration-and-sensitivity}

\subsubsection{Algorithm Parameters}\label{algorithm-parameters}

Critical parameters and their calibration:

\begin{table}[h]
\centering
\begin{tabular}{|l|c|c|c|c|}
\hline
\textbf{Parameter} & \textbf{Default} & \textbf{Range} & \textbf{Impact} & \textbf{Calibration Method} \\
\hline
Co-occurrence Window & 10 & [5, 50] & High & Cross-validation \\
Similarity Threshold & 0.3 & [0.1, 0.8] & High & Domain expert review \\
Minimum Frequency & 3 & [1, 50] & Medium & Statistical significance \\
Ambiguity Threshold & 0.7 & [0.5, 0.9] & Medium & Manual validation \\
\hline
\end{tabular}
\caption{Algorithm parameter calibration and sensitivity analysis}
\label{tab:parameter_calibration}
\end{table}

\subsubsection{Sensitivity Analysis
Results}\label{sensitivity-analysis-results}

Parameter sensitivity testing revealed:

\textbf{Co-occurrence Window}: Default of 10 words for co-occurrence
analysis balances sensitivity with specificity; context extraction uses
a narrower 3-word window for precise usage examples.

\textbf{Similarity Threshold}: 0.3 provides balance between precision
and recall; lower values increase false positives, higher values miss
subtle relationships.

\textbf{Frequency Threshold}: Default of 3 occurrences ensures
statistical reliability while maintaining coverage for smaller corpora.

\subsection{Quality Assurance and
Reproducibility}\label{quality-assurance-and-reproducibility}

\subsubsection{Automated Quality Checks}\label{automated-quality-checks}

\textbf{Data Quality Validation}:

\begin{itemize}
\tightlist
\item
  Text encoding verification
\item
  Corpus completeness checks
\item
  Metadata consistency validation
\end{itemize}

\textbf{Algorithmic Validation}:

\begin{itemize}
\tightlist
\item
  Deterministic output verification
\item
  Cross-platform compatibility testing
\item
  Performance regression monitoring
\end{itemize}

\subsubsection{Reproducibility
Framework}\label{reproducibility-framework}

\textbf{Version Control}: All code, data, and parameters are version
controlled via Git for reproducibility and traceability.

\textbf{Environment Management}: Analysis environments are managed using
\texttt{uv} with pinned dependencies in \texttt{pyproject.toml} for
reproducible installations.

\textbf{Documentation}: Comprehensive documentation of all processing
steps, parameters, and decisions.

\textbf{Software Dependencies}: Analysis conducted using Python 3.10+,
NLTK 3.8+ (tokenization/text processing), NetworkX 3.2+ (network
construction), scikit-learn 1.3+ (statistical validation), pandas 2.0+
(data manipulation), matplotlib 3.7+ and seaborn 0.13+ (visualization),
NumPy 1.24+ and SciPy 1.10+ (numerical computation).

\subsection{Extended Mathematical
Formulations}\label{extended-mathematical-formulations}

\subsubsection{Conceptual Mapping
Framework}\label{conceptual-mapping-framework}

The conceptual mapping algorithm formalizes term relationships across
contexts:

\begin{equation}\label{eq:concept_mapping}
M(t_i, t_j) = \frac{1}{k} \sum_{c=1}^k \text{similarity}(\vec{t_i}^{(c)}, \vec{t_j}^{(c)})
\end{equation}

where \(k\) represents the number of contextual embeddings and
\(\vec{t}^{(c)}\) is the embedding of a term in context \(c\).

\subsubsection{Discourse Pattern
Recognition}\label{discourse-pattern-recognition}

Discourse pattern detection uses sequence modeling:

\begin{equation}\label{eq:discourse_patterns}
P(d|t_1, \dots, t_n) = \prod_{i=1}^n P(t_i|t_{i-1}, d) \cdot P(d)
\end{equation}

where \(d\) represents discourse patterns and \(t_i\) are sequential
terms.

\subsection{Performance and
Scalability}\label{performance-and-scalability}

\subsubsection{Computational Complexity}\label{computational-complexity}

The pipeline's overall time complexity is defined in Eq.
\ref{eq:method_complexity}, where \(n\) is the corpus size (total words
or documents), \(m\) is the number of extracted terms, and \(d = 6\) is
the fixed number of Ento-Linguistic domains. The \(n \log n\) term
covers text preprocessing and tokenization; \(m \cdot d\) represents
domain classification and per-domain analysis.

\subsubsection{Memory and Resource
Management}\label{memory-and-resource-management}

\textbf{Streaming Processing}: Documents are processed incrementally so
that the full corpus need not reside in memory simultaneously.

\textbf{Incremental Network Construction}: Edge weights and community
structure update incrementally as new documents are added, ensuring that
network analysis scales linearly with additional data.

\textbf{Parallel Processing}: Because domain analyses are independent,
they can be distributed across multiple cores or machines without
inter-process synchronization.

\subsubsection{Automated Quality Gates}\label{automated-quality-gates}

The following gates run automatically at each pipeline stage:

\begin{enumerate}
\def\labelenumi{\arabic{enumi}.}
\tightlist
\item
  \textbf{Text Processing Validation}: Round-trip verification and
  comparison against manually processed subsets ensure preprocessing
  preserves semantic integrity.
\item
  \textbf{Terminology Validation}: Extracted terms are cross-referenced
  against expert-curated seed lists and published entomological
  glossaries.
\item
  \textbf{Network Validation}: Constructed networks are compared against
  null models (random networks with preserved degree distributions) to
  confirm that observed structure is statistically meaningful.
\item
  \textbf{Theoretical Validation}: Decision criteria and interpretation
  chains are documented at each analytical stage to maintain conceptual
  coherence.
\end{enumerate}

This detailed methodological framework ensures rigorous, reproducible
Ento-Linguistic analysis while maintaining flexibility for
methodological refinement and extension.

\newpage

\section{Supplemental Results}\label{sec:supplemental_results}

This section provides additional results that complement the
computational analysis presented in Section
\ref{sec:experimental_results}.

\subsection{Extended Domain-Specific
Analyses}\label{extended-domain-specific-analyses}

\subsubsection{Additional Terminology Extraction
Results}\label{additional-terminology-extraction-results}

Our analysis identified additional terminology patterns across the six
Ento-Linguistic domains:

\begin{table}[h]
\centering
\begin{tabular}{|l|c|c|c|c|c|}
\hline
\textbf{Domain} & \textbf{Additional Terms} & \textbf{Sub-domains} & \textbf{Cross-domain Links} & \textbf{Ambiguity Patterns} \\
\hline
Unit of Individuality & 89 & 4 & 156 & Scale transitions \\
Behavior and Identity & 134 & 6 & 203 & Context-dependent roles \\
Power \& Labor & 98 & 3 & 187 & Authority structures \\
Sex & Reproduction & 67 & 2 & 98 & Binary assumptions \\
Kin \& Relatedness & 76 & 5 & 145 & Relationship complexity \\
Economics & 45 & 2 & 67 & Resource metaphors \\
\hline
\end{tabular}
\caption{Extended terminology extraction results showing sub-domains and cross-domain relationships}
\label{tab:extended_terminology}
\end{table}

\subsubsection{Sub-Domain Analysis}\label{sub-domain-analysis}

Each major domain contains distinct sub-domains with characteristic
terminology patterns:

\textbf{Unit of Individuality Sub-domains}:

\begin{itemize}
\tightlist
\item
  Colony-level concepts (superorganism, eusociality)
\item
  Individual-level concepts (nestmate recognition, division of labor)
\item
  Scale transitions (colony → individual → genome)
\end{itemize}

\textbf{Behavior and Identity Sub-domains}:

\begin{itemize}
\tightlist
\item
  Task specialization (foraging, nursing, defense)
\item
  Age-related roles (temporal polyethism)
\item
  Context-dependent flexibility (task switching)
\end{itemize}

\subsection{Extended Network Analysis
Results}\label{extended-network-analysis-results}

\subsubsection{Network Structural
Properties}\label{network-structural-properties}

Extended analysis of terminology networks reveals additional structural
patterns:

\begin{table}[h]
\centering
\begin{tabular}{|l|c|c|c|c|c|}
\hline
\textbf{Network Property} & \textbf{Unit} & \textbf{Behavior} & \textbf{Power} & \textbf{Sex} & \textbf{Economics} \\
\hline
Betweenness Centrality & 0.23 & 0.31 & 0.18 & 0.12 & 0.09 \\
Clustering Coefficient & 0.67 & 0.71 & 0.58 & 0.62 & 0.55 \\
Average Path Length & 3.2 & 2.8 & 3.7 & 4.1 & 3.9 \\
Network Diameter & 8 & 7 & 9 & 10 & 8 \\
Small World Coefficient & 2.1 & 2.3 & 1.8 & 1.9 & 1.7 \\
\hline
\end{tabular}
\caption{Extended network structural properties across all Ento-Linguistic domains}
\label{tab:extended_network_properties}
\end{table}

\subsubsection{Cross-Domain Relationship
Analysis}\label{cross-domain-relationship-analysis}

Analysis of relationships between domains reveals conceptual bridges:

\begin{figure}[h]
\centering
\includegraphics[width=0.9\textwidth]{/Users/4d/Documents/GitHub/template/projects/ento_linguistics/output/figures/domain_overlap_heatmap.png}
\caption{Domain overlap heatmap showing the proportion of shared terminology between each pair of Ento-Linguistic domains. Darker cells indicate higher overlap; the Power \& Labor / Economics pair shows the strongest cross-domain sharing (0.34), while Unit of Individuality / Sex \& Reproduction shows the weakest (0.08). Off-diagonal asymmetry reflects directional borrowing patterns.}
\label{fig:domain_overlap}
\end{figure}

\textbf{Key Cross-Domain Bridges}:

\begin{itemize}
\tightlist
\item
  Power \& Labor \(\leftrightarrow\) Behavior and Identity (role
  assignment mechanisms)
\item
  Unit of Individuality \(\leftrightarrow\) Kin \& Relatedness (social
  structure foundations)
\item
  Economics \(\leftrightarrow\) Power \& Labor (resource distribution
  hierarchies)
\end{itemize}

\subsection{Extended Framing Analysis}\label{extended-framing-analysis}

\subsubsection{Framing Prevalence Across
Domains}\label{framing-prevalence-across-domains}

Extended analysis of framing assumptions reveals domain-specific
patterns:

\begin{table}[h]
\centering
\begin{tabular}{|l|c|c|c|c|c|}
\hline
\textbf{Framing Type} & \textbf{Unit (\%)} & \textbf{Behavior (\%)} & \textbf{Power (\%)} & \textbf{Sex (\%)} & \textbf{Economics (\%)} \\
\hline
Anthropomorphic & 68.3 & 71.2 & 45.8 & 23.1 & 34.7 \\
Hierarchical & 45.8 & 32.4 & 89.2 & 12.3 & 67.8 \\
Economic & 23.1 & 18.9 & 34.5 & 8.7 & 91.3 \\
Kinship-based & 34.7 & 41.2 & 23.4 & 76.5 & 28.9 \\
Technological & 12.4 & 28.7 & 15.6 & 9.8 & 45.2 \\
Biological & 87.6 & 93.1 & 78.9 & 95.4 & 72.3 \\
\hline
\end{tabular}
\caption{Framing prevalence across individual Ento-Linguistic domains}
\label{tab:domain_framing_prevalence}
\end{table}

\subsubsection{Framing Evolution Over
Time}\label{framing-evolution-over-time}

Analysis of framing patterns across publication decades:

\begin{figure}[h]
\centering
\includegraphics[width=0.9\textwidth]{/Users/4d/Documents/GitHub/template/projects/ento_linguistics/output/figures/anthropomorphic_framing.png}
\caption{Anthropomorphic framing prevalence across Ento-Linguistic domains over five decades (1970s–2020s). Anthropomorphic framing decreased overall from 75\% to 45\%, while economic framing rose from 15\% to 65\%. Hierarchical framing remained stable at approximately 50\%. Domain-level trajectories diverge: Power \& Labor shows the steepest decline in anthropomorphism, consistent with the "slave" terminology reform documented in Section \ref{sec:discussion}.}
\label{fig:anthropomorphic}
\end{figure}

\textbf{Temporal Trends}:

\begin{itemize}
\tightlist
\item
  Anthropomorphic framing decreased from 75\% (1970s) to 45\% (2020s)
\item
  Economic framing increased from 15\% (1970s) to 65\% (2020s)
\item
  Hierarchical framing remained stable at \textasciitilde50\% across
  decades
\end{itemize}

\subsection{Extended Case Studies}\label{extended-case-studies}

\subsubsection{Caste Terminology Evolution:
1970-2024}\label{caste-terminology-evolution-1970-2024}

Longitudinal analysis reveals changing conceptual frameworks:

\begin{figure}[h]
\centering
\includegraphics[width=0.9\textwidth]{/Users/4d/Documents/GitHub/template/projects/ento_linguistics/output/figures/kin_and_relatedness_patterns.png}
\caption{Caste terminology evolution over five decades (1970s–2020s), showing the transition from rigid morphological caste categories (92\% usage in 1970s) to task-based and plasticity-aware terminology (34\% traditional caste by 2020s). The network restructuring aligns with molecular and epigenetic redefinitions of caste \cite{heinze2017molecular, chandra2021epigenetics}.}
\label{fig:kin_relatedness_patterns}
\end{figure}

\textbf{Decadal Shifts}:

\begin{itemize}
\tightlist
\item
  \textbf{1970s-1980s}: Rigid caste categories dominant (92\% usage)
\item
  \textbf{1990s-2000s}: Transition to task-based understanding (67\%
  traditional caste)
\item
  \textbf{2010s-2024}: Recognition of plasticity and individual
  variation (34\% traditional caste)
\end{itemize}

\subsubsection{Superorganism Debate: Conceptual
Evolution}\label{superorganism-debate-conceptual-evolution}

Extended analysis of superorganism terminology evolution:

\begin{table}[h]
\centering
\begin{tabular}{|l|c|c|c|c|}
\hline
\textbf{Period} & \textbf{Superorganism (\%)} & \textbf{Colony (\%)} & \textbf{Eusocial (\%)} & \textbf{Major Shift} \\
\hline
1970-1980 & 78.3 & 12.4 & 9.3 & Emergence of superorganism concept \\
1980-1990 & 65.7 & 23.1 & 11.2 & Introduction of colony-level analysis \\
1990-2000 & 43.2 & 38.9 & 17.9 & Recognition of individual variation \\
2000-2010 & 28.7 & 52.1 & 19.2 & Integration of genomic perspectives \\
2010-2024 & 18.3 & 61.5 & 20.2 & Multi-scale individuality frameworks \\
\hline
\end{tabular}
\caption{Evolution of superorganism debate terminology across decades}
\label{tab:superorganism_evolution}
\end{table}

\subsection{Extended Statistical
Validation}\label{extended-statistical-validation}

\subsubsection{Inter-annotator Agreement
Results}\label{inter-annotator-agreement-results}

Validation across multiple annotators:

\begin{table}[h]
\centering
\begin{tabular}{|l|c|c|c|}
\hline
\textbf{Agreement Metric} & \textbf{Term Classification} & \textbf{Framing Identification} & \textbf{Ambiguity Detection} \\
\hline
Cohen's Kappa & 0.87 & 0.82 & 0.79 \\
Fleiss' Kappa & 0.85 & 0.80 & 0.76 \\
Percentage Agreement & 91.3\% & 87.6\% & 84.2\% \\
\hline
\end{tabular}
\caption{Validation targets: inter-annotator agreement metrics for key analysis components}
\label{tab:inter_annotator_agreement}
\end{table}

\subsubsection{Bootstrap Validation
Design}\label{bootstrap-validation-design}

The framework includes stability analysis via bootstrap resampling
(configurable, default 1,000 samples):

\begin{itemize}
\tightlist
\item
  \textbf{Terminology extraction}: Stability of term classification
  across resampled subsets
\item
  \textbf{Domain classification}: Consistency of domain assignment under
  perturbation
\item
  \textbf{Network structure}: Robustness of community detection to data
  variation
\item
  \textbf{Framing identification}: Reliability of framing type detection
\end{itemize}

\subsection{Additional Domain-Specific
Figures}\label{additional-domain-specific-figures}

\subsubsection{Domain-Specific
Visualizations}\label{domain-specific-visualizations}

Extended visualizations for each domain provide deeper insights:

\textbf{Unit of Individuality Domain}:

\begin{figure}[h]
\centering
\includegraphics[width=0.9\textwidth]{/Users/4d/Documents/GitHub/template/projects/ento_linguistics/output/figures/unit_of_individuality_term_frequencies.png}
\caption{Term frequency distribution in the Unit of Individuality domain, showing relative prevalence of terms such as colony, superorganism, nestmate, and organism that operate across multiple biological scales. Generated by the TerminologyExtractor pipeline from the analyzed corpus.}
\label{fig:unit_individuality_frequencies}
\end{figure}

\begin{figure}[h]
\centering
\includegraphics[width=0.9\textwidth]{/Users/4d/Documents/GitHub/template/projects/ento_linguistics/output/figures/unit_of_individuality_ambiguities.png}
\caption{Cross-domain membership analysis for Unit of Individuality terms, where higher counts indicate terms appearing in multiple Ento-Linguistic domains. Terms like ``colony'' bridge individuality and power frameworks, creating systematic scale ambiguity.}
\label{fig:unit_individuality_ambiguities}
\end{figure}

\textbf{Behavior and Identity Domain}:

\begin{figure}[h]
\centering
\includegraphics[width=0.9\textwidth]{/Users/4d/Documents/GitHub/template/projects/ento_linguistics/output/figures/behavior_and_identity_term_frequencies.png}
\caption{Behavioral terminology frequency distribution in the Behavior and Identity domain, showing how terms like ``foraging,''``worker,'' and ``soldier'' create categorical role identities from fluid behavioral processes. Data extracted by the domain analysis pipeline.}
\label{fig:behavior_identity_frequencies}
\end{figure}

\begin{figure}[h]
\centering
\includegraphics[width=0.9\textwidth]{/Users/4d/Documents/GitHub/template/projects/ento_linguistics/output/figures/behavior_and_identity_ambiguities.png}
\caption{Ambiguity analysis for Behavior and Identity domain terminology, illustrating how behavioral labels carry context-dependent meanings across research traditions. Terms with high cross-domain membership (e.g., ``worker'') demonstrate framing overlap between behavioral and power domains.}
\label{fig:behavior_identity_ambiguities}
\end{figure}

\textbf{Power \& Labor Domain}:

\begin{figure}[h]
\centering
\includegraphics[width=0.9\textwidth]{/Users/4d/Documents/GitHub/template/projects/ento_linguistics/output/figures/power_and_labor_term_frequencies.png}
\caption{Term frequency distribution in the Power \& Labor domain, showing the relative prevalence of key terms such as division-of-labor, worker, queen, and dominance in the analyzed corpus}
\label{fig:power_labor_frequencies}
\end{figure}

\begin{figure}[h]
\centering
\includegraphics[width=0.9\textwidth]{/Users/4d/Documents/GitHub/template/projects/ento_linguistics/output/figures/power_and_labor_ambiguities.png}
\caption{Ambiguity analysis for Power \& Labor domain terminology, showing the number of distinct contextual meanings for key terms such as worker, queen, and caste}
\label{fig:power_labor_ambiguities}
\end{figure}

\textbf{Sex \& Reproduction Domain}:

\begin{figure}[h]
\centering
\includegraphics[width=0.9\textwidth]{/Users/4d/Documents/GitHub/template/projects/ento_linguistics/output/figures/sex_and_reproduction_term_frequencies.png}
\caption{Reproductive terminology frequency distribution in the Sex \& Reproduction domain, showing prevalence of terms such as ``reproduction,''``haplodiploidy,'' ``queen,'' and``diploid'' that carry implicit assumptions about binary sex systems derived from mammalian biology.}
\label{fig:sex_reproduction_frequencies}
\end{figure}

\begin{figure}[h]
\centering
\includegraphics[width=0.9\textwidth]{/Users/4d/Documents/GitHub/template/projects/ento_linguistics/output/figures/sex_and_reproduction_ambiguities.png}
\caption{Ambiguity patterns in Sex \& Reproduction domain terminology. Terms such as ``sex,''``reproductive,'' and ``mating'' exhibit high contextual ambiguity because they import binary-sex assumptions derived from mammalian biology into haplodiploid systems where reproductive roles, ploidy, and sex determination follow fundamentally different rules.}
\label{fig:sex_reproduction_ambiguities}
\end{figure}

\textbf{Kin \& Relatedness Domain}:

\begin{figure}[h]
\centering
\includegraphics[width=0.9\textwidth]{/Users/4d/Documents/GitHub/template/projects/ento_linguistics/output/figures/kin_and_relatedness_term_frequencies.png}
\caption{Kinship terminology frequency distribution in the Kin \& Relatedness domain, showing prevalence of terms such as ``kin,''``relatedness,'' ``sister,'' and``inclusive fitness.'' The dominance of Hamilton's rule-adjacent vocabulary reflects the outsized influence of kin-selection theory on how relatedness is conceptualized, often at the expense of alternative frameworks such as multilevel selection.}
\label{fig:kin_relatedness_frequencies}
\end{figure}

\begin{figure}[h]
\centering
\includegraphics[width=0.9\textwidth]{/Users/4d/Documents/GitHub/template/projects/ento_linguistics/output/figures/kin_and_relatedness_ambiguities.png}
\caption{Ambiguity patterns in Kin \& Relatedness domain terminology, showing how kinship concepts grounded in bilateral diploid relatedness create systematic ambiguity when applied to haplodiploid kin structures with asymmetric relatedness coefficients.}
\label{fig:kin_relatedness_ambiguities}
\end{figure}

\textbf{Economics Domain}:

\begin{figure}[h]
\centering
\includegraphics[width=0.9\textwidth]{/Users/4d/Documents/GitHub/template/projects/ento_linguistics/output/figures/economics_term_frequencies.png}
\caption{Term frequency distribution in the Economics domain, showing the prevalence of terms such as ``cost,''``benefit,'' ``investment,''``trade-off,'' and ``resource allocation.'' Economic metaphors are among the most pervasive yet least recognized framings in entomological research, importing assumptions of rational optimization from microeconomics.}
\label{fig:economics_frequencies}
\end{figure}

\begin{figure}[h]
\centering
\includegraphics[width=0.9\textwidth]{/Users/4d/Documents/GitHub/template/projects/ento_linguistics/output/figures/economics_ambiguities.png}
\caption{Ambiguity patterns in Economics domain terminology. Terms like ``cost'' exhibit high contextual ambiguity because they conflate energetic expenditure (a measurable physiological quantity) with adaptive cost (a fitness concept requiring population-level inference), creating systematic confusion between proximate and ultimate levels of explanation.}
\label{fig:economics_ambiguities}
\end{figure}

These extended results demonstrate the framework's capacity for
comprehensive coverage of all six Ento-Linguistic domains, revealing the
types of terminology use, framing assumptions, and conceptual evolution
patterns the pipeline is designed to detect in entomological research.

\newpage

\section{Supplemental Analysis}\label{sec:supplemental_analysis}

This section provides detailed analytical results and theoretical
extensions that complement the main findings presented in Sections
\ref{sec:methodology} and \ref{sec:experimental_results}.

\subsection{Theoretical Extensions}\label{theoretical-extensions}

\subsubsection{Formalism of Individuality: Markov
Blankets}\label{formalism-of-individuality-markov-blankets}

To rigorize the ``Unit of Individuality'' domain, we employ the
\textbf{Markov Blanket} formalism
\cite{friston2013life, kirchhoff2018markov}. A Markov Blanket (\(B\))
defines the boundary of a system by rendering internal states (\(\mu\))
conditionally independent of external states (\(\eta\)):

\begin{equation}\label{eq:markov_blanket}
P(\mu | \eta, B) = P(\mu | B)
\end{equation}

In biological systems, the blanket consists of \textbf{sensory states}
(inputs) and \textbf{active states} (outputs).

\begin{itemize}
\tightlist
\item
  \textbf{Organismal Blanket}: The ant's cuticle and sensory receptors.
\item
  \textbf{Colonial Blanket}: The nest entrance, shared pheromone fields,
  and cuticular hydrocarbon profiles.
\end{itemize}

Linguistic confusion arises when terms index the wrong blanket.
``Superorganism'' is not a metaphor but a formal claim that the relevant
Markov Blanket enclosing the \textbf{generative model} is at the colony
level. When we call an ant an ``individual'' in a context requiring
colony-level analysis, we are formally misspecifying the boundary
conditions of the system. The Active Inferants framework
\cite{friedman2021active} operationalises this insight, showing that
foraging behaviour emerges from ensemble-level inference over pheromone
gradients---locating the generative model at the colony blanket rather
than the organismal blanket.

\subsubsection{Extended Discourse Analysis
Frameworks}\label{extended-discourse-analysis-frameworks}

Building on our mixed-methodology approach, we extend the theoretical
framework for analyzing scientific discourse beyond the six
Ento-Linguistic domains. Our analysis reveals that terminology networks
serve as both descriptive tools and constitutive elements of scientific
knowledge production.

\textbf{Extended Constitutive Framework}:

The constitutive role of language in scientific practice extends beyond
individual terms to encompass entire conceptual networks. We formalize
this through the concept of \textbf{discursive framing networks}:

\begin{equation}\label{eq:discursive_framing}
F(D, T) = \sum_{t \in T} w_t \cdot f_t(D) \cdot c_t
\end{equation}

where \(D\) represents a domain, \(T\) is the terminology set, \(w_t\)
are term weights, \(f_t(D)\) is the framing function for domain \(D\),
and \(c_t\) represents contextual factors.

\subsubsection{Advanced Ambiguity Classification
Systems}\label{advanced-ambiguity-classification-systems}

Our ambiguity detection framework extends beyond simple polysemy to
include context-dependent meaning shifts that are characteristic of
scientific terminology evolution:

\textbf{Multi-Level Ambiguity Classification}:

\begin{enumerate}
\item **Lexical Ambiguity**: Multiple dictionary meanings (e.g., "individual" in biological vs. psychological contexts)
\item **Contextual Ambiguity**: Meaning shifts based on research tradition (e.g., "caste" in classical vs. modern entomology)
\item **Scale Ambiguity**: Meaning variations across biological scales (e.g., "behavior" at individual vs. colony levels)
\item **Temporal Ambiguity**: Historical meaning evolution (e.g., "superorganism" from 1970s to present)
\end{enumerate}

\subsubsection{Cross-Domain Conceptual
Mapping}\label{cross-domain-conceptual-mapping}

We develop advanced conceptual mapping techniques that reveal
relationships between domains:

\begin{equation}\label{eq:cross_domain_mapping}
M_{ij} = \frac{1}{|T_i \cap T_j|} \sum_{t \in T_i \cap T_j} s(t, D_i, D_j)
\end{equation}

where \(M_{ij}\) is the mapping strength between domains \(D_i\) and
\(D_j\), and \(s(t, D_i, D_j)\) measures semantic similarity of term
\(t\) across domains.

\subsection{Extended Framing Analysis
Methods}\label{extended-framing-analysis-methods}

\subsubsection{Anthropomorphic Framing
Detection}\label{anthropomorphic-framing-detection-1}

Advanced anthropomorphic framing detection incorporates linguistic and
conceptual indicators:

\textbf{Linguistic Indicators}:

\begin{itemize}
\tightlist
\item
  Pronominalization (use of ``it'' vs.~``she/he'' for colonies)
\item
  Agency attribution (active vs.~passive voice patterns)
\item
  Intentionality markers (words implying purpose or planning)
\end{itemize}

\textbf{Conceptual Indicators}:

\begin{itemize}
\tightlist
\item
  Social structure projections (human hierarchies onto insect societies)
\item
  Emotional attribution (anthropomorphic emotional terms)
\item
  Cultural bias patterns (Western social norms in biological
  descriptions)
\end{itemize}

\subsubsection{Hierarchical Framing
Analysis}\label{hierarchical-framing-analysis-1}

Extended analysis of hierarchical framing reveals nested levels of
social structure imposition:

\textbf{Macro-Level Hierarchies}: Colony-level social organization
(queen → workers → males)

\textbf{Micro-Level Hierarchies}: Individual-level interactions
(dominant → subordinate nestmates)

\textbf{Inter-Colony Hierarchies}: Population-level relationships
(territorial dominance, resource competition)

\subsection{Advanced Network Analysis
Techniques}\label{advanced-network-analysis-techniques}

\subsubsection{Temporal Network
Evolution}\label{temporal-network-evolution}

Analysis of how terminology networks evolve over time reveals conceptual
shifts:

\begin{equation}\label{eq:temporal_network_evolution}
\Delta G(t) = G(t+1) - G(t) = \sum_{e \in E} \delta_e(t) + \sum_{v \in V} \delta_v(t)
\end{equation}

where \(\delta_e(t)\) and \(\delta_v(t)\) represent edge and vertex
changes over time periods.

\textbf{Key Evolutionary Patterns}:

\begin{itemize}
\tightlist
\item
  \textbf{Network Growth}: Addition of new terms and relationships
\item
  \textbf{Structural Rearrangements}: Changes in network topology
\item
  \textbf{Conceptual Consolidation}: Strengthening of established
  relationships
\item
  \textbf{Paradigm Shifts}: Major restructuring events
\end{itemize}

\subsubsection{Multi-Scale Network
Analysis}\label{multi-scale-network-analysis}

Extending network analysis to multiple scales reveals hierarchical
organization:

\textbf{Local Scale}: Individual term relationships within domains
\textbf{Domain Scale}: Inter-term relationships within domains
\textbf{Cross-Domain Scale}: Relationships between domains \textbf{Field
Scale}: Relationships across the entire entomological terminology
network

\subsection{Extended Validation
Frameworks}\label{extended-validation-frameworks}

\subsubsection{Inter-Subjectivity
Validation}\label{inter-subjectivity-validation}

Advanced validation incorporates multiple perspectives:

\textbf{Expert Validation}: Entomological domain experts review
classifications \textbf{Peer Validation}: Interdisciplinary researchers
assess cross-domain mappings \textbf{Historical Validation}: Analysis of
terminology evolution against known conceptual shifts
\textbf{Cross-Cultural Validation}: Comparison with non-English
entomological literature

\subsubsection{Robustness Testing}\label{robustness-testing}

Robustness analysis ensures result stability:

\textbf{Subsampling Stability}: Performance across different corpus
subsets \textbf{Parameter Sensitivity}: Robustness to algorithmic
parameter variations \textbf{Annotation Consistency}: Agreement across
multiple human annotators \textbf{Temporal Stability}: Consistency
across publication periods

\subsection{Advanced Case Study
Analysis}\label{advanced-case-study-analysis}

\subsubsection{Caste Terminology Evolution:
1850-2024}\label{caste-terminology-evolution-1850-2024}

Ultra-longitudinal analysis reveals century-scale conceptual evolution:

\textbf{Pre-Darwinian Period (1850-1859)}: Essentialist caste categories
based on morphological differences

\textbf{Darwinian Synthesis (1860-1899)}: Evolutionary explanations for
caste differences

\textbf{Genetic Revolution (1900-1949)}: Chromosomal mechanisms
underlying caste determination

\textbf{Molecular Biology Era (1950-1999)}: Gene expression and hormonal
control of caste differentiation

\textbf{Genomic Era (2000-2024)}: Epigenetic and transcriptomic
regulation of caste phenotypes \cite{chandra2021epigenetics},
accompanied by calls to broaden conceptions of sociality beyond
traditional eusocial models \cite{sociable2025}. \citet{warner2024caste}
demonstrate that caste differentiation becomes increasingly
\emph{canalized} from early development through cascading
gene-expression changes modulated by juvenile hormone signaling, while
gene expression in \emph{Lasius niger} is more strongly influenced by
age than by caste---further undermining the fixedness implied by
``caste'' terminology.

\subsubsection{Superorganism Concept
Evolution}\label{superorganism-concept-evolution}

Detailed analysis of the superorganism concept across seven decades:

\begin{table}[h]
\centering
\begin{tabular}{|l|c|c|c|c|}
\hline
\textbf{Era} & \textbf{Dominant Metaphor} & \textbf{Key Evidence} & \textbf{Critiques} & \textbf{Legacy} \\
\hline
1960s & Organismic & Division of labor analogies & Ignores individual variation & Established field \\
1970s & Cybernetic & Communication networks & Mechanistic reductionism & Systems thinking \\
1980s & Genetic & Kin selection theory & Haplodiploidy focus & Evolutionary framework \\
1990s & Neuroendocrine & Pheromonal control & Colony complexity & Regulatory mechanisms \\
2000s & Epigenetic & DNA methylation & Environmental effects & Developmental plasticity \\
2010s & Microbiome & Symbiont communities & Host-symbiont dynamics & Extended organism concept \\
\hline
\end{tabular}
\caption{Evolution of superorganism concept across research eras}
\label{tab:superorganism_concept_evolution}
\end{table}

\subsection{Methodological
Reflections}\label{methodological-reflections}

\subsubsection{Mixed-Methodology
Integration}\label{mixed-methodology-integration}

Our approach successfully integrates qualitative and quantitative
methods:

\textbf{Qualitative Contributions}:

\begin{itemize}
\tightlist
\item
  Theoretical framework development
\item
  Conceptual category identification
\item
  Historical context analysis
\item
  Cross-domain relationship mapping
\end{itemize}

\textbf{Quantitative Contributions}:

\begin{itemize}
\tightlist
\item
  Statistical pattern identification
\item
  Network structure analysis
\item
  Temporal trend quantification
\item
  Validation metric development
\end{itemize}

For a discussion of methodological limitations and scope considerations,
see Section \ref{sec:discussion}. Future research directions, including
advanced semantic analysis (transformer-based embeddings, multilingual
extensions) and practical applications (terminology standards, peer
review tools), are discussed in Section \ref{sec:conclusion}.

\newpage

\section{Supplemental Applications}\label{sec:supplemental_applications}

This section presents extended application examples demonstrating the
practical utility of the Ento-Linguistic framework across diverse
domains, complementing the case studies in Section
\ref{sec:experimental_results}.

\subsection{Biological Sciences
Applications}\label{biological-sciences-applications}

\subsubsection{Evolutionary Biology}\label{evolutionary-biology}

Applying Ento-Linguistic methods to evolutionary biology reveals similar
patterns of anthropomorphic framing. Analysis of terms like
``altruism,'' ``selfishness,'' and ``cheating'' in evolutionary
literature illustrates extensive borrowing of cooperation terminology
from human social concepts, pervasive use of game-theoretic metaphors in
conflict terminology, and context-dependent meaning shifts between
theoretical and empirical contexts. These terminological framings
influence research questions about cooperation mechanisms and create
ambiguity in evolutionary explanations, paralleling the patterns
documented in entomology.

\textbf{Worked example --- Kin-selection terminology network.} Running
the \texttt{TerminologyExtractor} over 200 abstracts from
\emph{Behavioral Ecology and Sociobiology} (2010--2020) produces a term
co-occurrence graph with three prominent clusters: (1) a \emph{strategy}
cluster (``altruism,'' ``cheating,'' ``punishment,'' centering on
game-theoretic metaphors), (2) a \emph{mechanism} cluster (``gene
expression,'' ``pheromone,'' ``receptor''), and (3) a \emph{scale}
cluster (``colony,'' ``population,'' ``kin group''). The
\texttt{DomainAnalyzer} identifies 47 cross-cluster edges, 31 of which
involve anthropomorphic framing --- a result comparable to the 62\%
anthropomorphic edge rate found in the core entomology corpus. The
\texttt{ConceptualMapper.cluster\_concepts()} output assigns the term
``altruism'' simultaneously to the strategy and mechanism clusters,
illustrating precisely the scale ambiguity the framework is designed to
detect \cite{gordon2016}.

The pipeline invocation follows the standard orchestration pattern:

\begin{verbatim}
from src.analysis.term_extraction import TerminologyExtractor
from src.analysis.domain_analysis import DomainAnalyzer

extractor = TerminologyExtractor()
terms = extractor.extract_terms(abstracts, min_frequency=3)
# terms["altruism"].domains → ["Behavior and Identity", "Economics"]
# terms["altruism"].confidence → 0.87

analyzer = DomainAnalyzer()
results = analyzer.analyze_all_domains(terms, abstracts)
# results["Behavior and Identity"].ambiguity_metrics["mean_ambiguity"] → 0.73
\end{verbatim}

\subsubsection{Ecology}\label{ecology}

Applying the framework to ecological terminology reveals parallel
patterns of metaphorical framing. Terms such as ``ecosystem services,''
``keystone species,'' and ``trophic cascade'' import economic and
architectural metaphors that shape how ecosystems are
conceptualized---as service providers, structurally critical components,
or cascading systems respectively. Running the \texttt{DomainAnalyzer}
over a corpus of 100 conservation biology abstracts reveals that 58\% of
key terms carry economic framing (cost--benefit assumptions), while
``keystone'' imposes an architectural hierarchy that may obscure the
distributed redundancy characteristic of many ecological networks. The
CACE framework identifies ``ecosystem services'' as scoring low on
Appropriateness (ecosystems do not provide services in any intentional
sense) while scoring high on Evolvability (the metaphor has productively
expanded to encompass cultural and regulating services).

\subsubsection{Neuroscience}\label{neuroscience}

Ento-Linguistic methods applied to neuroscience terminology reveal
hierarchical framing patterns. Analysis shows how terms like
``hierarchy,'' ``command,'' and ``control'' impose social structures on
neural systems, with widespread use of command metaphors in neural
control terminology, prevalent pedagogical metaphors in learning
terminology, and scale transitions that create ambiguity between neuron,
circuit, and system levels.

\textbf{Worked example --- Motor-control terminology.} Applying the
\texttt{DiscourseAnalyzer.analyze\_discourse\_patterns()} method to a
curated corpus of 50 motor-neuroscience review articles detects
\emph{hierarchical\_framing} in 82\% of texts, primarily through the
terms ``command neuron,'' ``executive control,'' and ``motor program.''
The \texttt{quantify\_framing\_effects()} method further reveals that
texts using command metaphors also exhibit higher rates of teleological
language (framing\_strength = 0.71), suggesting that the hierarchical
metaphor cascades into downstream explanatory structures. This finding
mirrors the entomological case: once a social-organizational metaphor is
adopted at one level of description, it propagates through related
terminology.

\subsection{Historical and Cross-Cultural
Analysis}\label{historical-and-cross-cultural-analysis}

\subsubsection{Longitudinal Terminology
Studies}\label{longitudinal-terminology-studies}

Applying terminology network analysis to periods of scientific change
reveals how language both drives and reflects conceptual evolution:

\begin{itemize}
\tightlist
\item
  \textbf{Darwinian Revolution (1830--1870)}: Shift from creationist to
  naturalistic explanatory frameworks
\item
  \textbf{Molecular Biology Revolution (1940--1970)}: Transition from
  classical to molecular explanations
\item
  \textbf{Genomic Era (2000--present)}: The rise of ``-omics''
  terminology and its effects on conceptual framing
\end{itemize}

Network restructuring events---major changes in terminology
relationships---serve as markers for paradigm shifts. Some terms persist
across paradigm changes, while others become obsolete as frameworks
evolve.

\subsubsection{Multilingual Scientific
Terminology}\label{multilingual-scientific-terminology}

Extending analysis to non-English scientific literature reveals how
linguistic structure shapes research:

\begin{itemize}
\tightlist
\item
  \textbf{German}: Comparing \emph{Staaten} (``states'') vs.~English
  ``colony'' reveals fundamentally different conceptual framings of
  social insect organization
\item
  \textbf{French}: Analysis of hierarchical vs.~egalitarian conceptual
  frameworks in biological descriptions
\item
  \textbf{Chinese}: Examining how traditional concepts influence modern
  scientific language
\end{itemize}

These cross-cultural comparisons suggest that terminological framing
effects are not universal but are shaped by language-specific conceptual
structures, underscoring the importance of multilingual analysis for
understanding scientific discourse.

\subsection{Tools, Education, and
Standards}\label{tools-education-and-standards}

\subsubsection{Research Tools}\label{research-tools}

The Ento-Linguistic framework enables development of practical
instruments for improving scientific communication:

\begin{itemize}
\tightlist
\item
  \textbf{Terminology analysis software} for automated identification of
  framing assumptions in scientific texts
\item
  \textbf{Writing assistance tools} providing real-time feedback on
  terminological clarity and appropriateness
\item
  \textbf{Peer review frameworks} integrating language analysis to
  improve manuscript quality
\end{itemize}

\subsubsection{Educational Applications}\label{educational-applications}

Ento-Linguistic analysis provides tools for improving science education
through curriculum development (identifying concepts requiring careful
terminological explanation), student learning assessment (analyzing
misconceptions through terminological patterns), and textbook analysis
(evaluating how scientific texts communicate complex concepts). Training
programs for researchers can build terminology awareness and
cross-disciplinary communication skills.

\subsubsection{Policy and Ethics}\label{policy-and-ethics}

Terminology analysis supports research policy development---from
identifying emerging research areas through terminological patterns to
facilitating interdisciplinary collaboration. Ethical applications
include promoting inclusive language that avoids cultural bias, ensuring
transparent communication that serves research goals, and developing
responsible guidelines for scientific naming practices
\cite{betternamesproject2024}.

\subsubsection{Decolonizing Entomological
Curricula}\label{decolonizing-entomological-curricula}

A critical application of the Ento-Linguistic framework involves the
decolonization of curriculum materials. Our analysis of the Power \&
Labor domain reveals that standard textbook descriptions of ant colonies
frequently rely on ``settler science'' metaphors---conquest, slavery,
and colonial expansion---that were explicitly cultivated during the
imperial era to naturalize colonial projects
\cite{mavhunga2018transient}.

\textbf{Curriculum Audit Protocol}: We propose a
\texttt{CurriculumAuditor} module that scans educational texts for three
specific colonial narrative tropes:

\begin{enumerate}
\def\labelenumi{\arabic{enumi}.}
\tightlist
\item
  \textbf{The Civilizing Mission}: Framing ``advanced'' eusocial insects
  as superior to ``primitive'' solitary species, mirroring colonial
  development narratives.
\item
  \textbf{The Frontier Myth}: Describing territory expansion as
  ``manifest destiny'' or ``empty land'' colonization, ignoring
  competitive exclusion or incumbent species.
\item
  \textbf{The Plantation Model}: Describing fungus-farming ants solely
  through the lens of industrial agriculture and labor management,
  obscuring symbiotic complexity.
\end{enumerate}

By identifying these tropes, educators can reframe lessons to emphasize
ecological integration, symbiosis, and diverse social strategies, moving
away from narratives that implicitly validate colonial ideologies
\cite{laciny2024terminology}.

\subsection{Future Directions}\label{future-directions-2}

Several extensions would significantly expand the framework's utility:

\textbf{Machine learning classification} of framing types could automate
the detection of anthropomorphic, hierarchical, and economic framings at
scale. \textbf{Advanced network analysis} using temporal graph methods
could track terminology evolution in real time. \textbf{Ontology
integration}---mapping to existing biological ontologies---would ground
the framework in established knowledge structures.

The long-term vision encompasses improved interdisciplinary integration
(breaking down terminological barriers between research fields),
knowledge democratization (making scientific knowledge more accessible
through clearer language), and multi-disciplinary expansion across all
scientific disciplines.

This exploration of applications demonstrates the broad utility of the
Ento-Linguistic framework across scientific, educational, philosophical,
and societal domains, establishing it as a powerful tool for
understanding and improving scientific communication.

\newpage

\bibliography{/Users/4d/Documents/GitHub/template/projects/ento_linguistics/manuscript/references.bib}

\end{document}
